\subsection{Семинар <<Композиция>> (2 часа)}

\subsubsection*{Теория}

\tikzset{
  class/.style={
    rectangle,
    draw,
    minimum width=3cm,
    minimum height=1.2cm,
    align=center
  }
}


\begin{center}
\begin{tikzpicture}[node distance=3cm]

% Класс Поезд
\node[class] (Train) {Поезд};

% Класс Вагон
\node[class, right=of Train] (Carriage) {Вагон};

% Связь композиции: Поезд -> Вагон
% Заполненный ромб у Поезда (композиция)
\draw[thick] (Train.east) -- ++(1cm,0) coordinate (mid);
\fill (Train.east) -- ++(0.3,0.2) -- ++(0,-0.4) -- cycle; % ромб
\draw[thick] (mid) -- (Carriage.west);

% Мультипликация
\node[above] at (mid) {1};
\node[above] at ($(Carriage.west)!0.5!(mid)$) {*};

\end{tikzpicture}
\end{center}

Композиция — это концепция, позволяющая моделировать отношения между классами в программе. Она представляет собой один из способов организации взаимодействия классов. Композиция позволяет создавать сложные типы, комбинируя объекты других типов. Это означает, что один класс («целое») может содержать экземпляр другого класса («часть») как своё поле.

Классы, содержащие объекты другxих классов, обычно называются \emph{композитами} (\emph{composites}), а классы, используемые для создания более сложных типов, называются \emph{компонентами} (\emph{components}).

Композиция позволяет повторно использовать код путём включения объектов других классов, избегая при этом наследования.

\textbf{Вопросы для подготовки к защите}

\begin{itemize}
\item Что такое композиция?
    \item Сравните композицию и наследование.
    \item Опишите достоинства и недостатки указания типов аргументов и результатов у функций.
\end{itemize}

В ходе выполнения используйте свойство инкапсуляции (даже если в формулировке 
задания это не учтено).

\subsubsection{Задача 1}

    \begin{enumerate}
\item[1] \textbf{Формирование состава груженых контейнеров}
\begin{enumerate}
    \item Создайте класс \texttt{Container}, который будет представлять собой контейнер с грузом. В конструкторе класса \texttt{Container} инициализируйте значения контейнера и груза из списков \texttt{NumList} и \texttt{MasList}, которые объявлены как общие атрибуты класса. \texttt{NumList: list[str]} — это список контейнеров (не менее 14): 
    \begin{center}
        \texttt{[''Контейнер\_1'', ''Контейнер\_2'', \dots, ''Контейнер\_14'']}
    \end{center}
    \texttt{MasList: list[str]} — это список грузов для контейнеров (не менее 4):
    \begin{center}
        \texttt{[''Электроника'', ''Мебель'', ''Одежда'', ''Продукты'']}
    \end{center}
    Конструктор должен иметь сигнатуру: \texttt{\_\_init\_\_(self) -> None}.

    \item Создайте класс \texttt{TrainOfContainers}, который будет представлять собой состав, состоящий из моделей контейнеров. В конструкторе класса \texttt{TrainOfContainers} инициализируйте список контейнеров \texttt{self.train: list[Container]} длиной 56.

    \item Добавьте метод \texttt{shuffle(self) -> None} в класс \texttt{TrainOfContainers}, который будет перемешивать контейнеры в списке \texttt{self.train}.

    \item Добавьте метод \texttt{get(self, i: int) -> Container}, который будет возвращать $i$-й контейнер и груз из списка \texttt{self.train}.

    \item Создайте экземпляр класса \texttt{TrainOfContainers} и вызовите метод \texttt{shuffle} для перемешивания контейнеров.

    \item Создайте цикл, который будет запрашивать у пользователя номер контейнера из состава и выводить информацию о выбранном контейнере.

    \item Повторите шаги 5–6 до тех пор, пока пользователь не выберет все контейнеры или не завершит выбор.

    \item В конце программы выводите сообщение о завершении выбора контейнеров.

    \item Убедитесь, что пользователь вводит корректные номера контейнеров и что программа обрабатывает ошибки, связанные с вводом пользователя.

    \item Проверьте работу программы, используя различные комбинации номеров контейнеров и грузов.
\end{enumerate}

\item[2] \textbf{Формирование состава почтовых посылок}
\begin{enumerate}
    \item Создайте класс \texttt{Parcel}, который будет представлять собой посылку с содержимым. В конструкторе класса \texttt{Parcel} инициализируйте значения посылки и содержимого из списков \texttt{NumList} и \texttt{MasList}, которые объявлены как общие атрибуты класса. \texttt{NumList: list[str]} — это список посылок (не менее 14): 
    \begin{center}
        \texttt{[''Посылка\_1'', ''Посылка\_2'', \dots, ''Посылка\_14'']}
    \end{center}
    \texttt{MasList: list[str]} — это список содержимого посылок (не менее 4):
    \begin{center}
        \texttt{[''Книги'', ''Игрушки'', ''Косметика'', ''Спортивный инвентарь'']}
    \end{center}
    Конструктор должен иметь сигнатуру: \texttt{\_\_init\_\_(self) -> None}.

    \item Создайте класс \texttt{TrainOfParcels}, который будет представлять собой состав, состоящий из моделей посылок. В конструкторе класса \texttt{TrainOfParcels} инициализируйте список посылок \texttt{self.train: list[Parcel]} длиной 56.

    \item Добавьте метод \texttt{shuffle(self) -> None} в класс \texttt{TrainOfParcels}, который будет перемешивать посылки в списке \texttt{self.train}.

    \item Добавьте метод \texttt{get(self, i: int) -> Parcel}, который будет возвращать $i$-ю посылку и её содержимое из списка \texttt{self.train}.

    \item Создайте экземпляр класса \texttt{TrainOfParcels} и вызовите метод \texttt{shuffle} для перемешивания посылок.

    \item Создайте цикл, который будет запрашивать у пользователя номер посылки из состава и выводить информацию о выбранной посылке.

    \item Повторите шаги 5–6 до тех пор, пока пользователь не выберет все посылки или не завершит выбор.

    \item В конце программы выводите сообщение о завершении выбора посылок.

    \item Убедитесь, что пользователь вводит корректные номера посылок и что программа обрабатывает ошибки, связанные с вводом пользователя.

    \item Проверьте работу программы, используя различные комбинации номеров посылок и содержимого.
\end{enumerate}

\item[3] \textbf{Формирование автовоза с автомобилями}
\begin{enumerate}
    \item Создайте класс \texttt{Car}, который будет представлять собой автомобиль на автовозе. В конструкторе класса \texttt{Car} инициализируйте значения автомобиля и его марки из списков \texttt{NumList} и \texttt{MasList}, которые объявлены как общие атрибуты класса. \texttt{NumList: list[str]} — это список автомобилей (не менее 14): 
    \begin{center}
        \texttt{[''Автомобиль\_1'', ''Автомобиль\_2'', \dots, ''Автомобиль\_14'']}
    \end{center}
    \texttt{MasList: list[str]} — это список марок автомобилей (не менее 4):
    \begin{center}
        \texttt{[''Toyota'', ''BMW'', ''Lada'', ''Tesla'']}
    \end{center}
    Конструктор должен иметь сигнатуру: \texttt{\_\_init\_\_(self) -> None}.

    \item Создайте класс \texttt{CarCarrier}, который будет представлять собой автовоз, состоящий из моделей автомобилей. В конструкторе класса \texttt{CarCarrier} инициализируйте список автомобилей \texttt{self.train: list[Car]} длиной 56.

    \item Добавьте метод \texttt{shuffle(self) -> None} в класс \texttt{CarCarrier}, который будет перемешивать автомобили в списке \texttt{self.train}.

    \item Добавьте метод \texttt{get(self, i: int) -> Car}, который будет возвращать $i$-й автомобиль и его марку из списка \texttt{self.train}.

    \item Создайте экземпляр класса \texttt{CarCarrier} и вызовите метод \texttt{shuffle} для перемешивания автомобилей.

    \item Создайте цикл, который будет запрашивать у пользователя номер автомобиля на автовозе и выводить информацию о выбранном автомобиле.

    \item Повторите шаги 5–6 до тех пор, пока пользователь не выберет все автомобили или не завершит выбор.

    \item В конце программы выводите сообщение о завершении выбора автомобилей.

    \item Убедитесь, что пользователь вводит корректные номера автомобилей и что программа обрабатывает ошибки, связанные с вводом пользователя.

    \item Проверьте работу программы, используя различные комбинации номеров автомобилей и марок.
\end{enumerate}

\item[4] \textbf{Формирование багажного состава из чемоданов}
\begin{enumerate}
    \item Создайте класс \texttt{Suitcase}, который будет представлять собой чемодан с владельцем. В конструкторе класса \texttt{Suitcase} инициализируйте значения чемодана и владельца из списков \texttt{NumList} и \texttt{MasList}, которые объявлены как общие атрибуты класса. \texttt{NumList: list[str]} — это список чемоданов (не менее 14): 
    \begin{center}
        \texttt{[''Чемодан\_1'', ''Чемодан\_2'', \dots, ''Чемодан\_14'']}
    \end{center}
    \texttt{MasList: list[str]} — это список владельцев (не менее 4):
    \begin{center}
        \texttt{[''Иванов'', ''Петров'', ''Сидоров'', ''Кузнецов'']}
    \end{center}
    Конструктор должен иметь сигнатуру: \texttt{\_\_init\_\_(self) -> None}.

    \item Создайте класс \texttt{BaggageTrain}, который будет представлять собой багажный состав, состоящий из моделей чемоданов. В конструкторе класса \texttt{BaggageTrain} инициализируйте список чемоданов \texttt{self.train: list[Suitcase]} длиной 56.

    \item Добавьте метод \texttt{shuffle(self) -> None} в класс \texttt{BaggageTrain}, который будет перемешивать чемоданы в списке \texttt{self.train}.

    \item Добавьте метод \texttt{get(self, i: int) -> Suitcase}, который будет возвращать $i$-й чемодан и его владельца из списка \texttt{self.train}.

    \item Создайте экземпляр класса \texttt{BaggageTrain} и вызовите метод \texttt{shuffle} для перемешивания чемоданов.

    \item Создайте цикл, который будет запрашивать у пользователя номер чемодана из состава и выводить информацию о выбранном чемодане.

    \item Повторите шаги 5–6 до тех пор, пока пользователь не выберет все чемоданы или не завершит выбор.

    \item В конце программы выводите сообщение о завершении выбора чемоданов.

    \item Убедитесь, что пользователь вводит корректные номера чемоданов и что программа обрабатывает ошибки, связанные с вводом пользователя.

    \item Проверьте работу программы, используя различные комбинации номеров чемоданов и владельцев.
\end{enumerate}

\item[5] \textbf{Формирование складского состава из ящиков}
\begin{enumerate}
    \item Создайте класс \texttt{Box}, который будет представлять собой ящик с содержимым. В конструкторе класса \texttt{Box} инициализируйте значения ящика и содержимого из списков \texttt{NumList} и \texttt{MasList}, которые объявлены как общие атрибуты класса. \texttt{NumList: list[str]} — это список ящиков (не менее 14): 
    \begin{center}
        \texttt{[''Ящик\_1'', ''Ящик\_2'', \dots, ''Ящик\_14'']}
    \end{center}
    \texttt{MasList: list[str]} — это список типов содержимого (не менее 4):
    \begin{center}
        \texttt{[''Инструменты'', ''Запчасти'', ''Химикаты'', ''Упаковка'']}
    \end{center}
    Конструктор должен иметь сигнатуру: \texttt{\_\_init\_\_(self) -> None}.

    \item Создайте класс \texttt{WarehouseTrain}, который будет представлять собой состав, состоящий из моделей ящиков. В конструкторе класса \texttt{WarehouseTrain} инициализируйте список ящиков \texttt{self.train: list[Box]} длиной 56.

    \item Добавьте метод \texttt{shuffle(self) -> None} в класс \texttt{WarehouseTrain}, который будет перемешивать ящики в списке \texttt{self.train}.

    \item Добавьте метод \texttt{get(self, i: int) -> Box}, который будет возвращать $i$-й ящик и его содержимое из списка \texttt{self.train}.

    \item Создайте экземпляр класса \texttt{WarehouseTrain} и вызовите метод \texttt{shuffle} для перемешивания ящиков.

    \item Создайте цикл, который будет запрашивать у пользователя номер ящика из состава и выводить информацию о выбранном ящике.

    \item Повторите шаги 5–6 до тех пор, пока пользователь не выберет все ящики или не завершит выбор.

    \item В конце программы выводите сообщение о завершении выбора ящиков.

    \item Убедитесь, что пользователь вводит корректные номера ящиков и что программа обрабатывает ошибки, связанные с вводом пользователя.

    \item Проверьте работу программы, используя различные комбинации номеров ящиков и содержимого.
\end{enumerate}

\item[6] \textbf{Формирование состава морских судов с грузом}
\begin{enumerate}
    \item Создайте класс \texttt{Ship}, который будет представлять собой судно с грузом. В конструкторе класса \texttt{Ship} инициализируйте значения судна и груза из списков \texttt{NumList} и \texttt{MasList}, которые объявлены как общие атрибуты класса. \texttt{NumList: list[str]} — это список судов (не менее 14): 
    \begin{center}
        \texttt{[''Судно\_1'', ''Судно\_2'', \dots, ''Судно\_14'']}
    \end{center}
    \texttt{MasList: list[str]} — это список грузов (не менее 4):
    \begin{center}
        \texttt{[''Нефть'', ''Уголь'', ''Зерно'', ''Лес'']}
    \end{center}
    Конструктор должен иметь сигнатуру: \texttt{\_\_init\_\_(self) -> None}.

    \item Создайте класс \texttt{Fleet}, который будет представлять собой флотилию, состоящую из моделей судов. В конструкторе класса \texttt{Fleet} инициализируйте список судов \texttt{self.train: list[Ship]} длиной 56.

    \item Добавьте метод \texttt{shuffle(self) -> None} в класс \texttt{Fleet}, который будет перемешивать суда в списке \texttt{self.train}.

    \item Добавьте метод \texttt{get(self, i: int) -> Ship}, который будет возвращать $i$-е судно и его груз из списка \texttt{self.train}.

    \item Создайте экземпляр класса \texttt{Fleet} и вызовите метод \texttt{shuffle} для перемешивания судов.

    \item Создайте цикл, который будет запрашивать у пользователя номер судна из флотилии и выводить информацию о выбранном судне.

    \item Повторите шаги 5–6 до тех пор, пока пользователь не выберет все суда или не завершит выбор.

    \item В конце программы выводите сообщение о завершении выбора судов.

    \item Убедитесь, что пользователь вводит корректные номера судов и что программа обрабатывает ошибки, связанные с вводом пользователя.

    \item Проверьте работу программы, используя различные комбинации номеров судов и грузов.
\end{enumerate}

\item[7] \textbf{Формирование состава ракет-носителей}
\begin{enumerate}
    \item Создайте класс \texttt{Rocket}, который будет представлять собой ракету с полезной нагрузкой. В конструкторе класса \texttt{Rocket} инициализируйте значения ракеты и нагрузки из списков \texttt{NumList} и \texttt{MasList}, которые объявлены как общие атрибуты класса. \texttt{NumList: list[str]} — это список ракет (не менее 14): 
    \begin{center}
        \texttt{[''Ракета\_1'', ''Ракета\_2'', \dots, ''Ракета\_14'']}
    \end{center}
    \texttt{MasList: list[str]} — это список типов нагрузки (не менее 4):
    \begin{center}
        \texttt{[''Спутник'', ''Грузовой модуль'', ''Экипаж'', ''Научное оборудование'']}
    \end{center}
    Конструктор должен иметь сигнатуру: \texttt{\_\_init\_\_(self) -> None}.

    \item Создайте класс \texttt{RocketTrain}, который будет представлять собой состав ракет. В конструкторе класса \texttt{RocketTrain} инициализируйте список ракет \texttt{self.train: list[Rocket]} длиной 56.

    \item Добавьте метод \texttt{shuffle(self) -> None} в класс \texttt{RocketTrain}, который будет перемешивать ракеты в списке \texttt{self.train}.

    \item Добавьте метод \texttt{get(self, i: int) -> Rocket}, который будет возвращать $i$-ю ракету и её нагрузку из списка \texttt{self.train}.

    \item Создайте экземпляр класса \texttt{RocketTrain} и вызовите метод \texttt{shuffle} для перемешивания ракет.

    \item Создайте цикл, который будет запрашивать у пользователя номер ракеты и выводить информацию о выбранной ракете.

    \item Повторите шаги 5–6 до тех пор, пока пользователь не выберет все ракеты или не завершит выбор.

    \item В конце программы выводите сообщение о завершении выбора ракет.

    \item Убедитесь, что пользователь вводит корректные номера ракет и что программа обрабатывает ошибки, связанные с вводом пользователя.

    \item Проверьте работу программы, используя различные комбинации номеров ракет и нагрузок.
\end{enumerate}

\item[8] \textbf{Формирование состава дронов с грузом}
\begin{enumerate}
    \item Создайте класс \texttt{Drone}, который будет представлять собой дрон с миссией. В конструкторе класса \texttt{Drone} инициализируйте значения дрона и миссии из списков \texttt{NumList} и \texttt{MasList}, которые объявлены как общие атрибуты класса. \texttt{NumList: list[str]} — это список дронов (не менее 14): 
    \begin{center}
        \texttt{[''Дрон\_1'', ''Дрон\_2'', \dots, ''Дрон\_14'']}
    \end{center}
    \texttt{MasList: list[str]} — это список миссий (не менее 4):
    \begin{center}
        \texttt{[''Фотосъёмка'', ''Доставка'', ''Разведка'', ''Мониторинг'']}
    \end{center}
    Конструктор должен иметь сигнатуру: \texttt{\_\_init\_\_(self) -> None}.

    \item Создайте класс \texttt{DroneSquadron}, который будет представлять собой эскадрилью, состоящую из моделей дронов. В конструкторе класса \texttt{DroneSquadron} инициализируйте список дронов \texttt{self.train: list[Drone]} длиной 56.

    \item Добавьте метод \texttt{shuffle(self) -> None} в класс \texttt{DroneSquadron}, который будет перемешивать дроны в списке \texttt{self.train}.

    \item Добавьте метод \texttt{get(self, i: int) -> Drone}, который будет возвращать $i$-й дрон и его миссию из списка \texttt{self.train}.

    \item Создайте экземпляр класса \texttt{DroneSquadron} и вызовите метод \texttt{shuffle} для перемешивания дронов.

    \item Создайте цикл, который будет запрашивать у пользователя номер дрона и выводить информацию о выбранном дроне.

    \item Повторите шаги 5–6 до тех пор, пока пользователь не выберет все дроны или не завершит выбор.

    \item В конце программы выводите сообщение о завершении выбора дронов.

    \item Убедитесь, что пользователь вводит корректные номера дронов и что программа обрабатывает ошибки, связанные с вводом пользователя.

    \item Проверьте работу программы, используя различные комбинации номеров дронов и миссий.
\end{enumerate}

\item[9] \textbf{Формирование состава тележек в супермаркете}
\begin{enumerate}
    \item Создайте класс \texttt{Trolley}, который будет представлять собой тележку с типом покупателя. В конструкторе класса \texttt{Trolley} инициализируйте значения тележки и типа покупателя из списков \texttt{NumList} и \texttt{MasList}, которые объявлены как общие атрибуты класса. \texttt{NumList: list[str]} — это список тележек (не менее 14): 
    \begin{center}
        \texttt{[''Тележка\_1'', ''Тележка\_2'', \dots, ''Тележка\_14'']}
    \end{center}
    \texttt{MasList: list[str]} — это список типов покупателей (не менее 4):
    \begin{center}
        \texttt{[''Семья'', ''Студент'', ''Пенсионер'', ''Турист'']}
    \end{center}
    Конструктор должен иметь сигнатуру: \texttt{\_\_init\_\_(self) -> None}.

    \item Создайте класс \texttt{TrolleyTrain}, который будет представлять собой состав тележек. В конструкторе класса \texttt{TrolleyTrain} инициализируйте список тележек \texttt{self.train: list[Trolley]} длиной 56.

    \item Добавьте метод \texttt{shuffle(self) -> None} в класс \texttt{TrolleyTrain}, который будет перемешивать тележки в списке \texttt{self.train}.

    \item Добавьте метод \texttt{get(self, i: int) -> Trolley}, который будет возвращать $i$-ю тележку и тип её покупателя из списка \texttt{self.train}.

    \item Создайте экземпляр класса \texttt{TrolleyTrain} и вызовите метод \texttt{shuffle} для перемешивания тележек.

    \item Создайте цикл, который будет запрашивать у пользователя номер тележки и выводить информацию о ней.

    \item Повторите шаги 5–6 до тех пор, пока пользователь не выберет все тележки или не завершит выбор.

    \item В конце программы выводите сообщение о завершении выбора тележек.

    \item Убедитесь, что пользователь вводит корректные номера тележек и что программа обрабатывает ошибки, связанные с вводом пользователя.

    \item Проверьте работу программы, используя различные комбинации номеров тележек и типов покупателей.
\end{enumerate}

\item[10] \textbf{Формирование состава камер хранения}
\begin{enumerate}
    \item Создайте класс \texttt{Locker}, который будет представлять собой камеру хранения с содержимым. В конструкторе класса \texttt{Locker} инициализируйте значения камеры и содержимого из списков \texttt{NumList} и \texttt{MasList}, которые объявлены как общие атрибуты класса. \texttt{NumList: list[str]} — это список камер (не менее 14): 
    \begin{center}
        \texttt{[''Камера\_1'', ''Камера\_2'', \dots, ''Камера\_14'']}
    \end{center}
    \texttt{MasList: list[str]} — это список типов содержимого (не менее 4):
    \begin{center}
        \texttt{[''Велосипед'', ''Чемодан'', ''Инструменты'', ''Спортинвентарь'']}
    \end{center}
    Конструктор должен иметь сигнатуру: \texttt{\_\_init\_\_(self) -> None}.

    \item Создайте класс \texttt{StorageTrain}, который будет представлять собой состав камер хранения. В конструкторе класса \texttt{StorageTrain} инициализируйте список камер \texttt{self.train: list[Locker]} длиной 56.

    \item Добавьте метод \texttt{shuffle(self) -> None} в класс \texttt{StorageTrain}, который будет перемешивать камеры в списке \texttt{self.train}.

    \item Добавьте метод \texttt{get(self, i: int) -> Locker}, который будет возвращать $i$-ю камеру и её содержимое из списка \texttt{self.train}.

    \item Создайте экземпляр класса \texttt{StorageTrain} и вызовите метод \texttt{shuffle} для перемешивания камер.

    \item Создайте цикл, который будет запрашивать у пользователя номер камеры и выводить информацию о ней.

    \item Повторите шаги 5–6 до тех пор, пока пользователь не выберет все камеры или не завершит выбор.

    \item В конце программы выводите сообщение о завершении выбора камер.

    \item Убедитесь, что пользователь вводит корректные номера камер и что программа обрабатывает ошибки, связанные с вводом пользователя.

    \item Проверьте работу программы, используя различные комбинации номеров камер и содержимого.
\end{enumerate}

\item[11] \textbf{Формирование состава самолётов с бортами}
\begin{enumerate}
    \item Создайте класс \texttt{Aircraft}, который будет представлять собой самолёт с типом рейса. В конструкторе класса \texttt{Aircraft} инициализируйте значения самолёта и типа рейса из списков \texttt{NumList} и \texttt{MasList}, которые объявлены как общие атрибуты класса. \texttt{NumList: list[str]} — это список самолётов (не менее 14): 
    \begin{center}
        \texttt{[''Борт\_1'', ''Борт\_2'', \dots, ''Борт\_14'']}
    \end{center}
    \texttt{MasList: list[str]} — это список типов рейсов (не менее 4):
    \begin{center}
        \texttt{[''Пассажирский'', ''Грузовой'', ''Военный'', ''Санитарный'']}
    \end{center}
    Конструктор должен иметь сигнатуру: \texttt{\_\_init\_\_(self) -> None}.

    \item Создайте класс \texttt{AirFleet}, который будет представлять собой воздушный флот, состоящий из моделей самолётов. В конструкторе класса \texttt{AirFleet} инициализируйте список самолётов \texttt{self.train: list[Aircraft]} длиной 56.

    \item Добавьте метод \texttt{shuffle(self) -> None} в класс \texttt{AirFleet}, который будет перемешивать самолёты в списке \texttt{self.train}.

    \item Добавьте метод \texttt{get(self, i: int) -> Aircraft}, который будет возвращать $i$-й самолёт и его тип рейса из списка \texttt{self.train}.

    \item Создайте экземпляр класса \texttt{AirFleet} и вызовите метод \texttt{shuffle} для перемешивания самолётов.

    \item Создайте цикл, который будет запрашивать у пользователя номер самолёта и выводить информацию о нём.

    \item Повторите шаги 5–6 до тех пор, пока пользователь не выберет все самолёты или не завершит выбор.

    \item В конце программы выводите сообщение о завершении выбора самолётов.

    \item Убедитесь, что пользователь вводит корректные номера самолётов и что программа обрабатывает ошибки, связанные с вводом пользователя.

    \item Проверьте работу программы, используя различные комбинации номеров самолётов и типов рейсов.
\end{enumerate}

\item[12] \textbf{Формирование состава танкеров с жидкостями}
\begin{enumerate}
    \item Создайте класс \texttt{Tanker}, который будет представлять собой танкер с жидкостью. В конструкторе класса \texttt{Tanker} инициализируйте значения танкера и жидкости из списков \texttt{NumList} и \texttt{MasList}, которые объявлены как общие атрибуты класса. \texttt{NumList: list[str]} — это список танкеров (не менее 14): 
    \begin{center}
        \texttt{[''Танкер\_1'', ''Танкер\_2'', \dots, ''Танкер\_14'']}
    \end{center}
    \texttt{MasList: list[str]} — это список жидкостей (не менее 4):
    \begin{center}
        \texttt{[''Вода'', ''Молоко'', ''Топливо'', ''Химикаты'']}
    \end{center}
    Конструктор должен иметь сигнатуру: \texttt{\_\_init\_\_(self) -> None}.

    \item Создайте класс \texttt{TankerConvoy}, который будет представлять собой конвой танкеров. В конструкторе класса \texttt{TankerConvoy} инициализируйте список танкеров \texttt{self.train: list[Tanker]} длиной 56.

    \item Добавьте метод \texttt{shuffle(self) -> None} в класс \texttt{TankerConvoy}, который будет перемешивать танкеры в списке \texttt{self.train}.

    \item Добавьте метод \texttt{get(self, i: int) -> Tanker}, который будет возвращать $i$-й танкер и его жидкость из списка \texttt{self.train}.

    \item Создайте экземпляр класса \texttt{TankerConvoy} и вызовите метод \texttt{shuffle} для перемешивания танкеров.

    \item Создайте цикл, который будет запрашивать у пользователя номер танкера и выводить информацию о нём.

    \item Повторите шаги 5–6 до тех пор, пока пользователь не выберет все танкеры или не завершит выбор.

    \item В конце программы выводите сообщение о завершении выбора танкеров.

    \item Убедитесь, что пользователь вводит корректные номера танкеров и что программа обрабатывает ошибки, связанные с вводом пользователя.

    \item Проверьте работу программы, используя различные комбинации номеров танкеров и жидкостей.
\end{enumerate}

\item[13] \textbf{Формирование состава паллет на складе}
\begin{enumerate}
    \item Создайте класс \texttt{Pallet}, который будет представлять собой паллету с товаром. В конструкторе класса \texttt{Pallet} инициализируйте значения паллеты и товара из списков \texttt{NumList} и \texttt{MasList}, которые объявлены как общие атрибуты класса. \texttt{NumList: list[str]} — это список паллет (не менее 14): 
    \begin{center}
        \texttt{[''Паллета\_1'', ''Паллета\_2'', \dots, ''Паллета\_14'']}
    \end{center}
    \texttt{MasList: list[str]} — это список типов товаров (не менее 4):
    \begin{center}
        \texttt{[''Напитки'', ''Консервы'', ''Бытовая химия'', ''Бумажная продукция'']}
    \end{center}
    Конструктор должен иметь сигнатуру: \texttt{\_\_init\_\_(self) -> None}.

    \item Создайте класс \texttt{PalletTrain}, который будет представлять собой состав паллет. В конструкторе класса \texttt{PalletTrain} инициализируйте список паллет \texttt{self.train: list[Pallet]} длиной 56.

    \item Добавьте метод \texttt{shuffle(self) -> None} в класс \texttt{PalletTrain}, который будет перемешивать паллеты в списке \texttt{self.train}.

    \item Добавьте метод \texttt{get(self, i: int) -> Pallet}, который будет возвращать $i$-ю паллету и её товар из списка \texttt{self.train}.

    \item Создайте экземпляр класса \texttt{PalletTrain} и вызовите метод \texttt{shuffle} для перемешивания паллет.

    \item Создайте цикл, который будет запрашивать у пользователя номер паллеты и выводить информацию о ней.

    \item Повторите шаги 5–6 до тех пор, пока пользователь не выберет все паллеты или не завершит выбор.

    \item В конце программы выводите сообщение о завершении выбора паллет.

    \item Убедитесь, что пользователь вводит корректные номера паллет и что программа обрабатывает ошибки, связанные с вводом пользователя.

    \item Проверьте работу программы, используя различные комбинации номеров паллет и товаров.
\end{enumerate}

\item[14] \textbf{Формирование состава вагонов-цистерн}
\begin{enumerate}
    \item Создайте класс \texttt{TankWagon}, который будет представлять собой цистерну с содержимым. В конструкторе класса \texttt{TankWagon} инициализируйте значения цистерны и содержимого из списков \texttt{NumList} и \texttt{MasList}, которые объявлены как общие атрибуты класса. \texttt{NumList: list[str]} — это список цистерн (не менее 14): 
    \begin{center}
        \texttt{[''Цистерна\_1'', ''Цистерна\_2'', \dots, ''Цистерна\_14'']}
    \end{center}
    \texttt{MasList: list[str]} — это список типов содержимого (не менее 4):
    \begin{center}
        \texttt{[''Бензин'', ''Дизель'', ''Газ'', ''Вода'']}
    \end{center}
    Конструктор должен иметь сигнатуру: \texttt{\_\_init\_\_(self) -> None}.

    \item Создайте класс \texttt{TankTrain}, который будет представлять собой состав цистерн. В конструкторе класса \texttt{TankTrain} инициализируйте список цистерн \texttt{self.train: list[TankWagon]} длиной 56.

    \item Добавьте метод \texttt{shuffle(self) -> None} в класс \texttt{TankTrain}, который будет перемешивать цистерны в списке \texttt{self.train}.

    \item Добавьте метод \texttt{get(self, i: int) -> TankWagon}, который будет возвращать $i$-ю цистерну и её содержимое из списка \texttt{self.train}.

    \item Создайте экземпляр класса \texttt{TankTrain} и вызовите метод \texttt{shuffle} для перемешивания цистерн.

    \item Создайте цикл, который будет запрашивать у пользователя номер цистерны и выводить информацию о ней.

    \item Повторите шаги 5–6 до тех пор, пока пользователь не выберет все цистерны или не завершит выбор.

    \item В конце программы выводите сообщение о завершении выбора цистерн.

    \item Убедитесь, что пользователь вводит корректные номера цистерн и что программа обрабатывает ошибки, связанные с вводом пользователя.

    \item Проверьте работу программы, используя различные комбинации номеров цистерн и содержимого.
\end{enumerate}

\item[15] \textbf{Формирование состава промышленных роботов}
\begin{enumerate}
    \item Создайте класс \texttt{Robot}, который будет представлять собой робота с модулем. В конструкторе класса \texttt{Robot} инициализируйте значения робота и модуля из списков \texttt{NumList} и \texttt{MasList}, которые объявлены как общие атрибуты класса. \texttt{NumList: list[str]} — это список роботов (не менее 14): 
    \begin{center}
        \texttt{[''Робот\_1'', ''Робот\_2'', \dots, ''Робот\_14'']}
    \end{center}
    \texttt{MasList: list[str]} — это список модулей (не менее 4):
    \begin{center}
        \texttt{[''Манипулятор'', ''Камера'', ''Сенсор'', ''Батарея'']}
    \end{center}
    Конструктор должен иметь сигнатуру: \texttt{\_\_init\_\_(self) -> None}.

    \item Создайте класс \texttt{RobotLine}, который будет представлять собой производственную линию роботов. В конструкторе класса \texttt{RobotLine} инициализируйте список роботов \texttt{self.train: list[Robot]} длиной 56.

    \item Добавьте метод \texttt{shuffle(self) -> None} в класс \texttt{RobotLine}, который будет перемешивать роботов в списке \texttt{self.train}.

    \item Добавьте метод \texttt{get(self, i: int) -> Robot}, который будет возвращать $i$-го робота и его модуль из списка \texttt{self.train}.

    \item Создайте экземпляр класса \texttt{RobotLine} и вызовите метод \texttt{shuffle} для перемешивания роботов.

    \item Создайте цикл, который будет запрашивать у пользователя номер робота и выводить информацию о нём.

    \item Повторите шаги 5–6 до тех пор, пока пользователь не выберет всех роботов или не завершит выбор.

    \item В конце программы выводите сообщение о завершении выбора роботов.

    \item Убедитесь, что пользователь вводит корректные номера роботов и что программа обрабатывает ошибки, связанные с вводом пользователя.

    \item Проверьте работу программы, используя различные комбинации номеров роботов и модулей.
\end{enumerate}

\item[16] \textbf{Формирование состава клеток с животными}
\begin{enumerate}
    \item Создайте класс \texttt{Cage}, который будет представлять собой клетку с животным. В конструкторе класса \texttt{Cage} инициализируйте значения клетки и животного из списков \texttt{NumList} и \texttt{MasList}, которые объявлены как общие атрибуты класса. \texttt{NumList: list[str]} — это список клеток (не менее 14): 
    \begin{center}
        \texttt{[''Клетка\_1'', ''Клетка\_2'', \dots, ''Клетка\_14'']}
    \end{center}
    \texttt{MasList: list[str]} — это список животных (не менее 4):
    \begin{center}
        \texttt{[''Собака'', ''Кошка'', ''Попугай'', ''Кролик'']}
    \end{center}
    Конструктор должен иметь сигнатуру: \texttt{\_\_init\_\_(self) -> None}.

    \item Создайте класс \texttt{ZooTrain}, который будет представлять собой состав клеток. В конструкторе класса \texttt{ZooTrain} инициализируйте список клеток \texttt{self.train: list[Cage]} длиной 56.

    \item Добавьте метод \texttt{shuffle(self) -> None} в класс \texttt{ZooTrain}, который будет перемешивать клетки в списке \texttt{self.train}.

    \item Добавьте метод \texttt{get(self, i: int) -> Cage}, который будет возвращать $i$-ю клетку и её животное из списка \texttt{self.train}.

    \item Создайте экземпляр класса \texttt{ZooTrain} и вызовите метод \texttt{shuffle} для перемешивания клеток.

    \item Создайте цикл, который будет запрашивать у пользователя номер клетки и выводить информацию о ней.

    \item Повторите шаги 5–6 до тех пор, пока пользователь не выберет все клетки или не завершит выбор.

    \item В конце программы выводите сообщение о завершении выбора клеток.

    \item Убедитесь, что пользователь вводит корректные номера клеток и что программа обрабатывает ошибки, связанные с вводом пользователя.

    \item Проверьте работу программы, используя различные комбинации номеров клеток и животных.
\end{enumerate}

\item[17] \textbf{Формирование состава прицепов на автодороге}
\begin{enumerate}
    \item Создайте класс \texttt{Trailer}, который будет представлять собой прицеп с грузом. В конструкторе класса \texttt{Trailer} инициализируйте значения прицепа и груза из списков \texttt{NumList} и \texttt{MasList}, которые объявлены как общие атрибуты класса. \texttt{NumList: list[str]} — это список прицепов (не менее 14): 
    \begin{center}
        \texttt{[''Прицеп\_1'', ''Прицеп\_2'', \dots, ''Прицеп\_14'']}
    \end{center}
    \texttt{MasList: list[str]} — это список типов груза (не менее 4):
    \begin{center}
        \texttt{[''Строительные материалы'', ''Мебель'', ''Техника'', ''Сельхозпродукция'']}
    \end{center}
    Конструктор должен иметь сигнатуру: \texttt{\_\_init\_\_(self) -> None}.

    \item Создайте класс \texttt{TrailerConvoy}, который будет представлять собой конвой прицепов. В конструкторе класса \texttt{TrailerConvoy} инициализируйте список прицепов \texttt{self.train: list[Trailer]} длиной 56.

    \item Добавьте метод \texttt{shuffle(self) -> None} в класс \texttt{TrailerConvoy}, который будет перемешивать прицепы в списке \texttt{self.train}.

    \item Добавьте метод \texttt{get(self, i: int) -> Trailer}, который будет возвращать $i$-й прицеп и его груз из списка \texttt{self.train}.

    \item Создайте экземпляр класса \texttt{TrailerConvoy} и вызовите метод \texttt{shuffle} для перемешивания прицепов.

    \item Создайте цикл, который будет запрашивать у пользователя номер прицепа и выводить информацию о нём.

    \item Повторите шаги 5–6 до тех пор, пока пользователь не выберет все прицепы или не завершит выбор.

    \item В конце программы выводите сообщение о завершении выбора прицепов.

    \item Убедитесь, что пользователь вводит корректные номера прицепов и что программа обрабатывает ошибки, связанные с вводом пользователя.

    \item Проверьте работу программы, используя различные комбинации номеров прицепов и грузов.
\end{enumerate}

\item[18] \textbf{Формирование состава морских контейнеровозов}
\begin{enumerate}
    \item Создайте класс \texttt{Vessel}, который будет представлять собой контейнеровоз с типом контейнера. В конструкторе класса \texttt{Vessel} инициализируйте значения судна и типа контейнера из списков \texttt{NumList} и \texttt{MasList}, которые объявлены как общие атрибуты класса. \texttt{NumList: list[str]} — это список судов (не менее 14): 
    \begin{center}
        \texttt{[''Корабль\_1'', ''Корабль\_2'', \dots, ''Корабль\_14'']}
    \end{center}
    \texttt{MasList: list[str]} — это список типов контейнеров (не менее 4):
    \begin{center}
        \texttt{[''20ft'', ''40ft'', ''Рефрижератор'', ''Открытый'']}
    \end{center}
    Конструктор должен иметь сигнатуру: \texttt{\_\_init\_\_(self) -> None}.

    \item Создайте класс \texttt{VesselFleet}, который будет представлять собой флот контейнеровозов. В конструкторе класса \texttt{VesselFleet} инициализируйте список судов \texttt{self.train: list[Vessel]} длиной 56.

    \item Добавьте метод \texttt{shuffle(self) -> None} в класс \texttt{VesselFleet}, который будет перемешивать суда в списке \texttt{self.train}.

    \item Добавьте метод \texttt{get(self, i: int) -> Vessel}, который будет возвращать $i$-е судно и тип его контейнера из списка \texttt{self.train}.

    \item Создайте экземпляр класса \texttt{VesselFleet} и вызовите метод \texttt{shuffle} для перемешивания судов.

    \item Создайте цикл, который будет запрашивать у пользователя номер судна и выводить информацию о нём.

    \item Повторите шаги 5–6 до тех пор, пока пользователь не выберет все суда или не завершит выбор.

    \item В конце программы выводите сообщение о завершении выбора судов.

    \item Убедитесь, что пользователь вводит корректные номера судов и что программа обрабатывает ошибки, связанные с вводом пользователя.

    \item Проверьте работу программы, используя различные комбинации номеров судов и типов контейнеров.
\end{enumerate}

\item[19] \textbf{Формирование состава банковских сейфов}
\begin{enumerate}
    \item Создайте класс \texttt{Safe}, который будет представлять собой сейф с содержимым. В конструкторе класса \texttt{Safe} инициализируйте значения сейфа и содержимого из списков \texttt{NumList} и \texttt{MasList}, которые объявлены как общие атрибуты класса. \texttt{NumList: list[str]} — это список сейфов (не менее 14): 
    \begin{center}
        \texttt{[''Сейф\_1'', ''Сейф\_2'', \dots, ''Сейф\_14'']}
    \end{center}
    \texttt{MasList: list[str]} — это список типов содержимого (не менее 4):
    \begin{center}
        \texttt{[''Документы'', ''Драгоценности'', ''Деньги'', ''Антиквариат'']}
    \end{center}
    Конструктор должен иметь сигнатуру: \texttt{\_\_init\_\_(self) -> None}.

    \item Создайте класс \texttt{VaultTrain}, который будет представлять собой состав сейфов. В конструкторе класса \texttt{VaultTrain} инициализируйте список сейфов \texttt{self.train: list[Safe]} длиной 56.

    \item Добавьте метод \texttt{shuffle(self) -> None} в класс \texttt{VaultTrain}, который будет перемешивать сейфы в списке \texttt{self.train}.

    \item Добавьте метод \texttt{get(self, i: int) -> Safe}, который будет возвращать $i$-й сейф и его содержимое из списка \texttt{self.train}.

    \item Создайте экземпляр класса \texttt{VaultTrain} и вызовите метод \texttt{shuffle} для перемешивания сейфов.

    \item Создайте цикл, который будет запрашивать у пользователя номер сейфа и выводить информацию о нём.

    \item Повторите шаги 5–6 до тех пор, пока пользователь не выберет все сейфы или не завершит выбор.

    \item В конце программы выводите сообщение о завершении выбора сейфов.

    \item Убедитесь, что пользователь вводит корректные номера сейфов и что программа обрабатывает ошибки, связанные с вводом пользователя.

    \item Проверьте работу программы, используя различные комбинации номеров сейфов и содержимого.
\end{enumerate}

\item[20] \textbf{Формирование состава капсул экспресс-доставки}
\begin{enumerate}
    \item Создайте класс \texttt{Capsule}, который будет представлять собой капсулу с грузом. В конструкторе класса \texttt{Capsule} инициализируйте значения капсулы и груза из списков \texttt{NumList} и \texttt{MasList}, которые объявлены как общие атрибуты класса. \texttt{NumList: list[str]} — это список капсул (не менее 14): 
    \begin{center}
        \texttt{[''Капсула\_1'', ''Капсула\_2'', \dots, ''Капсула\_14'']}
    \end{center}
    \texttt{MasList: list[str]} — это список типов груза (не менее 4):
    \begin{center}
        \texttt{[''Медикаменты'', ''Еда'', ''Посылки'', ''Образцы'']}
    \end{center}
    Конструктор должен иметь сигнатуру: \texttt{\_\_init\_\_(self) -> None}.

    \item Создайте класс \texttt{CapsuleTrain}, который будет представлять собой состав капсул. В конструкторе класса \texttt{CapsuleTrain} инициализируйте список капсул \texttt{self.train: list[Capsule]} длиной 56.

    \item Добавьте метод \texttt{shuffle(self) -> None} в класс \texttt{CapsuleTrain}, который будет перемешивать капсулы в списке \texttt{self.train}.

    \item Добавьте метод \texttt{get(self, i: int) -> Capsule}, который будет возвращать $i$-ю капсулу и её груз из списка \texttt{self.train}.

    \item Создайте экземпляр класса \texttt{CapsuleTrain} и вызовите метод \texttt{shuffle} для перемешивания капсул.

    \item Создайте цикл, который будет запрашивать у пользователя номер капсулы и выводить информацию о ней.

    \item Повторите шаги 5–6 до тех пор, пока пользователь не выберет все капсулы или не завершит выбор.

    \item В конце программы выводите сообщение о завершении выбора капсул.

    \item Убедитесь, что пользователь вводит корректные номера капсул и что программа обрабатывает ошибки, связанные с вводом пользователя.

    \item Проверьте работу программы, используя различные комбинации номеров капсул и грузов.
\end{enumerate}

\item[21] \textbf{Формирование состава тележек в аэропорту}
\begin{enumerate}
    \item Создайте класс \texttt{Trolley}, который будет представлять собой тележку с типом пассажира. В конструкторе класса \texttt{Trolley} инициализируйте значения тележки и типа пассажира из списков \texttt{NumList} и \texttt{MasList}, которые объявлены как общие атрибуты класса. \texttt{NumList: list[str]} — это список тележек (не менее 14): 
    \begin{center}
        \texttt{[''Тележка\_1'', ''Тележка\_2'', \dots, ''Тележка\_14'']}
    \end{center}
    \texttt{MasList: list[str]} — это список типов пассажиров (не менее 4):
    \begin{center}
        \texttt{[''Бизнес'', ''Эконом'', ''Первый класс'', ''Транзит'']}
    \end{center}
    Конструктор должен иметь сигнатуру: \texttt{\_\_init\_\_(self) -> None}.

    \item Создайте класс \texttt{AirportTrolleyTrain}, который будет представлять собой состав тележек. В конструкторе класса \texttt{AirportTrolleyTrain} инициализируйте список тележек \texttt{self.train: list[Trolley]} длиной 56.

    \item Добавьте метод \texttt{shuffle(self) -> None} в класс \texttt{AirportTrolleyTrain}, который будет перемешивать тележки в списке \texttt{self.train}.

    \item Добавьте метод \texttt{get(self, i: int) -> Trolley}, который будет возвращать $i$-ю тележку и тип её пассажира из списка \texttt{self.train}.

    \item Создайте экземпляр класса \texttt{AirportTrolleyTrain} и вызовите метод \texttt{shuffle} для перемешивания тележек.

    \item Создайте цикл, который будет запрашивать у пользователя номер тележки и выводить информацию о ней.

    \item Повторите шаги 5–6 до тех пор, пока пользователь не выберет все тележки или не завершит выбор.

    \item В конце программы выводите сообщение о завершении выбора тележек.

    \item Убедитесь, что пользователь вводит корректные номера тележек и что программа обрабатывает ошибки, связанные с вводом пользователя.

    \item Проверьте работу программы, используя различные комбинации номеров тележек и типов пассажиров.
\end{enumerate}

\item[22] \textbf{Формирование состава мобильных платформ с оборудованием}
\begin{enumerate}
    \item Создайте класс \texttt{Platform}, который будет представлять собой платформу с оборудованием. В конструкторе класса \texttt{Platform} инициализируйте значения платформы и оборудования из списков \texttt{NumList} и \texttt{MasList}, которые объявлены как общие атрибуты класса. \texttt{NumList: list[str]} — это список платформ (не менее 14): 
    \begin{center}
        \texttt{[''Платформа\_1'', ''Платформа\_2'', \dots, ''Платформа\_14'']}
    \end{center}
    \texttt{MasList: list[str]} — это список типов оборудования (не менее 4):
    \begin{center}
        \texttt{[''Генератор'', ''Компрессор'', ''Насос'', ''Сварочный аппарат'']}
    \end{center}
    Конструктор должен иметь сигнатуру: \texttt{\_\_init\_\_(self) -> None}.

    \item Создайте класс \texttt{PlatformTrain}, который будет представлять собой состав платформ. В конструкторе класса \texttt{PlatformTrain} инициализируйте список платформ \texttt{self.train: list[Platform]} длиной 56.

    \item Добавьте метод \texttt{shuffle(self) -> None} в класс \texttt{PlatformTrain}, который будет перемешивать платформы в списке \texttt{self.train}.

    \item Добавьте метод \texttt{get(self, i: int) -> Platform}, который будет возвращать $i$-ю платформу и её оборудование из списка \texttt{self.train}.

    \item Создайте экземпляр класса \texttt{PlatformTrain} и вызовите метод \texttt{shuffle} для перемешивания платформ.

    \item Создайте цикл, который будет запрашивать у пользователя номер платформы и выводить информацию о ней.

    \item Повторите шаги 5–6 до тех пор, пока пользователь не выберет все платформы или не завершит выбор.

    \item В конце программы выводите сообщение о завершении выбора платформ.

    \item Убедитесь, что пользователь вводит корректные номера платформ и что программа обрабатывает ошибки, связанные с вводом пользователя.

    \item Проверьте работу программы, используя различные комбинации номеров платформ и оборудования.
\end{enumerate}

\item[23] \textbf{Формирование состава ящиков с инструментами}
\begin{enumerate}
    \item Создайте класс \texttt{Toolbox}, который будет представлять собой ящик с набором инструментов. В конструкторе класса \texttt{Toolbox} инициализируйте значения ящика и набора из списков \texttt{NumList} и \texttt{MasList}, которые объявлены как общие атрибуты класса. \texttt{NumList: list[str]} — это список ящиков (не менее 14): 
    \begin{center}
        \texttt{[''Ящик\_1'', ''Ящик\_2'', \dots, ''Ящик\_14'']}
    \end{center}
    \texttt{MasList: list[str]} — это список типов наборов (не менее 4):
    \begin{center}
        \texttt{[''Слесарный'', ''Электромонтажный'', ''Столярный'', ''Автомобильный'']}
    \end{center}
    Конструктор должен иметь сигнатуру: \texttt{\_\_init\_\_(self) -> None}.

    \item Создайте класс \texttt{ToolTrain}, который будет представлять собой состав ящиков. В конструкторе класса \texttt{ToolTrain} инициализируйте список ящиков \texttt{self.train: list[Toolbox]} длиной 56.

    \item Добавьте метод \texttt{shuffle(self) -> None} в класс \texttt{ToolTrain}, который будет перемешивать ящики в списке \texttt{self.train}.

    \item Добавьте метод \texttt{get(self, i: int) -> Toolbox}, который будет возвращать $i$-й ящик и его набор инструментов из списка \texttt{self.train}.

    \item Создайте экземпляр класса \texttt{ToolTrain} и вызовите метод \texttt{shuffle} для перемешивания ящиков.

    \item Создайте цикл, который будет запрашивать у пользователя номер ящика и выводить информацию о нём.

    \item Повторите шаги 5–6 до тех пор, пока пользователь не выберет все ящики или не завершит выбор.

    \item В конце программы выводите сообщение о завершении выбора ящиков.

    \item Убедитесь, что пользователь вводит корректные номера ящиков и что программа обрабатывает ошибки, связанные с вводом пользователя.

    \item Проверьте работу программы, используя различные комбинации номеров ящиков и наборов инструментов.
\end{enumerate}

\item[24] \textbf{Формирование состава подводных аппаратов}
\begin{enumerate}
    \item Создайте класс \texttt{Submersible}, который будет представлять собой аппарат с миссией. В конструкторе класса \texttt{Submersible} инициализируйте значения аппарата и миссии из списков \texttt{NumList} и \texttt{MasList}, которые объявлены как общие атрибуты класса. \texttt{NumList: list[str]} — это список аппаратов (не менее 14): 
    \begin{center}
        \texttt{[''Аппарат\_1'', ''Аппарат\_2'', \dots, ''Аппарат\_14'']}
    \end{center}
    \texttt{MasList: list[str]} — это список миссий (не менее 4):
    \begin{center}
        \texttt{[''Исследование'', ''Спасение'', ''Инспекция'', ''Добыча'']}
    \end{center}
    Конструктор должен иметь сигнатуру: \texttt{\_\_init\_\_(self) -> None}.

    \item Создайте класс \texttt{SubmersibleSquadron}, который будет представлять собой эскадрилью аппаратов. В конструкторе класса \texttt{SubmersibleSquadron} инициализируйте список аппаратов \texttt{self.train: list[Submersible]} длиной 56.

    \item Добавьте метод \texttt{shuffle(self) -> None} в класс \texttt{SubmersibleSquadron}, который будет перемешивать аппараты в списке \texttt{self.train}.

    \item Добавьте метод \texttt{get(self, i: int) -> Submersible}, который будет возвращать $i$-й аппарат и его миссию из списка \texttt{self.train}.

    \item Создайте экземпляр класса \texttt{SubmersibleSquadron} и вызовите метод \texttt{shuffle} для перемешивания аппаратов.

    \item Создайте цикл, который будет запрашивать у пользователя номер аппарата и выводить информацию о нём.

    \item Повторите шаги 5–6 до тех пор, пока пользователь не выберет все аппараты или не завершит выбор.

    \item В конце программы выводите сообщение о завершении выбора аппаратов.

    \item Убедитесь, что пользователь вводит корректные номера аппаратов и что программа обрабатывает ошибки, связанные с вводом пользователя.

    \item Проверьте работу программы, используя различные комбинации номеров аппаратов и миссий.
\end{enumerate}

\item[25] \textbf{Формирование состава контейнеров с растениями}
\begin{enumerate}
    \item Создайте класс \texttt{Planter}, который будет представлять собой контейнер с растением. В конструкторе класса \texttt{Planter} инициализируйте значения контейнера и растения из списков \texttt{NumList} и \texttt{MasList}, которые объявлены как общие атрибуты класса. \texttt{NumList: list[str]} — это список контейнеров (не менее 14): 
    \begin{center}
        \texttt{[''Контейнер\_1'', ''Контейнер\_2'', \dots, ''Контейнер\_14'']}
    \end{center}
    \texttt{MasList: list[str]} — это список растений (не менее 4):
    \begin{center}
        \texttt{[''Орхидеи'', ''Кактусы'', ''Пальмы'', ''Бонсаи'']}
    \end{center}
    Конструктор должен иметь сигнатуру: \texttt{\_\_init\_\_(self) -> None}.

    \item Создайте класс \texttt{GreenTrain}, который будет представлять собой состав контейнеров. В конструкторе класса \texttt{GreenTrain} инициализируйте список контейнеров \texttt{self.train: list[Planter]} длиной 56.

    \item Добавьте метод \texttt{shuffle(self) -> None} в класс \texttt{GreenTrain}, который будет перемешивать контейнеры в списке \texttt{self.train}.

    \item Добавьте метод \texttt{get(self, i: int) -> Planter}, который будет возвращать $i$-й контейнер и его растение из списка \texttt{self.train}.

    \item Создайте экземпляр класса \texttt{GreenTrain} и вызовите метод \texttt{shuffle} для перемешивания контейнеров.

    \item Создайте цикл, который будет запрашивать у пользователя номер контейнера и выводить информацию о нём.

    \item Повторите шаги 5–6 до тех пор, пока пользователь не выберет все контейнеры или не завершит выбор.

    \item В конце программы выводите сообщение о завершении выбора контейнеров.

    \item Убедитесь, что пользователь вводит корректные номера контейнеров и что программа обрабатывает ошибки, связанные с вводом пользователя.

    \item Проверьте работу программы, используя различные комбинации номеров контейнеров и растений.
\end{enumerate}

\item[26] \textbf{Формирование состава машин скорой помощи}
\begin{enumerate}
    \item Создайте класс \texttt{Ambulance}, который будет представлять собой машину с типом бригады. В конструкторе класса \texttt{Ambulance} инициализируйте значения машины и типа бригады из списков \texttt{NumList} и \texttt{MasList}, которые объявлены как общие атрибуты класса. \texttt{NumList: list[str]} — это список машин (не менее 14): 
    \begin{center}
        \texttt{[''Машина\_1'', ''Машина\_2'', \dots, ''Машина\_14'']}
    \end{center}
    \texttt{MasList: list[str]} — это список типов бригад (не менее 4):
    \begin{center}
        \texttt{[''Травматологи'', ''Кардиологи'', ''Психиатры'', ''Реаниматологи'']}
    \end{center}
    Конструктор должен иметь сигнатуру: \texttt{\_\_init\_\_(self) -> None}.

    \item Создайте класс \texttt{AmbulanceConvoy}, который будет представлять собой конвой машин. В конструкторе класса \texttt{AmbulanceConvoy} инициализируйте список машин \texttt{self.train: list[Ambulance]} длиной 56.

    \item Добавьте метод \texttt{shuffle(self) -> None} в класс \texttt{AmbulanceConvoy}, который будет перемешивать машины в списке \texttt{self.train}.

    \item Добавьте метод \texttt{get(self, i: int) -> Ambulance}, который будет возвращать $i$-ю машину и её бригаду из списка \texttt{self.train}.

    \item Создайте экземпляр класса \texttt{AmbulanceConvoy} и вызовите метод \texttt{shuffle} для перемешивания машин.

    \item Создайте цикл, который будет запрашивать у пользователя номер машины и выводить информацию о ней.

    \item Повторите шаги 5–6 до тех пор, пока пользователь не выберет все машины или не завершит выбор.

    \item В конце программы выводите сообщение о завершении выбора машин.

    \item Убедитесь, что пользователь вводит корректные номера машин и что программа обрабатывает ошибки, связанные с вводом пользователя.

    \item Проверьте работу программы, используя различные комбинации номеров машин и типов бригад.
\end{enumerate}

\item[27] \textbf{Формирование состава пожарных машин}
\begin{enumerate}
    \item Создайте класс \texttt{FireTruck}, который будет представлять собой пожарную машину со специализацией. В конструкторе класса \texttt{FireTruck} инициализируйте значения машины и специализации из списков \texttt{NumList} и \texttt{MasList}, которые объявлены как общие атрибуты класса. \texttt{NumList: list[str]} — это список машин (не менее 14): 
    \begin{center}
        \texttt{[''Машина\_1'', ''Машина\_2'', \dots, ''Машина\_14'']}
    \end{center}
    \texttt{MasList: list[str]} — это список специализаций (не менее 4):
    \begin{center}
        \texttt{[''Тушение'', ''Спасение'', ''Химзащита'', ''Высотные работы'']}
    \end{center}
    Конструктор должен иметь сигнатуру: \texttt{\_\_init\_\_(self) -> None}.

    \item Создайте класс \texttt{FireTrain}, который будет представлять собой состав пожарных машин. В конструкторе класса \texttt{FireTrain} инициализируйте список машин \texttt{self.train: list[FireTruck]} длиной 56.

    \item Добавьте метод \texttt{shuffle(self) -> None} в класс \texttt{FireTrain}, который будет перемешивать машины в списке \texttt{self.train}.

    \item Добавьте метод \texttt{get(self, i: int) -> FireTruck}, который будет возвращать $i$-ю машину и её специализацию из списка \texttt{self.train}.

    \item Создайте экземпляр класса \texttt{FireTrain} и вызовите метод \texttt{shuffle} для перемешивания машин.

    \item Создайте цикл, который будет запрашивать у пользователя номер машины и выводить информацию о ней.

    \item Повторите шаги 5–6 до тех пор, пока пользователь не выберет все машины или не завершит выбор.

    \item В конце программы выводите сообщение о завершении выбора машин.

    \item Убедитесь, что пользователь вводит корректные номера машин и что программа обрабатывает ошибки, связанные с вводом пользователя.

    \item Проверьте работу программы, используя различные комбинации номеров машин и специализаций.
\end{enumerate}

\item[28] \textbf{Формирование состава эвакуаторов}
\begin{enumerate}
    \item Создайте класс \texttt{TowTruck}, который будет представлять собой эвакуатор с типом транспортного средства. В конструкторе класса \texttt{TowTruck} инициализируйте значения эвакуатора и типа ТС из списков \texttt{NumList} и \texttt{MasList}, которые объявлены как общие атрибуты класса. \texttt{NumList: list[str]} — это список эвакуаторов (не менее 14): 
    \begin{center}
        \texttt{[''Эвакуатор\_1'', ''Эвакуатор\_2'', \dots, ''Эвакуатор\_14'']}
    \end{center}
    \texttt{MasList: list[str]} — это список типов ТС (не менее 4):
    \begin{center}
        \texttt{[''Легковой'', ''Грузовик'', ''Мотоцикл'', ''Автобус'']}
    \end{center}
    Конструктор должен иметь сигнатуру: \texttt{\_\_init\_\_(self) -> None}.

    \item Создайте класс \texttt{TowTrain}, который будет представлять собой состав эвакуаторов. В конструкторе класса \texttt{TowTrain} инициализируйте список эвакуаторов \texttt{self.train: list[TowTruck]} длиной 56.

    \item Добавьте метод \texttt{shuffle(self) -> None} в класс \texttt{TowTrain}, который будет перемешивать эвакуаторы в списке \texttt{self.train}.

    \item Добавьте метод \texttt{get(self, i: int) -> TowTruck}, который будет возвращать $i$-й эвакуатор и тип его ТС из списка \texttt{self.train}.

    \item Создайте экземпляр класса \texttt{TowTrain} и вызовите метод \texttt{shuffle} для перемешивания эвакуаторов.

    \item Создайте цикл, который будет запрашивать у пользователя номер эвакуатора и выводить информацию о нём.

    \item Повторите шаги 5–6 до тех пор, пока пользователь не выберет все эвакуаторы или не завершит выбор.

    \item В конце программы выводите сообщение о завершении выбора эвакуаторов.

    \item Убедитесь, что пользователь вводит корректные номера эвакуаторов и что программа обрабатывает ошибки, связанные с вводом пользователя.

    \item Проверьте работу программы, используя различные комбинации номеров эвакуаторов и типов ТС.
\end{enumerate}

\item[29] \textbf{Формирование состава контейнеров с лекарствами}
\begin{enumerate}
    \item Создайте класс \texttt{MedBox}, который будет представлять собой контейнер с типом лекарств. В конструкторе класса \texttt{MedBox} инициализируйте значения контейнера и типа лекарств из списков \texttt{NumList} и \texttt{MasList}, которые объявлены как общие атрибуты класса. \texttt{NumList: list[str]} — это список контейнеров (не менее 14): 
    \begin{center}
        \texttt{[''Контейнер\_1'', ''Контейнер\_2'', \dots, ''Контейнер\_14'']}
    \end{center}
    \texttt{MasList: list[str]} — это список типов лекарств (не менее 4):
    \begin{center}
        \texttt{[''Антибиотики'', ''Вакцины'', ''Обезболивающие'', ''Витамины'']}
    \end{center}
    Конструктор должен иметь сигнатуру: \texttt{\_\_init\_\_(self) -> None}.

    \item Создайте класс \texttt{MedTrain}, который будет представлять собой состав контейнеров. В конструкторе класса \texttt{MedTrain} инициализируйте список контейнеров \texttt{self.train: list[MedBox]} длиной 56.

    \item Добавьте метод \texttt{shuffle(self) -> None} в класс \texttt{MedTrain}, который будет перемешивать контейнеры в списке \texttt{self.train}.

    \item Добавьте метод \texttt{get(self, i: int) -> MedBox}, который будет возвращать $i$-й контейнер и его лекарства из списка \texttt{self.train}.

    \item Создайте экземпляр класса \texttt{MedTrain} и вызовите метод \texttt{shuffle} для перемешивания контейнеров.

    \item Создайте цикл, который будет запрашивать у пользователя номер контейнера и выводить информацию о нём.

    \item Повторите шаги 5–6 до тех пор, пока пользователь не выберет все контейнеры или не завершит выбор.

    \item В конце программы выводите сообщение о завершении выбора контейнеров.

    \item Убедитесь, что пользователь вводит корректные номера контейнеров и что программа обрабатывает ошибки, связанные с вводом пользователя.

    \item Проверьте работу программы, используя различные комбинации номеров контейнеров и типов лекарств.
\end{enumerate}

\item[30] \textbf{Формирование состава транспорта с опасными грузами}
\begin{enumerate}
    \item Создайте класс \texttt{HazmatTruck}, который будет представлять собой грузовик с классом опасности. В конструкторе класса \texttt{HazmatTruck} инициализируйте значения грузовика и класса опасности из списков \texttt{NumList} и \texttt{MasList}, которые объявлены как общие атрибуты класса. \texttt{NumList: list[str]} — это список грузовиков (не менее 14): 
    \begin{center}
        \texttt{[''Грузовик\_1'', ''Грузовик\_2'', \dots, ''Грузовик\_14'']}
    \end{center}
    \texttt{MasList: list[str]} — это список классов опасности (не менее 4):
    \begin{center}
        \texttt{[''Взрывчатка'', ''Газы'', ''Легковоспламеняющиеся'', ''Токсичные'']}
    \end{center}
    Конструктор должен иметь сигнатуру: \texttt{\_\_init\_\_(self) -> None}.

    \item Создайте класс \texttt{HazmatConvoy}, который будет представлять собой конвой грузовиков. В конструкторе класса \texttt{HazmatConvoy} инициализируйте список грузовиков \texttt{self.train: list[HazmatTruck]} длиной 56.

    \item Добавьте метод \texttt{shuffle(self) -> None} в класс \texttt{HazmatConvoy}, который будет перемешивать грузовики в списке \texttt{self.train}.

    \item Добавьте метод \texttt{get(self, i: int) -> HazmatTruck}, который будет возвращать $i$-й грузовик и его класс опасности из списка \texttt{self.train}.

    \item Создайте экземпляр класса \texttt{HazmatConvoy} и вызовите метод \texttt{shuffle} для перемешивания грузовиков.

    \item Создайте цикл, который будет запрашивать у пользователя номер грузовика и выводить информацию о нём.

    \item Повторите шаги 5–6 до тех пор, пока пользователь не выберет все грузовики или не завершит выбор.

    \item В конце программы выводите сообщение о завершении выбора грузовиков.

    \item Убедитесь, что пользователь вводит корректные номера грузовиков и что программа обрабатывает ошибки, связанные с вводом пользователя.

    \item Проверьте работу программы, используя различные комбинации номеров грузовиков и классов опасности.
\end{enumerate}

\item[31] \textbf{Формирование состава курьерских пакетов}
\begin{enumerate}
    \item Создайте класс \texttt{Package}, который будет представлять собой пакет с типом доставки. В конструкторе класса \texttt{Package} инициализируйте значения пакета и типа доставки из списков \texttt{NumList} и \texttt{MasList}, которые объявлены как общие атрибуты класса. \texttt{NumList: list[str]} — это список пакетов (не менее 14): 
    \begin{center}
        \texttt{[''Пакет\_1'', ''Пакет\_2'', \dots, ''Пакет\_14'']}
    \end{center}
    \texttt{MasList: list[str]} — это список типов доставки (не менее 4):
    \begin{center}
        \texttt{[''Экспресс'', ''Стандарт'', ''Международный'', ''Хрупкий'']}
    \end{center}
    Конструктор должен иметь сигнатуру: \texttt{\_\_init\_\_(self) -> None}.

    \item Создайте класс \texttt{PackageTrain}, который будет представлять собой состав пакетов. В конструкторе класса \texttt{PackageTrain} инициализируйте список пакетов \texttt{self.train: list[Package]} длиной 56.

    \item Добавьте метод \texttt{shuffle(self) -> None} в класс \texttt{PackageTrain}, который будет перемешивать пакеты в списке \texttt{self.train}.

    \item Добавьте метод \texttt{get(self, i: int) -> Package}, который будет возвращать $i$-й пакет и его тип доставки из списка \texttt{self.train}.

    \item Создайте экземпляр класса \texttt{PackageTrain} и вызовите метод \texttt{shuffle} для перемешивания пакетов.

    \item Создайте цикл, который будет запрашивать у пользователя номер пакета и выводить информацию о нём.

    \item Повторите шаги 5–6 до тех пор, пока пользователь не выберет все пакеты или не завершит выбор.

    \item В конце программы выводите сообщение о завершении выбора пакетов.

    \item Убедитесь, что пользователь вводит корректные номера пакетов и что программа обрабатывает ошибки, связанные с вводом пользователя.

    \item Проверьте работу программы, используя различные комбинации номеров пакетов и типов доставки.
\end{enumerate}

\item[32] \textbf{Формирование состава мобильных медицинских лабораторий}
\begin{enumerate}
    \item Создайте класс \texttt{LabVan}, который будет представлять собой лабораторию с типом анализа. В конструкторе класса \texttt{LabVan} инициализируйте значения лаборатории и типа анализа из списков \texttt{NumList} и \texttt{MasList}, которые объявлены как общие атрибуты класса. \texttt{NumList: list[str]} — это список лабораторий (не менее 14): 
    \begin{center}
        \texttt{[''Лаборатория\_1'', ''Лаборатория\_2'', \dots, ''Лаборатория\_14'']}
    \end{center}
    \texttt{MasList: list[str]} — это список типов анализов (не менее 4):
    \begin{center}
        \texttt{[''PCR'', ''Анализы крови'', ''Токсикология'', ''Микробиология'']}
    \end{center}
    Конструктор должен иметь сигнатуру: \texttt{\_\_init\_\_(self) -> None}.

    \item Создайте класс \texttt{LabConvoy}, который будет представлять собой конвой лабораторий. В конструкторе класса \texttt{LabConvoy} инициализируйте список лабораторий \texttt{self.train: list[LabVan]} длиной 56.

    \item Добавьте метод \texttt{shuffle(self) -> None} в класс \texttt{LabConvoy}, который будет перемешивать лаборатории в списке \texttt{self.train}.

    \item Добавьте метод \texttt{get(self, i: int) -> LabVan}, который будет возвращать $i$-ю лабораторию и её тип анализа из списка \texttt{self.train}.

    \item Создайте экземпляр класса \texttt{LabConvoy} и вызовите метод \texttt{shuffle} для перемешивания лабораторий.

    \item Создайте цикл, который будет запрашивать у пользователя номер лаборатории и выводить информацию о ней.

    \item Повторите шаги 5–6 до тех пор, пока пользователь не выберет все лаборатории или не завершит выбор.

    \item В конце программы выводите сообщение о завершении выбора лабораторий.

    \item Убедитесь, что пользователь вводит корректные номера лабораторий и что программа обрабатывает ошибки, связанные с вводом пользователя.

    \item Проверьте работу программы, используя различные комбинации номеров лабораторий и типов анализов.
\end{enumerate}

\item[33] \textbf{Формирование состава контейнеров с артефактами}
\begin{enumerate}
    \item Создайте класс \texttt{ArtifactCase}, который будет представлять собой кейс с происхождением артефакта. В конструкторе класса \texttt{ArtifactCase} инициализируйте значения кейса и происхождения из списков \texttt{NumList} и \texttt{MasList}, которые объявлены как общие атрибуты класса. \texttt{NumList: list[str]} — это список кейсов (не менее 14): 
    \begin{center}
        \texttt{[''Кейс\_1'', ''Кейс\_2'', \dots, ''Кейс\_14'']}
    \end{center}
    \texttt{MasList: list[str]} — это список происхождений (не менее 4):
    \begin{center}
        \texttt{[''Египет'', ''Греция'', ''Мезоамерика'', ''Древний Китай'']}
    \end{center}
    Конструктор должен иметь сигнатуру: \texttt{\_\_init\_\_(self) -> None}.

    \item Создайте класс \texttt{ArtifactTrain}, который будет представлять собой состав кейсов. В конструкторе класса \texttt{ArtifactTrain} инициализируйте список кейсов \texttt{self.train: list[ArtifactCase]} длиной 56.

    \item Добавьте метод \texttt{shuffle(self) -> None} в класс \texttt{ArtifactTrain}, который будет перемешивать кейсы в списке \texttt{self.train}.

    \item Добавьте метод \texttt{get(self, i: int) -> ArtifactCase}, который будет возвращать $i$-й кейс и происхождение его артефакта из списка \texttt{self.train}.

    \item Создайте экземпляр класса \texttt{ArtifactTrain} и вызовите метод \texttt{shuffle} для перемешивания кейсов.

    \item Создайте цикл, который будет запрашивать у пользователя номер кейса и выводить информацию о нём.

    \item Повторите шаги 5–6 до тех пор, пока пользователь не выберет все кейсы или не завершит выбор.

    \item В конце программы выводите сообщение о завершении выбора кейсов.

    \item Убедитесь, что пользователь вводит корректные номера кейсов и что программа обрабатывает ошибки, связанные с вводом пользователя.

    \item Проверьте работу программы, используя различные комбинации номеров кейсов и происхождений.
\end{enumerate}

\item[34] \textbf{Формирование состава беспилотных грузовиков}
\begin{enumerate}
    \item Создайте класс \texttt{AutonomousTruck}, который будет представлять собой грузовик с типом маршрута. В конструкторе класса \texttt{AutonomousTruck} инициализируйте значения грузовика и маршрута из списков \texttt{NumList} и \texttt{MasList}, которые объявлены как общие атрибуты класса. \texttt{NumList: list[str]} — это список грузовиков (не менее 14): 
    \begin{center}
        \texttt{[''Грузовик\_1'', ''Грузовик\_2'', \dots, ''Грузовик\_14'']}
    \end{center}
    \texttt{MasList: list[str]} — это список типов маршрутов (не менее 4):
    \begin{center}
        \texttt{[''Город'', ''Шоссе'', ''Горы'', ''Пустыня'']}
    \end{center}
    Конструктор должен иметь сигнатуру: \texttt{\_\_init\_\_(self) -> None}.

    \item Создайте класс \texttt{TruckConvoy}, который будет представлять собой конвой грузовиков. В конструкторе класса \texttt{TruckConvoy} инициализируйте список грузовиков \texttt{self.train: list[AutonomousTruck]} длиной 56.

    \item Добавьте метод \texttt{shuffle(self) -> None} в класс \texttt{TruckConvoy}, который будет перемешивать грузовики в списке \texttt{self.train}.

    \item Добавьте метод \texttt{get(self, i: int) -> AutonomousTruck}, который будет возвращать $i$-й грузовик и его маршрут из списка \texttt{self.train}.

    \item Создайте экземпляр класса \texttt{TruckConvoy} и вызовите метод \texttt{shuffle} для перемешивания грузовиков.

    \item Создайте цикл, который будет запрашивать у пользователя номер грузовика и выводить информацию о нём.

    \item Повторите шаги 5–6 до тех пор, пока пользователь не выберет все грузовики или не завершит выбор.

    \item В конце программы выводите сообщение о завершении выбора грузовиков.

    \item Убедитесь, что пользователь вводит корректные номера грузовиков и что программа обрабатывает ошибки, связанные с вводом пользователя.

    \item Проверьте работу программы, используя различные комбинации номеров грузовиков и маршрутов.
\end{enumerate}

\item[35] \textbf{Формирование состава грузовых вагонов} (оригинальный вариант)
\begin{enumerate}
    \item Создайте класс \texttt{Vagon}, который будет представлять собой вагон с грузом. В конструкторе класса \texttt{Vagon} инициализируйте значения вагона и груза из списков \texttt{NumList} и \texttt{MasList}, которые объявлены как общие атрибуты класса. \texttt{NumList: list[str]} — это список крытых вагонов (не менее 14): 
    \begin{center}
        \texttt{[''Вагон\_1'', ''Вагон\_2'', \dots, ''Вагон\_14'']}
    \end{center}
    \texttt{MasList: list[str]} — это список грузов для крытых вагонов (не менее 4):
    \begin{center}
        \texttt{[''Станки'', ''Автозапчасти'', ''Бумага'', ''Керамическая плитка'']}
    \end{center}
    Конструктор должен иметь сигнатуру: \texttt{\_\_init\_\_(self) -> None}.

    \item Создайте класс \texttt{TrainOfVagons}, который будет представлять собой грузовой поезд, состоящий из моделей вагонов. В конструкторе класса \texttt{TrainOfVagons} инициализируйте список вагонов \texttt{self.train: list[Vagon]} длиной 56.

    \item Добавьте метод \texttt{shuffle(self) -> None} в класс \texttt{TrainOfVagons}, который будет перемешивать вагоны в списке \texttt{self.train}.

    \item Добавьте метод \texttt{get(self, i: int) -> Vagon}, который будет возвращать $i$-й вагон и груз из списка \texttt{self.train}.

    \item Создайте экземпляр класса \texttt{TrainOfVagons} и вызовите метод \texttt{shuffle} для перемешивания вагонов.

    \item Создайте цикл, который будет запрашивать у пользователя номер вагона из поезда и выводить информацию о выбранном вагоне.

    \item Повторите шаги 5–6 до тех пор, пока пользователь не выберет все вагоны или не завершит выбор.

    \item В конце программы выводите сообщение о завершении выбора вагонов.

    \item Убедитесь, что пользователь вводит корректные номера вагонов и что программа обрабатывает ошибки, связанные с вводом пользователя.

    \item Проверьте работу программы, используя различные комбинации номеров вагонов и грузов.
\end{enumerate}

\end{enumerate}

\subsubsection{Задача 2}

    \begin{enumerate}
\item[1] \textbf{Расчёт площади стен в зависимости от наличия встроенных картин и зеркал}
\begin{enumerate}
    \item Создайте два класса: \texttt{PictureMirror} и \texttt{Room}.  
    Класс \texttt{PictureMirror} представляет отдельный встроенный элемент (картину или зеркало). Его конструктор принимает два аргумента:  
    \texttt{width} — ширина элемента в метрах (положительное дробное число),  
    \texttt{height} — высота элемента в метрах (положительное дробное число).  
    Объект этого класса хранит только эти два значения.  
    Класс \texttt{Room} описывает прямоугольную комнату. Его конструктор принимает три аргумента:  
    \texttt{width} — ширина комнаты в метрах,  
    \texttt{length} — длина комнаты в метрах,  
    \texttt{height} — высота стен в метрах.  
    Все значения должны быть положительными. Объект \texttt{Room} хранит геометрические размеры комнаты и список объектов \texttt{PictureMirror}, изначально пустой.

    \item В классе \texttt{Room} реализуйте следующие методы:  
    \begin{itemize}
        \item \texttt{add\_item(self, item: PictureMirror) -> None} — добавляет переданный объект \texttt{PictureMirror} в внутренний список встроенных элементов комнаты. Метод не проверяет, помещается ли элемент на стене; предполагается, что все элементы корректно размещены.
        \item \texttt{get\_area\_to\_cover(self) -> float} — вычисляет и возвращает площадь стен, подлежащую отделке. Общая площадь стен комнаты рассчитывается по формуле \(2 \cdot \texttt{height} \cdot (\texttt{width} + \texttt{length})\). Из этой площади вычитается суммарная площадь всех встроенных элементов (каждый элемент вносит вклад \(\texttt{width} \cdot \texttt{height}\)). Результат не может быть отрицательным: если суммарная площадь элементов превышает площадь стен, метод возвращает 0.0.
        \item \texttt{get\_panels\_count(self, panel\_width: float, panel\_height: float) -> int} — рассчитывает минимальное количество декоративных панелей, необходимых для отделки вычисленной ранее площади. Площадь одной панели равна \(\texttt{panel\_width} \cdot \texttt{panel\_height}\). Количество панелей определяется как результат деления площади под отделку на площадь одной панели, округлённый вверх до ближайшего целого (поскольку панели продаются только целиком).
    \end{itemize}

    \item Создайте три различных экземпляра класса \texttt{Room} с разными размерами и разным набором встроенных элементов (например, комната без элементов, комната с одной большой картиной, комната с несколькими зеркалами). Для каждого экземпляра вызовите методы \texttt{add\_item} (при необходимости), \texttt{get\_area\_to\_cover} и \texttt{get\_panels\_count}, чтобы продемонстрировать корректность реализации.

    \item Запросите у пользователя данные для одной комнаты: ширину, длину и высоту комнаты (все — дробные числа), а также ширину и высоту одной декоративной панели (дробные числа).

    \item Выведите на экран два значения: площадь стен под отделку (в квадратных метрах, с дробной частью) и минимальное количество необходимых панелей (целое число, округлённое вверх).
\end{enumerate}

\item[2] \textbf{Расчёт площади стен в зависимости от наличия встроенных настенных устройств}
\begin{enumerate}
    \item Создайте два класса: \texttt{WallDevice} и \texttt{Room}.  
    Класс \texttt{WallDevice} представляет отдельное настенное устройство (например, панель управления). Его конструктор принимает два аргумента:  
    \texttt{width} — ширина устройства в метрах (положительное дробное число),  
    \texttt{height} — высота устройства в метрах (положительное дробное число).  
    Объект хранит только эти два значения.  
    Класс \texttt{Room} описывает прямоугольную комнату. Его конструктор принимает три аргумента:  
    \texttt{width} — ширина комнаты в метрах,  
    \texttt{length} — длина комнаты в метрах,  
    \texttt{height} — высота стен в метрах.  
    Все значения должны быть положительными. Объект \texttt{Room} хранит размеры комнаты и список объектов \texttt{WallDevice}, изначально пустой.

    \item В классе \texttt{Room} реализуйте следующие методы:  
    \begin{itemize}
        \item \texttt{add\_device(self, dev: WallDevice) -> None} — добавляет переданный объект \texttt{WallDevice} в список встроенных устройств комнаты.
        \item \texttt{get\_area\_to\_cover(self) -> float} — вычисляет площадь стен под отделку: из общей площади стен \(2 \cdot \texttt{height} \cdot (\texttt{width} + \texttt{length})\) вычитается суммарная площадь всех устройств. Результат не может быть меньше нуля.
        \item \texttt{get\_tiles\_count(self, tile\_width: float, tile\_height: float) -> int} — рассчитывает количество керамических плиток, необходимых для облицовки. Площадь одной плитки равна \(\texttt{tile\_width} \cdot \texttt{tile\_height}\). Количество плиток — это результат деления площади под отделку на площадь плитки, округлённый вверх до целого числа.
    \end{itemize}

    \item Создайте три различных экземпляра класса \texttt{Room} с разными параметрами и разным числом устройств. Для каждого вызовите методы для получения площади и количества плиток.

    \item Запросите у пользователя размеры комнаты (ширина, длина, высота) и размеры одной плитки (ширина и высота), все — дробные числа.

    \item Выведите площадь стен под облицовку (м²) и минимальное количество плиток (целое число, округлённое вверх).
\end{enumerate}

\item[3] \textbf{Расчёт площади стен в зависимости от наличия встроенных светильников и бра}
\begin{enumerate}
    \item Создайте два класса: \texttt{Lamp} и \texttt{Room}.  
    Класс \texttt{Lamp} представляет один настенный светильник или бра. Его конструктор принимает:  
    \texttt{width} — ширина светильника в метрах,  
    \texttt{height} — высота светильника в метрах.  
    Оба значения — положительные дробные числа.  
    Класс \texttt{Room} описывает комнату. Его конструктор принимает:  
    \texttt{width}, \texttt{length}, \texttt{height} — размеры комнаты в метрах (все положительные).  
    Объект \texttt{Room} хранит эти размеры и список объектов \texttt{Lamp}.

    \item В классе \texttt{Room} реализуйте методы:  
    \begin{itemize}
        \item \texttt{add\_lamp(self, lamp: Lamp) -> None} — добавляет светильник в список встроенных элементов.
        \item \texttt{get\_area\_to\_cover(self) -> float} — возвращает площадь стен без учёта площадей всех светильников. Общая площадь стен: \(2 \cdot \texttt{height} \cdot (\texttt{width} + \texttt{length})\). Из неё вычитается сумма площадей всех светильников. Результат $\geqslant$ 0.
        \item \texttt{get\_wallpaper\_rolls(self, roll\_width: float, roll\_length: float) -> int} — вычисляет количество рулонов обоев. Площадь одного рулона: \(\texttt{roll\_width} \cdot \texttt{roll\_length}\). Количество рулонов — частное от деления площади под оклейку на площадь рулона, округлённое вверх до целого.
    \end{itemize}

    \item Создайте три разных экземпляра \texttt{Room} (с разным числом светильников) и протестируйте методы.

    \item Запросите у пользователя размеры комнаты и размеры одного рулона обоев (все — дробные числа).

    \item Выведите площадь под оклейку (м²) и количество рулонов обоев (целое число, округлённое вверх).
\end{enumerate}

\item[4] \textbf{Расчёт площади стен в зависимости от наличия встроенных полок и стеллажей}
\begin{enumerate}
    \item Создайте два класса: \texttt{Shelf} и \texttt{Room}.  
    Класс \texttt{Shelf} описывает одну полку или стеллаж. Его конструктор принимает:  
    \texttt{width} — ширина в метрах,  
    \texttt{height} — высота в метрах (оба — положительные дробные числа).  
    Класс \texttt{Room} описывает комнату с параметрами:  
    \texttt{width}, \texttt{length}, \texttt{height} — размеры комнаты в метрах.  
    Объект \texttt{Room} хранит список объектов \texttt{Shelf}.

    \item В классе \texttt{Room} реализуйте методы:  
    \begin{itemize}
        \item \texttt{add\_shelf(self, shelf: Shelf) -> None} — добавляет полку в комнату.
        \item \texttt{get\_area\_to\_cover(self) -> float} — площадь стен под покраску: общая площадь стен минус суммарная площадь всех полок (результат $\geqslant$ 0).
        \item \texttt{get\_paint\_liters(self, coverage: float) -> float} — вычисляет необходимый объём краски в литрах. Аргумент \texttt{coverage} задаёт, сколько квадратных метров можно покрыть одним литром краски (м²/л). Объём краски = площадь под покраску / \texttt{coverage}. Результат может быть дробным, так как краску можно купить нецелыми банками.
    \end{itemize}

    \item Создайте три различных комнаты с разным числом полок и проверьте работу методов.

    \item Запросите у пользователя размеры комнаты и значение \texttt{coverage} (дробное число).

    \item Выведите площадь под покраску (м²) и необходимое количество литров краски (с дробной частью).
\end{enumerate}

\item[5] \textbf{Расчёт площади стен в зависимости от наличия встроенных розеток и выключателей}
\begin{enumerate}
    \item Создайте два класса: \texttt{SocketSwitch} и \texttt{Room}.  
    Класс \texttt{SocketSwitch} представляет одну розетку или выключатель. Его конструктор принимает:  
    \texttt{width} — ширина в метрах,  
    \texttt{height} — высота в метрах (положительные дробные числа).  
    Класс \texttt{Room} описывает комнату с размерами \texttt{width}, \texttt{length}, \texttt{height} (все — положительные дробные числа) и хранит список объектов \texttt{SocketSwitch}.

    \item В классе \texttt{Room} реализуйте методы:  
    \begin{itemize}
        \item \texttt{add\_electrical(self, el: SocketSwitch) -> None} — добавляет электроарматуру в комнату.
        \item \texttt{get\_area\_to\_cover(self) -> float} — площадь стен под штукатурку: общая площадь стен минус сумма площадей всех розеток и выключателей (результат $\geqslant$ 0).
        \item \texttt{get\_plaster\_bags(self, bag\_coverage: float) -> int} — количество мешков штукатурки. Аргумент \texttt{bag\_coverage} — сколько квадратных метров покрывает один мешок (м²/мешок). Количество мешков = площадь под штукатурку / \texttt{bag\_coverage}, округлённое вверх до целого.
    \end{itemize}

    \item Создайте три разных экземпляра \texttt{Room} с разным числом электроустройств и протестируйте методы.

    \item Запросите у пользователя размеры комнаты и \texttt{bag\_coverage} (дробное число).

    \item Выведите площадь под штукатурку (м²) и количество мешков (целое число, округлённое вверх).
\end{enumerate}

\item[6] \textbf{Расчёт площади стен в зависимости от наличия встроенных вентиляционных решёток}
\begin{enumerate}
    \item Создайте два класса: \texttt{VentGrille} и \texttt{Room}.  
    Класс \texttt{VentGrille} описывает одну вентиляционную решётку. Его конструктор принимает:  
    \texttt{width} — ширина решётки в метрах,  
    \texttt{height} — высота решётки в метрах (положительные дробные числа).  
    Класс \texttt{Room} описывает комнату с размерами \texttt{width}, \texttt{length}, \texttt{height} и хранит список объектов \texttt{VentGrille}.

    \item В классе \texttt{Room} реализуйте методы:  
    \begin{itemize}
        \item \texttt{add\_vent(self, vent: VentGrille) -> None} — добавляет решётку в комнату.
        \item \texttt{get\_area\_to\_cover(self) -> float} — площадь стен под обшивку: общая площадь стен минус сумма площадей всех решёток (результат $\geqslant$ 0).
        \item \texttt{get\_panel\_sheets(self, sheet\_width: float, sheet\_height: float) -> int} — количество листов панелей. Площадь одного листа = \(\texttt{sheet\_width} \cdot \texttt{sheet\_height}\). Количество листов — частное от деления площади под обшивку на площадь листа, округлённое вверх до целого.
    \end{itemize}

    \item Создайте три различных комнаты с разным числом решёток и проверьте методы.

    \item Запросите у пользователя размеры комнаты и размеры одного листа панели (все — дробные числа).

    \item Выведите площадь под обшивку (м²) и количество листов (целое число, округлённое вверх).
\end{enumerate}

\item[7] \textbf{Расчёт площади стен в зависимости от наличия встроенных настенных кондиционеров}
\begin{enumerate}
    \item Создайте два класса: \texttt{WallAC} и \texttt{Room}.  
    Класс \texttt{WallAC} представляет один настенный кондиционер. Его конструктор принимает:  
    \texttt{width} — ширина в метрах,  
    \texttt{height} — высота в метрах (положительные дробные числа).  
    Класс \texttt{Room} описывает комнату с размерами \texttt{width}, \texttt{length}, \texttt{height} и хранит список объектов \texttt{WallAC}.

    \item В классе \texttt{Room} реализуйте методы:  
    \begin{itemize}
        \item \texttt{add\_ac(self, ac: WallAC) -> None} — добавляет кондиционер в комнату.
        \item \texttt{get\_area\_to\_cover(self) -> float} — площадь стен под оклейку обоями: общая площадь стен минус сумма площадей всех кондиционеров (результат $\geqslant$ 0).
        \item \texttt{get\_wallpaper\_rolls(self, roll\_width: float, roll\_length: float) -> int} — количество рулонов обоев. Площадь рулона = \(\texttt{roll\_width} \cdot \texttt{roll\_length}\). Количество рулонов — частное от деления площади под оклейку на площадь рулона, округлённое вверх до целого.
    \end{itemize}

    \item Создайте три разных экземпляра \texttt{Room} с разным числом кондиционеров и протестируйте методы.

    \item Запросите у пользователя размеры комнаты и размеры рулона обоев (все — дробные числа).

    \item Выведите площадь под оклейку (м²) и количество рулонов (целое число, округлённое вверх).
\end{enumerate}

\item[8] \textbf{Расчёт площади стен в зависимости от наличия встроенных настенных экранов}
\begin{enumerate}
    \item Создайте два класса: \texttt{WallScreen} и \texttt{Room}.  
    Класс \texttt{WallScreen} описывает один настенный экран. Его конструктор принимает:  
    \texttt{width} — ширина в метрах,  
    \texttt{height} — высота в метрах (положительные дробные числа).  
    Класс \texttt{Room} описывает комнату с размерами \texttt{width}, \texttt{length}, \texttt{height} и хранит список объектов \texttt{WallScreen}.

    \item В классе \texttt{Room} реализуйте методы:  
    \begin{itemize}
        \item \texttt{add\_screen(self, scr: WallScreen) -> None} — добавляет экран в комнату.
        \item \texttt{get\_area\_to\_cover(self) -> float} — площадь стен под покраску: общая площадь стен минус сумма площадей всех экранов (результат $\geqslant$ 0).
        \item \texttt{get\_paint\_cans(self, can\_coverage: float) -> int} — количество банок краски. Аргумент \texttt{can\_coverage} — сколько квадратных метров покрывает одна банка (м²/банка). Количество банок = площадь под покраску / \texttt{can\_coverage}, округлённое вверх до целого.
    \end{itemize}

    \item Создайте три различных комнаты с разным числом экранов и проверьте работу методов.

    \item Запросите у пользователя размеры комнаты и \texttt{can\_coverage} (дробное число).

    \item Выведите площадь под покраску (м²) и количество банок (целое число, округлённое вверх).
\end{enumerate}

\item[9] \textbf{Расчёт площади стен в зависимости от наличия встроенных настенных сейфов}
\begin{enumerate}
    \item Создайте два класса: \texttt{WallSafe} и \texttt{Room}.  
    Класс \texttt{WallSafe} представляет один настенный сейф. Его конструктор принимает:  
    \texttt{width} — ширина в метрах,  
    \texttt{height} — высота в метрах (положительные дробные числа).  
    Класс \texttt{Room} описывает комнату с размерами \texttt{width}, \texttt{length}, \texttt{height} и хранит список объектов \texttt{WallSafe}.

    \item В классе \texttt{Room} реализуйте методы:  
    \begin{itemize}
        \item \texttt{add\_safe(self, safe: WallSafe) -> None} — добавляет сейф в комнату.
        \item \texttt{get\_area\_to\_cover(self) -> float} — площадь стен под облицовку: общая площадь стен минус сумма площадей всех сейфов (результат $\geqslant$ 0).
        \item \texttt{get\_tiles\_count(self, tile\_width: float, tile\_height: float) -> int} — количество плиток. Площадь одной плитки = \(\texttt{tile\_width} \cdot \texttt{tile\_height}\). Количество плиток — частное от деления площади под облицовку на площадь плитки, округлённое вверх до целого.
    \end{itemize}

    \item Создайте три разных экземпляра \texttt{Room} с разным числом сейфов и протестируйте методы.

    \item Запросите у пользователя размеры комнаты и размеры плитки (все — дробные числа).

    \item Выведите площадь под облицовку (м²) и количество плиток (целое число, округлённое вверх).
\end{enumerate}

\item[10] \textbf{Расчёт площади стен в зависимости от наличия встроенных настенных досок}
\begin{enumerate}
    \item Создайте два класса: \texttt{WallBoard} и \texttt{Room}.  
    Класс \texttt{WallBoard} описывает одну настенную доску. Его конструктор принимает:  
    \texttt{width} — ширина в метрах,  
    \texttt{height} — высота в метрах (положительные дробные числа).  
    Класс \texttt{Room} описывает комнату с размерами \texttt{width}, \texttt{length}, \texttt{height} и хранит список объектов \texttt{WallBoard}.

    \item В классе \texttt{Room} реализуйте методы:  
    \begin{itemize}
        \item \texttt{add\_board(self, board: WallBoard) -> None} — добавляет доску в комнату.
        \item \texttt{get\_area\_to\_cover(self) -> float} — площадь стен под драпировку: общая площадь стен минус сумма площадей всех досок (результат $\geqslant$ 0).
        \item \texttt{get\_fabric\_meters(self, fabric\_width: float) -> float} — необходимая длина ткани в метрах. Аргумент \texttt{fabric\_width} — ширина ткани в метрах. Длина ткани = площадь под драпировку / \texttt{fabric\_width}. Результат может быть дробным, так как ткань продаётся погонными метрами.
    \end{itemize}

    \item Создайте три различных комнаты с разным числом досок и проверьте методы.

    \item Запросите у пользователя размеры комнаты и ширину ткани (дробные числа).

    \item Выведите площадь под драпировку (м²) и количество метров ткани (с дробной частью).
\end{enumerate}

\item[11] \textbf{Расчёт площади стен в зависимости от наличия встроенных настенных календарей}
\begin{enumerate}
    \item Создайте два класса: \texttt{WallCalendar} и \texttt{Room}.  
    Класс \texttt{WallCalendar} представляет один настенный календарь. Его конструктор принимает:  
    \texttt{width} — ширина в метрах,  
    \texttt{height} — высота в метрах (положительные дробные числа).  
    Класс \texttt{Room} описывает комнату с размерами \texttt{width}, \texttt{length}, \texttt{height} и хранит список объектов \texttt{WallCalendar}.

    \item В классе \texttt{Room} реализуйте методы:  
    \begin{itemize}
        \item \texttt{add\_calendar(self, cal: WallCalendar) -> None} — добавляет календарь в комнату.
        \item \texttt{get\_area\_to\_cover(self) -> float} — площадь стен под оклейку: общая площадь стен минус сумма площадей всех календарей (результат $\geqslant$ 0).
        \item \texttt{get\_wallpaper\_rolls(self, roll\_width: float, roll\_length: float) -> int} — количество рулонов обоев. Площадь рулона = \(\texttt{roll\_width} \cdot \texttt{roll\_length}\). Количество рулонов — частное от деления площади под оклейку на площадь рулона, округлённое вверх до целого.
    \end{itemize}

    \item Создайте три разных экземпляра \texttt{Room} с разным числом календарей и протестируйте методы.

    \item Запросите у пользователя размеры комнаты и размеры рулона обоев (все — дробные числа).

    \item Выведите площадь под оклейку (м²) и количество рулонов (целое число, округлённое вверх).
\end{enumerate}

\item[12] \textbf{Расчёт площади стен в зависимости от наличия встроенных настенных карт}
\begin{enumerate}
    \item Создайте два класса: \texttt{WallMap} и \texttt{Room}.  
    Класс \texttt{WallMap} описывает одну настенную карту. Его конструктор принимает:  
    \texttt{width} — ширина в метрах,  
    \texttt{height} — высота в метрах (положительные дробные числа).  
    Класс \texttt{Room} описывает комнату с размерами \texttt{width}, \texttt{length}, \texttt{height} и хранит список объектов \texttt{WallMap}.

    \item В классе \texttt{Room} реализуйте методы:  
    \begin{itemize}
        \item \texttt{add\_map(self, map: WallMap) -> None} — добавляет карту в комнату.
        \item \texttt{get\_area\_to\_cover(self) -> float} — площадь стен под покраску: общая площадь стен минус сумма площадей всех карт (результат $\geqslant$ 0).
        \item \texttt{get\_paint\_liters(self, coverage: float) -> float} — объём краски в литрах. Аргумент \texttt{coverage} — расход краски (м²/л). Объём = площадь под покраску / \texttt{coverage}. Результат может быть дробным.
    \end{itemize}

    \item Создайте три различных комнаты с разным числом карт и проверьте методы.

    \item Запросите у пользователя размеры комнаты и \texttt{coverage} (дробное число).

    \item Выведите площадь под покраску (м²) и количество литров краски (с дробной частью).
\end{enumerate}

\item[13] \textbf{Расчёт площади стен в зависимости от наличия встроенных настенных террариумов}
\begin{enumerate}
    \item Создайте два класса: \texttt{WallTerrarium} и \texttt{Room}.  
    Класс \texttt{WallTerrarium} представляет один настенный террариум. Его конструктор принимает:  
    \texttt{width} — ширина в метрах,  
    \texttt{height} — высота в метрах (положительные дробные числа).  
    Класс \texttt{Room} описывает комнату с размерами \texttt{width}, \texttt{length}, \texttt{height} и хранит список объектов \texttt{WallTerrarium}.

    \item В классе \texttt{Room} реализуйте методы:  
    \begin{itemize}
        \item \texttt{add\_terrarium(self, terr: WallTerrarium) -> None} — добавляет террариум в комнату.
        \item \texttt{get\_area\_to\_cover(self) -> float} — площадь стен под отделку: общая площадь стен минус сумма площадей всех террариумов (результат $\geqslant$ 0).
        \item \texttt{get\_panels\_count(self, panel\_width: float, panel\_height: float) -> int} — количество декоративных панелей. Площадь одной панели = \(\texttt{panel\_width} \cdot \texttt{panel\_height}\). Количество панелей — частное от деления площади под отделку на площадь панели, округлённое вверх до целого.
    \end{itemize}

    \item Создайте три разных экземпляра \texttt{Room} с разным числом террариумов и протестируйте методы.

    \item Запросите у пользователя размеры комнаты и размеры панели (все — дробные числа).

    \item Выведите площадь под отделку (м²) и количество панелей (целое число, округлённое вверх).
\end{enumerate}

\item[14] \textbf{Расчёт площади стен в зависимости от наличия встроенных настенных аквариумов}
\begin{enumerate}
    \item Создайте два класса: \texttt{WallAquarium} и \texttt{Room}.  
    Класс \texttt{WallAquarium} описывает один настенный аквариум. Его конструктор принимает:  
    \texttt{width} — ширина в метрах,  
    \texttt{height} — высота в метрах (положительные дробные числа).  
    Класс \texttt{Room} описывает комнату с размерами \texttt{width}, \texttt{length}, \texttt{height} и хранит список объектов \texttt{WallAquarium}.

    \item В классе \texttt{Room} реализуйте методы:  
    \begin{itemize}
        \item \texttt{add\_aquarium(self, aq: WallAquarium) -> None} — добавляет аквариум в комнату.
        \item \texttt{get\_area\_to\_cover(self) -> float} — площадь стен под облицовку: общая площадь стен минус сумма площадей всех аквариумов (результат $\geqslant$ 0).
        \item \texttt{get\_tiles\_count(self, tile\_width: float, tile\_height: float) -> int} — количество плиток. Площадь одной плитки = \(\texttt{tile\_width} \cdot \texttt{tile\_height}\). Количество плиток — частное от деления площади под облицовку на площадь плитки, округлённое вверх до целого.
    \end{itemize}

    \item Создайте три различных комнаты с разным числом аквариумов и проверьте методы.

    \item Запросите у пользователя размеры комнаты и размеры плитки (все — дробные числа).

    \item Выведите площадь под облицовку (м²) и количество плиток (целое число, округлённое вверх).
\end{enumerate}

\item[15] \textbf{Расчёт площади стен в зависимости от наличия встроенных настенных динамиков}
\begin{enumerate}
    \item Создайте два класса: \texttt{WallSpeaker} и \texttt{Room}.  
    Класс \texttt{WallSpeaker} представляет один настенный динамик. Его конструктор принимает:  
    \texttt{width} — ширина в метрах,  
    \texttt{height} — высота в метрах (положительные дробные числа).  
    Класс \texttt{Room} описывает комнату с размерами \texttt{width}, \texttt{length}, \texttt{height} и хранит список объектов \texttt{WallSpeaker}.

    \item В классе \texttt{Room} реализуйте методы:  
    \begin{itemize}
        \item \texttt{add\_speaker(self, sp: WallSpeaker) -> None} — добавляет динамик в комнату.
        \item \texttt{get\_area\_to\_cover(self) -> float} — площадь стен под оклейку: общая площадь стен минус сумма площадей всех динамиков (результат $\geqslant$ 0).
        \item \texttt{get\_wallpaper\_rolls(self, roll\_width: float, roll\_length: float) -> int} — количество рулонов обоев. Площадь рулона = \(\texttt{roll\_width} \cdot \texttt{roll\_length}\). Количество рулонов — частное от деления площади под оклейку на площадь рулона, округлённое вверх до целого.
    \end{itemize}

    \item Создайте три разных экземпляра \texttt{Room} с разным числом динамиков и протестируйте методы.

    \item Запросите у пользователя размеры комнаты и размеры рулона обоев (все — дробные числа).

    \item Выведите площадь под оклейку (м²) и количество рулонов (целое число, округлённое вверх).
\end{enumerate}

\item[16] \textbf{Расчёт площади стен в зависимости от наличия встроенных настенных датчиков}
\begin{enumerate}
    \item Создайте два класса: \texttt{WallSensor} и \texttt{Room}.  
    Класс \texttt{WallSensor} описывает один настенный датчик. Его конструктор принимает:  
    \texttt{width} — ширина в метрах,  
    \texttt{height} — высота в метрах (положительные дробные числа).  
    Класс \texttt{Room} описывает комнату с размерами \texttt{width}, \texttt{length}, \texttt{height} и хранит список объектов \texttt{WallSensor}.

    \item В классе \texttt{Room} реализуйте методы:  
    \begin{itemize}
        \item \texttt{add\_sensor(self, sens: WallSensor) -> None} — добавляет датчик в комнату.
        \item \texttt{get\_area\_to\_cover(self) -> float} — площадь стен под покраску: общая площадь стен минус сумма площадей всех датчиков (результат $\geqslant$ 0).
        \item \texttt{get\_paint\_cans(self, can\_coverage: float) -> int} — количество банок краски. Аргумент \texttt{can\_coverage} — покрытие одной банки (м²/банка). Количество банок = площадь под покраску / \texttt{can\_coverage}, округлённое вверх до целого.
    \end{itemize}

    \item Создайте три различных комнаты с разным числом датчиков и проверьте методы.

    \item Запросите у пользователя размеры комнаты и \texttt{can\_coverage} (дробное число).

    \item Выведите площадь под покраску (м²) и количество банок (целое число, округлённое вверх).
\end{enumerate}

\item[17] \textbf{Расчёт площади стен в зависимости от наличия встроенных настенных панно}
\begin{enumerate}
    \item Создайте два класса: \texttt{WallPanel} и \texttt{Room}.  
    Класс \texttt{WallPanel} представляет одно настенное панно. Его конструктор принимает:  
    \texttt{width} — ширина в метрах,  
    \texttt{height} — высота в метрах (положительные дробные числа).  
    Класс \texttt{Room} описывает комнату с размерами \texttt{width}, \texttt{length}, \texttt{height} и хранит список объектов \texttt{WallPanel}.

    \item В классе \texttt{Room} реализуйте методы:  
    \begin{itemize}
        \item \texttt{add\_panel(self, p: WallPanel) -> None} — добавляет панно в комнату.
        \item \texttt{get\_area\_to\_cover(self) -> float} — площадь стен под драпировку: общая площадь стен минус сумма площадей всех панно (результат $\geqslant$ 0).
        \item \texttt{get\_fabric\_meters(self, fabric\_width: float) -> float} — длина ткани в метрах. Аргумент \texttt{fabric\_width} — ширина ткани. Длина = площадь под драпировку / \texttt{fabric\_width}. Результат может быть дробным.
    \end{itemize}

    \item Создайте три разных экземпляра \texttt{Room} с разным числом панно и протестируйте методы.

    \item Запросите у пользователя размеры комнаты и ширину ткани (дробные числа).

    \item Выведите площадь под драпировку (м²) и количество метров ткани (с дробной частью).
\end{enumerate}

\item[18] \textbf{Расчёт площади стен в зависимости от наличия встроенных настенных рамок}
\begin{enumerate}
    \item Создайте два класса: \texttt{WallFrame} и \texttt{Room}.  
    Класс \texttt{WallFrame} описывает одну настенную рамку. Его конструктор принимает:  
    \texttt{width} — ширина в метрах,  
    \texttt{height} — высота в метрах (положительные дробные числа).  
    Класс \texttt{Room} описывает комнату с размерами \texttt{width}, \texttt{length}, \texttt{height} и хранит список объектов \texttt{WallFrame}.

    \item В классе \texttt{Room} реализуйте методы:  
    \begin{itemize}
        \item \texttt{add\_frame(self, f: WallFrame) -> None} — добавляет рамку в комнату.
        \item \texttt{get\_area\_to\_cover(self) -> float} — площадь стен под оклейку: общая площадь стен минус сумма площадей всех рамок (результат $\geqslant$ 0).
        \item \texttt{get\_wallpaper\_rolls(self, roll\_width: float, roll\_length: float) -> int} — количество рулонов обоев. Площадь рулона = \(\texttt{roll\_width} \cdot \texttt{roll\_length}\). Количество рулонов — частное от деления площади под оклейку на площадь рулона, округлённое вверх до целого.
    \end{itemize}

    \item Создайте три различных комнаты с разным числом рамок и проверьте методы.

    \item Запросите у пользователя размеры комнаты и размеры рулона обоев (все — дробные числа).

    \item Выведите площадь под оклейку (м²) и количество рулонов (целое число, округлённое вверх).
\end{enumerate}

\item[19] \textbf{Расчёт площади стен в зависимости от наличия встроенных настенных витрин}
\begin{enumerate}
    \item Создайте два класса: \texttt{WallShowcase} и \texttt{Room}.  
    Класс \texttt{WallShowcase} представляет одну настенную витрину. Его конструктор принимает:  
    \texttt{width} — ширина в метрах,  
    \texttt{height} — высота в метрах (положительные дробные числа).  
    Класс \texttt{Room} описывает комнату с размерами \texttt{width}, \texttt{length}, \texttt{height} и хранит список объектов \texttt{WallShowcase}.

    \item В классе \texttt{Room} реализуйте методы:  
    \begin{itemize}
        \item \texttt{add\_showcase(self, sc: WallShowcase) -> None} — добавляет витрину в комнату.
        \item \texttt{get\_area\_to\_cover(self) -> float} — площадь стен под облицовку: общая площадь стен минус сумма площадей всех витрин (результат $\geqslant$ 0).
        \item \texttt{get\_tiles\_count(self, tile\_width: float, tile\_height: float) -> int} — количество плиток. Площадь одной плитки = \(\texttt{tile\_width} \cdot \texttt{tile\_height}\). Количество плиток — частное от деления площади под облицовку на площадь плитки, округлённое вверх до целого.
    \end{itemize}

    \item Создайте три разных экземпляра \texttt{Room} с разным числом витрин и протестируйте методы.

    \item Запросите у пользователя размеры комнаты и размеры плитки (все — дробные числа).

    \item Выведите площадь под облицовку (м²) и количество плиток (целое число, округлённое вверх).
\end{enumerate}

\item[20] \textbf{Расчёт площади стен в зависимости от наличия встроенных настенных кронштейнов}
\begin{enumerate}
    \item Создайте два класса: \texttt{WallBracket} и \texttt{Room}.  
    Класс \texttt{WallBracket} описывает один настенный кронштейн. Его конструктор принимает:  
    \texttt{width} — ширина в метрах,  
    \texttt{height} — высота в метрах (положительные дробные числа).  
    Класс \texttt{Room} описывает комнату с размерами \texttt{width}, \texttt{length}, \texttt{height} и хранит список объектов \texttt{WallBracket}.

    \item В классе \texttt{Room} реализуйте методы:  
    \begin{itemize}
        \item \texttt{add\_bracket(self, br: WallBracket) -> None} — добавляет кронштейн в комнату.
        \item \texttt{get\_area\_to\_cover(self) -> float} — площадь стен под покраску: общая площадь стен минус сумма площадей всех кронштейнов (результат $\geqslant$ 0).
        \item \texttt{get\_paint\_liters(self, coverage: float) -> float} — объём краски в литрах. Аргумент \texttt{coverage} — расход (м²/л). Объём = площадь под покраску / \texttt{coverage}. Результат может быть дробным.
    \end{itemize}

    \item Создайте три различных комнаты с разным числом кронштейнов и проверьте методы.

    \item Запросите у пользователя размеры комнаты и \texttt{coverage} (дробное число).

    \item Выведите площадь под покраску (м²) и количество литров краски (с дробной частью).
\end{enumerate}

\item[21] \textbf{Расчёт площади стен в зависимости от наличия встроенных настенных жалюзи}
\begin{enumerate}
    \item Создайте два класса: \texttt{WallBlind} и \texttt{Room}.  
    Класс \texttt{WallBlind} представляет одни настенные жалюзи. Его конструктор принимает:  
    \texttt{width} — ширина в метрах,  
    \texttt{height} — высота в метрах (положительные дробные числа).  
    Класс \texttt{Room} описывает комнату с размерами \texttt{width}, \texttt{length}, \texttt{height} и хранит список объектов \texttt{WallBlind}.

    \item В классе \texttt{Room} реализуйте методы:  
    \begin{itemize}
        \item \texttt{add\_blind(self, bl: WallBlind) -> None} — добавляет жалюзи в комнату.
        \item \texttt{get\_area\_to\_cover(self) -> float} — площадь стен под драпировку: общая площадь стен минус сумма площадей всех жалюзи (результат $\geqslant$ 0).
        \item \texttt{get\_fabric\_meters(self, fabric\_width: float) -> float} — длина ткани в метрах. Аргумент \texttt{fabric\_width} — ширина ткани. Длина = площадь под драпировку / \texttt{fabric\_width}. Результат может быть дробным.
    \end{itemize}

    \item Создайте три разных экземпляра \texttt{Room} с разным числом жалюзи и протестируйте методы.

    \item Запросите у пользователя размеры комнаты и ширину ткани (дробные числа).

    \item Выведите площадь под драпировку (м²) и количество метров ткани (с дробной частью).
\end{enumerate}

\item[22] \textbf{Расчёт площади стен в зависимости от наличия встроенных настенных флагов}
\begin{enumerate}
    \item Создайте два класса: \texttt{WallFlag} и \texttt{Room}.  
    Класс \texttt{WallFlag} описывает один настенный флаг. Его конструктор принимает:  
    \texttt{width} — ширина в метрах,  
    \texttt{height} — высота в метрах (положительные дробные числа).  
    Класс \texttt{Room} описывает комнату с размерами \texttt{width}, \texttt{length}, \texttt{height} и хранит список объектов \texttt{WallFlag}.

    \item В классе \texttt{Room} реализуйте методы:  
    \begin{itemize}
        \item \texttt{add\_flag(self, fl: WallFlag) -> None} — добавляет флаг в комнату.
        \item \texttt{get\_area\_to\_cover(self) -> float} — площадь стен под оклейку: общая площадь стен минус сумма площадей всех флагов (результат $\geqslant$ 0).
        \item \texttt{get\_wallpaper\_rolls(self, roll\_width: float, roll\_length: float) -> int} — количество рулонов обоев. Площадь рулона = \(\texttt{roll\_width} \cdot \texttt{roll\_length}\). Количество рулонов — частное от деления площади под оклейку на площадь рулона, округлённое вверх до целого.
    \end{itemize}

    \item Создайте три различных комнаты с разным числом флагов и проверьте методы.

    \item Запросите у пользователя размеры комнаты и размеры рулона обоев (все — дробные числа).

    \item Выведите площадь под оклейку (м²) и количество рулонов (целое число, округлённое вверх).
\end{enumerate}

\item[23] \textbf{Расчёт площади стен в зависимости от наличия встроенных грифельных досок}
\begin{enumerate}
    \item Создайте два класса: \texttt{Chalkboard} и \texttt{Room}.  
    Класс \texttt{Chalkboard} представляет одну грифельную доску. Его конструктор принимает:  
    \texttt{width} — ширина в метрах,  
    \texttt{height} — высота в метрах (положительные дробные числа).  
    Класс \texttt{Room} описывает комнату с размерами \texttt{width}, \texttt{length}, \texttt{height} и хранит список объектов \texttt{Chalkboard}.

    \item В классе \texttt{Room} реализуйте методы:  
    \begin{itemize}
        \item \texttt{add\_board(self, cb: Chalkboard) -> None} — добавляет доску в комнату.
        \item \texttt{get\_area\_to\_cover(self) -> float} — площадь стен под покраску: общая площадь стен минус сумма площадей всех досок (результат $\geqslant$ 0).
        \item \texttt{get\_paint\_cans(self, can\_coverage: float) -> int} — количество банок краски. Аргумент \texttt{can\_coverage} — покрытие одной банки (м²/банка). Количество банок = площадь под покраску / \texttt{can\_coverage}, округлённое вверх до целого.
    \end{itemize}

    \item Создайте три разных экземпляра \texttt{Room} с разным числом досок и протестируйте методы.

    \item Запросите у пользователя размеры комнаты и \texttt{can\_coverage} (дробное число).

    \item Выведите площадь под покраску (м²) и количество банок (целое число, округлённое вверх).
\end{enumerate}

\item[24] \textbf{Расчёт площади стен в зависимости от наличия встроенных маркерных досок}
\begin{enumerate}
    \item Создайте два класса: \texttt{Whiteboard} и \texttt{Room}.  
    Класс \texttt{Whiteboard} описывает одну маркерную доску. Его конструктор принимает:  
    \texttt{width} — ширина в метрах,  
    \texttt{height} — высота в метрах (положительные дробные числа).  
    Класс \texttt{Room} описывает комнату с размерами \texttt{width}, \texttt{length}, \texttt{height} и хранит список объектов \texttt{Whiteboard}.

    \item В классе \texttt{Room} реализуйте методы:  
    \begin{itemize}
        \item \texttt{add\_board(self, wb: Whiteboard) -> None} — добавляет доску в комнату.
        \item \texttt{get\_area\_to\_cover(self) -> float} — площадь стен под отделку: общая площадь стен минус сумма площадей всех досок (результат $\geqslant$ 0).
        \item \texttt{get\_panels\_count(self, panel\_width: float, panel\_height: float) -> int} — количество декоративных панелей. Площадь одной панели = \(\texttt{panel\_width} \cdot \texttt{panel\_height}\). Количество панелей — частное от деления площади под отделку на площадь панели, округлённое вверх до целого.
    \end{itemize}

    \item Создайте три различных комнаты с разным числом досок и проверьте методы.

    \item Запросите у пользователя размеры комнаты и размеры панели (все — дробные числа).

    \item Выведите площадь под отделку (м²) и количество панелей (целое число, округлённое вверх).
\end{enumerate}

\item[25] \textbf{Расчёт площади стен в зависимости от наличия встроенных зеркал}
\begin{enumerate}
    \item Создайте два класса: \texttt{Mirror} и \texttt{Room}.  
    Класс \texttt{Mirror} представляет одно зеркало. Его конструктор принимает:  
    \texttt{width} — ширина в метрах,  
    \texttt{height} — высота в метрах (положительные дробные числа).  
    Класс \texttt{Room} описывает комнату с размерами \texttt{width}, \texttt{length}, \texttt{height} и хранит список объектов \texttt{Mirror}.

    \item В классе \texttt{Room} реализуйте методы:  
    \begin{itemize}
        \item \texttt{add\_mirror(self, m: Mirror) -> None} — добавляет зеркало в комнату.
        \item \texttt{get\_area\_to\_cover(self) -> float} — площадь стен под облицовку: общая площадь стен минус сумма площадей всех зеркал (результат $\geqslant$ 0).
        \item \texttt{get\_tiles\_count(self, tile\_width: float, tile\_height: float) -> int} — количество плиток. Площадь одной плитки = \(\texttt{tile\_width} \cdot \texttt{tile\_height}\). Количество плиток — частное от деления площади под облицовку на площадь плитки, округлённое вверх до целого.
    \end{itemize}

    \item Создайте три разных экземпляра \texttt{Room} с разным числом зеркал и протестируйте методы.

    \item Запросите у пользователя размеры комнаты и размеры плитки (все — дробные числа).

    \item Выведите площадь под облицовку (м²) и количество плиток (целое число, округлённое вверх).
\end{enumerate}

\item[26] \textbf{Расчёт площади стен в зависимости от наличия встроенных часов}
\begin{enumerate}
    \item Создайте два класса: \texttt{WallClock} и \texttt{Room}.  
    Класс \texttt{WallClock} описывает одни настенные часы. Его конструктор принимает:  
    \texttt{width} — ширина в метрах,  
    \texttt{height} — высота в метрах (положительные дробные числа).  
    Класс \texttt{Room} описывает комнату с размерами \texttt{width}, \texttt{length}, \texttt{height} и хранит список объектов \texttt{WallClock}.

    \item В классе \texttt{Room} реализуйте методы:  
    \begin{itemize}
        \item \texttt{add\_clock(self, cl: WallClock) -> None} — добавляет часы в комнату.
        \item \texttt{get\_area\_to\_cover(self) -> float} — площадь стен под оклейку: общая площадь стен минус сумма площадей всех часов (результат $\geqslant$ 0).
        \item \texttt{get\_wallpaper\_rolls(self, roll\_width: float, roll\_length: float) -> int} — количество рулонов обоев. Площадь рулона = \(\texttt{roll\_width} \cdot \texttt{roll\_length}\). Количество рулонов — частное от деления площади под оклейку на площадь рулона, округлённое вверх до целого.
    \end{itemize}

    \item Создайте три различных комнаты с разным числом часов и проверьте методы.

    \item Запросите у пользователя размеры комнаты и размеры рулона обоев (все — дробные числа).

    \item Выведите площадь под оклейку (м²) и количество рулонов (целое число, округлённое вверх).
\end{enumerate}

\item[27] \textbf{Расчёт площади стен в зависимости от наличия встроенных термометров}
\begin{enumerate}
    \item Создайте два класса: \texttt{Thermometer} и \texttt{Room}.  
    Класс \texttt{Thermometer} представляет один настенный термометр. Его конструктор принимает:  
    \texttt{width} — ширина в метрах,  
    \texttt{height} — высота в метрах (положительные дробные числа).  
    Класс \texttt{Room} описывает комнату с размерами \texttt{width}, \texttt{length}, \texttt{height} и хранит список объектов \texttt{Thermometer}.

    \item В классе \texttt{Room} реализуйте методы:  
    \begin{itemize}
        \item \texttt{add\_thermometer(self, t: Thermometer) -> None} — добавляет термометр в комнату.
        \item \texttt{get\_area\_to\_cover(self) -> float} — площадь стен под покраску: общая площадь стен минус сумма площадей всех термометров (результат $\geqslant$ 0).
        \item \texttt{get\_paint\_liters(self, coverage: float) -> float} — объём краски в литрах. Аргумент \texttt{coverage} — расход (м²/л). Объём = площадь под покраску / \texttt{coverage}. Результат может быть дробным.
    \end{itemize}

    \item Создайте три разных экземпляра \texttt{Room} с разным числом термометров и протестируйте методы.

    \item Запросите у пользователя размеры комнаты и \texttt{coverage} (дробное число).

    \item Выведите площадь под покраску (м²) и количество литров краски (с дробной частью).
\end{enumerate}

\item[28] \textbf{Расчёт площади стен в зависимости от наличия встроенных барометров}
\begin{enumerate}
    \item Создайте два класса: \texttt{Barometer} и \texttt{Room}.  
    Класс \texttt{Barometer} описывает один настенный барометр. Его конструктор принимает:  
    \texttt{width} — ширина в метрах,  
    \texttt{height} — высота в метрах (положительные дробные числа).  
    Класс \texttt{Room} описывает комнату с размерами \texttt{width}, \texttt{length}, \texttt{height} и хранит список объектов \texttt{Barometer}.

    \item В классе \texttt{Room} реализуйте методы:  
    \begin{itemize}
        \item \texttt{add\_barometer(self, b: Barometer) -> None} — добавляет барометр в комнату.
        \item \texttt{get\_area\_to\_cover(self) -> float} — площадь стен под драпировку: общая площадь стен минус сумма площадей всех барометров (результат $\geqslant$ 0).
        \item \texttt{get\_fabric\_meters(self, fabric\_width: float) -> float} — длина ткани в метрах. Аргумент \texttt{fabric\_width} — ширина ткани. Длина = площадь под драпировку / \texttt{fabric\_width}. Результат может быть дробным.
    \end{itemize}

    \item Создайте три различных комнаты с разным числом барометров и проверьте методы.

    \item Запросите у пользователя размеры комнаты и ширину ткани (дробные числа).

    \item Выведите площадь под драпировку (м²) и количество метров ткани (с дробной частью).
\end{enumerate}

\item[29] \textbf{Расчёт площади стен в зависимости от наличия встроенных гидрометров}
\begin{enumerate}
    \item Создайте два класса: \texttt{Hygrometer} и \texttt{Room}.  
    Класс \texttt{Hygrometer} представляет один настенный гидрометр. Его конструктор принимает:  
    \texttt{width} — ширина в метрах,  
    \texttt{height} — высота в метрах (положительные дробные числа).  
    Класс \texttt{Room} описывает комнату с размерами \texttt{width}, \texttt{length}, \texttt{height} и хранит список объектов \texttt{Hygrometer}.

    \item В классе \texttt{Room} реализуйте методы:  
    \begin{itemize}
        \item \texttt{add\_hygrometer(self, h: Hygrometer) -> None} — добавляет гидрометр в комнату.
        \item \texttt{get\_area\_to\_cover(self) -> float} — площадь стен под оклейку: общая площадь стен минус сумма площадей всех гидрометров (результат $\geqslant$ 0).
        \item \texttt{get\_wallpaper\_rolls(self, roll\_width: float, roll\_length: float) -> int} — количество рулонов обоев. Площадь рулона = \(\texttt{roll\_width} \cdot \texttt{roll\_length}\). Количество рулонов — частное от деления площади под оклейку на площадь рулона, округлённое вверх до целого.
    \end{itemize}

    \item Создайте три разных экземпляра \texttt{Room} с разным числом гидрометров и протестируйте методы.

    \item Запросите у пользователя размеры комнаты и размеры рулона обоев (все — дробные числа).

    \item Выведите площадь под оклейку (м²) и количество рулонов (целое число, округлённое вверх).
\end{enumerate}

\item[30] \textbf{Расчёт площади стен в зависимости от наличия встроенных настенных растений}
\begin{enumerate}
    \item Создайте два класса: \texttt{WallPlant} и \texttt{Room}.  
    Класс \texttt{WallPlant} описывает одно настенное растение (в кашпо или модуле). Его конструктор принимает:  
    \texttt{width} — ширина в метрах,  
    \texttt{height} — высота в метрах (положительные дробные числа).  
    Класс \texttt{Room} описывает комнату с размерами \texttt{width}, \texttt{length}, \texttt{height} и хранит список объектов \texttt{WallPlant}.

    \item В классе \texttt{Room} реализуйте методы:  
    \begin{itemize}
        \item \texttt{add\_plant(self, p: WallPlant) -> None} — добавляет растение в комнату.
        \item \texttt{get\_area\_to\_cover(self) -> float} — площадь стен под покраску: общая площадь стен минус сумма площадей всех растений (результат $\geqslant$ 0).
        \item \texttt{get\_paint\_cans(self, can\_coverage: float) -> int} — количество банок краски. Аргумент \texttt{can\_coverage} — покрытие одной банки (м²/банка). Количество банок = площадь под покраску / \texttt{can\_coverage}, округлённое вверх до целого.
    \end{itemize}

    \item Создайте три различных комнаты с разным числом растений и проверьте методы.

    \item Запросите у пользователя размеры комнаты и \texttt{can\_coverage} (дробное число).

    \item Выведите площадь под покраску (м²) и количество банок (целое число, округлённое вверх).
\end{enumerate}

\item[31] \textbf{Расчёт площади стен в зависимости от наличия встроенных настенных фонарей}
\begin{enumerate}
    \item Создайте два класса: \texttt{WallLantern} и \texttt{Room}.  
    Класс \texttt{WallLantern} представляет один настенный фонарь. Его конструктор принимает:  
    \texttt{width} — ширина в метрах,  
    \texttt{height} — высота в метрах (положительные дробные числа).  
    Класс \texttt{Room} описывает комнату с размерами \texttt{width}, \texttt{length}, \texttt{height} и хранит список объектов \texttt{WallLantern}.

    \item В классе \texttt{Room} реализуйте методы:  
    \begin{itemize}
        \item \texttt{add\_lantern(self, l: WallLantern) -> None} — добавляет фонарь в комнату.
        \item \texttt{get\_area\_to\_cover(self) -> float} — площадь стен под отделку: общая площадь стен минус сумма площадей всех фонарей (результат $\geqslant$ 0).
        \item \texttt{get\_panels\_count(self, panel\_width: float, panel\_height: float) -> int} — количество декоративных панелей. Площадь одной панели = \(\texttt{panel\_width} \cdot \texttt{panel\_height}\). Количество панелей — частное от деления площади под отделку на площадь панели, округлённое вверх до целого.
    \end{itemize}

    \item Создайте три разных экземпляра \texttt{Room} с разным числом фонарей и протестируйте методы.

    \item Запросите у пользователя размеры комнаты и размеры панели (все — дробные числа).

    \item Выведите площадь под отделку (м²) и количество панелей (целое число, округлённое вверх).
\end{enumerate}

\item[32] \textbf{Расчёт площади стен в зависимости от наличия встроенных настенных вентиляторов}
\begin{enumerate}
    \item Создайте два класса: \texttt{WallFan} и \texttt{Room}.  
    Класс \texttt{WallFan} описывает один настенный вентилятор. Его конструктор принимает:  
    \texttt{width} — ширина в метрах,  
    \texttt{height} — высота в метрах (положительные дробные числа).  
    Класс \texttt{Room} описывает комнату с размерами \texttt{width}, \texttt{length}, \texttt{height} и хранит список объектов \texttt{WallFan}.

    \item В классе \texttt{Room} реализуйте методы:  
    \begin{itemize}
        \item \texttt{add\_fan(self, f: WallFan) -> None} — добавляет вентилятор в комнату.
        \item \texttt{get\_area\_to\_cover(self) -> float} — площадь стен под облицовку: общая площадь стен минус сумма площадей всех вентиляторов (результат $\geqslant$ 0).
        \item \texttt{get\_tiles\_count(self, tile\_width: float, tile\_height: float) -> int} — количество плиток. Площадь одной плитки = \(\texttt{tile\_width} \cdot \texttt{tile\_height}\). Количество плиток — частное от деления площади под облицовку на площадь плитки, округлённое вверх до целого.
    \end{itemize}

    \item Создайте три различных комнаты с разным числом вентиляторов и проверьте методы.

    \item Запросите у пользователя размеры комнаты и размеры плитки (все — дробные числа).

    \item Выведите площадь под облицовку (м²) и количество плиток (целое число, округлённое вверх).
\end{enumerate}

\item[33] \textbf{Расчёт площади стен в зависимости от наличия встроенных увлажнителей}
\begin{enumerate}
    \item Создайте два класса: \texttt{WallHumidifier} и \texttt{Room}.  
    Класс \texttt{WallHumidifier} представляет один настенный увлажнитель. Его конструктор принимает:  
    \texttt{width} — ширина в метрах,  
    \texttt{height} — высота в метрах (положительные дробные числа).  
    Класс \texttt{Room} описывает комнату с размерами \texttt{width}, \texttt{length}, \texttt{height} и хранит список объектов \texttt{WallHumidifier}.

    \item В классе \texttt{Room} реализуйте методы:  
    \begin{itemize}
        \item \texttt{add\_humidifier(self, h: WallHumidifier) -> None} — добавляет увлажнитель в комнату.
        \item \texttt{get\_area\_to\_cover(self) -> float} — площадь стен под оклейку: общая площадь стен минус сумма площадей всех увлажнителей (результат $\geqslant$ 0).
        \item \texttt{get\_wallpaper\_rolls(self, roll\_width: float, roll\_length: float) -> int} — количество рулонов обоев. Площадь рулона = \(\texttt{roll\_width} \cdot \texttt{roll\_length}\). Количество рулонов — частное от деления площади под оклейку на площадь рулона, округлённое вверх до целого.
    \end{itemize}

    \item Создайте три разных экземпляра \texttt{Room} с разным числом увлажнителей и протестируйте методы.

    \item Запросите у пользователя размеры комнаты и размеры рулона обоев (все — дробные числа).

    \item Выведите площадь под оклейку (м²) и количество рулонов (целое число, округлённое вверх).
\end{enumerate}

\item[34] \textbf{Расчёт площади стен в зависимости от наличия встроенных обогревателей}
\begin{enumerate}
    \item Создайте два класса: \texttt{WallHeater} и \texttt{Room}.  
    Класс \texttt{WallHeater} описывает один настенный обогреватель. Его конструктор принимает:  
    \texttt{width} — ширина в метрах,  
    \texttt{height} — высота в метрах (положительные дробные числа).  
    Класс \texttt{Room} описывает комнату с размерами \texttt{width}, \texttt{length}, \texttt{height} и хранит список объектов \texttt{WallHeater}.

    \item В классе \texttt{Room} реализуйте методы:  
    \begin{itemize}
        \item \texttt{add\_heater(self, h: WallHeater) -> None} — добавляет обогреватель в комнату.
        \item \texttt{get\_area\_to\_cover(self) -> float} — площадь стен под покраску: общая площадь стен минус сумма площадей всех обогревателей (результат $\geqslant$ 0).
        \item \texttt{get\_paint\_liters(self, coverage: float) -> float} — объём краски в литрах. Аргумент \texttt{coverage} — расход (м²/л). Объём = площадь под покраску / \texttt{coverage}. Результат может быть дробным.
    \end{itemize}

    \item Создайте три различных комнаты с разным числом обогревателей и проверьте методы.

    \item Запросите у пользователя размеры комнаты и \texttt{coverage} (дробное число).

    \item Выведите площадь под покраску (м²) и количество литров краски (с дробной частью).
\end{enumerate}

\item[35] \textbf{Расчёт площади стен в зависимости от наличия окон и дверей}
\begin{enumerate}
    \item Создайте два класса: \texttt{WinDoor} и \texttt{Room}.  
    Класс \texttt{WinDoor} представляет один проём (окно или дверь). Его конструктор принимает:  
    \texttt{width} — ширина проёма в метрах,  
    \texttt{height} — высота проёма в метрах (положительные дробные числа).  
    Класс \texttt{Room} описывает комнату с размерами \texttt{width}, \texttt{length}, \texttt{height} и хранит список объектов \texttt{WinDoor}.

    \item В классе \texttt{Room} реализуйте методы:  
    \begin{itemize}
        \item \texttt{add\_windoor(self, wd: WinDoor) -> None} — добавляет проём в комнату.
        \item \texttt{get\_area\_to\_cover(self) -> float} — площадь стен под оклейку: общая площадь стен минус сумма площадей всех проёмов (результат $\geqslant$ 0).
        \item \texttt{get\_rolls\_count(self, roll\_width: float, roll\_length: float) -> int} — количество рулонов обоев. Площадь одного рулона = \(\texttt{roll\_width} \cdot \texttt{roll\_length}\). Количество рулонов — частное от деления площади под оклейку на площадь рулона, округлённое вверх до целого.
    \end{itemize}

    \item Создайте три разных экземпляра \texttt{Room} с разным числом проёмов и протестируйте методы.

    \item Запросите у пользователя размеры комнаты и размеры рулона обоев (все — дробные числа).

    \item Выведите площадь под оклейку (м²) и количество рулонов обоев (целое число, округлённое вверх).
\end{enumerate}
\end{enumerate}

\subsubsection{Задача 3}

\begin{enumerate}
    \item \textbf{Моделирование поединка между двумя дуэлянтами}

\begin{enumerate}
    \item Импортируйте функцию \texttt{randint} из модуля \texttt{random}.

    \item Создайте класс \texttt{Duelist} («Дуэлянт»).  
    В конструкторе класса должны задаваться имя дуэлянта и его начальное здоровье (по умолчанию — 100 единиц).  
    Также реализуйте следующие методы:
    \begin{itemize}
        \item \texttt{set\_name} — позволяет изменить имя дуэлянта;
        \item \texttt{attack} — моделирует атаку на другого дуэлянта: генерирует случайный урон в диапазоне от 10 до 30 и уменьшает здоровье противника на эту величину.
    \end{itemize}

    \item Создайте класс \texttt{Duel} («Дуэль»).  
    Его конструктор принимает двух дуэлянтов и сохраняет их как внутренние атрибуты. Также в конструкторе инициализируется пустая строка для хранения результата поединка.

    \item Реализуйте метод \texttt{fight}, который моделирует сам поединок:
    \begin{itemize}
        \item Поединок продолжается, пока у обоих дуэлянтов здоровье больше нуля;
        \item На каждом шаге случайным образом (с равной вероятностью) выбирается, кто из дуэлянтов наносит удар;
        \item После каждой атаки, если здоровье любого из участников стало меньше или равно нулю, оно устанавливается в ноль.
    \end{itemize}

    \item После завершения поединка определите его исход:
    \begin{itemize}
        \item Если у первого дуэлянта осталось здоровье, а у второго — нет, побеждает первый;
        \item Если у второго осталось здоровье, а у первого — нет, побеждает второй;
        \item Если здоровье обоих участников равно нулю, объявляется ничья.
    \end{itemize}
    Результат сохраняется в виде понятной строки (например, «Алексей побеждает!» или «Ничья!»).

    \item Добавьте метод \texttt{who\_wins}, который выводит на экран сохранённый результат поединка.
\end{enumerate}

\item \textbf{Моделирование боя между двумя боксёрами}

\begin{enumerate}
    \item Импортируйте функцию \texttt{randint} из модуля \texttt{random}.

    \item Создайте класс \texttt{Boxer} («Боксёр»).  
    В конструкторе задаются имя и начальное здоровье (по умолчанию — 100).  
    Реализуйте методы:
    \begin{itemize}
        \item \texttt{set\_name} — изменение имени боксёра;
        \item \texttt{punch} — нанесение удара противнику с уроном от 10 до 30.
    \end{itemize}

    \item Создайте класс \texttt{BoxingMatch} («Боксёрский поединок»).  
    Его конструктор принимает двух боксёров и инициализирует атрибут для хранения результата.

    \item Реализуйте метод \texttt{match}, моделирующий бой по тем же правилам, что и в первом задании: случайный выбор атакующего, цикл до тех пор, пока у одного из участников не закончится здоровье, коррекция здоровья до нуля при необходимости.

    \item После завершения боя определите победителя или ничью и сохраните результат в виде строки.

    \item Добавьте метод \texttt{announce\_winner}, выводящий результат на экран.
\end{enumerate}

\item \textbf{Моделирование поединка между двумя шахматистами}

\begin{enumerate}
    \item Импортируйте функцию \texttt{randint}.

    \item Создайте класс \texttt{ChessPlayer} («Шахматист»).  
    В конструкторе задаются имя и начальное здоровье (по умолчанию — 100).  
    Реализуйте методы:
    \begin{itemize}
        \item \texttt{set\_name} — изменение имени;
        \item \texttt{play\_move} — «ход» в рамках метафорического интеллектуального поединка, наносящий урон от 10 до 30.
    \end{itemize}

    \item Создайте класс \texttt{ChessGame} («Шахматная партия»), принимающий двух шахматистов и хранящий результат.

    \item Реализуйте метод \texttt{simulate}, моделирующий поединок: случайный выбор ходящего, цикл до обнуления здоровья одного или обоих участников.

    \item Определите победителя или ничью по оставшемуся здоровью.

    \item Добавьте метод \texttt{show\_result}, выводящий итог партии.
\end{enumerate}

\item \textbf{Моделирование схватки между двумя борцами}

\begin{enumerate}
    \item Импортируйте функцию \texttt{randint}.

    \item Создайте класс \texttt{Wrestler} («Борец») с именем и здоровьем (по умолчанию — 100).  
    Методы:
    \begin{itemize}
        \item \texttt{set\_name} — изменение имени;
        \item \texttt{grapple} — захват, наносящий урон от 10 до 30.
    \end{itemize}

    \item Создайте класс \texttt{WrestlingMatch} («Борцовский поединок»), принимающий двух борцов.

    \item Реализуйте метод \texttt{compete}, моделирующий схватку по стандартной схеме: случайный выбор атакующего, цикл до поражения одного или обоих.

    \item Определите исход схватки и сохраните его в виде строки.

    \item Добавьте метод \texttt{declare\_champion}, выводящий победителя.
\end{enumerate}

\item \textbf{Моделирование битвы между двумя магами}

\begin{enumerate}
    \item Импортируйте функцию \texttt{randint}.

    \item Создайте класс \texttt{Mage} («Маг») с именем и здоровьем (по умолчанию — 100).  
    Методы:
    \begin{itemize}
        \item \texttt{set\_name} — изменение имени;
        \item \texttt{cast\_spell} — заклинание, наносящее урон от 10 до 30.
    \end{itemize}

    \item Создайте класс \texttt{MagicDuel} («Магическая дуэль»), принимающий двух магов.

    \item Реализуйте метод \texttt{duel}, моделирующий битву по стандартной логике.

    \item Определите победителя или ничью.

    \item Добавьте метод \texttt{reveal\_winner}, выводящий результат.
\end{enumerate}

\item \textbf{Моделирование поединка между двумя киберспортсменами}

\begin{enumerate}
    \item Импортируйте функцию \texttt{randint}.

    \item Создайте класс \texttt{Gamer} («Киберспортсмен») с именем и здоровьем (по умолчанию — 100).  
    Методы:
    \begin{itemize}
        \item \texttt{set\_name} — изменение имени;
        \item \texttt{make\_move} — игровой ход в метафоре «кибер-боя», наносящий урон от 10 до 30.
    \end{itemize}

    \item Создайте класс \texttt{EsportsMatch} («Кибертурнир»), принимающий двух игроков.

    \item Реализуйте метод \texttt{play}, моделирующий поединок.

    \item Определите победителя или ничью.

    \item Добавьте метод \texttt{show\_champion}, выводящий чемпиона.
\end{enumerate}

\item \textbf{Моделирование соревнования между двумя пловцами}

\begin{enumerate}
    \item Импортируйте функцию \texttt{randint}.

    \item Создайте класс \texttt{Swimmer} («Пловец») с именем и здоровьем (по умолчанию — 100).  
    Методы:
    \begin{itemize}
        \item \texttt{set\_name} — изменение имени;
        \item \texttt{swim\_lap} — заплыв в игровой интерпретации как «атака», наносящая урон от 10 до 30.
    \end{itemize}

    \item Создайте класс \texttt{SwimRace} («Заплыв»), принимающий двух пловцов.

    \item Реализуйте метод \texttt{race}, моделирующий соревнование.

    \item Определите победителя или ничью.

    \item Добавьте метод \texttt{announce\_medal}, выводящий результат.
\end{enumerate}

\item \textbf{Моделирование боя между двумя роботами}

\begin{enumerate}
    \item Импортируйте функцию \texttt{randint}.

    \item Создайте класс \texttt{RobotFighter} («Боевой робот») с именем и здоровьем (по умолчанию — 100).  
    Методы:
    \begin{itemize}
        \item \texttt{set\_name} — изменение имени;
        \item \texttt{strike} — удар, наносящий урон от 10 до 30.
    \end{itemize}

    \item Создайте класс \texttt{RobotBattle} («Робобой»), принимающий двух роботов.

    \item Реализуйте метод \texttt{fight}, моделирующий сражение.

    \item Определите победителя или ничью.

    \item Добавьте метод \texttt{display\_result}, выводящий результат.
\end{enumerate}

\item \textbf{Моделирование дуэли между двумя ковбоями}

\begin{enumerate}
    \item Импортируйте функцию \texttt{randint}.

    \item Создайте класс \texttt{Cowboy} («Ковбой») с именем и здоровьем (по умолчанию — 100).  
    Методы:
    \begin{itemize}
        \item \texttt{set\_name} — изменение имени;
        \item \texttt{draw} — выстрел, наносящий урон от 10 до 30.
    \end{itemize}

    \item Создайте класс \texttt{Showdown} («Разборка»), принимающий двух ковбоев.

    \item Реализуйте метод \texttt{shootout}, моделирующий перестрелку.

    \item Определите победителя или ничью.

    \item Добавьте метод \texttt{proclaim\_winner}, выводящий результат.
\end{enumerate}

\item \textbf{Моделирование битвы между двумя ниндзя}

\begin{enumerate}
    \item Импортируйте функцию \texttt{randint}.

    \item Создайте класс \texttt{Ninja} («Ниндзя») с именем и здоровьем (по умолчанию — 100).  
    Методы:
    \begin{itemize}
        \item \texttt{set\_name} — изменение имени;
        \item \texttt{throw\_shuriken} — бросок сюрикена, наносящий урон от 10 до 30.
    \end{itemize}

    \item Создайте класс \texttt{NinjaClash} («Столкновение ниндзя»), принимающий двух бойцов.

    \item Реализуйте метод \texttt{clash}, моделирующий битву.

    \item Определите победителя или ничью.

    \item Добавьте метод \texttt{declare\_victor}, выводящий результат.
\end{enumerate}

\item \textbf{Моделирование поединка между двумя пиратами}

\begin{enumerate}
    \item Импортируйте функцию \texttt{randint}.

    \item Создайте класс \texttt{Pirate} («Пират») с именем и здоровьем (по умолчанию — 100).  
    Методы:
    \begin{itemize}
        \item \texttt{set\_name} — изменение имени;
        \item \texttt{sword\_fight} — удар мечом, наносящий урон от 10 до 30.
    \end{itemize}

    \item Создайте класс \texttt{PirateDuel} («Пиратская дуэль»), принимающий двух пиратов.

    \item Реализуйте метод \texttt{battle}, моделирующий схватку.

    \item Определите победителя или ничью.

    \item Добавьте метод \texttt{shout\_winner}, выводящий результат.
\end{enumerate}

\item \textbf{Моделирование схватки между двумя гладиаторами}

\begin{enumerate}
    \item Импортируйте функцию \texttt{randint}.

    \item Создайте класс \texttt{Gladiator} («Гладиатор») с именем и здоровьем (по умолчанию — 100).  
    Методы:
    \begin{itemize}
        \item \texttt{set\_name} — изменение имени;
        \item \texttt{attack\_with\_sword} — удар мечом, наносящий урон от 10 до 30.
    \end{itemize}

    \item Создайте класс \texttt{ArenaFight} («Арена»), принимающий двух гладиаторов.

    \item Реализуйте метод \texttt{fight\_to\_death}, моделирующий бой до конца.

    \item Определите победителя или ничью.

    \item Добавьте метод \texttt{crowd\_cheers}, выводящий результат.
\end{enumerate}

\item \textbf{Моделирование поединка между двумя самураями}

\begin{enumerate}
    \item Импортируйте функцию \texttt{randint}.

    \item Создайте класс \texttt{Samurai} («Самурай») с именем и здоровьем (по умолчанию — 100).  
    Методы:
    \begin{itemize}
        \item \texttt{set\_name} — изменение имени;
        \item \texttt{katana\_strike} — удар катаной, наносящий урон от 10 до 30.
    \end{itemize}

    \item Создайте класс \texttt{SamuraiDuel} («Самурайская дуэль»), принимающий двух самураев.

    \item Реализуйте метод \texttt{duel}, моделирующий поединок.

    \item Определите победителя или ничью.

    \item Добавьте метод \texttt{bow\_to\_winner}, выводящий результат.
\end{enumerate}

\item \textbf{Моделирование поединка между двумя драконами}

\begin{enumerate}
    \item Импортируйте функцию \texttt{randint}.

    \item Создайте класс \texttt{Dragon} («Дракон») с именем и здоровьем (по умолчанию — 100).  
    Методы:
    \begin{itemize}
        \item \texttt{set\_name} — изменение имени;
        \item \texttt{breathe\_fire} — огненное дыхание, наносящее урон от 10 до 30.
    \end{itemize}

    \item Создайте класс \texttt{DragonBattle} («Битва драконов»), принимающий двух драконов.

    \item Реализуйте метод \texttt{clash}, моделирующий сражение.

    \item Определите победителя или ничью.

    \item Добавьте метод \texttt{roar\_victory}, выводящий результат.
\end{enumerate}

\item \textbf{Моделирование битвы между двумя титанами}

\begin{enumerate}
    \item Импортируйте функцию \texttt{randint}.

    \item Создайте класс \texttt{Titan} («Титан») с именем и здоровьем (по умолчанию — 100).  
    Методы:
    \begin{itemize}
        \item \texttt{set\_name} — изменение имени;
        \item \texttt{stomp} — топот, наносящий урон от 10 до 30.
    \end{itemize}

    \item Создайте класс \texttt{TitanClash} («Столкновение титанов»), принимающий двух титанов.

    \item Реализуйте метод \texttt{battle}, моделирующий битву.

    \item Определите победителя или ничью.

    \item Добавьте метод \texttt{earth\_shakes}, выводящий результат.
\end{enumerate}

\item \textbf{Моделирование поединка между двумя рыцарями}

\begin{enumerate}
    \item Импортируйте функцию \texttt{randint}.

    \item Создайте класс \texttt{Knight} («Рыцарь») с именем и здоровьем (по умолчанию — 100).  
    Методы:
    \begin{itemize}
        \item \texttt{set\_name} — изменение имени;
        \item \texttt{lance\_charge} — рывок с копьём, наносящий урон от 10 до 30.
    \end{itemize}

    \item Создайте класс \texttt{Joust} («Турнир»), принимающий двух рыцарей.

    \item Реализуйте метод \texttt{tournament}, моделирующий поединок.

    \item Определите победителя или ничью.

    \item Добавьте метод \texttt{king\_declares}, выводящий результат.
\end{enumerate}

\item \textbf{Моделирование поединка между двумя ведьмами}

\begin{enumerate}
    \item Импортируйте функцию \texttt{randint}.

    \item Создайте класс \texttt{Witch} («Ведьма») с именем и здоровьем (по умолчанию — 100).  
    Методы:
    \begin{itemize}
        \item \texttt{set\_name} — изменение имени;
        \item \texttt{brew\_curse} — наложение проклятия, наносящего урон от 10 до 30.
    \end{itemize}

    \item Создайте класс \texttt{WitchDuel} («Ведьмин поединок»), принимающий двух ведьм.

    \item Реализуйте метод \texttt{hex\_battle}, моделирующий битву.

    \item Определите победителя или ничью.

    \item Добавьте метод \texttt{cackle\_in\_triumph}, выводящий результат.
\end{enumerate}

\item \textbf{Моделирование боя между двумя зомби}

\begin{enumerate}
    \item Импортируйте функцию \texttt{randint}.

    \item Создайте класс \texttt{Zombie} («Зомби») с именем и здоровьем (по умолчанию — 100).  
    Методы:
    \begin{itemize}
        \item \texttt{set\_name} — изменение имени;
        \item \texttt{bite} — укус, наносящий урон от 10 до 30.
    \end{itemize}

    \item Создайте класс \texttt{ZombieFight} («Зомби-битва»), принимающий двух зомби.

    \item Реализуйте метод \texttt{apocalypse}, моделирующий сражение.

    \item Определите победителя или ничью.

    \item Добавьте метод \texttt{groan\_winner}, выводящий результат.
\end{enumerate}

\item \textbf{Моделирование схватки между двумя вампирами}

\begin{enumerate}
    \item Импортируйте функцию \texttt{randint}.

    \item Создайте класс \texttt{Vampire} («Вампир») с именем и здоровьем (по умолчанию — 100).  
    Методы:
    \begin{itemize}
        \item \texttt{set\_name} — изменение имени;
        \item \texttt{drain} — высасывание жизненных сил, наносящее урон от 10 до 30.
    \end{itemize}

    \item Создайте класс \texttt{VampireDuel} («Вампирская дуэль»), принимающий двух вампиров.

    \item Реализуйте метод \texttt{night\_fight}, моделирующий ночную битву.

    \item Определите победителя или ничью.

    \item Добавьте метод \texttt{howl\_at\_moon}, выводящий результат.
\end{enumerate}

\item \textbf{Моделирование битвы между двумя оборотнями}

\begin{enumerate}
    \item Импортируйте функцию \texttt{randint}.

    \item Создайте класс \texttt{Werewolf} («Оборотень») с именем и здоровьем (по умолчанию — 100).  
    Методы:
    \begin{itemize}
        \item \texttt{set\_name} — изменение имени;
        \item \texttt{claw} — удар когтями, наносящий урон от 10 до 30.
    \end{itemize}

    \item Создайте класс \texttt{MoonBattle} («Лунная битва»), принимающий двух оборотней.

    \item Реализуйте метод \texttt{howl\_and\_fight}, моделирующий сражение.

    \item Определите победителя или ничью.

    \item Добавьте метод \texttt{moon\_witnesses}, выводящий результат.
\end{enumerate}

\item \textbf{Моделирование поединка между двумя призраками}

\begin{enumerate}
    \item Импортируйте функцию \texttt{randint}.

    \item Создайте класс \texttt{Ghost} («Призрак») с именем и здоровьем (по умолчанию — 100).  
    Методы:
    \begin{itemize}
        \item \texttt{set\_name} — изменение имени;
        \item \texttt{terrify} — устрашение, наносящее урон от 10 до 30.
    \end{itemize}

    \item Создайте класс \texttt{HauntedDuel} («Призрачная дуэль»), принимающий двух призраков.

    \item Реализуйте метод \texttt{scare\_off}, моделирующий поединок.

    \item Определите победителя или ничью.

    \item Добавьте метод \texttt{echo\_victory}, выводящий результат.
\end{enumerate}

\item \textbf{Моделирование боя между двумя гоблинами}

\begin{enumerate}
    \item Импортируйте функцию \texttt{randint}.

    \item Создайте класс \texttt{Goblin} («Гоблин») с именем и здоровьем (по умолчанию — 100).  
    Методы:
    \begin{itemize}
        \item \texttt{set\_name} — изменение имени;
        \item \texttt{stab} — удар кинжалом, наносящий урон от 10 до 30.
    \end{itemize}

    \item Создайте класс \texttt{GoblinSkirmish} («Гоблинская стычка»), принимающий двух гоблинов.

    \item Реализуйте метод \texttt{loot\_fight}, моделирующий драку.

    \item Определите победителя или ничью.

    \item Добавьте метод \texttt{squeal\_winner}, выводящий результат.
\end{enumerate}

\item \textbf{Моделирование схватки между двумя орками}

\begin{enumerate}
    \item Импортируйте функцию \texttt{randint}.

    \item Создайте класс \texttt{Orc} («Орк») с именем и здоровьем (по умолчанию — 100).  
    Методы:
    \begin{itemize}
        \item \texttt{set\_name} — изменение имени;
        \item \texttt{bash} — мощный удар, наносящий урон от 10 до 30.
    \end{itemize}

    \item Создайте класс \texttt{OrcBattle} («Орковская битва»), принимающий двух орков.

    \item Реализуйте метод \texttt{war\_cry}, моделирующий сражение.

    \item Определите победителя или ничью.

    \item Добавьте метод \texttt{grunt\_victory}, выводящий результат.
\end{enumerate}

\item \textbf{Моделирование битвы между двумя эльфами}

\begin{enumerate}
    \item Импортируйте функцию \texttt{randint}.

    \item Создайте класс \texttt{Elf} («Эльф») с именем и здоровьем (по умолчанию — 100).  
    Методы:
    \begin{itemize}
        \item \texttt{set\_name} — изменение имени;
        \item \texttt{arrow\_shot} — выстрел из лука, наносящий урон от 10 до 30.
    \end{itemize}

    \item Создайте класс \texttt{ElvenDuel} («Эльфийская дуэль»), принимающий двух эльфов.

    \item Реализуйте метод \texttt{forest\_clash}, моделирующий поединок.

    \item Определите победителя или ничью.

    \item Добавьте метод \texttt{whisper\_winner}, выводящий результат.
\end{enumerate}

\item \textbf{Моделирование поединка между двумя гномами}

\begin{enumerate}
    \item Импортируйте функцию \texttt{randint}.

    \item Создайте класс \texttt{Dwarf} («Гном») с именем и здоровьем (по умолчанию — 100).  
    Методы:
    \begin{itemize}
        \item \texttt{set\_name} — изменение имени;
        \item \texttt{swing\_axe} — удар топором, наносящий урон от 10 до 30.
    \end{itemize}

    \item Создайте класс \texttt{DwarfFight} («Гномья драка»), принимающий двух гномов.

    \item Реализуйте метод \texttt{mine\_battle}, моделирующий сражение.

    \item Определите победителя или ничью.

    \item Добавьте метод \texttt{roar\_ale}, выводящий результат.
\end{enumerate}

\item \textbf{Моделирование боя между двумя кентаврами}

\begin{enumerate}
    \item Импортируйте функцию \texttt{randint}.

    \item Создайте класс \texttt{Centaur} («Кентавр») с именем и здоровьем (по умолчанию — 100).  
    Методы:
    \begin{itemize}
        \item \texttt{set\_name} — изменение имени;
        \item \texttt{gallop\_attack} — атака в галопе, наносящая урон от 10 до 30.
    \end{itemize}

    \item Создайте класс \texttt{CentaurClash} («Столкновение кентавров»), принимающий двух кентавров.

    \item Реализуйте метод \texttt{plain\_duel}, моделирующий поединок.

    \item Определите победителя или ничью.

    \item Добавьте метод \texttt{neigh\_victory}, выводящий результат.
\end{enumerate}

\item \textbf{Моделирование схватки между двумя минотаврами}

\begin{enumerate}
    \item Импортируйте функцию \texttt{randint}.

    \item Создайте класс \texttt{Minotaur} («Минотавр») с именем и здоровьем (по умолчанию — 100).  
    Методы:
    \begin{itemize}
        \item \texttt{set\_name} — изменение имени;
        \item \texttt{gore} — удар рогами, наносящий урон от 10 до 30.
    \end{itemize}

    \item Создайте класс \texttt{LabyrinthFight} («Лабиринтная битва»), принимающий двух минотавров.

    \item Реализуйте метод \texttt{maze\_battle}, моделирующий сражение.

    \item Определите победителя или ничью.

    \item Добавьте метод \texttt{bellow\_winner}, выводящий результат.
\end{enumerate}

\item \textbf{Моделирование битвы между двумя фениксами}

\begin{enumerate}
    \item Импортируйте функцию \texttt{randint}.

    \item Создайте класс \texttt{Phoenix} («Феникс») с именем и здоровьем (по умолчанию — 100).  
    Методы:
    \begin{itemize}
        \item \texttt{set\_name} — изменение имени;
        \item \texttt{rebirth\_strike} — удар, связанный с возрождением, наносящий урон от 10 до 30.
    \end{itemize}

    \item Создайте класс \texttt{PhoenixClash} («Столкновение фениксов»), принимающий двух фениксов.

    \item Реализуйте метод \texttt{ash\_duel}, моделирующий поединок.

    \item Определите победителя или ничью.

    \item Добавьте метод \texttt{soar\_victorious}, выводящий результат.
\end{enumerate}

\item \textbf{Моделирование поединка между двумя единорогами}

\begin{enumerate}
    \item Импортируйте функцию \texttt{randint}.

    \item Создайте класс \texttt{Unicorn} («Единорог») с именем и здоровьем (по умолчанию — 100).  
    Методы:
    \begin{itemize}
        \item \texttt{set\_name} — изменение имени;
        \item \texttt{horn\_charge} — удар рогом, наносящий урон от 10 до 30.
    \end{itemize}

    \item Создайте класс \texttt{UnicornDuel} («Дуэль единорогов»), принимающий двух единорогов.

    \item Реализуйте метод \texttt{meadow\_clash}, моделирующий поединок.

    \item Определите победителя или ничью.

    \item Добавьте метод \texttt{gallop\_in\_glory}, выводящий результат.
\end{enumerate}

\item \textbf{Моделирование боя между двумя троллями}

\begin{enumerate}
    \item Импортируйте функцию \texttt{randint}.

    \item Создайте класс \texttt{Troll} («Тролль») с именем и здоровьем (по умолчанию — 100).  
    Методы:
    \begin{itemize}
        \item \texttt{set\_name} — изменение имени;
        \item \texttt{club\_smash} — удар дубиной, наносящий урон от 10 до 30.
    \end{itemize}

    \item Создайте класс \texttt{TrollFight} («Троллья драка»), принимающий двух троллей.

    \item Реализуйте метод \texttt{bridge\_battle}, моделирующий сражение.

    \item Определите победителя или ничью.

    \item Добавьте метод \texttt{grunt\_and\_laugh}, выводящий результат.
\end{enumerate}

\item \textbf{Моделирование схватки между двумя грифонами}

\begin{enumerate}
    \item Импортируйте функцию \texttt{randint}.

    \item Создайте класс \texttt{Griffin} («Грифон») с именем и здоровьем (по умолчанию — 100).  
    Методы:
    \begin{itemize}
        \item \texttt{set\_name} — изменение имени;
        \item \texttt{dive\_attack} — пикирующая атака, наносящая урон от 10 до 30.
    \end{itemize}

    \item Создайте класс \texttt{GriffinClash} («Столкновение грифонов»), принимающий двух грифонов.

    \item Реализуйте метод \texttt{aerial\_duel}, моделирующий воздушный поединок.

    \item Определите победителя или ничью.

    \item Добавьте метод \texttt{screech\_victory}, выводящий результат.
\end{enumerate}

\item \textbf{Моделирование битвы между двумя драконоборцами}

\begin{enumerate}
    \item Импортируйте функцию \texttt{randint}.

    \item Создайте класс \texttt{Dragonslayer} («Драконоборец») с именем и здоровьем (по умолчанию — 100).  
    Методы:
    \begin{itemize}
        \item \texttt{set\_name} — изменение имени;
        \item \texttt{slay} — удар, направленный на убийство, наносящий урон от 10 до 30.
    \end{itemize}

    \item Создайте класс \texttt{SlayerDuel} («Дуэль драконоборцев»), принимающий двух героев.

    \item Реализуйте метод \texttt{heroic\_fight}, моделирующий битву.

    \item Определите победителя или ничью.

    \item Добавьте метод \texttt{bard\_sings}, выводящий результат.
\end{enumerate}

\item \textbf{Моделирование поединка между двумя наёмниками}

\begin{enumerate}
    \item Импортируйте функцию \texttt{randint}.

    \item Создайте класс \texttt{Mercenary} («Наёмник») с именем и здоровьем (по умолчанию — 100).  
    Методы:
    \begin{itemize}
        \item \texttt{set\_name} — изменение имени;
        \item \texttt{strike\_for\_hire} — удар за плату, наносящий урон от 10 до 30.
    \end{itemize}

    \item Создайте класс \texttt{MercenaryClash} («Стычка наёмников»), принимающий двух бойцов.

    \item Реализуйте метод \texttt{contract\_battle}, моделирующий сражение.

    \item Определите победителя или ничью.

    \item Добавьте метод \texttt{count\_coins}, выводящий результат.
\end{enumerate}

\item \textbf{Моделирование боя между двумя ассасинами}

\begin{enumerate}
    \item Импортируйте функцию \texttt{randint}.

    \item Создайте класс \texttt{Assassin} («Ассасин») с именем и здоровьем (по умолчанию — 100).  
    Методы:
    \begin{itemize}
        \item \texttt{set\_name} — изменение имени;
        \item \texttt{backstab} — удар в спину, наносящий урон от 10 до 30.
    \end{itemize}

    \item Создайте класс \texttt{ShadowDuel} («Теневая дуэль»), принимающий двух ассасинов.

    \item Реализуйте метод \texttt{night\_kill}, моделирующий ночной бой.

    \item Определите победителя или ничью.

    \item Добавьте метод \texttt{vanish\_in\_dark}, выводящий результат.
\end{enumerate}

\item \textbf{Моделирование сражения между двумя солдатами} (оригинальный вариант)

\begin{enumerate}
    \item Импортируйте функцию \texttt{randint} из модуля \texttt{random}.

    \item Создайте класс \texttt{Soldier} («Солдат»).  
    В конструкторе задаются имя и начальное здоровье (по умолчанию — 100).  
    Реализуйте методы:
    \begin{itemize}
        \item \texttt{set\_name} — изменение имени;
        \item \texttt{attack} — атака противника с уроном от 10 до 30.
    \end{itemize}

    \item Создайте класс \texttt{Battle} («Сражение»), принимающий двух солдат и хранящий результат.

    \item Реализуйте метод \texttt{battle}, моделирующий бой:
    \begin{itemize}
        \item Поединок продолжается, пока у обоих солдат здоровье больше нуля;
        \item Атакующий выбирается случайно;
        \item После каждой атаки здоровье, упавшее до нуля или ниже, устанавливается в ноль.
    \end{itemize}

    \item После завершения боя определите исход: победа одного из солдат или ничья — и сохраните результат в виде строки.

    \item Добавьте метод \texttt{who\_win}, выводящий результат на экран.
\end{enumerate}

\end{enumerate}