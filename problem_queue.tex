\begin{enumerate}
\item Написать программу на Python, которая создает класс Queue для представления структуры данных очереди с инкапсуляцией. Класс должен содержать методы enqueue, dequeue и is\_empty, которые реализуют операции добавления элементов в очередь, удаления элементов из очереди и проверки пустоты очереди соответственно. Программа также должна создавать экземпляр класса Queue, добавлять элементы в очередь, удалять элементы из очереди и выводить информацию о состоянии очереди на экран.

Инструкции:
\begin{enumerate}
    \item Создайте класс Queue с методом \_\_init\_\_, который инициализирует пустую очередь (внутренний список \_elements). Принимает необязательный параметр max\_size (по умолчанию None — без ограничений).
    \item Создайте метод enqueue, который принимает элемент и добавляет его в конец очереди, только если не превышает max\_size. Если превышает — выбрасывает ValueError("Очередь переполнена").
    \item Создайте метод dequeue, который удаляет и возвращает элемент из начала очереди. Если очередь пуста — выбрасывает IndexError("Очередь пуста").
    \item Создайте метод is\_empty, который возвращает True, если очередь пуста, и False в противном случае.
    \item Создайте приватный метод \_debug\_list (только для отладки, не включайте в задание студентам; в решении можно использовать queue.\_elements) для вывода внутреннего состояния.
    \item Создайте экземпляр класса Queue с max\_size=5.
    \item Добавьте элементы: 100, 200, 300, 400, 500.
    \item Попытайтесь добавить 600 — должно вызвать исключение (перехватите его и выведите сообщение).
    \item Выведите текущее состояние очереди.
    \item Вызовите dequeue дважды, выводя каждый раз удаленный элемент.
    \item Выведите обновленное состояние очереди.
\end{enumerate}

Пример использования:
\begin{lstlisting}[language=Python]
queue = Queue(max_size=5)
queue.enqueue(100)
queue.enqueue(200)
queue.enqueue(300)
queue.enqueue(400)
queue.enqueue(500)

try:
    queue.enqueue(600)
except ValueError as e:
    print("Ошибка:", e)

print("Current Queue:", queue._elements)  # только для проверки

dequeued_item = queue.dequeue()
print("Dequeued item:", dequeued_item)

dequeued_item = queue.dequeue()
print("Dequeued item:", dequeued_item)

print("Updated Queue:", queue._elements)
\end{lstlisting}

\item Написать программу на Python, которая создает класс Queue для представления структуры данных очереди с инкапсуляцией. Класс должен содержать методы enqueue, dequeue и is\_empty, которые реализуют операции добавления элементов в очередь, удаления элементов из очереди и проверки пустоты очереди соответственно. Программа также должна создавать экземпляр класса Queue, добавлять элементы в очередь, удалять элементы из очереди и выводить информацию о состоянии очереди на экран.

Инструкции:
\begin{enumerate}
    \item Создайте класс Queue с методом \_\_init\_\_, который инициализирует пустую очередь (список \_items). Принимает параметр allow\_duplicates=True. Если False, то не добавляет элемент, если он уже есть в очереди.
    \item Создайте метод enqueue, который принимает элемент. Если allow\_duplicates=False и элемент уже есть в очереди — не добавляет и возвращает False. Иначе — добавляет в конец и возвращает True.
    \item Создайте метод dequeue, который удаляет и возвращает первый элемент. Если очередь пуста — возвращает None (не выбрасывает исключение).
    \item Создайте метод is\_empty, который возвращает True, если очередь пуста, и False в противном случае.
    \item Создайте экземпляр класса Queue с allow\_duplicates=False.
    \item Добавьте элементы: 10, 20, 10 (не добавится), 30, 20 (не добавится), 40.
    \item Выведите текущее состояние очереди.
    \item Вызовите dequeue трижды, выводя каждый раз удаленный элемент.
    \item Выведите обновленное состояние очереди.
\end{enumerate}

Пример использования:
\begin{lstlisting}[language=Python]
queue = Queue(allow_duplicates=False)
print(queue.enqueue(10))  # True
print(queue.enqueue(20))  # True
print(queue.enqueue(10))  # False
print(queue.enqueue(30))  # True
print(queue.enqueue(20))  # False
print(queue.enqueue(40))  # True

print("Current Queue:", queue._items)

for _ in range(3):
    dequeued_item = queue.dequeue()
    print("Dequeued item:", dequeued_item)

print("Updated Queue:", queue._items)
\end{lstlisting}

\item Написать программу на Python, которая создает класс Queue для представления структуры данных очереди с инкапсуляцией. Класс должен содержать методы enqueue, dequeue и is\_empty, которые реализуют операции добавления элементов в очередь, удаления элементов из очереди и проверки пустоты очереди соответственно. Программа также должна создавать экземпляр класса Queue, добавлять элементы в очередь, удалять элементы из очереди и выводить информацию о состоянии очереди на экран.

Инструкции:
\begin{enumerate}
    \item Создайте класс Queue с методом \_\_init\_\_, который инициализирует пустую очередь (список \_data). Принимает параметр auto\_reverse=False. Если True, то enqueue добавляет в начало, а dequeue удаляет с конца (поведение стека, но интерфейс очереди).
    \item Создайте метод enqueue, который добавляет элемент: если auto\_reverse=False — в конец, если True — в начало.
    \item Создайте метод dequeue, который удаляет и возвращает элемент: если auto\_reverse=False — из начала, если True — из конца. Если очередь пуста — выбрасывает IndexError("Пусто").
    \item Создайте метод is\_empty, который возвращает True, если очередь пуста, и False в противном случае.
    \item Создайте экземпляр класса Queue с auto\_reverse=True.
    \item Добавьте элементы: 1, 2, 3, 4, 5.
    \item Выведите текущее состояние очереди.
    \item Вызовите dequeue дважды, выводя каждый раз удаленный элемент.
    \item Выведите обновленное состояние очереди.
\end{enumerate}

Пример использования:
\begin{lstlisting}[language=Python]
queue = Queue(auto_reverse=True)
queue.enqueue(1)
queue.enqueue(2)
queue.enqueue(3)
queue.enqueue(4)
queue.enqueue(5)

print("Current Queue:", queue._data)  # [5,4,3,2,1]

dequeued_item = queue.dequeue()  # удаляет 1
print("Dequeued item:", dequeued_item)

dequeued_item = queue.dequeue()  # удаляет 2
print("Dequeued item:", dequeued_item)

print("Updated Queue:", queue._data)  # [5,4,3]
\end{lstlisting}

\item Написать программу на Python, которая создает класс Queue для представления структуры данных очереди с инкапсуляцией. Класс должен содержать методы enqueue, dequeue и is\_empty, которые реализуют операции добавления элементов в очередь, удаления элементов из очереди и проверки пустоты очереди соответственно. Программа также должна создавать экземпляр класса Queue, добавлять элементы в очередь, удалять элементы из очереди и выводить информацию о состоянии очереди на экран.

Инструкции:
\begin{enumerate}
    \item Создайте класс Queue с методом \_\_init\_\_, который инициализирует пустую очередь (список \_buffer). Принимает параметр dequeue\_all\_at\_once=False. Если True, то dequeue возвращает список всех элементов и очищает очередь.
    \item Создайте метод enqueue, который добавляет элемент в конец очереди.
    \item Создайте метод dequeue, который, если dequeue\_all\_at\_once=False, удаляет и возвращает первый элемент. Если True — возвращает список всех элементов и очищает очередь. Если очередь пуста — возвращает пустой список [].
    \item Создайте метод is\_empty, который возвращает True, если очередь пуста, и False в противном случае.
    \item Создайте экземпляр класса Queue с dequeue\_all\_at\_once=True.
    \item Добавьте элементы: 5, 15, 25, 35.
    \item Выведите текущее состояние очереди.
    \item Вызовите dequeue (вернет [5,15,25,35] и очистит очередь).
    \item Выведите результат dequeue и состояние очереди после вызова.
\end{enumerate}

Пример использования:
\begin{lstlisting}[language=Python]
queue = Queue(dequeue_all_at_once=True)
queue.enqueue(5)
queue.enqueue(15)
queue.enqueue(25)
queue.enqueue(35)

print("Current Queue:", queue._buffer)

dequeued_items = queue.dequeue()
print("Dequeued items:", dequeued_items)  # [5,15,25,35]
print("Updated Queue:", queue._buffer)    # []
\end{lstlisting}

\item Написать программу на Python, которая создает класс Queue для представления структуры данных очереди с инкапсуляцией. Класс должен содержать методы enqueue, dequeue и is\_empty, которые реализуют операции добавления элементов в очередь, удаления элементов из очереди и проверки пустоты очереди соответственно. Программа также должна создавать экземпляр класса Queue, добавлять элементы в очередь, удалять элементы из очереди и выводить информацию о состоянии очереди на экран.

Инструкции:
\begin{enumerate}
    \item Создайте класс Queue с методом \_\_init\_\_, который инициализирует пустую очередь (список \_store). Принимает параметр on\_enqueue\_callback=None — функция, вызываемая при каждом добавлении (с аргументом — добавленным элементом).
    \item Создайте метод enqueue, который добавляет элемент в конец и, если on\_enqueue\_callback не None, вызывает её с элементом.
    \item Создайте метод dequeue, который удаляет и возвращает первый элемент. Если очередь пуста — выбрасывает IndexError("Нельзя извлечь из пустой очереди").
    \item Создайте метод is\_empty, который возвращает True, если очередь пуста, и False в противном случае.
    \item Создайте функцию printer(x): print(f"[+] Добавлен: {x}")
    \item Создайте экземпляр класса Queue, передав printer в on\_enqueue\_callback.
    \item Добавьте элементы: 101, 202, 303.
    \item Выведите текущее состояние очереди.
    \item Вызовите dequeue, выведите удаленный элемент.
    \item Выведите обновленное состояние очереди.
\end{enumerate}

Пример использования:
\begin{lstlisting}[language=Python]
def printer(x):
    print(f"[+] Добавлен: {x}")

queue = Queue(on_enqueue_callback=printer)
queue.enqueue(101)  # [+] Добавлен: 101
queue.enqueue(202)  # [+] Добавлен: 202
queue.enqueue(303)  # [+] Добавлен: 303

print("Current Queue:", queue._store)

dequeued_item = queue.dequeue()
print("Dequeued item:", dequeued_item)

print("Updated Queue:", queue._store)
\end{lstlisting}

\item Написать программу на Python, которая создает класс Queue для представления структуры данных очереди с инкапсуляцией. Класс должен содержать методы enqueue, dequeue и is\_empty, которые реализуют операции добавления элементов в очередь, удаления элементов из очереди и проверки пустоты очереди соответственно. Программа также должна создавать экземпляр класса Queue, добавлять элементы в очередь, удалять элементы из очереди и выводить информацию о состоянии очереди на экран.

Инструкции:
\begin{enumerate}
    \item Создайте класс Queue с методом \_\_init\_\_, который инициализирует пустую очередь (список \_pool). Принимает параметр compress\_on\_enqueue=False. Если True, то при добавлении элемента, равного последнему в очереди, вместо добавления увеличивает счетчик дубликатов у последнего элемента (хранит пары (элемент, счетчик)).
    \item Создайте метод enqueue, который, если compress\_on\_enqueue=True и очередь не пуста и элемент == последний\_элемент, увеличивает счетчик последнего элемента. Иначе — добавляет новый элемент (со счетчиком 1, если режим сжатия включен).
    \item Создайте метод dequeue, который удаляет первый элемент. Если режим сжатия включен и счетчик >1, уменьшает счетчик и возвращает элемент. Если счетчик=1, удаляет элемент. Если очередь пуста — выбрасывает IndexError("Очередь пуста").
    \item Создайте метод is\_empty, который возвращает True, если очередь пуста, и False в противном случае.
    \item Создайте экземпляр класса Queue с compress\_on\_enqueue=True.
    \item Добавьте элементы: 7, 7, 7, 14, 14, 21.
    \item Выведите текущее состояние очереди (внутреннее представление).
    \item Вызовите dequeue трижды, выводя каждый раз удаленный элемент.
    \item Выведите обновленное состояние очереди.
\end{enumerate}

Пример использования:
\begin{lstlisting}[language=Python]
queue = Queue(compress_on_enqueue=True)
queue.enqueue(7)
queue.enqueue(7)
queue.enqueue(7)
queue.enqueue(14)
queue.enqueue(14)
queue.enqueue(21)

print("Current Queue:", queue._pool)  # [(7,3), (14,2), (21,1)]

for _ in range(3):
    dequeued_item = queue.dequeue()
    print("Dequeued item:", dequeued_item)  # 7, 7, 7

print("Updated Queue:", queue._pool)  # [(14,2), (21,1)]
\end{lstlisting}

\item Написать программу на Python, которая создает класс Queue для представления структуры данных очереди с инкапсуляцией. Класс должен содержать методы enqueue, dequeue и is\_empty, которые реализуют операции добавления элементов в очередь, удаления элементов из очереди и проверки пустоты очереди соответственно. Программа также должна создавать экземпляр класса Queue, добавлять элементы в очередь, удалять элементы из очереди и выводить информацию о состоянии очереди на экран.

Инструкции:
\begin{enumerate}
    \item Создайте класс Queue с методом \_\_init\_\_, который инициализирует пустую очередь (список \_line). Принимает параметр immutable\_dequeue=False. Если True, то dequeue возвращает первый элемент, но не удаляет его.
    \item Создайте метод enqueue, который добавляет элемент в конец очереди.
    \item Создайте метод dequeue, который, если immutable\_dequeue=False, удаляет и возвращает первый элемент. Если True — возвращает первый элемент, не удаляя его. Если очередь пуста — возвращает None.
    \item Создайте метод is\_empty, который возвращает True, если очередь пуста, и False в противном случае.
    \item Создайте экземпляр класса Queue с immutable\_dequeue=True.
    \item Добавьте элементы: 1, 3, 5.
    \item Выведите текущее состояние очереди.
    \item Вызовите dequeue дважды, выводя каждый раз результат (должен быть 1 оба раза).
    \item Выведите состояние очереди (не должно измениться).
\end{enumerate}

Пример использования:
\begin{lstlisting}[language=Python]
queue = Queue(immutable_dequeue=True)
queue.enqueue(1)
queue.enqueue(3)
queue.enqueue(5)

print("Current Queue:", queue._line)

print("Dequeued item:", queue.dequeue())  # 1
print("Dequeued item:", queue.dequeue())  # 1 (не удалилось)

print("Updated Queue:", queue._line)  # [1,3,5]
\end{lstlisting}

\item Написать программу на Python, которая создает класс Queue для представления структуры данных очереди с инкапсуляцией. Класс должен содержать методы enqueue, dequeue и is\_empty, которые реализуют операции добавления элементов в очередь, удаления элементов из очереди и проверки пустоты очереди соответственно. Программа также должна создавать экземпляр класса Queue, добавлять элементы в очередь, удалять элементы из очереди и выводить информацию о состоянии очереди на экран.

Инструкции:
\begin{enumerate}
    \item Создайте класс Queue с методом \_\_init\_\_, который инициализирует пустую очередь (список \_stream). Принимает параметр track\_history=False. Если True, сохраняет историю всех когда-либо добавленных элементов (даже удаленных) в отдельном списке \_history.
    \item Создайте метод enqueue, который добавляет элемент в конец \_stream и, если track\_history=True, добавляет его в \_history.
    \item Создайте метод dequeue, который удаляет и возвращает первый элемент из \_stream. Если очередь пуста — выбрасывает IndexError("Пусто").
    \item Создайте метод is\_empty, который возвращает True, если \_stream пуст, и False в противном случае.
    \item Создайте метод get\_history (только если track\_history=True), возвращающий копию \_history.
    \item Создайте экземпляр класса Queue с track\_history=True.
    \item Добавьте элементы: 2, 4, 6.
    \item Вызовите dequeue (вернет 2).
    \item Добавьте 8.
    \item Выведите текущую очередь и историю.
\end{enumerate}

Пример использования:
\begin{lstlisting}[language=Python]
queue = Queue(track_history=True)
queue.enqueue(2)
queue.enqueue(4)
queue.enqueue(6)
queue.dequeue()  # 2
queue.enqueue(8)

print("Current Queue:", queue._stream)    # [4,6,8]
print("History:", queue.get_history())    # [2,4,6,8]
\end{lstlisting}

\item Написать программу на Python, которая создает класс Queue для представления структуры данных очереди с инкапсуляцией. Класс должен содержать методы enqueue, dequeue и is\_empty, которые реализуют операции добавления элементов в очередь, удаления элементов из очереди и проверки пустоты очереди соответственно. Программа также должна создавать экземпляр класса Queue, добавлять элементы в очередь, удалять элементы из очереди и выводить информацию о состоянии очереди на экран.

Инструкции:
\begin{enumerate}
    \item Создайте класс Queue с методом \_\_init\_\_, который инициализирует пустую очередь (список \_flow). Принимает параметр enqueue\_only\_even=False. Если True, то добавляются только четные числа.
    \item Создайте метод enqueue, который добавляет элемент в конец, только если enqueue\_only\_even=False или элемент четный.
    \item Создайте метод dequeue, который удаляет и возвращает первый элемент. Если очередь пуста — выбрасывает IndexError("Очередь пуста").
    \item Создайте метод is\_empty, который возвращает True, если очередь пуста, и False в противном случае.
    \item Создайте экземпляр класса Queue с enqueue\_only\_even=True.
    \item Добавьте элементы: 1 (игнорируется), 2, 3 (игнорируется), 4, 5 (игнорируется), 6.
    \item Выведите текущее состояние очереди.
    \item Вызовите dequeue, выведите удаленный элемент.
    \item Выведите обновленное состояние очереди.
\end{enumerate}

Пример использования:
\begin{lstlisting}[language=Python]
queue = Queue(enqueue_only_even=True)
queue.enqueue(1)  # игнорируется
queue.enqueue(2)
queue.enqueue(3)  # игнорируется
queue.enqueue(4)
queue.enqueue(5)  # игнорируется
queue.enqueue(6)

print("Current Queue:", queue._flow)  # [2,4,6]

dequeued_item = queue.dequeue()
print("Dequeued item:", dequeued_item)  # 2

print("Updated Queue:", queue._flow)  # [4,6]
\end{lstlisting}

\item Написать программу на Python, которая создает класс Queue для представления структуры данных очереди с инкапсуляцией. Класс должен содержать методы enqueue, dequeue и is\_empty, которые реализуют операции добавления элементов в очередь, удаления элементов из очереди и проверки пустоты очереди соответственно. Программа также должна создавать экземпляр класса Queue, добавлять элементы в очередь, удалять элементы из очереди и выводить информацию о состоянии очереди на экран.

Инструкции:
\begin{enumerate}
    \item Создайте класс Queue с методом \_\_init\_\_, который инициализирует пустую очередь (список \_pipe). Принимает параметр reverse\_dequeue=False. Если True, то dequeue удаляет и возвращает не первый, а последний элемент.
    \item Создайте метод enqueue, который добавляет элемент в конец очереди.
    \item Создайте метод dequeue, который, если reverse\_dequeue=False, удаляет и возвращает первый элемент. Если True — удаляет и возвращает последний элемент. Если очередь пуста — выбрасывает IndexError("Пусто").
    \item Создайте метод is\_empty, который возвращает True, если очередь пуста, и False в противном случае.
    \item Создайте экземпляр класса Queue с reverse\_dequeue=True.
    \item Добавьте элементы: 10, 20, 30.
    \item Выведите текущее состояние очереди.
    \item Вызовите dequeue — должен вернуться 30 (последний).
    \item Выведите обновленное состояние очереди.
\end{enumerate}

Пример использования:
\begin{lstlisting}[language=Python]
queue = Queue(reverse_dequeue=True)
queue.enqueue(10)
queue.enqueue(20)
queue.enqueue(30)

print("Current Queue:", queue._pipe)  # [10,20,30]

dequeued_item = queue.dequeue()  # 30
print("Dequeued item:", dequeued_item)

print("Updated Queue:", queue._pipe)  # [10,20]
\end{lstlisting}

\item Написать программу на Python, которая создает класс Queue для представления структуры данных очереди с инкапсуляцией. Класс должен содержать методы enqueue, dequeue и is\_empty, которые реализуют операции добавления элементов в очередь, удаления элементов из очереди и проверки пустоты очереди соответственно. Программа также должна создавать экземпляр класса Queue, добавлять элементы в очередь, удалять элементы из очереди и выводить информацию о состоянии очереди на экран.

Инструкции:
\begin{enumerate}
    \item Создайте класс Queue с методом \_\_init\_\_, который инициализирует пустую очередь (список \_channel). Принимает параметр enqueue\_with\_timestamp=False. Если True, то при добавлении сохраняет пару (элемент, time.time()).
    \item Создайте метод enqueue, который, если enqueue\_with\_timestamp=True, добавляет (элемент, timestamp). Иначе — элемент.
    \item Создайте метод dequeue, который удаляет и возвращает первый элемент (или пару). Если очередь пуста — выбрасывает IndexError("Очередь пуста").
    \item Создайте метод is\_empty, который возвращает True, если очередь пуста, и False в противном случае.
    \item Создайте экземпляр класса Queue с enqueue\_with\_timestamp=True.
    \item Добавьте элементы: "first", "second", "third".
    \item Выведите текущее состояние очереди.
    \item Вызовите dequeue, выведите результат (пару).
    \item Выведите обновленное состояние очереди.
\end{enumerate}

Пример использования:
\begin{lstlisting}[language=Python]
import time

queue = Queue(enqueue_with_timestamp=True)
queue.enqueue("first")
queue.enqueue("second")
queue.enqueue("third")

print("Current Queue:", queue._channel)

dequeued_item = queue.dequeue()
print("Dequeued item:", dequeued_item)  # ('first', timestamp)

print("Updated Queue:", queue._channel)
\end{lstlisting}

\item Написать программу на Python, которая создает класс Queue для представления структуры данных очереди с инкапсуляцией. Класс должен содержать методы enqueue, dequeue и is\_empty, которые реализуют операции добавления элементов в очередь, удаления элементов из очереди и проверки пустоты очереди соответственно. Программа также должна создавать экземпляр класса Queue, добавлять элементы в очередь, удалять элементы из очереди и выводить информацию о состоянии очереди на экран.

Инструкции:
\begin{enumerate}
    \item Создайте класс Queue с методом \_\_init\_\_, который инициализирует пустую очередь (список \_tube). Принимает параметр enqueue\_pairs=False. Если True, то enqueue принимает два аргумента (key, value) и сохраняет кортеж (key, value).
    \item Создайте метод enqueue, который, если enqueue\_pairs=False, принимает один элемент. Если True — два аргумента и сохраняет кортеж.
    \item Создайте метод dequeue, который удаляет и возвращает первый элемент (или кортеж). Если очередь пуста — выбрасывает IndexError("Пусто").
    \item Создайте метод is\_empty, который возвращает True, если очередь пуста, и False в противном случае.
    \item Создайте экземпляр класса Queue с enqueue\_pairs=True.
    \item Добавьте пары: ("a", 1), ("b", 2), ("c", 3).
    \item Выведите текущее состояние очереди.
    \item Вызовите dequeue, выведите результат.
    \item Выведите обновленное состояние очереди.
\end{enumerate}

Пример использования:
\begin{lstlisting}[language=Python]
queue = Queue(enqueue_pairs=True)
queue.enqueue("a", 1)
queue.enqueue("b", 2)
queue.enqueue("c", 3)

print("Current Queue:", queue._tube)

dequeued_item = queue.dequeue()
print("Dequeued item:", dequeued_item)  # ('a', 1)

print("Updated Queue:", queue._tube)
\end{lstlisting}

\item Написать программу на Python, которая создает класс Queue для представления структуры данных очереди с инкапсуляцией. Класс должен содержать методы enqueue, dequeue и is\_empty, которые реализуют операции добавления элементов в очередь, удаления элементов из очереди и проверки пустоты очереди соответственно. Программа также должна создавать экземпляр класса Queue, добавлять элементы в очередь, удалять элементы из очереди и выводить информацию о состоянии очереди на экран.

Инструкции:
\begin{enumerate}
    \item Создайте класс Queue с методом \_\_init\_\_, который инициализирует пустую очередь (список \_conduit). Принимает параметр auto\_dedup=False. Если True, то при добавлении, если элемент уже есть в очереди, сначала удаляет все его предыдущие вхождения.
    \item Создайте метод enqueue, который, если auto\_dedup=True и элемент уже есть, удаляет все его вхождения, затем добавляет в конец. Иначе — просто добавляет.
    \item Создайте метод dequeue, который удаляет и возвращает первый элемент. Если очередь пуста — выбрасывает IndexError("Очередь пуста").
    \item Создайте метод is\_empty, который возвращает True, если очередь пуста, и False в противном случае.
    \item Создайте экземпляр класса Queue с auto\_dedup=True.
    \item Добавьте элементы: 1, 2, 1, 3, 2, 4.
    \item Выведите текущее состояние очереди.
    \item Вызовите dequeue, выведите удаленный элемент.
    \item Выведите обновленное состояние очереди.
\end{enumerate}

Пример использования:
\begin{lstlisting}[language=Python]
queue = Queue(auto_dedup=True)
queue.enqueue(1)  # [1]
queue.enqueue(2)  # [1,2]
queue.enqueue(1)  # удаляет старую 1 -> [2,1]
queue.enqueue(3)  # [2,1,3]
queue.enqueue(2)  # удаляет 2 -> [1,3,2]
queue.enqueue(4)  # [1,3,2,4]

print("Current Queue:", queue._conduit)

dequeued_item = queue.dequeue()
print("Dequeued item:", dequeued_item)  # 1

print("Updated Queue:", queue._conduit)  # [3,2,4]
\end{lstlisting}

\item Написать программу на Python, которая создает класс Queue для представления структуры данных очереди с инкапсуляцией. Класс должен содержать методы enqueue, dequeue и is\_empty, которые реализуют операции добавления элементов в очередь, удаления элементов из очереди и проверки пустоты очереди соответственно. Программа также должна создавать экземпляр класса Queue, добавлять элементы в очередь, удалять элементы из очереди и выводить информацию о состоянии очереди на экран.

Инструкции:
\begin{enumerate}
    \item Создайте класс Queue с методом \_\_init\_\_, который инициализирует пустую очередь (список \_duct). Принимает параметр enqueue\_if\_max=False. Если True, то элемент добавляется только если он больше всех текущих элементов в очереди.
    \item Создайте метод enqueue, который добавляет элемент, только если enqueue\_if\_max=False или элемент > всех элементов в очереди.
    \item Создайте метод dequeue, который удаляет и возвращает первый элемент. Если очередь пуста — выбрасывает IndexError("Пусто").
    \item Создайте метод is\_empty, который возвращает True, если очередь пуста, и False в противном случае.
    \item Создайте экземпляр класса Queue с enqueue\_if\_max=True.
    \item Добавьте элементы: 5, 3 (не добавится), 10, 7 (не добавится), 15.
    \item Выведите текущее состояние очереди.
    \item Вызовите dequeue, выведите удаленный элемент.
    \item Выведите обновленное состояние очереди.
\end{enumerate}

Пример использования:
\begin{lstlisting}[language=Python]
queue = Queue(enqueue_if_max=True)
queue.enqueue(5)
queue.enqueue(3)   # не добавится
queue.enqueue(10)
queue.enqueue(7)   # не добавится
queue.enqueue(15)

print("Current Queue:", queue._duct)  # [5,10,15]

dequeued_item = queue.dequeue()
print("Dequeued item:", dequeued_item)  # 5

print("Updated Queue:", queue._duct)  # [10,15]
\end{lstlisting}

\item Написать программу на Python, которая создает класс Queue для представления структуры данных очереди с инкапсуляцией. Класс должен содержать методы enqueue, dequeue и is\_empty, которые реализуют операции добавления элементов в очередь, удаления элементов из очереди и проверки пустоты очереди соответственно. Программа также должна создавать экземпляр класса Queue, добавлять элементы в очередь, удалять элементы из очереди и выводить информацию о состоянии очереди на экран.

Инструкции:
\begin{enumerate}
    \item Создайте класс Queue с методом \_\_init\_\_, который инициализирует пустую очередь (список \_pipe). Принимает параметр cumulative=False. Если True, то при добавлении элемент становится element + последний\_элемент (если очередь не пуста). Первый элемент добавляется как есть.
    \item Создайте метод enqueue, который, если cumulative=True и очередь не пуста, добавляет element + последний\_элемент. Иначе — element.
    \item Создайте метод dequeue, который удаляет и возвращает первый элемент. Если очередь пуста — выбрасывает IndexError("Очередь пуста").
    \item Создайте метод is\_empty, который возвращает True, если очередь пуста, и False в противном случае.
    \item Создайте экземпляр класса Queue с cumulative=True.
    \item Добавьте элементы: 1, 2, 3, 4.
    \item Выведите текущее состояние очереди.
    \item Вызовите dequeue, выведите удаленный элемент.
    \item Выведите обновленное состояние очереди.
\end{enumerate}

Пример использования:
\begin{lstlisting}[language=Python]
queue = Queue(cumulative=True)
queue.enqueue(1)  # [1]
queue.enqueue(2)  # [1, 1+2=3]
queue.enqueue(3)  # [1,3, 3+3=6]
queue.enqueue(4)  # [1,3,6, 6+4=10]

print("Current Queue:", queue._pipe)

dequeued_item = queue.dequeue()
print("Dequeued item:", dequeued_item)  # 1

print("Updated Queue:", queue._pipe)  # [3,6,10]
\end{lstlisting}

\item Написать программу на Python, которая создает класс Queue для представления структуры данных очереди с инкапсуляцией. Класс должен содержать методы enqueue, dequeue и is\_empty, которые реализуют операции добавления элементов в очередь, удаления элементов из очереди и проверки пустоты очереди соответственно. Программа также должна создавать экземпляр класса Queue, добавлять элементы в очередь, удалять элементы из очереди и выводить информацию о состоянии очереди на экран.

Инструкции:
\begin{enumerate}
    \item Создайте класс Queue с методом \_\_init\_\_, который инициализирует пустую очередь (список \_line). Принимает параметр enqueue\_squared=False. Если True, то при добавлении сохраняется element**2.
    \item Создайте метод enqueue, который добавляет element**2, если enqueue\_squared=True, иначе — element.
    \item Создайте метод dequeue, который удаляет и возвращает первый элемент. Если очередь пуста — выбрасывает IndexError("Пусто").
    \item Создайте метод is\_empty, который возвращает True, если очередь пуста, и False в противном случае.
    \item Создайте экземпляр класса Queue с enqueue\_squared=True.
    \item Добавьте элементы: 2, 3, 4, 5.
    \item Выведите текущее состояние очереди.
    \item Вызовите dequeue, выведите удаленный элемент.
    \item Выведите обновленное состояние очереди.
\end{enumerate}

Пример использования:
\begin{lstlisting}[language=Python]
queue = Queue(enqueue_squared=True)
queue.enqueue(2)  # 4
queue.enqueue(3)  # 9
queue.enqueue(4)  # 16
queue.enqueue(5)  # 25

print("Current Queue:", queue._line)

dequeued_item = queue.dequeue()
print("Dequeued item:", dequeued_item)  # 4

print("Updated Queue:", queue._line)  # [9,16,25]
\end{lstlisting}

\item Написать программу на Python, которая создает класс Queue для представления структуры данных очереди с инкапсуляцией. Класс должен содержать методы enqueue, dequeue и is\_empty, которые реализуют операции добавления элементов в очередь, удаления элементов из очереди и проверки пустоты очереди соответственно. Программа также должна создавать экземпляр класса Queue, добавлять элементы в очередь, удалять элементы из очереди и выводить информацию о состоянии очереди на экран.

Инструкции:
\begin{enumerate}
    \item Создайте класс Queue с методом \_\_init\_\_, который инициализирует пустую очередь (список \_stream). Принимает параметр enqueue\_absolute=False. Если True, то при добавлении сохраняется abs(element).
    \item Создайте метод enqueue, который добавляет abs(element), если enqueue\_absolute=True, иначе — element.
    \item Создайте метод dequeue, который удаляет и возвращает первый элемент. Если очередь пуста — выбрасывает IndexError("Очередь пуста").
    \item Создайте метод is\_empty, который возвращает True, если очередь пуста, и False в противном случае.
    \item Создайте экземпляр класса Queue с enqueue\_absolute=True.
    \item Добавьте элементы: -5, 3, -8, 2.
    \item Выведите текущее состояние очереди.
    \item Вызовите dequeue, выведите удаленный элемент.
    \item Выведите обновленное состояние очереди.
\end{enumerate}

Пример использования:
\begin{lstlisting}[language=Python]
queue = Queue(enqueue_absolute=True)
queue.enqueue(-5)  # 5
queue.enqueue(3)   # 3
queue.enqueue(-8)  # 8
queue.enqueue(2)   # 2

print("Current Queue:", queue._stream)

dequeued_item = queue.dequeue()
print("Dequeued item:", dequeued_item)  # 5

print("Updated Queue:", queue._stream)  # [3,8,2]
\end{lstlisting}

\item Написать программу на Python, которая создает класс Queue для представления структуры данных очереди с инкапсуляцией. Класс должен содержать методы enqueue, dequeue и is\_empty, которые реализуют операции добавления элементов в очередь, удаления элементов из очереди и проверки пустоты очереди соответственно. Программа также должна создавать экземпляр класса Queue, добавлять элементы в очередь, удалять элементы из очереди и выводить информацию о состоянии очереди на экран.

Инструкции:
\begin{enumerate}
    \item Создайте класс Queue с методом \_\_init\_\_, который инициализирует пустую очередь (список \_buffer). Принимает параметр enqueue\_rounded=False. Если True, то при добавлении сохраняется round(element).
    \item Создайте метод enqueue, который добавляет round(element), если enqueue\_rounded=True, иначе — element.
    \item Создайте метод dequeue, который удаляет и возвращает первый элемент. Если очередь пуста — выбрасывает IndexError("Пусто").
    \item Создайте метод is\_empty, который возвращает True, если очередь пуста, и False в противном случае.
    \item Создайте экземпляр класса Queue с enqueue\_rounded=True.
    \item Добавьте элементы: 3.2, 4.7, 5.1, 6.9.
    \item Выведите текущее состояние очереди.
    \item Вызовите dequeue, выведите удаленный элемент.
    \item Выведите обновленное состояние очереди.
\end{enumerate}

Пример использования:
\begin{lstlisting}[language=Python]
queue = Queue(enqueue_rounded=True)
queue.enqueue(3.2)  # 3
queue.enqueue(4.7)  # 5
queue.enqueue(5.1)  # 5
queue.enqueue(6.9)  # 7

print("Current Queue:", queue._buffer)

dequeued_item = queue.dequeue()
print("Dequeued item:", dequeued_item)  # 3

print("Updated Queue:", queue._buffer)  # [5,5,7]
\end{lstlisting}

\item Написать программу на Python, которая создает класс Queue для представления структуры данных очереди с инкапсуляцией. Класс должен содержать методы enqueue, dequeue и is\_empty, которые реализуют операции добавления элементов в очередь, удаления элементов из очереди и проверки пустоты очереди соответственно. Программа также должна создавать экземпляр класса Queue, добавлять элементы в очередь, удалять элементы из очереди и выводить информацию о состоянии очереди на экран.

Инструкции:
\begin{enumerate}
    \item Создайте класс Queue с методом \_\_init\_\_, который инициализирует пустую очередь (список \_store). Принимает параметр enqueue\_negated=False. Если True, то при добавлении сохраняется -element.
    \item Создайте метод enqueue, который добавляет -element, если enqueue\_negated=True, иначе — element.
    \item Создайте метод dequeue, который удаляет и возвращает первый элемент. Если очередь пуста — выбрасывает IndexError("Очередь пуста").
    \item Создайте метод is\_empty, который возвращает True, если очередь пуста, и False в противном случае.
    \item Создайте экземпляр класса Queue с enqueue\_negated=True.
    \item Добавьте элементы: 10, 20, 30, 40.
    \item Выведите текущее состояние очереди.
    \item Вызовите dequeue, выведите удаленный элемент.
    \item Выведите обновленное состояние очереди.
\end{enumerate}

Пример использования:
\begin{lstlisting}[language=Python]
queue = Queue(enqueue_negated=True)
queue.enqueue(10)  # -10
queue.enqueue(20)  # -20
queue.enqueue(30)  # -30
queue.enqueue(40)  # -40

print("Current Queue:", queue._store)

dequeued_item = queue.dequeue()
print("Dequeued item:", dequeued_item)  # -10

print("Updated Queue:", queue._store)  # [-20,-30,-40]
\end{lstlisting}

\item Написать программу на Python, которая создает класс Queue для представления структуры данных очереди с инкапсуляцией. Класс должен содержать методы enqueue, dequeue и is\_empty, которые реализуют операции добавления элементов в очередь, удаления элементов из очереди и проверки пустоты очереди соответственно. Программа также должна создавать экземпляр класса Queue, добавлять элементы в очередь, удалять элементы из очереди и выводить информацию о состоянии очереди на экран.

Инструкции:
\begin{enumerate}
    \item Создайте класс Queue с методом \_\_init\_\_, который инициализирует пустую очередь (список \_pool). Принимает параметр enqueue\_doubled=False. Если True, то при добавлении сохраняется element * 2.
    \item Создайте метод enqueue, который добавляет element * 2, если enqueue\_doubled=True, иначе — element.
    \item Создайте метод dequeue, который удаляет и возвращает первый элемент. Если очередь пуста — выбрасывает IndexError("Пусто").
    \item Создайте метод is\_empty, который возвращает True, если очередь пуста, и False в противном случае.
    \item Создайте экземпляр класса Queue с enqueue\_doubled=True.
    \item Добавьте элементы: 1, 2, 3, 4.
    \item Выведите текущее состояние очереди.
    \item Вызовите dequeue, выведите удаленный элемент.
    \item Выведите обновленное состояние очереди.
\end{enumerate}

Пример использования:
\begin{lstlisting}[language=Python]
queue = Queue(enqueue_doubled=True)
queue.enqueue(1)  # 2
queue.enqueue(2)  # 4
queue.enqueue(3)  # 6
queue.enqueue(4)  # 8

print("Current Queue:", queue._pool)

dequeued_item = queue.dequeue()
print("Dequeued item:", dequeued_item)  # 2

print("Updated Queue:", queue._pool)  # [4,6,8]
\end{lstlisting}

\item Написать программу на Python, которая создает класс Queue для представления структуры данных очереди с инкапсуляцией. Класс должен содержать методы enqueue, dequeue и is\_empty, которые реализуют операции добавления элементов в очередь, удаления элементов из очереди и проверки пустоты очереди соответственно. Программа также должна создавать экземпляр класса Queue, добавлять элементы в очередь, удалять элементы из очереди и выводить информацию о состоянии очереди на экран.

Инструкции:
\begin{enumerate}
    \item Создайте класс Queue с методом \_\_init\_\_, который инициализирует пустую очередь (список \_reservoir). Принимает параметр enqueue\_halved=False. Если True, то при добавлении сохраняется element / 2.0.
    \item Создайте метод enqueue, который добавляет element / 2.0, если enqueue\_halved=True, иначе — element.
    \item Создайте метод dequeue, который удаляет и возвращает первый элемент. Если очередь пуста — выбрасывает IndexError("Очередь пуста").
    \item Создайте метод is\_empty, который возвращает True, если очередь пуста, и False в противном случае.
    \item Создайте экземпляр класса Queue с enqueue\_halved=True.
    \item Добавьте элементы: 4, 8, 12, 16.
    \item Выведите текущее состояние очереди.
    \item Вызовите dequeue, выведите удаленный элемент.
    \item Выведите обновленное состояние очереди.
\end{enumerate}

Пример использования:
\begin{lstlisting}[language=Python]
queue = Queue(enqueue_halved=True)
queue.enqueue(4)   # 2.0
queue.enqueue(8)   # 4.0
queue.enqueue(12)  # 6.0
queue.enqueue(16)  # 8.0

print("Current Queue:", queue._reservoir)

dequeued_item = queue.dequeue()
print("Dequeued item:", dequeued_item)  # 2.0

print("Updated Queue:", queue._reservoir)  # [4.0,6.0,8.0]
\end{lstlisting}

\item Написать программу на Python, которая создает класс Queue для представления структуры данных очереди с инкапсуляцией. Класс должен содержать методы enqueue, dequeue и is\_empty, которые реализуют операции добавления элементов в очередь, удаления элементов из очереди и проверки пустоты очереди соответственно. Программа также должна создавать экземпляр класса Queue, добавлять элементы в очередь, удалять элементы из очереди и выводить информацию о состоянии очереди на экран.

Инструкции:
\begin{enumerate}
    \item Создайте класс Queue с методом \_\_init\_\_, который инициализирует пустую очередь (список \_tank). Принимает параметр enqueue\_as\_string=False. Если True, то при добавлении сохраняется str(element).
    \item Создайте метод enqueue, который добавляет str(element), если enqueue\_as\_string=True, иначе — element.
    \item Создайте метод dequeue, который удаляет и возвращает первый элемент. Если очередь пуста — выбрасывает IndexError("Пусто").
    \item Создайте метод is\_empty, который возвращает True, если очередь пуста, и False в противном случае.
    \item Создайте экземпляр класса Queue с enqueue\_as\_string=True.
    \item Добавьте элементы: 100, 200, 300, 400.
    \item Выведите текущее состояние очереди.
    \item Вызовите dequeue, выведите удаленный элемент.
    \item Выведите обновленное состояние очереди.
\end{enumerate}

Пример использования:
\begin{lstlisting}[language=Python]
queue = Queue(enqueue_as_string=True)
queue.enqueue(100)  # "100"
queue.enqueue(200)  # "200"
queue.enqueue(300)  # "300"
queue.enqueue(400)  # "400"

print("Current Queue:", queue._tank)

dequeued_item = queue.dequeue()
print("Dequeued item:", dequeued_item)  # "100"

print("Updated Queue:", queue._tank)  # ["200","300","400"]
\end{lstlisting}

\item Написать программу на Python, которая создает класс Queue для представления структуры данных очереди с инкапсуляцией. Класс должен содержать методы enqueue, dequeue и is\_empty, которые реализуют операции добавления элементов в очередь, удаления элементов из очереди и проверки пустоты очереди соответственно. Программа также должна создавать экземпляр класса Queue, добавлять элементы в очередь, удалять элементы из очереди и выводить информацию о состоянии очереди на экран.

Инструкции:
\begin{enumerate}
    \item Создайте класс Queue с методом \_\_init\_\_, который инициализирует пустую очередь (список \_container). Принимает параметр enqueue\_with\_index=False. Если True, то при добавлении сохраняется кортеж (element, порядковый\_номер\_добавления).
    \item Создайте метод enqueue, который добавляет (element, self.\_counter), где \_counter — внутренний счетчик, увеличивающийся при каждом добавлении. Иначе — element.
    \item Создайте метод dequeue, который удаляет и возвращает первый элемент (или кортеж). Если очередь пуста — выбрасывает IndexError("Очередь пуста").
    \item Создайте метод is\_empty, который возвращает True, если очередь пуста, и False в противном случае.
    \item Создайте экземпляр класса Queue с enqueue\_with\_index=True.
    \item Добавьте элементы: "alpha", "beta", "gamma".
    \item Выведите текущее состояние очереди.
    \item Вызовите dequeue, выведите удаленный элемент.
    \item Выведите обновленное состояние очереди.
\end{enumerate}

Пример использования:
\begin{lstlisting}[language=Python]
queue = Queue(enqueue_with_index=True)
queue.enqueue("alpha")  # ("alpha", 0)
queue.enqueue("beta")   # ("beta", 1)
queue.enqueue("gamma")  # ("gamma", 2)

print("Current Queue:", queue._container)

dequeued_item = queue.dequeue()
print("Dequeued item:", dequeued_item)  # ('alpha', 0)

print("Updated Queue:", queue._container)  # [('beta',1), ('gamma',2)]
\end{lstlisting}

\item Написать программу на Python, которая создает класс Queue для представления структуры данных очереди с инкапсуляцией. Класс должен содержать методы enqueue, dequeue и is\_empty, которые реализуют операции добавления элементов в очередь, удаления элементов из очереди и проверки пустоты очереди соответственно. Программа также должна создавать экземпляр класса Queue, добавлять элементы в очередь, удалять элементы из очереди и выводить информацию о состоянии очереди на экран.

Инструкции:
\begin{enumerate}
    \item Создайте класс Queue с методом \_\_init\_\_, который инициализирует пустую очередь (список \_vessel). Принимает параметр enqueue\_unique\_rear=False. Если True, то при добавлении, если элемент равен текущему последнему, он не добавляется.
    \item Создайте метод enqueue, который добавляет элемент, только если enqueue\_unique\_rear=False или очередь пуста или element != последний\_элемент.
    \item Создайте метод dequeue, который удаляет и возвращает первый элемент. Если очередь пуста — выбрасывает IndexError("Пусто").
    \item Создайте метод is\_empty, который возвращает True, если очередь пуста, и False в противном случае.
    \item Создайте экземпляр класса Queue с enqueue\_unique\_rear=True.
    \item Добавьте элементы: 1, 2, 2, 3, 3, 3, 4.
    \item Выведите текущее состояние очереди.
    \item Вызовите dequeue, выведите удаленный элемент.
    \item Выведите обновленное состояние очереди.
\end{enumerate}

Пример использования:
\begin{lstlisting}[language=Python]
queue = Queue(enqueue_unique_rear=True)
queue.enqueue(1)
queue.enqueue(2)
queue.enqueue(2)  # не добавится
queue.enqueue(3)
queue.enqueue(3)  # не добавится
queue.enqueue(3)  # не добавится
queue.enqueue(4)

print("Current Queue:", queue._vessel)  # [1,2,3,4]

dequeued_item = queue.dequeue()
print("Dequeued item:", dequeued_item)  # 1

print("Updated Queue:", queue._vessel)  # [2,3,4]
\end{lstlisting}

\item Написать программу на Python, которая создает класс Queue для представления структуры данных очереди с инкапсуляцией. Класс должен содержать методы enqueue, dequeue и is\_empty, которые реализуют операции добавления элементов в очередь, удаления элементов из очереди и проверки пустоты очереди соответственно. Программа также должна создавать экземпляр класса Queue, добавлять элементы в очередь, удалять элементы из очереди и выводить информацию о состоянии очереди на экран.

Инструкции:
\begin{enumerate}
    \item Создайте класс Queue с методом \_\_init\_\_, который инициализирует пустую очередь (список \_bin). Принимает параметр enqueue\_even\_only=False. Если True, то добавляются только четные числа.
    \item Создайте метод enqueue, который добавляет элемент, только если enqueue\_even\_only=False или element \% 2 == 0.
    \item Создайте метод dequeue, который удаляет и возвращает первый элемент. Если очередь пуста — выбрасывает IndexError("Очередь пуста").
    \item Создайте метод is\_empty, который возвращает True, если очередь пуста, и False в противном случае.
    \item Создайте экземпляр класса Queue с enqueue\_even\_only=True.
    \item Добавьте элементы: 1 (не добавится), 2, 3 (не добавится), 4, 5 (не добавится), 6.
    \item Выведите текущее состояние очереди.
    \item Вызовите dequeue, выведите удаленный элемент.
    \item Выведите обновленное состояние очереди.
\end{enumerate}

Пример использования:
\begin{lstlisting}[language=Python]
queue = Queue(enqueue_even_only=True)
queue.enqueue(1)  # нет
queue.enqueue(2)
queue.enqueue(3)  # нет
queue.enqueue(4)
queue.enqueue(5)  # нет
queue.enqueue(6)

print("Current Queue:", queue._bin)  # [2,4,6]

dequeued_item = queue.dequeue()
print("Dequeued item:", dequeued_item)  # 2

print("Updated Queue:", queue._bin)  # [4,6]
\end{lstlisting}

\item Написать программу на Python, которая создает класс Queue для представления структуры данных очереди с инкапсуляцией. Класс должен содержать методы enqueue, dequeue и is\_empty, которые реализуют операции добавления элементов в очередь, удаления элементов из очереди и проверки пустоты очереди соответственно. Программа также должна создавать экземпляр класса Queue, добавлять элементы в очередь, удалять элементы из очереди и выводить информацию о состоянии очереди на экран.

Инструкции:
\begin{enumerate}
    \item Создайте класс Queue с методом \_\_init\_\_, который инициализирует пустую очередь (список \_box). Принимает параметр enqueue\_odd\_only=False. Если True, то добавляются только нечетные числа.
    \item Создайте метод enqueue, который добавляет элемент, только если enqueue\_odd\_only=False или element \% 2 != 0.
    \item Создайте метод dequeue, который удаляет и возвращает первый элемент. Если очередь пуста — выбрасывает IndexError("Пусто").
    \item Создайте метод is\_empty, который возвращает True, если очередь пуста, и False в противном случае.
    \item Создайте экземпляр класса Queue с enqueue\_odd\_only=True.
    \item Добавьте элементы: 2 (не добавится), 1, 4 (не добавится), 3, 6 (не добавится), 5.
    \item Выведите текущее состояние очереди.
    \item Вызовите dequeue, выведите удаленный элемент.
    \item Выведите обновленное состояние очереди.
\end{enumerate}

Пример использования:
\begin{lstlisting}[language=Python]
queue = Queue(enqueue_odd_only=True)
queue.enqueue(2)  # нет
queue.enqueue(1)
queue.enqueue(4)  # нет
queue.enqueue(3)
queue.enqueue(6)  # нет
queue.enqueue(5)

print("Current Queue:", queue._box)  # [1,3,5]

dequeued_item = queue.dequeue()
print("Dequeued item:", dequeued_item)  # 1

print("Updated Queue:", queue._box)  # [3,5]
\end{lstlisting}

\item Написать программу на Python, которая создает класс Queue для представления структуры данных очереди с инкапсуляцией. Класс должен содержать методы enqueue, dequeue и is\_empty, которые реализуют операции добавления элементов в очередь, удаления элементов из очереди и проверки пустоты очереди соответственно. Программа также должна создавать экземпляр класса Queue, добавлять элементы в очередь, удалять элементы из очереди и выводить информацию о состоянии очереди на экран.

Инструкции:
\begin{enumerate}
    \item Создайте класс Queue с методом \_\_init\_\_, который инициализирует пустую очередь (список \_crate). Принимает параметр enqueue\_positive\_only=False. Если True, то добавляются только положительные числа (>0).
    \item Создайте метод enqueue, который добавляет элемент, только если enqueue\_positive\_only=False или element > 0.
    \item Создайте метод dequeue, который удаляет и возвращает первый элемент. Если очередь пуста — выбрасывает IndexError("Очередь пуста").
    \item Создайте метод is\_empty, который возвращает True, если очередь пуста, и False в противном случае.
    \item Создайте экземпляр класса Queue с enqueue\_positive\_only=True.
    \item Добавьте элементы: -1 (не добавится), 0 (не добавится), 1, 2, -5 (не добавится), 3.
    \item Выведите текущее состояние очереди.
    \item Вызовите dequeue, выведите удаленный элемент.
    \item Выведите обновленное состояние очереди.
\end{enumerate}

Пример использования:
\begin{lstlisting}[language=Python]
queue = Queue(enqueue_positive_only=True)
queue.enqueue(-1)  # нет
queue.enqueue(0)   # нет
queue.enqueue(1)
queue.enqueue(2)
queue.enqueue(-5)  # нет
queue.enqueue(3)

print("Current Queue:", queue._crate)  # [1,2,3]

dequeued_item = queue.dequeue()
print("Dequeued item:", dequeued_item)  # 1

print("Updated Queue:", queue._crate)  # [2,3]
\end{lstlisting}

\item Написать программу на Python, которая создает класс Queue для представления структуры данных очереди с инкапсуляцией. Класс должен содержать методы enqueue, dequeue и is\_empty, которые реализуют операции добавления элементов в очередь, удаления элементов из очереди и проверки пустоты очереди соответственно. Программа также должна создавать экземпляр класса Queue, добавлять элементы в очередь, удалять элементы из очереди и выводить информацию о состоянии очереди на экран.

Инструкции:
\begin{enumerate}
    \item Создайте класс Queue с методом \_\_init\_\_, который инициализирует пустую очередь (список \_carton). Принимает параметр enqueue\_nonzero\_only=False. Если True, то добавляются только ненулевые числа.
    \item Создайте метод enqueue, который добавляет элемент, только если enqueue\_nonzero\_only=False или element != 0.
    \item Создайте метод dequeue, который удаляет и возвращает первый элемент. Если очередь пуста — выбрасывает IndexError("Пусто").
    \item Создайте метод is\_empty, который возвращает True, если очередь пуста, и False в противном случае.
    \item Создайте экземпляр класса Queue с enqueue\_nonzero\_only=True.
    \item Добавьте элементы: 0 (не добавится), 5, 0 (не добавится), 10, 15.
    \item Выведите текущее состояние очереди.
    \item Вызовите dequeue, выведите удаленный элемент.
    \item Выведите обновленное состояние очереди.
\end{enumerate}

Пример использования:
\begin{lstlisting}[language=Python]
queue = Queue(enqueue_nonzero_only=True)
queue.enqueue(0)   # нет
queue.enqueue(5)
queue.enqueue(0)   # нет
queue.enqueue(10)
queue.enqueue(15)

print("Current Queue:", queue._carton)  # [5,10,15]

dequeued_item = queue.dequeue()
print("Dequeued item:", dequeued_item)  # 5

print("Updated Queue:", queue._carton)  # [10,15]
\end{lstlisting}

\item Написать программу на Python, которая создает класс Queue для представления структуры данных очереди с инкапсуляцией. Класс должен содержать методы enqueue, dequeue и is\_empty, которые реализуют операции добавления элементов в очередь, удаления элементов из очереди и проверки пустоты очереди соответственно. Программа также должна создавать экземпляр класса Queue, добавлять элементы в очередь, удалять элементы из очереди и выводить информацию о состоянии очереди на экран.

Инструкции:
\begin{enumerate}
    \item Создайте класс Queue с методом \_\_init\_\_, который инициализирует пустую очередь (список \_package). Принимает параметр enqueue\_prime\_only=False. Если True, то добавляются только простые числа (реализуйте простую проверку).
    \item Создайте метод enqueue, который добавляет элемент, только если enqueue\_prime\_only=False или element — простое число.
    \item Создайте метод dequeue, который удаляет и возвращает первый элемент. Если очередь пуста — выбрасывает IndexError("Очередь пуста").
    \item Создайте метод is\_empty, который возвращает True, если очередь пуста, и False в противном случае.
    \item Создайте вспомогательную функцию is\_prime(n) (вне класса).
    \item Создайте экземпляр класса Queue с enqueue\_prime\_only=True.
    \item Добавьте элементы: 4 (не простое), 5 (простое), 6 (не простое), 7 (простое), 8 (не простое), 11 (простое).
    \item Выведите текущее состояние очереди.
    \item Вызовите dequeue, выведите удаленный элемент.
    \item Выведите обновленное состояние очереди.
\end{enumerate}

Пример использования:
\begin{lstlisting}[language=Python]
def is_prime(n):
    if n < 2:
        return False
    for i in range(2, int(n**0.5)+1):
        if n % i == 0:
            return False
    return True

queue = Queue(enqueue_prime_only=True)
queue.enqueue(4)   # нет
queue.enqueue(5)   # да
queue.enqueue(6)   # нет
queue.enqueue(7)   # да
queue.enqueue(8)   # нет
queue.enqueue(11)  # да

print("Current Queue:", queue._package)  # [5,7,11]

dequeued_item = queue.dequeue()
print("Dequeued item:", dequeued_item)  # 5

print("Updated Queue:", queue._package)  # [7,11]
\end{lstlisting}

\item Написать программу на Python, которая создает класс Queue для представления структуры данных очереди с инкапсуляцией. Класс должен содержать методы enqueue, dequeue и is\_empty, которые реализуют операции добавления элементов в очередь, удаления элементов из очереди и проверки пустоты очереди соответственно. Программа также должна создавать экземпляр класса Queue, добавлять элементы в очередь, удалять элементы из очереди и выводить информацию о состоянии очереди на экран.

Инструкции:
\begin{enumerate}
    \item Создайте класс Queue с методом \_\_init\_\_, который инициализирует пустую очередь (список \_parcel). Принимает параметр enqueue\_fibonacci\_only=False. Если True, то добавляются только числа Фибоначчи (до 100: 0,1,1,2,3,5,8,13,21,34,55,89).
    \item Создайте метод enqueue, который добавляет элемент, только если enqueue\_fibonacci\_only=False или element входит в FIB\_SET.
    \item Создайте метод dequeue, который удаляет и возвращает первый элемент. Если очередь пуста — выбрасывает IndexError("Пусто").
    \item Создайте метод is\_empty, который возвращает True, если очередь пуста, и False в противном случае.
    \item Создайте экземпляр класса Queue с enqueue\_fibonacci\_only=True.
    \item Добавьте элементы: 4 (не Фибоначчи), 5 (Фибоначчи), 6 (не Фибоначчи), 8 (Фибоначчи), 7 (не Фибоначчи), 13 (Фибоначчи).
    \item Выведите текущее состояние очереди.
    \item Вызовите dequeue, выведите удаленный элемент.
    \item Выведите обновленное состояние очереди.
\end{enumerate}

Пример использования:
\begin{lstlisting}[language=Python]
FIB_SET = {0, 1, 2, 3, 5, 8, 13, 21, 34, 55, 89}

queue = Queue(enqueue_fibonacci_only=True)
queue.enqueue(4)   # нет
queue.enqueue(5)   # да
queue.enqueue(6)   # нет
queue.enqueue(8)   # да
queue.enqueue(7)   # нет
queue.enqueue(13)  # да

print("Current Queue:", queue._parcel)  # [5,8,13]

dequeued_item = queue.dequeue()
print("Dequeued item:", dequeued_item)  # 5

print("Updated Queue:", queue._parcel)  # [8,13]
\end{lstlisting}

\item Написать программу на Python, которая создает класс Queue для представления структуры данных очереди с инкапсуляцией. Класс должен содержать методы enqueue, dequeue и is\_empty, которые реализуют операции добавления элементов в очередь, удаления элементов из очереди и проверки пустоты очереди соответственно. Программа также должна создавать экземпляр класса Queue, добавлять элементы в очередь, удалять элементы из очереди и выводить информацию о состоянии очереди на экран.

Инструкции:
\begin{enumerate}
    \item Создайте класс Queue с методом \_\_init\_\_, который инициализирует пустую очередь (список \_sack). Принимает параметр enqueue\_palindrome\_only=False. Если True, то добавляются только числа-палиндромы.
    \item Создайте метод enqueue, который добавляет элемент, только если enqueue\_palindrome\_only=False или element — палиндром (str(element) == str(element)[::-1]).
    \item Создайте метод dequeue, который удаляет и возвращает первый элемент. Если очередь пуста — выбрасывает IndexError("Очередь пуста").
    \item Создайте метод is\_empty, который возвращает True, если очередь пуста, и False в противном случае.
    \item Создайте экземпляр класса Queue с enqueue\_palindrome\_only=True.
    \item Добавьте элементы: 12 (не палиндром), 22 (палиндром), 34 (не палиндром), 55 (палиндром), 123 (не палиндром), 121 (палиндром).
    \item Выведите текущее состояние очереди.
    \item Вызовите dequeue, выведите удаленный элемент.
    \item Выведите обновленное состояние очереди.
\end{enumerate}

Пример использования:
\begin{lstlisting}[language=Python]
queue = Queue(enqueue_palindrome_only=True)
queue.enqueue(12)   # нет
queue.enqueue(22)   # да
queue.enqueue(34)   # нет
queue.enqueue(55)   # да
queue.enqueue(123)  # нет
queue.enqueue(121)  # да

print("Current Queue:", queue._sack)  # [22,55,121]

dequeued_item = queue.dequeue()
print("Dequeued item:", dequeued_item)  # 22

print("Updated Queue:", queue._sack)  # [55,121]
\end{lstlisting}

\item Написать программу на Python, которая создает класс Queue для представления структуры данных очереди с инкапсуляцией. Класс должен содержать методы enqueue, dequeue и is\_empty, которые реализуют операции добавления элементов в очередь, удаления элементов из очереди и проверки пустоты очереди соответственно. Программа также должна создавать экземпляр класса Queue, добавлять элементы в очередь, удалять элементы из очереди и выводить информацию о состоянии очереди на экран.

Инструкции:
\begin{enumerate}
    \item Создайте класс Queue с методом \_\_init\_\_, который инициализирует пустую очередь (список \_bag). Принимает параметр enqueue\_power\_of\_two=False. Если True, то добавляются только степени двойки.
    \item Создайте метод enqueue, который добавляет элемент, только если enqueue\_power\_of\_two=False или element > 0 и (element \& (element-1)) == 0.
    \item Создайте метод dequeue, который удаляет и возвращает первый элемент. Если очередь пуста — выбрасывает IndexError("Пусто").
    \item Создайте метод is\_empty, который возвращает True, если очередь пуста, и False в противном случае.
    \item Создайте экземпляр класса Queue с enqueue\_power\_of\_two=True.
    \item Добавьте элементы: 3 (не степень), 4 (степень), 5 (не степень), 8 (степень), 9 (не степень), 16 (степень).
    \item Выведите текущее состояние очереди.
    \item Вызовите dequeue, выведите удаленный элемент.
    \item Выведите обновленное состояние очереди.
\end{enumerate}

Пример использования:
\begin{lstlisting}[language=Python]
queue = Queue(enqueue_power_of_two=True)
queue.enqueue(3)   # нет
queue.enqueue(4)   # да
queue.enqueue(5)   # нет
queue.enqueue(8)   # да
queue.enqueue(9)   # нет
queue.enqueue(16)  # да

print("Current Queue:", queue._bag)  # [4,8,16]

dequeued_item = queue.dequeue()
print("Dequeued item:", dequeued_item)  # 4

print("Updated Queue:", queue._bag)  # [8,16]
\end{lstlisting}

\item Написать программу на Python, которая создает класс Queue для представления структуры данных очереди с инкапсуляцией. Класс должен содержать методы enqueue, dequeue и is\_empty, которые реализуют операции добавления элементов в очередь, удаления элементов из очереди и проверки пустоты очереди соответственно. Программа также должна создавать экземпляр класса Queue, добавлять элементы в очередь, удалять элементы из очереди и выводить информацию о состоянии очереди на экран.

Инструкции:
\begin{enumerate}
    \item Создайте класс Queue с методом \_\_init\_\_, который инициализирует пустую очередь (список \_suitcase). Принимает параметр enqueue\_divisible\_by\_three=False. Если True, то добавляются только числа, делящиеся на 3.
    \item Создайте метод enqueue, который добавляет элемент, только если enqueue\_divisible\_by\_three=False или element \% 3 == 0.
    \item Создайте метод dequeue, который удаляет и возвращает первый элемент. Если очередь пуста — выбрасывает IndexError("Очередь пуста").
    \item Создайте метод is\_empty, который возвращает True, если очередь пуста, и False в противном случае.
    \item Создайте экземпляр класса Queue с enqueue\_divisible\_by\_three=True.
    \item Добавьте элементы: 1 (нет), 3 (да), 4 (нет), 6 (да), 7 (нет), 9 (да).
    \item Выведите текущее состояние очереди.
    \item Вызовите dequeue, выведите удаленный элемент.
    \item Выведите обновленное состояние очереди.
\end{enumerate}

Пример использования:
\begin{lstlisting}[language=Python]
queue = Queue(enqueue_divisible_by_three=True)
queue.enqueue(1)  # нет
queue.enqueue(3)  # да
queue.enqueue(4)  # нет
queue.enqueue(6)  # да
queue.enqueue(7)  # нет
queue.enqueue(9)  # да

print("Current Queue:", queue._suitcase)  # [3,6,9]

dequeued_item = queue.dequeue()
print("Dequeued item:", dequeued_item)  # 3

print("Updated Queue:", queue._suitcase)  # [6,9]
\end{lstlisting}

\item Написать программу на Python, которая создает класс Queue для представления структуры данных очереди с инкапсуляцией. Класс должен содержать методы enqueue, dequeue и is\_empty, которые реализуют операции добавления элементов в очередь, удаления элементов из очереди и проверки пустоты очереди соответственно. Программа также должна создавать экземпляр класса Queue, добавлять элементы в очередь, удалять элементы из очереди и выводить информацию о состоянии очереди на экран.

Инструкции:
\begin{enumerate}
    \item Создайте класс Queue с методом \_\_init\_\_, который инициализирует пустую очередь (список \_luggage). Принимает параметр enqueue\_greater\_than\_prev=False. Если True, то элемент добавляется только если он строго больше предыдущего добавленного элемента (первый — всегда).
    \item Создайте метод enqueue, который добавляет элемент, только если enqueue\_greater\_than\_prev=False или очередь пуста или element > последний\_элемент.
    \item Создайте метод dequeue, который удаляет и возвращает первый элемент. Если очередь пуста — выбрасывает IndexError("Пусто").
    \item Создайте метод is\_empty, который возвращает True, если очередь пуста, и False в противном случае.
    \item Создайте экземпляр класса Queue с enqueue\_greater\_than\_prev=True.
    \item Добавьте элементы: 5, 3 (не добавится), 7, 6 (не добавится), 10, 8 (не добавится).
    \item Выведите текущее состояние очереди.
    \item Вызовите dequeue, выведите удаленный элемент.
    \item Выведите обновленное состояние очереди.
\end{enumerate}

Пример использования:
\begin{lstlisting}[language=Python]
queue = Queue(enqueue_greater_than_prev=True)
queue.enqueue(5)
queue.enqueue(3)  # нет
queue.enqueue(7)
queue.enqueue(6)  # нет
queue.enqueue(10)
queue.enqueue(8)  # нет

print("Current Queue:", queue._luggage)  # [5,7,10]

dequeued_item = queue.dequeue()
print("Dequeued item:", dequeued_item)  # 5

print("Updated Queue:", queue._luggage)  # [7,10]
\end{lstlisting}

\item Написать программу на Python, которая создает класс Queue для представления структуры данных очереди с инкапсуляцией. Класс должен содержать методы enqueue, dequeue и is\_empty, которые реализуют операции добавления элементов в очередь, удаления элементов из очереди и проверки пустоты очереди соответственно. Программа также должна создавать экземпляр класса Queue, добавлять элементы в очередь, удалять элементы из очереди и выводить информацию о состоянии очереди на экран.

Инструкции:
\begin{enumerate}
    \item Создайте класс Queue с методом \_\_init\_\_, который инициализирует пустую очередь (список \_trunk). Принимает параметр enqueue\_less\_than\_prev=False. Если True, то элемент добавляется только если он строго меньше предыдущего добавленного элемента (первый — всегда).
    \item Создайте метод enqueue, который добавляет элемент, только если enqueue\_less\_than\_prev=False или очередь пуста или element < последний\_элемент.
    \item Создайте метод dequeue, который удаляет и возвращает первый элемент. Если очередь пуста — выбрасывает IndexError("Очередь пуста").
    \item Создайте метод is\_empty, который возвращает True, если очередь пуста, и False в противном случае.
    \item Создайте экземпляр класса Queue с enqueue\_less\_than\_prev=True.
    \item Добавьте элементы: 10, 15 (не добавится), 8, 9 (не добавится), 5, 7 (не добавится).
    \item Выведите текущее состояние очереди.
    \item Вызовите dequeue, выведите удаленный элемент.
    \item Выведите обновленное состояние очереди.
\end{enumerate}

Пример использования:
\begin{lstlisting}[language=Python]
queue = Queue(enqueue_less_than_prev=True)
queue.enqueue(10)
queue.enqueue(15)  # нет
queue.enqueue(8)
queue.enqueue(9)   # нет
queue.enqueue(5)
queue.enqueue(7)   # нет

print("Current Queue:", queue._trunk)  # [10,8,5]

dequeued_item = queue.dequeue()
print("Dequeued item:", dequeued_item)  # 10

print("Updated Queue:", queue._trunk)  # [8,5]
\end{lstlisting}

\end{enumerate}