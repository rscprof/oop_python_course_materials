\begin{enumerate}
\item[1] \textbf{Формирование состава груженых контейнеров}
\begin{enumerate}
    \item Создайте класс \texttt{Container}, который будет представлять собой контейнер с грузом. В конструкторе класса \texttt{Container} инициализируйте значения контейнера и груза из списков \texttt{NumList} и \texttt{MasList}, которые объявлены как общие атрибуты класса. \texttt{NumList: list[str]} — это список контейнеров (не менее 14): 
    \begin{center}
        \texttt{[''Контейнер\_1'', ''Контейнер\_2'', \dots, ''Контейнер\_14'']}
    \end{center}
    \texttt{MasList: list[str]} — это список грузов для контейнеров (не менее 4):
    \begin{center}
        \texttt{[''Электроника'', ''Мебель'', ''Одежда'', ''Продукты'']}
    \end{center}
    Конструктор должен иметь сигнатуру: \texttt{\_\_init\_\_(self) -> None}.

    \item Создайте класс \texttt{TrainOfContainers}, который будет представлять собой состав, состоящий из моделей контейнеров. В конструкторе класса \texttt{TrainOfContainers} инициализируйте список контейнеров \texttt{self.train: list[Container]} длиной 56.

    \item Добавьте метод \texttt{shuffle(self) -> None} в класс \texttt{TrainOfContainers}, который будет перемешивать контейнеры в списке \texttt{self.train}.

    \item Добавьте метод \texttt{get(self, i: int) -> Container}, который будет возвращать $i$-й контейнер и груз из списка \texttt{self.train}.

    \item Создайте экземпляр класса \texttt{TrainOfContainers} и вызовите метод \texttt{shuffle} для перемешивания контейнеров.

    \item Создайте цикл, который будет запрашивать у пользователя номер контейнера из состава и выводить информацию о выбранном контейнере.

    \item Повторите шаги 5–6 до тех пор, пока пользователь не выберет все контейнеры или не завершит выбор.

    \item В конце программы выводите сообщение о завершении выбора контейнеров.

    \item Убедитесь, что пользователь вводит корректные номера контейнеров и что программа обрабатывает ошибки, связанные с вводом пользователя.

    \item Проверьте работу программы, используя различные комбинации номеров контейнеров и грузов.
\end{enumerate}

\item[2] \textbf{Формирование состава почтовых посылок}
\begin{enumerate}
    \item Создайте класс \texttt{Parcel}, который будет представлять собой посылку с содержимым. В конструкторе класса \texttt{Parcel} инициализируйте значения посылки и содержимого из списков \texttt{NumList} и \texttt{MasList}, которые объявлены как общие атрибуты класса. \texttt{NumList: list[str]} — это список посылок (не менее 14): 
    \begin{center}
        \texttt{[''Посылка\_1'', ''Посылка\_2'', \dots, ''Посылка\_14'']}
    \end{center}
    \texttt{MasList: list[str]} — это список содержимого посылок (не менее 4):
    \begin{center}
        \texttt{[''Книги'', ''Игрушки'', ''Косметика'', ''Спортивный инвентарь'']}
    \end{center}
    Конструктор должен иметь сигнатуру: \texttt{\_\_init\_\_(self) -> None}.

    \item Создайте класс \texttt{TrainOfParcels}, который будет представлять собой состав, состоящий из моделей посылок. В конструкторе класса \texttt{TrainOfParcels} инициализируйте список посылок \texttt{self.train: list[Parcel]} длиной 56.

    \item Добавьте метод \texttt{shuffle(self) -> None} в класс \texttt{TrainOfParcels}, который будет перемешивать посылки в списке \texttt{self.train}.

    \item Добавьте метод \texttt{get(self, i: int) -> Parcel}, который будет возвращать $i$-ю посылку и её содержимое из списка \texttt{self.train}.

    \item Создайте экземпляр класса \texttt{TrainOfParcels} и вызовите метод \texttt{shuffle} для перемешивания посылок.

    \item Создайте цикл, который будет запрашивать у пользователя номер посылки из состава и выводить информацию о выбранной посылке.

    \item Повторите шаги 5–6 до тех пор, пока пользователь не выберет все посылки или не завершит выбор.

    \item В конце программы выводите сообщение о завершении выбора посылок.

    \item Убедитесь, что пользователь вводит корректные номера посылок и что программа обрабатывает ошибки, связанные с вводом пользователя.

    \item Проверьте работу программы, используя различные комбинации номеров посылок и содержимого.
\end{enumerate}

\item[3] \textbf{Формирование автовоза с автомобилями}
\begin{enumerate}
    \item Создайте класс \texttt{Car}, который будет представлять собой автомобиль на автовозе. В конструкторе класса \texttt{Car} инициализируйте значения автомобиля и его марки из списков \texttt{NumList} и \texttt{MasList}, которые объявлены как общие атрибуты класса. \texttt{NumList: list[str]} — это список автомобилей (не менее 14): 
    \begin{center}
        \texttt{[''Автомобиль\_1'', ''Автомобиль\_2'', \dots, ''Автомобиль\_14'']}
    \end{center}
    \texttt{MasList: list[str]} — это список марок автомобилей (не менее 4):
    \begin{center}
        \texttt{[''Toyota'', ''BMW'', ''Lada'', ''Tesla'']}
    \end{center}
    Конструктор должен иметь сигнатуру: \texttt{\_\_init\_\_(self) -> None}.

    \item Создайте класс \texttt{CarCarrier}, который будет представлять собой автовоз, состоящий из моделей автомобилей. В конструкторе класса \texttt{CarCarrier} инициализируйте список автомобилей \texttt{self.train: list[Car]} длиной 56.

    \item Добавьте метод \texttt{shuffle(self) -> None} в класс \texttt{CarCarrier}, который будет перемешивать автомобили в списке \texttt{self.train}.

    \item Добавьте метод \texttt{get(self, i: int) -> Car}, который будет возвращать $i$-й автомобиль и его марку из списка \texttt{self.train}.

    \item Создайте экземпляр класса \texttt{CarCarrier} и вызовите метод \texttt{shuffle} для перемешивания автомобилей.

    \item Создайте цикл, который будет запрашивать у пользователя номер автомобиля на автовозе и выводить информацию о выбранном автомобиле.

    \item Повторите шаги 5–6 до тех пор, пока пользователь не выберет все автомобили или не завершит выбор.

    \item В конце программы выводите сообщение о завершении выбора автомобилей.

    \item Убедитесь, что пользователь вводит корректные номера автомобилей и что программа обрабатывает ошибки, связанные с вводом пользователя.

    \item Проверьте работу программы, используя различные комбинации номеров автомобилей и марок.
\end{enumerate}

\item[4] \textbf{Формирование багажного состава из чемоданов}
\begin{enumerate}
    \item Создайте класс \texttt{Suitcase}, который будет представлять собой чемодан с владельцем. В конструкторе класса \texttt{Suitcase} инициализируйте значения чемодана и владельца из списков \texttt{NumList} и \texttt{MasList}, которые объявлены как общие атрибуты класса. \texttt{NumList: list[str]} — это список чемоданов (не менее 14): 
    \begin{center}
        \texttt{[''Чемодан\_1'', ''Чемодан\_2'', \dots, ''Чемодан\_14'']}
    \end{center}
    \texttt{MasList: list[str]} — это список владельцев (не менее 4):
    \begin{center}
        \texttt{[''Иванов'', ''Петров'', ''Сидоров'', ''Кузнецов'']}
    \end{center}
    Конструктор должен иметь сигнатуру: \texttt{\_\_init\_\_(self) -> None}.

    \item Создайте класс \texttt{BaggageTrain}, который будет представлять собой багажный состав, состоящий из моделей чемоданов. В конструкторе класса \texttt{BaggageTrain} инициализируйте список чемоданов \texttt{self.train: list[Suitcase]} длиной 56.

    \item Добавьте метод \texttt{shuffle(self) -> None} в класс \texttt{BaggageTrain}, который будет перемешивать чемоданы в списке \texttt{self.train}.

    \item Добавьте метод \texttt{get(self, i: int) -> Suitcase}, который будет возвращать $i$-й чемодан и его владельца из списка \texttt{self.train}.

    \item Создайте экземпляр класса \texttt{BaggageTrain} и вызовите метод \texttt{shuffle} для перемешивания чемоданов.

    \item Создайте цикл, который будет запрашивать у пользователя номер чемодана из состава и выводить информацию о выбранном чемодане.

    \item Повторите шаги 5–6 до тех пор, пока пользователь не выберет все чемоданы или не завершит выбор.

    \item В конце программы выводите сообщение о завершении выбора чемоданов.

    \item Убедитесь, что пользователь вводит корректные номера чемоданов и что программа обрабатывает ошибки, связанные с вводом пользователя.

    \item Проверьте работу программы, используя различные комбинации номеров чемоданов и владельцев.
\end{enumerate}

\item[5] \textbf{Формирование складского состава из ящиков}
\begin{enumerate}
    \item Создайте класс \texttt{Box}, который будет представлять собой ящик с содержимым. В конструкторе класса \texttt{Box} инициализируйте значения ящика и содержимого из списков \texttt{NumList} и \texttt{MasList}, которые объявлены как общие атрибуты класса. \texttt{NumList: list[str]} — это список ящиков (не менее 14): 
    \begin{center}
        \texttt{[''Ящик\_1'', ''Ящик\_2'', \dots, ''Ящик\_14'']}
    \end{center}
    \texttt{MasList: list[str]} — это список типов содержимого (не менее 4):
    \begin{center}
        \texttt{[''Инструменты'', ''Запчасти'', ''Химикаты'', ''Упаковка'']}
    \end{center}
    Конструктор должен иметь сигнатуру: \texttt{\_\_init\_\_(self) -> None}.

    \item Создайте класс \texttt{WarehouseTrain}, который будет представлять собой состав, состоящий из моделей ящиков. В конструкторе класса \texttt{WarehouseTrain} инициализируйте список ящиков \texttt{self.train: list[Box]} длиной 56.

    \item Добавьте метод \texttt{shuffle(self) -> None} в класс \texttt{WarehouseTrain}, который будет перемешивать ящики в списке \texttt{self.train}.

    \item Добавьте метод \texttt{get(self, i: int) -> Box}, который будет возвращать $i$-й ящик и его содержимое из списка \texttt{self.train}.

    \item Создайте экземпляр класса \texttt{WarehouseTrain} и вызовите метод \texttt{shuffle} для перемешивания ящиков.

    \item Создайте цикл, который будет запрашивать у пользователя номер ящика из состава и выводить информацию о выбранном ящике.

    \item Повторите шаги 5–6 до тех пор, пока пользователь не выберет все ящики или не завершит выбор.

    \item В конце программы выводите сообщение о завершении выбора ящиков.

    \item Убедитесь, что пользователь вводит корректные номера ящиков и что программа обрабатывает ошибки, связанные с вводом пользователя.

    \item Проверьте работу программы, используя различные комбинации номеров ящиков и содержимого.
\end{enumerate}

\item[6] \textbf{Формирование состава морских судов с грузом}
\begin{enumerate}
    \item Создайте класс \texttt{Ship}, который будет представлять собой судно с грузом. В конструкторе класса \texttt{Ship} инициализируйте значения судна и груза из списков \texttt{NumList} и \texttt{MasList}, которые объявлены как общие атрибуты класса. \texttt{NumList: list[str]} — это список судов (не менее 14): 
    \begin{center}
        \texttt{[''Судно\_1'', ''Судно\_2'', \dots, ''Судно\_14'']}
    \end{center}
    \texttt{MasList: list[str]} — это список грузов (не менее 4):
    \begin{center}
        \texttt{[''Нефть'', ''Уголь'', ''Зерно'', ''Лес'']}
    \end{center}
    Конструктор должен иметь сигнатуру: \texttt{\_\_init\_\_(self) -> None}.

    \item Создайте класс \texttt{Fleet}, который будет представлять собой флотилию, состоящую из моделей судов. В конструкторе класса \texttt{Fleet} инициализируйте список судов \texttt{self.train: list[Ship]} длиной 56.

    \item Добавьте метод \texttt{shuffle(self) -> None} в класс \texttt{Fleet}, который будет перемешивать суда в списке \texttt{self.train}.

    \item Добавьте метод \texttt{get(self, i: int) -> Ship}, который будет возвращать $i$-е судно и его груз из списка \texttt{self.train}.

    \item Создайте экземпляр класса \texttt{Fleet} и вызовите метод \texttt{shuffle} для перемешивания судов.

    \item Создайте цикл, который будет запрашивать у пользователя номер судна из флотилии и выводить информацию о выбранном судне.

    \item Повторите шаги 5–6 до тех пор, пока пользователь не выберет все суда или не завершит выбор.

    \item В конце программы выводите сообщение о завершении выбора судов.

    \item Убедитесь, что пользователь вводит корректные номера судов и что программа обрабатывает ошибки, связанные с вводом пользователя.

    \item Проверьте работу программы, используя различные комбинации номеров судов и грузов.
\end{enumerate}

\item[7] \textbf{Формирование состава ракет-носителей}
\begin{enumerate}
    \item Создайте класс \texttt{Rocket}, который будет представлять собой ракету с полезной нагрузкой. В конструкторе класса \texttt{Rocket} инициализируйте значения ракеты и нагрузки из списков \texttt{NumList} и \texttt{MasList}, которые объявлены как общие атрибуты класса. \texttt{NumList: list[str]} — это список ракет (не менее 14): 
    \begin{center}
        \texttt{[''Ракета\_1'', ''Ракета\_2'', \dots, ''Ракета\_14'']}
    \end{center}
    \texttt{MasList: list[str]} — это список типов нагрузки (не менее 4):
    \begin{center}
        \texttt{[''Спутник'', ''Грузовой модуль'', ''Экипаж'', ''Научное оборудование'']}
    \end{center}
    Конструктор должен иметь сигнатуру: \texttt{\_\_init\_\_(self) -> None}.

    \item Создайте класс \texttt{RocketTrain}, который будет представлять собой состав ракет. В конструкторе класса \texttt{RocketTrain} инициализируйте список ракет \texttt{self.train: list[Rocket]} длиной 56.

    \item Добавьте метод \texttt{shuffle(self) -> None} в класс \texttt{RocketTrain}, который будет перемешивать ракеты в списке \texttt{self.train}.

    \item Добавьте метод \texttt{get(self, i: int) -> Rocket}, который будет возвращать $i$-ю ракету и её нагрузку из списка \texttt{self.train}.

    \item Создайте экземпляр класса \texttt{RocketTrain} и вызовите метод \texttt{shuffle} для перемешивания ракет.

    \item Создайте цикл, который будет запрашивать у пользователя номер ракеты и выводить информацию о выбранной ракете.

    \item Повторите шаги 5–6 до тех пор, пока пользователь не выберет все ракеты или не завершит выбор.

    \item В конце программы выводите сообщение о завершении выбора ракет.

    \item Убедитесь, что пользователь вводит корректные номера ракет и что программа обрабатывает ошибки, связанные с вводом пользователя.

    \item Проверьте работу программы, используя различные комбинации номеров ракет и нагрузок.
\end{enumerate}

\item[8] \textbf{Формирование состава дронов с грузом}
\begin{enumerate}
    \item Создайте класс \texttt{Drone}, который будет представлять собой дрон с миссией. В конструкторе класса \texttt{Drone} инициализируйте значения дрона и миссии из списков \texttt{NumList} и \texttt{MasList}, которые объявлены как общие атрибуты класса. \texttt{NumList: list[str]} — это список дронов (не менее 14): 
    \begin{center}
        \texttt{[''Дрон\_1'', ''Дрон\_2'', \dots, ''Дрон\_14'']}
    \end{center}
    \texttt{MasList: list[str]} — это список миссий (не менее 4):
    \begin{center}
        \texttt{[''Фотосъёмка'', ''Доставка'', ''Разведка'', ''Мониторинг'']}
    \end{center}
    Конструктор должен иметь сигнатуру: \texttt{\_\_init\_\_(self) -> None}.

    \item Создайте класс \texttt{DroneSquadron}, который будет представлять собой эскадрилью, состоящую из моделей дронов. В конструкторе класса \texttt{DroneSquadron} инициализируйте список дронов \texttt{self.train: list[Drone]} длиной 56.

    \item Добавьте метод \texttt{shuffle(self) -> None} в класс \texttt{DroneSquadron}, который будет перемешивать дроны в списке \texttt{self.train}.

    \item Добавьте метод \texttt{get(self, i: int) -> Drone}, который будет возвращать $i$-й дрон и его миссию из списка \texttt{self.train}.

    \item Создайте экземпляр класса \texttt{DroneSquadron} и вызовите метод \texttt{shuffle} для перемешивания дронов.

    \item Создайте цикл, который будет запрашивать у пользователя номер дрона и выводить информацию о выбранном дроне.

    \item Повторите шаги 5–6 до тех пор, пока пользователь не выберет все дроны или не завершит выбор.

    \item В конце программы выводите сообщение о завершении выбора дронов.

    \item Убедитесь, что пользователь вводит корректные номера дронов и что программа обрабатывает ошибки, связанные с вводом пользователя.

    \item Проверьте работу программы, используя различные комбинации номеров дронов и миссий.
\end{enumerate}

\item[9] \textbf{Формирование состава тележек в супермаркете}
\begin{enumerate}
    \item Создайте класс \texttt{Trolley}, который будет представлять собой тележку с типом покупателя. В конструкторе класса \texttt{Trolley} инициализируйте значения тележки и типа покупателя из списков \texttt{NumList} и \texttt{MasList}, которые объявлены как общие атрибуты класса. \texttt{NumList: list[str]} — это список тележек (не менее 14): 
    \begin{center}
        \texttt{[''Тележка\_1'', ''Тележка\_2'', \dots, ''Тележка\_14'']}
    \end{center}
    \texttt{MasList: list[str]} — это список типов покупателей (не менее 4):
    \begin{center}
        \texttt{[''Семья'', ''Студент'', ''Пенсионер'', ''Турист'']}
    \end{center}
    Конструктор должен иметь сигнатуру: \texttt{\_\_init\_\_(self) -> None}.

    \item Создайте класс \texttt{TrolleyTrain}, который будет представлять собой состав тележек. В конструкторе класса \texttt{TrolleyTrain} инициализируйте список тележек \texttt{self.train: list[Trolley]} длиной 56.

    \item Добавьте метод \texttt{shuffle(self) -> None} в класс \texttt{TrolleyTrain}, который будет перемешивать тележки в списке \texttt{self.train}.

    \item Добавьте метод \texttt{get(self, i: int) -> Trolley}, который будет возвращать $i$-ю тележку и тип её покупателя из списка \texttt{self.train}.

    \item Создайте экземпляр класса \texttt{TrolleyTrain} и вызовите метод \texttt{shuffle} для перемешивания тележек.

    \item Создайте цикл, который будет запрашивать у пользователя номер тележки и выводить информацию о ней.

    \item Повторите шаги 5–6 до тех пор, пока пользователь не выберет все тележки или не завершит выбор.

    \item В конце программы выводите сообщение о завершении выбора тележек.

    \item Убедитесь, что пользователь вводит корректные номера тележек и что программа обрабатывает ошибки, связанные с вводом пользователя.

    \item Проверьте работу программы, используя различные комбинации номеров тележек и типов покупателей.
\end{enumerate}

\item[10] \textbf{Формирование состава камер хранения}
\begin{enumerate}
    \item Создайте класс \texttt{Locker}, который будет представлять собой камеру хранения с содержимым. В конструкторе класса \texttt{Locker} инициализируйте значения камеры и содержимого из списков \texttt{NumList} и \texttt{MasList}, которые объявлены как общие атрибуты класса. \texttt{NumList: list[str]} — это список камер (не менее 14): 
    \begin{center}
        \texttt{[''Камера\_1'', ''Камера\_2'', \dots, ''Камера\_14'']}
    \end{center}
    \texttt{MasList: list[str]} — это список типов содержимого (не менее 4):
    \begin{center}
        \texttt{[''Велосипед'', ''Чемодан'', ''Инструменты'', ''Спортинвентарь'']}
    \end{center}
    Конструктор должен иметь сигнатуру: \texttt{\_\_init\_\_(self) -> None}.

    \item Создайте класс \texttt{StorageTrain}, который будет представлять собой состав камер хранения. В конструкторе класса \texttt{StorageTrain} инициализируйте список камер \texttt{self.train: list[Locker]} длиной 56.

    \item Добавьте метод \texttt{shuffle(self) -> None} в класс \texttt{StorageTrain}, который будет перемешивать камеры в списке \texttt{self.train}.

    \item Добавьте метод \texttt{get(self, i: int) -> Locker}, который будет возвращать $i$-ю камеру и её содержимое из списка \texttt{self.train}.

    \item Создайте экземпляр класса \texttt{StorageTrain} и вызовите метод \texttt{shuffle} для перемешивания камер.

    \item Создайте цикл, который будет запрашивать у пользователя номер камеры и выводить информацию о ней.

    \item Повторите шаги 5–6 до тех пор, пока пользователь не выберет все камеры или не завершит выбор.

    \item В конце программы выводите сообщение о завершении выбора камер.

    \item Убедитесь, что пользователь вводит корректные номера камер и что программа обрабатывает ошибки, связанные с вводом пользователя.

    \item Проверьте работу программы, используя различные комбинации номеров камер и содержимого.
\end{enumerate}

\item[11] \textbf{Формирование состава самолётов с бортами}
\begin{enumerate}
    \item Создайте класс \texttt{Aircraft}, который будет представлять собой самолёт с типом рейса. В конструкторе класса \texttt{Aircraft} инициализируйте значения самолёта и типа рейса из списков \texttt{NumList} и \texttt{MasList}, которые объявлены как общие атрибуты класса. \texttt{NumList: list[str]} — это список самолётов (не менее 14): 
    \begin{center}
        \texttt{[''Борт\_1'', ''Борт\_2'', \dots, ''Борт\_14'']}
    \end{center}
    \texttt{MasList: list[str]} — это список типов рейсов (не менее 4):
    \begin{center}
        \texttt{[''Пассажирский'', ''Грузовой'', ''Военный'', ''Санитарный'']}
    \end{center}
    Конструктор должен иметь сигнатуру: \texttt{\_\_init\_\_(self) -> None}.

    \item Создайте класс \texttt{AirFleet}, который будет представлять собой воздушный флот, состоящий из моделей самолётов. В конструкторе класса \texttt{AirFleet} инициализируйте список самолётов \texttt{self.train: list[Aircraft]} длиной 56.

    \item Добавьте метод \texttt{shuffle(self) -> None} в класс \texttt{AirFleet}, который будет перемешивать самолёты в списке \texttt{self.train}.

    \item Добавьте метод \texttt{get(self, i: int) -> Aircraft}, который будет возвращать $i$-й самолёт и его тип рейса из списка \texttt{self.train}.

    \item Создайте экземпляр класса \texttt{AirFleet} и вызовите метод \texttt{shuffle} для перемешивания самолётов.

    \item Создайте цикл, который будет запрашивать у пользователя номер самолёта и выводить информацию о нём.

    \item Повторите шаги 5–6 до тех пор, пока пользователь не выберет все самолёты или не завершит выбор.

    \item В конце программы выводите сообщение о завершении выбора самолётов.

    \item Убедитесь, что пользователь вводит корректные номера самолётов и что программа обрабатывает ошибки, связанные с вводом пользователя.

    \item Проверьте работу программы, используя различные комбинации номеров самолётов и типов рейсов.
\end{enumerate}

\item[12] \textbf{Формирование состава танкеров с жидкостями}
\begin{enumerate}
    \item Создайте класс \texttt{Tanker}, который будет представлять собой танкер с жидкостью. В конструкторе класса \texttt{Tanker} инициализируйте значения танкера и жидкости из списков \texttt{NumList} и \texttt{MasList}, которые объявлены как общие атрибуты класса. \texttt{NumList: list[str]} — это список танкеров (не менее 14): 
    \begin{center}
        \texttt{[''Танкер\_1'', ''Танкер\_2'', \dots, ''Танкер\_14'']}
    \end{center}
    \texttt{MasList: list[str]} — это список жидкостей (не менее 4):
    \begin{center}
        \texttt{[''Вода'', ''Молоко'', ''Топливо'', ''Химикаты'']}
    \end{center}
    Конструктор должен иметь сигнатуру: \texttt{\_\_init\_\_(self) -> None}.

    \item Создайте класс \texttt{TankerConvoy}, который будет представлять собой конвой танкеров. В конструкторе класса \texttt{TankerConvoy} инициализируйте список танкеров \texttt{self.train: list[Tanker]} длиной 56.

    \item Добавьте метод \texttt{shuffle(self) -> None} в класс \texttt{TankerConvoy}, который будет перемешивать танкеры в списке \texttt{self.train}.

    \item Добавьте метод \texttt{get(self, i: int) -> Tanker}, который будет возвращать $i$-й танкер и его жидкость из списка \texttt{self.train}.

    \item Создайте экземпляр класса \texttt{TankerConvoy} и вызовите метод \texttt{shuffle} для перемешивания танкеров.

    \item Создайте цикл, который будет запрашивать у пользователя номер танкера и выводить информацию о нём.

    \item Повторите шаги 5–6 до тех пор, пока пользователь не выберет все танкеры или не завершит выбор.

    \item В конце программы выводите сообщение о завершении выбора танкеров.

    \item Убедитесь, что пользователь вводит корректные номера танкеров и что программа обрабатывает ошибки, связанные с вводом пользователя.

    \item Проверьте работу программы, используя различные комбинации номеров танкеров и жидкостей.
\end{enumerate}

\item[13] \textbf{Формирование состава паллет на складе}
\begin{enumerate}
    \item Создайте класс \texttt{Pallet}, который будет представлять собой паллету с товаром. В конструкторе класса \texttt{Pallet} инициализируйте значения паллеты и товара из списков \texttt{NumList} и \texttt{MasList}, которые объявлены как общие атрибуты класса. \texttt{NumList: list[str]} — это список паллет (не менее 14): 
    \begin{center}
        \texttt{[''Паллета\_1'', ''Паллета\_2'', \dots, ''Паллета\_14'']}
    \end{center}
    \texttt{MasList: list[str]} — это список типов товаров (не менее 4):
    \begin{center}
        \texttt{[''Напитки'', ''Консервы'', ''Бытовая химия'', ''Бумажная продукция'']}
    \end{center}
    Конструктор должен иметь сигнатуру: \texttt{\_\_init\_\_(self) -> None}.

    \item Создайте класс \texttt{PalletTrain}, который будет представлять собой состав паллет. В конструкторе класса \texttt{PalletTrain} инициализируйте список паллет \texttt{self.train: list[Pallet]} длиной 56.

    \item Добавьте метод \texttt{shuffle(self) -> None} в класс \texttt{PalletTrain}, который будет перемешивать паллеты в списке \texttt{self.train}.

    \item Добавьте метод \texttt{get(self, i: int) -> Pallet}, который будет возвращать $i$-ю паллету и её товар из списка \texttt{self.train}.

    \item Создайте экземпляр класса \texttt{PalletTrain} и вызовите метод \texttt{shuffle} для перемешивания паллет.

    \item Создайте цикл, который будет запрашивать у пользователя номер паллеты и выводить информацию о ней.

    \item Повторите шаги 5–6 до тех пор, пока пользователь не выберет все паллеты или не завершит выбор.

    \item В конце программы выводите сообщение о завершении выбора паллет.

    \item Убедитесь, что пользователь вводит корректные номера паллет и что программа обрабатывает ошибки, связанные с вводом пользователя.

    \item Проверьте работу программы, используя различные комбинации номеров паллет и товаров.
\end{enumerate}

\item[14] \textbf{Формирование состава вагонов-цистерн}
\begin{enumerate}
    \item Создайте класс \texttt{TankWagon}, который будет представлять собой цистерну с содержимым. В конструкторе класса \texttt{TankWagon} инициализируйте значения цистерны и содержимого из списков \texttt{NumList} и \texttt{MasList}, которые объявлены как общие атрибуты класса. \texttt{NumList: list[str]} — это список цистерн (не менее 14): 
    \begin{center}
        \texttt{[''Цистерна\_1'', ''Цистерна\_2'', \dots, ''Цистерна\_14'']}
    \end{center}
    \texttt{MasList: list[str]} — это список типов содержимого (не менее 4):
    \begin{center}
        \texttt{[''Бензин'', ''Дизель'', ''Газ'', ''Вода'']}
    \end{center}
    Конструктор должен иметь сигнатуру: \texttt{\_\_init\_\_(self) -> None}.

    \item Создайте класс \texttt{TankTrain}, который будет представлять собой состав цистерн. В конструкторе класса \texttt{TankTrain} инициализируйте список цистерн \texttt{self.train: list[TankWagon]} длиной 56.

    \item Добавьте метод \texttt{shuffle(self) -> None} в класс \texttt{TankTrain}, который будет перемешивать цистерны в списке \texttt{self.train}.

    \item Добавьте метод \texttt{get(self, i: int) -> TankWagon}, который будет возвращать $i$-ю цистерну и её содержимое из списка \texttt{self.train}.

    \item Создайте экземпляр класса \texttt{TankTrain} и вызовите метод \texttt{shuffle} для перемешивания цистерн.

    \item Создайте цикл, который будет запрашивать у пользователя номер цистерны и выводить информацию о ней.

    \item Повторите шаги 5–6 до тех пор, пока пользователь не выберет все цистерны или не завершит выбор.

    \item В конце программы выводите сообщение о завершении выбора цистерн.

    \item Убедитесь, что пользователь вводит корректные номера цистерн и что программа обрабатывает ошибки, связанные с вводом пользователя.

    \item Проверьте работу программы, используя различные комбинации номеров цистерн и содержимого.
\end{enumerate}

\item[15] \textbf{Формирование состава промышленных роботов}
\begin{enumerate}
    \item Создайте класс \texttt{Robot}, который будет представлять собой робота с модулем. В конструкторе класса \texttt{Robot} инициализируйте значения робота и модуля из списков \texttt{NumList} и \texttt{MasList}, которые объявлены как общие атрибуты класса. \texttt{NumList: list[str]} — это список роботов (не менее 14): 
    \begin{center}
        \texttt{[''Робот\_1'', ''Робот\_2'', \dots, ''Робот\_14'']}
    \end{center}
    \texttt{MasList: list[str]} — это список модулей (не менее 4):
    \begin{center}
        \texttt{[''Манипулятор'', ''Камера'', ''Сенсор'', ''Батарея'']}
    \end{center}
    Конструктор должен иметь сигнатуру: \texttt{\_\_init\_\_(self) -> None}.

    \item Создайте класс \texttt{RobotLine}, который будет представлять собой производственную линию роботов. В конструкторе класса \texttt{RobotLine} инициализируйте список роботов \texttt{self.train: list[Robot]} длиной 56.

    \item Добавьте метод \texttt{shuffle(self) -> None} в класс \texttt{RobotLine}, который будет перемешивать роботов в списке \texttt{self.train}.

    \item Добавьте метод \texttt{get(self, i: int) -> Robot}, который будет возвращать $i$-го робота и его модуль из списка \texttt{self.train}.

    \item Создайте экземпляр класса \texttt{RobotLine} и вызовите метод \texttt{shuffle} для перемешивания роботов.

    \item Создайте цикл, который будет запрашивать у пользователя номер робота и выводить информацию о нём.

    \item Повторите шаги 5–6 до тех пор, пока пользователь не выберет всех роботов или не завершит выбор.

    \item В конце программы выводите сообщение о завершении выбора роботов.

    \item Убедитесь, что пользователь вводит корректные номера роботов и что программа обрабатывает ошибки, связанные с вводом пользователя.

    \item Проверьте работу программы, используя различные комбинации номеров роботов и модулей.
\end{enumerate}

\item[16] \textbf{Формирование состава клеток с животными}
\begin{enumerate}
    \item Создайте класс \texttt{Cage}, который будет представлять собой клетку с животным. В конструкторе класса \texttt{Cage} инициализируйте значения клетки и животного из списков \texttt{NumList} и \texttt{MasList}, которые объявлены как общие атрибуты класса. \texttt{NumList: list[str]} — это список клеток (не менее 14): 
    \begin{center}
        \texttt{[''Клетка\_1'', ''Клетка\_2'', \dots, ''Клетка\_14'']}
    \end{center}
    \texttt{MasList: list[str]} — это список животных (не менее 4):
    \begin{center}
        \texttt{[''Собака'', ''Кошка'', ''Попугай'', ''Кролик'']}
    \end{center}
    Конструктор должен иметь сигнатуру: \texttt{\_\_init\_\_(self) -> None}.

    \item Создайте класс \texttt{ZooTrain}, который будет представлять собой состав клеток. В конструкторе класса \texttt{ZooTrain} инициализируйте список клеток \texttt{self.train: list[Cage]} длиной 56.

    \item Добавьте метод \texttt{shuffle(self) -> None} в класс \texttt{ZooTrain}, который будет перемешивать клетки в списке \texttt{self.train}.

    \item Добавьте метод \texttt{get(self, i: int) -> Cage}, который будет возвращать $i$-ю клетку и её животное из списка \texttt{self.train}.

    \item Создайте экземпляр класса \texttt{ZooTrain} и вызовите метод \texttt{shuffle} для перемешивания клеток.

    \item Создайте цикл, который будет запрашивать у пользователя номер клетки и выводить информацию о ней.

    \item Повторите шаги 5–6 до тех пор, пока пользователь не выберет все клетки или не завершит выбор.

    \item В конце программы выводите сообщение о завершении выбора клеток.

    \item Убедитесь, что пользователь вводит корректные номера клеток и что программа обрабатывает ошибки, связанные с вводом пользователя.

    \item Проверьте работу программы, используя различные комбинации номеров клеток и животных.
\end{enumerate}

\item[17] \textbf{Формирование состава прицепов на автодороге}
\begin{enumerate}
    \item Создайте класс \texttt{Trailer}, который будет представлять собой прицеп с грузом. В конструкторе класса \texttt{Trailer} инициализируйте значения прицепа и груза из списков \texttt{NumList} и \texttt{MasList}, которые объявлены как общие атрибуты класса. \texttt{NumList: list[str]} — это список прицепов (не менее 14): 
    \begin{center}
        \texttt{[''Прицеп\_1'', ''Прицеп\_2'', \dots, ''Прицеп\_14'']}
    \end{center}
    \texttt{MasList: list[str]} — это список типов груза (не менее 4):
    \begin{center}
        \texttt{[''Строительные материалы'', ''Мебель'', ''Техника'', ''Сельхозпродукция'']}
    \end{center}
    Конструктор должен иметь сигнатуру: \texttt{\_\_init\_\_(self) -> None}.

    \item Создайте класс \texttt{TrailerConvoy}, который будет представлять собой конвой прицепов. В конструкторе класса \texttt{TrailerConvoy} инициализируйте список прицепов \texttt{self.train: list[Trailer]} длиной 56.

    \item Добавьте метод \texttt{shuffle(self) -> None} в класс \texttt{TrailerConvoy}, который будет перемешивать прицепы в списке \texttt{self.train}.

    \item Добавьте метод \texttt{get(self, i: int) -> Trailer}, который будет возвращать $i$-й прицеп и его груз из списка \texttt{self.train}.

    \item Создайте экземпляр класса \texttt{TrailerConvoy} и вызовите метод \texttt{shuffle} для перемешивания прицепов.

    \item Создайте цикл, который будет запрашивать у пользователя номер прицепа и выводить информацию о нём.

    \item Повторите шаги 5–6 до тех пор, пока пользователь не выберет все прицепы или не завершит выбор.

    \item В конце программы выводите сообщение о завершении выбора прицепов.

    \item Убедитесь, что пользователь вводит корректные номера прицепов и что программа обрабатывает ошибки, связанные с вводом пользователя.

    \item Проверьте работу программы, используя различные комбинации номеров прицепов и грузов.
\end{enumerate}

\item[18] \textbf{Формирование состава морских контейнеровозов}
\begin{enumerate}
    \item Создайте класс \texttt{Vessel}, который будет представлять собой контейнеровоз с типом контейнера. В конструкторе класса \texttt{Vessel} инициализируйте значения судна и типа контейнера из списков \texttt{NumList} и \texttt{MasList}, которые объявлены как общие атрибуты класса. \texttt{NumList: list[str]} — это список судов (не менее 14): 
    \begin{center}
        \texttt{[''Корабль\_1'', ''Корабль\_2'', \dots, ''Корабль\_14'']}
    \end{center}
    \texttt{MasList: list[str]} — это список типов контейнеров (не менее 4):
    \begin{center}
        \texttt{[''20ft'', ''40ft'', ''Рефрижератор'', ''Открытый'']}
    \end{center}
    Конструктор должен иметь сигнатуру: \texttt{\_\_init\_\_(self) -> None}.

    \item Создайте класс \texttt{VesselFleet}, который будет представлять собой флот контейнеровозов. В конструкторе класса \texttt{VesselFleet} инициализируйте список судов \texttt{self.train: list[Vessel]} длиной 56.

    \item Добавьте метод \texttt{shuffle(self) -> None} в класс \texttt{VesselFleet}, который будет перемешивать суда в списке \texttt{self.train}.

    \item Добавьте метод \texttt{get(self, i: int) -> Vessel}, который будет возвращать $i$-е судно и тип его контейнера из списка \texttt{self.train}.

    \item Создайте экземпляр класса \texttt{VesselFleet} и вызовите метод \texttt{shuffle} для перемешивания судов.

    \item Создайте цикл, который будет запрашивать у пользователя номер судна и выводить информацию о нём.

    \item Повторите шаги 5–6 до тех пор, пока пользователь не выберет все суда или не завершит выбор.

    \item В конце программы выводите сообщение о завершении выбора судов.

    \item Убедитесь, что пользователь вводит корректные номера судов и что программа обрабатывает ошибки, связанные с вводом пользователя.

    \item Проверьте работу программы, используя различные комбинации номеров судов и типов контейнеров.
\end{enumerate}

\item[19] \textbf{Формирование состава банковских сейфов}
\begin{enumerate}
    \item Создайте класс \texttt{Safe}, который будет представлять собой сейф с содержимым. В конструкторе класса \texttt{Safe} инициализируйте значения сейфа и содержимого из списков \texttt{NumList} и \texttt{MasList}, которые объявлены как общие атрибуты класса. \texttt{NumList: list[str]} — это список сейфов (не менее 14): 
    \begin{center}
        \texttt{[''Сейф\_1'', ''Сейф\_2'', \dots, ''Сейф\_14'']}
    \end{center}
    \texttt{MasList: list[str]} — это список типов содержимого (не менее 4):
    \begin{center}
        \texttt{[''Документы'', ''Драгоценности'', ''Деньги'', ''Антиквариат'']}
    \end{center}
    Конструктор должен иметь сигнатуру: \texttt{\_\_init\_\_(self) -> None}.

    \item Создайте класс \texttt{VaultTrain}, который будет представлять собой состав сейфов. В конструкторе класса \texttt{VaultTrain} инициализируйте список сейфов \texttt{self.train: list[Safe]} длиной 56.

    \item Добавьте метод \texttt{shuffle(self) -> None} в класс \texttt{VaultTrain}, который будет перемешивать сейфы в списке \texttt{self.train}.

    \item Добавьте метод \texttt{get(self, i: int) -> Safe}, который будет возвращать $i$-й сейф и его содержимое из списка \texttt{self.train}.

    \item Создайте экземпляр класса \texttt{VaultTrain} и вызовите метод \texttt{shuffle} для перемешивания сейфов.

    \item Создайте цикл, который будет запрашивать у пользователя номер сейфа и выводить информацию о нём.

    \item Повторите шаги 5–6 до тех пор, пока пользователь не выберет все сейфы или не завершит выбор.

    \item В конце программы выводите сообщение о завершении выбора сейфов.

    \item Убедитесь, что пользователь вводит корректные номера сейфов и что программа обрабатывает ошибки, связанные с вводом пользователя.

    \item Проверьте работу программы, используя различные комбинации номеров сейфов и содержимого.
\end{enumerate}

\item[20] \textbf{Формирование состава капсул экспресс-доставки}
\begin{enumerate}
    \item Создайте класс \texttt{Capsule}, который будет представлять собой капсулу с грузом. В конструкторе класса \texttt{Capsule} инициализируйте значения капсулы и груза из списков \texttt{NumList} и \texttt{MasList}, которые объявлены как общие атрибуты класса. \texttt{NumList: list[str]} — это список капсул (не менее 14): 
    \begin{center}
        \texttt{[''Капсула\_1'', ''Капсула\_2'', \dots, ''Капсула\_14'']}
    \end{center}
    \texttt{MasList: list[str]} — это список типов груза (не менее 4):
    \begin{center}
        \texttt{[''Медикаменты'', ''Еда'', ''Посылки'', ''Образцы'']}
    \end{center}
    Конструктор должен иметь сигнатуру: \texttt{\_\_init\_\_(self) -> None}.

    \item Создайте класс \texttt{CapsuleTrain}, который будет представлять собой состав капсул. В конструкторе класса \texttt{CapsuleTrain} инициализируйте список капсул \texttt{self.train: list[Capsule]} длиной 56.

    \item Добавьте метод \texttt{shuffle(self) -> None} в класс \texttt{CapsuleTrain}, который будет перемешивать капсулы в списке \texttt{self.train}.

    \item Добавьте метод \texttt{get(self, i: int) -> Capsule}, который будет возвращать $i$-ю капсулу и её груз из списка \texttt{self.train}.

    \item Создайте экземпляр класса \texttt{CapsuleTrain} и вызовите метод \texttt{shuffle} для перемешивания капсул.

    \item Создайте цикл, который будет запрашивать у пользователя номер капсулы и выводить информацию о ней.

    \item Повторите шаги 5–6 до тех пор, пока пользователь не выберет все капсулы или не завершит выбор.

    \item В конце программы выводите сообщение о завершении выбора капсул.

    \item Убедитесь, что пользователь вводит корректные номера капсул и что программа обрабатывает ошибки, связанные с вводом пользователя.

    \item Проверьте работу программы, используя различные комбинации номеров капсул и грузов.
\end{enumerate}

\item[21] \textbf{Формирование состава тележек в аэропорту}
\begin{enumerate}
    \item Создайте класс \texttt{Trolley}, который будет представлять собой тележку с типом пассажира. В конструкторе класса \texttt{Trolley} инициализируйте значения тележки и типа пассажира из списков \texttt{NumList} и \texttt{MasList}, которые объявлены как общие атрибуты класса. \texttt{NumList: list[str]} — это список тележек (не менее 14): 
    \begin{center}
        \texttt{[''Тележка\_1'', ''Тележка\_2'', \dots, ''Тележка\_14'']}
    \end{center}
    \texttt{MasList: list[str]} — это список типов пассажиров (не менее 4):
    \begin{center}
        \texttt{[''Бизнес'', ''Эконом'', ''Первый класс'', ''Транзит'']}
    \end{center}
    Конструктор должен иметь сигнатуру: \texttt{\_\_init\_\_(self) -> None}.

    \item Создайте класс \texttt{AirportTrolleyTrain}, который будет представлять собой состав тележек. В конструкторе класса \texttt{AirportTrolleyTrain} инициализируйте список тележек \texttt{self.train: list[Trolley]} длиной 56.

    \item Добавьте метод \texttt{shuffle(self) -> None} в класс \texttt{AirportTrolleyTrain}, который будет перемешивать тележки в списке \texttt{self.train}.

    \item Добавьте метод \texttt{get(self, i: int) -> Trolley}, который будет возвращать $i$-ю тележку и тип её пассажира из списка \texttt{self.train}.

    \item Создайте экземпляр класса \texttt{AirportTrolleyTrain} и вызовите метод \texttt{shuffle} для перемешивания тележек.

    \item Создайте цикл, который будет запрашивать у пользователя номер тележки и выводить информацию о ней.

    \item Повторите шаги 5–6 до тех пор, пока пользователь не выберет все тележки или не завершит выбор.

    \item В конце программы выводите сообщение о завершении выбора тележек.

    \item Убедитесь, что пользователь вводит корректные номера тележек и что программа обрабатывает ошибки, связанные с вводом пользователя.

    \item Проверьте работу программы, используя различные комбинации номеров тележек и типов пассажиров.
\end{enumerate}

\item[22] \textbf{Формирование состава мобильных платформ с оборудованием}
\begin{enumerate}
    \item Создайте класс \texttt{Platform}, который будет представлять собой платформу с оборудованием. В конструкторе класса \texttt{Platform} инициализируйте значения платформы и оборудования из списков \texttt{NumList} и \texttt{MasList}, которые объявлены как общие атрибуты класса. \texttt{NumList: list[str]} — это список платформ (не менее 14): 
    \begin{center}
        \texttt{[''Платформа\_1'', ''Платформа\_2'', \dots, ''Платформа\_14'']}
    \end{center}
    \texttt{MasList: list[str]} — это список типов оборудования (не менее 4):
    \begin{center}
        \texttt{[''Генератор'', ''Компрессор'', ''Насос'', ''Сварочный аппарат'']}
    \end{center}
    Конструктор должен иметь сигнатуру: \texttt{\_\_init\_\_(self) -> None}.

    \item Создайте класс \texttt{PlatformTrain}, который будет представлять собой состав платформ. В конструкторе класса \texttt{PlatformTrain} инициализируйте список платформ \texttt{self.train: list[Platform]} длиной 56.

    \item Добавьте метод \texttt{shuffle(self) -> None} в класс \texttt{PlatformTrain}, который будет перемешивать платформы в списке \texttt{self.train}.

    \item Добавьте метод \texttt{get(self, i: int) -> Platform}, который будет возвращать $i$-ю платформу и её оборудование из списка \texttt{self.train}.

    \item Создайте экземпляр класса \texttt{PlatformTrain} и вызовите метод \texttt{shuffle} для перемешивания платформ.

    \item Создайте цикл, который будет запрашивать у пользователя номер платформы и выводить информацию о ней.

    \item Повторите шаги 5–6 до тех пор, пока пользователь не выберет все платформы или не завершит выбор.

    \item В конце программы выводите сообщение о завершении выбора платформ.

    \item Убедитесь, что пользователь вводит корректные номера платформ и что программа обрабатывает ошибки, связанные с вводом пользователя.

    \item Проверьте работу программы, используя различные комбинации номеров платформ и оборудования.
\end{enumerate}

\item[23] \textbf{Формирование состава ящиков с инструментами}
\begin{enumerate}
    \item Создайте класс \texttt{Toolbox}, который будет представлять собой ящик с набором инструментов. В конструкторе класса \texttt{Toolbox} инициализируйте значения ящика и набора из списков \texttt{NumList} и \texttt{MasList}, которые объявлены как общие атрибуты класса. \texttt{NumList: list[str]} — это список ящиков (не менее 14): 
    \begin{center}
        \texttt{[''Ящик\_1'', ''Ящик\_2'', \dots, ''Ящик\_14'']}
    \end{center}
    \texttt{MasList: list[str]} — это список типов наборов (не менее 4):
    \begin{center}
        \texttt{[''Слесарный'', ''Электромонтажный'', ''Столярный'', ''Автомобильный'']}
    \end{center}
    Конструктор должен иметь сигнатуру: \texttt{\_\_init\_\_(self) -> None}.

    \item Создайте класс \texttt{ToolTrain}, который будет представлять собой состав ящиков. В конструкторе класса \texttt{ToolTrain} инициализируйте список ящиков \texttt{self.train: list[Toolbox]} длиной 56.

    \item Добавьте метод \texttt{shuffle(self) -> None} в класс \texttt{ToolTrain}, который будет перемешивать ящики в списке \texttt{self.train}.

    \item Добавьте метод \texttt{get(self, i: int) -> Toolbox}, который будет возвращать $i$-й ящик и его набор инструментов из списка \texttt{self.train}.

    \item Создайте экземпляр класса \texttt{ToolTrain} и вызовите метод \texttt{shuffle} для перемешивания ящиков.

    \item Создайте цикл, который будет запрашивать у пользователя номер ящика и выводить информацию о нём.

    \item Повторите шаги 5–6 до тех пор, пока пользователь не выберет все ящики или не завершит выбор.

    \item В конце программы выводите сообщение о завершении выбора ящиков.

    \item Убедитесь, что пользователь вводит корректные номера ящиков и что программа обрабатывает ошибки, связанные с вводом пользователя.

    \item Проверьте работу программы, используя различные комбинации номеров ящиков и наборов инструментов.
\end{enumerate}

\item[24] \textbf{Формирование состава подводных аппаратов}
\begin{enumerate}
    \item Создайте класс \texttt{Submersible}, который будет представлять собой аппарат с миссией. В конструкторе класса \texttt{Submersible} инициализируйте значения аппарата и миссии из списков \texttt{NumList} и \texttt{MasList}, которые объявлены как общие атрибуты класса. \texttt{NumList: list[str]} — это список аппаратов (не менее 14): 
    \begin{center}
        \texttt{[''Аппарат\_1'', ''Аппарат\_2'', \dots, ''Аппарат\_14'']}
    \end{center}
    \texttt{MasList: list[str]} — это список миссий (не менее 4):
    \begin{center}
        \texttt{[''Исследование'', ''Спасение'', ''Инспекция'', ''Добыча'']}
    \end{center}
    Конструктор должен иметь сигнатуру: \texttt{\_\_init\_\_(self) -> None}.

    \item Создайте класс \texttt{SubmersibleSquadron}, который будет представлять собой эскадрилью аппаратов. В конструкторе класса \texttt{SubmersibleSquadron} инициализируйте список аппаратов \texttt{self.train: list[Submersible]} длиной 56.

    \item Добавьте метод \texttt{shuffle(self) -> None} в класс \texttt{SubmersibleSquadron}, который будет перемешивать аппараты в списке \texttt{self.train}.

    \item Добавьте метод \texttt{get(self, i: int) -> Submersible}, который будет возвращать $i$-й аппарат и его миссию из списка \texttt{self.train}.

    \item Создайте экземпляр класса \texttt{SubmersibleSquadron} и вызовите метод \texttt{shuffle} для перемешивания аппаратов.

    \item Создайте цикл, который будет запрашивать у пользователя номер аппарата и выводить информацию о нём.

    \item Повторите шаги 5–6 до тех пор, пока пользователь не выберет все аппараты или не завершит выбор.

    \item В конце программы выводите сообщение о завершении выбора аппаратов.

    \item Убедитесь, что пользователь вводит корректные номера аппаратов и что программа обрабатывает ошибки, связанные с вводом пользователя.

    \item Проверьте работу программы, используя различные комбинации номеров аппаратов и миссий.
\end{enumerate}

\item[25] \textbf{Формирование состава контейнеров с растениями}
\begin{enumerate}
    \item Создайте класс \texttt{Planter}, который будет представлять собой контейнер с растением. В конструкторе класса \texttt{Planter} инициализируйте значения контейнера и растения из списков \texttt{NumList} и \texttt{MasList}, которые объявлены как общие атрибуты класса. \texttt{NumList: list[str]} — это список контейнеров (не менее 14): 
    \begin{center}
        \texttt{[''Контейнер\_1'', ''Контейнер\_2'', \dots, ''Контейнер\_14'']}
    \end{center}
    \texttt{MasList: list[str]} — это список растений (не менее 4):
    \begin{center}
        \texttt{[''Орхидеи'', ''Кактусы'', ''Пальмы'', ''Бонсаи'']}
    \end{center}
    Конструктор должен иметь сигнатуру: \texttt{\_\_init\_\_(self) -> None}.

    \item Создайте класс \texttt{GreenTrain}, который будет представлять собой состав контейнеров. В конструкторе класса \texttt{GreenTrain} инициализируйте список контейнеров \texttt{self.train: list[Planter]} длиной 56.

    \item Добавьте метод \texttt{shuffle(self) -> None} в класс \texttt{GreenTrain}, который будет перемешивать контейнеры в списке \texttt{self.train}.

    \item Добавьте метод \texttt{get(self, i: int) -> Planter}, который будет возвращать $i$-й контейнер и его растение из списка \texttt{self.train}.

    \item Создайте экземпляр класса \texttt{GreenTrain} и вызовите метод \texttt{shuffle} для перемешивания контейнеров.

    \item Создайте цикл, который будет запрашивать у пользователя номер контейнера и выводить информацию о нём.

    \item Повторите шаги 5–6 до тех пор, пока пользователь не выберет все контейнеры или не завершит выбор.

    \item В конце программы выводите сообщение о завершении выбора контейнеров.

    \item Убедитесь, что пользователь вводит корректные номера контейнеров и что программа обрабатывает ошибки, связанные с вводом пользователя.

    \item Проверьте работу программы, используя различные комбинации номеров контейнеров и растений.
\end{enumerate}

\item[26] \textbf{Формирование состава машин скорой помощи}
\begin{enumerate}
    \item Создайте класс \texttt{Ambulance}, который будет представлять собой машину с типом бригады. В конструкторе класса \texttt{Ambulance} инициализируйте значения машины и типа бригады из списков \texttt{NumList} и \texttt{MasList}, которые объявлены как общие атрибуты класса. \texttt{NumList: list[str]} — это список машин (не менее 14): 
    \begin{center}
        \texttt{[''Машина\_1'', ''Машина\_2'', \dots, ''Машина\_14'']}
    \end{center}
    \texttt{MasList: list[str]} — это список типов бригад (не менее 4):
    \begin{center}
        \texttt{[''Травматологи'', ''Кардиологи'', ''Психиатры'', ''Реаниматологи'']}
    \end{center}
    Конструктор должен иметь сигнатуру: \texttt{\_\_init\_\_(self) -> None}.

    \item Создайте класс \texttt{AmbulanceConvoy}, который будет представлять собой конвой машин. В конструкторе класса \texttt{AmbulanceConvoy} инициализируйте список машин \texttt{self.train: list[Ambulance]} длиной 56.

    \item Добавьте метод \texttt{shuffle(self) -> None} в класс \texttt{AmbulanceConvoy}, который будет перемешивать машины в списке \texttt{self.train}.

    \item Добавьте метод \texttt{get(self, i: int) -> Ambulance}, который будет возвращать $i$-ю машину и её бригаду из списка \texttt{self.train}.

    \item Создайте экземпляр класса \texttt{AmbulanceConvoy} и вызовите метод \texttt{shuffle} для перемешивания машин.

    \item Создайте цикл, который будет запрашивать у пользователя номер машины и выводить информацию о ней.

    \item Повторите шаги 5–6 до тех пор, пока пользователь не выберет все машины или не завершит выбор.

    \item В конце программы выводите сообщение о завершении выбора машин.

    \item Убедитесь, что пользователь вводит корректные номера машин и что программа обрабатывает ошибки, связанные с вводом пользователя.

    \item Проверьте работу программы, используя различные комбинации номеров машин и типов бригад.
\end{enumerate}

\item[27] \textbf{Формирование состава пожарных машин}
\begin{enumerate}
    \item Создайте класс \texttt{FireTruck}, который будет представлять собой пожарную машину со специализацией. В конструкторе класса \texttt{FireTruck} инициализируйте значения машины и специализации из списков \texttt{NumList} и \texttt{MasList}, которые объявлены как общие атрибуты класса. \texttt{NumList: list[str]} — это список машин (не менее 14): 
    \begin{center}
        \texttt{[''Машина\_1'', ''Машина\_2'', \dots, ''Машина\_14'']}
    \end{center}
    \texttt{MasList: list[str]} — это список специализаций (не менее 4):
    \begin{center}
        \texttt{[''Тушение'', ''Спасение'', ''Химзащита'', ''Высотные работы'']}
    \end{center}
    Конструктор должен иметь сигнатуру: \texttt{\_\_init\_\_(self) -> None}.

    \item Создайте класс \texttt{FireTrain}, который будет представлять собой состав пожарных машин. В конструкторе класса \texttt{FireTrain} инициализируйте список машин \texttt{self.train: list[FireTruck]} длиной 56.

    \item Добавьте метод \texttt{shuffle(self) -> None} в класс \texttt{FireTrain}, который будет перемешивать машины в списке \texttt{self.train}.

    \item Добавьте метод \texttt{get(self, i: int) -> FireTruck}, который будет возвращать $i$-ю машину и её специализацию из списка \texttt{self.train}.

    \item Создайте экземпляр класса \texttt{FireTrain} и вызовите метод \texttt{shuffle} для перемешивания машин.

    \item Создайте цикл, который будет запрашивать у пользователя номер машины и выводить информацию о ней.

    \item Повторите шаги 5–6 до тех пор, пока пользователь не выберет все машины или не завершит выбор.

    \item В конце программы выводите сообщение о завершении выбора машин.

    \item Убедитесь, что пользователь вводит корректные номера машин и что программа обрабатывает ошибки, связанные с вводом пользователя.

    \item Проверьте работу программы, используя различные комбинации номеров машин и специализаций.
\end{enumerate}

\item[28] \textbf{Формирование состава эвакуаторов}
\begin{enumerate}
    \item Создайте класс \texttt{TowTruck}, который будет представлять собой эвакуатор с типом транспортного средства. В конструкторе класса \texttt{TowTruck} инициализируйте значения эвакуатора и типа ТС из списков \texttt{NumList} и \texttt{MasList}, которые объявлены как общие атрибуты класса. \texttt{NumList: list[str]} — это список эвакуаторов (не менее 14): 
    \begin{center}
        \texttt{[''Эвакуатор\_1'', ''Эвакуатор\_2'', \dots, ''Эвакуатор\_14'']}
    \end{center}
    \texttt{MasList: list[str]} — это список типов ТС (не менее 4):
    \begin{center}
        \texttt{[''Легковой'', ''Грузовик'', ''Мотоцикл'', ''Автобус'']}
    \end{center}
    Конструктор должен иметь сигнатуру: \texttt{\_\_init\_\_(self) -> None}.

    \item Создайте класс \texttt{TowTrain}, который будет представлять собой состав эвакуаторов. В конструкторе класса \texttt{TowTrain} инициализируйте список эвакуаторов \texttt{self.train: list[TowTruck]} длиной 56.

    \item Добавьте метод \texttt{shuffle(self) -> None} в класс \texttt{TowTrain}, который будет перемешивать эвакуаторы в списке \texttt{self.train}.

    \item Добавьте метод \texttt{get(self, i: int) -> TowTruck}, который будет возвращать $i$-й эвакуатор и тип его ТС из списка \texttt{self.train}.

    \item Создайте экземпляр класса \texttt{TowTrain} и вызовите метод \texttt{shuffle} для перемешивания эвакуаторов.

    \item Создайте цикл, который будет запрашивать у пользователя номер эвакуатора и выводить информацию о нём.

    \item Повторите шаги 5–6 до тех пор, пока пользователь не выберет все эвакуаторы или не завершит выбор.

    \item В конце программы выводите сообщение о завершении выбора эвакуаторов.

    \item Убедитесь, что пользователь вводит корректные номера эвакуаторов и что программа обрабатывает ошибки, связанные с вводом пользователя.

    \item Проверьте работу программы, используя различные комбинации номеров эвакуаторов и типов ТС.
\end{enumerate}

\item[29] \textbf{Формирование состава контейнеров с лекарствами}
\begin{enumerate}
    \item Создайте класс \texttt{MedBox}, который будет представлять собой контейнер с типом лекарств. В конструкторе класса \texttt{MedBox} инициализируйте значения контейнера и типа лекарств из списков \texttt{NumList} и \texttt{MasList}, которые объявлены как общие атрибуты класса. \texttt{NumList: list[str]} — это список контейнеров (не менее 14): 
    \begin{center}
        \texttt{[''Контейнер\_1'', ''Контейнер\_2'', \dots, ''Контейнер\_14'']}
    \end{center}
    \texttt{MasList: list[str]} — это список типов лекарств (не менее 4):
    \begin{center}
        \texttt{[''Антибиотики'', ''Вакцины'', ''Обезболивающие'', ''Витамины'']}
    \end{center}
    Конструктор должен иметь сигнатуру: \texttt{\_\_init\_\_(self) -> None}.

    \item Создайте класс \texttt{MedTrain}, который будет представлять собой состав контейнеров. В конструкторе класса \texttt{MedTrain} инициализируйте список контейнеров \texttt{self.train: list[MedBox]} длиной 56.

    \item Добавьте метод \texttt{shuffle(self) -> None} в класс \texttt{MedTrain}, который будет перемешивать контейнеры в списке \texttt{self.train}.

    \item Добавьте метод \texttt{get(self, i: int) -> MedBox}, который будет возвращать $i$-й контейнер и его лекарства из списка \texttt{self.train}.

    \item Создайте экземпляр класса \texttt{MedTrain} и вызовите метод \texttt{shuffle} для перемешивания контейнеров.

    \item Создайте цикл, который будет запрашивать у пользователя номер контейнера и выводить информацию о нём.

    \item Повторите шаги 5–6 до тех пор, пока пользователь не выберет все контейнеры или не завершит выбор.

    \item В конце программы выводите сообщение о завершении выбора контейнеров.

    \item Убедитесь, что пользователь вводит корректные номера контейнеров и что программа обрабатывает ошибки, связанные с вводом пользователя.

    \item Проверьте работу программы, используя различные комбинации номеров контейнеров и типов лекарств.
\end{enumerate}

\item[30] \textbf{Формирование состава транспорта с опасными грузами}
\begin{enumerate}
    \item Создайте класс \texttt{HazmatTruck}, который будет представлять собой грузовик с классом опасности. В конструкторе класса \texttt{HazmatTruck} инициализируйте значения грузовика и класса опасности из списков \texttt{NumList} и \texttt{MasList}, которые объявлены как общие атрибуты класса. \texttt{NumList: list[str]} — это список грузовиков (не менее 14): 
    \begin{center}
        \texttt{[''Грузовик\_1'', ''Грузовик\_2'', \dots, ''Грузовик\_14'']}
    \end{center}
    \texttt{MasList: list[str]} — это список классов опасности (не менее 4):
    \begin{center}
        \texttt{[''Взрывчатка'', ''Газы'', ''Легковоспламеняющиеся'', ''Токсичные'']}
    \end{center}
    Конструктор должен иметь сигнатуру: \texttt{\_\_init\_\_(self) -> None}.

    \item Создайте класс \texttt{HazmatConvoy}, который будет представлять собой конвой грузовиков. В конструкторе класса \texttt{HazmatConvoy} инициализируйте список грузовиков \texttt{self.train: list[HazmatTruck]} длиной 56.

    \item Добавьте метод \texttt{shuffle(self) -> None} в класс \texttt{HazmatConvoy}, который будет перемешивать грузовики в списке \texttt{self.train}.

    \item Добавьте метод \texttt{get(self, i: int) -> HazmatTruck}, который будет возвращать $i$-й грузовик и его класс опасности из списка \texttt{self.train}.

    \item Создайте экземпляр класса \texttt{HazmatConvoy} и вызовите метод \texttt{shuffle} для перемешивания грузовиков.

    \item Создайте цикл, который будет запрашивать у пользователя номер грузовика и выводить информацию о нём.

    \item Повторите шаги 5–6 до тех пор, пока пользователь не выберет все грузовики или не завершит выбор.

    \item В конце программы выводите сообщение о завершении выбора грузовиков.

    \item Убедитесь, что пользователь вводит корректные номера грузовиков и что программа обрабатывает ошибки, связанные с вводом пользователя.

    \item Проверьте работу программы, используя различные комбинации номеров грузовиков и классов опасности.
\end{enumerate}

\item[31] \textbf{Формирование состава курьерских пакетов}
\begin{enumerate}
    \item Создайте класс \texttt{Package}, который будет представлять собой пакет с типом доставки. В конструкторе класса \texttt{Package} инициализируйте значения пакета и типа доставки из списков \texttt{NumList} и \texttt{MasList}, которые объявлены как общие атрибуты класса. \texttt{NumList: list[str]} — это список пакетов (не менее 14): 
    \begin{center}
        \texttt{[''Пакет\_1'', ''Пакет\_2'', \dots, ''Пакет\_14'']}
    \end{center}
    \texttt{MasList: list[str]} — это список типов доставки (не менее 4):
    \begin{center}
        \texttt{[''Экспресс'', ''Стандарт'', ''Международный'', ''Хрупкий'']}
    \end{center}
    Конструктор должен иметь сигнатуру: \texttt{\_\_init\_\_(self) -> None}.

    \item Создайте класс \texttt{PackageTrain}, который будет представлять собой состав пакетов. В конструкторе класса \texttt{PackageTrain} инициализируйте список пакетов \texttt{self.train: list[Package]} длиной 56.

    \item Добавьте метод \texttt{shuffle(self) -> None} в класс \texttt{PackageTrain}, который будет перемешивать пакеты в списке \texttt{self.train}.

    \item Добавьте метод \texttt{get(self, i: int) -> Package}, который будет возвращать $i$-й пакет и его тип доставки из списка \texttt{self.train}.

    \item Создайте экземпляр класса \texttt{PackageTrain} и вызовите метод \texttt{shuffle} для перемешивания пакетов.

    \item Создайте цикл, который будет запрашивать у пользователя номер пакета и выводить информацию о нём.

    \item Повторите шаги 5–6 до тех пор, пока пользователь не выберет все пакеты или не завершит выбор.

    \item В конце программы выводите сообщение о завершении выбора пакетов.

    \item Убедитесь, что пользователь вводит корректные номера пакетов и что программа обрабатывает ошибки, связанные с вводом пользователя.

    \item Проверьте работу программы, используя различные комбинации номеров пакетов и типов доставки.
\end{enumerate}

\item[32] \textbf{Формирование состава мобильных медицинских лабораторий}
\begin{enumerate}
    \item Создайте класс \texttt{LabVan}, который будет представлять собой лабораторию с типом анализа. В конструкторе класса \texttt{LabVan} инициализируйте значения лаборатории и типа анализа из списков \texttt{NumList} и \texttt{MasList}, которые объявлены как общие атрибуты класса. \texttt{NumList: list[str]} — это список лабораторий (не менее 14): 
    \begin{center}
        \texttt{[''Лаборатория\_1'', ''Лаборатория\_2'', \dots, ''Лаборатория\_14'']}
    \end{center}
    \texttt{MasList: list[str]} — это список типов анализов (не менее 4):
    \begin{center}
        \texttt{[''PCR'', ''Анализы крови'', ''Токсикология'', ''Микробиология'']}
    \end{center}
    Конструктор должен иметь сигнатуру: \texttt{\_\_init\_\_(self) -> None}.

    \item Создайте класс \texttt{LabConvoy}, который будет представлять собой конвой лабораторий. В конструкторе класса \texttt{LabConvoy} инициализируйте список лабораторий \texttt{self.train: list[LabVan]} длиной 56.

    \item Добавьте метод \texttt{shuffle(self) -> None} в класс \texttt{LabConvoy}, который будет перемешивать лаборатории в списке \texttt{self.train}.

    \item Добавьте метод \texttt{get(self, i: int) -> LabVan}, который будет возвращать $i$-ю лабораторию и её тип анализа из списка \texttt{self.train}.

    \item Создайте экземпляр класса \texttt{LabConvoy} и вызовите метод \texttt{shuffle} для перемешивания лабораторий.

    \item Создайте цикл, который будет запрашивать у пользователя номер лаборатории и выводить информацию о ней.

    \item Повторите шаги 5–6 до тех пор, пока пользователь не выберет все лаборатории или не завершит выбор.

    \item В конце программы выводите сообщение о завершении выбора лабораторий.

    \item Убедитесь, что пользователь вводит корректные номера лабораторий и что программа обрабатывает ошибки, связанные с вводом пользователя.

    \item Проверьте работу программы, используя различные комбинации номеров лабораторий и типов анализов.
\end{enumerate}

\item[33] \textbf{Формирование состава контейнеров с артефактами}
\begin{enumerate}
    \item Создайте класс \texttt{ArtifactCase}, который будет представлять собой кейс с происхождением артефакта. В конструкторе класса \texttt{ArtifactCase} инициализируйте значения кейса и происхождения из списков \texttt{NumList} и \texttt{MasList}, которые объявлены как общие атрибуты класса. \texttt{NumList: list[str]} — это список кейсов (не менее 14): 
    \begin{center}
        \texttt{[''Кейс\_1'', ''Кейс\_2'', \dots, ''Кейс\_14'']}
    \end{center}
    \texttt{MasList: list[str]} — это список происхождений (не менее 4):
    \begin{center}
        \texttt{[''Египет'', ''Греция'', ''Мезоамерика'', ''Древний Китай'']}
    \end{center}
    Конструктор должен иметь сигнатуру: \texttt{\_\_init\_\_(self) -> None}.

    \item Создайте класс \texttt{ArtifactTrain}, который будет представлять собой состав кейсов. В конструкторе класса \texttt{ArtifactTrain} инициализируйте список кейсов \texttt{self.train: list[ArtifactCase]} длиной 56.

    \item Добавьте метод \texttt{shuffle(self) -> None} в класс \texttt{ArtifactTrain}, который будет перемешивать кейсы в списке \texttt{self.train}.

    \item Добавьте метод \texttt{get(self, i: int) -> ArtifactCase}, который будет возвращать $i$-й кейс и происхождение его артефакта из списка \texttt{self.train}.

    \item Создайте экземпляр класса \texttt{ArtifactTrain} и вызовите метод \texttt{shuffle} для перемешивания кейсов.

    \item Создайте цикл, который будет запрашивать у пользователя номер кейса и выводить информацию о нём.

    \item Повторите шаги 5–6 до тех пор, пока пользователь не выберет все кейсы или не завершит выбор.

    \item В конце программы выводите сообщение о завершении выбора кейсов.

    \item Убедитесь, что пользователь вводит корректные номера кейсов и что программа обрабатывает ошибки, связанные с вводом пользователя.

    \item Проверьте работу программы, используя различные комбинации номеров кейсов и происхождений.
\end{enumerate}

\item[34] \textbf{Формирование состава беспилотных грузовиков}
\begin{enumerate}
    \item Создайте класс \texttt{AutonomousTruck}, который будет представлять собой грузовик с типом маршрута. В конструкторе класса \texttt{AutonomousTruck} инициализируйте значения грузовика и маршрута из списков \texttt{NumList} и \texttt{MasList}, которые объявлены как общие атрибуты класса. \texttt{NumList: list[str]} — это список грузовиков (не менее 14): 
    \begin{center}
        \texttt{[''Грузовик\_1'', ''Грузовик\_2'', \dots, ''Грузовик\_14'']}
    \end{center}
    \texttt{MasList: list[str]} — это список типов маршрутов (не менее 4):
    \begin{center}
        \texttt{[''Город'', ''Шоссе'', ''Горы'', ''Пустыня'']}
    \end{center}
    Конструктор должен иметь сигнатуру: \texttt{\_\_init\_\_(self) -> None}.

    \item Создайте класс \texttt{TruckConvoy}, который будет представлять собой конвой грузовиков. В конструкторе класса \texttt{TruckConvoy} инициализируйте список грузовиков \texttt{self.train: list[AutonomousTruck]} длиной 56.

    \item Добавьте метод \texttt{shuffle(self) -> None} в класс \texttt{TruckConvoy}, который будет перемешивать грузовики в списке \texttt{self.train}.

    \item Добавьте метод \texttt{get(self, i: int) -> AutonomousTruck}, который будет возвращать $i$-й грузовик и его маршрут из списка \texttt{self.train}.

    \item Создайте экземпляр класса \texttt{TruckConvoy} и вызовите метод \texttt{shuffle} для перемешивания грузовиков.

    \item Создайте цикл, который будет запрашивать у пользователя номер грузовика и выводить информацию о нём.

    \item Повторите шаги 5–6 до тех пор, пока пользователь не выберет все грузовики или не завершит выбор.

    \item В конце программы выводите сообщение о завершении выбора грузовиков.

    \item Убедитесь, что пользователь вводит корректные номера грузовиков и что программа обрабатывает ошибки, связанные с вводом пользователя.

    \item Проверьте работу программы, используя различные комбинации номеров грузовиков и маршрутов.
\end{enumerate}

\item[35] \textbf{Формирование состава грузовых вагонов} (оригинальный вариант)
\begin{enumerate}
    \item Создайте класс \texttt{Vagon}, который будет представлять собой вагон с грузом. В конструкторе класса \texttt{Vagon} инициализируйте значения вагона и груза из списков \texttt{NumList} и \texttt{MasList}, которые объявлены как общие атрибуты класса. \texttt{NumList: list[str]} — это список крытых вагонов (не менее 14): 
    \begin{center}
        \texttt{[''Вагон\_1'', ''Вагон\_2'', \dots, ''Вагон\_14'']}
    \end{center}
    \texttt{MasList: list[str]} — это список грузов для крытых вагонов (не менее 4):
    \begin{center}
        \texttt{[''Станки'', ''Автозапчасти'', ''Бумага'', ''Керамическая плитка'']}
    \end{center}
    Конструктор должен иметь сигнатуру: \texttt{\_\_init\_\_(self) -> None}.

    \item Создайте класс \texttt{TrainOfVagons}, который будет представлять собой грузовой поезд, состоящий из моделей вагонов. В конструкторе класса \texttt{TrainOfVagons} инициализируйте список вагонов \texttt{self.train: list[Vagon]} длиной 56.

    \item Добавьте метод \texttt{shuffle(self) -> None} в класс \texttt{TrainOfVagons}, который будет перемешивать вагоны в списке \texttt{self.train}.

    \item Добавьте метод \texttt{get(self, i: int) -> Vagon}, который будет возвращать $i$-й вагон и груз из списка \texttt{self.train}.

    \item Создайте экземпляр класса \texttt{TrainOfVagons} и вызовите метод \texttt{shuffle} для перемешивания вагонов.

    \item Создайте цикл, который будет запрашивать у пользователя номер вагона из поезда и выводить информацию о выбранном вагоне.

    \item Повторите шаги 5–6 до тех пор, пока пользователь не выберет все вагоны или не завершит выбор.

    \item В конце программы выводите сообщение о завершении выбора вагонов.

    \item Убедитесь, что пользователь вводит корректные номера вагонов и что программа обрабатывает ошибки, связанные с вводом пользователя.

    \item Проверьте работу программы, используя различные комбинации номеров вагонов и грузов.
\end{enumerate}

\end{enumerate}