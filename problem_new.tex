\subsection{Семинар <<Структуры данных (закрепление) и \texttt{\_\_new\_\_}>>  
(2 часа)}

При создании подкласса неизменяемого встроенного типа данных 
(например, \texttt{float}, \texttt{str}, \texttt{int}) 
возникает проблема: значение объекта 
устанавливается \textit{в момент его создания}, 
и метод \texttt{\_\_init\_\_} вызывается уже 
\textit{после} этого, когда изменить базовое значение невозможно.

Кроме того, конструктор родительского неизменяемого типа 
(например, \texttt{float.\_\_new\_\_()}) часто не принимает 
дополнительные аргументы так же гибко, 
как \texttt{object.\_\_new\_\_()}, что приводит к ошибкам.

\textbf{Решение:} Использовать метод \texttt{\_\_new\_\_} для 
инициализации объекта \textit{в момент его создания}.

\begin{lstlisting}[language=Python, caption=Пример: Класс Distance с использованием \_\_new\_\_]
class Distance(float):
    def __new__(cls, value, unit):
        # 1. Создаем новый экземпляр float с заданным значением
        instance = super().__new__(cls, value)
        # 2. Настраиваем экземпляр, добавляя изменяемый атрибут
        instance.unit = unit
        # 3. Возвращаем настроенный экземпляр
        return instance

# Использование:
d = Distance(10.5, "km")
print(d)      # 10.5
print(d.unit) # km
d.unit = "m"  # Атрибут unit изменяем!
print(d.unit) # m
\end{lstlisting}

В этом примере \texttt{\_\_new\_\_} выполняет три шага:
\begin{enumerate}
    \item Создает новый экземпляр текущего класса \texttt{cls}, вызывая \texttt{super().\_\_new\_\_(cls, value)}. Это обращение к \texttt{float.\_\_new\_\_()}, который создает и инициализирует новый экземпляр \texttt{float}.
    \item Настраивает новый экземпляр, добавляя к нему изменяемый атрибут \texttt{unit}.
    \item Возвращает новый, настроенный экземпляр.
\end{enumerate}

Теперь класс \texttt{Distance} работает корректно, позволяя хранить единицы измерения в изменяемом атрибуте \texttt{unit}.

\textbf{Замечение}: для упрщения мы не применяли свойство ООП 
\textit{инкапсуляция} 
в примере.

\subsubsection {Задача 1 (Singleton)}

Реализуйте задание согласно своему варианту. Обратите внимание, что мы
не реализуем логику работы сложных вещей, а только её имитируем во всех 
вариантах. 

\textbf{Замечание}: Singleton -- это антипаттерн, в production его использовать
не стоит, но для учебных целей он хорош и, кроме того, знание его
сущности обязательно для разработчика.

\begin{enumerate}

\item Написать программу на Python, которая создает класс `DataBase` с использованием метода `\_\_new\_\_` для реализации паттерна Singleton (один экземпляр). Программа должна принимать параметры при создании и выводить сообщение при подключении.

Инструкции:
\begin{enumerate}
    \item Создайте класс `DataBase`.
    \item Добавьте приватный атрибут класса `\_instance` и инициализируйте его значением `None`.
    \item Переопределите метод `\_\_new\_\_`, чтобы он проверял, существует ли уже экземпляр. Если нет — создает новый с помощью `super().\_\_new\_\_(cls)` и присваивает его `\_instance`. Возвращает `\_instance`.
    \item Переопределите метод `\_\_init\_\_`, принимающий `user`, `psw`, `port`. Устанавливает эти атрибуты экземпляра, но только если они еще не были установлены (чтобы не перезаписывать при повторном "создании").
    \item Добавьте метод `connect`, который выводит сообщение "Подключение к БД: \{user\}, \{psw\}, \{port\}".
    \item Добавьте метод `\_\_del\_\_`, который выводит "Закрытие соединения с БД".
    \item Добавьте метод `get\_data`, который возвращает строку "Данные получены".
    \item Добавьте метод `set\_data`, который принимает `data` и выводит "Данные '\{data\}' записаны".
    \item Создайте два экземпляра `db1` и `db2` с разными параметрами.
    \item Вызовите `connect` для `db1`, затем для `db2`.
    \item Выведите `id(db1)` и `id(db2)` — они должны совпадать.
\end{enumerate}

Пример использования:
\begin{lstlisting}[language=Python]
db1 = DataBase("admin", "secret", 5432)
db2 = DataBase("user", "12345", 3306)  # Это тот же объект, что и db1!

db1.connect()
db2.connect()  # Выведет те же параметры, что и db1

print("ID db1:", id(db1))
print("ID db2:", id(db2))  # ID будут одинаковыми
\end{lstlisting}

\item Написать программу на Python, которая создает класс `ConnectionManager` с использованием метода `\_\_new\_\_` для реализации паттерна Singleton. Программа должна принимать параметры `host`, `username`, `timeout` при создании экземпляра.

Инструкции:
\begin{enumerate}
    \item Создайте класс `ConnectionManager`.
    \item Добавьте приватный атрибут класса `\_shared\_instance` и инициализируйте его значением `None`.
    \item Переопределите метод `\_\_new\_\_`, чтобы он возвращал существующий экземпляр, если он есть, или создавал новый.
    \item Переопределите метод `\_\_init\_\_`, принимающий `host`, `username`, `timeout`. Устанавливает атрибуты, только если они еще не заданы.
    \item Добавьте метод `establish`, который выводит "Соединение установлено с \{host\} под пользователем \{username\} (таймаут: \{timeout\})".
    \item Добавьте метод `\_\_del\_\_`, который выводит "Соединение разорвано".
    \item Добавьте метод `fetch`, который возвращает "Запрос выполнен".
    \item Добавьте метод `commit`, который принимает `transaction` и выводит "Транзакция '\{transaction\}' зафиксирована".
    \item Создайте два экземпляра `cm1` и `cm2` с разными параметрами.
    \item Вызовите `establish` для `cm1`, затем для `cm2`.
    \item Выведите `cm1 is cm2` — должно быть `True`.
\end{enumerate}

Пример использования:
\begin{lstlisting}[language=Python]
cm1 = ConnectionManager("localhost", "root", 30)
cm2 = ConnectionManager("remote.server", "guest", 60)

cm1.establish()
cm2.establish()  # Параметры будут от cm1

print("cm1 is cm2:", cm1 is cm2)  # True
\end{lstlisting}

\item Написать программу на Python, которая создает класс `ConfigLoader` с использованием метода `\_\_new\_\_` для реализации паттерна Singleton. Программа должна принимать параметры `config\_file`, `env`, `debug` при создании экземпляра.

Инструкции:
\begin{enumerate}
    \item Создайте класс `ConfigLoader`.
    \item Добавьте приватный атрибут класса `\_instance\_ref` и инициализируйте его значением `None`.
    \item Переопределите метод `\_\_new\_\_`, чтобы он обеспечивал единственный экземпляр.
    \item Переопределите метод `\_\_init\_\_`, принимающий `config\_file`, `env`, `debug`. Устанавливает атрибуты, только если они еще не заданы.
    \item Добавьте метод `load`, который выводит "Конфигурация загружена из '\{config\_file\}' для среды '\{env\}' (debug=\{debug\})".
    \item Добавьте метод `\_\_del\_\_`, который выводит "Конфигурация выгружена".
    \item Добавьте метод `get\_setting`, который принимает `key` и возвращает "Значение для \{key\}".
    \item Добавьте метод `set\_setting`, который принимает `key`, `value` и выводит "Настройка \{key\} установлена в \{value\}".
    \item Создайте два экземпляра `cfg1` и `cfg2` с разными параметрами.
    \item Вызовите `load` для `cfg1`, затем для `cfg2`.
    \item Проверьте, что `cfg1.debug == cfg2.debug` (должно быть `True`, если первый был создан с `debug=True`).
\end{enumerate}

Пример использования:
\begin{lstlisting}[language=Python]
cfg1 = ConfigLoader("app.yaml", "prod", True)
cfg2 = ConfigLoader("dev.yaml", "dev", False)

cfg1.load()
cfg2.load()  # Параметры будут от cfg1

print("Debug mode (cfg1):", cfg1.debug)
print("Debug mode (cfg2):", cfg2.debug)  # Будет True, как у cfg1
\end{lstlisting}

\item Написать программу на Python, которая создает класс `Logger` с использованием метода `\_\_new\_\_` для реализации паттерна Singleton. Программа должна принимать параметры `log\_level`, `output\_file`, `rotate` при создании экземпляра.

Инструкции:
\begin{enumerate}
    \item Создайте класс `Logger`.
    \item Добавьте приватный атрибут класса `\_the\_logger` и инициализируйте его значением `None`.
    \item Переопределите метод `\_\_new\_\_`, чтобы он возвращал единственный экземпляр.
    \item Переопределите метод `\_\_init\_\_`, принимающий `log\_level`, `output\_file`, `rotate`. Устанавливает атрибуты, только если они еще не заданы.
    \item Добавьте метод `log`, который принимает `message` и выводит "[\{log\_level\}] \{message\} -> \{output\_file\}".
    \item Добавьте метод `\_\_del\_\_`, который выводит "Логгер остановлен".
    \item Добавьте метод `set\_level`, который принимает `level` и устанавливает `self.log\_level = level`.
    \item Добавьте метод `flush`, который выводит "Буфер логов сброшен".
    \item Создайте два экземпляра `log1` и `log2` с разными параметрами.
    \item Вызовите `log` для `log1`, затем `set\_level("ERROR")` для `log2`.
    \item Вызовите `log` для `log1` снова — уровень должен измениться.
\end{enumerate}

Пример использования:
\begin{lstlisting}[language=Python]
log1 = Logger("INFO", "app.log", True)
log2 = Logger("DEBUG", "debug.log", False)

log1.log("Старт приложения")
log2.set\_level("ERROR")  # Меняет уровень для log1 тоже!
log1.log("Ошибка!")  # Выведет [ERROR] Ошибка! -> app.log
\end{lstlisting}

\item Написать программу на Python, которая создает класс `Cache` с использованием метода `\_\_new\_\_` для реализации паттерна Singleton. Программа должна принимать параметры `max\_size`, `ttl`, `strategy` при создании экземпляра.

Инструкции:
\begin{enumerate}
    \item Создайте класс `Cache`.
    \item Добавьте приватный атрибут класса `\_cache\_instance` и инициализируйте его значением `None`.
    \item Переопределите метод `\_\_new\_\_`, чтобы он обеспечивал единственный экземпляр.
    \item Переопределите метод `\_\_init\_\_`, принимающий `max\_size`, `ttl`, `strategy`. Устанавливает атрибуты, только если они еще не заданы.
    \item Добавьте метод `put`, который принимает `key`, `value` и выводит "Ключ '\{key\}' закеширован (стратегия: \{strategy\})".
    \item Добавьте метод `\_\_del\_\_`, который выводит "Кеш очищен".
    \item Добавьте метод `get`, который принимает `key` и возвращает "Значение для \{key\}".
    \item Добавьте метод `clear`, который выводит "Кеш принудительно очищен".
    \item Создайте два экземпляра `cache1` и `cache2` с разными параметрами.
    \item Вызовите `put` для `cache1`, затем `clear` для `cache2`.
    \item Проверьте, что `cache1.max\_size == cache2.max\_size`.
\end{enumerate}

Пример использования:
\begin{lstlisting}[language=Python]
cache1 = Cache(1000, 3600, "LRU")
cache2 = Cache(500, 1800, "FIFO")

cache1.put("user\_123", \{"name": "Alice"\})
cache2.clear()  # Очищает кеш cache1 тоже

print("Max size:", cache1.max\_size)  # 1000 (от первого вызова)
\end{lstlisting}

\item Написать программу на Python, которая создает класс `SessionHandler` с использованием метода `\_\_new\_\_` для реализации паттерна Singleton. Программа должна принимать параметры `session\_id`, `timeout`, `secure` при создании экземпляра.

Инструкции:
\begin{enumerate}
    \item Создайте класс `SessionHandler`.
    \item Добавьте приватный атрибут класса `\_handler` и инициализируйте его значением `None`.
    \item Переопределите метод `\_\_new\_\_`, чтобы он возвращал единственный экземпляр.
    \item Переопределите метод `\_\_init\_\_`, принимающий `session\_id`, `timeout`, `secure`. Устанавливает атрибуты, только если они еще не заданы.
    \item Добавьте метод `start`, который выводит "Сессия \{session\_id\} начата (timeout=\{timeout\}, secure=\{secure\})".
    \item Добавьте метод `\_\_del\_\_`, который выводит "Сессия завершена".
    \item Добавьте метод `get\_session\_data`, который возвращает "Данные сессии".
    \item Добавьте метод `invalidate`, который выводит "Сессия аннулирована".
    \item Создайте два экземпляра `sh1` и `sh2` с разными параметрами.
    \item Вызовите `start` для `sh1`, затем `invalidate` для `sh2`.
    \item Выведите `sh1.session\_id` и `sh2.session\_id` — они должны быть одинаковыми.
\end{enumerate}

Пример использования:
\begin{lstlisting}[language=Python]
sh1 = SessionHandler("SID-001", 1800, True)
sh2 = SessionHandler("SID-999", 600, False)

sh1.start()
sh2.invalidate()  # Аннулирует сессию sh1

print("Session ID sh1:", sh1.session\_id)  # SID-001
print("Session ID sh2:", sh2.session\_id)  # SID-001
\end{lstlisting}

\item Написать программу на Python, которая создает класс `ResourceManager` с использованием метода `\_\_new\_\_` для реализации паттерна Singleton. Программа должна принимать параметры `resource\_type`, `capacity`, `priority` при создании экземпляра.

Инструкции:
\begin{enumerate}
    \item Создайте класс `ResourceManager`.
    \item Добавьте приватный атрибут класса `\_manager` и инициализируйте его значением `None`.
    \item Переопределите метод `\_\_new\_\_`, чтобы он возвращал единственный экземпляр.
    \item Переопределите метод `\_\_init\_\_`, принимающий `resource\_type`, `capacity`, `priority`. Устанавливает атрибуты, только если они еще не заданы.
    \item Добавьте метод `allocate`, который выводит "Выделено \{capacity\} ресурсов типа \{resource\_type\} (приоритет: \{priority\})".
    \item Добавьте метод `\_\_del\_\_`, который выводит "Освобождение ресурсов".
    \item Добавьте метод `status`, который возвращает "Ресурсы доступны".
    \item Добавьте метод `release`, который выводит "Ресурсы освобождены".
    \item Создайте два экземпляра `rm1` и `rm2` с разными параметрами.
    \item Вызовите `allocate` для `rm1`, затем `release` для `rm2`.
    \item Проверьте, что `rm1.capacity == rm2.capacity`.
\end{enumerate}

Пример использования:
\begin{lstlisting}[language=Python]
rm1 = ResourceManager("CPU", 4, 1)
rm2 = ResourceManager("GPU", 2, 2)

rm1.allocate()
rm2.release()  # Освобождает ресурсы rm1

print("Capacity:", rm1.capacity)  # 4
\end{lstlisting}

\item Написать программу на Python, которая создает класс `PrinterPool` с использованием метода `\_\_new\_\_` для реализации паттерна Singleton. Программа должна принимать параметры `printer\_id`, `speed`, `color` при создании экземпляра.

Инструкции:
\begin{enumerate}
    \item Создайте класс `PrinterPool`.
    \item Добавьте приватный атрибут класса `\_pool` и инициализируйте его значением `None`.
    \item Переопределите метод `\_\_new\_\_`, чтобы он возвращал единственный экземпляр.
    \item Переопределите метод `\_\_init\_\_`, принимающий `printer\_id`, `speed`, `color`. Устанавливает атрибуты, только если они еще не заданы.
    \item Добавьте метод `print`, который выводит "Печать документа на принтере \{printer\_id\} (скорость: \{speed\}, цвет: \{color\})".
    \item Добавьте метод `\_\_del\_\_`, который выводит "Принтер выключен".
    \item Добавьте метод `get\_status`, который возвращает "Готов к печати".
    \item Добавьте метод `add\_job`, который принимает `job` и выводит "Добавлено задание: \{job\}".
    \item Создайте два экземпляра `pp1` и `pp2` с разными параметрами.
    \item Вызовите `print` для `pp1`, затем `add\_job` для `pp2`.
    \item Проверьте, что `pp1.speed == pp2.speed`.
\end{enumerate}

Пример использования:
\begin{lstlisting}[language=Python]
pp1 = PrinterPool("P100", 10, "Yes")
pp2 = PrinterPool("P200", 8, "No")

pp1.print()
pp2.add\_job("Report.pdf")  # Добавляет задание для pp1

print("Speed:", pp1.speed)  # 10
\end{lstlisting}

\item Написать программу на Python, которая создает класс `NetworkInterface` с использованием метода `\_\_new\_\_` для реализации паттерна Singleton. Программа должна принимать параметры `interface\_name`, `ip`, `mac` при создании экземпляра.

Инструкции:
\begin{enumerate}
    \item Создайте класс `NetworkInterface`.
    \item Добавьте приватный атрибут класса `\_interface` и инициализируйте его значением `None`.
    \item Переопределите метод `\_\_new\_\_`, чтобы он возвращал единственный экземпляр.
    \item Переопределите метод `\_\_init\_\_`, принимающий `interface\_name`, `ip`, `mac`. Устанавливает атрибуты, только если они еще не заданы.
    \item Добавьте метод `connect`, который выводит "Подключение к сети через \{interface\_name\} (\{ip\}, \{mac\})".
    \item Добавьте метод `\_\_del\_\_`, который выводит "Отключение от сети".
    \item Добавьте метод `ping`, который возвращает "Пинг успешен".
    \item Добавьте метод `configure`, который принимает `new\_ip` и выводит "IP изменен на \{new\_ip\}".
    \item Создайте два экземпляра `ni1` и `ni2` с разными параметрами.
    \item Вызовите `connect` для `ni1`, затем `configure` для `ni2`.
    \item Проверьте, что `ni1.ip == ni2.ip`.
\end{enumerate}

Пример использования:
\begin{lstlisting}[language=Python]
ni1 = NetworkInterface("eth0", "192.168.1.100", "AA:BB:CC:DD:EE:FF")
ni2 = NetworkInterface("wlan0", "192.168.1.101", "11:22:33:44:55:66")

ni1.connect()
ni2.configure("192.168.1.102")  # Изменяет IP для ni1

print("IP:", ni1.ip)  # 192.168.1.102
\end{lstlisting}

\item Написать программу на Python, которая создает класс `FileManager` с использованием метода `\_\_new\_\_` для реализации паттерна Singleton. Программа должна принимать параметры `path`, `mode`, `buffered` при создании экземпляра.

Инструкции:
\begin{enumerate}
    \item Создайте класс `FileManager`.
    \item Добавьте приватный атрибут класса `\_manager` и инициализируйте его значением `None`.
    \item Переопределите метод `\_\_new\_\_`, чтобы он возвращал единственный экземпляр.
    \item Переопределите метод `\_\_init\_\_`, принимающий `path`, `mode`, `buffered`. Устанавливает атрибуты, только если они еще не заданы.
    \item Добавьте метод `open`, который выводит "Открытие файла '\{path\}' в режиме '\{mode\}' (буферизация: \{buffered\})".
    \item Добавьте метод `\_\_del\_\_`, который выводит "Файл закрыт".
    \item Добавьте метод `read`, который возвращает "Данные прочитаны".
    \item Добавьте метод `write`, который принимает `data` и выводит "Данные '\{data\}' записаны".
    \item Создайте два экземпляра `fm1` и `fm2` с разными параметрами.
    \item Вызовите `open` для `fm1`, затем `write` для `fm2`.
    \item Проверьте, что `fm1.mode == fm2.mode`.
\end{enumerate}

Пример использования:
\begin{lstlisting}[language=Python]
fm1 = FileManager("data.txt", "r", True)
fm2 = FileManager("log.txt", "w", False)

fm1.open()
fm2.write("Hello")  # Записывает в файл fm1

print("Mode:", fm1.mode)  # r
\end{lstlisting}

\item Написать программу на Python, которая создает класс `DatabaseConnector` с использованием метода `\_\_new\_\_` для реализации паттерна Singleton. Программа должна принимать параметры `dbname`, `host`, `port` при создании экземпляра.

Инструкции:
\begin{enumerate}
    \item Создайте класс `DatabaseConnector`.
    \item Добавьте приватный атрибут класса `\_connector` и инициализируйте его значением `None`.
    \item Переопределите метод `\_\_new\_\_`, чтобы он возвращал единственный экземпляр.
    \item Переопределите метод `\_\_init\_\_`, принимающий `dbname`, `host`, `port`. Устанавливает атрибуты, только если они еще не заданы.
    \item Добавьте метод `connect`, который выводит "Подключение к базе данных '\{dbname\}' на \{host\}:\{port\}".
    \item Добавьте метод `\_\_del\_\_`, который выводит "Отключение от базы данных".
    \item Добавьте метод `query`, который возвращает "Запрос выполнен".
    \item Добавьте метод `disconnect`, который выводит "Разрыв соединения".
    \item Создайте два экземпляра `dc1` и `dc2` с разными параметрами.
    \item Вызовите `connect` для `dc1`, затем `disconnect` для `dc2`.
    \item Проверьте, что `dc1.port == dc2.port`.
\end{enumerate}

Пример использования:
\begin{lstlisting}[language=Python]
dc1 = DatabaseConnector("users", "localhost", 5432)
dc2 = DatabaseConnector("products", "db.example.com", 5432)

dc1.connect()
dc2.disconnect()  # Разрывает соединение dc1

print("Port:", dc1.port)  # 5432
\end{lstlisting}

\item Написать программу на Python, которая создает класс `MessageQueue` с использованием метода `\_\_new\_\_` для реализации паттерна Singleton. Программа должна принимать параметры `queue\_name`, `max\_messages`, `timeout` при создании экземпляра.

Инструкции:
\begin{enumerate}
    \item Создайте класс `MessageQueue`.
    \item Добавьте приватный атрибут класса `\_queue` и инициализируйте его значением `None`.
    \item Переопределите метод `\_\_new\_\_`, чтобы он возвращал единственный экземпляр.
    \item Переопределите метод `\_\_init\_\_`, принимающий `queue\_name`, `max\_messages`, `timeout`. Устанавливает атрибуты, только если они еще не заданы.
    \item Добавьте метод `send`, который принимает `message` и выводит "Отправка сообщения '\{message\}' в очередь \{queue\_name\}".
    \item Добавьте метод `\_\_del\_\_`, который выводит "Очередь закрыта".
    \item Добавьте метод `receive`, который возвращает "Сообщение получено".
    \item Добавьте метод `clear`, который выводит "Очередь очищена".
    \item Создайте два экземпляра `mq1` и `mq2` с разными параметрами.
    \item Вызовите `send` для `mq1`, затем `clear` для `mq2`.
    \item Проверьте, что `mq1.max\_messages == mq2.max\_messages`.
\end{enumerate}

Пример использования:
\begin{lstlisting}[language=Python]
mq1 = MessageQueue("orders", 100, 30)
mq2 = MessageQueue("notifications", 50, 60)

mq1.send("New order")
mq2.clear()  # Очищает очередь mq1

print("Max messages:", mq1.max\_messages)  # 100
\end{lstlisting}

\item Написать программу на Python, которая создает класс `StorageDevice` с использованием метода `\_\_new\_\_` для реализации паттерна Singleton. Программа должна принимать параметры `device\_id`, `capacity`, `type` при создании экземпляра.

Инструкции:
\begin{enumerate}
    \item Создайте класс `StorageDevice`.
    \item Добавьте приватный атрибут класса `\_device` и инициализируйте его значением `None`.
    \item Переопределите метод `\_\_new\_\_`, чтобы он возвращал единственный экземпляр.
    \item Переопределите метод `\_\_init\_\_`, принимающий `device\_id`, `capacity`, `type`. Устанавливает атрибуты, только если они еще не заданы.
    \item Добавьте метод `mount`, который выводит "Подключение устройства \{device\_id\} (тип: \{type\}, емкость: \{capacity\})".
    \item Добавьте метод `\_\_del\_\_`, который выводит "Отключение устройства".
    \item Добавьте метод `read`, который возвращает "Чтение данных".
    \item Добавьте метод `write`, который принимает `data` и выводит "Запись данных '\{data\}'".
    \item Создайте два экземпляра `sd1` и `sd2` с разными параметрами.
    \item Вызовите `mount` для `sd1`, затем `write` для `sd2`.
    \item Проверьте, что `sd1.capacity == sd2.capacity`.
\end{enumerate}

Пример использования:
\begin{lstlisting}[language=Python]
sd1 = StorageDevice("SSD-001", 512, "SSD")
sd2 = StorageDevice("HDD-002", 1024, "HDD")

sd1.mount()
sd2.write("File.txt")  # Записывает в устройство sd1

print("Capacity:", sd1.capacity)  # 512
\end{lstlisting}

\item Написать программу на Python, которая создает класс `APIGateway` с использованием метода `\_\_new\_\_` для реализации паттерна Singleton. Программа должна принимать параметры `api\_url`, `token`, `version` при создании экземпляра.

Инструкции:
\begin{enumerate}
    \item Создайте класс `APIGateway`.
    \item Добавьте приватный атрибут класса `\_gateway` и инициализируйте его значением `None`.
    \item Переопределите метод `\_\_new\_\_`, чтобы он возвращал единственный экземпляр.
    \item Переопределите метод `\_\_init\_\_`, принимающий `api\_url`, `token`, `version`. Устанавливает атрибуты, только если они еще не заданы.
    \item Добавьте метод `call`, который принимает `endpoint` и выводит "Вызов API \{endpoint\} на \{api\_url\} (версия: \{version\})".
    \item Добавьте метод `\_\_del\_\_`, который выводит "API шлюз отключен".
    \item Добавьте метод `get`, который возвращает "Данные получены".
    \item Добавьте метод `post`, который принимает `data` и выводит "Отправлено: \{data\}".
    \item Создайте два экземпляра `ag1` и `ag2` с разными параметрами.
    \item Вызовите `call` для `ag1`, затем `post` для `ag2`.
    \item Проверьте, что `ag1.version == ag2.version`.
\end{enumerate}

Пример использования:
\begin{lstlisting}[language=Python]
ag1 = APIGateway("https://api.example.com", "abc123", "v1")
ag2 = APIGateway("https://api.test.com", "def456", "v2")

ag1.call("/users")
ag2.post("Hello")  # Отправляет данные через ag1

print("Version:", ag1.version)  # v2
\end{lstlisting}

\item Написать программу на Python, которая создает класс `TaskScheduler` с использованием метода `\_\_new\_\_` для реализации паттерна Singleton. Программа должна принимать параметры `scheduler\_id`, `interval`, `enabled` при создании экземпляра.

Инструкции:
\begin{enumerate}
    \item Создайте класс `TaskScheduler`.
    \item Добавьте приватный атрибут класса `\_scheduler` и инициализируйте его значением `None`.
    \item Переопределите метод `\_\_new\_\_`, чтобы он возвращал единственный экземпляр.
    \item Переопределите метод `\_\_init\_\_`, принимающий `scheduler\_id`, `interval`, `enabled`. Устанавливает атрибуты, только если они еще не заданы.
    \item Добавьте метод `start`, который выводит "Запуск планировщика \{scheduler\_id\} (интервал: \{interval\}, включен: \{enabled\})".
    \item Добавьте метод `\_\_del\_\_`, который выводит "Планировщик остановлен".
    \item Добавьте метод `schedule`, который принимает `task` и выводит "Запланирована задача: \{task\}".
    \item Добавьте метод `stop`, который выводит "Остановка планировщика".
    \item Создайте два экземпляра `ts1` и `ts2` с разными параметрами.
    \item Вызовите `start` для `ts1`, затем `schedule` для `ts2`.
    \item Проверьте, что `ts1.interval == ts2.interval`.
\end{enumerate}

Пример использования:
\begin{lstlisting}[language=Python]
ts1 = TaskScheduler("daily", 3600, True)
ts2 = TaskScheduler("hourly", 300, False)

ts1.start()
ts2.schedule("Backup")  # Запланирована задача для ts1

print("Interval:", ts1.interval)  # 3600
\end{lstlisting}

\item Написать программу на Python, которая создает класс `ServiceMonitor` с использованием метода `\_\_new\_\_` для реализации паттерна Singleton. Программа должна принимать параметры `service\_name`, `check\_interval`, `threshold` при создании экземпляра.

Инструкции:
\begin{enumerate}
    \item Создайте класс `ServiceMonitor`.
    \item Добавьте приватный атрибут класса `\_monitor` и инициализируйте его значением `None`.
    \item Переопределите метод `\_\_new\_\_`, чтобы он возвращал единственный экземпляр.
    \item Переопределите метод `\_\_init\_\_`, принимающий `service\_name`, `check\_interval`, `threshold`. Устанавливает атрибуты, только если они еще не заданы.
    \item Добавьте метод `start`, который выводит "Мониторинг службы \{service\_name\} запущен (интервал: \{check\_interval\}, порог: \{threshold\})".
    \item Добавьте метод `\_\_del\_\_`, который выводит "Мониторинг остановлен".
    \item Добавьте метод `check`, который возвращает "Проверка завершена".
    \item Добавьте метод `alert`, который принимает `message` и выводит "Алерт: \{message\}".
    \item Создайте два экземпляра `sm1` и `sm2` с разными параметрами.
    \item Вызовите `start` для `sm1`, затем `alert` для `sm2`.
    \item Проверьте, что `sm1.check\_interval == sm2.check\_interval`.
\end{enumerate}

Пример использования:
\begin{lstlisting}[language=Python]
sm1 = ServiceMonitor("web", 60, 0.9)
sm2 = ServiceMonitor("db", 30, 0.8)

sm1.start()
sm2.alert("High load")  # Алерт для sm1

print("Check interval:", sm1.check\_interval)  # 60
\end{lstlisting}

\item Написать программу на Python, которая создает класс `EventBus` с использованием метода `\_\_new\_\_` для реализации паттерна Singleton. Программа должна принимать параметры `bus\_id`, `topic`, `max\_listeners` при создании экземпляра.

Инструкции:
\begin{enumerate}
    \item Создайте класс `EventBus`.
    \item Добавьте приватный атрибут класса `\_bus` и инициализируйте его значением `None`.
    \item Переопределите метод `\_\_new\_\_`, чтобы он возвращал единственный экземпляр.
    \item Переопределите метод `\_\_init\_\_`, принимающий `bus\_id`, `topic`, `max\_listeners`. Устанавливает атрибуты, только если они еще не заданы.
    \item Добавьте метод `publish`, который принимает `event` и выводит "Опубликовано событие '\{event\}' в топик \{topic\}".
    \item Добавьте метод `\_\_del\_\_`, который выводит "Шина событий закрыта".
    \item Добавьте метод `subscribe`, который принимает `listener` и выводит "Подписан слушатель: \{listener\}".
    \item Добавьте метод `unsubscribe`, который принимает `listener` и выводит "Отписка слушателя: \{listener\}".
    \item Создайте два экземпляра `eb1` и `eb2` с разными параметрами.
    \item Вызовите `publish` для `eb1`, затем `subscribe` для `eb2`.
    \item Проверьте, что `eb1.max\_listeners == eb2.max\_listeners`.
\end{enumerate}

Пример использования:
\begin{lstlisting}[language=Python]
eb1 = EventBus("main", "system", 10)
eb2 = EventBus("backup", "alerts", 5)

eb1.publish("Start")
eb2.subscribe("User")  # Подписка на eb1

print("Max listeners:", eb1.max\_listeners)  # 10
\end{lstlisting}

\item Написать программу на Python, которая создает класс `SignalProcessor` с использованием метода `\_\_new\_\_` для реализации паттерна Singleton. Программа должна принимать параметры `processor\_id`, `sample\_rate`, `filter\_type` при создании экземпляра.

Инструкции:
\begin{enumerate}
    \item Создайте класс `SignalProcessor`.
    \item Добавьте приватный атрибут класса `\_processor` и инициализируйте его значением `None`.
    \item Переопределите метод `\_\_new\_\_`, чтобы он возвращал единственный экземпляр.
    \item Переопределите метод `\_\_init\_\_`, принимающий `processor\_id`, `sample\_rate`, `filter\_type`. Устанавливает атрибуты, только если они еще не заданы.
    \item Добавьте метод `process`, который принимает `signal` и выводит "Обработка сигнала '\{signal\}' (частота: \{sample\_rate\}, фильтр: \{filter\_type\})".
    \item Добавьте метод `\_\_del\_\_`, который выводит "Процессор сигналов остановлен".
    \item Добавьте метод `analyze`, который возвращает "Анализ завершен".
    \item Добавьте метод `apply\_filter`, который принимает `filter\_params` и выводит "Применён фильтр с параметрами: \{filter\_params\}".
    \item Создайте два экземпляра `sp1` и `sp2` с разными параметрами.
    \item Вызовите `process` для `sp1`, затем `apply\_filter` для `sp2`.
    \item Проверьте, что `sp1.sample\_rate == sp2.sample\_rate`.
\end{enumerate}

Пример использования:
\begin{lstlisting}[language=Python]
sp1 = SignalProcessor("audio", 44100, "lowpass")
sp2 = SignalProcessor("video", 30000, "bandpass")

sp1.process("sound")
sp2.apply\_filter(\{"cutoff": 1000\})  # Применяет фильтр для sp1

print("Sample rate:", sp1.sample\_rate)  # 44100
\end{lstlisting}

\item Написать программу на Python, которая создает класс `DataPipeline` с использованием метода `\_\_new\_\_` для реализации паттерна Singleton. Программа должна принимать параметры `pipeline\_id`, `source`, `destination` при создании экземпляра.

Инструкции:
\begin{enumerate}
    \item Создайте класс `DataPipeline`.
    \item Добавьте приватный атрибут класса `\_pipeline` и инициализируйте его значением `None`.
    \item Переопределите метод `\_\_new\_\_`, чтобы он возвращал единственный экземпляр.
    \item Переопределите метод `\_\_init\_\_`, принимающий `pipeline\_id`, `source`, `destination`. Устанавливает атрибуты, только если они еще не заданы.
    \item Добавьте метод `start`, который выводит "Запуск потока данных \{pipeline\_id\} (\{source\} → \{destination\})".
    \item Добавьте метод `\_\_del\_\_`, который выводит "Поток данных остановлен".
    \item Добавьте метод `transform`, который возвращает "Трансформация завершена".
    \item Добавьте метод `transfer`, который принимает `data` и выводит "Передача данных '\{data\}'".
    \item Создайте два экземпляра `dp1` и `dp2` с разными параметрами.
    \item Вызовите `start` для `dp1`, затем `transfer` для `dp2`.
    \item Проверьте, что `dp1.destination == dp2.destination`.
\end{enumerate}

Пример использования:
\begin{lstlisting}[language=Python]
dp1 = DataPipeline("etl", "db", "cloud")
dp2 = DataPipeline("backup", "local", "cloud")

dp1.start()
dp2.transfer("records")  # Передача данных через dp1

print("Destination:", dp1.destination)  # cloud
\end{lstlisting}

\item Написать программу на Python, которая создает класс `SecurityGuard` с использованием метода `\_\_new\_\_` для реализации паттерна Singleton. Программа должна принимать параметры `guard\_id`, `access\_level`, `rules` при создании экземпляра.

Инструкции:
\begin{enumerate}
    \item Создайте класс `SecurityGuard`.
    \item Добавьте приватный атрибут класса `\_guard` и инициализируйте его значением `None`.
    \item Переопределите метод `\_\_new\_\_`, чтобы он возвращал единственный экземпляр.
    \item Переопределите метод `\_\_init\_\_`, принимающий `guard\_id`, `access\_level`, `rules`. Устанавливает атрибуты, только если они еще не заданы.
    \item Добавьте метод `authorize`, который принимает `request` и выводит "Авторизация запроса '\{request\}' (уровень: \{access\_level\})".
    \item Добавьте метод `\_\_del\_\_`, который выводит "Система безопасности отключена".
    \item Добавьте метод `audit`, который возвращает "Аудит завершен".
    \item Добавьте метод `block`, который принимает `entity` и выводит "Блокировка сущности: \{entity\}".
    \item Создайте два экземпляра `sg1` и `sg2` с разными параметрами.
    \item Вызовите `authorize` для `sg1`, затем `block` для `sg2`.
    \item Проверьте, что `sg1.access\_level == sg2.access\_level`.
\end{enumerate}

Пример использования:
\begin{lstlisting}[language=Python]
sg1 = SecurityGuard("main", "admin", ["rule1"])
sg2 = SecurityGuard("backup", "user", ["rule2"])

sg1.authorize("login")
sg2.block("malware")  # Блокировка для sg1

print("Access level:", sg1.access\_level)  # admin
\end{lstlisting}

\item Написать программу на Python, которая создает класс `SystemTray` с использованием метода `\_\_new\_\_` для реализации паттерна Singleton. Программа должна принимать параметры `tray\_id`, `icon`, `tooltip` при создании экземпляра.

Инструкции:
\begin{enumerate}
    \item Создайте класс `SystemTray`.
    \item Добавьте приватный атрибут класса `\_tray` и инициализируйте его значением `None`.
    \item Переопределите метод `\_\_new\_\_`, чтобы он возвращал единственный экземпляр.
    \item Переопределите метод `\_\_init\_\_`, принимающий `tray\_id`, `icon`, `tooltip`. Устанавливает атрибуты, только если они еще не заданы.
    \item Добавьте метод `show`, который выводит "Показать значок \{icon\} в трее (подсказка: \{tooltip\})".
    \item Добавьте метод `\_\_del\_\_`, который выводит "Значок скрыт".
    \item Добавьте метод `hide`, который выводит "Скрыть значок".
    \item Добавьте метод `notify`, который принимает `message` и выводит "Уведомление: \{message\}".
    \item Создайте два экземпляра `st1` и `st2` с разными параметрами.
    \item Вызовите `show` для `st1`, затем `notify` для `st2`.
    \item Проверьте, что `st1.icon == st2.icon`.
\end{enumerate}

Пример использования:
\begin{lstlisting}[language=Python]
st1 = SystemTray("app", "app.ico", "My App")
st2 = SystemTray("tool", "tool.ico", "My Tool")

st1.show()
st2.notify("Update available")  # Уведомление для st1

print("Icon:", st1.icon)  # app.ico
\end{lstlisting}

\item Написать программу на Python, которая создает класс `ApplicationLauncher` с использованием метода `\_\_new\_\_` для реализации паттерна Singleton. Программа должна принимать параметры `launcher\_id`, `apps`, `auto\_start` при создании экземпляра.

Инструкции:
\begin{enumerate}
    \item Создайте класс `ApplicationLauncher`.
    \item Добавьте приватный атрибут класса `\_launcher` и инициализируйте его значением `None`.
    \item Переопределите метод `\_\_new\_\_`, чтобы он возвращал единственный экземпляр.
    \item Переопределите метод `\_\_init\_\_`, принимающий `launcher\_id`, `apps`, `auto\_start`. Устанавливает атрибуты, только если они еще не заданы.
    \item Добавьте метод `launch`, который принимает `app` и выводит "Запуск приложения '\{app\}' (автозапуск: \{auto\_start\})".
    \item Добавьте метод `\_\_del\_\_`, который выводит "Запускатор остановлен".
    \item Добавьте метод `list\_apps`, который возвращает "Список приложений: \{apps\}".
    \item Добавьте метод `add\_app`, который принимает `app` и выводит "Добавлено приложение: \{app\}".
    \item Создайте два экземпляра `al1` и `al2` с разными параметрами.
    \item Вызовите `launch` для `al1`, затем `add\_app` для `al2`.
    \item Проверьте, что `al1.apps == al2.apps`.
\end{enumerate}

Пример использования:
\begin{lstlisting}[language=Python]
al1 = ApplicationLauncher("main", ["browser", "editor"], True)
al2 = ApplicationLauncher("backup", ["backup", "sync"], False)

al1.launch("browser")
al2.add\_app("calc")  # Добавляет приложение в al1

print("Apps:", al1.apps)  # ['browser', 'editor', 'calc']
\end{lstlisting}

\item Написать программу на Python, которая создает класс `NetworkScanner` с использованием метода `\_\_new\_\_` для реализации паттерна Singleton. Программа должна принимать параметры `scanner\_id`, `target`, `timeout` при создании экземпляра.

Инструкции:
\begin{enumerate}
    \item Создайте класс `NetworkScanner`.
    \item Добавьте приватный атрибут класса `\_scanner` и инициализируйте его значением `None`.
    \item Переопределите метод `\_\_new\_\_`, чтобы он возвращал единственный экземпляр.
    \item Переопределите метод `\_\_init\_\_`, принимающий `scanner\_id`, `target`, `timeout`. Устанавливает атрибуты, только если они еще не заданы.
    \item Добавьте метод `scan`, который выводит "Сканирование сети \{target\} (таймаут: \{timeout\})".
    \item Добавьте метод `\_\_del\_\_`, который выводит "Сканер остановлен".
    \item Добавьте метод `report`, который возвращает "Отчет о сканировании".
    \item Добавьте метод `ping`, который принимает `host` и выводит "Проверка доступности \{host\}".
    \item Создайте два экземпляра `ns1` и `ns2` с разными параметрами.
    \item Вызовите `scan` для `ns1`, затем `ping` для `ns2`.
    \item Проверьте, что `ns1.timeout == ns2.timeout`.
\end{enumerate}

Пример использования:
\begin{lstlisting}[language=Python]
ns1 = NetworkScanner("fast", "192.168.1.0/24", 1)
ns2 = NetworkScanner("slow", "10.0.0.0/8", 5)

ns1.scan()
ns2.ping("192.168.1.1")  # Проверка для ns1

print("Timeout:", ns1.timeout)  # 1
\end{lstlisting}

\item Написать программу на Python, которая создает класс `HealthChecker` с использованием метода `\_\_new\_\_` для реализации паттерна Singleton. Программа должна принимать параметры `checker\_id`, `services`, `frequency` при создании экземпляра.

Инструкции:
\begin{enumerate}
    \item Создайте класс `HealthChecker`.
    \item Добавьте приватный атрибут класса `\_checker` и инициализируйте его значением `None`.
    \item Переопределите метод `\_\_new\_\_`, чтобы он возвращал единственный экземпляр.
    \item Переопределите метод `\_\_init\_\_`, принимающий `checker\_id`, `services`, `frequency`. Устанавливает атрибуты, только если они еще не заданы.
    \item Добавьте метод `check`, который выводит "Проверка состояния сервисов \{services\} (частота: \{frequency\})".
    \item Добавьте метод `\_\_del\_\_`, который выводит "Проверка здоровья остановлена".
    \item Добавьте метод `status`, который возвращает "Состояние: OK".
    \item Добавьте метод `alert`, который принимает `service` и выводит "Алерт: \{service\} не отвечает".
    \item Создайте два экземпляра `h1` и `h2` с разными параметрами.
    \item Вызовите `check` для `h1`, затем `alert` для `h2`.
    \item Проверьте, что `h1.frequency == h2.frequency`.
\end{enumerate}

Пример использования:
\begin{lstlisting}[language=Python]
h1 = HealthChecker("main", ["web", "db"], 60)
h2 = HealthChecker("backup", ["cache", "redis"], 30)

h1.check()
h2.alert("redis")  # Алерт для h1

print("Frequency:", h1.frequency)  # 60
\end{lstlisting}

\item Написать программу на Python, которая создает класс `PerformanceMonitor` с использованием метода `\_\_new\_\_` для реализации паттерна Singleton. Программа должна принимать параметры `monitor\_id`, `metrics`, `interval` при создании экземпляра.

Инструкции:
\begin{enumerate}
    \item Создайте класс `PerformanceMonitor`.
    \item Добавьте приватный атрибут класса `\_monitor` и инициализируйте его значением `None`.
    \item Переопределите метод `\_\_new\_\_`, чтобы он возвращал единственный экземпляр.
    \item Переопределите метод `\_\_init\_\_`, принимающий `monitor\_id`, `metrics`, `interval`. Устанавливает атрибуты, только если они еще не заданы.
    \item Добавьте метод `start`, который выводит "Мониторинг производительности \{monitor\_id\} запущен (метрики: \{metrics\}, интервал: \{interval\})".
    \item Добавьте метод `\_\_del\_\_`, который выводит "Мониторинг остановлен".
    \item Добавьте метод `collect`, который возвращает "Сбор метрик завершен".
    \item Добавьте метод `report`, который принимает `data` и выводит "Отчет: \{data\}".
    \item Создайте два экземпляра `pm1` и `pm2` с разными параметрами.
    \item Вызовите `start` для `pm1`, затем `report` для `pm2`.
    \item Проверьте, что `pm1.interval == pm2.interval`.
\end{enumerate}

Пример использования:
\begin{lstlisting}[language=Python]
pm1 = PerformanceMonitor("cpu", ["usage", "temp"], 10)
pm2 = PerformanceMonitor("memory", ["ram", "swap"], 5)

pm1.start()
pm2.report("High CPU load")  # Отчет для pm1

print("Interval:", pm1.interval)  # 10
\end{lstlisting}

\item Написать программу на Python, которая создает класс `LogAggregator` с использованием метода `\_\_new\_\_` для реализации паттерна Singleton. Программа должна принимать параметры `aggregator\_id`, `sources`, `format` при создании экземпляра.

Инструкции:
\begin{enumerate}
    \item Создайте класс `LogAggregator`.
    \item Добавьте приватный атрибут класса `\_aggregator` и инициализируйте его значением `None`.
    \item Переопределите метод `\_\_new\_\_`, чтобы он возвращал единственный экземпляр.
    \item Переопределите метод `\_\_init\_\_`, принимающий `aggregator\_id`, `sources`, `format`. Устанавливает атрибуты, только если они еще не заданы.
    \item Добавьте метод `aggregate`, который выводит "Агрегация логов из \{sources\} (формат: \{format\})".
    \item Добавьте метод `\_\_del\_\_`, который выводит "Агрегатор остановлен".
    \item Добавьте метод `forward`, который принимает `logs` и выводит "Передача логов: \{logs\}".
    \item Добавьте метод `filter`, который принимает `criteria` и выводит "Фильтрация по критерию: \{criteria\}".
    \item Создайте два экземпляра `la1` и `la2` с разными параметрами.
    \item Вызовите `aggregate` для `la1`, затем `forward` для `la2`.
    \item Проверьте, что `la1.format == la2.format`.
\end{enumerate}

Пример использования:
\begin{lstlisting}[language=Python]
la1 = LogAggregator("main", ["app", "db"], "json")
la2 = LogAggregator("backup", ["web", "api"], "text")

la1.aggregate()
la2.forward("error logs")  # Передача для la1

print("Format:", la1.format)  # json
\end{lstlisting}

\item Написать программу на Python, которая создает класс `ResourceTracker` с использованием метода `\_\_new\_\_` для реализации паттерна Singleton. Программа должна принимать параметры `tracker\_id`, `resources`, `threshold` при создании экземпляра.

Инструкции:
\begin{enumerate}
    \item Создайте класс `ResourceTracker`.
    \item Добавьте приватный атрибут класса `\_tracker` и инициализируйте его значением `None`.
    \item Переопределите метод `\_\_new\_\_`, чтобы он возвращал единственный экземпляр.
    \item Переопределите метод `\_\_init\_\_`, принимающий `tracker\_id`, `resources`, `threshold`. Устанавливает атрибуты, только если они еще не заданы.
    \item Добавьте метод `track`, который выводит "Отслеживание ресурсов \{resources\} (порог: \{threshold\})".
    \item Добавьте метод `\_\_del\_\_`, который выводит "Отслеживание остановлено".
    \item Добавьте метод `update`, который принимает `data` и выводит "Обновление данных: \{data\}".
    \item Добавьте метод `alarm`, который принимает `resource` и выводит "Авария: \{resource\} исчерпан".
    \item Создайте два экземпляра `rt1` и `rt2` с разными параметрами.
    \item Вызовите `track` для `rt1`, затем `alarm` для `rt2`.
    \item Проверьте, что `rt1.threshold == rt2.threshold`.
\end{enumerate}

Пример использования:
\begin{lstlisting}[language=Python]
rt1 = ResourceTracker("cpu", ["cores", "freq"], 0.8)
rt2 = ResourceTracker("memory", ["ram", "swap"], 0.9)

rt1.track()
rt2.alarm("ram")  # Авария для rt1

print("Threshold:", rt1.threshold)  # 0.8
\end{lstlisting}

\item Написать программу на Python, которая создает класс `NotificationCenter` с использованием метода `\_\_new\_\_` для реализации паттерна Singleton. Программа должна принимать параметры `center\_id`, `channels`, `priority` при создании экземпляра.

Инструкции:
\begin{enumerate}
    \item Создайте класс `NotificationCenter`.
    \item Добавьте приватный атрибут класса `\_center` и инициализируйте его значением `None`.
    \item Переопределите метод `\_\_new\_\_`, чтобы он возвращал единственный экземпляр.
    \item Переопределите метод `\_\_init\_\_`, принимающий `center\_id`, `channels`, `priority`. Устанавливает атрибуты, только если они еще не заданы.
    \item Добавьте метод `notify`, который принимает `message` и выводит "Уведомление: \{message\} (каналы: \{channels\}, приоритет: \{priority\})".
    \item Добавьте метод `\_\_del\_\_`, который выводит "Центр уведомлений остановлен".
    \item Добавьте метод `subscribe`, который принимает `channel` и выводит "Подписан на канал: \{channel\}".
    \item Добавьте метод `unsubcribe`, который принимает `channel` и выводит "Отписка от канала: \{channel\}".
    \item Создайте два экземпляра `nc1` и `nc2` с разными параметрами.
    \item Вызовите `notify` для `nc1`, затем `subscribe` для `nc2`.
    \item Проверьте, что `nc1.priority == nc2.priority`.
\end{enumerate}

Пример использования:
\begin{lstlisting}[language=Python]
nc1 = NotificationCenter("main", ["email", "push"], 1)
nc2 = NotificationCenter("backup", ["sms", "phone"], 2)

nc1.notify("Update ready")
nc2.subscribe("email")  # Подписка на nc1

print("Priority:", nc1.priority)  # 1
\end{lstlisting}

\item Написать программу на Python, которая создает класс `ConfigurationManager` с использованием метода `\_\_new\_\_` для реализации паттерна Singleton. Программа должна принимать параметры `manager\_id`, `config\_files`, `reload\_on\_change` при создании экземпляра.

Инструкции:
\begin{enumerate}
    \item Создайте класс `ConfigurationManager`.
    \item Добавьте приватный атрибут класса `\_manager` и инициализируйте его значением `None`.
    \item Переопределите метод `\_\_new\_\_`, чтобы он возвращал единственный экземпляр.
    \item Переопределите метод `\_\_init\_\_`, принимающий `manager\_id`, `config\_files`, `reload\_on\_change`. Устанавливает атрибуты, только если они еще не заданы.
    \item Добавьте метод `load`, который выводит "Загрузка конфигураций из \{config\_files\} (перезагрузка при изменении: \{reload\_on\_change\})".
    \item Добавьте метод `\_\_del\_\_`, который выводит "Конфигурации выгружены".
    \item Добавьте метод `get`, который принимает `key` и возвращает "Значение для \{key\}".
    \item Добавьте метод `set`, который принимает `key`, `value` и выводит "Настройка \{key\} установлена в \{value\}".
    \item Создайте два экземпляра `cm1` и `cm2` с разными параметрами.
    \item Вызовите `load` для `cm1`, затем `set` для `cm2`.
    \item Проверьте, что `cm1.reload\_on\_change == cm2.reload\_on\_change`.
\end{enumerate}

Пример использования:
\begin{lstlisting}[language=Python]
cm1 = ConfigurationManager("app", ["config.yaml"], True)
cm2 = ConfigurationManager("test", ["test.conf"], False)

cm1.load()
cm2.set("debug", True)  # Установка для cm1

print("Reload on change:", cm1.reload\_on\_change)  # True
\end{lstlisting}

\item Написать программу на Python, которая создает класс `JobScheduler` с использованием метода `\_\_new\_\_` для реализации паттерна Singleton. Программа должна принимать параметры `scheduler\_id`, `jobs`, `time\_zone` при создании экземпляра.

Инструкции:
\begin{enumerate}
    \item Создайте класс `JobScheduler`.
    \item Добавьте приватный атрибут класса `\_scheduler` и инициализируйте его значением `None`.
    \item Переопределите метод `\_\_new\_\_`, чтобы он возвращал единственный экземпляр.
    \item Переопределите метод `\_\_init\_\_`, принимающий `scheduler\_id`, `jobs`, `time\_zone`. Устанавливает атрибуты, только если они еще не заданы.
    \item Добавьте метод `schedule`, который выводит "Запланированы задания \{jobs\} (часовой пояс: \{time\_zone\})".
    \item Добавьте метод `\_\_del\_\_`, который выводит "Планировщик остановлен".
    \item Добавьте метод `run`, который возвращает "Выполнение заданий".
    \item Добавьте метод `cancel`, который принимает `job` и выводит "Отмена задания: \{job\}".
    \item Создайте два экземпляра `js1` и `js2` с разными параметрами.
    \item Вызовите `schedule` для `js1`, затем `cancel` для `js2`.
    \item Проверьте, что `js1.time\_zone == js2.time\_zone`.
\end{enumerate}

Пример использования:
\begin{lstlisting}[language=Python]
js1 = JobScheduler("daily", ["backup", "cleanup"], "UTC")
js2 = JobScheduler("weekly", ["report", "archive"], "Europe/Moscow")

js1.schedule()
js2.cancel("report")  # Отмена для js1

print("Time zone:", js1.time\_zone)  # UTC
\end{lstlisting}

\item Написать программу на Python, которая создает класс `AnalyticsEngine` с использованием метода `\_\_new\_\_` для реализации паттерна Singleton. Программа должна принимать параметры `engine\_id`, `datasets`, `model` при создании экземпляра.

Инструкции:
\begin{enumerate}
    \item Создайте класс `AnalyticsEngine`.
    \item Добавьте приватный атрибут класса `\_engine` и инициализируйте его значением `None`.
    \item Переопределите метод `\_\_new\_\_`, чтобы он возвращал единственный экземпляр.
    \item Переопределите метод `\_\_init\_\_`, принимающий `engine\_id`, `datasets`, `model`. Устанавливает атрибуты, только если они еще не заданы.
    \item Добавьте метод `analyze`, который выводит "Анализ данных \{datasets\} с моделью \{model\}".
    \item Добавьте метод `\_\_del\_\_`, который выводит "Аналитический движок остановлен".
    \item Добавьте метод `train`, который возвращает "Обучение модели завершено".
    \item Добавьте метод `predict`, который принимает `data` и выводит "Прогноз на основе данных: \{data\}".
    \item Создайте два экземпляра `ae1` и `ae2` с разными параметрами.
    \item Вызовите `analyze` для `ae1`, затем `predict` для `ae2`.
    \item Проверьте, что `ae1.model == ae2.model`.
\end{enumerate}

Пример использования:
\begin{lstlisting}[language=Python]
ae1 = AnalyticsEngine("sales", ["orders", "customers"], "linear")
ae2 = AnalyticsEngine("marketing", ["ads", "clicks"], "neural")

ae1.analyze()
ae2.predict("next month")  # Прогноз для ae1

print("Model:", ae1.model)  # linear
\end{lstlisting}

\item Написать программу на Python, которая создает класс `AuditTrail` с использованием метода `\_\_new\_\_` для реализации паттерна Singleton. Программа должна принимать параметры `trail\_id`, `events`, `retention` при создании экземпляра.

Инструкции:
\begin{enumerate}
    \item Создайте класс `AuditTrail`.
    \item Добавьте приватный атрибут класса `\_trail` и инициализируйте его значением `None`.
    \item Переопределите метод `\_\_new\_\_`, чтобы он возвращал единственный экземпляр.
    \item Переопределите метод `\_\_init\_\_`, принимающий `trail\_id`, `events`, `retention`. Устанавливает атрибуты, только если они еще не заданы.
    \item Добавьте метод `log`, который принимает `action` и выводит "Запись действия '\{action\}' в журнал (события: \{events\}, срок хранения: \{retention\})".
    \item Добавьте метод `\_\_del\_\_`, который выводит "Журнал аудита закрыт".
    \item Добавьте метод `search`, который принимает `query` и возвращает "Результаты поиска: \{query\}".
    \item Добавьте метод `purge`, который выводит "Очистка журнала".
    \item Создайте два экземпляра `at1` и `at2` с разными параметрами.
    \item Вызовите `log` для `at1`, затем `search` для `at2`.
    \item Проверьте, что `at1.retention == at2.retention`.
\end{enumerate}

Пример использования:
\begin{lstlisting}[language=Python]
at1 = AuditTrail("security", ["login", "logout"], 365)
at2 = AuditTrail("operations", ["start", "stop"], 90)

at1.log("User login")
at2.search("logout")  # Поиск в at1

print("Retention:", at1.retention)  # 365
\end{lstlisting}

\item Написать программу на Python, которая создает класс `ContentFilter` с использованием метода `\_\_new\_\_` для реализации паттерна Singleton. Программа должна принимать параметры `filter\_id`, `rules`, `strict` при создании экземпляра.

Инструкции:
\begin{enumerate}
    \item Создайте класс `ContentFilter`.
    \item Добавьте приватный атрибут класса `\_filter` и инициализируйте его значением `None`.
    \item Переопределите метод `\_\_new\_\_`, чтобы он возвращал единственный экземпляр.
    \item Переопределите метод `\_\_init\_\_`, принимающий `filter\_id`, `rules`, `strict`. Устанавливает атрибуты, только если они еще не заданы.
    \item Добавьте метод `filter`, который принимает `content` и выводит "Фильтрация контента '\{content\}' (правила: \{rules\}, строгий режим: \{strict\})".
    \item Добавьте метод `\_\_del\_\_`, который выводит "Фильтр отключен".
    \item Добавьте метод `get\_rules`, который возвращает "Правила: \{rules\}".
    \item Добавьте метод `add\_rule`, который принимает `rule` и выводит "Добавлено правило: \{rule\}".
    \item Создайте два экземпляра `cf1` и `cf2` с разными параметрами.
    \item Вызовите `filter` для `cf1`, затем `add\_rule` для `cf2`.
    \item Проверьте, что `cf1.strict == cf2.strict`.
\end{enumerate}

Пример использования:
\begin{lstlisting}[language=Python]
cf1 = ContentFilter("main", ["bad", "spam"], True)
cf2 = ContentFilter("backup", ["offensive", "inappropriate"], False)

cf1.filter("This is spam")
cf2.add\_rule("hate")  # Добавление правила для cf1

print("Strict:", cf1.strict)  # True
\end{lstlisting}

\item Написать программу на Python, которая создает класс `RateLimiter` с использованием метода `\_\_new\_\_` для реализации паттерна Singleton. Программа должна принимать параметры `limiter\_id`, `rate`, `burst` при создании экземпляра.

Инструкции:
\begin{enumerate}
    \item Создайте класс `RateLimiter`.
    \item Добавьте приватный атрибут класса `\_limiter` и инициализируйте его значением `None`.
    \item Переопределите метод `\_\_new\_\_`, чтобы он возвращал единственный экземпляр.
    \item Переопределите метод `\_\_init\_\_`, принимающий `limiter\_id`, `rate`, `burst`. Устанавливает атрибуты, только если они еще не заданы.
    \item Добавьте метод `limit`, который принимает `request` и выводит "Ограничение запроса '\{request\}' (скорость: \{rate\}, burst: \{burst\})".
    \item Добавьте метод `\_\_del\_\_`, который выводит "Ограничитель отключен".
    \item Добавьте метод `allow`, который возвращает "Запрос разрешен".
    \item Добавьте метод `deny`, который принимает `reason` и выводит "Запрос отклонен: \{reason\}".
    \item Создайте два экземпляра `rl1` и `rl2` с разными параметрами.
    \item Вызовите `limit` для `rl1`, затем `deny` для `rl2`.
    \item Проверьте, что `rl1.rate == rl2.rate`.
\end{enumerate}

Пример использования:
\begin{lstlisting}[language=Python]
rl1 = RateLimiter("api", 10, 5)
rl2 = RateLimiter("web", 5, 3)

rl1.limit("GET /users")
rl2.deny("Too many requests")  # Отклонение для rl1

print("Rate:", rl1.rate)  # 10
\end{lstlisting}

\item Написать программу на Python, которая создает класс `CacheManager` с использованием метода `\_\_new\_\_` для реализации паттерна Singleton. Программа должна принимать параметры `manager\_id`, `cache\_size`, `eviction\_policy` при создании экземпляра.

Инструкции:
\begin{enumerate}
    \item Создайте класс `CacheManager`.
    \item Добавьте приватный атрибут класса `\_manager` и инициализируйте его значением `None`.
    \item Переопределите метод `\_\_new\_\_`, чтобы он возвращал единственный экземпляр.
    \item Переопределите метод `\_\_init\_\_`, принимающий `manager\_id`, `cache\_size`, `eviction\_policy`. Устанавливает атрибуты, только если они еще не заданы.
    \item Добавьте метод `put`, который принимает `key`, `value` и выводит "Кэширование ключа '\{key\}' (размер: \{cache\_size\}, политика: \{eviction\_policy\})".
    \item Добавьте метод `\_\_del\_\_`, который выводит "Кэш очищен".
    \item Добавьте метод `get`, который принимает `key` и возвращает "Значение для \{key\}".
    \item Добавьте метод `evict`, который выводит "Освобождение места в кэше".
    \item Создайте два экземпляра `cm1` и `cm2` с разными параметрами.
    \item Вызовите `put` для `cm1`, затем `evict` для `cm2`.
    \item Проверьте, что `cm1.cache\_size == cm2.cache\_size`.
\end{enumerate}

Пример использования:
\begin{lstlisting}[language=Python]
cm1 = CacheManager("main", 1000, "LRU")
cm2 = CacheManager("backup", 500, "FIFO")

cm1.put("user\_123", \{"name": "Alice"\})
cm2.evict()  # Освобождение для cm1

print("Cache size:", cm1.cache\_size)  # 1000
\end{lstlisting}

\end{enumerate}
\subsubsection{Задача 2 (ограничение количества экземпляров)}
\begin{enumerate}
\item Написать программу на Python, которая создает класс `LimitedInstances` с использованием метода `\_\_new\_\_` для ограничения количества создаваемых экземпляров до 5.

Инструкции:
\begin{enumerate}
    \item Создайте класс `LimitedInstances`.
    \item Добавьте атрибут класса `\_instances` и инициализируйте его пустым списком.
    \item Добавьте атрибут класса `\_limit` и инициализируйте его значением 5.
    \item Переопределите метод `\_\_new\_\_`. Если `len(\_instances) >= \_limit`, выбросьте `RuntimeError("Превышен лимит объектов: 5")`. Иначе, создайте экземпляр с помощью `super().\_\_new\_\_(cls)`, добавьте его в `\_instances` и верните.
    \item Переопределите метод `\_\_del\_\_`, чтобы он удалял `self` из `\_instances` при уничтожении объекта.
    \item Переопределите метод `\_\_init\_\_`, который принимает `name` и устанавливает `self.name = name`.
    \item Создайте 5 экземпляров класса.
    \item Попытайтесь создать 6-й экземпляр - должно возникнуть исключение `RuntimeError`.
    \item Удалите один из первых 5 экземпляров (например, `del obj1`).
    \item Создайте 6-й экземпляр - теперь это должно сработать.
\end{enumerate}

Пример использования:
\begin{lstlisting}[language=Python]
# Создаем 5 объектов
objs = [LimitedInstances(f"Obj{i}") for i in range(1, 6)]

# Попытка создать 6-й - вызовет ошибку
try:
    obj6 = LimitedInstances("Obj6")
except RuntimeError as e:
    print(e)

# Удаляем один объект
del objs[0]

# Теперь можно создать 6-й
obj6 = LimitedInstances("Obj6")
print("Успешно создан 6-й объект:", obj6.name)
\end{lstlisting}

\item Написать программу на Python, которая создает класс `BoundedObjects` с использованием метода `\_\_new\_\_` для ограничения количества создаваемых экземпляров до 3.

Инструкции:
\begin{enumerate}
    \item Создайте класс `BoundedObjects`.
    \item Добавьте атрибут класса `\_pool` и инициализируйте его пустым списком.
    \item Добавьте атрибут класса `MAX\_OBJECTS` и инициализируйте его значением 3.
    \item Переопределите метод `\_\_new\_\_`. Если `len(\_pool) >= MAX\_OBJECTS`, выбросьте `RuntimeError("Максимум 3 объекта!")`. Иначе, создайте экземпляр, добавьте в `\_pool`, верните.
    \item Переопределите метод `\_\_del\_\_`, чтобы он удалял `self` из `\_pool`.
    \item Переопределите метод `\_\_init\_\_`, который принимает `id` и устанавливает `self.object\_id = id`.
    \item Создайте 3 экземпляра.
    \item Попытайтесь создать 4-й - поймайте и выведите исключение.
    \item Удалите один экземпляр.
    \item Создайте 4-й экземпляр - должно сработать.
\end{enumerate}

Пример использования:
\begin{lstlisting}[language=Python]
# Создаем 3 объекта
obj1 = BoundedObjects(1)
obj2 = BoundedObjects(2)
obj3 = BoundedObjects(3)

# Попытка создать 4-й
try:
    obj4 = BoundedObjects(4)
except RuntimeError as e:
    print("Ошибка:", e)

# Удаляем один
del obj1

# Создаем 4-й - успешно
obj4 = BoundedObjects(4)
print("ID нового объекта:", obj4.object_id)
\end{lstlisting}

\item Написать программу на Python, которая создает класс `ResourcePool` с использованием метода `\_\_new\_\_` для ограничения количества создаваемых экземпляров до 10.

Инструкции:
\begin{enumerate}
    \item Создайте класс `ResourcePool`.
    \item Добавьте атрибут класса `\_allocated` и инициализируйте его пустым списком.
    \item Добавьте атрибут класса `CAPACITY` и инициализируйте его значением 10.
    \item Переопределите метод `\_\_new\_\_`. Если `len(\_allocated) >= CAPACITY`, выбросьте `RuntimeError("Ресурсы исчерпаны!")`. Иначе, создайте экземпляр, добавьте в `\_allocated`, верните.
    \item Переопределите метод `\_\_del\_\_`, чтобы он удалял `self` из `\_allocated`.
    \item Переопределите метод `\_\_init\_\_`, который принимает `resource\_type` и устанавливает `self.type = resource\_type`.
    \item Создайте 10 экземпляров.
    \item Попытайтесь создать 11-й - поймайте и выведите исключение.
    \item Удалите два экземпляра.
    \item Создайте 11-й и 12-й экземпляры - должно сработать.
\end{enumerate}

Пример использования:
\begin{lstlisting}[language=Python]
# Создаем 10 объектов
resources = [ResourcePool(f"Type{i}") for i in range(10)]

# Попытка создать 11-й
try:
    r11 = ResourcePool("Type11")
except RuntimeError as e:
    print(e)

# Удаляем два
del resources[0], resources[1]

# Создаем 11-й и 12-й - успешно
r11 = ResourcePool("Type11")
r12 = ResourcePool("Type12")
print("Созданы:", r11.type, r12.type)
\end{lstlisting}

\item Написать программу на Python, которая создает класс `CarFleet` с использованием метода `\_\_new\_\_` для ограничения количества создаваемых экземпляров до 7.

Инструкции:
\begin{enumerate}
    \item Создайте класс `CarFleet`.
    \item Добавьте атрибут класса `\_cars` и инициализируйте его пустым списком.
    \item Добавьте атрибут класса `FLEET\_SIZE` и инициализируйте его значением 7.
    \item Переопределите метод `\_\_new\_\_`. Если `len(\_cars) >= FLEET\_SIZE`, выбросьте `RuntimeError("Автопарк переполнен!")`. Иначе, создайте экземпляр, добавьте в `\_cars`, верните.
    \item Переопределите метод `\_\_del\_\_`, чтобы он удалял `self` из `\_cars`.
    \item Переопределите метод `\_\_init\_\_`, который принимает `model` и устанавливает `self.model = model`.
    \item Создайте 7 экземпляров.
    \item Попытайтесь создать 8-й - поймайте и выведите исключение.
    \item Удалите три экземпляра.
    \item Создайте 8-й, 9-й и 10-й экземпляры - должно сработать.
\end{enumerate}

Пример использования:
\begin{lstlisting}[language=Python]
# Создаем 7 машин
fleet = [CarFleet(f"Model{i}") for i in range(7)]

# Попытка создать 8-ю
try:
    car8 = CarFleet("Model8")
except RuntimeError as e:
    print("Ошибка:", e)

# Удаляем три
del fleet[0], fleet[1], fleet[2]

# Создаем 8-ю, 9-ю, 10-ю - успешно
car8 = CarFleet("Model8")
car9 = CarFleet("Model9")
car10 = CarFleet("Model10")
print("Новые модели:", car8.model, car9.model, car10.model)
\end{lstlisting}

\item Написать программу на Python, которая создает класс `StudentGroup` с использованием метода `\_\_new\_\_` для ограничения количества создаваемых экземпляров до 30.

Инструкции:
\begin{enumerate}
    \item Создайте класс `StudentGroup`.
    \item Добавьте атрибут класса `\_students` и инициализируйте его пустым списком.
    \item Добавьте атрибут класса `GROUP\_MAX` и инициализируйте его значением 30.
    \item Переопределите метод `\_\_new\_\_`. Если `len(\_students) >= GROUP\_MAX`, выбросьте `RuntimeError("Группа заполнена!")`. Иначе, создайте экземпляр, добавьте в `\_students`, верните.
    \item Переопределите метод `\_\_del\_\_`, чтобы он удалял `self` из `\_students`.
    \item Переопределите метод `\_\_init\_\_`, который принимает `student\_name` и устанавливает `self.name = student\_name`.
    \item Создайте 30 экземпляров.
    \item Попытайтесь создать 31-й - поймайте и выведите исключение.
    \item Удалите пять экземпляров.
    \item Создайте 31-й, 32-й, 33-й, 34-й, 35-й экземпляры - должно сработать.
\end{enumerate}

Пример использования:
\begin{lstlisting}[language=Python]
# Создаем 30 студентов
students = [StudentGroup(f"Student{i}") for i in range(30)]

# Попытка создать 31-го
try:
    s31 = StudentGroup("Alice")
except RuntimeError as e:
    print("Ошибка:", e)

# Удаляем пять
for i in range(5):
    del students[0]

# Создаем 31-го, 32-го, 33-го, 34-го, 35-го - успешно
new_students = [StudentGroup(f"New{i}") for i in range(31, 36)]
for s in new_students:
    print("Добавлен:", s.name)
\end{lstlisting}

\item Написать программу на Python, которая создает класс `TaskQueue` с использованием метода `\_\_new\_\_` для ограничения количества создаваемых экземпляров до 100.

Инструкции:
\begin{enumerate}
    \item Создайте класс `TaskQueue`.
    \item Добавьте атрибут класса `\_tasks` и инициализируйте его пустым списком.
    \item Добавьте атрибут класса `QUEUE\_LIMIT` и инициализируйте его значением 100.
    \item Переопределите метод `\_\_new\_\_`. Если `len(\_tasks) >= QUEUE\_LIMIT`, выбросьте `RuntimeError("Очередь задач переполнена!")`. Иначе, создайте экземпляр, добавьте в `\_tasks`, верните.
    \item Переопределите метод `\_\_del\_\_`, чтобы он удалял `self` из `\_tasks`.
    \item Переопределите метод `\_\_init\_\_`, который принимает `task\_name` и устанавливает `self.task = task\_name`.
    \item Создайте 100 экземпляров.
    \item Попытайтесь создать 101-й - поймайте и выведите исключение.
    \item Удалите десять экземпляров.
    \item Создайте 101-й, 102-й, ..., 110-й экземпляры - должно сработать.
\end{enumerate}

Пример использования:
\begin{lstlisting}[language=Python]
# Создаем 100 задач
tasks = [TaskQueue(f"Task{i}") for i in range(100)]

# Попытка создать 101-ю
try:
    t101 = TaskQueue("FinalTask")
except RuntimeError as e:
    print("Ошибка:", e)

# Удаляем 10 задач
for i in range(10):
    del tasks[0]

# Создаем 101-ю, 102-ю, ..., 110-ю - успешно
new_tasks = [TaskQueue(f"NewTask{i}") for i in range(101, 111)]
for t in new_tasks:
    print("Добавлена задача:", t.task)
\end{lstlisting}

\item Написать программу на Python, которая создает класс `ConnectionPool` с использованием метода `\_\_new\_\_` для ограничения количества создаваемых экземпляров до 8.

Инструкции:
\begin{enumerate}
    \item Создайте класс `ConnectionPool`.
    \item Добавьте атрибут класса `\_connections` и инициализируйте его пустым списком.
    \item Добавьте атрибут класса `POOL\_SIZE` и инициализируйте его значением 8.
    \item Переопределите метод `\_\_new\_\_`. Если `len(\_connections) >= POOL\_SIZE`, выбросьте `RuntimeError("Пул соединений полон!")`. Иначе, создайте экземпляр, добавьте в `\_connections`, верните.
    \item Переопределите метод `\_\_del\_\_`, чтобы он удалял `self` из `\_connections`.
    \item Переопределите метод `\_\_init\_\_`, который принимает `connection\_id` и устанавливает `self.id = connection\_id`.
    \item Создайте 8 экземпляров.
    \item Попытайтесь создать 9-й - поймайте и выведите исключение.
    \item Удалите четыре экземпляра.
    \item Создайте 9-й, 10-й, 11-й, 12-й экземпляры - должно сработать.
\end{enumerate}

Пример использования:
\begin{lstlisting}[language=Python]
# Создаем 8 соединений
pool = [ConnectionPool(f"Conn{i}") for i in range(8)]

# Попытка создать 9-е
try:
    conn9 = ConnectionPool("Conn9")
except RuntimeError as e:
    print("Ошибка:", e)

# Удаляем 4
del pool[0], pool[1], pool[2], pool[3]

# Создаем 9-е, 10-е, 11-е, 12-е - успешно
new_conns = [ConnectionPool(f"Conn{i}") for i in range(9, 13)]
for c in new_conns:
    print("Создано соединение:", c.id)
\end{lstlisting}

\item Написать программу на Python, которая создает класс `DeviceManager` с использованием метода `\_\_new\_\_` для ограничения количества создаваемых экземпляров до 15.

Инструкции:
\begin{enumerate}
    \item Создайте класс `DeviceManager`.
    \item Добавьте атрибут класса `\_devices` и инициализируйте его пустым списком.
    \item Добавьте атрибут класса `MANAGER\_LIMIT` и инициализируйте его значением 15.
    \item Переопределите метод `\_\_new\_\_`. Если `len(\_devices) >= MANAGER\_LIMIT`, выбросьте `RuntimeError("Менеджер устройств перегружен!")`. Иначе, создайте экземпляр, добавьте в `\_devices`, верните.
    \item Переопределите метод `\_\_del\_\_`, чтобы он удалял `self` из `\_devices`.
    \item Переопределите метод `\_\_init\_\_`, который принимает `device\_name` и устанавливает `self.device = device\_name`.
    \item Создайте 15 экземпляров.
    \item Попытайтесь создать 16-й - поймайте и выведите исключение.
    \item Удалите семь экземпляров.
    \item Создайте 16-й, 17-й, ..., 22-й экземпляры - должно сработать.
\end{enumerate}

Пример использования:
\begin{lstlisting}[language=Python]
# Создаем 15 устройств
devices = [DeviceManager(f"Device{i}") for i in range(15)]

# Попытка создать 16-е
try:
    d16 = DeviceManager("NewDevice")
except RuntimeError as e:
    print("Ошибка:", e)

# Удаляем 7
for i in range(7):
    del devices[0]

# Создаем 16-е, 17-е, ..., 22-е - успешно
new_devices = [DeviceManager(f"Device{i}") for i in range(16, 23)]
for d in new_devices:
    print("Добавлено устройство:", d.device)
\end{lstlisting}

\item Написать программу на Python, которая создает класс `SessionPool` с использованием метода `\_\_new\_\_` для ограничения количества создаваемых экземпляров до 6.

Инструкции:
\begin{enumerate}
    \item Создайте класс `SessionPool`.
    \item Добавьте атрибут класса `\_sessions` и инициализируйте его пустым списком.
    \item Добавьте атрибут класса `SESSION\_LIMIT` и инициализируйте его значением 6.
    \item Переопределите метод `\_\_new\_\_`. Если `len(\_sessions) >= SESSION\_LIMIT`, выбросьте `RuntimeError("Пул сессий исчерпан!")`. Иначе, создайте экземпляр, добавьте в `\_sessions`, верните.
    \item Переопределите метод `\_\_del\_\_`, чтобы он удалял `self` из `\_sessions`.
    \item Переопределите метод `\_\_init\_\_`, который принимает `session\_token` и устанавливает `self.token = session\_token`.
    \item Создайте 6 экземпляров.
    \item Попытайтесь создать 7-й - поймайте и выведите исключение.
    \item Удалите два экземпляра.
    \item Создайте 7-й и 8-й экземпляры - должно сработать.
\end{enumerate}

Пример использования:
\begin{lstlisting}[language=Python]
# Создаем 6 сессий
sessions = [SessionPool(f"Token{i}") for i in range(6)]

# Попытка создать 7-ю
try:
    s7 = SessionPool("Token7")
except RuntimeError as e:
    print("Ошибка:", e)

# Удаляем 2
del sessions[0], sessions[1]

# Создаем 7-ю и 8-ю - успешно
s7 = SessionPool("Token7")
s8 = SessionPool("Token8")
print("Созданы токены:", s7.token, s8.token)
\end{lstlisting}

\item Написать программу на Python, которая создает класс `ThreadPool` с использованием метода `\_\_new\_\_` для ограничения количества создаваемых экземпляров до 12.

Инструкции:
\begin{enumerate}
    \item Создайте класс `ThreadPool`.
    \item Добавьте атрибут класса `\_threads` и инициализируйте его пустым списком.
    \item Добавьте атрибут класса `THREAD\_MAX` и инициализируйте его значением 12.
    \item Переопределите метод `\_\_new\_\_`. Если `len(\_threads) >= THREAD\_MAX`, выбросьте `RuntimeError("Достигнут лимит потоков!")`. Иначе, создайте экземпляр, добавьте в `\_threads`, верните.
    \item Переопределите метод `\_\_del\_\_`, чтобы он удалял `self` из `\_threads`.
    \item Переопределите метод `\_\_init\_\_`, который принимает `thread\_id` и устанавливает `self.thread = thread\_id`.
    \item Создайте 12 экземпляров.
    \item Попытайтесь создать 13-й - поймайте и выведите исключение.
    \item Удалите три экземпляра.
    \item Создайте 13-й, 14-й, 15-й экземпляры - должно сработать.
\end{enumerate}

Пример использования:
\begin{lstlisting}[language=Python]
# Создаем 12 потоков
threads = [ThreadPool(f"Thread{i}") for i in range(12)]

# Попытка создать 13-й
try:
    t13 = ThreadPool("Thread13")
except RuntimeError as e:
    print("Ошибка:", e)

# Удаляем 3
del threads[0], threads[1], threads[2]

# Создаем 13-й, 14-й, 15-й - успешно
t13 = ThreadPool("Thread13")
t14 = ThreadPool("Thread14")
t15 = ThreadPool("Thread15")
print("Созданы потоки:", t13.thread, t14.thread, t15.thread)
\end{lstlisting}

\item Написать программу на Python, которая создает класс `CachePool` с использованием метода `\_\_new\_\_` для ограничения количества создаваемых экземпляров до 20.

Инструкции:
\begin{enumerate}
    \item Создайте класс `CachePool`.
    \item Добавьте атрибут класса `\_caches` и инициализируйте его пустым списком.
    \item Добавьте атрибут класса `CACHE\_LIMIT` и инициализируйте его значением 20.
    \item Переопределите метод `\_\_new\_\_`. Если `len(\_caches) >= CACHE\_LIMIT`, выбросьте `RuntimeError("Кэш-пул переполнен!")`. Иначе, создайте экземпляр, добавьте в `\_caches`, верните.
    \item Переопределите метод `\_\_del\_\_`, чтобы он удалял `self` из `\_caches`.
    \item Переопределите метод `\_\_init\_\_`, который принимает `cache\_key` и устанавливает `self.key = cache\_key`.
    \item Создайте 20 экземпляров.
    \item Попытайтесь создать 21-й - поймайте и выведите исключение.
    \item Удалите пять экземпляров.
    \item Создайте 21-й, 22-й, ..., 25-й экземпляры - должно сработать.
\end{enumerate}

Пример использования:
\begin{lstlisting}[language=Python]
# Создаем 20 кэшей
caches = [CachePool(f"Key{i}") for i in range(20)]

# Попытка создать 21-й
try:
    c21 = CachePool("Key21")
except RuntimeError as e:
    print("Ошибка:", e)

# Удаляем 5
for i in range(5):
    del caches[0]

# Создаем 21-й, 22-й, ..., 25-й - успешно
new_caches = [CachePool(f"Key{i}") for i in range(21, 26)]
for c in new_caches:
    print("Создан ключ:", c.key)
\end{lstlisting}

\item Написать программу на Python, которая создает класс `DatabasePool` с использованием метода `\_\_new\_\_` для ограничения количества создаваемых экземпляров до 4.

Инструкции:
\begin{enumerate}
    \item Создайте класс `DatabasePool`.
    \item Добавьте атрибут класса `\_databases` и инициализируйте его пустым списком.
    \item Добавьте атрибут класса `DB\_LIMIT` и инициализируйте его значением 4.
    \item Переопределите метод `\_\_new\_\_`. Если `len(\_databases) >= DB\_LIMIT`, выбросьте `RuntimeError("Базы данных: лимит превышен!")`. Иначе, создайте экземпляр, добавьте в `\_databases`, верните.
    \item Переопределите метод `\_\_del\_\_`, чтобы он удалял `self` из `\_databases`.
    \item Переопределите метод `\_\_init\_\_`, который принимает `db\_name` и устанавливает `self.name = db\_name`.
    \item Создайте 4 экземпляра.
    \item Попытайтесь создать 5-й - поймайте и выведите исключение.
    \item Удалите один экземпляр.
    \item Создайте 5-й экземпляр - должно сработать.
\end{enumerate}

Пример использования:
\begin{lstlisting}[language=Python]
# Создаем 4 базы
dbs = [DatabasePool(f"DB{i}") for i in range(4)]

# Попытка создать 5-ю
try:
    db5 = DatabasePool("DB5")
except RuntimeError as e:
    print("Ошибка:", e)

# Удаляем одну
del dbs[0]

# Создаем 5-ю - успешно
db5 = DatabasePool("DB5")
print("Создана база:", db5.name)
\end{lstlisting}

\item Написать программу на Python, которая создает класс `FileHandlerPool` с использованием метода `\_\_new\_\_` для ограничения количества создаваемых экземпляров до 9.

Инструкции:
\begin{enumerate}
    \item Создайте класс `FileHandlerPool`.
    \item Добавьте атрибут класса `\_handlers` и инициализируйте его пустым списком.
    \item Добавьте атрибут класса `HANDLER\_MAX` и инициализируйте его значением 9.
    \item Переопределите метод `\_\_new\_\_`. Если `len(\_handlers) >= HANDLER\_MAX`, выбросьте `RuntimeError("Слишком много обработчиков файлов!")`. Иначе, создайте экземпляр, добавьте в `\_handlers`, верните.
    \item Переопределите метод `\_\_del\_\_`, чтобы он удалял `self` из `\_handlers`.
    \item Переопределите метод `\_\_init\_\_`, который принимает `file\_path` и устанавливает `self.path = file\_path`.
    \item Создайте 9 экземпляров.
    \item Попытайтесь создать 10-й - поймайте и выведите исключение.
    \item Удалите четыре экземпляра.
    \item Создайте 10-й, 11-й, 12-й, 13-й экземпляры - должно сработать.
\end{enumerate}

Пример использования:
\begin{lstlisting}[language=Python]
# Создаем 9 обработчиков
handlers = [FileHandlerPool(f"/path/to/file{i}.txt") for i in range(9)]

# Попытка создать 10-й
try:
    h10 = FileHandlerPool("/path/to/newfile.txt")
except RuntimeError as e:
    print("Ошибка:", e)

# Удаляем 4
del handlers[0], handlers[1], handlers[2], handlers[3]

# Создаем 10-й, 11-й, 12-й, 13-й - успешно
new_handlers = [FileHandlerPool(f"/path/to/newfile{i}.txt") for i in range(10, 14)]
for h in new_handlers:
    print("Обработчик для:", h.path)
\end{lstlisting}

\item Написать программу на Python, которая создает класс `NetworkPool` с использованием метода `\_\_new\_\_` для ограничения количества создаваемых экземпляров до 11.

Инструкции:
\begin{enumerate}
    \item Создайте класс `NetworkPool`.
    \item Добавьте атрибут класса `\_networks` и инициализируйте его пустым списком.
    \item Добавьте атрибут класса `NETWORK\_CAP` и инициализируйте его значением 11.
    \item Переопределите метод `\_\_new\_\_`. Если `len(\_networks) >= NETWORK\_CAP`, выбросьте `RuntimeError("Сеть: превышен лимит!")`. Иначе, создайте экземпляр, добавьте в `\_networks`, верните.
    \item Переопределите метод `\_\_del\_\_`, чтобы он удалял `self` из `\_networks`.
    \item Переопределите метод `\_\_init\_\_`, который принимает `network\_id` и устанавливает `self.net\_id = network\_id`.
    \item Создайте 11 экземпляров.
    \item Попытайтесь создать 12-й - поймайте и выведите исключение.
    \item Удалите шесть экземпляров.
    \item Создайте 12-й, 13-й, ..., 17-й экземпляры - должно сработать.
\end{enumerate}

Пример использования:
\begin{lstlisting}[language=Python]
# Создаем 11 сетей
networks = [NetworkPool(f"Net{i}") for i in range(11)]

# Попытка создать 12-ю
try:
    n12 = NetworkPool("Net12")
except RuntimeError as e:
    print("Ошибка:", e)

# Удаляем 6
for i in range(6):
    del networks[0]

# Создаем 12-ю, 13-ю, ..., 17-ю - успешно
new_networks = [NetworkPool(f"Net{i}") for i in range(12, 18)]
for n in new_networks:
    print("Создана сеть:", n.net_id)
\end{lstlisting}

\item Написать программу на Python, которая создает класс `MemoryPool` с использованием метода `\_\_new\_\_` для ограничения количества создаваемых экземпляров до 25.

Инструкции:
\begin{enumerate}
    \item Создайте класс `MemoryPool`.
    \item Добавьте атрибут класса `\_blocks` и инициализируйте его пустым списком.
    \item Добавьте атрибут класса `MEMORY\_LIMIT` и инициализируйте его значением 25.
    \item Переопределите метод `\_\_new\_\_`. Если `len(\_blocks) >= MEMORY\_LIMIT`, выбросьте `RuntimeError("Память: лимит блоков превышен!")`. Иначе, создайте экземпляр, добавьте в `\_blocks`, верните.
    \item Переопределите метод `\_\_del\_\_`, чтобы он удалял `self` из `\_blocks`.
    \item Переопределите метод `\_\_init\_\_`, который принимает `block\_size` и устанавливает `self.size = block\_size`.
    \item Создайте 25 экземпляров.
    \item Попытайтесь создать 26-й - поймайте и выведите исключение.
    \item Удалите десять экземпляров.
    \item Создайте 26-й, 27-й, ..., 35-й экземпляры - должно сработать.
\end{enumerate}

Пример использования:
\begin{lstlisting}[language=Python]
# Создаем 25 блоков
blocks = [MemoryPool(f"Size{i}") for i in range(25)]

# Попытка создать 26-й
try:
    b26 = MemoryPool("Size26")
except RuntimeError as e:
    print("Ошибка:", e)

# Удаляем 10
for i in range(10):
    del blocks[0]

# Создаем 26-й, 27-й, ..., 35-й - успешно
new_blocks = [MemoryPool(f"Size{i}") for i in range(26, 36)]
for b in new_blocks:
    print("Создан блок размером:", b.size)
\end{lstlisting}

\item Написать программу на Python, которая создает класс `ProcessPool` с использованием метода `\_\_new\_\_` для ограничения количества создаваемых экземпляров до 16.

Инструкции:
\begin{enumerate}
    \item Создайте класс `ProcessPool`.
    \item Добавьте атрибут класса `\_processes` и инициализируйте его пустым списком.
    \item Добавьте атрибут класса `PROCESS\_MAX` и инициализируйте его значением 16.
    \item Переопределите метод `\_\_new\_\_`. Если `len(\_processes) >= PROCESS\_MAX`, выбросьте `RuntimeError("Процессы: лимит превышен!")`. Иначе, создайте экземпляр, добавьте в `\_processes`, верните.
    \item Переопределите метод `\_\_del\_\_`, чтобы он удалял `self` из `\_processes`.
    \item Переопределите метод `\_\_init\_\_`, который принимает `process\_name` и устанавливает `self.name = process\_name`.
    \item Создайте 16 экземпляров.
    \item Попытайтесь создать 17-й - поймайте и выведите исключение.
    \item Удалите восемь экземпляров.
    \item Создайте 17-й, 18-й, ..., 24-й экземпляры - должно сработать.
\end{enumerate}

Пример использования:
\begin{lstlisting}[language=Python]
# Создаем 16 процессов
processes = [ProcessPool(f"Proc{i}") for i in range(16)]

# Попытка создать 17-й
try:
    p17 = ProcessPool("Proc17")
except RuntimeError as e:
    print("Ошибка:", e)

# Удаляем 8
for i in range(8):
    del processes[0]

# Создаем 17-й, 18-й, ..., 24-й - успешно
new_processes = [ProcessPool(f"Proc{i}") for i in range(17, 25)]
for p in new_processes:
    print("Запущен процесс:", p.name)
\end{lstlisting}

\item Написать программу на Python, которая создает класс `BufferPool` с использованием метода `\_\_new\_\_` для ограничения количества создаваемых экземпляров до 18.

Инструкции:
\begin{enumerate}
    \item Создайте класс `BufferPool`.
    \item Добавьте атрибут класса `\_buffers` и инициализируйте его пустым списком.
    \item Добавьте атрибут класса `BUFFER\_SIZE` и инициализируйте его значением 18.
    \item Переопределите метод `\_\_new\_\_`. Если `len(\_buffers) >= BUFFER\_SIZE`, выбросьте `RuntimeError("Буфер: переполнение!")`. Иначе, создайте экземпляр, добавьте в `\_buffers`, верните.
    \item Переопределите метод `\_\_del\_\_`, чтобы он удалял `self` из `\_buffers`.
    \item Переопределите метод `\_\_init\_\_`, который принимает `buffer\_id` и устанавливает `self.id = buffer\_id`.
    \item Создайте 18 экземпляров.
    \item Попытайтесь создать 19-й - поймайте и выведите исключение.
    \item Удалите девять экземпляров.
    \item Создайте 19-й, 20-й, ..., 27-й экземпляры - должно сработать.
\end{enumerate}

Пример использования:
\begin{lstlisting}[language=Python]
# Создаем 18 буферов
buffers = [BufferPool(f"Buf{i}") for i in range(18)]

# Попытка создать 19-й
try:
    b19 = BufferPool("Buf19")
except RuntimeError as e:
    print("Ошибка:", e)

# Удаляем 9
for i in range(9):
    del buffers[0]

# Создаем 19-й, 20-й, ..., 27-й - успешно
new_buffers = [BufferPool(f"Buf{i}") for i in range(19, 28)]
for b in new_buffers:
    print("Создан буфер:", b.id)
\end{lstlisting}

\item Написать программу на Python, которая создает класс `ChannelPool` с использованием метода `\_\_new\_\_` для ограничения количества создаваемых экземпляров до 13.

Инструкции:
\begin{enumerate}
    \item Создайте класс `ChannelPool`.
    \item Добавьте атрибут класса `\_channels` и инициализируйте его пустым списком.
    \item Добавьте атрибут класса `CHANNEL\_LIMIT` и инициализируйте его значением 13.
    \item Переопределите метод `\_\_new\_\_`. Если `len(\_channels) >= CHANNEL\_LIMIT`, выбросьте `RuntimeError("Каналы: лимит исчерпан!")`. Иначе, создайте экземпляр, добавьте в `\_channels`, верните.
    \item Переопределите метод `\_\_del\_\_`, чтобы он удалял `self` из `\_channels`.
    \item Переопределите метод `\_\_init\_\_`, который принимает `channel\_name` и устанавливает `self.name = channel\_name`.
    \item Создайте 13 экземпляров.
    \item Попытайтесь создать 14-й - поймайте и выведите исключение.
    \item Удалите три экземпляра.
    \item Создайте 14-й, 15-й, 16-й экземпляры - должно сработать.
\end{enumerate}

Пример использования:
\begin{lstlisting}[language=Python]
# Создаем 13 каналов
channels = [ChannelPool(f"Channel{i}") for i in range(13)]

# Попытка создать 14-й
try:
    c14 = ChannelPool("Channel14")
except RuntimeError as e:
    print("Ошибка:", e)

# Удаляем 3
del channels[0], channels[1], channels[2]

# Создаем 14-й, 15-й, 16-й - успешно
c14 = ChannelPool("Channel14")
c15 = ChannelPool("Channel15")
c16 = ChannelPool("Channel16")
print("Созданы каналы:", c14.name, c15.name, c16.name)
\end{lstlisting}

\item Написать программу на Python, которая создает класс `SocketPool` с использованием метода `\_\_new\_\_` для ограничения количества создаваемых экземпляров до 22.

Инструкции:
\begin{enumerate}
    \item Создайте класс `SocketPool`.
    \item Добавьте атрибут класса `\_sockets` и инициализируйте его пустым списком.
    \item Добавьте атрибут класса `SOCKET\_MAX` и инициализируйте его значением 22.
    \item Переопределите метод `\_\_new\_\_`. Если `len(\_sockets) >= SOCKET\_MAX`, выбросьте `RuntimeError("Сокеты: лимит превышен!")`. Иначе, создайте экземпляр, добавьте в `\_sockets`, верните.
    \item Переопределите метод `\_\_del\_\_`, чтобы он удалял `self` из `\_sockets`.
    \item Переопределите метод `\_\_init\_\_`, который принимает `socket\_port` и устанавливает `self.port = socket\_port`.
    \item Создайте 22 экземпляра.
    \item Попытайтесь создать 23-й - поймайте и выведите исключение.
    \item Удалите одиннадцать экземпляров.
    \item Создайте 23-й, 24-й, ..., 33-й экземпляры - должно сработать.
\end{enumerate}

Пример использования:
\begin{lstlisting}[language=Python]
# Создаем 22 сокета
sockets = [SocketPool(8000 + i) for i in range(22)]

# Попытка создать 23-й
try:
    s23 = SocketPool(8022)
except RuntimeError as e:
    print("Ошибка:", e)

# Удаляем 11
for i in range(11):
    del sockets[0]

# Создаем 23-й, 24-й, ..., 33-й - успешно
new_sockets = [SocketPool(8022 + i) for i in range(11)]
for s in new_sockets:
    print("Создан сокет на порту:", s.port)
\end{lstlisting}

\item Написать программу на Python, которая создает класс `LockPool` с использованием метода `\_\_new\_\_` для ограничения количества создаваемых экземпляров до 14.

Инструкции:
\begin{enumerate}
    \item Создайте класс `LockPool`.
    \item Добавьте атрибут класса `\_locks` и инициализируйте его пустым списком.
    \item Добавьте атрибут класса `LOCK\_COUNT` и инициализируйте его значением 14.
    \item Переопределите метод `\_\_new\_\_`. Если `len(\_locks) >= LOCK\_COUNT`, выбросьте `RuntimeError("Замки: все заняты!")`. Иначе, создайте экземпляр, добавьте в `\_locks`, верните.
    \item Переопределите метод `\_\_del\_\_`, чтобы он удалял `self` из `\_locks`.
    \item Переопределите метод `\_\_init\_\_`, который принимает `lock\_name` и устанавливает `self.name = lock\_name`.
    \item Создайте 14 экземпляров.
    \item Попытайтесь создать 15-й - поймайте и выведите исключение.
    \item Удалите семь экземпляров.
    \item Создайте 15-й, 16-й, ..., 21-й экземпляры - должно сработать.
\end{enumerate}

Пример использования:
\begin{lstlisting}[language=Python]
# Создаем 14 замков
locks = [LockPool(f"Lock{i}") for i in range(14)]

# Попытка создать 15-й
try:
    l15 = LockPool("Lock15")
except RuntimeError as e:
    print("Ошибка:", e)

# Удаляем 7
for i in range(7):
    del locks[0]

# Создаем 15-й, 16-й, ..., 21-й - успешно
new_locks = [LockPool(f"Lock{i}") for i in range(15, 22)]
for l in new_locks:
    print("Создан замок:", l.name)
\end{lstlisting}

\item Написать программу на Python, которая создает класс `QueuePool` с использованием метода `\_\_new\_\_` для ограничения количества создаваемых экземпляров до 19.

Инструкции:
\begin{enumerate}
    \item Создайте класс `QueuePool`.
    \item Добавьте атрибут класса `\_queues` и инициализируйте его пустым списком.
    \item Добавьте атрибут класса `QUEUE\_COUNT` и инициализируйте его значением 19.
    \item Переопределите метод `\_\_new\_\_`. Если `len(\_queues) >= QUEUE\_COUNT`, выбросьте `RuntimeError("Очереди: лимит достигнут!")`. Иначе, создайте экземпляр, добавьте в `\_queues`, верните.
    \item Переопределите метод `\_\_del\_\_`, чтобы он удалял `self` из `\_queues`.
    \item Переопределите метод `\_\_init\_\_`, который принимает `queue\_name` и устанавливает `self.name = queue\_name`.
    \item Создайте 19 экземпляров.
    \item Попытайтесь создать 20-й - поймайте и выведите исключение.
    \item Удалите десять экземпляров.
    \item Создайте 20-й, 21-й, ..., 29-й экземпляры - должно сработать.
\end{enumerate}

Пример использования:
\begin{lstlisting}[language=Python]
# Создаем 19 очередей
queues = [QueuePool(f"Queue{i}") for i in range(19)]

# Попытка создать 20-ю
try:
    q20 = QueuePool("Queue20")
except RuntimeError as e:
    print("Ошибка:", e)

# Удаляем 10
for i in range(10):
    del queues[0]

# Создаем 20-ю, 21-ю, ..., 29-ю - успешно
new_queues = [QueuePool(f"Queue{i}") for i in range(20, 30)]
for q in new_queues:
    print("Создана очередь:", q.name)
\end{lstlisting}

\item Написать программу на Python, которая создает класс `SemaphorePool` с использованием метода `\_\_new\_\_` для ограничения количества создаваемых экземпляров до 8.

Инструкции:
\begin{enumerate}
    \item Создайте класс `SemaphorePool`.
    \item Добавьте атрибут класса `\_semaphores` и инициализируйте его пустым списком.
    \item Добавьте атрибут класса `SEMA\_LIMIT` и инициализируйте его значением 8.
    \item Переопределите метод `\_\_new\_\_`. Если `len(\_semaphores) >= SEMA\_LIMIT`, выбросьте `RuntimeError("Семафоры: лимит превышен!")`. Иначе, создайте экземпляр, добавьте в `\_semaphores`, верните.
    \item Переопределите метод `\_\_del\_\_`, чтобы он удалял `self` из `\_semaphores`.
    \item Переопределите метод `\_\_init\_\_`, который принимает `sema\_id` и устанавливает `self.id = sema\_id`.
    \item Создайте 8 экземпляров.
    \item Попытайтесь создать 9-й - поймайте и выведите исключение.
    \item Удалите четыре экземпляра.
    \item Создайте 9-й, 10-й, 11-й, 12-й экземпляры - должно сработать.
\end{enumerate}

Пример использования:
\begin{lstlisting}[language=Python]
# Создаем 8 семафоров
semas = [SemaphorePool(f"Sema{i}") for i in range(8)]

# Попытка создать 9-й
try:
    s9 = SemaphorePool("Sema9")
except RuntimeError as e:
    print("Ошибка:", e)

# Удаляем 4
del semas[0], semas[1], semas[2], semas[3]

# Создаем 9-й, 10-й, 11-й, 12-й - успешно
s9 = SemaphorePool("Sema9")
s10 = SemaphorePool("Sema10")
s11 = SemaphorePool("Sema11")
s12 = SemaphorePool("Sema12")
print("Созданы семафоры:", s9.id, s10.id, s11.id, s12.id)
\end{lstlisting}

\item Написать программу на Python, которая создает класс `TimerPool` с использованием метода `\_\_new\_\_` для ограничения количества создаваемых экземпляров до 21.

Инструкции:
\begin{enumerate}
    \item Создайте класс `TimerPool`.
    \item Добавьте атрибут класса `\_timers` и инициализируйте его пустым списком.
    \item Добавьте атрибут класса `TIMER\_MAX` и инициализируйте его значением 21.
    \item Переопределите метод `\_\_new\_\_`. Если `len(\_timers) >= TIMER\_MAX`, выбросьте `RuntimeError("Таймеры: лимит исчерпан!")`. Иначе, создайте экземпляр, добавьте в `\_timers`, верните.
    \item Переопределите метод `\_\_del\_\_`, чтобы он удалял `self` из `\_timers`.
    \item Переопределите метод `\_\_init\_\_`, который принимает `timer\_duration` и устанавливает `self.duration = timer\_duration`.
    \item Создайте 21 экземпляр.
    \item Попытайтесь создать 22-й - поймайте и выведите исключение.
    \item Удалите одиннадцать экземпляров.
    \item Создайте 22-й, 23-й, ..., 32-й экземпляры - должно сработать.
\end{enumerate}

Пример использования:
\begin{lstlisting}[language=Python]
# Создаем 21 таймер
timers = [TimerPool(i * 10) for i in range(21)]

# Попытка создать 22-й
try:
    t22 = TimerPool(220)
except RuntimeError as e:
    print("Ошибка:", e)

# Удаляем 11
for i in range(11):
    del timers[0]

# Создаем 22-й, 23-й, ..., 32-й - успешно
new_timers = [TimerPool(i * 10) for i in range(22, 33)]
for t in new_timers:
    print("Создан таймер на:", t.duration, "сек")
\end{lstlisting}

\item Написать программу на Python, которая создает класс `WorkerPool` с использованием метода `\_\_new\_\_` для ограничения количества создаваемых экземпляров до 23.

Инструкции:
\begin{enumerate}
    \item Создайте класс `WorkerPool`.
    \item Добавьте атрибут класса `\_workers` и инициализируйте его пустым списком.
    \item Добавьте атрибут класса `WORKER\_LIMIT` и инициализируйте его значением 23.
    \item Переопределите метод `\_\_new\_\_`. Если `len(\_workers) >= WORKER\_LIMIT`, выбросьте `RuntimeError("Рабочие: лимит превышен!")`. Иначе, создайте экземпляр, добавьте в `\_workers`, верните.
    \item Переопределите метод `\_\_del\_\_`, чтобы он удалял `self` из `\_workers`.
    \item Переопределите метод `\_\_init\_\_`, который принимает `worker\_id` и устанавливает `self.id = worker\_id`.
    \item Создайте 23 экземпляра.
    \item Попытайтесь создать 24-й - поймайте и выведите исключение.
    \item Удалите двенадцать экземпляров.
    \item Создайте 24-й, 25-й, ..., 35-й экземпляры - должно сработать.
\end{enumerate}

Пример использования:
\begin{lstlisting}[language=Python]
# Создаем 23 рабочих
workers = [WorkerPool(f"Worker{i}") for i in range(23)]

# Попытка создать 24-го
try:
    w24 = WorkerPool("Worker24")
except RuntimeError as e:
    print("Ошибка:", e)

# Удаляем 12
for i in range(12):
    del workers[0]

# Создаем 24-го, 25-го, ..., 35-го - успешно
new_workers = [WorkerPool(f"Worker{i}") for i in range(24, 36)]
for w in new_workers:
    print("Создан рабочий:", w.id)
\end{lstlisting}

\item Написать программу на Python, которая создает класс `JobPool` с использованием метода `\_\_new\_\_` для ограничения количества создаваемых экземпляров до 26.

Инструкции:
\begin{enumerate}
    \item Создайте класс `JobPool`.
    \item Добавьте атрибут класса `\_jobs` и инициализируйте его пустым списком.
    \item Добавьте атрибут класса `JOB\_CAP` и инициализируйте его значением 26.
    \item Переопределите метод `\_\_new\_\_`. Если `len(\_jobs) >= JOB\_CAP`, выбросьте `RuntimeError("Задания: лимит превышен!")`. Иначе, создайте экземпляр, добавьте в `\_jobs`, верните.
    \item Переопределите метод `\_\_del\_\_`, чтобы он удалял `self` из `\_jobs`.
    \item Переопределите метод `\_\_init\_\_`, который принимает `job\_name` и устанавливает `self.name = job\_name`.
    \item Создайте 26 экземпляров.
    \item Попытайтесь создать 27-й - поймайте и выведите исключение.
    \item Удалите тринадцать экземпляров.
    \item Создайте 27-й, 28-й, ..., 39-й экземпляры - должно сработать.
\end{enumerate}

Пример использования:
\begin{lstlisting}[language=Python]
# Создаем 26 заданий
jobs = [JobPool(f"Job{i}") for i in range(26)]

# Попытка создать 27-е
try:
    j27 = JobPool("Job27")
except RuntimeError as e:
    print("Ошибка:", e)

# Удаляем 13
for i in range(13):
    del jobs[0]

# Создаем 27-е, 28-е, ..., 39-е - успешно
new_jobs = [JobPool(f"Job{i}") for i in range(27, 40)]
for j in new_jobs:
    print("Создано задание:", j.name)
\end{lstlisting}

\item Написать программу на Python, которая создает класс `RequestPool` с использованием метода `\_\_new\_\_` для ограничения количества создаваемых экземпляров до 27.

Инструкции:
\begin{enumerate}
    \item Создайте класс `RequestPool`.
    \item Добавьте атрибут класса `\_requests` и инициализируйте его пустым списком.
    \item Добавьте атрибут класса `REQUEST\_MAX` и инициализируйте его значением 27.
    \item Переопределите метод `\_\_new\_\_`. Если `len(\_requests) >= REQUEST\_MAX`, выбросьте `RuntimeError("Запросы: лимит превышен!")`. Иначе, создайте экземпляр, добавьте в `\_requests`, верните.
    \item Переопределите метод `\_\_del\_\_`, чтобы он удалял `self` из `\_requests`.
    \item Переопределите метод `\_\_init\_\_`, который принимает `request\_url` и устанавливает `self.url = request\_url`.
    \item Создайте 27 экземпляров.
    \item Попытайтесь создать 28-й - поймайте и выведите исключение.
    \item Удалите четырнадцать экземпляров.
    \item Создайте 28-й, 29-й, ..., 41-й экземпляры - должно сработать.
\end{enumerate}

Пример использования:
\begin{lstlisting}[language=Python]
# Создаем 27 запросов
requests = [RequestPool(f"http://site{i}.com") for i in range(27)]

# Попытка создать 28-й
try:
    r28 = RequestPool("http://newsite.com")
except RuntimeError as e:
    print("Ошибка:", e)

# Удаляем 14
for i in range(14):
    del requests[0]

# Создаем 28-й, 29-й, ..., 41-й - успешно
new_requests = [RequestPool(f"http://newsite{i}.com") for i in range(28, 42)]
for r in new_requests:
    print("Создан запрос к:", r.url)
\end{lstlisting}

\item Написать программу на Python, которая создает класс `EventPool` с использованием метода `\_\_new\_\_` для ограничения количества создаваемых экземпляров до 28.

Инструкции:
\begin{enumerate}
    \item Создайте класс `EventPool`.
    \item Добавьте атрибут класса `\_events` и инициализируйте его пустым списком.
    \item Добавьте атрибут класса `EVENT\_LIMIT` и инициализируйте его значением 28.
    \item Переопределите метод `\_\_new\_\_`. Если `len(\_events) >= EVENT\_LIMIT`, выбросьте `RuntimeError("События: лимит превышен!")`. Иначе, создайте экземпляр, добавьте в `\_events`, верните.
    \item Переопределите метод `\_\_del\_\_`, чтобы он удалял `self` из `\_events`.
    \item Переопределите метод `\_\_init\_\_`, который принимает `event\_type` и устанавливает `self.type = event\_type`.
    \item Создайте 28 экземпляров.
    \item Попытайтесь создать 29-е - поймайте и выведите исключение.
    \item Удалите пятнадцать экземпляров.
    \item Создайте 29-е, 30-е, ..., 43-е экземпляры - должно сработать.
\end{enumerate}

Пример использования:
\begin{lstlisting}[language=Python]
# Создаем 28 событий
events = [EventPool(f"Event{i}") for i in range(28)]

# Попытка создать 29-е
try:
    e29 = EventPool("Event29")
except RuntimeError as e:
    print("Ошибка:", e)

# Удаляем 15
for i in range(15):
    del events[0]

# Создаем 29-е, 30-е, ..., 43-е - успешно
new_events = [EventPool(f"Event{i}") for i in range(29, 44)]
for e in new_events:
    print("Создано событие типа:", e.type)
\end{lstlisting}

\item Написать программу на Python, которая создает класс `MessagePool` с использованием метода `\_\_new\_\_` для ограничения количества создаваемых экземпляров до 29.

Инструкции:
\begin{enumerate}
    \item Создайте класс `MessagePool`.
    \item Добавьте атрибут класса `\_messages` и инициализируйте его пустым списком.
    \item Добавьте атрибут класса `MSG\_MAX` и инициализируйте его значением 29.
    \item Переопределите метод `\_\_new\_\_`. Если `len(\_messages) >= MSG\_MAX`, выбросьте `RuntimeError("Сообщения: лимит превышен!")`. Иначе, создайте экземпляр, добавьте в `\_messages`, верните.
    \item Переопределите метод `\_\_del\_\_`, чтобы он удалял `self` из `\_messages`.
    \item Переопределите метод `\_\_init\_\_`, который принимает `message\_text` и устанавливает `self.text = message\_text`.
    \item Создайте 29 экземпляров.
    \item Попытайтесь создать 30-й - поймайте и выведите исключение.
    \item Удалите шестнадцать экземпляров.
    \item Создайте 30-й, 31-й, ..., 45-й экземпляры - должно сработать.
\end{enumerate}

Пример использования:
\begin{lstlisting}[language=Python]
# Создаем 29 сообщений
messages = [MessagePool(f"Message{i}") for i in range(29)]

# Попытка создать 30-е
try:
    m30 = MessagePool("Message30")
except RuntimeError as e:
    print("Ошибка:", e)

# Удаляем 16
for i in range(16):
    del messages[0]

# Создаем 30-е, 31-е, ..., 45-е - успешно
new_messages = [MessagePool(f"Message{i}") for i in range(30, 46)]
for m in new_messages:
    print("Создано сообщение:", m.text)
\end{lstlisting}

\item Написать программу на Python, которая создает класс `NotificationPool` с использованием метода `\_\_new\_\_` для ограничения количества создаваемых экземпляров до 31.

Инструкции:
\begin{enumerate}
    \item Создайте класс `NotificationPool`.
    \item Добавьте атрибут класса `\_notifications` и инициализируйте его пустым списком.
    \item Добавьте атрибут класса `NOTIF\_LIMIT` и инициализируйте его значением 31.
    \item Переопределите метод `\_\_new\_\_`. Если `len(\_notifications) >= NOTIF\_LIMIT`, выбросьте `RuntimeError("Уведомления: лимит превышен!")`. Иначе, создайте экземпляр, добавьте в `\_notifications`, верните.
    \item Переопределите метод `\_\_del\_\_`, чтобы он удалял `self` из `\_notifications`.
    \item Переопределите метод `\_\_init\_\_`, который принимает `notification\_title` и устанавливает `self.title = notification\_title`.
    \item Создайте 31 экземпляр.
    \item Попытайтесь создать 32-й - поймайте и выведите исключение.
    \item Удалите семнадцать экземпляров.
    \item Создайте 32-й, 33-й, ..., 48-й экземпляры - должно сработать.
\end{enumerate}

Пример использования:
\begin{lstlisting}[language=Python]
# Создаем 31 уведомление
notifications = [NotificationPool(f"Notif{i}") for i in range(31)]

# Попытка создать 32-е
try:
    n32 = NotificationPool("Notif32")
except RuntimeError as e:
    print("Ошибка:", e)

# Удаляем 17
for i in range(17):
    del notifications[0]

# Создаем 32-е, 33-е, ..., 48-е - успешно
new_notifications = [NotificationPool(f"Notif{i}") for i in range(32, 49)]
for n in new_notifications:
    print("Создано уведомление:", n.title)
\end{lstlisting}
\item Написать программу на Python, которая создает класс `LoggerPool` с использованием метода `\_\_new\_\_` для ограничения количества создаваемых экземпляров до 5.

Инструкции:
\begin{enumerate}
    \item Создайте класс `LoggerPool`.
    \item Добавьте атрибут класса `\_loggers` и инициализируйте его пустым списком.
    \item Добавьте атрибут класса `LOGGER\_LIMIT` и инициализируйте его значением 5.
    \item Переопределите метод `\_\_new\_\_`. Если `len(\_loggers) >= LOGGER\_LIMIT`, выбросьте `RuntimeError("Логгеры: лимит превышен!")`. Иначе, создайте экземпляр, добавьте в `\_loggers`, верните.
    \item Переопределите метод `\_\_del\_\_`, чтобы он удалял `self` из `\_loggers`.
    \item Переопределите метод `\_\_init\_\_`, который принимает `logger\_name` и устанавливает `self.name = logger\_name`.
    \item Создайте 5 экземпляров.
    \item Попытайтесь создать 6-й - поймайте и выведите исключение.
    \item Удалите два экземпляра.
    \item Создайте 6-й и 7-й экземпляры - должно сработать.
\end{enumerate}

Пример использования:
\begin{lstlisting}[language=Python]
# Создаем 5 логгеров
loggers = [LoggerPool(f"Logger{i}") for i in range(5)]

# Попытка создать 6-й
try:
    l6 = LoggerPool("Logger6")
except RuntimeError as e:
    print("Ошибка:", e)

# Удаляем 2
del loggers[0], loggers[1]

# Создаем 6-й и 7-й - успешно
l6 = LoggerPool("Logger6")
l7 = LoggerPool("Logger7")
print("Созданы логгеры:", l6.name, l7.name)
\end{lstlisting}

\item Написать программу на Python, которая создает класс `ConfigPool` с использованием метода `\_\_new\_\_` для ограничения количества создаваемых экземпляров до 12.

Инструкции:
\begin{enumerate}
    \item Создайте класс `ConfigPool`.
    \item Добавьте атрибут класса `\_configs` и инициализируйте его пустым списком.
    \item Добавьте атрибут класса `CONFIG\_MAX` и инициализируйте его значением 12.
    \item Переопределите метод `\_\_new\_\_`. Если `len(\_configs) >= CONFIG\_MAX`, выбросьте `RuntimeError("Конфигурации: лимит превышен!")`. Иначе, создайте экземпляр, добавьте в `\_configs`, верните.
    \item Переопределите метод `\_\_del\_\_`, чтобы он удалял `self` из `\_configs`.
    \item Переопределите метод `\_\_init\_\_`, который принимает `config\_name` и устанавливает `self.name = config\_name`.
    \item Создайте 12 экземпляров.
    \item Попытайтесь создать 13-й - поймайте и выведите исключение.
    \item Удалите шесть экземпляров.
    \item Создайте 13-й, 14-й, ..., 18-й экземпляры - должно сработать.
\end{enumerate}

Пример использования:
\begin{lstlisting}[language=Python]
# Создаем 12 конфигураций
configs = [ConfigPool(f"Config{i}") for i in range(12)]

# Попытка создать 13-ю
try:
    c13 = ConfigPool("Config13")
except RuntimeError as e:
    print("Ошибка:", e)

# Удаляем 6
for i in range(6):
    del configs[0]

# Создаем 13-ю, 14-ю, ..., 18-ю - успешно
new_configs = [ConfigPool(f"Config{i}") for i in range(13, 19)]
for c in new_configs:
    print("Создана конфигурация:", c.name)
\end{lstlisting}

\item Написать программу на Python, которая создает класс `PluginPool` с использованием метода `\_\_new\_\_` для ограничения количества создаваемых экземпляров до 10.

Инструкции:
\begin{enumerate}
    \item Создайте класс `PluginPool`.
    \item Добавьте атрибут класса `\_plugins` и инициализируйте его пустым списком.
    \item Добавьте атрибут класса `PLUGIN\_CAP` и инициализируйте его значением 10.
    \item Переопределите метод `\_\_new\_\_`. Если `len(\_plugins) >= PLUGIN\_CAP`, выбросьте `RuntimeError("Плагины: лимит исчерпан!")`. Иначе, создайте экземпляр, добавьте в `\_plugins`, верните.
    \item Переопределите метод `\_\_del\_\_`, чтобы он удалял `self` из `\_plugins`.
    \item Переопределите метод `\_\_init\_\_`, который принимает `plugin\_id` и устанавливает `self.id = plugin\_id`.
    \item Создайте 10 экземпляров.
    \item Попытайтесь создать 11-й - поймайте и выведите исключение.
    \item Удалите пять экземпляров.
    \item Создайте 11-й, 12-й, ..., 15-й экземпляры - должно сработать.
\end{enumerate}

Пример использования:
\begin{lstlisting}[language=Python]
# Создаем 10 плагинов
plugins = [PluginPool(f"Plugin{i}") for i in range(10)]

# Попытка создать 11-й
try:
    p11 = PluginPool("Plugin11")
except RuntimeError as e:
    print("Ошибка:", e)

# Удаляем 5
for i in range(5):
    del plugins[0]

# Создаем 11-й, 12-й, ..., 15-й - успешно
new_plugins = [PluginPool(f"Plugin{i}") for i in range(11, 16)]
for p in new_plugins:
    print("Создан плагин:", p.id)
\end{lstlisting}

\item Написать программу на Python, которая создает класс `ServicePool` с использованием метода `\_\_new\_\_` для ограничения количества создаваемых экземпляров до 8.

Инструкции:
\begin{enumerate}
    \item Создайте класс `ServicePool`.
    \item Добавьте атрибут класса `\_services` и инициализируйте его пустым списком.
    \item Добавьте атрибут класса `SERVICE\_LIMIT` и инициализируйте его значением 8.
    \item Переопределите метод `\_\_new\_\_`. Если `len(\_services) >= SERVICE\_LIMIT`, выбросьте `RuntimeError("Сервисы: лимит превышен!")`. Иначе, создайте экземпляр, добавьте в `\_services`, верните.
    \item Переопределите метод `\_\_del\_\_`, чтобы он удалял `self` из `\_services`.
    \item Переопределите метод `\_\_init\_\_`, который принимает `service\_name` и устанавливает `self.name = service\_name`.
    \item Создайте 8 экземпляров.
    \item Попытайтесь создать 9-й - поймайте и выведите исключение.
    \item Удалите четыре экземпляра.
    \item Создайте 9-й, 10-й, 11-й, 12-й экземпляры - должно сработать.
\end{enumerate}

Пример использования:
\begin{lstlisting}[language=Python]
# Создаем 8 сервисов
services = [ServicePool(f"Service{i}") for i in range(8)]

# Попытка создать 9-й
try:
    s9 = ServicePool("Service9")
except RuntimeError as e:
    print("Ошибка:", e)

# Удаляем 4
del services[0], services[1], services[2], services[3]

# Создаем 9-й, 10-й, 11-й, 12-й - успешно
s9 = ServicePool("Service9")
s10 = ServicePool("Service10")
s11 = ServicePool("Service11")
s12 = ServicePool("Service12")
print("Созданы сервисы:", s9.name, s10.name, s11.name, s12.name)
\end{lstlisting}

\item Написать программу на Python, которая создает класс `CacheEntryPool` с использованием метода `\_\_new\_\_` для ограничения количества создаваемых экземпляров до 15.

Инструкции:
\begin{enumerate}
    \item Создайте класс `CacheEntryPool`.
    \item Добавьте атрибут класса `\_entries` и инициализируйте его пустым списком.
    \item Добавьте атрибут класса `ENTRY\_MAX` и инициализируйте его значением 15.
    \item Переопределите метод `\_\_new\_\_`. Если `len(\_entries) >= ENTRY\_MAX`, выбросьте `RuntimeError("Кэш-записи: лимит превышен!")`. Иначе, создайте экземпляр, добавьте в `\_entries`, верните.
    \item Переопределите метод `\_\_del\_\_`, чтобы он удалял `self` из `\_entries`.
    \item Переопределите метод `\_\_init\_\_`, который принимает `entry\_key` и устанавливает `self.key = entry\_key`.
    \item Создайте 15 экземпляров.
    \item Попытайтесь создать 16-й - поймайте и выведите исключение.
    \item Удалите семь экземпляров.
    \item Создайте 16-й, 17-й, ..., 22-й экземпляры - должно сработать.
\end{enumerate}

Пример использования:
\begin{lstlisting}[language=Python]
# Создаем 15 записей
entries = [CacheEntryPool(f"Key{i}") for i in range(15)]

# Попытка создать 16-ю
try:
    e16 = CacheEntryPool("Key16")
except RuntimeError as e:
    print("Ошибка:", e)

# Удаляем 7
for i in range(7):
    del entries[0]

# Создаем 16-ю, 17-ю, ..., 22-ю - успешно
new_entries = [CacheEntryPool(f"Key{i}") for i in range(16, 23)]
for e in new_entries:
    print("Создана запись с ключом:", e.key)
\end{lstlisting}

\item Написать программу на Python, которая создает класс `ConnectionHandlerPool` с использованием метода `\_\_new\_\_` для ограничения количества создаваемых экземпляров до 20.

Инструкции:
\begin{enumerate}
    \item Создайте класс `ConnectionHandlerPool`.
    \item Добавьте атрибут класса `\_handlers` и инициализируйте его пустым списком.
    \item Добавьте атрибут класса `HANDLER\_LIMIT` и инициализируйте его значением 20.
    \item Переопределите метод `\_\_new\_\_`. Если `len(\_handlers) >= HANDLER\_LIMIT`, выбросьте `RuntimeError("Обработчики соединений: лимит превышен!")`. Иначе, создайте экземпляр, добавьте в `\_handlers`, верните.
    \item Переопределите метод `\_\_del\_\_`, чтобы он удалял `self` из `\_handlers`.
    \item Переопределите метод `\_\_init\_\_`, который принимает `handler\_id` и устанавливает `self.id = handler\_id`.
    \item Создайте 20 экземпляров.
    \item Попытайтесь создать 21-й - поймайте и выведите исключение.
    \item Удалите десять экземпляров.
    \item Создайте 21-й, 22-й, ..., 30-й экземпляры - должно сработать.
\end{enumerate}

Пример использования:
\begin{lstlisting}[language=Python]
# Создаем 20 обработчиков
handlers = [ConnectionHandlerPool(f"H{i}") for i in range(20)]

# Попытка создать 21-й
try:
    h21 = ConnectionHandlerPool("H21")
except RuntimeError as e:
    print("Ошибка:", e)

# Удаляем 10
for i in range(10):
    del handlers[0]

# Создаем 21-й, 22-й, ..., 30-й - успешно
new_handlers = [ConnectionHandlerPool(f"H{i}") for i in range(21, 31)]
for h in new_handlers:
    print("Создан обработчик:", h.id)
\end{lstlisting}

\end{enumerate}
\subsubsection{Задача 3 (именование)}

\begin{enumerate}
\item Написать программу на Python, которая создает класс `Vagon` с использованием метода `\_\_new\_\_` для контроля именования. Имена должны начинаться с "vagon\_". Метод `\_\_init\_\_` должен быть пустым.

Инструкции:
\begin{enumerate}
    \item Создайте класс `Vagon`.
    \item Добавьте атрибут класса `numbers` и инициализируйте его пустым словарем.
    \item Переопределите метод `\_\_new\_\_`, принимающий `cls`, `name`, `number`.
    \item В `\_\_new\_\_`: если `name` не начинается с "vagon\_", выбросьте `ValueError("Имя должно начинаться с 'vagon\_'")`.
    \item Извлеките номер вагона: `vagon\_number = name[6:]` (удаляем "vagon\_").
    \item Создайте экземпляр: `instance = super().\_\_new\_\_(cls)`.
    \item Добавьте номер в словарь: `cls.numbers[vagon\_number] = instance`.
    \item Установите атрибут экземпляра: `setattr(instance, f"v{vagon\_number}", number)`.
    \item Верните `instance`.
    \item Переопределите метод `\_\_init\_\_` как пустой: `def \_\_init\_\_(self, *args, **kwargs): pass`.
    \item Создайте объект `v1` с именем "vagon\_1" и номером 101.
    \item Создайте объект `v2` с именем "vagon\_2" и номером 102.
    \item Попытайтесь создать объект с именем "car\_3" — должно возникнуть исключение `ValueError`.
    \item Выведите `v1.v1` и `v2.v2`.
    \item Выведите `Vagon.numbers`.
\end{enumerate}

Пример использования:
\begin{lstlisting}[language=Python]
v1 = Vagon("vagon_1", 101)
v2 = Vagon("vagon_2", 102)

try:
    v3 = Vagon("car_3", 103)
except ValueError as e:
    print("Ошибка:", e)

print("v1.v1:", v1.v1)  # 101
print("v2.v2:", v2.v2)  # 102
print("Vagon.numbers:", Vagon.numbers)
\end{lstlisting}

\item Написать программу на Python, которая создает класс `Room` с использованием метода `\_\_new\_\_` для контроля именования. Имена должны начинаться с "room\_". Метод `\_\_init\_\_` должен быть пустым.

Инструкции:
\begin{enumerate}
    \item Создайте класс `Room`.
    \item Добавьте атрибут класса `registry` и инициализируйте его пустым словарем.
    \item Переопределите метод `\_\_new\_\_`, принимающий `cls`, `name`, `capacity`.
    \item В `\_\_new\_\_`: если `name` не начинается с "room\_", выбросьте `ValueError("Недопустимое имя комнаты")`.
    \item Извлеките номер комнаты: `room\_num = name[5:]`.
    \item Создайте экземпляр: `instance = super().\_\_new\_\_(cls)`.
    \item Добавьте номер в словарь: `cls.registry[room\_num] = instance`.
    \item Установите атрибут экземпляра: `setattr(instance, f"r{room\_num}", capacity)`.
    \item Верните `instance`.
    \item Переопределите метод `\_\_init\_\_` как пустой.
    \item Создайте объект `r1` с именем "room\_101" и вместимостью 50.
    \item Создайте объект `r2` с именем "room\_202" и вместимостью 30.
    \item Попытайтесь создать объект с именем "hall\_A" — поймайте исключение.
    \item Выведите `r1.r101` и `r2.r202`.
    \item Выведите `Room.registry`.
\end{enumerate}

Пример использования:
\begin{lstlisting}[language=Python]
r1 = Room("room_101", 50)
r2 = Room("room_202", 30)

try:
    r3 = Room("hall_A", 100)
except ValueError as e:
    print("Ошибка:", e)

print("r1.r101:", r1.r101)  # 50
print("r2.r202:", r2.r202)  # 30
print("Room.registry:", Room.registry)
\end{lstlisting}

\item Написать программу на Python, которая создает класс `Device` с использованием метода `\_\_new\_\_` для контроля именования. Имена должны начинаться с "dev\_". Метод `\_\_init\_\_` должен быть пустым.

Инструкции:
\begin{enumerate}
    \item Создайте класс `Device`.
    \item Добавьте атрибут класса `inventory` и инициализируйте его пустым словарем.
    \item Переопределите метод `\_\_new\_\_`, принимающий `cls`, `name`, `model`.
    \item В `\_\_new\_\_`: если `name` не начинается с "dev\_", выбросьте `ValueError("Неверный префикс устройства")`.
    \item Извлеките ID устройства: `dev\_id = name[4:]`.
    \item Создайте экземпляр: `instance = super().\_\_new\_\_(cls)`.
    \item Добавьте ID в словарь: `cls.inventory[dev\_id] = instance`.
    \item Установите атрибут экземпляра: `setattr(instance, f"d{dev\_id}", model)`.
    \item Верните `instance`.
    \item Переопределите метод `\_\_init\_\_` как пустой.
    \item Создайте объект `d1` с именем "dev\_001" и моделью "X1".
    \item Создайте объект `d2` с именем "dev\_002" и моделью "Y2".
    \item Попытайтесь создать объект с именем "sensor\_01" — поймайте исключение.
    \item Выведите `d1.d001` и `d2.d002`.
    \item Выведите `Device.inventory`.
\end{enumerate}

Пример использования:
\begin{lstlisting}[language=Python]
d1 = Device("dev_001", "X1")
d2 = Device("dev_002", "Y2")

try:
    d3 = Device("sensor_01", "Z3")
except ValueError as e:
    print("Ошибка:", e)

print("d1.d001:", d1.d001)  # X1
print("d2.d002:", d2.d002)  # Y2
print("Device.inventory:", Device.inventory)
\end{lstlisting}

\item Написать программу на Python, которая создает класс `Book` с использованием метода `\_\_new\_\_` для контроля именования. Имена должны начинаться с "book\_". Метод `\_\_init\_\_` должен быть пустым.

Инструкции:
\begin{enumerate}
    \item Создайте класс `Book`.
    \item Добавьте атрибут класса `catalog` и инициализируйте его пустым словарем.
    \item Переопределите метод `\_\_new\_\_`, принимающий `cls`, `name`, `author`.
    \item В `\_\_new\_\_`: если `name` не начинается с "book\_", выбросьте `ValueError("Книга должна иметь префикс 'book\_'")`.
    \item Извлеките ID книги: `book\_id = name[5:]`.
    \item Создайте экземпляр: `instance = super().\_\_new\_\_(cls)`.
    \item Добавьте ID в словарь: `cls.catalog[book\_id] = instance`.
    \item Установите атрибут экземпляра: `setattr(instance, f"b{book\_id}", author)`.
    \item Верните `instance`.
    \item Переопределите метод `\_\_init\_\_` как пустой.
    \item Создайте объект `b1` с именем "book\_001" и автором "Толстой".
    \item Создайте объект `b2` с именем "book\_002" и автором "Достоевский".
    \item Попытайтесь создать объект с именем "magazine\_01" — поймайте исключение.
    \item Выведите `b1.b001` и `b2.b002`.
    \item Выведите `Book.catalog`.
\end{enumerate}

Пример использования:
\begin{lstlisting}[language=Python]
b1 = Book("book_001", "Толстой")
b2 = Book("book_002", "Достоевский")

try:
    b3 = Book("magazine_01", "Пушкин")
except ValueError as e:
    print("Ошибка:", e)

print("b1.b001:", b1.b001)  # Толстой
print("b2.b002:", b2.b002)  # Достоевский
print("Book.catalog:", Book.catalog)
\end{lstlisting}

\item Написать программу на Python, которая создает класс `File` с использованием метода `\_\_new\_\_` для контроля именования. Имена должны начинаться с "file\_". Метод `\_\_init\_\_` должен быть пустым.

Инструкции:
\begin{enumerate}
    \item Создайте класс `File`.
    \item Добавьте атрибут класса `index` и инициализируйте его пустым словарем.
    \item Переопределите метод `\_\_new\_\_`, принимающий `cls`, `name`, `size`.
    \item В `\_\_new\_\_`: если `name` не начинается с "file\_", выбросьте `ValueError("Файл должен иметь префикс 'file\_'")`.
    \item Извлеките ID файла: `file\_id = name[5:]`.
    \item Создайте экземпляр: `instance = super().\_\_new\_\_(cls)`.
    \item Добавьте ID в словарь: `cls.index[file\_id] = instance`.
    \item Установите атрибут экземпляра: `setattr(instance, f"f{file\_id}", size)`.
    \item Верните `instance`.
    \item Переопределите метод `\_\_init\_\_` как пустой.
    \item Создайте объект `f1` с именем "file\_config" и размером 1024.
    \item Создайте объект `f2` с именем "file\_data" и размером 2048.
    \item Попытайтесь создать объект с именем "document\_1" — поймайте исключение.
    \item Выведите `f1.fconfig` и `f2.fdata`.
    \item Выведите `File.index`.
\end{enumerate}

Пример использования:
\begin{lstlisting}[language=Python]
f1 = File("file_config", 1024)
f2 = File("file_data", 2048)

try:
    f3 = File("document_1", 512)
except ValueError as e:
    print("Ошибка:", e)

print("f1.fconfig:", f1.fconfig)  # 1024
print("f2.fdata:", f2.fdata)      # 2048
print("File.index:", File.index)
\end{lstlisting}

\item Написать программу на Python, которая создает класс `User` с использованием метода `\_\_new\_\_` для контроля именования. Имена должны начинаться с "user\_". Метод `\_\_init\_\_` должен быть пустым.

Инструкции:
\begin{enumerate}
    \item Создайте класс `User`.
    \item Добавьте атрибут класса `directory` и инициализируйте его пустым словарем.
    \item Переопределите метод `\_\_new\_\_`, принимающий `cls`, `name`, `email`.
    \item В `\_\_new\_\_`: если `name` не начинается с "user\_", выбросьте `ValueError("Пользователь должен иметь префикс 'user\_'")`.
    \item Извлеките ID пользователя: `user\_id = name[5:]`.
    \item Создайте экземпляр: `instance = super().\_\_new\_\_(cls)`.
    \item Добавьте ID в словарь: `cls.directory[user\_id] = instance`.
    \item Установите атрибут экземпляра: `setattr(instance, f"u{user\_id}", email)`.
    \item Верните `instance`.
    \item Переопределите метод `\_\_init\_\_` как пустой.
    \item Создайте объект `u1` с именем "user\_alice" и email "alice@example.com".
    \item Создайте объект `u2` с именем "user\_bob" и email "bob@example.com".
    \item Попытайтесь создать объект с именем "admin\_john" — поймайте исключение.
    \item Выведите `u1.ualice` и `u2.ubob`.
    \item Выведите `User.directory`.
\end{enumerate}

Пример использования:
\begin{lstlisting}[language=Python]
u1 = User("user_alice", "alice@example.com")
u2 = User("user_bob", "bob@example.com")

try:
    u3 = User("admin_john", "john@example.com")
except ValueError as e:
    print("Ошибка:", e)

print("u1.ualice:", u1.ualice)  # alice@example.com
print("u2.ubob:", u2.ubob)      # bob@example.com
print("User.directory:", User.directory)
\end{lstlisting}

\item Написать программу на Python, которая создает класс `Product` с использованием метода `\_\_new\_\_` для контроля именования. Имена должны начинаться с "prod\_". Метод `\_\_init\_\_` должен быть пустым.

Инструкции:
\begin{enumerate}
    \item Создайте класс `Product`.
    \item Добавьте атрибут класса `warehouse` и инициализируйте его пустым словарем.
    \item Переопределите метод `\_\_new\_\_`, принимающий `cls`, `name`, `price`.
    \item В `\_\_new\_\_`: если `name` не начинается с "prod\_", выбросьте `ValueError("Продукт должен иметь префикс 'prod\_'")`.
    \item Извлеките ID продукта: `prod\_id = name[5:]`.
    \item Создайте экземпляр: `instance = super().\_\_new\_\_(cls)`.
    \item Добавьте ID в словарь: `cls.warehouse[prod\_id] = instance`.
    \item Установите атрибут экземпляра: `setattr(instance, f"p{prod\_id}", price)`.
    \item Верните `instance`.
    \item Переопределите метод `\_\_init\_\_` как пустой.
    \item Создайте объект `p1` с именем "prod\_laptop" и ценой 999.
    \item Создайте объект `p2` с именем "prod\_mouse" и ценой 25.
    \item Попытайтесь создать объект с именем "item\_keyboard" — поймайте исключение.
    \item Выведите `p1.plaptop` и `p2.pmouse`.
    \item Выведите `Product.warehouse`.
\end{enumerate}

Пример использования:
\begin{lstlisting}[language=Python]
p1 = Product("prod_laptop", 999)
p2 = Product("prod_mouse", 25)

try:
    p3 = Product("item_keyboard", 50)
except ValueError as e:
    print("Ошибка:", e)

print("p1.plaptop:", p1.plaptop)  # 999
print("p2.pmouse:", p2.pmouse)    # 25
print("Product.warehouse:", Product.warehouse)
\end{lstlisting}

\item Написать программу на Python, которая создает класс `Employee` с использованием метода `\_\_new\_\_` для контроля именования. Имена должны начинаться с "emp\_". Метод `\_\_init\_\_` должен быть пустым.

Инструкции:
\begin{enumerate}
    \item Создайте класс `Employee`.
    \item Добавьте атрибут класса `staff` и инициализируйте его пустым словарем.
    \item Переопределите метод `\_\_new\_\_`, принимающий `cls`, `name`, `department`.
    \item В `\_\_new\_\_`: если `name` не начинается с "emp\_", выбросьте `ValueError("Сотрудник должен иметь префикс 'emp\_'")`.
    \item Извлеките ID сотрудника: `emp\_id = name[4:]`.
    \item Создайте экземпляр: `instance = super().\_\_new\_\_(cls)`.
    \item Добавьте ID в словарь: `cls.staff[emp\_id] = instance`.
    \item Установите атрибут экземпляра: `setattr(instance, f"e{emp\_id}", department)`.
    \item Верните `instance`.
    \item Переопределите метод `\_\_init\_\_` как пустой.
    \item Создайте объект `e1` с именем "emp\_001" и отделом "IT".
    \item Создайте объект `e2` с именем "emp\_002" и отделом "HR".
    \item Попытайтесь создать объект с именем "worker\_003" — поймайте исключение.
    \item Выведите `e1.e001` и `e2.e002`.
    \item Выведите `Employee.staff`.
\end{enumerate}

Пример использования:
\begin{lstlisting}[language=Python]
e1 = Employee("emp_001", "IT")
e2 = Employee("emp_002", "HR")

try:
    e3 = Employee("worker_003", "Sales")
except ValueError as e:
    print("Ошибка:", e)

print("e1.e001:", e1.e001)  # IT
print("e2.e002:", e2.e002)  # HR
print("Employee.staff:", Employee.staff)
\end{lstlisting}

\item Написать программу на Python, которая создает класс `Order` с использованием метода `\_\_new\_\_` для контроля именования. Имена должны начинаться с "order\_". Метод `\_\_init\_\_` должен быть пустым.

Инструкции:
\begin{enumerate}
    \item Создайте класс `Order`.
    \item Добавьте атрибут класса `ledger` и инициализируйте его пустым словарем.
    \item Переопределите метод `\_\_new\_\_`, принимающий `cls`, `name`, `total`.
    \item В `\_\_new\_\_`: если `name` не начинается с "order\_", выбросьте `ValueError("Заказ должен иметь префикс 'order\_'")`.
    \item Извлеките ID заказа: `order\_id = name[6:]`.
    \item Создайте экземпляр: `instance = super().\_\_new\_\_(cls)`.
    \item Добавьте ID в словарь: `cls.ledger[order\_id] = instance`.
    \item Установите атрибут экземпляра: `setattr(instance, f"o{order\_id}", total)`.
    \item Верните `instance`.
    \item Переопределите метод `\_\_init\_\_` как пустой.
    \item Создайте объект `o1` с именем "order\_1001" и суммой 150.0.
    \item Создайте объект `o2` с именем "order\_1002" и суммой 89.99.
    \item Попытайтесь создать объект с именем "purchase\_1003" — поймайте исключение.
    \item Выведите `o1.o1001` и `o2.o1002`.
    \item Выведите `Order.ledger`.
\end{enumerate}

Пример использования:
\begin{lstlisting}[language=Python]
o1 = Order("order_1001", 150.0)
o2 = Order("order_1002", 89.99)

try:
    o3 = Order("purchase_1003", 200.0)
except ValueError as e:
    print("Ошибка:", e)

print("o1.o1001:", o1.o1001)  # 150.0
print("o2.o1002:", o2.o1002)  # 89.99
print("Order.ledger:", Order.ledger)
\end{lstlisting}

\item Написать программу на Python, которая создает класс `Ticket` с использованием метода `\_\_new\_\_` для контроля именования. Имена должны начинаться с "ticket\_". Метод `\_\_init\_\_` должен быть пустым.

Инструкции:
\begin{enumerate}
    \item Создайте класс `Ticket`.
    \item Добавьте атрибут класса `database` и инициализируйте его пустым словарем.
    \item Переопределите метод `\_\_new\_\_`, принимающий `cls`, `name`, `priority`.
    \item В `\_\_new\_\_`: если `name` не начинается с "ticket\_", выбросьте `ValueError("Тикет должен иметь префикс 'ticket\_'")`.
    \item Извлеките ID тикета: `ticket\_id = name[7:]`.
    \item Создайте экземпляр: `instance = super().\_\_new\_\_(cls)`.
    \item Добавьте ID в словарь: `cls.database[ticket\_id] = instance`.
    \item Установите атрибут экземпляра: `setattr(instance, f"t{ticket\_id}", priority)`.
    \item Верните `instance`.
    \item Переопределите метод `\_\_init\_\_` как пустой.
    \item Создайте объект `t1` с именем "ticket\_001" и приоритетом "High".
    \item Создайте объект `t2` с именем "ticket\_002" и приоритетом "Low".
    \item Попытайтесь создать объект с именем "issue\_003" — поймайте исключение.
    \item Выведите `t1.t001` и `t2.t002`.
    \item Выведите `Ticket.database`.
\end{enumerate}

Пример использования:
\begin{lstlisting}[language=Python]
t1 = Ticket("ticket_001", "High")
t2 = Ticket("ticket_002", "Low")

try:
    t3 = Ticket("issue_003", "Medium")
except ValueError as e:
    print("Ошибка:", e)

print("t1.t001:", t1.t001)  # High
print("t2.t002:", t2.t002)  # Low
print("Ticket.database:", Ticket.database)
\end{lstlisting}

\item Написать программу на Python, которая создает класс `Project` с использованием метода `\_\_new\_\_` для контроля именования. Имена должны начинаться с "proj\_". Метод `\_\_init\_\_` должен быть пустым.

Инструкции:
\begin{enumerate}
    \item Создайте класс `Project`.
    \item Добавьте атрибут класса `portfolio` и инициализируйте его пустым словарем.
    \item Переопределите метод `\_\_new\_\_`, принимающий `cls`, `name`, `status`.
    \item В `\_\_new\_\_`: если `name` не начинается с "proj\_", выбросьте `ValueError("Проект должен иметь префикс 'proj\_'")`.
    \item Извлеките ID проекта: `proj\_id = name[5:]`.
    \item Создайте экземпляр: `instance = super().\_\_new\_\_(cls)`.
    \item Добавьте ID в словарь: `cls.portfolio[proj\_id] = instance`.
    \item Установите атрибут экземпляра: `setattr(instance, f"pr{proj\_id}", status)`.
    \item Верните `instance`.
    \item Переопределите метод `\_\_init\_\_` как пустой.
    \item Создайте объект `pr1` с именем "proj\_alpha" и статусом "Active".
    \item Создайте объект `pr2` с именем "proj\_beta" и статусом "Inactive".
    \item Попытайтесь создать объект с именем "task\_gamma" — поймайте исключение.
    \item Выведите `pr1.pralpha` и `pr2.prbeta`.
    \item Выведите `Project.portfolio`.
\end{enumerate}

Пример использования:
\begin{lstlisting}[language=Python]
pr1 = Project("proj_alpha", "Active")
pr2 = Project("proj_beta", "Inactive")

try:
    pr3 = Project("task_gamma", "Pending")
except ValueError as e:
    print("Ошибка:", e)

print("pr1.pralpha:", pr1.pralpha)  # Active
print("pr2.prbeta:", pr2.prbeta)    # Inactive
print("Project.portfolio:", Project.portfolio)
\end{lstlisting}

\item Написать программу на Python, которая создает класс `Sensor` с использованием метода `\_\_new\_\_` для контроля именования. Имена должны начинаться с "sensor\_". Метод `\_\_init\_\_` должен быть пустым.

Инструкции:
\begin{enumerate}
    \item Создайте класс `Sensor`.
    \item Добавьте атрибут класса `registry` и инициализируйте его пустым словарем.
    \item Переопределите метод `\_\_new\_\_`, принимающий `cls`, `name`, `type`.
    \item В `\_\_new\_\_`: если `name` не начинается с "sensor\_", выбросьте `ValueError("Сенсор должен иметь префикс 'sensor\_'")`.
    \item Извлеките ID сенсора: `sensor\_id = name[7:]`.
    \item Создайте экземпляр: `instance = super().\_\_new\_\_(cls)`.
    \item Добавьте ID в словарь: `cls.registry[sensor\_id] = instance`.
    \item Установите атрибут экземпляра: `setattr(instance, f"s{sensor\_id}", type)`.
    \item Верните `instance`.
    \item Переопределите метод `\_\_init\_\_` как пустой.
    \item Создайте объект `s1` с именем "sensor\_temp" и типом "Temperature".
    \item Создайте объект `s2` с именем "sensor\_humid" и типом "Humidity".
    \item Попытайтесь создать объект с именем "device\_press" — поймайте исключение.
    \item Выведите `s1.stemp` и `s2.shumid`.
    \item Выведите `Sensor.registry`.
\end{enumerate}

Пример использования:
\begin{lstlisting}[language=Python]
s1 = Sensor("sensor_temp", "Temperature")
s2 = Sensor("sensor_humid", "Humidity")

try:
    s3 = Sensor("device_press", "Pressure")
except ValueError as e:
    print("Ошибка:", e)

print("s1.stemp:", s1.stemp)    # Temperature
print("s2.shumid:", s2.shumid)  # Humidity
print("Sensor.registry:", Sensor.registry)
\end{lstlisting}

\item Написать программу на Python, которая создает класс `Vehicle` с использованием метода `\_\_new\_\_` для контроля именования. Имена должны начинаться с "veh\_". Метод `\_\_init\_\_` должен быть пустым.

Инструкции:
\begin{enumerate}
    \item Создайте класс `Vehicle`.
    \item Добавьте атрибут класса `fleet` и инициализируйте его пустым словарем.
    \item Переопределите метод `\_\_new\_\_`, принимающий `cls`, `name`, `model`.
    \item В `\_\_new\_\_`: если `name` не начинается с "veh\_", выбросьте `ValueError("Транспортное средство должно иметь префикс 'veh\_'")`.
    \item Извлеките ID транспорта: `veh\_id = name[4:]`.
    \item Создайте экземпляр: `instance = super().\_\_new\_\_(cls)`.
    \item Добавьте ID в словарь: `cls.fleet[veh\_id] = instance`.
    \item Установите атрибут экземпляра: `setattr(instance, f"v{veh\_id}", model)`.
    \item Верните `instance`.
    \item Переопределите метод `\_\_init\_\_` как пустой.
    \item Создайте объект `v1` с именем "veh\_car1" и моделью "Sedan".
    \item Создайте объект `v2` с именем "veh\_truck1" и моделью "Pickup".
    \item Попытайтесь создать объект с именем "bike\_01" — поймайте исключение.
    \item Выведите `v1.vcar1` и `v2.vtruck1`.
    \item Выведите `Vehicle.fleet`.
\end{enumerate}

Пример использования:
\begin{lstlisting}[language=Python]
v1 = Vehicle("veh_car1", "Sedan")
v2 = Vehicle("veh_truck1", "Pickup")

try:
    v3 = Vehicle("bike_01", "Mountain")
except ValueError as e:
    print("Ошибка:", e)

print("v1.vcar1:", v1.vcar1)    # Sedan
print("v2.vtruck1:", v2.vtruck1) # Pickup
print("Vehicle.fleet:", Vehicle.fleet)
\end{lstlisting}

\item Написать программу на Python, которая создает класс `Animal` с использованием метода `\_\_new\_\_` для контроля именования. Имена должны начинаться с "animal\_". Метод `\_\_init\_\_` должен быть пустым.

Инструкции:
\begin{enumerate}
    \item Создайте класс `Animal`.
    \item Добавьте атрибут класса `zoo` и инициализируйте его пустым словарем.
    \item Переопределите метод `\_\_new\_\_`, принимающий `cls`, `name`, `species`.
    \item В `\_\_new\_\_`: если `name` не начинается с "animal\_", выбросьте `ValueError("Животное должно иметь префикс 'animal\_'")`.
    \item Извлеките ID животного: `animal\_id = name[7:]`.
    \item Создайте экземпляр: `instance = super().\_\_new\_\_(cls)`.
    \item Добавьте ID в словарь: `cls.zoo[animal\_id] = instance`.
    \item Установите атрибут экземпляра: `setattr(instance, f"a{animal\_id}", species)`.
    \item Верните `instance`.
    \item Переопределите метод `\_\_init\_\_` как пустой.
    \item Создайте объект `a1` с именем "animal\_lion" и видом "Panthera leo".
    \item Создайте объект `a2` с именем "animal\_elephant" и видом "Loxodonta".
    \item Попытайтесь создать объект с именем "creature\_tiger" — поймайте исключение.
    \item Выведите `a1.alion` и `a2.aelephant`.
    \item Выведите `Animal.zoo`.
\end{enumerate}

Пример использования:
\begin{lstlisting}[language=Python]
a1 = Animal("animal_lion", "Panthera leo")
a2 = Animal("animal_elephant", "Loxodonta")

try:
    a3 = Animal("creature_tiger", "Panthera tigris")
except ValueError as e:
    print("Ошибка:", e)

print("a1.alion:", a1.alion)        # Panthera leo
print("a2.aelephant:", a2.aelephant) # Loxodonta
print("Animal.zoo:", Animal.zoo)
\end{lstlisting}

\item Написать программу на Python, которая создает класс `Plant` с использованием метода `\_\_new\_\_` для контроля именования. Имена должны начинаться с "plant\_". Метод `\_\_init\_\_` должен быть пустым.

Инструкции:
\begin{enumerate}
    \item Создайте класс `Plant`.
    \item Добавьте атрибут класса `greenhouse` и инициализируйте его пустым словарем.
    \item Переопределите метод `\_\_new\_\_`, принимающий `cls`, `name`, `family`.
    \item В `\_\_new\_\_`: если `name` не начинается с "plant\_", выбросьте `ValueError("Растение должно иметь префикс 'plant\_'")`.
    \item Извлеките ID растения: `plant\_id = name[6:]`.
    \item Создайте экземпляр: `instance = super().\_\_new\_\_(cls)`.
    \item Добавьте ID в словарь: `cls.greenhouse[plant\_id] = instance`.
    \item Установите атрибут экземпляра: `setattr(instance, f"pl{plant\_id}", family)`.
    \item Верните `instance`.
    \item Переопределите метод `\_\_init\_\_` как пустой.
    \item Создайте объект `pl1` с именем "plant\_rose" и семейством "Rosaceae".
    \item Создайте объект `pl2` с именем "plant\_oak" и семейством "Fagaceae".
    \item Попытайтесь создать объект с именем "tree\_pine" — поймайте исключение.
    \item Выведите `pl1.plrose` и `pl2.ploak`.
    \item Выведите `Plant.greenhouse`.
\end{enumerate}

Пример использования:
\begin{lstlisting}[language=Python]
pl1 = Plant("plant_rose", "Rosaceae")
pl2 = Plant("plant_oak", "Fagaceae")

try:
    pl3 = Plant("tree_pine", "Pinaceae")
except ValueError as e:
    print("Ошибка:", e)

print("pl1.plrose:", pl1.plrose)  # Rosaceae
print("pl2.ploak:", pl2.ploak)    # Fagaceae
print("Plant.greenhouse:", Plant.greenhouse)
\end{lstlisting}

\item Написать программу на Python, которая создает класс `Planet` с использованием метода `\_\_new\_\_` для контроля именования. Имена должны начинаться с "planet\_". Метод `\_\_init\_\_` должен быть пустым.

Инструкции:
\begin{enumerate}
    \item Создайте класс `Planet`.
    \item Добавьте атрибут класса `solar\_system` и инициализируйте его пустым словарем.
    \item Переопределите метод `\_\_new\_\_`, принимающий `cls`, `name`, `type`.
    \item В `\_\_new\_\_`: если `name` не начинается с "planet\_", выбросьте `ValueError("Планета должна иметь префикс 'planet\_'")`.
    \item Извлеките ID планеты: `planet\_id = name[7:]`.
    \item Создайте экземпляр: `instance = super().\_\_new\_\_(cls)`.
    \item Добавьте ID в словарь: `cls.solar\_system[planet\_id] = instance`.
    \item Установите атрибут экземпляра: `setattr(instance, f"pn{planet\_id}", type)`.
    \item Верните `instance`.
    \item Переопределите метод `\_\_init\_\_` как пустой.
    \item Создайте объект `pn1` с именем "planet\_earth" и типом "Terrestrial".
    \item Создайте объект `pn2` с именем "planet\_jupiter" и типом "Gas Giant".
    \item Попытайтесь создать объект с именем "star\_sun" — поймайте исключение.
    \item Выведите `pn1.pnearth` и `pn2.pnjupiter`.
    \item Выведите `Planet.solar\_system`.
\end{enumerate}

Пример использования:
\begin{lstlisting}[language=Python]
pn1 = Planet("planet_earth", "Terrestrial")
pn2 = Planet("planet_jupiter", "Gas Giant")

try:
    pn3 = Planet("star_sun", "Star")
except ValueError as e:
    print("Ошибка:", e)

print("pn1.pnearth:", pn1.pnearth)    # Terrestrial
print("pn2.pnjupiter:", pn2.pnjupiter) # Gas Giant
print("Planet.solar_system:", Planet.solar_system)
\end{lstlisting}

\item Написать программу на Python, которая создает класс `Star` с использованием метода `\_\_new\_\_` для контроля именования. Имена должны начинаться с "star\_". Метод `\_\_init\_\_` должен быть пустым.

Инструкции:
\begin{enumerate}
    \item Создайте класс `Star`.
    \item Добавьте атрибут класса `galaxy` и инициализируйте его пустым словарем.
    \item Переопределите метод `\_\_new\_\_`, принимающий `cls`, `name`, `class\_type`.
    \item В `\_\_new\_\_`: если `name` не начинается с "star\_", выбросьте `ValueError("Звезда должна иметь префикс 'star\_'")`.
    \item Извлеките ID звезды: `star\_id = name[5:]`.
    \item Создайте экземпляр: `instance = super().\_\_new\_\_(cls)`.
    \item Добавьте ID в словарь: `cls.galaxy[star\_id] = instance`.
    \item Установите атрибут экземпляра: `setattr(instance, f"st{star\_id}", class\_type)`.
    \item Верните `instance`.
    \item Переопределите метод `\_\_init\_\_` как пустой.
    \item Создайте объект `st1` с именем "star\_sol" и классом "G2V".
    \item Создайте объект `st2` с именем "star\_proxima" и классом "M5.5V".
    \item Попытайтесь создать объект с именем "nova\_1" — поймайте исключение.
    \item Выведите `st1.stsol` и `st2.stproxima`.
    \item Выведите `Star.galaxy`.
\end{enumerate}

Пример использования:
\begin{lstlisting}[language=Python]
st1 = Star("star_sol", "G2V")
st2 = Star("star_proxima", "M5.5V")

try:
    st3 = Star("nova_1", "Variable")
except ValueError as e:
    print("Ошибка:", e)

print("st1.stsol:", st1.stsol)      # G2V
print("st2.stproxima:", st2.stproxima) # M5.5V
print("Star.galaxy:", Star.galaxy)
\end{lstlisting}

\item Написать программу на Python, которая создает класс `Galaxy` с использованием метода `\_\_new\_\_` для контроля именования. Имена должны начинаться с "galaxy\_". Метод `\_\_init\_\_` должен быть пустым.

Инструкции:
\begin{enumerate}
    \item Создайте класс `Galaxy`.
    \item Добавьте атрибут класса `universe` и инициализируйте его пустым словарем.
    \item Переопределите метод `\_\_new\_\_`, принимающий `cls`, `name`, `type`.
    \item В `\_\_new\_\_`: если `name` не начинается с "galaxy\_", выбросьте `ValueError("Галактика должна иметь префикс 'galaxy\_'")`.
    \item Извлеките ID галактики: `galaxy\_id = name[7:]`.
    \item Создайте экземпляр: `instance = super().\_\_new\_\_(cls)`.
    \item Добавьте ID в словарь: `cls.universe[galaxy\_id] = instance`.
    \item Установите атрибут экземпляра: `setattr(instance, f"g{galaxy\_id}", type)`.
    \item Верните `instance`.
    \item Переопределите метод `\_\_init\_\_` как пустой.
    \item Создайте объект `g1` с именем "galaxy\_milkyway" и типом "Spiral".
    \item Создайте объект `g2` с именем "galaxy\_andromeda" и типом "Spiral".
    \item Попытайтесь создать объект с именем "cluster\_virgo" — поймайте исключение.
    \item Выведите `g1.gmilkyway` и `g2.gandromeda`.
    \item Выведите `Galaxy.universe`.
\end{enumerate}

Пример использования:
\begin{lstlisting}[language=Python]
g1 = Galaxy("galaxy_milkyway", "Spiral")
g2 = Galaxy("galaxy_andromeda", "Spiral")

try:
    g3 = Galaxy("cluster_virgo", "Cluster")
except ValueError as e:
    print("Ошибка:", e)

print("g1.gmilkyway:", g1.gmilkyway)    # Spiral
print("g2.gandromeda:", g2.gandromeda)  # Spiral
print("Galaxy.universe:", Galaxy.universe)
\end{lstlisting}

\item Написать программу на Python, которая создает класс `Constellation` с использованием метода `\_\_new\_\_` для контроля именования. Имена должны начинаться с "const\_". Метод `\_\_init\_\_` должен быть пустым.

Инструкции:
\begin{enumerate}
    \item Создайте класс `Constellation`.
    \item Добавьте атрибут класса `sky\_map` и инициализируйте его пустым словарем.
    \item Переопределите метод `\_\_new\_\_`, принимающий `cls`, `name`, `stars`.
    \item В `\_\_new\_\_`: если `name` не начинается с "const\_", выбросьте `ValueError("Созвездие должно иметь префикс 'const\_'")`.
    \item Извлеките ID созвездия: `const\_id = name[6:]`.
    \item Создайте экземпляр: `instance = super().\_\_new\_\_(cls)`.
    \item Добавьте ID в словарь: `cls.sky\_map[const\_id] = instance`.
    \item Установите атрибут экземпляра: `setattr(instance, f"c{const\_id}", stars)`.
    \item Верните `instance`.
    \item Переопределите метод `\_\_init\_\_` как пустой.
    \item Создайте объект `c1` с именем "const\_orion" и количеством звезд 81.
    \item Создайте объект `c2` с именем "const\_ursa" и количеством звезд 20.
    \item Попытайтесь создать объект с именем "asterism\_bigdipper" — поймайте исключение.
    \item Выведите `c1.corion` и `c2.cursa`.
    \item Выведите `Constellation.sky\_map`.
\end{enumerate}

Пример использования:
\begin{lstlisting}[language=Python]
c1 = Constellation("const_orion", 81)
c2 = Constellation("const_ursa", 20)

try:
    c3 = Constellation("asterism_bigdipper", 7)
except ValueError as e:
    print("Ошибка:", e)

print("c1.corion:", c1.corion)  # 81
print("c2.cursa:", c2.cursa)    # 20
print("Constellation.sky_map:", Constellation.sky_map)
\end{lstlisting}

\item Написать программу на Python, которая создает класс `Asteroid` с использованием метода `\_\_new\_\_` для контроля именования. Имена должны начинаться с "ast\_". Метод `\_\_init\_\_` должен быть пустым.

Инструкции:
\begin{enumerate}
    \item Создайте класс `Asteroid`.
    \item Добавьте атрибут класса `belt` и инициализируйте его пустым словарем.
    \item Переопределите метод `\_\_new\_\_`, принимающий `cls`, `name`, `diameter`.
    \item В `\_\_new\_\_`: если `name` не начинается с "ast\_", выбросьте `ValueError("Астероид должен иметь префикс 'ast\_'")`.
    \item Извлеките ID астероида: `ast\_id = name[4:]`.
    \item Создайте экземпляр: `instance = super().\_\_new\_\_(cls)`.
    \item Добавьте ID в словарь: `cls.belt[ast\_id] = instance`.
    \item Установите атрибут экземпляра: `setattr(instance, f"a{ast\_id}", diameter)`.
    \item Верните `instance`.
    \item Переопределите метод `\_\_init\_\_` как пустой.
    \item Создайте объект `a1` с именем "ast\_ceres" и диаметром 939.
    \item Создайте объект `a2` с именем "ast\_vesta" и диаметром 525.
    \item Попытайтесь создать объект с именем "meteor\_id8" — поймайте исключение.
    \item Выведите `a1.aceres` и `a2.avesta`.
    \item Выведите `Asteroid.belt`.
\end{enumerate}

Пример использования:
\begin{lstlisting}[language=Python]
a1 = Asteroid("ast_ceres", 939)
a2 = Asteroid("ast_vesta", 525)

try:
    a3 = Asteroid("meteor_id8", 10)
except ValueError as e:
    print("Ошибка:", e)

print("a1.aceres:", a1.aceres)  # 939
print("a2.avesta:", a2.avesta)  # 525
print("Asteroid.belt:", Asteroid.belt)
\end{lstlisting}

\item Написать программу на Python, которая создает класс `Comet` с использованием метода `\_\_new\_\_` для контроля именования. Имена должны начинаться с "comet\_". Метод `\_\_init\_\_` должен быть пустым.

Инструкции:
\begin{enumerate}
    \item Создайте класс `Comet`.
    \item Добавьте атрибут класса `orbits` и инициализируйте его пустым словарем.
    \item Переопределите метод `\_\_new\_\_`, принимающий `cls`, `name`, `period`.
    \item В `\_\_new\_\_`: если `name` не начинается с "comet\_", выбросьте `ValueError("Комета должна иметь префикс 'comet\_'")`.
    \item Извлеките ID кометы: `comet\_id = name[6:]`.
    \item Создайте экземпляр: `instance = super().\_\_new\_\_(cls)`.
    \item Добавьте ID в словарь: `cls.orbits[comet\_id] = instance`.
    \item Установите атрибут экземпляра: `setattr(instance, f"cm{comet\_id}", period)`.
    \item Верните `instance`.
    \item Переопределите метод `\_\_init\_\_` как пустой.
    \item Создайте объект `cm1` с именем "comet\_halley" и периодом 76.
    \item Создайте объект `cm2` с именем "comet\_encke" и периодом 3.3.
    \item Попытайтесь создать объект с именем "meteor\_shower" — поймайте исключение.
    \item Выведите `cm1.cmhalley` и `cm2.cmencke`.
    \item Выведите `Comet.orbits`.
\end{enumerate}

Пример использования:
\begin{lstlisting}[language=Python]
cm1 = Comet("comet_halley", 76)
cm2 = Comet("comet_encke", 3.3)

try:
    cm3 = Comet("meteor_shower", 0.1)
except ValueError as e:
    print("Ошибка:", e)

print("cm1.cmhalley:", cm1.cmhalley)  # 76
print("cm2.cmencke:", cm2.cmencke)    # 3.3
print("Comet.orbits:", Comet.orbits)
\end{lstlisting}

\item Написать программу на Python, которая создает класс `Satellite` с использованием метода `\_\_new\_\_` для контроля именования. Имена должны начинаться с "sat\_". Метод `\_\_init\_\_` должен быть пустым.

Инструкции:
\begin{enumerate}
    \item Создайте класс `Satellite`.
    \item Добавьте атрибут класса `orbiters` и инициализируйте его пустым словарем.
    \item Переопределите метод `\_\_new\_\_`, принимающий `cls`, `name`, `planet`.
    \item В `\_\_new\_\_`: если `name` не начинается с "sat\_", выбросьте `ValueError("Спутник должен иметь префикс 'sat\_'")`.
    \item Извлеките ID спутника: `sat\_id = name[4:]`.
    \item Создайте экземпляр: `instance = super().\_\_new\_\_(cls)`.
    \item Добавьте ID в словарь: `cls.orbiters[sat\_id] = instance`.
    \item Установите атрибут экземпляра: `setattr(instance, f"s{sat\_id}", planet)`.
    \item Верните `instance`.
    \item Переопределите метод `\_\_init\_\_` как пустой.
    \item Создайте объект `s1` с именем "sat\_moon" и планетой "Earth".
    \item Создайте объект `s2` с именем "sat\_phobos" и планетой "Mars".
    \item Попытайтесь создать объект с именем "rover\_curiosity" — поймайте исключение.
    \item Выведите `s1.smoon` и `s2.sphobos`.
    \item Выведите `Satellite.orbiters`.
\end{enumerate}

Пример использования:
\begin{lstlisting}[language=Python]
s1 = Satellite("sat_moon", "Earth")
s2 = Satellite("sat_phobos", "Mars")

try:
    s3 = Satellite("rover_curiosity", "Mars")
except ValueError as e:
    print("Ошибка:", e)

print("s1.smoon:", s1.smoon)    # Earth
print("s2.sphobos:", s2.sphobos) # Mars
print("Satellite.orbiters:", Satellite.orbiters)
\end{lstlisting}

\item Написать программу на Python, которая создает класс `Rocket` с использованием метода `\_\_new\_\_` для контроля именования. Имена должны начинаться с "rocket\_". Метод `\_\_init\_\_` должен быть пустым.

Инструкции:
\begin{enumerate}
    \item Создайте класс `Rocket`.
    \item Добавьте атрибут класса `launchpad` и инициализируйте его пустым словарем.
    \item Переопределите метод `\_\_new\_\_`, принимающий `cls`, `name`, `payload`.
    \item В `\_\_new\_\_`: если `name` не начинается с "rocket\_", выбросьте `ValueError("Ракета должна иметь префикс 'rocket\_'")`.
    \item Извлеките ID ракеты: `rocket\_id = name[7:]`.
    \item Создайте экземпляр: `instance = super().\_\_new\_\_(cls)`.
    \item Добавьте ID в словарь: `cls.launchpad[rocket\_id] = instance`.
    \item Установите атрибут экземпляра: `setattr(instance, f"r{rocket\_id}", payload)`.
    \item Верните `instance`.
    \item Переопределите метод `\_\_init\_\_` как пустой.
    \item Создайте объект `r1` с именем "rocket\_falcon9" и полезной нагрузкой "Starlink".
    \item Создайте объект `r2` с именем "rocket\_atlas" и полезной нагрузкой "GPS".
    \item Попытайтесь создать объект с именем "drone\_delivery" — поймайте исключение.
    \item Выведите `r1.rfalcon9` и `r2.ratlas`.
    \item Выведите `Rocket.launchpad`.
\end{enumerate}

Пример использования:
\begin{lstlisting}[language=Python]
r1 = Rocket("rocket_falcon9", "Starlink")
r2 = Rocket("rocket_atlas", "GPS")

try:
    r3 = Rocket("drone_delivery", "Package")
except ValueError as e:
    print("Ошибка:", e)

print("r1.rfalcon9:", r1.rfalcon9)  # Starlink
print("r2.ratlas:", r2.ratlas)      # GPS
print("Rocket.launchpad:", Rocket.launchpad)
\end{lstlisting}

\item Написать программу на Python, которая создает класс `Drone` с использованием метода `\_\_new\_\_` для контроля именования. Имена должны начинаться с "drone\_". Метод `\_\_init\_\_` должен быть пустым.

Инструкции:
\begin{enumerate}
    \item Создайте класс `Drone`.
    \item Добавьте атрибут класса `fleet` и инициализируйте его пустым словарем.
    \item Переопределите метод `\_\_new\_\_`, принимающий `cls`, `name`, `range`.
    \item В `\_\_new\_\_`: если `name` не начинается с "drone\_", выбросьте `ValueError("Дрон должен иметь префикс 'drone\_'")`.
    \item Извлеките ID дрона: `drone\_id = name[6:]`.
    \item Создайте экземпляр: `instance = super().\_\_new\_\_(cls)`.
    \item Добавьте ID в словарь: `cls.fleet[drone\_id] = instance`.
    \item Установите атрибут экземпляра: `setattr(instance, f"d{drone\_id}", range)`.
    \item Верните `instance`.
    \item Переопределите метод `\_\_init\_\_` как пустой.
    \item Создайте объект `d1` с именем "drone\_x1" и дальностью 5.
    \item Создайте объект `d2` с именем "drone\_x2" и дальностью 10.
    \item Попытайтесь создать объект с именем "robot\_r1" — поймайте исключение.
    \item Выведите `d1.dx1` и `d2.dx2`.
    \item Выведите `Drone.fleet`.
\end{enumerate}

Пример использования:
\begin{lstlisting}[language=Python]
d1 = Drone("drone_x1", 5)
d2 = Drone("drone_x2", 10)

try:
    d3 = Drone("robot_r1", 2)
except ValueError as e:
    print("Ошибка:", e)

print("d1.dx1:", d1.dx1)  # 5
print("d2.dx2:", d2.dx2)  # 10
print("Drone.fleet:", Drone.fleet)
\end{lstlisting}

\item Написать программу на Python, которая создает класс `Robot` с использованием метода `\_\_new\_\_` для контроля именования. Имена должны начинаться с "robot\_". Метод `\_\_init\_\_` должен быть пустым.

Инструкции:
\begin{enumerate}
    \item Создайте класс `Robot`.
    \item Добавьте атрибут класса `factory` и инициализируйте его пустым словарем.
    \item Переопределите метод `\_\_new\_\_`, принимающий `cls`, `name`, `function`.
    \item В `\_\_new\_\_`: если `name` не начинается с "robot\_", выбросьте `ValueError("Робот должен иметь префикс 'robot\_'")`.
    \item Извлеките ID робота: `robot\_id = name[6:]`.
    \item Создайте экземпляр: `instance = super().\_\_new\_\_(cls)`.
    \item Добавьте ID в словарь: `cls.factory[robot\_id] = instance`.
    \item Установите атрибут экземпляра: `setattr(instance, f"rb{robot\_id}", function)`.
    \item Верните `instance`.
    \item Переопределите метод `\_\_init\_\_` как пустой.
    \item Создайте объект `rb1` с именем "robot\_arm" и функцией "Assembly".
    \item Создайте объект `rb2` с именем "robot\_cleaner" и функцией "Cleaning".
    \item Попытайтесь создать объект с именем "android\_unit" — поймайте исключение.
    \item Выведите `rb1.rbarm` и `rb2.rbcleaner`.
    \item Выведите `Robot.factory`.
\end{enumerate}

Пример использования:
\begin{lstlisting}[language=Python]
rb1 = Robot("robot_arm", "Assembly")
rb2 = Robot("robot_cleaner", "Cleaning")

try:
    rb3 = Robot("android_unit", "General")
except ValueError as e:
    print("Ошибка:", e)

print("rb1.rbarm:", rb1.rbarm)      # Assembly
print("rb2.rbcleaner:", rb2.rbcleaner) # Cleaning
print("Robot.factory:", Robot.factory)
\end{lstlisting}

\item Написать программу на Python, которая создает класс `AI` с использованием метода `\_\_new\_\_` для контроля именования. Имена должны начинаться с "ai\_". Метод `\_\_init\_\_` должен быть пустым.

Инструкции:
\begin{enumerate}
    \item Создайте класс `AI`.
    \item Добавьте атрибут класса `network` и инициализируйте его пустым словарем.
    \item Переопределите метод `\_\_new\_\_`, принимающий `cls`, `name`, `capability`.
    \item В `\_\_new\_\_`: если `name` не начинается с "ai\_", выбросьте `ValueError("ИИ должен иметь префикс 'ai\_'")`.
    \item Извлеките ID ИИ: `ai\_id = name[3:]`.
    \item Создайте экземпляр: `instance = super().\_\_new\_\_(cls)`.
    \item Добавьте ID в словарь: `cls.network[ai\_id] = instance`.
    \item Установите атрибут экземпляра: `setattr(instance, f"a{ai\_id}", capability)`.
    \item Верните `instance`.
    \item Переопределите метод `\_\_init\_\_` как пустой.
    \item Создайте объект `a1` с именем "ai\_alpha" и возможностью "NLP".
    \item Создайте объект `a2` с именем "ai\_beta" и возможностью "CV".
    \item Попытайтесь создать объект с именем "ml\_model" — поймайте исключение.
    \item Выведите `a1.aalpha` и `a2.abeta`.
    \item Выведите `AI.network`.
\end{enumerate}

Пример использования:
\begin{lstlisting}[language=Python]
a1 = AI("ai_alpha", "NLP")
a2 = AI("ai_beta", "CV")

try:
    a3 = AI("ml_model", "Regression")
except ValueError as e:
    print("Ошибка:", e)

print("a1.aalpha:", a1.aalpha)  # NLP
print("a2.abeta:", a2.abeta)    # CV
print("AI.network:", AI.network)
\end{lstlisting}

\item Написать программу на Python, которая создает класс `MLModel` с использованием метода `\_\_new\_\_` для контроля именования. Имена должны начинаться с "model\_". Метод `\_\_init\_\_` должен быть пустым.

Инструкции:
\begin{enumerate}
    \item Создайте класс `MLModel`.
    \item Добавьте атрибут класса `repository` и инициализируйте его пустым словарем.
    \item Переопределите метод `\_\_new\_\_`, принимающий `cls`, `name`, `algorithm`.
    \item В `\_\_new\_\_`: если `name` не начинается с "model\_", выбросьте `ValueError("Модель должна иметь префикс 'model\_'")`.
    \item Извлеките ID модели: `model\_id = name[6:]`.
    \item Создайте экземпляр: `instance = super().\_\_new\_\_(cls)`.
    \item Добавьте ID в словарь: `cls.repository[model\_id] = instance`.
    \item Установите атрибут экземпляра: `setattr(instance, f"m{model\_id}", algorithm)`.
    \item Верните `instance`.
    \item Переопределите метод `\_\_init\_\_` как пустой.
    \item Создайте объект `m1` с именем "model\_logreg" и алгоритмом "Logistic Regression".
    \item Создайте объект `m2` с именем "model\_svm" и алгоритмом "Support Vector Machine".
    \item Попытайтесь создать объект с именем "algo\_randomforest" — поймайте исключение.
    \item Выведите `m1.mlogreg` и `m2.msvm`.
    \item Выведите `MLModel.repository`.
\end{enumerate}

Пример использования:
\begin{lstlisting}[language=Python]
m1 = MLModel("model_logreg", "Logistic Regression")
m2 = MLModel("model_svm", "Support Vector Machine")

try:
    m3 = MLModel("algo_randomforest", "Random Forest")
except ValueError as e:
    print("Ошибка:", e)

print("m1.mlogreg:", m1.mlogreg)  # Logistic Regression
print("m2.msvm:", m2.msvm)        # Support Vector Machine
print("MLModel.repository:", MLModel.repository)
\end{lstlisting}

\item Написать программу на Python, которая создает класс `Dataset` с использованием метода `\_\_new\_\_` для контроля именования. Имена должны начинаться с "dataset\_". Метод `\_\_init\_\_` должен быть пустым.

Инструкции:
\begin{enumerate}
    \item Создайте класс `Dataset`.
    \item Добавьте атрибут класса `catalog` и инициализируйте его пустым словарем.
    \item Переопределите метод `\_\_new\_\_`, принимающий `cls`, `name`, `size`.
    \item В `\_\_new\_\_`: если `name` не начинается с "dataset\_", выбросьте `ValueError("Набор данных должен иметь префикс 'dataset\_'")`.
    \item Извлеките ID набора данных: `dataset\_id = name[8:]`.
    \item Создайте экземпляр: `instance = super().\_\_new\_\_(cls)`.
    \item Добавьте ID в словарь: `cls.catalog[dataset\_id] = instance`.
    \item Установите атрибут экземпляра: `setattr(instance, f"ds{dataset\_id}", size)`.
    \item Верните `instance`.
    \item Переопределите метод `\_\_init\_\_` как пустой.
    \item Создайте объект `ds1` с именем "dataset\_train" и размером 10000.
    \item Создайте объект `ds2` с именем "dataset\_test" и размером 2000.
    \item Попытайтесь создать объект с именем "data\_validation" — поймайте исключение.
    \item Выведите `ds1.dstrain` и `ds2.dstest`.
    \item Выведите `Dataset.catalog`.
\end{enumerate}

Пример использования:
\begin{lstlisting}[language=Python]
ds1 = Dataset("dataset_train", 10000)
ds2 = Dataset("dataset_test", 2000)

try:
    ds3 = Dataset("data_validation", 2000)
except ValueError as e:
    print("Ошибка:", e)

print("ds1.dstrain:", ds1.dstrain)  # 10000
print("ds2.dstest:", ds2.dstest)    # 2000
print("Dataset.catalog:", Dataset.catalog)
\end{lstlisting}

\item Написать программу на Python, которая создает класс `Feature` с использованием метода `\_\_new\_\_` для контроля именования. Имена должны начинаться с "feat\_". Метод `\_\_init\_\_` должен быть пустым.

Инструкции:
\begin{enumerate}
    \item Создайте класс `Feature`.
    \item Добавьте атрибут класса `registry` и инициализируйте его пустым словарем.
    \item Переопределите метод `\_\_new\_\_`, принимающий `cls`, `name`, `type`.
    \item В `\_\_new\_\_`: если `name` не начинается с "feat\_", выбросьте `ValueError("Признак должен иметь префикс 'feat\_'")`.
    \item Извлеките ID признака: `feat\_id = name[5:]`.
    \item Создайте экземпляр: `instance = super().\_\_new\_\_(cls)`.
    \item Добавьте ID в словарь: `cls.registry[feat\_id] = instance`.
    \item Установите атрибут экземпляра: `setattr(instance, f"f{feat\_id}", type)`.
    \item Верните `instance`.
    \item Переопределите метод `\_\_init\_\_` как пустой.
    \item Создайте объект `f1` с именем "feat\_age" и типом "Numeric".
    \item Создайте объект `f2` с именем "feat\_gender" и типом "Categorical".
    \item Попытайтесь создать объект с именем "attr\_income" — поймайте исключение.
    \item Выведите `f1.fage` и `f2.fgender`.
    \item Выведите `Feature.registry`.
\end{enumerate}

Пример использования:
\begin{lstlisting}[language=Python]
f1 = Feature("feat_age", "Numeric")
f2 = Feature("feat_gender", "Categorical")

try:
    f3 = Feature("attr_income", "Numeric")
except ValueError as e:
    print("Ошибка:", e)

print("f1.fage:", f1.fage)      # Numeric
print("f2.fgender:", f2.fgender) # Categorical
print("Feature.registry:", Feature.registry)
\end{lstlisting}

\item Написать программу на Python, которая создает класс `Label` с использованием метода `\_\_new\_\_` для контроля именования. Имена должны начинаться с "label\_". Метод `\_\_init\_\_` должен быть пустым.

Инструкции:
\begin{enumerate}
    \item Создайте класс `Label`.
    \item Добавьте атрибут класса `index` и инициализируйте его пустым словарем.
    \item Переопределите метод `\_\_new\_\_`, принимающий `cls`, `name`, `class\_name`.
    \item В `\_\_new\_\_`: если `name` не начинается с "label\_", выбросьте `ValueError("Метка должна иметь префикс 'label\_'")`.
    \item Извлеките ID метки: `label\_id = name[6:]`.
    \item Создайте экземпляр: `instance = super().\_\_new\_\_(cls)`.
    \item Добавьте ID в словарь: `cls.index[label\_id] = instance`.
    \item Установите атрибут экземпляра: `setattr(instance, f"l{label\_id}", class\_name)`.
    \item Верните `instance`.
    \item Переопределите метод `\_\_init\_\_` как пустой.
    \item Создайте объект `l1` с именем "label\_cat" и классом "Animal".
    \item Создайте объект `l2` с именем "label\_car" и классом "Vehicle".
    \item Попытайтесь создать объект с именем "tag\_dog" — поймайте исключение.
    \item Выведите `l1.lcat` и `l2.lcar`.
    \item Выведите `Label.index`.
\end{enumerate}

Пример использования:
\begin{lstlisting}[language=Python]
l1 = Label("label_cat", "Animal")
l2 = Label("label_car", "Vehicle")

try:
    l3 = Label("tag_dog", "Animal")
except ValueError as e:
    print("Ошибка:", e)

print("l1.lcat:", l1.lcat)  # Animal
print("l2.lcar:", l2.lcar)  # Vehicle
print("Label.index:", Label.index)
\end{lstlisting}

\item Написать программу на Python, которая создает класс `Layer` с использованием метода `\_\_new\_\_` для контроля именования. Имена должны начинаться с "layer\_". Метод `\_\_init\_\_` должен быть пустым.

Инструкции:
\begin{enumerate}
    \item Создайте класс `Layer`.
    \item Добавьте атрибут класса `stack` и инициализируйте его пустым словарем.
    \item Переопределите метод `\_\_new\_\_`, принимающий `cls`, `name`, `neurons`.
    \item В `\_\_new\_\_`: если `name` не начинается с "layer\_", выбросьте `ValueError("Слой должен иметь префикс 'layer\_'")`.
    \item Извлеките ID слоя: `layer\_id = name[6:]`.
    \item Создайте экземпляр: `instance = super().\_\_new\_\_(cls)`.
    \item Добавьте ID в словарь: `cls.stack[layer\_id] = instance`.
    \item Установите атрибут экземпляра: `setattr(instance, f"ly{layer\_id}", neurons)`.
    \item Верните `instance`.
    \item Переопределите метод `\_\_init\_\_` как пустой.
    \item Создайте объект `ly1` с именем "layer\_input" и нейронами 784.
    \item Создайте объект `ly2` с именем "layer\_hidden" и нейронами 128.
    \item Попытайтесь создать объект с именем "unit\_output" — поймайте исключение.
    \item Выведите `ly1.lyinput` и `ly2.lyhidden`.
    \item Выведите `Layer.stack`.
\end{enumerate}

Пример использования:
\begin{lstlisting}[language=Python]
ly1 = Layer("layer_input", 784)
ly2 = Layer("layer_hidden", 128)

try:
    ly3 = Layer("unit_output", 10)
except ValueError as e:
    print("Ошибка:", e)

print("ly1.lyinput:", ly1.lyinput)    # 784
print("ly2.lyhidden:", ly2.lyhidden)  # 128
print("Layer.stack:", Layer.stack)
\end{lstlisting}

\item Написать программу на Python, которая создает класс `Neuron` с использованием метода `\_\_new\_\_` для контроля именования. Имена должны начинаться с "neuron\_". Метод `\_\_init\_\_` должен быть пустым.

Инструкции:
\begin{enumerate}
    \item Создайте класс `Neuron`.
    \item Добавьте атрибут класса `brain` и инициализируйте его пустым словарем.
    \item Переопределите метод `\_\_new\_\_`, принимающий `cls`, `name`, `activation`.
    \item В `\_\_new\_\_`: если `name` не начинается с "neuron\_", выбросьте `ValueError("Нейрон должен иметь префикс 'neuron\_'")`.
    \item Извлеките ID нейрона: `neuron\_id = name[7:]`.
    \item Создайте экземпляр: `instance = super().\_\_new\_\_(cls)`.
    \item Добавьте ID в словарь: `cls.brain[neuron\_id] = instance`.
    \item Установите атрибут экземпляра: `setattr(instance, f"n{neuron\_id}", activation)`.
    \item Верните `instance`.
    \item Переопределите метод `\_\_init\_\_` как пустой.
    \item Создайте объект `n1` с именем "neuron\_1" и активацией "ReLU".
    \item Создайте объект `n2` с именем "neuron\_2" и активацией "Sigmoid".
    \item Попытайтесь создать объект с именем "cell\_3" — поймайте исключение.
    \item Выведите `n1.n1` и `n2.n2`.
    \item Выведите `Neuron.brain`.
\end{enumerate}

Пример использования:
\begin{lstlisting}[language=Python]
n1 = Neuron("neuron_1", "ReLU")
n2 = Neuron("neuron_2", "Sigmoid")

try:
    n3 = Neuron("cell_3", "Tanh")
except ValueError as e:
    print("Ошибка:", e)

print("n1.n1:", n1.n1)  # ReLU
print("n2.n2:", n2.n2)  # Sigmoid
print("Neuron.brain:", Neuron.brain)
\end{lstlisting}

\item Написать программу на Python, которая создает класс `Synapse` с использованием метода `\_\_new\_\_` для контроля именования. Имена должны начинаться с "syn\_". Метод `\_\_init\_\_` должен быть пустым.

Инструкции:
\begin{enumerate}
    \item Создайте класс `Synapse`.
    \item Добавьте атрибут класса `connections` и инициализируйте его пустым словарем.
    \item Переопределите метод `\_\_new\_\_`, принимающий `cls`, `name`, `weight`.
    \item В `\_\_new\_\_`: если `name` не начинается с "syn\_", выбросьте `ValueError("Синапс должен иметь префикс 'syn\_'")`.
    \item Извлеките ID синапса: `syn\_id = name[4:]`.
    \item Создайте экземпляр: `instance = super().\_\_new\_\_(cls)`.
    \item Добавьте ID в словарь: `cls.connections[syn\_id] = instance`.
    \item Установите атрибут экземпляра: `setattr(instance, f"s{syn\_id}", weight)`.
    \item Верните `instance`.
    \item Переопределите метод `\_\_init\_\_` как пустой.
    \item Создайте объект `s1` с именем "syn\_a1b1" и весом 0.5.
    \item Создайте объект `s2` с именем "syn\_a2b2" и весом -0.3.
    \item Попытайтесь создать объект с именем "link\_x1y1" — поймайте исключение.
    \item Выведите `s1.sa1b1` и `s2.sa2b2`.
    \item Выведите `Synapse.connections`.
\end{enumerate}

Пример использования:
\begin{lstlisting}[language=Python]
s1 = Synapse("syn_a1b1", 0.5)
s2 = Synapse("syn_a2b2", -0.3)

try:
    s3 = Synapse("link_x1y1", 0.8)
except ValueError as e:
    print("Ошибка:", e)

print("s1.sa1b1:", s1.sa1b1)  # 0.5
print("s2.sa2b2:", s2.sa2b2)  # -0.3
print("Synapse.connections:", Synapse.connections)
\end{lstlisting}

\item Написать программу на Python, которая создает класс `Container` с использованием метода `\_\_new\_\_` для контроля именования. Имена должны начинаться с "container\_". Метод `\_\_init\_\_` должен быть пустым.
Инструкции:
\begin{enumerate}
    \item Создайте класс `Container`.
    \item Добавьте атрибут класса `depot` и инициализируйте его пустым словарем.
    \item Переопределите метод `\_\_new\_\_`, принимающий `cls`, `name`, `volume`.
    \item В `\_\_new\_\_`: если `name` не начинается с "container\_", выбросьте `ValueError("Контейнер должен иметь префикс 'container\_'")`.
    \item Извлеките ID контейнера: `container\_id = name[9:]`.
    \item Создайте экземпляр: `instance = super().\_\_new\_\_(cls)`.
    \item Добавьте ID в словарь: `cls.depot[container\_id] = instance`.
    \item Установите атрибут экземпляра: `setattr(instance, f"cn{container\_id}", volume)`.
    \item Верните `instance`.
    \item Переопределите метод `\_\_init\_\_` как пустой.
    \item Создайте объект `cn1` с именем "container\_20ft" и объемом 33.
    \item Создайте объект `cn2` с именем "container\_40ft" и объемом 67.
    \item Попытайтесь создать объект с именем "box\_small" — поймайте исключение.
    \item Выведите `cn1.cn20ft` и `cn2.cn40ft`.
    \item Выведите `Container.depot`.
\end{enumerate}
Пример использования:
\begin{lstlisting}[language=Python]
cn1 = Container("container_20ft", 33)
cn2 = Container("container_40ft", 67)
try:
    cn3 = Container("box_small", 1)
except ValueError as e:
    print("Ошибка:", e)
print("cn1.cn20ft:", cn1.cn20ft)  # 33
print("cn2.cn40ft:", cn2.cn40ft)  # 67
print("Container.depot:", Container.depot)
\end{lstlisting}


\item Написать программу на Python, которая создает класс `Module` с использованием метода `\_\_new\_\_` для контроля именования. Имена должны начинаться с "mod\_". Метод `\_\_init\_\_` должен быть пустым.
Инструкции:
\begin{enumerate}
    \item Создайте класс `Module`.
    \item Добавьте атрибут класса `system` и инициализируйте его пустым словарем.
    \item Переопределите метод `\_\_new\_\_`, принимающий `cls`, `name`, `version`.
    \item В `\_\_new\_\_`: если `name` не начинается с "mod\_", выбросьте `ValueError("Модуль должен иметь префикс 'mod\_'")`.
    \item Извлеките ID модуля: `mod\_id = name[4:]`.
    \item Создайте экземпляр: `instance = super().\_\_new\_\_(cls)`.
    \item Добавьте ID в словарь: `cls.system[mod\_id] = instance`.
    \item Установите атрибут экземпляра: `setattr(instance, f"md{mod\_id}", version)`.
    \item Верните `instance`.
    \item Переопределите метод `\_\_init\_\_` как пустой.
    \item Создайте объект `md1` с именем "mod\_auth" и версией "1.2.0".
    \item Создайте объект `md2` с именем "mod\_payment" и версией "3.0.1".
    \item Попытайтесь создать объект с именем "lib\_utils" — поймайте исключение.
    \item Выведите `md1.mdauth` и `md2.mdpayment`.
    \item Выведите `Module.system`.
\end{enumerate}
Пример использования:
\begin{lstlisting}[language=Python]
md1 = Module("mod_auth", "1.2.0")
md2 = Module("mod_payment", "3.0.1")
try:
    md3 = Module("lib_utils", "0.9.5")
except ValueError as e:
    print("Ошибка:", e)
print("md1.mdauth:", md1.mdauth)    # 1.2.0
print("md2.mdpayment:", md2.mdpayment) # 3.0.1
print("Module.system:", Module.system)
\end{lstlisting}
\end{enumerate}

\subsubsection{Задача 4}

\begin{enumerate}
\item[1] Написать программу на Python, которая создает класс \texttt{ShoppingCart} для представления корзины покупок. Класс должен содержать методы для добавления и удаления товаров, а также вычисления общего количества. Программа также должна создавать экземпляр класса \texttt{ShoppingCart}, добавлять товары в корзину, удалять товары из корзины и выводить информацию о корзине на экран.

\begin{itemize}
    \item Создайте класс \texttt{ShoppingCart} с методом \texttt{\_\_init\_\_}, который создает пустой список товаров.
    \item Создайте метод \texttt{add\_item}, который принимает название товара и количество в качестве аргументов и добавляет их в список товаров.
    \item Создайте метод \texttt{remove\_item}, который удаляет товар из списка товаров по его названию.
    \item Создайте метод \texttt{calculate\_total}, который вычисляет и возвращает общее количество всех товаров в корзине.
    \item Создайте экземпляр класса \texttt{ShoppingCart} и добавьте товары в корзину.
    \item Выведите информацию о текущих товарах в корзине на экран.
    \item Выведите общее количество всех товаров в корзине на экран.
    \item Удалите товар из корзины и выведите обновленную информацию о товарах в корзине на экран.
    \item Выведите общее количество всех товаров в корзине после удаления товара на экран.
\end{itemize}

\textbf{Пример использования:}

\begin{verbatim}
cart = ShoppingCart()
cart.add_item("Картофель", 100)
cart.add_item("Капуста", 200)
cart.add_item("Апельсин", 150)
print("Число товаров в корзине:")
for item in cart.items:
    print(item[0], "-", item[1])
total_qty = cart.calculate_total()
print("Общее количество:", total_qty)
cart.remove_item("Апельсин")
print("Обновление числа покупок в корзине после удаления апельсина:")
for item in cart.items:
    print(item[0], "-", item[1])
total_qty = cart.calculate_total()
print("Общее количество:", total_qty)
\end{verbatim}

\textbf{Вывод:}
\begin{verbatim}
Число товаров в корзине:
Картофель - 100
Капуста - 200
Апельсин - 150
Общее количество: 450
Обновление числа покупок в корзине после удаления апельсина:
Картофель - 100
Капуста - 200
Общее количество: 300
\end{verbatim}

\item[2] Написать программу на Python, которая создает класс \texttt{BookCollection} для представления коллекции книг. Класс должен содержать методы для добавления и удаления книг, а также подсчета общего количества страниц. Программа также должна создавать экземпляр класса \texttt{BookCollection}, добавлять книги в коллекцию, удалять книги из коллекции и выводить информацию о коллекции на экран.

\begin{itemize}
    \item Создайте класс \texttt{BookCollection} с методом \texttt{\_\_init\_\_}, который создает пустой список книг.
    \item Создайте метод \texttt{add\_book}, который принимает название книги и количество страниц в качестве аргументов и добавляет их в список книг.
    \item Создайте метод \texttt{remove\_book}, который удаляет книгу из списка по её названию.
    \item Создайте метод \texttt{total\_pages}, который вычисляет и возвращает общее количество страниц всех книг в коллекции.
    \item Создайте экземпляр класса \texttt{BookCollection} и добавьте книги в коллекцию.
    \item Выведите информацию о текущих книгах в коллекции на экран.
    \item Выведите общее количество страниц всех книг на экран.
    \item Удалите книгу из коллекции и выведите обновленную информацию о книгах на экран.
    \item Выведите общее количество страниц после удаления книги на экран.
\end{itemize}

\textbf{Пример использования:}

\begin{verbatim}
collection = BookCollection()
collection.add_book("Война и мир", 1225)
collection.add_book("Преступление и наказание", 671)
collection.add_book("Мастер и Маргарита", 480)
print("Книги в коллекции:")
for book in collection.books:
    print(book[0], "-", book[1], "стр.")
total = collection.total_pages()
print("Общее количество страниц:", total)
collection.remove_book("Преступление и наказание")
print("Книги после удаления 'Преступления и наказания':")
for book in collection.books:
    print(book[0], "-", book[1], "стр.")
total = collection.total_pages()
print("Общее количество страниц:", total)
\end{verbatim}

\textbf{Вывод:}
\begin{verbatim}
Книги в коллекции:
Война и мир - 1225 стр.
Преступление и наказание - 671 стр.
Мастер и Маргарита - 480 стр.
Общее количество страниц: 2376
Книги после удаления 'Преступления и наказания':
Война и мир - 1225 стр.
Мастер и Маргарита - 480 стр.
Общее количество страниц: 1705
\end{verbatim}

\item[3] Написать программу на Python, которая создает класс \texttt{Inventory} для представления складского запаса. Класс должен содержать методы для добавления и удаления предметов, а также вычисления общего количества единиц товара. Программа также должна создавать экземпляр класса \texttt{Inventory}, добавлять предметы на склад, удалять предметы со склада и выводить информацию о запасах на экран.

\begin{itemize}
    \item Создайте класс \texttt{Inventory} с методом \texttt{\_\_init\_\_}, который создает пустой список предметов.
    \item Создайте метод \texttt{add\_item}, который принимает название предмета и количество единиц в качестве аргументов и добавляет их в список.
    \item Создайте метод \texttt{remove\_item}, который удаляет предмет из списка по его названию.
    \item Создайте метод \texttt{total\_count}, который вычисляет и возвращает общее количество всех единиц товара на складе.
    \item Создайте экземпляр класса \texttt{Inventory} и добавьте предметы на склад.
    \item Выведите информацию о текущих предметах на складе на экран.
    \item Выведите общее количество единиц товара на экран.
    \item Удалите предмет со склада и выведите обновленную информацию о предметах на экран.
    \item Выведите общее количество единиц товара после удаления предмета на экран.
\end{itemize}

\textbf{Пример использования:}

\begin{verbatim}
inv = Inventory()
inv.add_item("Молотки", 50)
inv.add_item("Отвертки", 120)
inv.add_item("Гвозди", 1000)
print("Предметы на складе:")
for item in inv.items:
    print(item[0], "-", item[1])
total = inv.total_count()
print("Общее количество:", total)
inv.remove_item("Отвертки")
print("Предметы после удаления отверток:")
for item in inv.items:
    print(item[0], "-", item[1])
total = inv.total_count()
print("Общее количество:", total)
\end{verbatim}

\textbf{Вывод:}
\begin{verbatim}
Предметы на складе:
Молотки - 50
Отвертки - 120
Гвозди - 1000
Общее количество: 1170
Предметы после удаления отверток:
Молотки - 50
Гвозди - 1000
Общее количество: 1050
\end{verbatim}

\item[4] Написать программу на Python, которая создает класс \texttt{Playlist} для представления музыкального плейлиста. Класс должен содержать методы для добавления и удаления треков, а также подсчета общего времени воспроизведения. Программа также должна создавать экземпляр класса \texttt{Playlist}, добавлять треки в плейлист, удалять треки из плейлиста и выводить информацию о плейлисте на экран.

\begin{itemize}
    \item Создайте класс \texttt{Playlist} с методом \texttt{\_\_init\_\_}, который создает пустой список треков.
    \item Создайте метод \texttt{add\_track}, который принимает название трека и его длительность (в секундах) в качестве аргументов и добавляет их в список.
    \item Создайте метод \texttt{remove\_track}, который удаляет трек из списка по его названию.
    \item Создайте метод \texttt{total\_duration}, который вычисляет и возвращает общую длительность всех треков в плейлисте (в секундах).
    \item Создайте экземпляр класса \texttt{Playlist} и добавьте треки в плейлист.
    \item Выведите информацию о текущих треках в плейлисте на экран.
    \item Выведите общую длительность всех треков на экран.
    \item Удалите трек из плейлиста и выведите обновленную информацию о треках на экран.
    \item Выведите общую длительность после удаления трека на экран.
\end{itemize}

\textbf{Пример использования:}

\begin{verbatim}
pl = Playlist()
pl.add_track("Bohemian Rhapsody", 354)
pl.add_track("Imagine", 183)
pl.add_track("Smells Like Teen Spirit", 301)
print("Треки в плейлисте:")
for track in pl.tracks:
    print(track[0], "-", track[1], "сек.")
total = pl.total_duration()
print("Общая длительность:", total, "сек.")
pl.remove_track("Imagine")
print("Треки после удаления 'Imagine':")
for track in pl.tracks:
    print(track[0], "-", track[1], "сек.")
total = pl.total_duration()
print("Общая длительность:", total, "сек.")
\end{verbatim}

\textbf{Вывод:}
\begin{verbatim}
Треки в плейлисте:
Bohemian Rhapsody - 354 сек.
Imagine - 183 сек.
Smells Like Teen Spirit - 301 сек.
Общая длительность: 838 сек.
Треки после удаления 'Imagine':
Bohemian Rhapsody - 354 сек.
Smells Like Teen Spirit - 301 сек.
Общая длительность: 655 сек.
\end{verbatim}

\item[5] Написать программу на Python, которая создает класс \texttt{StudentGrades} для представления оценок студента. Класс должен содержать методы для добавления и удаления оценок, а также вычисления среднего балла. Программа также должна создавать экземпляр класса \texttt{StudentGrades}, добавлять оценки, удалять оценки и выводить информацию об успеваемости на экран.

\begin{itemize}
    \item Создайте класс \texttt{StudentGrades} с методом \texttt{\_\_init\_\_}, который создает пустой список оценок.
    \item Создайте метод \texttt{add\_grade}, который принимает название предмета и оценку в качестве аргументов и добавляет их в список.
    \item Создайте метод \texttt{remove\_grade}, который удаляет оценку по названию предмета.
    \item Создайте метод \texttt{average\_grade}, который вычисляет и возвращает средний балл по всем предметам.
    \item Создайте экземпляр класса \texttt{StudentGrades} и добавьте оценки по разным предметам.
    \item Выведите информацию о текущих оценках на экран.
    \item Выведите средний балл на экран.
    \item Удалите оценку по одному из предметов и выведите обновленную информацию.
    \item Выведите средний балл после удаления оценки на экран.
\end{itemize}

\textbf{Пример использования:}

\begin{verbatim}
grades = StudentGrades()
grades.add_grade("Математика", 5)
grades.add_grade("Физика", 4)
grades.add_grade("Информатика", 5)
print("Оценки студента:")
for subject, grade in grades.grades:
    print(subject, "-", grade)
avg = grades.average_grade()
print("Средний балл:", round(avg, 2))
grades.remove_grade("Физика")
print("Оценки после удаления Физики:")
for subject, grade in grades.grades:
    print(subject, "-", grade)
avg = grades.average_grade()
print("Средний балл:", round(avg, 2))
\end{verbatim}

\textbf{Вывод:}
\begin{verbatim}
Оценки студента:
Математика - 5
Физика - 4
Информатика - 5
Средний балл: 4.67
Оценки после удаления Физики:
Математика - 5
Информатика - 5
Средний балл: 5.0
\end{verbatim}

\item[6] Написать программу на Python, которая создает класс \texttt{TaskList} для представления списка задач. Класс должен содержать методы для добавления и удаления задач, а также подсчета общего количества задач. Программа также должна создавать экземпляр класса \texttt{TaskList}, добавлять задачи, удалять задачи и выводить информацию о списке задач на экран.

\begin{itemize}
    \item Создайте класс \texttt{TaskList} с методом \texttt{\_\_init\_\_}, который создает пустой список задач.
    \item Создайте метод \texttt{add\_task}, который принимает описание задачи и приоритет (целое число) в качестве аргументов и добавляет их в список.
    \item Создайте метод \texttt{remove\_task}, который удаляет задачу из списка по её описанию.
    \item Создайте метод \texttt{task\_count}, который возвращает общее количество задач в списке.
    \item Создайте экземпляр класса \texttt{TaskList} и добавьте несколько задач.
    \item Выведите информацию о текущих задачах на экран.
    \item Выведите общее количество задач на экран.
    \item Удалите одну из задач и выведите обновленный список задач.
    \item Выведите общее количество задач после удаления на экран.
\end{itemize}

\textbf{Пример использования:}

\begin{verbatim}
tasks = TaskList()
tasks.add_task("Написать отчет", 1)
tasks.add_task("Проверить почту", 3)
tasks.add_task("Подготовить презентацию", 2)
print("Список задач:")
for desc, priority in tasks.tasks:
    print(desc, "(приоритет", priority, ")")
count = tasks.task_count()
print("Всего задач:", count)
tasks.remove_task("Проверить почту")
print("Список задач после удаления 'Проверить почту':")
for desc, priority in tasks.tasks:
    print(desc, "(приоритет", priority, ")")
count = tasks.task_count()
print("Всего задач:", count)
\end{verbatim}

\textbf{Вывод:}
\begin{verbatim}
Список задач:
Написать отчет (приоритет 1 )
Проверить почту (приоритет 3 )
Подготовить презентацию (приоритет 2 )
Всего задач: 3
Список задач после удаления 'Проверить почту':
Написать отчет (приоритет 1 )
Подготовить презентацию (приоритет 2 )
Всего задач: 2
\end{verbatim}

\item[7] Написать программу на Python, которая создает класс \texttt{BankAccount} для представления банковского счета. Класс должен содержать методы для добавления и снятия средств, а также получения текущего баланса. Программа также должна создавать экземпляр класса \texttt{BankAccount}, выполнять операции пополнения и снятия, и выводить информацию о балансе на экран.

\begin{itemize}
    \item Создайте класс \texttt{BankAccount} с методом \texttt{\_\_init\_\_}, который инициализирует баланс нулём.
    \item Создайте метод \texttt{deposit}, который принимает сумму и увеличивает баланс на неё.
    \item Создайте метод \texttt{withdraw}, который принимает сумму и уменьшает баланс на неё (если достаточно средств).
    \item Создайте метод \texttt{get\_balance}, который возвращает текущий баланс.
    \item Создайте экземпляр класса \texttt{BankAccount}.
    \item Выполните несколько операций пополнения счета.
    \item Выведите текущий баланс на экран.
    \item Выполните операцию снятия средств и выведите обновленный баланс.
    \item Выведите окончательный баланс на экран.
\end{itemize}

\textbf{Пример использования:}

\begin{verbatim}
account = BankAccount()
account.deposit(1000)
account.deposit(500)
print("Баланс после пополнений:", account.get_balance())
account.withdraw(300)
print("Баланс после снятия 300:", account.get_balance())
account.withdraw(200)
print("Окончательный баланс:", account.get_balance())
\end{verbatim}

\textbf{Вывод:}
\begin{verbatim}
Баланс после пополнений: 1500
Баланс после снятия 300: 1200
Окончательный баланс: 1000
\end{verbatim}

\item[8] Написать программу на Python, которая создает класс \texttt{Library} для представления библиотеки. Класс должен содержать методы для добавления и удаления книг, а также подсчета общего количества книг. Программа также должна создавать экземпляр класса \texttt{Library}, добавлять книги, удалять книги и выводить информацию о фонде на экран.

\begin{itemize}
    \item Создайте класс \texttt{Library} с методом \texttt{\_\_init\_\_}, который создает пустой список книг.
    \item Создайте метод \texttt{add\_book}, который принимает название книги и автора в качестве аргументов и добавляет их в список.
    \item Создайте метод \texttt{remove\_book}, который удаляет книгу из списка по её названию.
    \item Создайте метод \texttt{book\_count}, который возвращает общее количество книг в библиотеке.
    \item Создайте экземпляр класса \texttt{Library} и добавьте несколько книг.
    \item Выведите информацию о текущих книгах на экран.
    \item Выведите общее количество книг на экран.
    \item Удалите одну из книг и выведите обновленный список.
    \item Выведите общее количество книг после удаления на экран.
\end{itemize}

\textbf{Пример использования:}

\begin{verbatim}
lib = Library()
lib.add_book("1984", "Джордж Оруэлл")
lib.add_book("Гарри Поттер", "Дж.К. Роулинг")
lib.add_book("Гордость и предубеждение", "Джейн Остин")
print("Книги в библиотеке:")
for title, author in lib.books:
    print(title, "—", author)
count = lib.book_count()
print("Всего книг:", count)
lib.remove_book("Гарри Поттер")
print("Книги после удаления 'Гарри Поттера':")
for title, author in lib.books:
    print(title, "—", author)
count = lib.book_count()
print("Всего книг:", count)
\end{verbatim}

\textbf{Вывод:}
\begin{verbatim}
Книги в библиотеке:
1984 — Джордж Оруэлл
Гарри Поттер — Дж.К. Роулинг
Гордость и предубеждение — Джейн Остин
Всего книг: 3
Книги после удаления 'Гарри Поттера':
1984 — Джордж Оруэлл
Гордость и предубеждение — Джейн Остин
Всего книг: 2
\end{verbatim}

\item[9] Написать программу на Python, которая создает класс \texttt{GroceryList} для представления списка покупок. Класс должен содержать методы для добавления и удаления продуктов, а также подсчета общего количества позиций. Программа также должна создавать экземпляр класса \texttt{GroceryList}, добавлять продукты, удалять продукты и выводить информацию о списке на экран.

\begin{itemize}
    \item Создайте класс \texttt{GroceryList} с методом \texttt{\_\_init\_\_}, который создает пустой список продуктов.
    \item Создайте метод \texttt{add\_product}, который принимает название продукта и количество в качестве аргументов и добавляет их в список.
    \item Создайте метод \texttt{remove\_product}, который удаляет продукт из списка по его названию.
    \item Создайте метод \texttt{total\_items}, который возвращает общее количество различных продуктов в списке.
    \item Создайте экземпляр класса \texttt{GroceryList} и добавьте несколько продуктов.
    \item Выведите информацию о текущих продуктах на экран.
    \item Выведите общее количество позиций на экран.
    \item Удалите один из продуктов и выведите обновленный список.
    \item Выведите общее количество позиций после удаления на экран.
\end{itemize}

\textbf{Пример использования:}

\begin{verbatim}
grocery = GroceryList()
grocery.add_product("Молоко", 2)
grocery.add_product("Хлеб", 1)
grocery.add_product("Яйца", 12)
print("Список покупок:")
for name, qty in grocery.products:
    print(name, "-", qty)
count = grocery.total_items()
print("Всего позиций:", count)
grocery.remove_product("Хлеб")
print("Список после удаления хлеба:")
for name, qty in grocery.products:
    print(name, "-", qty)
count = grocery.total_items()
print("Всего позиций:", count)
\end{verbatim}

\textbf{Вывод:}
\begin{verbatim}
Список покупок:
Молоко - 2
Хлеб - 1
Яйца - 12
Всего позиций: 3
Список после удаления хлеба:
Молоко - 2
Яйца - 12
Всего позиций: 2
\end{verbatim}

\item[10] Написать программу на Python, которая создает класс \texttt{ContactList} для представления списка контактов. Класс должен содержать методы для добавления и удаления контактов, а также подсчета общего количества контактов. Программа также должна создавать экземпляр класса \texttt{ContactList}, добавлять контакты, удалять контакты и выводить информацию о списке на экран.

\begin{itemize}
    \item Создайте класс \texttt{ContactList} с методом \texttt{\_\_init\_\_}, который создает пустой список контактов.
    \item Создайте метод \texttt{add\_contact}, который принимает имя и номер телефона в качестве аргументов и добавляет их в список.
    \item Создайте метод \texttt{remove\_contact}, который удаляет контакт из списка по имени.
    \item Создайте метод \texttt{contact\_count}, который возвращает общее количество контактов в списке.
    \item Создайте экземпляр класса \texttt{ContactList} и добавьте несколько контактов.
    \item Выведите информацию о текущих контактах на экран.
    \item Выведите общее количество контактов на экран.
    \item Удалите один из контактов и выведите обновленный список.
    \item Выведите общее количество контактов после удаления на экран.
\end{itemize}

\textbf{Пример использования:}

\begin{verbatim}
contacts = ContactList()
contacts.add_contact("Анна", "+79001234567")
contacts.add_contact("Борис", "+79007654321")
contacts.add_contact("Виктория", "+79001112233")
print("Контакты:")
for name, phone in contacts.contacts:
    print(name, "-", phone)
count = contacts.contact_count()
print("Всего контактов:", count)
contacts.remove_contact("Борис")
print("Контакты после удаления Бориса:")
for name, phone in contacts.contacts:
    print(name, "-", phone)
count = contacts.contact_count()
print("Всего контактов:", count)
\end{verbatim}

\textbf{Вывод:}
\begin{verbatim}
Контакты:
Анна - +79001234567
Борис - +79007654321
Виктория - +79001112233
Всего контактов: 3
Контакты после удаления Бориса:
Анна - +79001234567
Виктория - +79001112233
Всего контактов: 2
\end{verbatim}

\item[11] Написать программу на Python, которая создает класс \texttt{MovieCollection} для представления коллекции фильмов. Класс должен содержать методы для добавления и удаления фильмов, а также подсчета общего количества фильмов. Программа также должна создавать экземпляр класса \texttt{MovieCollection}, добавлять фильмы, удалять фильмы и выводить информацию о коллекции на экран.

\begin{itemize}
    \item Создайте класс \texttt{MovieCollection} с методом \texttt{\_\_init\_\_}, который создает пустой список фильмов.
    \item Создайте метод \texttt{add\_movie}, который принимает название фильма и год выпуска в качестве аргументов и добавляет их в список.
    \item Создайте метод \texttt{remove\_movie}, который удаляет фильм из списка по его названию.
    \item Создайте метод \texttt{movie\_count}, который возвращает общее количество фильмов в коллекции.
    \item Создайте экземпляр класса \texttt{MovieCollection} и добавьте несколько фильмов.
    \item Выведите информацию о текущих фильмах на экран.
    \item Выведите общее количество фильмов на экран.
    \item Удалите один из фильмов и выведите обновленный список.
    \item Выведите общее количество фильмов после удаления на экран.
\end{itemize}

\textbf{Пример использования:}

\begin{verbatim}
movies = MovieCollection()
movies.add_movie("Крёстный отец", 1972)
movies.add_movie("Побег из Шоушенка", 1994)
movies.add_movie("Тёмный рыцарь", 2008)
print("Фильмы в коллекции:")
for title, year in movies.movies:
    print(title, "(", year, ")")
count = movies.movie_count()
print("Всего фильмов:", count)
movies.remove_movie("Побег из Шоушенка")
print("Фильмы после удаления 'Побега из Шоушенка':")
for title, year in movies.movies:
    print(title, "(", year, ")")
count = movies.movie_count()
print("Всего фильмов:", count)
\end{verbatim}

\textbf{Вывод:}
\begin{verbatim}
Фильмы в коллекции:
Крёстный отец ( 1972 )
Побег из Шоушенка ( 1994 )
Тёмный рыцарь ( 2008 )
Всего фильмов: 3
Фильмы после удаления 'Побега из Шоушенка':
Крёстный отец ( 1972 )
Тёмный рыцарь ( 2008 )
Всего фильмов: 2
\end{verbatim}

\item[12] Написать программу на Python, которая создает класс \texttt{RecipeBook} для представления кулинарной книги. Класс должен содержать методы для добавления и удаления рецептов, а также подсчета общего количества рецептов. Программа также должна создавать экземпляр класса \texttt{RecipeBook}, добавлять рецепты, удалять рецепты и выводить информацию о книге на экран.

\begin{itemize}
    \item Создайте класс \texttt{RecipeBook} с методом \texttt{\_\_init\_\_}, который создает пустой список рецептов.
    \item Создайте метод \texttt{add\_recipe}, который принимает название блюда и время приготовления (в минутах) в качестве аргументов и добавляет их в список.
    \item Создайте метод \texttt{remove\_recipe}, который удаляет рецепт из списка по названию блюда.
    \item Создайте метод \texttt{recipe\_count}, который возвращает общее количество рецептов в книге.
    \item Создайте экземпляр класса \texttt{RecipeBook} и добавьте несколько рецептов.
    \item Выведите информацию о текущих рецептах на экран.
    \item Выведите общее количество рецептов на экран.
    \item Удалите один из рецептов и выведите обновленный список.
    \item Выведите общее количество рецептов после удаления на экран.
\end{itemize}

\textbf{Пример использования:}

\begin{verbatim}
recipes = RecipeBook()
recipes.add_recipe("Борщ", 60)
recipes.add_recipe("Омлет", 10)
recipes.add_recipe("Паста", 20)
print("Рецепты в книге:")
for dish, time in recipes.recipes:
    print(dish, "-", time, "мин.")
count = recipes.recipe_count()
print("Всего рецептов:", count)
recipes.remove_recipe("Омлет")
print("Рецепты после удаления омлета:")
for dish, time in recipes.recipes:
    print(dish, "-", time, "мин.")
count = recipes.recipe_count()
print("Всего рецептов:", count)
\end{verbatim}

\textbf{Вывод:}
\begin{verbatim}
Рецепты в книге:
Борщ - 60 мин.
Омлет - 10 мин.
Паста - 20 мин.
Всего рецептов: 3
Рецепты после удаления омлета:
Борщ - 60 мин.
Паста - 20 мин.
Всего рецептов: 2
\end{verbatim}

\item[13] Написать программу на Python, которая создает класс \texttt{CarGarage} для представления автосервиса. Класс должен содержать методы для добавления и удаления автомобилей, а также подсчета общего количества машин. Программа также должна создавать экземпляр класса \texttt{CarGarage}, добавлять автомобили, удалять автомобили и выводить информацию о гараже на экран.

\begin{itemize}
    \item Создайте класс \texttt{CarGarage} с методом \texttt{\_\_init\_\_}, который создает пустой список автомобилей.
    \item Создайте метод \texttt{add\_car}, который принимает марку и модель автомобиля в качестве аргументов и добавляет их в список.
    \item Создайте метод \texttt{remove\_car}, который удаляет автомобиль из списка по марке.
    \item Создайте метод \texttt{car\_count}, который возвращает общее количество автомобилей в гараже.
    \item Создайте экземпляр класса \texttt{CarGarage} и добавьте несколько автомобилей.
    \item Выведите информацию о текущих автомобилях на экран.
    \item Выведите общее количество машин на экран.
    \item Удалите один из автомобилей и выведите обновленный список.
    \item Выведите общее количество машин после удаления на экран.
\end{itemize}

\textbf{Пример использования:}

\begin{verbatim}
garage = CarGarage()
garage.add_car("Toyota", "Camry")
garage.add_car("BMW", "X5")
garage.add_car("Ford", "Focus")
print("Автомобили в гараже:")
for brand, model in garage.cars:
    print(brand, model)
count = garage.car_count()
print("Всего автомобилей:", count)
garage.remove_car("BMW")
print("Автомобили после удаления BMW:")
for brand, model in garage.cars:
    print(brand, model)
count = garage.car_count()
print("Всего автомобилей:", count)
\end{verbatim}

\textbf{Вывод:}
\begin{verbatim}
Автомобили в гараже:
Toyota Camry
BMW X5
Ford Focus
Всего автомобилей: 3
Автомобили после удаления BMW:
Toyota Camry
Ford Focus
Всего автомобилей: 2
\end{verbatim}

\item[14] Написать программу на Python, которая создает класс \texttt{PetStore} для представления зоомагазина. Класс должен содержать методы для добавления и удаления животных, а также подсчета общего количества питомцев. Программа также должна создавать экземпляр класса \texttt{PetStore}, добавлять животных, удалять животных и выводить информацию о магазине на экран.

\begin{itemize}
    \item Создайте класс \texttt{PetStore} с методом \texttt{\_\_init\_\_}, который создает пустой список животных.
    \item Создайте метод \texttt{add\_pet}, который принимает вид животного и количество в качестве аргументов и добавляет их в список.
    \item Создайте метод \texttt{remove\_pet}, который удаляет животное из списка по виду.
    \item Создайте метод \texttt{total\_pets}, который возвращает общее количество всех питомцев в магазине.
    \item Создайте экземпляр класса \texttt{PetStore} и добавьте несколько видов животных.
    \item Выведите информацию о текущих животных на экран.
    \item Выведите общее количество питомцев на экран.
    \item Удалите один из видов животных и выведите обновленный список.
    \item Выведите общее количество питомцев после удаления на экран.
\end{itemize}

\textbf{Пример использования:}

\begin{verbatim}
store = PetStore()
store.add_pet("Кошки", 5)
store.add_pet("Собаки", 3)
store.add_pet("Попугаи", 10)
print("Животные в магазине:")
for species, count in store.pets:
    print(species, "-", count)
total = store.total_pets()
print("Всего питомцев:", total)
store.remove_pet("Собаки")
print("Животные после удаления собак:")
for species, count in store.pets:
    print(species, "-", count)
total = store.total_pets()
print("Всего питомцев:", total)
\end{verbatim}

\textbf{Вывод:}
\begin{verbatim}
Животные в магазине:
Кошки - 5
Собаки - 3
Попугаи - 10
Всего питомцев: 18
Животные после удаления собак:
Кошки - 5
Попугаи - 10
Всего питомцев: 15
\end{verbatim}

\item[15] Написать программу на Python, которая создает класс \texttt{CourseRoster} для представления списка студентов на курсе. Класс должен содержать методы для добавления и удаления студентов, а также подсчета общего количества учащихся. Программа также должна создавать экземпляр класса \texttt{CourseRoster}, добавлять студентов, удалять студентов и выводить информацию о курсе на экран.

\begin{itemize}
    \item Создайте класс \texttt{CourseRoster} с методом \texttt{\_\_init\_\_}, который создает пустой список студентов.
    \item Создайте метод \texttt{enroll\_student}, который принимает имя студента и его ID в качестве аргументов и добавляет их в список.
    \item Создайте метод \texttt{drop\_student}, который удаляет студента из списка по имени.
    \item Создайте метод \texttt{student\_count}, который возвращает общее количество студентов на курсе.
    \item Создайте экземпляр класса \texttt{CourseRoster} и добавьте несколько студентов.
    \item Выведите информацию о текущих студентах на экран.
    \item Выведите общее количество студентов на экран.
    \item Удалите одного из студентов и выведите обновленный список.
    \item Выведите общее количество студентов после удаления на экран.
\end{itemize}

\textbf{Пример использования:}

\begin{verbatim}
roster = CourseRoster()
roster.enroll_student("Иван", 101)
roster.enroll_student("Мария", 102)
roster.enroll_student("Алексей", 103)
print("Студенты на курсе:")
for name, sid in roster.students:
    print(name, "(ID:", sid, ")")
count = roster.student_count()
print("Всего студентов:", count)
roster.drop_student("Мария")
print("Студенты после отчисления Марии:")
for name, sid in roster.students:
    print(name, "(ID:", sid, ")")
count = roster.student_count()
print("Всего студентов:", count)
\end{verbatim}

\textbf{Вывод:}
\begin{verbatim}
Студенты на курсе:
Иван (ID: 101 )
Мария (ID: 102 )
Алексей (ID: 103 )
Всего студентов: 3
Студенты после отчисления Марии:
Иван (ID: 101 )
Алексей (ID: 103 )
Всего студентов: 2
\end{verbatim}

\item[16] Написать программу на Python, которая создает класс \texttt{TravelItinerary} для представления туристического маршрута. Класс должен содержать методы для добавления и удаления мест, а также подсчета общего количества пунктов назначения. Программа также должна создавать экземпляр класса \texttt{TravelItinerary}, добавлять места, удалять места и выводить информацию о маршруте на экран.

\begin{itemize}
    \item Создайте класс \texttt{TravelItinerary} с методом \texttt{\_\_init\_\_}, который создает пустой список мест.
    \item Создайте метод \texttt{add\_destination}, который принимает название города и количество дней пребывания в качестве аргументов и добавляет их в список.
    \item Создайте метод \texttt{remove\_destination}, который удаляет место из списка по названию города.
    \item Создайте метод \texttt{destination\_count}, который возвращает общее количество пунктов назначения в маршруте.
    \item Создайте экземпляр класса \texttt{TravelItinerary} и добавьте несколько городов.
    \item Выведите информацию о текущих местах на экран.
    \item Выведите общее количество пунктов назначения на экран.
    \item Удалите один из городов и выведите обновленный маршрут.
    \item Выведите общее количество пунктов назначения после удаления на экран.
\end{itemize}

\textbf{Пример использования:}

\begin{verbatim}
itinerary = TravelItinerary()
itinerary.add_destination("Париж", 4)
itinerary.add_destination("Рим", 3)
itinerary.add_destination("Барселона", 5)
print("Маршрут путешествия:")
for city, days in itinerary.destinations:
    print(city, "-", days, "дней")
count = itinerary.destination_count()
print("Всего пунктов:", count)
itinerary.remove_destination("Рим")
print("Маршрут после удаления Рима:")
for city, days in itinerary.destinations:
    print(city, "-", days, "дней")
count = itinerary.destination_count()
print("Всего пунктов:", count)
\end{verbatim}

\textbf{Вывод:}
\begin{verbatim}
Маршрут путешествия:
Париж - 4 дней
Рим - 3 дней
Барселона - 5 дней
Всего пунктов: 3
Маршрут после удаления Рима:
Париж - 4 дней
Барселона - 5 дней
Всего пунктов: 2
\end{verbatim}

\item[17] Написать программу на Python, которая создает класс \texttt{FitnessTracker} для представления тренировочного плана. Класс должен содержать методы для добавления и удаления упражнений, а также подсчета общего количества подходов. Программа также должна создавать экземпляр класса \texttt{FitnessTracker}, добавлять упражнения, удалять упражнения и выводить информацию о плане на экран.

\begin{itemize}
    \item Создайте класс \texttt{FitnessTracker} с методом \texttt{\_\_init\_\_}, который создает пустой список упражнений.
    \item Создайте метод \texttt{add\_exercise}, который принимает название упражнения и количество подходов в качестве аргументов и добавляет их в список.
    \item Создайте метод \texttt{remove\_exercise}, который удаляет упражнение из списка по его названию.
    \item Создайте метод \texttt{total\_sets}, который возвращает общее количество подходов по всем упражнениям.
    \item Создайте экземпляр класса \texttt{FitnessTracker} и добавьте несколько упражнений.
    \item Выведите информацию о текущих упражнениях на экран.
    \item Выведите общее количество подходов на экран.
    \item Удалите одно из упражнений и выведите обновленный план.
    \item Выведите общее количество подходов после удаления на экран.
\end{itemize}

\textbf{Пример использования:}

\begin{verbatim}
tracker = FitnessTracker()
tracker.add_exercise("Приседания", 4)
tracker.add_exercise("Отжимания", 3)
tracker.add_exercise("Подтягивания", 5)
print("Тренировочный план:")
for ex, sets in tracker.exercises:
    print(ex, "-", sets, "подходов")
total = tracker.total_sets()
print("Всего подходов:", total)
tracker.remove_exercise("Отжимания")
print("План после удаления отжиманий:")
for ex, sets in tracker.exercises:
    print(ex, "-", sets, "подходов")
total = tracker.total_sets()
print("Всего подходов:", total)
\end{verbatim}

\textbf{Вывод:}
\begin{verbatim}
Тренировочный план:
Приседания - 4 подходов
Отжимания - 3 подходов
Подтягивания - 5 подходов
Всего подходов: 12
План после удаления отжиманий:
Приседания - 4 подходов
Подтягивания - 5 подходов
Всего подходов: 9
\end{verbatim}

\item[18] Написать программу на Python, которая создает класс \texttt{ExpenseTracker} для представления расходов. Класс должен содержать методы для добавления и удаления трат, а также подсчета общей суммы расходов. Программа также должна создавать экземпляр класса \texttt{ExpenseTracker}, добавлять расходы, удалять расходы и выводить информацию о тратах на экран.

\begin{itemize}
    \item Создайте класс \texttt{ExpenseTracker} с методом \texttt{\_\_init\_\_}, который создает пустой список расходов.
    \item Создайте метод \texttt{add\_expense}, который принимает категорию и сумму в качестве аргументов и добавляет их в список.
    \item Создайте метод \texttt{remove\_expense}, который удаляет расход из списка по категории.
    \item Создайте метод \texttt{total\_expenses}, который возвращает общую сумму всех расходов.
    \item Создайте экземпляр класса \texttt{ExpenseTracker} и добавьте несколько расходов.
    \item Выведите информацию о текущих тратах на экран.
    \item Выведите общую сумму расходов на экран.
    \item Удалите один из расходов и выведите обновленный список.
    \item Выведите общую сумму расходов после удаления на экран.
\end{itemize}

\textbf{Пример использования:}

\begin{verbatim}
expenses = ExpenseTracker()
expenses.add_expense("Продукты", 2500)
expenses.add_expense("Транспорт", 800)
expenses.add_expense("Развлечения", 1200)
print("Расходы:")
for cat, amount in expenses.expenses:
    print(cat, "-", amount, "руб.")
total = expenses.total_expenses()
print("Общая сумма расходов:", total, "руб.")
expenses.remove_expense("Транспорт")
print("Расходы после удаления транспорта:")
for cat, amount in expenses.expenses:
    print(cat, "-", amount, "руб.")
total = expenses.total_expenses()
print("Общая сумма расходов:", total, "руб.")
\end{verbatim}

\textbf{Вывод:}
\begin{verbatim}
Расходы:
Продукты - 2500 руб.
Транспорт - 800 руб.
Развлечения - 1200 руб.
Общая сумма расходов: 4500 руб.
Расходы после удаления транспорта:
Продукты - 2500 руб.
Развлечения - 1200 руб.
Общая сумма расходов: 3700 руб.
\end{verbatim}

\item[19] Написать программу на Python, которая создает класс \texttt{ProjectTasks} для представления задач проекта. Класс должен содержать методы для добавления и удаления задач, а также подсчета общего количества задач. Программа также должна создавать экземпляр класса \texttt{ProjectTasks}, добавлять задачи, удалять задачи и выводить информацию о проекте на экран.

\begin{itemize}
    \item Создайте класс \texttt{ProjectTasks} с методом \texttt{\_\_init\_\_}, который создает пустой список задач.
    \item Создайте метод \texttt{add\_task}, который принимает описание задачи и срок выполнения (в днях) в качестве аргументов и добавляет их в список.
    \item Создайте метод \texttt{remove\_task}, который удаляет задачу из списка по её описанию.
    \item Создайте метод \texttt{task\_count}, который возвращает общее количество задач в проекте.
    \item Создайте экземпляр класса \texttt{ProjectTasks} и добавьте несколько задач.
    \item Выведите информацию о текущих задачах на экран.
    \item Выведите общее количество задач на экран.
    \item Удалите одну из задач и выведите обновленный список.
    \item Выведите общее количество задач после удаления на экран.
\end{itemize}

\textbf{Пример использования:}

\begin{verbatim}
project = ProjectTasks()
project.add_task("Разработка интерфейса", 5)
project.add_task("Тестирование", 3)
project.add_task("Документация", 2)
print("Задачи проекта:")
for desc, days in project.tasks:
    print(desc, "-", days, "дней")
count = project.task_count()
print("Всего задач:", count)
project.remove_task("Тестирование")
print("Задачи после удаления тестирования:")
for desc, days in project.tasks:
    print(desc, "-", days, "дней")
count = project.task_count()
print("Всего задач:", count)
\end{verbatim}

\textbf{Вывод:}
\begin{verbatim}
Задачи проекта:
Разработка интерфейса - 5 дней
Тестирование - 3 дней
Документация - 2 дней
Всего задач: 3
Задачи после удаления тестирования:
Разработка интерфейса - 5 дней
Документация - 2 дней
Всего задач: 2
\end{verbatim}

\item[20] Написать программу на Python, которая создает класс \texttt{EventSchedule} для представления расписания мероприятий. Класс должен содержать методы для добавления и удаления событий, а также подсчета общего количества мероприятий. Программа также должна создавать экземпляр класса \texttt{EventSchedule}, добавлять события, удалять события и выводить информацию о расписании на экран.

\begin{itemize}
    \item Создайте класс \texttt{EventSchedule} с методом \texttt{\_\_init\_\_}, который создает пустой список мероприятий.
    \item Создайте метод \texttt{add\_event}, который принимает название мероприятия и дату проведения в качестве аргументов и добавляет их в список.
    \item Создайте метод \texttt{remove\_event}, который удаляет мероприятие из списка по его названию.
    \item Создайте метод \texttt{event\_count}, который возвращает общее количество мероприятий в расписании.
    \item Создайте экземпляр класса \texttt{EventSchedule} и добавьте несколько мероприятий.
    \item Выведите информацию о текущих мероприятиях на экран.
    \item Выведите общее количество мероприятий на экран.
    \item Удалите одно из мероприятий и выведите обновленное расписание.
    \item Выведите общее количество мероприятий после удаления на экран.
\end{itemize}

\textbf{Пример использования:}

\begin{verbatim}
schedule = EventSchedule()
schedule.add_event("Конференция", "15.05.2024")
schedule.add_event("Воркшоп", "20.05.2024")
schedule.add_event("Выставка", "25.05.2024")
print("Расписание мероприятий:")
for name, date in schedule.events:
    print(name, "-", date)
count = schedule.event_count()
print("Всего мероприятий:", count)
schedule.remove_event("Воркшоп")
print("Расписание после удаления воркшопа:")
for name, date in schedule.events:
    print(name, "-", date)
count = schedule.event_count()
print("Всего мероприятий:", count)
\end{verbatim}

\textbf{Вывод:}
\begin{verbatim}
Расписание мероприятий:
Конференция - 15.05.2024
Воркшоп - 20.05.2024
Выставка - 25.05.2024
Всего мероприятий: 3
Расписание после удаления воркшопа:
Конференция - 15.05.2024
Выставка - 25.05.2024
Всего мероприятий: 2
\end{verbatim}

\item[21] Написать программу на Python, которая создает класс \texttt{GardenPlanner} для представления садового участка. Класс должен содержать методы для добавления и удаления растений, а также подсчета общего количества видов растений. Программа также должна создавать экземпляр класса \texttt{GardenPlanner}, добавлять растения, удалять растения и выводить информацию о саде на экран.

\begin{itemize}
    \item Создайте класс \texttt{GardenPlanner} с методом \texttt{\_\_init\_\_}, который создает пустой список растений.
    \item Создайте метод \texttt{add\_plant}, который принимает название растения и количество экземпляров в качестве аргументов и добавляет их в список.
    \item Создайте метод \texttt{remove\_plant}, который удаляет растение из списка по его названию.
    \item Создайте метод \texttt{plant\_count}, который возвращает общее количество различных видов растений в саду.
    \item Создайте экземпляр класса \texttt{GardenPlanner} и добавьте несколько растений.
    \item Выведите информацию о текущих растениях на экран.
    \item Выведите общее количество видов растений на экран.
    \item Удалите одно из растений и выведите обновленный список.
    \item Выведите общее количество видов растений после удаления на экран.
\end{itemize}

\textbf{Пример использования:}

\begin{verbatim}
garden = GardenPlanner()
garden.add_plant("Розы", 10)
garden.add_plant("Тюльпаны", 20)
garden.add_plant("Лаванда", 5)
print("Растения в саду:")
for name, qty in garden.plants:
    print(name, "-", qty)
count = garden.plant_count()
print("Всего видов растений:", count)
garden.remove_plant("Тюльпаны")
print("Растения после удаления тюльпанов:")
for name, qty in garden.plants:
    print(name, "-", qty)
count = garden.plant_count()
print("Всего видов растений:", count)
\end{verbatim}

\textbf{Вывод:}
\begin{verbatim}
Растения в саду:
Розы - 10
Тюльпаны - 20
Лаванда - 5
Всего видов растений: 3
Растения после удаления тюльпанов:
Розы - 10
Лаванда - 5
Всего видов растений: 2
\end{verbatim}

\item[22] Написать программу на Python, которая создает класс \texttt{Warehouse} для представления склада товаров. Класс должен содержать методы для добавления и удаления товаров, а также подсчета общего количества типов товаров. Программа также должна создавать экземпляр класса \texttt{Warehouse}, добавлять товары, удалять товары и выводить информацию о складе на экран.

\begin{itemize}
    \item Создайте класс \texttt{Warehouse} с методом \texttt{\_\_init\_\_}, который создает пустой список товаров.
    \item Создайте метод \texttt{add\_product}, который принимает название товара и количество единиц в качестве аргументов и добавляет их в список.
    \item Создайте метод \texttt{remove\_product}, который удаляет товар из списка по его названию.
    \item Создайте метод \texttt{product\_types}, который возвращает общее количество различных типов товаров на складе.
    \item Создайте экземпляр класса \texttt{Warehouse} и добавьте несколько товаров.
    \item Выведите информацию о текущих товарах на экран.
    \item Выведите общее количество типов товаров на экран.
    \item Удалите один из товаров и выведите обновленный список.
    \item Выведите общее количество типов товаров после удаления на экран.
\end{itemize}

\textbf{Пример использования:}

\begin{verbatim}
warehouse = Warehouse()
warehouse.add_product("Стулья", 50)
warehouse.add_product("Столы", 20)
warehouse.add_product("Лампы", 100)
print("Товары на складе:")
for name, qty in warehouse.products:
    print(name, "-", qty)
types = warehouse.product_types()
print("Всего типов товаров:", types)
warehouse.remove_product("Столы")
print("Товары после удаления столов:")
for name, qty in warehouse.products:
    print(name, "-", qty)
types = warehouse.product_types()
print("Всего типов товаров:", types)
\end{verbatim}

\textbf{Вывод:}
\begin{verbatim}
Товары на складе:
Стулья - 50
Столы - 20
Лампы - 100
Всего типов товаров: 3
Товары после удаления столов:
Стулья - 50
Лампы - 100
Всего типов товаров: 2
\end{verbatim}

\item[23] Написать программу на Python, которая создает класс \texttt{GameInventory} для представления инвентаря игрока. Класс должен содержать методы для добавления и удаления предметов, а также подсчета общего количества типов предметов. Программа также должна создавать экземпляр класса \texttt{GameInventory}, добавлять предметы, удалять предметы и выводить информацию об инвентаре на экран.

\begin{itemize}
    \item Создайте класс \texttt{GameInventory} с методом \texttt{\_\_init\_\_}, который создает пустой список предметов.
    \item Создайте метод \texttt{add\_item}, который принимает название предмета и количество в качестве аргументов и добавляет их в список.
    \item Создайте метод \texttt{remove\_item}, который удаляет предмет из списка по его названию.
    \item Создайте метод \texttt{item\_types}, который возвращает общее количество различных типов предметов в инвентаре.
    \item Создайте экземпляр класса \texttt{GameInventory} и добавьте несколько предметов.
    \item Выведите информацию о текущих предметах на экран.
    \item Выведите общее количество типов предметов на экран.
    \item Удалите один из предметов и выведите обновленный инвентарь.
    \item Выведите общее количество типов предметов после удаления на экран.
\end{itemize}

\textbf{Пример использования:}

\begin{verbatim}
inventory = GameInventory()
inventory.add_item("Меч", 1)
inventory.add_item("Зелье", 5)
inventory.add_item("Щит", 1)
print("Инвентарь игрока:")
for name, qty in inventory.items:
    print(name, "-", qty)
types = inventory.item_types()
print("Всего типов предметов:", types)
inventory.remove_item("Зелье")
print("Инвентарь после удаления зелий:")
for name, qty in inventory.items:
    print(name, "-", qty)
types = inventory.item_types()
print("Всего типов предметов:", types)
\end{verbatim}

\textbf{Вывод:}
\begin{verbatim}
Инвентарь игрока:
Меч - 1
Зелье - 5
Щит - 1
Всего типов предметов: 3
Инвентарь после удаления зелий:
Меч - 1
Щит - 1
Всего типов предметов: 2
\end{verbatim}

\item[24] Написать программу на Python, которая создает класс \texttt{MusicAlbum} для представления музыкального альбома. Класс должен содержать методы для добавления и удаления треков, а также подсчета общего количества треков. Программа также должна создавать экземпляр класса \texttt{MusicAlbum}, добавлять треки, удалять треки и выводить информацию об альбоме на экран.

\begin{itemize}
    \item Создайте класс \texttt{MusicAlbum} с методом \texttt{\_\_init\_\_}, который создает пустой список треков.
    \item Создайте метод \texttt{add\_track}, который принимает название трека и его длительность (в секундах) в качестве аргументов и добавляет их в список.
    \item Создайте метод \texttt{remove\_track}, который удаляет трек из списка по его названию.
    \item Создайте метод \texttt{track\_count}, который возвращает общее количество треков в альбоме.
    \item Создайте экземпляр класса \texttt{MusicAlbum} и добавьте несколько треков.
    \item Выведите информацию о текущих треках на экран.
    \item Выведите общее количество треков на экран.
    \item Удалите один из треков и выведите обновленный список.
    \item Выведите общее количество треков после удаления на экран.
\end{itemize}

\textbf{Пример использования:}

\begin{verbatim}
album = MusicAlbum()
album.add_track("Yesterday", 125)
album.add_track("Hey Jude", 431)
album.add_track("Let It Be", 243)
print("Треки в альбоме:")
for name, duration in album.tracks:
    print(name, "-", duration, "сек.")
count = album.track_count()
print("Всего треков:", count)
album.remove_track("Hey Jude")
print("Треки после удаления 'Hey Jude':")
for name, duration in album.tracks:
    print(name, "-", duration, "сек.")
count = album.track_count()
print("Всего треков:", count)
\end{verbatim}

\textbf{Вывод:}
\begin{verbatim}
Треки в альбоме:
Yesterday - 125 сек.
Hey Jude - 431 сек.
Let It Be - 243 сек.
Всего треков: 3
Треки после удаления 'Hey Jude':
Yesterday - 125 сек.
Let It Be - 243 сек.
Всего треков: 2
\end{verbatim}

\item[25] Написать программу на Python, которая создает класс \texttt{EmployeeRoster} для представления списка сотрудников. Класс должен содержать методы для добавления и удаления сотрудников, а также подсчета общего количества работников. Программа также должна создавать экземпляр класса \texttt{EmployeeRoster}, добавлять сотрудников, удалять сотрудников и выводить информацию о персонале на экран.

\begin{itemize}
    \item Создайте класс \texttt{EmployeeRoster} с методом \texttt{\_\_init\_\_}, который создает пустой список сотрудников.
    \item Создайте метод \texttt{hire\_employee}, который принимает имя сотрудника и его должность в качестве аргументов и добавляет их в список.
    \item Создайте метод \texttt{fire\_employee}, который удаляет сотрудника из списка по имени.
    \item Создайте метод \texttt{employee\_count}, который возвращает общее количество сотрудников.
    \item Создайте экземпляр класса \texttt{EmployeeRoster} и добавьте несколько сотрудников.
    \item Выведите информацию о текущих сотрудниках на экран.
    \item Выведите общее количество работников на экран.
    \item Удалите одного из сотрудников и выведите обновленный список.
    \item Выведите общее количество работников после удаления на экран.
\end{itemize}

\textbf{Пример использования:}

\begin{verbatim}
roster = EmployeeRoster()
roster.hire_employee("Елена", "Менеджер")
roster.hire_employee("Дмитрий", "Разработчик")
roster.hire_employee("Ольга", "Дизайнер")
print("Сотрудники компании:")
for name, position in roster.employees:
    print(name, "-", position)
count = roster.employee_count()
print("Всего сотрудников:", count)
roster.fire_employee("Дмитрий")
print("Сотрудники после увольнения Дмитрия:")
for name, position in roster.employees:
    print(name, "-", position)
count = roster.employee_count()
print("Всего сотрудников:", count)
\end{verbatim}

\textbf{Вывод:}
\begin{verbatim}
Сотрудники компании:
Елена - Менеджер
Дмитрий - Разработчик
Ольга - Дизайнер
Всего сотрудников: 3
Сотрудники после увольнения Дмитрия:
Елена - Менеджер
Ольга - Дизайнер
Всего сотрудников: 2
\end{verbatim}

\item[26] Написать программу на Python, которая создает класс \texttt{ShoppingWishlist} для представления списка желаний покупателя. Класс должен содержать методы для добавления и удаления товаров, а также подсчета общего количества позиций. Программа также должна создавать экземпляр класса \texttt{ShoppingWishlist}, добавлять товары, удалять товары и выводить информацию о списке желаний на экран.

\begin{itemize}
    \item Создайте класс \texttt{ShoppingWishlist} с методом \texttt{\_\_init\_\_}, который создает пустой список товаров.
    \item Создайте метод \texttt{add\_item}, который принимает название товара и его приоритет (от 1 до 5) в качестве аргументов и добавляет их в список.
    \item Создайте метод \texttt{remove\_item}, который удаляет товар из списка по его названию.
    \item Создайте метод \texttt{item\_count}, который возвращает общее количество товаров в списке желаний.
    \item Создайте экземпляр класса \texttt{ShoppingWishlist} и добавьте несколько товаров.
    \item Выведите информацию о текущих товарах на экран.
    \item Выведите общее количество позиций на экран.
    \item Удалите один из товаров и выведите обновленный список.
    \item Выведите общее количество позиций после удаления на экран.
\end{itemize}

\textbf{Пример использования:}

\begin{verbatim}
wishlist = ShoppingWishlist()
wishlist.add_item("Наушники", 5)
wishlist.add_item("Книга", 3)
wishlist.add_item("Флешка", 2)
print("Список желаний:")
for name, priority in wishlist.items:
    print(name, "(приоритет", priority, ")")
count = wishlist.item_count()
print("Всего позиций:", count)
wishlist.remove_item("Книга")
print("Список после удаления книги:")
for name, priority in wishlist.items:
    print(name, "(приоритет", priority, ")")
count = wishlist.item_count()
print("Всего позиций:", count)
\end{verbatim}

\textbf{Вывод:}
\begin{verbatim}
Список желаний:
Наушники (приоритет 5 )
Книга (приоритет 3 )
Флешка (приоритет 2 )
Всего позиций: 3
Список после удаления книги:
Наушники (приоритет 5 )
Флешка (приоритет 2 )
Всего позиций: 2
\end{verbatim}

\item[27] Написать программу на Python, которая создает класс \texttt{DietPlan} для представления плана питания. Класс должен содержать методы для добавления и удаления блюд, а также подсчета общего количества приемов пищи. Программа также должна создавать экземпляр класса \texttt{DietPlan}, добавлять блюда, удалять блюда и выводить информацию о плане на экран.

\begin{itemize}
    \item Создайте класс \texttt{DietPlan} с методом \texttt{\_\_init\_\_}, который создает пустой список блюд.
    \item Создайте метод \texttt{add\_meal}, который принимает название блюда и количество калорий в качестве аргументов и добавляет их в список.
    \item Создайте метод \texttt{remove\_meal}, который удаляет блюдо из списка по его названию.
    \item Создайте метод \texttt{meal\_count}, который возвращает общее количество блюд в плане.
    \item Создайте экземпляр класса \texttt{DietPlan} и добавьте несколько блюд.
    \item Выведите информацию о текущих блюдах на экран.
    \item Выведите общее количество приемов пищи на экран.
    \item Удалите одно из блюд и выведите обновленный план.
    \item Выведите общее количество приемов пищи после удаления на экран.
\end{itemize}

\textbf{Пример использования:}

\begin{verbatim}
diet = DietPlan()
diet.add_meal("Овсянка", 300)
diet.add_meal("Салат", 150)
diet.add_meal("Курица", 400)
print("План питания:")
for dish, calories in diet.meals:
    print(dish, "-", calories, "ккал")
count = diet.meal_count()
print("Всего приемов пищи:", count)
diet.remove_meal("Салат")
print("План после удаления салата:")
for dish, calories in diet.meals:
    print(dish, "-", calories, "ккал")
count = diet.meal_count()
print("Всего приемов пищи:", count)
\end{verbatim}

\textbf{Вывод:}
\begin{verbatim}
План питания:
Овсянка - 300 ккал
Салат - 150 ккал
Курица - 400 ккал
Всего приемов пищи: 3
План после удаления салата:
Овсянка - 300 ккал
Курица - 400 ккал
Всего приемов пищи: 2
\end{verbatim}

\item[28] Написать программу на Python, которая создает класс \texttt{PhotoAlbum} для представления фотоальбома. Класс должен содержать методы для добавления и удаления фотографий, а также подсчета общего количества снимков. Программа также должна создавать экземпляр класса \texttt{PhotoAlbum}, добавлять фотографии, удалять фотографии и выводить информацию об альбоме на экран.

\begin{itemize}
    \item Создайте класс \texttt{PhotoAlbum} с методом \texttt{\_\_init\_\_}, который создает пустой список фотографий.
    \item Создайте метод \texttt{add\_photo}, который принимает название фотографии и дату съемки в качестве аргументов и добавляет их в список.
    \item Создайте метод \texttt{remove\_photo}, который удаляет фотографию из списка по её названию.
    \item Создайте метод \texttt{photo\_count}, который возвращает общее количество фотографий в альбоме.
    \item Создайте экземпляр класса \texttt{PhotoAlbum} и добавьте несколько фотографий.
    \item Выведите информацию о текущих фотографиях на экран.
    \item Выведите общее количество снимков на экран.
    \item Удалите одну из фотографий и выведите обновленный альбом.
    \item Выведите общее количество снимков после удаления на экран.
\end{itemize}

\textbf{Пример использования:}

\begin{verbatim}
album = PhotoAlbum()
album.add_photo("Пляж", "2023-07-15")
album.add_photo("Горы", "2023-08-20")
album.add_photo("Семья", "2023-12-25")
print("Фотографии в альбоме:")
for name, date in album.photos:
    print(name, "-", date)
count = album.photo_count()
print("Всего фотографий:", count)
album.remove_photo("Горы")
print("Фотографии после удаления 'Горы':")
for name, date in album.photos:
    print(name, "-", date)
count = album.photo_count()
print("Всего фотографий:", count)
\end{verbatim}

\textbf{Вывод:}
\begin{verbatim}
Фотографии в альбоме:
Пляж - 2023-07-15
Горы - 2023-08-20
Семья - 2023-12-25
Всего фотографий: 3
Фотографии после удаления 'Горы':
Пляж - 2023-07-15
Семья - 2023-12-25
Всего фотографий: 2
\end{verbatim}

\item[29] Написать программу на Python, которая создает класс \texttt{StudyMaterials} для представления учебных материалов. Класс должен содержать методы для добавления и удаления материалов, а также подсчета общего количества ресурсов. Программа также должна создавать экземпляр класса \texttt{StudyMaterials}, добавлять материалы, удалять материалы и выводить информацию о ресурсах на экран.

\begin{itemize}
    \item Создайте класс \texttt{StudyMaterials} с методом \texttt{\_\_init\_\_}, который создает пустой список материалов.
    \item Создайте метод \texttt{add\_material}, который принимает название материала и тип (например, "книга", "видео", "статья") в качестве аргументов и добавляет их в список.
    \item Создайте метод \texttt{remove\_material}, который удаляет материал из списка по его названию.
    \item Создайте метод \texttt{material\_count}, который возвращает общее количество учебных материалов.
    \item Создайте экземпляр класса \texttt{StudyMaterials} и добавьте несколько материалов.
    \item Выведите информацию о текущих материалах на экран.
    \item Выведите общее количество ресурсов на экран.
    \item Удалите один из материалов и выведите обновленный список.
    \item Выведите общее количество ресурсов после удаления на экран.
\end{itemize}

\textbf{Пример использования:}

\begin{verbatim}
materials = StudyMaterials()
materials.add_material("Алгоритмы", "книга")
materials.add_material("Python для начинающих", "видео")
materials.add_material("Структуры данных", "статья")
print("Учебные материалы:")
for name, mtype in materials.materials:
    print(name, "-", mtype)
count = materials.material_count()
print("Всего материалов:", count)
materials.remove_material("Python для начинающих")
print("Материалы после удаления видео:")
for name, mtype in materials.materials:
    print(name, "-", mtype)
count = materials.material_count()
print("Всего материалов:", count)
\end{verbatim}

\textbf{Вывод:}
\begin{verbatim}
Учебные материалы:
Алгоритмы - книга
Python для начинающих - видео
Структуры данных - статья
Всего материалов: 3
Материалы после удаления видео:
Алгоритмы - книга
Структуры данных - статья
Всего материалов: 2
\end{verbatim}

\item[30] Написать программу на Python, которая создает класс \texttt{ArtCollection} для представления коллекции произведений искусства. Класс должен содержать методы для добавления и удаления работ, а также подсчета общего количества экспонатов. Программа также должна создавать экземпляр класса \texttt{ArtCollection}, добавлять работы, удалять работы и выводить информацию о коллекции на экран.

\begin{itemize}
    \item Создайте класс \texttt{ArtCollection} с методом \texttt{\_\_init\_\_}, который создает пустой список произведений.
    \item Создайте метод \texttt{add\_artwork}, который принимает название работы и имя художника в качестве аргументов и добавляет их в список.
    \item Создайте метод \texttt{remove\_artwork}, который удаляет работу из списка по её названию.
    \item Создайте метод \texttt{artwork\_count}, который возвращает общее количество произведений в коллекции.
    \item Создайте экземпляр класса \texttt{ArtCollection} и добавьте несколько работ.
    \item Выведите информацию о текущих произведениях на экран.
    \item Выведите общее количество экспонатов на экран.
    \item Удалите одну из работ и выведите обновленный список.
    \item Выведите общее количество экспонатов после удаления на экран.
\end{itemize}

\textbf{Пример использования:}

\begin{verbatim}
art = ArtCollection()
art.add_artwork("Звёздная ночь", "Ван Гог")
art.add_artwork("Мона Лиза", "Леонардо да Винчи")
art.add_artwork("Крик", "Мунк")
print("Произведения в коллекции:")
for title, artist in art.artworks:
    print(title, "—", artist)
count = art.artwork_count()
print("Всего экспонатов:", count)
art.remove_artwork("Мона Лиза")
print("Произведения после удаления 'Моны Лизы':")
for title, artist in art.artworks:
    print(title, "—", artist)
count = art.artwork_count()
print("Всего экспонатов:", count)
\end{verbatim}

\textbf{Вывод:}
\begin{verbatim}
Произведения в коллекции:
Звёздная ночь — Ван Гог
Мона Лиза — Леонардо да Винчи
Крик — Мунк
Всего экспонатов: 3
Произведения после удаления 'Моны Лизы':
Звёздная ночь — Ван Гог
Крик — Мунк
Всего экспонатов: 2
\end{verbatim}

\item[31] Написать программу на Python, которая создает класс \texttt{FlightSchedule} для представления расписания рейсов. Класс должен содержать методы для добавления и удаления рейсов, а также подсчета общего количества перелетов. Программа также должна создавать экземпляр класса \texttt{FlightSchedule}, добавлять рейсы, удалять рейсы и выводить информацию о расписании на экран.

\begin{itemize}
    \item Создайте класс \texttt{FlightSchedule} с методом \texttt{\_\_init\_\_}, который создает пустой список рейсов.
    \item Создайте метод \texttt{add\_flight}, который принимает номер рейса и пункт назначения в качестве аргументов и добавляет их в список.
    \item Создайте метод \texttt{remove\_flight}, который удаляет рейс из списка по его номеру.
    \item Создайте метод \texttt{flight\_count}, который возвращает общее количество рейсов в расписании.
    \item Создайте экземпляр класса \texttt{FlightSchedule} и добавьте несколько рейсов.
    \item Выведите информацию о текущих рейсах на экран.
    \item Выведите общее количество перелетов на экран.
    \item Удалите один из рейсов и выведите обновленное расписание.
    \item Выведите общее количество перелетов после удаления на экран.
\end{itemize}

\textbf{Пример использования:}

\begin{verbatim}
flights = FlightSchedule()
flights.add_flight("SU123", "Париж")
flights.add_flight("SU456", "Лондон")
flights.add_flight("SU789", "Рим")
print("Рейсы:")
for num, dest in flights.flights:
    print(num, "-", dest)
count = flights.flight_count()
print("Всего рейсов:", count)
flights.remove_flight("SU456")
print("Рейсы после отмены SU456:")
for num, dest in flights.flights:
    print(num, "-", dest)
count = flights.flight_count()
print("Всего рейсов:", count)
\end{verbatim}

\textbf{Вывод:}
\begin{verbatim}
Рейсы:
SU123 - Париж
SU456 - Лондон
SU789 - Рим
Всего рейсов: 3
Рейсы после отмены SU456:
SU123 - Париж
SU789 - Рим
Всего рейсов: 2
\end{verbatim}

\item[32] Написать программу на Python, которая создает класс \texttt{RecipeIngredients} для представления ингредиентов рецепта. Класс должен содержать методы для добавления и удаления ингредиентов, а также подсчета общего количества компонентов. Программа также должна создавать экземпляр класса \texttt{RecipeIngredients}, добавлять ингредиенты, удалять ингредиенты и выводить информацию о рецепте на экран.

\begin{itemize}
    \item Создайте класс \texttt{RecipeIngredients} с методом \texttt{\_\_init\_\_}, который создает пустой список ингредиентов.
    \item Создайте метод \texttt{add\_ingredient}, который принимает название ингредиента и количество (в граммах или штуках) в качестве аргументов и добавляет их в список.
    \item Создайте метод \texttt{remove\_ingredient}, который удаляет ингредиент из списка по его названию.
    \item Создайте метод \texttt{ingredient\_count}, который возвращает общее количество ингредиентов в рецепте.
    \item Создайте экземпляр класса \texttt{RecipeIngredients} и добавьте несколько ингредиентов.
    \item Выведите информацию о текущих ингредиентах на экран.
    \item Выведите общее количество компонентов на экран.
    \item Удалите один из ингредиентов и выведите обновленный список.
    \item Выведите общее количество компонентов после удаления на экран.
\end{itemize}

\textbf{Пример использования:}

\begin{verbatim}
recipe = RecipeIngredients()
recipe.add_ingredient("Мука", 200)
recipe.add_ingredient("Сахар", 100)
recipe.add_ingredient("Яйца", 2)
print("Ингредиенты рецепта:")
for name, qty in recipe.ingredients:
    print(name, "-", qty)
count = recipe.ingredient_count()
print("Всего ингредиентов:", count)
recipe.remove_ingredient("Сахар")
print("Ингредиенты после удаления сахара:")
for name, qty in recipe.ingredients:
    print(name, "-", qty)
count = recipe.ingredient_count()
print("Всего ингредиентов:", count)
\end{verbatim}

\textbf{Вывод:}
\begin{verbatim}
Ингредиенты рецепта:
Мука - 200
Сахар - 100
Яйца - 2
Всего ингредиентов: 3
Ингредиенты после удаления сахара:
Мука - 200
Яйца - 2
Всего ингредиентов: 2
\end{verbatim}

\item[33] Написать программу на Python, которая создает класс \texttt{WorkoutPlan} для представления плана тренировок. Класс должен содержать методы для добавления и удаления упражнений, а также подсчета общего количества упражнений. Программа также должна создавать экземпляр класса \texttt{WorkoutPlan}, добавлять упражнения, удалять упражнения и выводить информацию о плане на экран.

\begin{itemize}
    \item Создайте класс \texttt{WorkoutPlan} с методом \texttt{\_\_init\_\_}, который создает пустой список упражнений.
    \item Создайте метод \texttt{add\_exercise}, который принимает название упражнения и количество повторений в качестве аргументов и добавляет их в список.
    \item Создайте метод \texttt{remove\_exercise}, который удаляет упражнение из списка по его названию.
    \item Создайте метод \texttt{exercise\_count}, который возвращает общее количество упражнений в плане.
    \item Создайте экземпляр класса \texttt{WorkoutPlan} и добавьте несколько упражнений.
    \item Выведите информацию о текущих упражнениях на экран.
    \item Выведите общее количество упражнений на экран.
    \item Удалите одно из упражнений и выведите обновленный план.
    \item Выведите общее количество упражнений после удаления на экран.
\end{itemize}

\textbf{Пример использования:}

\begin{verbatim}
workout = WorkoutPlan()
workout.add_exercise("Бег", 30)
workout.add_exercise("Планка", 3)
workout.add_exercise("Приседания", 20)
print("План тренировки:")
for name, reps in workout.exercises:
    print(name, "-", reps)
count = workout.exercise_count()
print("Всего упражнений:", count)
workout.remove_exercise("Планка")
print("План после удаления планки:")
for name, reps in workout.exercises:
    print(name, "-", reps)
count = workout.exercise_count()
print("Всего упражнений:", count)
\end{verbatim}

\textbf{Вывод:}
\begin{verbatim}
План тренировки:
Бег - 30
Планка - 3
Приседания - 20
Всего упражнений: 3
План после удаления планки:
Бег - 30
Приседания - 20
Всего упражнений: 2
\end{verbatim}

\item[34] Написать программу на Python, которая создает класс \texttt{InventoryManager} для представления управления запасами. Класс должен содержать методы для добавления и удаления товаров, а также подсчета общего количества типов товаров. Программа также должна создавать экземпляр класса \texttt{InventoryManager}, добавлять товары, удалять товары и выводить информацию о запасах на экран.

\begin{itemize}
    \item Создайте класс \texttt{InventoryManager} с методом \texttt{\_\_init\_\_}, который создает пустой список товаров.
    \item Создайте метод \texttt{add\_product}, который принимает название товара и количество единиц в качестве аргументов и добавляет их в список.
    \item Создайте метод \texttt{remove\_product}, который удаляет товар из списка по его названию.
    \item Создайте метод \texttt{product\_types}, который возвращает общее количество различных типов товаров.
    \item Создайте экземпляр класса \texttt{InventoryManager} и добавьте несколько товаров.
    \item Выведите информацию о текущих товарах на экран.
    \item Выведите общее количество типов товаров на экран.
    \item Удалите один из товаров и выведите обновленный список.
    \item Выведите общее количество типов товаров после удаления на экран.
\end{itemize}

\textbf{Пример использования:}

\begin{verbatim}
inv = InventoryManager()
inv.add_product("Мыло", 100)
inv.add_product("Шампунь", 50)
inv.add_product("Зубная паста", 75)
print("Товары на складе:")
for name, qty in inv.products:
    print(name, "-", qty)
types = inv.product_types()
print("Всего типов товаров:", types)
inv.remove_product("Шампунь")
print("Товары после удаления шампуня:")
for name, qty in inv.products:
    print(name, "-", qty)
types = inv.product_types()
print("Всего типов товаров:", types)
\end{verbatim}

\textbf{Вывод:}
\begin{verbatim}
Товары на складе:
Мыло - 100
Шампунь - 50
Зубная паста - 75
Всего типов товаров: 3
Товары после удаления шампуня:
Мыло - 100
Зубная паста - 75
Всего типов товаров: 2
\end{verbatim}

\item[35] Написать программу на Python, которая создает класс \texttt{EventGuestList} для представления списка гостей мероприятия. Класс должен содержать методы для добавления и удаления гостей, а также подсчета общего количества приглашенных. Программа также должна создавать экземпляр класса \texttt{EventGuestList}, добавлять гостей, удалять гостей и выводить информацию о списке на экран.

\begin{itemize}
    \item Создайте класс \texttt{EventGuestList} с методом \texttt{\_\_init\_\_}, который создает пустой список гостей.
    \item Создайте метод \texttt{add\_guest}, который принимает имя гостя и его статус (например, "подтвержден", "ожидает") в качестве аргументов и добавляет их в список.
    \item Создайте метод \texttt{remove\_guest}, который удаляет гостя из списка по имени.
    \item Создайте метод \texttt{guest\_count}, который возвращает общее количество гостей в списке.
    \item Создайте экземпляр класса \texttt{EventGuestList} и добавьте несколько гостей.
    \item Выведите информацию о текущих гостях на экран.
    \item Выведите общее количество приглашенных на экран.
    \item Удалите одного из гостей и выведите обновленный список.
    \item Выведите общее количество приглашенных после удаления на экран.
\end{itemize}

\textbf{Пример использования:}

\begin{verbatim}
guests = EventGuestList()
guests.add_guest("Андрей", "подтвержден")
guests.add_guest("Светлана", "ожидает")
guests.add_guest("Михаил", "подтвержден")
print("Список гостей:")
for name, status in guests.guests:
    print(name, "-", status)
count = guests.guest_count()
print("Всего гостей:", count)
guests.remove_guest("Светлана")
print("Список после отмены Светланы:")
for name, status in guests.guests:
    print(name, "-", status)
count = guests.guest_count()
print("Всего гостей:", count)
\end{verbatim}

\textbf{Вывод:}
\begin{verbatim}
Список гостей:
Андрей - подтвержден
Светлана - ожидает
Михаил - подтвержден
Всего гостей: 3
Список после отмены Светланы:
Андрей - подтвержден
Михаил - подтвержден
Всего гостей: 2
\end{verbatim}
\end{enumerate}
\subsubsection{Задача 5}

\begin{enumerate}
\item[1] Написать программу на Python, которая создает класс \texttt{Bank}, представляющий банк. Класс должен содержать методы для создания учетных записей клиентов, внесения депозитов, снятия средств и проверки баланса. Программа также должна создавать экземпляр класса \texttt{Bank}, создавать учетные записи клиентов, вносить депозиты, снимать средства и проверять баланс.

\begin{itemize}
    \item Создайте класс \texttt{Bank} с методом \texttt{\_\_init\_\_}, который создает пустой словарь клиентов.
    \item Создайте метод \texttt{create\_account}, который принимает номер счета и начальный баланс в качестве аргументов. Метод должен проверять, существует ли уже номер счета в словаре клиентов. Если это так, он должен выводить сообщение об ошибке. В противном случае, он должен добавить номер счета в словарь клиентов с начальным балансом в качестве значения.
    \item Создайте метод \texttt{make\_deposit}, который принимает номер счета и сумму в качестве аргументов. Метод должен проверять, существует ли номер счета в словаре клиентов. Если это так, он должен добавить сумму к текущему балансу счета. Если номер счета не существует, он должен вывести сообщение об ошибке.
    \item Создайте метод \texttt{make\_withdrawal}, который принимает номер счета и сумму в качестве аргументов. Метод должен проверять, существует ли номер счета в словаре клиентов. Если это так, он должен проверить, достаточно ли средств на счете для снятия. Если это так, он должен вычесть сумму из текущего баланса счета. В противном случае, он должен вывести сообщение об ошибке, указывающее на недостаточность средств. Если номер счета не существует, он должен вывести сообщение об ошибке.
    \item Создайте метод \texttt{check\_balance}, который принимает номер счета в качестве аргумента. Метод должен проверять, существует ли номер счета в словаре клиентов. Если это так, он должен извлечь и вывести текущий баланс счета. Если номер счета не существует, он должен вывести сообщение об ошибке.
    \item Создайте экземпляр класса \texttt{Bank} и создайте учетные записи клиентов.
    \item Вносите депозиты на счета клиентов.
    \item Снимайте средства со счетов клиентов.
    \item Проверяйте баланс счетов клиентов.
\end{itemize}

\textbf{Пример использования:}

\begin{verbatim}
bank = Bank()
acno1 = "SB-123"
damt1 = 1000
print("Новый номер счета: ", acno1, " Внесенная сумма: ", damt1)
bank.create_account(acno1, damt1)
acno2 = "SB-124"
damt2 = 1500
print("Новый номер счета: ", acno2, " Внесенная сумма: ", damt2)
bank.create_account(acno2, damt2)
wamt1 = 600
print("\nДепозит средств: ", wamt1, " на счет № ", acno1)
bank.make_deposit(acno1, wamt1)
wamt2 = 350
print("Вывод средств: ", wamt2, " со счета № ", acno2)
bank.make_withdrawal(acno2, wamt2)
print("Номер расчетного счета: ", acno1)
bank.check_balance(acno1)
print("Номер расчетного счета: ", acno2)
bank.check_balance(acno2)
wamt3 = 1200
print("Вывод средств: ", wamt3, " со счета № ", acno2)
bank.make_withdrawal(acno2, wamt3)
acno3 = "SB-134"
print("Проверка баланса счета № ", acno3)
bank.check_balance(acno3)  # Non-existent account number
\end{verbatim}

\item[2] Написать программу на Python, которая создает класс \texttt{CreditUnion}, представляющий кредитный союз. Класс должен содержать методы для открытия счетов участников, пополнения баланса, снятия денег и запроса текущего состояния счета. Программа также должна создавать экземпляр класса \texttt{CreditUnion}, открывать счета, выполнять операции и проверять балансы.

\begin{itemize}
    \item Создайте класс \texttt{CreditUnion} с методом \texttt{\_\_init\_\_}, инициализирующим пустой словарь счетов.
    \item Создайте метод \texttt{open\_account}, принимающий идентификатор счета и стартовый остаток. Если счет уже существует, выведите ошибку; иначе — добавьте запись.
    \item Создайте метод \texttt{deposit}, принимающий идентификатор счета и сумму. Если счет существует, увеличьте баланс; иначе — сообщите об ошибке.
    \item Создайте метод \texttt{withdraw}, принимающий идентификатор счета и сумму. Если счет существует и средств достаточно, уменьшите баланс; иначе — выведите соответствующую ошибку.
    \item Создайте метод \texttt{get\_balance}, принимающий идентификатор счета. Если счет существует, выведите его баланс; иначе — сообщите об ошибке.
    \item Создайте экземпляр \texttt{CreditUnion}.
    \item Откройте несколько счетов.
    \item Выполните пополнения.
    \item Выполните снятия.
    \item Проверьте балансы.
\end{itemize}

\textbf{Пример использования:}

\begin{verbatim}
cu = CreditUnion()
cu.open_account("CU-001", 2000)
cu.open_account("CU-002", 500)
cu.deposit("CU-001", 300)
cu.withdraw("CU-002", 200)
cu.get_balance("CU-001")
cu.get_balance("CU-002")
cu.withdraw("CU-002", 400)  # недостаточно средств
cu.get_balance("CU-999")     # несуществующий счет
\end{verbatim}

\item[3] Написать программу на Python, которая создает класс \texttt{SavingsBank}, моделирующий сберегательный банк. Класс должен поддерживать создание счетов, внесение вкладов, снятие средств и проверку баланса. Программа должна демонстрировать работу всех методов на примере нескольких счетов.

\begin{itemize}
    \item Создайте класс \texttt{SavingsBank} с методом \texttt{\_\_init\_\_}, инициализирующим пустой словарь \texttt{accounts}.
    \item Метод \texttt{add\_account} принимает номер счета и начальный депозит. Если счет уже есть — ошибка; иначе — добавление.
    \item Метод \texttt{credit} принимает номер счета и сумму. При наличии счета — пополнение; иначе — ошибка.
    \item Метод \texttt{debit} принимает номер счета и сумму. При наличии счета и достаточном балансе — снятие; иначе — ошибка.
    \item Метод \texttt{show\_balance} принимает номер счета и выводит баланс или сообщение об ошибке.
    \item Создайте экземпляр \texttt{SavingsBank}.
    \item Добавьте два счета.
    \item Пополните один из них.
    \item Снимите средства с другого.
    \item Проверьте балансы обоих и несуществующего счета.
\end{itemize}

\textbf{Пример использования:}

\begin{verbatim}
sb = SavingsBank()
sb.add_account("SAV-101", 1000)
sb.add_account("SAV-102", 800)
sb.credit("SAV-101", 200)
sb.debit("SAV-102", 300)
sb.show_balance("SAV-101")
sb.show_balance("SAV-102")
sb.debit("SAV-102", 600)  # недостаточно
sb.show_balance("SAV-999") # несуществует
\end{verbatim}

\item[4] Написать программу на Python, которая создает класс \texttt{DigitalWallet}, представляющий цифровой кошелек. Класс должен поддерживать регистрацию кошельков, пополнение, списание и проверку баланса.

\begin{itemize}
    \item Создайте класс \texttt{DigitalWallet} с методом \texttt{\_\_init\_\_}, создающим пустой словарь \texttt{wallets}.
    \item Метод \texttt{register\_wallet} принимает ID кошелька и начальный баланс. Если ID занят — ошибка; иначе — регистрация.
    \item Метод \texttt{top\_up} принимает ID и сумму. При существовании кошелька — пополнение; иначе — ошибка.
    \item Метод \texttt{spend} принимает ID и сумму. При наличии кошелька и достаточном балансе — списание; иначе — ошибка.
    \item Метод \texttt{get\_wallet\_balance} принимает ID и выводит баланс или сообщение об ошибке.
    \item Создайте экземпляр \texttt{DigitalWallet}.
    \item Зарегистрируйте два кошелька.
    \item Пополните один.
    \item Потратьте с другого.
    \item Проверьте балансы и попытайтесь проверить несуществующий.
\end{itemize}

\textbf{Пример использования:}

\begin{verbatim}
dw = DigitalWallet()
dw.register_wallet("WAL-01", 500)
dw.register_wallet("WAL-02", 300)
dw.top_up("WAL-01", 100)
dw.spend("WAL-02", 150)
dw.get_wallet_balance("WAL-01")
dw.get_wallet_balance("WAL-02")
dw.spend("WAL-02", 200)  # недостаточно
dw.get_wallet_balance("WAL-99")  # несуществует
\end{verbatim}

\item[5] Написать программу на Python, которая создает класс \texttt{PaymentSystem}, моделирующий систему платежей. Класс должен поддерживать создание счетов, зачисление средств, списание и проверку баланса.

\begin{itemize}
    \item Создайте класс \texttt{PaymentSystem} с методом \texttt{\_\_init\_\_}, инициализирующим пустой словарь \texttt{accounts}.
    \item Метод \texttt{create\_user\_account} принимает идентификатор и начальный баланс. Если уже есть — ошибка; иначе — создание.
    \item Метод \texttt{credit\_account} принимает ID и сумму. При наличии счета — зачисление; иначе — ошибка.
    \item Метод \texttt{debit\_account} принимает ID и сумму. При наличии счета и достаточном балансе — списание; иначе — ошибка.
    \item Метод \texttt{check\_account\_balance} принимает ID и выводит баланс или ошибку.
    \item Создайте экземпляр \texttt{PaymentSystem}.
    \item Создайте два счета.
    \item Зачислите средства на один.
    \item Спишите с другого.
    \item Проверьте балансы и несуществующий счет.
\end{itemize}

\textbf{Пример использования:}

\begin{verbatim}
ps = PaymentSystem()
ps.create_user_account("USR-1", 1200)
ps.create_user_account("USR-2", 700)
ps.credit_account("USR-1", 300)
ps.debit_account("USR-2", 200)
ps.check_account_balance("USR-1")
ps.check_account_balance("USR-2")
ps.debit_account("USR-2", 600)  # недостаточно
ps.check_account_balance("USR-999")  # несуществует
\end{verbatim}

\item[6] Написать программу на Python, которая создает класс \texttt{MicroFinance}, представляющий микрофинансовую организацию. Класс должен поддерживать открытие счетов, пополнение, снятие и проверку баланса.

\begin{itemize}
    \item Создайте класс \texttt{MicroFinance} с методом \texttt{\_\_init\_\_}, создающим пустой словарь \texttt{clients}.
    \item Метод \texttt{open\_client\_account} принимает номер счета и стартовый баланс. Если счет существует — ошибка; иначе — открытие.
    \item Метод \texttt{fund\_account} принимает номер счета и сумму. При наличии счета — пополнение; иначе — ошибка.
    \item Метод \texttt{withdraw\_funds} принимает номер счета и сумму. При наличии счета и достаточном балансе — снятие; иначе — ошибка.
    \item Метод \texttt{view\_balance} принимает номер счета и выводит баланс или сообщение об ошибке.
    \item Создайте экземпляр \texttt{MicroFinance}.
    \item Откройте два счета.
    \item Пополните один.
    \item Снимите с другого.
    \item Проверьте балансы и несуществующий счет.
\end{itemize}

\textbf{Пример использования:}

\begin{verbatim}
mf = MicroFinance()
mf.open_client_account("MF-201", 900)
mf.open_client_account("MF-202", 400)
mf.fund_account("MF-201", 100)
mf.withdraw_funds("MF-202", 150)
mf.view_balance("MF-201")
mf.view_balance("MF-202")
mf.withdraw_funds("MF-202", 300)  # недостаточно
mf.view_balance("MF-999")  # несуществует
\end{verbatim}

\item[7] Написать программу на Python, которая создает класс \texttt{OnlineBank}, моделирующий онлайн-банк. Класс должен поддерживать регистрацию счетов, депозиты, выводы и проверку баланса.

\begin{itemize}
    \item Создайте класс \texttt{OnlineBank} с методом \texttt{\_\_init\_\_}, инициализирующим пустой словарь \texttt{accounts}.
    \item Метод \texttt{register\_account} принимает ID и начальный баланс. Если ID занят — ошибка; иначе — регистрация.
    \item Метод \texttt{deposit\_funds} принимает ID и сумму. При наличии счета — пополнение; иначе — ошибка.
    \item Метод \texttt{withdraw\_funds} принимает ID и сумму. При наличии счета и достаточном балансе — снятие; иначе — ошибка.
    \item Метод \texttt{check\_current\_balance} принимает ID и выводит баланс или ошибку.
    \item Создайте экземпляр \texttt{OnlineBank}.
    \item Зарегистрируйте два счета.
    \item Пополните один.
    \item Снимите с другого.
    \item Проверьте балансы и несуществующий счет.
\end{itemize}

\textbf{Пример использования:}

\begin{verbatim}
ob = OnlineBank()
ob.register_account("ONB-501", 1500)
ob.register_account("ONB-502", 600)
ob.deposit_funds("ONB-501", 200)
ob.withdraw_funds("ONB-502", 250)
ob.check_current_balance("ONB-501")
ob.check_current_balance("ONB-502")
ob.withdraw_funds("ONB-502", 400)  # недостаточно
ob.check_current_balance("ONB-999")  # несуществует
\end{verbatim}

\item[8] Написать программу на Python, которая создает класс \texttt{FinTechApp}, представляющий финтех-приложение. Класс должен поддерживать создание аккаунтов, пополнение, снятие и проверку баланса.

\begin{itemize}
    \item Создайте класс \texttt{FinTechApp} с методом \texttt{\_\_init\_\_}, создающим пустой словарь \texttt{users}.
    \item Метод \texttt{create\_user} принимает логин и начальный баланс. Если логин занят — ошибка; иначе — создание.
    \item Метод \texttt{add\_money} принимает логин и сумму. При наличии аккаунта — пополнение; иначе — ошибка.
    \item Метод \texttt{remove\_money} принимает логин и сумму. При наличии аккаунта и достаточном балансе — снятие; иначе — ошибка.
    \item Метод \texttt{get\_user\_balance} принимает логин и выводит баланс или ошибку.
    \item Создайте экземпляр \texttt{FinTechApp}.
    \item Создайте двух пользователей.
    \item Пополните одного.
    \item Снимите у другого.
    \item Проверьте балансы и несуществующего пользователя.
\end{itemize}

\textbf{Пример использования:}

\begin{verbatim}
ft = FinTechApp()
ft.create_user("alice", 2000)
ft.create_user("bob", 800)
ft.add_money("alice", 300)
ft.remove_money("bob", 200)
ft.get_user_balance("alice")
ft.get_user_balance("bob")
ft.remove_money("bob", 700)  # недостаточно
ft.get_user_balance("charlie")  # несуществует
\end{verbatim}

\item[9] Написать программу на Python, которая создает класс \texttt{CryptoWallet}, моделирующий криптовалютный кошелек. Класс должен поддерживать создание кошельков, пополнение, перевод и проверку баланса.

\begin{itemize}
    \item Создайте класс \texttt{CryptoWallet} с методом \texttt{\_\_init\_\_}, инициализирующим пустой словарь \texttt{wallets}.
    \item Метод \texttt{generate\_wallet} принимает адрес и начальный баланс. Если адрес уже есть — ошибка; иначе — создание.
    \item Метод \texttt{receive\_coins} принимает адрес и сумму. При наличии кошелька — пополнение; иначе — ошибка.
    \item Метод \texttt{send\_coins} принимает адрес и сумму. При наличии кошелька и достаточном балансе — списание; иначе — ошибка.
    \item Метод \texttt{check\_wallet\_balance} принимает адрес и выводит баланс или ошибку.
    \item Создайте экземпляр \texttt{CryptoWallet}.
    \item Создайте два кошелька.
    \item Пополните один.
    \item Отправьте с другого.
    \item Проверьте балансы и несуществующий адрес.
\end{itemize}

\textbf{Пример использования:}

\begin{verbatim}
cw = CryptoWallet()
cw.generate_wallet("0x1a2b", 10.5)
cw.generate_wallet("0x3c4d", 5.0)
cw.receive_coins("0x1a2b", 2.0)
cw.send_coins("0x3c4d", 1.5)
cw.check_wallet_balance("0x1a2b")
cw.check_wallet_balance("0x3c4d")
cw.send_coins("0x3c4d", 4.0)  # недостаточно
cw.check_wallet_balance("0x9999")  # несуществует
\end{verbatim}

\item[10] Написать программу на Python, которая создает класс \texttt{StudentFund}, представляющий студенческий фонд. Класс должен поддерживать создание счетов студентов, внесение средств, снятие и проверку баланса.

\begin{itemize}
    \item Создайте класс \texttt{StudentFund} с методом \texttt{\_\_init\_\_}, создающим пустой словарь \texttt{students}.
    \item Метод \texttt{enroll\_student} принимает ID студента и начальный грант. Если ID уже есть — ошибка; иначе — зачисление.
    \item Метод \texttt{add\_grant} принимает ID и сумму. При наличии студента — пополнение; иначе — ошибка.
    \item Метод \texttt{use\_funds} принимает ID и сумму. При наличии студента и достаточном балансе — списание; иначе — ошибка.
    \item Метод \texttt{view\_student\_balance} принимает ID и выводит баланс или ошибку.
    \item Создайте экземпляр \texttt{StudentFund}.
    \item Зачислите двух студентов.
    \item Пополните одного.
    \item Снимите у другого.
    \item Проверьте балансы и несуществующего студента.
\end{itemize}

\textbf{Пример использования:}

\begin{verbatim}
sf = StudentFund()
sf.enroll_student("STU-01", 5000)
sf.enroll_student("STU-02", 3000)
sf.add_grant("STU-01", 1000)
sf.use_funds("STU-02", 800)
sf.view_student_balance("STU-01")
sf.view_student_balance("STU-02")
sf.use_funds("STU-02", 2500)  # недостаточно
sf.view_student_balance("STU-99")  # несуществует
\end{verbatim}

\item[11] Написать программу на Python, которая создает класс \texttt{GameCurrency}, моделирующий внутриигровую валюту. Класс должен поддерживать создание аккаунтов игроков, начисление монет, трату и проверку баланса.

\begin{itemize}
    \item Создайте класс \texttt{GameCurrency} с методом \texttt{\_\_init\_\_}, инициализирующим пустой словарь \texttt{players}.
    \item Метод \texttt{create\_player} принимает ник и начальный баланс. Если ник занят — ошибка; иначе — создание.
    \item Метод \texttt{award\_coins} принимает ник и сумму. При наличии игрока — начисление; иначе — ошибка.
    \item Метод \texttt{spend\_coins} принимает ник и сумму. При наличии игрока и достаточном балансе — списание; иначе — ошибка.
    \item Метод \texttt{get\_player\_balance} принимает ник и выводит баланс или ошибку.
    \item Создайте экземпляр \texttt{GameCurrency}.
    \item Создайте двух игроков.
    \item Начислите одному.
    \item Потратьте у другого.
    \item Проверьте балансы и несуществующего игрока.
\end{itemize}

\textbf{Пример использования:}

\begin{verbatim}
gc = GameCurrency()
gc.create_player("hero1", 100)
gc.create_player("hero2", 75)
gc.award_coins("hero1", 25)
gc.spend_coins("hero2", 30)
gc.get_player_balance("hero1")
gc.get_player_balance("hero2")
gc.spend_coins("hero2", 50)  # недостаточно
gc.get_player_balance("hero99")  # несуществует
\end{verbatim}

\item[12] Написать программу на Python, которая создает класс \texttt{CharityFund}, представляющий благотворительный фонд. Класс должен поддерживать создание счетов доноров, получение пожертвований, выдачу средств и проверку баланса.

\begin{itemize}
    \item Создайте класс \texttt{CharityFund} с методом \texttt{\_\_init\_\_}, создающим пустой словарь \texttt{donors}.
    \item Метод \texttt{register\_donor} принимает ID и начальный взнос. Если ID есть — ошибка; иначе — регистрация.
    \item Метод \texttt{accept\_donation} принимает ID и сумму. При наличии донора — пополнение; иначе — ошибка.
    \item Метод \texttt{distribute\_funds} принимает ID и сумму. При наличии донора и достаточном балансе — списание; иначе — ошибка.
    \item Метод \texttt{check\_donor\_balance} принимает ID и выводит баланс или ошибку.
    \item Создайте экземпляр \texttt{CharityFund}.
    \item Зарегистрируйте двух доноров.
    \item Примите пожертвование от одного.
    \item Распределите средства от другого.
    \item Проверьте балансы и несуществующего донора.
\end{itemize}

\textbf{Пример использования:}

\begin{verbatim}
cf = CharityFund()
cf.register_donor("DON-1", 2000)
cf.register_donor("DON-2", 1500)
cf.accept_donation("DON-1", 500)
cf.distribute_funds("DON-2", 600)
cf.check_donor_balance("DON-1")
cf.check_donor_balance("DON-2")
cf.distribute_funds("DON-2", 1000)  # недостаточно
cf.check_donor_balance("DON-99")  # несуществует
\end{verbatim}

\item[13] Написать программу на Python, которая создает класс \texttt{TravelWallet}, моделирующий кошелек для путешествий. Класс должен поддерживать создание профилей, пополнение, оплату и проверку баланса.

\begin{itemize}
    \item Создайте класс \texttt{TravelWallet} с методом \texttt{\_\_init\_\_}, инициализирующим пустой словарь \texttt{profiles}.
    \item Метод \texttt{create\_profile} принимает имя профиля и начальный бюджет. Если профиль существует — ошибка; иначе — создание.
    \item Метод \texttt{load\_funds} принимает имя профиля и сумму. При наличии профиля — пополнение; иначе — ошибка.
    \item Метод \texttt{pay\_expense} принимает имя профиля и сумму. При наличии профиля и достаточном балансе — списание; иначе — ошибка.
    \item Метод \texttt{check\_budget} принимает имя профиля и выводит баланс или ошибку.
    \item Создайте экземпляр \texttt{TravelWallet}.
    \item Создайте два профиля.
    \item Пополните один.
    \item Оплатите по другому.
    \item Проверьте балансы и несуществующий профиль.
\end{itemize}

\textbf{Пример использования:}

\begin{verbatim}
tw = TravelWallet()
tw.create_profile("ParisTrip", 3000)
tw.create_profile("TokyoTrip", 2500)
tw.load_funds("ParisTrip", 500)
tw.pay_expense("TokyoTrip", 700)
tw.check_budget("ParisTrip")
tw.check_budget("TokyoTrip")
tw.pay_expense("TokyoTrip", 2000)  # недостаточно
tw.check_budget("LondonTrip")  # несуществует
\end{verbatim}

\item[14] Написать программу на Python, которая создает класс \texttt{SchoolFund}, представляющий школьный фонд. Класс должен поддерживать создание счетов классов, внесение средств, расход и проверку баланса.

\begin{itemize}
    \item Создайте класс \texttt{SchoolFund} с методом \texttt{\_\_init\_\_}, создающим пустой словарь \texttt{classes}.
    \item Метод \texttt{add\_class} принимает номер класса и начальный бюджет. Если класс уже есть — ошибка; иначе — добавление.
    \item Метод \texttt{collect\_money} принимает номер класса и сумму. При наличии класса — пополнение; иначе — ошибка.
    \item Метод \texttt{spend\_money} принимает номер класса и сумму. При наличии класса и достаточном бюджете — списание; иначе — ошибка.
    \item Метод \texttt{get\_class\_balance} принимает номер класса и выводит баланс или ошибку.
    \item Создайте экземпляр \texttt{SchoolFund}.
    \item Добавьте два класса.
    \item Соберите средства у одного.
    \item Потратьте у другого.
    \item Проверьте балансы и несуществующий класс.
\end{itemize}

\textbf{Пример использования:}

\begin{verbatim}
sf = SchoolFund()
sf.add_class("10A", 1200)
sf.add_class("11B", 900)
sf.collect_money("10A", 300)
sf.spend_money("11B", 400)
sf.get_class_balance("10A")
sf.get_class_balance("11B")
sf.spend_money("11B", 600)  # недостаточно
sf.get_class_balance("12C")  # несуществует
\end{verbatim}

\item[15] Написать программу на Python, которая создает класс \texttt{ClubAccount}, моделирующий счет клуба. Класс должен поддерживать создание счетов участников, пополнение взносами, снятие на мероприятия и проверку баланса.

\begin{itemize}
    \item Создайте класс \texttt{ClubAccount} с методом \texttt{\_\_init\_\_}, инициализирующим пустой словарь \texttt{members}.
    \item Метод \texttt{join\_club} принимает ID участника и вступительный взнос. Если ID есть — ошибка; иначе — добавление.
    \item Метод \texttt{pay\_dues} принимает ID и сумму. При наличии участника — пополнение; иначе — ошибка.
    \item Метод \texttt{request\_funds} принимает ID и сумму. При наличии участника и достаточном балансе — списание; иначе — ошибка.
    \item Метод \texttt{check\_member\_balance} принимает ID и выводит баланс или ошибку.
    \item Создайте экземпляр \texttt{ClubAccount}.
    \item Зарегистрируйте двух участников.
    \item Внесите взносы за одного.
    \item Запросите средства у другого.
    \item Проверьте балансы и несуществующего участника.
\end{itemize}

\textbf{Пример использования:}

\begin{verbatim}
ca = ClubAccount()
ca.join_club("MEM-01", 500)
ca.join_club("MEM-02", 400)
ca.pay_dues("MEM-01", 100)
ca.request_funds("MEM-02", 150)
ca.check_member_balance("MEM-01")
ca.check_member_balance("MEM-02")
ca.request_funds("MEM-02", 300)  # недостаточно
ca.check_member_balance("MEM-99")  # несуществует
\end{verbatim}

\item[16] Написать программу на Python, которая создает класс \texttt{ProjectBudget}, представляющий бюджет проекта. Класс должен поддерживать создание проектов, выделение средств, расход и проверку остатка.

\begin{itemize}
    \item Создайте класс \texttt{ProjectBudget} с методом \texttt{\_\_init\_\_}, создающим пустой словарь \texttt{projects}.
    \item Метод \texttt{initiate\_project} принимает код проекта и начальный бюджет. Если проект существует — ошибка; иначе — инициализация.
    \item Метод \texttt{allocate\_funds} принимает код проекта и сумму. При наличии проекта — пополнение; иначе — ошибка.
    \item Метод \texttt{expend\_funds} принимает код проекта и сумму. При наличии проекта и достаточном бюджете — списание; иначе — ошибка.
    \item Метод \texttt{check\_project\_balance} принимает код проекта и выводит баланс или ошибку.
    \item Создайте экземпляр \texttt{ProjectBudget}.
    \item Инициируйте два проекта.
    \item Выделите средства одному.
    \item Потратьте у другого.
    \item Проверьте балансы и несуществующий проект.
\end{itemize}

\textbf{Пример использования:}

\begin{verbatim}
pb = ProjectBudget()
pb.initiate_project("PRJ-Alpha", 10000)
pb.initiate_project("PRJ-Beta", 8000)
pb.allocate_funds("PRJ-Alpha", 2000)
pb.expend_funds("PRJ-Beta", 3000)
pb.check_project_balance("PRJ-Alpha")
pb.check_project_balance("PRJ-Beta")
pb.expend_funds("PRJ-Beta", 6000)  # недостаточно
pb.check_project_balance("PRJ-Gamma")  # несуществует
\end{verbatim}

\item[17] Написать программу на Python, которая создает класс \texttt{EventFund}, моделирующий фонд мероприятия. Класс должен поддерживать создание событий, сбор средств, оплату расходов и проверку баланса.

\begin{itemize}
    \item Создайте класс \texttt{EventFund} с методом \texttt{\_\_init\_\_}, инициализирующим пустой словарь \texttt{events}.
    \item Метод \texttt{create\_event} принимает название события и стартовый бюджет. Если событие есть — ошибка; иначе — создание.
    \item Метод \texttt{collect\_sponsorship} принимает название и сумму. При наличии события — пополнение; иначе — ошибка.
    \item Метод \texttt{pay\_vendor} принимает название и сумму. При наличии события и достаточном бюджете — списание; иначе — ошибка.
    \item Метод \texttt{view\_event\_balance} принимает название и выводит баланс или ошибку.
    \item Создайте экземпляр \texttt{EventFund}.
    \item Создайте два события.
    \item Соберите спонсорские средства для одного.
    \item Оплатите поставщика для другого.
    \item Проверьте балансы и несуществующее событие.
\end{itemize}

\textbf{Пример использования:}

\begin{verbatim}
ef = EventFund()
ef.create_event("Conference", 5000)
ef.create_event("Workshop", 3000)
ef.collect_sponsorship("Conference", 1500)
ef.pay_vendor("Workshop", 1000)
ef.view_event_balance("Conference")
ef.view_event_balance("Workshop")
ef.pay_vendor("Workshop", 2500)  # недостаточно
ef.view_event_balance("Seminar")  # несуществует
\end{verbatim}

\item[18] Написать программу на Python, которая создает класс \texttt{PersonalFinance}, представляющий личные финансы. Класс должен поддерживать создание категорий, пополнение доходами, списание расходами и проверку баланса.

\begin{itemize}
    \item Создайте класс \texttt{PersonalFinance} с методом \texttt{\_\_init\_\_}, создающим пустой словарь \texttt{categories}.
    \item Метод \texttt{add\_category} принимает название категории и начальный баланс. Если категория есть — ошибка; иначе — добавление.
    \item Метод \texttt{record\_income} принимает название и сумму. При наличии категории — пополнение; иначе — ошибка.
    \item Метод \texttt{record\_expense} принимает название и сумму. При наличии категории и достаточном балансе — списание; иначе — ошибка.
    \item Метод \texttt{check\_category\_balance} принимает название и выводит баланс или ошибку.
    \item Создайте экземпляр \texttt{PersonalFinance}.
    \item Добавьте две категории.
    \item Запишите доход в одну.
    \item Запишите расход в другую.
    \item Проверьте балансы и несуществующую категорию.
\end{itemize}

\textbf{Пример использования:}

\begin{verbatim}
pf = PersonalFinance()
pf.add_category("Salary", 25000)
pf.add_category("Entertainment", 2000)
pf.record_income("Salary", 5000)
pf.record_expense("Entertainment", 800)
pf.check_category_balance("Salary")
pf.check_category_balance("Entertainment")
pf.record_expense("Entertainment", 1500)  # недостаточно
pf.check_category_balance("Travel")  # несуществует
\end{verbatim}

\item[19] Написать программу на Python, которая создает класс \texttt{InvestmentAccount}, моделирующий инвестиционный счет. Класс должен поддерживать создание счетов, внесение капитала, снятие прибыли и проверку баланса.

\begin{itemize}
    \item Создайте класс \texttt{InvestmentAccount} с методом \texttt{\_\_init\_\_}, инициализирующим пустой словарь \texttt{accounts}.
    \item Метод \texttt{open\_investment} принимает ID счета и начальный капитал. Если счет есть — ошибка; иначе — открытие.
    \item Метод \texttt{invest\_more} принимает ID и сумму. При наличии счета — пополнение; иначе — ошибка.
    \item Метод \texttt{withdraw\_profit} принимает ID и сумму. При наличии счета и достаточном балансе — списание; иначе — ошибка.
    \item Метод \texttt{check\_investment\_balance} принимает ID и выводит баланс или ошибку.
    \item Создайте экземпляр \texttt{InvestmentAccount}.
    \item Откройте два счета.
    \item Инвестируйте дополнительно в один.
    \item Снимите прибыль с другого.
    \item Проверьте балансы и несуществующий счет.
\end{itemize}

\textbf{Пример использования:}

\begin{verbatim}
ia = InvestmentAccount()
ia.open_investment("INV-01", 10000)
ia.open_investment("INV-02", 7000)
ia.invest_more("INV-01", 2000)
ia.withdraw_profit("INV-02", 1500)
ia.check_investment_balance("INV-01")
ia.check_investment_balance("INV-02")
ia.withdraw_profit("INV-02", 6000)  # недостаточно
ia.check_investment_balance("INV-99")  # несуществует
\end{verbatim}

\item[20] Написать программу на Python, которая создает класс \texttt{FamilyBudget}, представляющий семейный бюджет. Класс должен поддерживать создание членов семьи, пополнение общими доходами, списание личными расходами и проверку баланса.

\begin{itemize}
    \item Создайте класс \texttt{FamilyBudget} с методом \texttt{\_\_init\_\_}, создающим пустой словарь \texttt{members}.
    \item Метод \texttt{add\_family\_member} принимает имя и начальный вклад. Если имя есть — ошибка; иначе — добавление.
    \item Метод \texttt{add\_income} принимает имя и сумму. При наличии члена — пополнение; иначе — ошибка.
    \item Метод \texttt{deduct\_expense} принимает имя и сумму. При наличии члена и достаточном балансе — списание; иначе — ошибка.
    \item Метод \texttt{check\_member\_balance} принимает имя и выводит баланс или ошибку.
    \item Создайте экземпляр \texttt{FamilyBudget}.
    \item Добавьте двух членов семьи.
    \item Добавьте доход одному.
    \item Спишите расход у другого.
    \item Проверьте балансы и несуществующего члена.
\end{itemize}

\textbf{Пример использования:}

\begin{verbatim}
fb = FamilyBudget()
fb.add_family_member("Mother", 20000)
fb.add_family_member("Father", 25000)
fb.add_income("Mother", 5000)
fb.deduct_expense("Father", 3000)
fb.check_member_balance("Mother")
fb.check_member_balance("Father")
fb.deduct_expense("Father", 23000)  # недостаточно
fb.check_member_balance("Child")  # несуществует
\end{verbatim}

\item[21] Написать программу на Python, которая создает класс \texttt{StartupFund}, моделирующий фонд стартапа. Класс должен поддерживать создание стартапов, привлечение инвестиций, оплату расходов и проверку баланса.

\begin{itemize}
    \item Создайте класс \texttt{StartupFund} с методом \texttt{\_\_init\_\_}, инициализирующим пустой словарь \texttt{startups}.
    \item Метод \texttt{launch\_startup} принимает название и начальный капитал. Если стартап есть — ошибка; иначе — запуск.
    \item Метод \texttt{attract\_investment} принимает название и сумму. При наличии стартапа — пополнение; иначе — ошибка.
    \item Метод \texttt{cover\_costs} принимает название и сумму. При наличии стартапа и достаточном балансе — списание; иначе — ошибка.
    \item Метод \texttt{check\_startup\_balance} принимает название и выводит баланс или ошибку.
    \item Создайте экземпляр \texttt{StartupFund}.
    \item Запустите два стартапа.
    \item Привлеките инвестиции в один.
    \item Покройте расходы другого.
    \item Проверьте балансы и несуществующий стартап.
\end{itemize}

\textbf{Пример использования:}

\begin{verbatim}
sf = StartupFund()
sf.launch_startup("TechApp", 50000)
sf.launch_startup("EcoShop", 30000)
sf.attract_investment("TechApp", 20000)
sf.cover_costs("EcoShop", 10000)
sf.check_startup_balance("TechApp")
sf.check_startup_balance("EcoShop")
sf.cover_costs("EcoShop", 25000)  # недостаточно
sf.check_startup_balance("FoodDelivery")  # несуществует
\end{verbatim}

\item[22] Написать программу на Python, которая создает класс \texttt{NonProfitAccount}, представляющий счет некоммерческой организации. Класс должен поддерживать создание проектов, получение грантов, оплату деятельности и проверку баланса.

\begin{itemize}
    \item Создайте класс \texttt{NonProfitAccount} с методом \texttt{\_\_init\_\_}, создающим пустой словарь \texttt{projects}.
    \item Метод \texttt{initiate\_nonprofit\_project} принимает ID и начальный грант. Если проект есть — ошибка; иначе — инициализация.
    \item Метод \texttt{receive\_grant} принимает ID и сумму. При наличии проекта — пополнение; иначе — ошибка.
    \item Метод \texttt{pay\_operational\_costs} принимает ID и сумму. При наличии проекта и достаточном балансе — списание; иначе — ошибка.
    \item Метод \texttt{check\_project\_funds} принимает ID и выводит баланс или ошибку.
    \item Создайте экземпляр \texttt{NonProfitAccount}.
    \item Инициируйте два проекта.
    \item Получите грант на один.
    \item Оплатите расходы другого.
    \item Проверьте балансы и несуществующий проект.
\end{itemize}

\textbf{Пример использования:}

\begin{verbatim}
np = NonProfitAccount()
np.initiate_nonprofit_project("EDU-01", 15000)
np.initiate_nonprofit_project("HEALTH-02", 12000)
np.receive_grant("EDU-01", 5000)
np.pay_operational_costs("HEALTH-02", 4000)
np.check_project_funds("EDU-01")
np.check_project_funds("HEALTH-02")
np.pay_operational_costs("HEALTH-02", 9000)  # недостаточно
np.check_project_funds("ENV-99")  # несуществует
\end{verbatim}

\item[23] Написать программу на Python, которая создает класс \texttt{FreelancerWallet}, моделирующий кошелек фрилансера. Класс должен поддерживать создание профилей, получение оплаты, оплату налогов и проверку баланса.

\begin{itemize}
    \item Создайте класс \texttt{FreelancerWallet} с методом \texttt{\_\_init\_\_}, инициализирующим пустой словарь \texttt{freelancers}.
    \item Метод \texttt{register\_freelancer} принимает ник и начальный баланс. Если ник есть — ошибка; иначе — регистрация.
    \item Метод \texttt{receive\_payment} принимает ник и сумму. При наличии фрилансера — пополнение; иначе — ошибка.
    \item Метод \texttt{pay\_taxes} принимает ник и сумму. При наличии фрилансера и достаточном балансе — списание; иначе — ошибка.
    \item Метод \texttt{check\_freelancer\_balance} принимает ник и выводит баланс или ошибку.
    \item Создайте экземпляр \texttt{FreelancerWallet}.
    \item Зарегистрируйте двух фрилансеров.
    \item Получите оплату для одного.
    \item Оплатите налоги для другого.
    \item Проверьте балансы и несуществующего фрилансера.
\end{itemize}

\textbf{Пример использования:}

\begin{verbatim}
fw = FreelancerWallet()
fw.register_freelancer("dev_alex", 0)
fw.register_freelancer("design_maria", 0)
fw.receive_payment("dev_alex", 10000)
fw.receive_payment("design_maria", 2500)
fw.pay_taxes("design_maria", 2000)
fw.check_freelancer_balance("dev_alex")
fw.check_freelancer_balance("design_maria")
fw.pay_taxes("design_maria", 1000)  # недостаточно
fw.check_freelancer_balance("writer_john")  # несуществует
\end{verbatim}

\item[24] Написать программу на Python, которая создает класс \texttt{RentalIncome}, представляющий доход от аренды. Класс должен поддерживать создание объектов недвижимости, получение арендной платы, оплату расходов и проверку баланса.

\begin{itemize}
    \item Создайте класс \texttt{RentalIncome} с методом \texttt{\_\_init\_\_}, создающим пустой словарь \texttt{properties}.
    \item Метод \texttt{add\_property} принимает адрес и начальный баланс. Если адрес есть — ошибка; иначе — добавление.
    \item Метод \texttt{collect\_rent} принимает адрес и сумму. При наличии объекта — пополнение; иначе — ошибка.
    \item Метод \texttt{pay\_maintenance} принимает адрес и сумму. При наличии объекта и достаточном балансе — списание; иначе — ошибка.
    \item Метод \texttt{check\_property\_balance} принимает адрес и выводит баланс или ошибку.
    \item Создайте экземпляр \texttt{RentalIncome}.
    \item Добавьте два объекта.
    \item Соберите арендную плату с одного.
    \item Оплатите обслуживание другого.
    \item Проверьте балансы и несуществующий адрес.
\end{itemize}

\textbf{Пример использования:}

\begin{verbatim}
ri = RentalIncome()
ri.add_property("123 Main St", 0)
ri.add_property("456 Oak Ave", 0)
ri.collect_rent("123 Main St", 2000)
ri.collect_rent("456 Oak Ave", 700)
ri.pay_maintenance("456 Oak Ave", 300)
ri.check_property_balance("123 Main St")
ri.check_property_balance("456 Oak Ave")
ri.pay_maintenance("456 Oak Ave", 500)  # недостаточно
ri.check_property_balance("789 Pine Rd")  # несуществует
\end{verbatim}

\item[25] Написать программу на Python, которая создает класс \texttt{ScholarshipFund}, моделирующий стипендиальный фонд. Класс должен поддерживать создание получателей, выдачу стипендий, возврат средств и проверку баланса.

\begin{itemize}
    \item Создайте класс \texttt{ScholarshipFund} с методом \texttt{\_\_init\_\_}, инициализирующим пустой словарь \texttt{recipients}.
    \item Метод \texttt{enroll\_recipient} принимает ID и начальную стипендию. Если ID есть — ошибка; иначе — зачисление.
    \item Метод \texttt{award\_scholarship} принимает ID и сумму. При наличии получателя — пополнение; иначе — ошибка.
    \item Метод \texttt{return\_funds} принимает ID и сумму. При наличии получателя и достаточном балансе — списание; иначе — ошибка.
    \item Метод \texttt{check\_recipient\_balance} принимает ID и выводит баланс или ошибку.
    \item Создайте экземпляр \texttt{ScholarshipFund}.
    \item Зачислите двух получателей.
    \item Выдайте стипендию одному.
    \item Примите возврат от другого.
    \item Проверьте балансы и несуществующего получателя.
\end{itemize}

\textbf{Пример использования:}

\begin{verbatim}
sf = ScholarshipFund()
sf.enroll_recipient("SCH-01", 5000)
sf.enroll_recipient("SCH-02", 4000)
sf.award_scholarship("SCH-01", 1000)
sf.return_funds("SCH-02", 500)
sf.check_recipient_balance("SCH-01")
sf.check_recipient_balance("SCH-02")
sf.return_funds("SCH-02", 4000)  # недостаточно
sf.check_recipient_balance("SCH-99")  # несуществует
\end{verbatim}

\item[26] Написать программу на Python, которая создает класс \texttt{Crowdfunding}, представляющий краудфандинговую платформу. Класс должен поддерживать создание кампаний, сбор средств, возврат пожертвований и проверку баланса.

\begin{itemize}
    \item Создайте класс \texttt{Crowdfunding} с методом \texttt{\_\_init\_\_}, создающим пустой словарь \texttt{campaigns}.
    \item Метод \texttt{start\_campaign} принимает название и начальный баланс. Если кампания есть — ошибка; иначе — создание.
    \item Метод \texttt{donate} принимает название и сумму. При наличии кампании — пополнение; иначе — ошибка.
    \item Метод \texttt{refund} принимает название и сумму. При наличии кампании и достаточном балансе — списание; иначе — ошибка.
    \item Метод \texttt{check\_campaign\_balance} принимает название и выводит баланс или ошибку.
    \item Создайте экземпляр \texttt{Crowdfunding}.
    \item Запустите две кампании.
    \item Пожертвуйте в одну.
    \item Верните средства из другой.
    \item Проверьте балансы и несуществующую кампанию.
\end{itemize}

\textbf{Пример использования:}

\begin{verbatim}
cf = Crowdfunding()
cf.start_campaign("BookPublish", 10000)
cf.start_campaign("ArtExhibit", 8000)
cf.donate("BookPublish", 3000)
cf.refund("ArtExhibit", 500)
cf.check_campaign_balance("BookPublish")
cf.check_campaign_balance("ArtExhibit")
cf.refund("ArtExhibit", 8000)  # недостаточно
cf.check_campaign_balance("FilmProject")  # несуществует
\end{verbatim}

\item[27] Написать программу на Python, которая создает класс \texttt{PiggyBank}, моделирующий копилку. Класс должен поддерживать создание копилок, добавление монет, извлечение средств и проверку баланса.

\begin{itemize}
    \item Создайте класс \texttt{PiggyBank} с методом \texttt{\_\_init\_\_}, инициализирующим пустой словарь \texttt{banks}.
    \item Метод \texttt{create\_piggy} принимает имя и начальную сумму. Если имя есть — ошибка; иначе — создание.
    \item Метод \texttt{add\_coins} принимает имя и сумму. При наличии копилки — пополнение; иначе — ошибка.
    \item Метод \texttt{break\_piggy} принимает имя и сумму. При наличии копилки и достаточном балансе — списание; иначе — ошибка.
    \item Метод \texttt{check\_piggy\_balance} принимает имя и выводит баланс или ошибку.
    \item Создайте экземпляр \texttt{PiggyBank}.
    \item Создайте две копилки.
    \item Добавьте монеты в одну.
    \item Разбейте другую частично.
    \item Проверьте балансы и несуществующую копилку.
\end{itemize}

\textbf{Пример использования:}

\begin{verbatim}
pb = PiggyBank()
pb.create_piggy("Vacation", 200)
pb.create_piggy("Gadget", 150)
pb.add_coins("Vacation", 100)
pb.break_piggy("Gadget", 50)
pb.check_piggy_balance("Vacation")
pb.check_piggy_balance("Gadget")
pb.break_piggy("Gadget", 120)  # недостаточно
pb.check_piggy_balance("Car")  # несуществует
\end{verbatim}

\item[28] Написать программу на Python, которая создает класс \texttt{BusinessAccount}, представляющий бизнес-счет. Класс должен поддерживать создание компаний, зачисление выручки, оплату счетов и проверку баланса.

\begin{itemize}
    \item Создайте класс \texttt{BusinessAccount} с методом \texttt{\_\_init\_\_}, создающим пустой словарь \texttt{companies}.
    \item Метод \texttt{register\_business} принимает название и начальный капитал. Если компания есть — ошибка; иначе — регистрация.
    \item Метод \texttt{record\_revenue} принимает название и сумму. При наличии компании — пополнение; иначе — ошибка.
    \item Метод \texttt{pay\_bills} принимает название и сумму. При наличии компании и достаточном балансе — списание; иначе — ошибка.
    \item Метод \texttt{check\_business\_balance} принимает название и выводит баланс или ошибку.
    \item Создайте экземпляр \texttt{BusinessAccount}.
    \item Зарегистрируйте две компании.
    \item Запишите выручку одной.
    \item Оплатите счета другой.
    \item Проверьте балансы и несуществующую компанию.
\end{itemize}

\textbf{Пример использования:}

\begin{verbatim}
ba = BusinessAccount()
ba.register_business("TechCorp", 50000)
ba.register_business("CafeLtd", 20000)
ba.record_revenue("TechCorp", 15000)
ba.pay_bills("CafeLtd", 3000)
ba.check_business_balance("TechCorp")
ba.check_business_balance("CafeLtd")
ba.pay_bills("CafeLtd", 18000)  # недостаточно
ba.check_business_balance("ShopInc")  # несуществует
\end{verbatim}

\item[29] Написать программу на Python, которая создает класс \texttt{GrantManager}, моделирующий управление грантами. Класс должен поддерживать создание грантов, выделение средств, отчетность и проверку баланса.

\begin{itemize}
    \item Создайте класс \texttt{GrantManager} с методом \texttt{\_\_init\_\_}, инициализирующим пустой словарь \texttt{grants}.
    \item Метод \texttt{issue\_grant} принимает код и сумму. Если код есть — ошибка; иначе — создание.
    \item Метод \texttt{disburse\_funds} принимает код и сумму. При наличии гранта и достаточном балансе — списание (выдача средств); иначе — ошибка.
    \item Метод \texttt{submit\_report} принимает код и сумму. При наличии гранта и достаточном балансе — списание; иначе — ошибка.
    \item Метод \texttt{check\_grant\_status} принимает код и выводит баланс или ошибку.
    \item Создайте экземпляр \texttt{GrantManager}.
    \item Выдайте два гранта.
    \item Распределите средства по одному.
    \item Подайте отчет по другому.
    \item Проверьте статусы и несуществующий грант.
\end{itemize}

\textbf{Пример использования:}

\begin{verbatim}
gm = GrantManager()
gm.issue_grant("GR-2024-01", 10000)
gm.issue_grant("GR-2024-02", 8000)
gm.disburse_funds("GR-2024-01", 4000)
gm.submit_report("GR-2024-02", 2000)
gm.check_grant_status("GR-2024-01")
gm.check_grant_status("GR-2024-02")
gm.submit_report("GR-2024-02", 7000)  # недостаточно
gm.check_grant_status("GR-2024-99")  # несуществует
\end{verbatim}

\item[30] Написать программу на Python, которая создает класс \texttt{SubscriptionService}, представляющий сервис подписок. Класс должен поддерживать создание пользователей, оплату подписок, возврат средств и проверку баланса.

\begin{itemize}
    \item Создайте класс \texttt{SubscriptionService} с методом \texttt{\_\_init\_\_}, создающим пустой словарь \texttt{subscribers}.
    \item Метод \texttt{subscribe\_user} принимает email и начальный баланс. Если email есть — ошибка; иначе — подписка.
    \item Метод \texttt{charge\_payment} принимает email и сумму. При наличии пользователя — пополнение; иначе — ошибка.
    \item Метод \texttt{refund\_payment} принимает email и сумму. При наличии пользователя и достаточном балансе — списание; иначе — ошибка.
    \item Метод \texttt{check\_subscription\_balance} принимает email и выводит баланс или ошибку.
    \item Создайте экземпляр \texttt{SubscriptionService}.
    \item Подпишите двух пользователей.
    \item Спишите оплату с одного.
    \item Верните средства другому.
    \item Проверьте балансы и несуществующий email.
\end{itemize}

\textbf{Пример использования:}

\begin{verbatim}
ss = SubscriptionService()
ss.subscribe_user("user1@example.com", 100)
ss.subscribe_user("user2@example.com", 80)
ss.charge_payment("user1@example.com", 20)
ss.refund_payment("user2@example.com", 10)
ss.check_subscription_balance("user1@example.com")
ss.check_subscription_balance("user2@example.com")
ss.refund_payment("user2@example.com", 80)  # недостаточно
ss.check_subscription_balance("user3@example.com")  # несуществует
\end{verbatim}

\item[31] Написать программу на Python, которая создает класс \texttt{LoyaltyProgram}, моделирующий программу лояльности. Класс должен поддерживать создание участников, начисление баллов, списание за вознаграждения и проверку баланса.

\begin{itemize}
    \item Создайте класс \texttt{LoyaltyProgram} с методом \texttt{\_\_init\_\_}, инициализирующим пустой словарь \texttt{members}.
    \item Метод \texttt{enroll\_member} принимает ID и начальные баллы. Если ID есть — ошибка; иначе — зачисление.
    \item Метод \texttt{earn\_points} принимает ID и количество. При наличии участника — пополнение; иначе — ошибка.
    \item Метод \texttt{redeem\_points} принимает ID и количество. При наличии участника и достаточном балансе — списание; иначе — ошибка.
    \item Метод \texttt{check\_points\_balance} принимает ID и выводит баланс или ошибку.
    \item Создайте экземпляр \texttt{LoyaltyProgram}.
    \item Зачислите двух участников.
    \item Начислите баллы одному.
    \item Спишите у другого.
    \item Проверьте балансы и несуществующего участника.
\end{itemize}

\textbf{Пример использования:}

\begin{verbatim}
lp = LoyaltyProgram()
lp.enroll_member("MEM-101", 500)
lp.enroll_member("MEM-102", 300)
lp.earn_points("MEM-101", 200)
lp.redeem_points("MEM-102", 100)
lp.check_points_balance("MEM-101")
lp.check_points_balance("MEM-102")
lp.redeem_points("MEM-102", 250)  # недостаточно
lp.check_points_balance("MEM-999")  # несуществует
\end{verbatim}

\item[32] Написать программу на Python, которая создает класс \texttt{UtilityBill}, представляющий оплату коммунальных услуг. Класс должен поддерживать создание лицевых счетов, внесение платежей, списание задолженностей и проверку баланса.

\begin{itemize}
    \item Создайте класс \texttt{UtilityBill} с методом \texttt{\_\_init\_\_}, создающим пустой словарь \texttt{accounts}.
    \item Метод \texttt{create\_utility\_account} принимает номер и начальный долг. Если номер есть — ошибка; иначе — создание.
    \item Метод \texttt{make\_payment} принимает номер и сумму. При наличии счета — пополнение; иначе — ошибка.
    \item Метод \texttt{apply\_charges} принимает номер и сумму. При наличии счета и достаточном балансе — списание; иначе — ошибка.
    \item Метод \texttt{check\_account\_status} принимает номер и выводит баланс или ошибку.
    \item Создайте экземпляр \texttt{UtilityBill}.
    \item Создайте два счета.
    \item Внесите платеж по одному.
    \item Начислите плату по другому.
    \item Проверьте статусы и несуществующий счет.
\end{itemize}

\textbf{Пример использования:}

\begin{verbatim}
ub = UtilityBill()
ub.create_utility_account("UTIL-01", 0)
ub.create_utility_account("UTIL-02", 0)
ub.make_payment("UTIL-01", 1500)
ub.make_payment("UTIL-02", 1200)
ub.apply_charges("UTIL-02", 800)
ub.check_account_status("UTIL-01")
ub.check_account_status("UTIL-02")
ub.apply_charges("UTIL-02", 1000)  # недостаточно
ub.check_account_status("UTIL-99")  # несуществует
\end{verbatim}

\item[33] Написать программу на Python, которая создает класс \texttt{InsuranceFund}, моделирующий страховой фонд. Класс должен поддерживать создание полисов, уплату премий, выплату возмещений и проверку баланса.

\begin{itemize}
    \item Создайте класс \texttt{InsuranceFund} с методом \texttt{\_\_init\_\_}, инициализирующим пустой словарь \texttt{policies}.
    \item Метод \texttt{issue\_policy} принимает номер полиса и начальный взнос. Если полис есть — ошибка; иначе — выдача.
    \item Метод \texttt{pay\_premium} принимает номер и сумму. При наличии полиса — пополнение; иначе — ошибка.
    \item Метод \texttt{process\_claim} принимает номер и сумму. При наличии полиса и достаточном балансе — списание; иначе — ошибка.
    \item Метод \texttt{check\_policy\_balance} принимает номер и выводит баланс или ошибку.
    \item Создайте экземпляр \texttt{InsuranceFund}.
    \item Выдайте два полиса.
    \item Уплатите премию по одному.
    \item Обработайте заявку по другому.
    \item Проверьте балансы и несуществующий полис.
\end{itemize}

\textbf{Пример использования:}

\begin{verbatim}
ifund = InsuranceFund()
ifund.issue_policy("POL-501", 10000)
ifund.issue_policy("POL-502", 8000)
ifund.pay_premium("POL-501", 2000)
ifund.process_claim("POL-502", 3000)
ifund.check_policy_balance("POL-501")
ifund.check_policy_balance("POL-502")
ifund.process_claim("POL-502", 6000)  # недостаточно
ifund.check_policy_balance("POL-999")  # несуществует
\end{verbatim}

\item[34] Написать программу на Python, которая создает класс \texttt{DonationBox}, представляющий ящик для пожертвований. Класс должен поддерживать создание ящиков, сбор средств, выдачу помощи и проверку баланса.

\begin{itemize}
    \item Создайте класс \texttt{DonationBox} с методом \texttt{\_\_init\_\_}, создающим пустой словарь \texttt{boxes}.
    \item Метод \texttt{install\_box} принимает локацию и начальный сбор. Если локация есть — ошибка; иначе — установка.
    \item Метод \texttt{collect\_donations} принимает локацию и сумму. При наличии ящика — пополнение; иначе — ошибка.
    \item Метод \texttt{distribute\_aid} принимает локацию и сумму. При наличии ящика и достаточном балансе — списание; иначе — ошибка.
    \item Метод \texttt{check\_box\_balance} принимает локацию и выводит баланс или ошибку.
    \item Создайте экземпляр \texttt{DonationBox}.
    \item Установите два ящика.
    \item Соберите пожертвования в один.
    \item Распределите помощь из другого.
    \item Проверьте балансы и несуществующую локацию.
\end{itemize}

\textbf{Пример использования:}

\begin{verbatim}
db = DonationBox()
db.install_box("Hospital", 0)
db.install_box("School", 0)
db.collect_donations("Hospital", 5000)
db.collect_donations("School", 1500)
db.distribute_aid("School", 1000)
db.check_box_balance("Hospital")
db.check_box_balance("School")
db.distribute_aid("School", 2000)  # недостаточно
db.check_box_balance("Park")  # несуществует
\end{verbatim}

\item[35] Написать программу на Python, которая создает класс \texttt{RewardWallet}, моделирующий кошелек вознаграждений. Класс должен поддерживать создание кошельков, начисление бонусов, списание за покупки и проверку баланса.

\begin{itemize}
    \item Создайте класс \texttt{RewardWallet} с методом \texttt{\_\_init\_\_}, инициализирующим пустой словарь \texttt{wallets}.
    \item Метод \texttt{activate\_wallet} принимает ID и начальные бонусы. Если ID есть — ошибка; иначе — активация.
    \item Метод \texttt{award\_bonus} принимает ID и сумму. При наличии кошелька — пополнение; иначе — ошибка.
    \item Метод \texttt{redeem\_reward} принимает ID и сумму. При наличии кошелька и достаточном балансе — списание; иначе — ошибка.
    \item Метод \texttt{check\_reward\_balance} принимает ID и выводит баланс или ошибку.
    \item Создайте экземпляр \texttt{RewardWallet}.
    \item Активируйте два кошелька.
    \item Начислите бонусы одному.
    \item Потратьте у другого.
    \item Проверьте балансы и несуществующий ID.
\end{itemize}

\textbf{Пример использования:}

\begin{verbatim}
rw = RewardWallet()
rw.activate_wallet("RW-001", 1000)
rw.activate_wallet("RW-002", 800)
rw.award_bonus("RW-001", 200)
rw.redeem_reward("RW-002", 300)
rw.check_reward_balance("RW-001")
rw.check_reward_balance("RW-002")
rw.redeem_reward("RW-002", 600)  # недостаточно
rw.check_reward_balance("RW-999")  # несуществует
\end{verbatim}
\end{enumerate}
