\begin{enumerate}
    \item Написать программу на Python, которая создает класс Stack для представления стека с инкапсуляцией внутреннего состояния. Класс должен содержать методы push, pop, is\_empty, size и peek, которые реализуют операции вталкивания, выталкивания, проверки пустоты, получения размера и просмотра вершины стека соответственно. Программа также должна создавать экземпляр класса Stack, вталкивать в него элементы, выталкивать элементы и выводить информацию о стеке на экран.

Инструкции:
\begin{enumerate}
    \item Создайте класс Stack с методом \_\_init\_\_, который принимает необязательный аргумент initial\_element. Если он передан, стек инициализируется с этим элементом (в виде списка из одного элемента), иначе — пустым списком.
    \item Создайте метод push, который принимает элемент в качестве аргумента и вталкивает его в стек только в том случае, если он не равен текущему верхнему элементу (если стек не пуст). Если стек пуст, элемент добавляется без проверки.
    \item Создайте метод pop, который выталкивает верхний элемент из стека и возвращает его. Если стек пуст, метод должен вернуть None и вывести сообщение "Стек пуст — извлечение невозможно" в стандартный поток ошибок (sys.stderr).
    \item Создайте метод is\_empty, который возвращает True, если стек пуст, и False в противном случае.
    \item Создайте метод size, который возвращает текущее количество элементов в стеке.
    \item Создайте метод peek, который возвращает верхний элемент стека, если стек не пуст. Если стек пуст, возвращает None и выводит сообщение "Стек пуст — просмотр невозможен" в sys.stderr.
    \item Создайте экземпляр класса Stack, передав в конструктор начальный элемент 10.
    \item Последовательно вызовите метод push с аргументами: 10, 20, 20, 30, 40 (обратите внимание, что повторяющийся элемент 20 не должен быть добавлен дважды подряд).
    \item Выведите размер стека и верхний элемент.
    \item Вызовите метод pop дважды, каждый раз выводя вытолкнутый элемент.
    \item После каждого pop выводите текущий размер стека и результат вызова peek.
\end{enumerate}

Пример использования:
\begin{lstlisting}[language=Python]
import sys

stack = Stack(10)
stack.push(10)   # не добавится, т.к. равен верхнему
stack.push(20)   # добавится
stack.push(20)   # не добавится, т.к. равен верхнему
stack.push(30)
stack.push(40)

print("Размер стека:", stack.size())
print("Верхний элемент:", stack.peek())

popped = stack.pop()
print("Вытолкнут:", popped)
print("Размер после pop:", stack.size())
print("Верхний элемент:", stack.peek())

popped = stack.pop()
print("Вытолкнут:", popped)
print("Размер после pop:", stack.size())
print("Верхний элемент:", stack.peek())
\end{lstlisting}

\item Написать программу на Python, которая создает класс Stack для представления стека с инкапсуляцией. Класс должен содержать методы push, pop, is\_empty, size и peek, которые реализуют операции вталкивания, выталкивания, проверки пустоты, получения размера и просмотра вершины стека соответственно. Программа также должна создавать экземпляр класса Stack, вталкивать в него элементы, выталкивать элементы и выводить информацию о стеке на экран.

Инструкции:
\begin{enumerate}
    \item Создайте класс Stack с методом \_\_init\_\_, который инициализирует пустой стек. Дополнительно принимает необязательный параметр max\_size, ограничивающий максимальное количество элементов в стеке (по умолчанию — None, то есть без ограничений).
    \item Создайте метод push, который принимает два аргумента: element и force=False. Элемент добавляется в стек, только если не превышает max\_size. Если force=True, то элемент добавляется даже при превышении лимита (с заменой самого нижнего элемента, если стек полон).
    \item Создайте метод pop, который выталкивает верхний элемент из стека и возвращает его. Если стек пуст, возвращает строку "Стек пуст".
    \item Создайте метод is\_empty, который возвращает True, если стек пуст, и False в противном случае.
    \item Создайте метод size, который возвращает текущее количество элементов в стеке.
    \item Создайте метод peek, который возвращает верхний элемент стека, если стек не пуст. Если стек пуст, возвращает строку "Нет элементов для просмотра".
    \item Создайте экземпляр класса Stack с max\_size=3.
    \item Последовательно вызовите push с элементами 5, 15, 25 (все добавятся).
    \item Попытайтесь добавить 35 без force — не должно добавиться.
    \item Добавьте 35 с force=True — должен замениться нижний элемент (5), стек станет [15, 25, 35].
    \item Выведите размер стека и верхний элемент.
    \item Вызовите pop и выведите результат.
    \item Повторите вывод размера и верхнего элемента.
\end{enumerate}

Пример использования:
\begin{lstlisting}[language=Python]
stack = Stack(max_size=3)
stack.push(5)
stack.push(15)
stack.push(25)
stack.push(35)          # не добавится
stack.push(35, force=True)  # добавится с заменой нижнего

print("Размер стека:", stack.size())
print("Верхний элемент:", stack.peek())

popped = stack.pop()
print("Вытолкнут:", popped)
print("Размер после pop:", stack.size())
print("Верхний элемент:", stack.peek())
\end{lstlisting}

\item Написать программу на Python, которая создает класс Stack для представления стека с инкапсуляцией. Класс должен содержать методы push, pop, is\_empty, size и peek, которые реализуют операции вталкивания, выталкивания, проверки пустоты, получения размера и просмотра вершины стека соответственно. Программа также должна создавать экземпляр класса Stack, вталкивать в него элементы, выталкивать элементы и выводить информацию о стеке на экран.

Инструкции:
\begin{enumerate}
    \item Создайте класс Stack с методом \_\_init\_\_, который инициализирует пустой стек. Может принимать список элементов в качестве аргумента items, который будет использован для первоначального заполнения стека (в порядке, как в списке: первый элемент — внизу стека).
    \item Создайте метод push, который принимает один элемент и добавляет его в стек. Если добавляемый элемент отрицательный, он не добавляется, а в sys.stderr выводится предупреждение "Отрицательные значения не допускаются".
    \item Создайте метод pop, который выталкивает верхний элемент из стека и возвращает его. Если стек пуст, выбрасывает исключение IndexError с сообщением "pop from empty stack".
    \item Создайте метод is\_empty, который возвращает True, если стек пуст, и False в противном случае.
    \item Создайте метод size, который возвращает текущее количество элементов в стеке.
    \item Создайте метод peek, который возвращает верхний элемент стека, если стек не пуст. Если стек пуст, выбрасывает исключение IndexError с сообщением "peek from empty stack".
    \item Создайте экземпляр класса Stack, передав в конструктор список [1, 2, 3].
    \item Добавьте элементы 4, -5 (не добавится), 6.
    \item Выведите размер стека и результат peek.
    \item Вызовите pop трижды, каждый раз выводя результат.
    \item После каждого pop проверяйте is\_empty и выводите результат.
\end{enumerate}

Пример использования:
\begin{lstlisting}[language=Python]
import sys

stack = Stack([1, 2, 3])
stack.push(4)
stack.push(-5)  # не добавится, выведет предупреждение
stack.push(6)

print("Размер стека:", stack.size())
print("Верхний элемент:", stack.peek())

for _ in range(3):
    popped = stack.pop()
    print("Вытолкнут:", popped)
    print("Стек пуст?", stack.is_empty())
\end{lstlisting}

\item Написать программу на Python, которая создает класс Stack для представления стека с инкапсуляцией. Класс должен содержать методы push, pop, is\_empty, size и peek, которые реализуют операции вталкивания, выталкивания, проверки пустоты, получения размера и просмотра вершины стека соответственно. Программа также должна создавать экземпляр класса Stack, вталкивать в него элементы, выталкивать элементы и выводить информацию о стеке на экран.

Инструкции:
\begin{enumerate}
    \item Создайте класс Stack с методом \_\_init\_\_, который инициализирует пустой стек. Принимает необязательный аргумент allow\_duplicates (по умолчанию True). Если False, то дубликаты (элементы, уже присутствующие в стеке) не добавляются.
    \item Создайте метод push, который принимает элемент и добавляет его в стек, только если allow\_duplicates=True или если такого элемента еще нет в стеке. Возвращает True, если элемент добавлен, и False — если не добавлен.
    \item Создайте метод pop, который выталкивает верхний элемент из стека и возвращает его. Если стек пуст, возвращает None.
    \item Создайте метод is\_empty, который возвращает True, если стек пуст, и False в противном случае.
    \item Создайте метод size, который возвращает текущее количество элементов в стеке.
    \item Создайте метод peek, который возвращает верхний элемент стека, если стек не пуст. Если стек пуст, возвращает None.
    \item Создайте экземпляр класса Stack с allow\_duplicates=False.
    \item Добавьте элементы 10, 20, 10 (второй 10 не добавится), 30.
    \item Выведите размер стека и верхний элемент.
    \item Вызовите pop, выведите результат.
    \item Повторите вывод размера и верхнего элемента.
\end{enumerate}

Пример использования:
\begin{lstlisting}[language=Python]
stack = Stack(allow_duplicates=False)
print(stack.push(10))  # True
print(stack.push(20))  # True
print(stack.push(10))  # False (дубликат)
print(stack.push(30))  # True

print("Размер стека:", stack.size())
print("Верхний элемент:", stack.peek())

popped = stack.pop()
print("Вытолкнут:", popped)
print("Размер после pop:", stack.size())
print("Верхний элемент:", stack.peek())
\end{lstlisting}

\item Написать программу на Python, которая создает класс Stack для представления стека с инкапсуляцией. Класс должен содержать методы push, pop, is\_empty, size и peek, которые реализуют операции вталкивания, выталкивания, проверки пустоты, получения размера и просмотра вершины стека соответственно. Программа также должна создавать экземпляр класса Stack, вталкивать в него элементы, выталкивать элементы и выводить информацию о стеке на экран.

Инструкции:
\begin{enumerate}
    \item Создайте класс Stack с методом \_\_init\_\_, который инициализирует пустой стек. Может принимать параметр name (строка) для именования стека (используется только для отладки, не влияет на логику).
    \item Создайте метод push, который принимает элемент и добавляет его в стек. Если элемент не является числом (int или float), он не добавляется, а в sys.stderr выводится сообщение "Только числовые значения разрешены".
    \item Создайте метод pop, который выталкивает верхний элемент из стека и возвращает его. Если стек пуст, возвращает None.
    \item Создайте метод is\_empty, который возвращает True, если стек пуст, и False в противном случае.
    \item Создайте метод size, который возвращает текущее количество элементов в стеке.
    \item Создайте метод peek, который возвращает верхний элемент стека, если стек не пуст. Если стек пуст, возвращает None.
    \item Создайте экземпляр класса Stack с именем "NumericStack".
    \item Добавьте элементы: 3.14, 42, "hello" (не добавится), 100, [1,2] (не добавится).
    \item Выведите размер стека и верхний элемент.
    \item Вызовите pop дважды, выводя каждый раз результат.
    \item После каждого pop выводите размер стека.
\end{enumerate}

Пример использования:
\begin{lstlisting}[language=Python]
import sys

stack = Stack(name="NumericStack")
stack.push(3.14)
stack.push(42)
stack.push("hello")   # не добавится
stack.push(100)
stack.push([1,2])     # не добавится

print("Размер стека:", stack.size())
print("Верхний элемент:", stack.peek())

popped = stack.pop()
print("Вытолкнут:", popped)
print("Размер после pop:", stack.size())

popped = stack.pop()
print("Вытолкнут:", popped)
print("Размер после pop:", stack.size())
\end{lstlisting}

\item Написать программу на Python, которая создает класс Stack для представления стека с инкапсуляцией. Класс должен содержать методы push, pop, is\_empty, size и peek, которые реализуют операции вталкивания, выталкивания, проверки пустоты, получения размера и просмотра вершины стека соответственно. Программа также должна создавать экземпляр класса Stack, вталкивать в него элементы, выталкивать элементы и выводить информацию о стеке на экран.

Инструкции:
\begin{enumerate}
    \item Создайте класс Stack с методом \_\_init\_\_, который инициализирует пустой стек. Принимает необязательный параметр auto\_reverse=False. Если True, то при добавлении элемента он вставляется не наверх, а вниз стека (реализуя поведение, обратное обычному стеку).
    \item Создайте метод push, который принимает элемент и добавляет его: если auto\_reverse=False — наверх (как обычно), если True — вниз (в начало внутреннего списка).
    \item Создайте метод pop, который выталкивает верхний элемент из стека (последний добавленный, если auto\_reverse=False, или первый добавленный, если auto\_reverse=True) и возвращает его. Если стек пуст, возвращает "EMPTY".
    \item Создайте метод is\_empty, который возвращает True, если стек пуст, и False в противном случае.
    \item Создайте метод size, который возвращает текущее количество элементов в стеке.
    \item Создайте метод peek, который возвращает верхний элемент стека (последний в списке, если auto\_reverse=False, или первый, если auto\_reverse=True), если стек не пуст. Если стек пуст, возвращает "NO ELEMENT".
    \item Создайте экземпляр класса Stack с auto\_reverse=True.
    \item Добавьте элементы: 1, 2, 3 (в стеке будет [3, 2, 1], где 3 — верх).
    \item Выведите размер стека и результат peek (должен быть 3).
    \item Вызовите pop, выведите результат (должен быть 3).
    \item Повторите вывод размера и peek (теперь верх — 2).
\end{enumerate}

Пример использования:
\begin{lstlisting}[language=Python]
stack = Stack(auto_reverse=True)
stack.push(1)
stack.push(2)
stack.push(3)  # стек: [3,2,1], верх - 3

print("Размер стека:", stack.size())
print("Верхний элемент:", stack.peek())

popped = stack.pop()
print("Вытолкнут:", popped)  # 3
print("Размер после pop:", stack.size())
print("Верхний элемент:", stack.peek())  # 2
\end{lstlisting}

\item Написать программу на Python, которая создает класс Stack для представления стека с инкапсуляцией. Класс должен содержать методы push, pop, is\_empty, size и peek, которые реализуют операции вталкивания, выталкивания, проверки пустоты, получения размера и просмотра вершины стека соответственно. Программа также должна создавать экземпляр класса Stack, вталкивать в него элементы, выталкивать элементы и выводить информацию о стеке на экран.

Инструкции:
\begin{enumerate}
    \item Создайте класс Stack с методом \_\_init\_\_, который инициализирует пустой стек. Принимает параметр case\_sensitive=True. Используется только если элементы — строки.
    \item Создайте метод push, который принимает элемент. Если элемент — строка и case\_sensitive=False, то перед добавлением преобразует её в нижний регистр. Добавляет элемент в стек.
    \item Создайте метод pop, который выталкивает верхний элемент из стека и возвращает его. Если стек пуст, возвращает пустую строку "".
    \item Создайте метод is\_empty, который возвращает True, если стек пуст, и False в противном случае.
    \item Создайте метод size, который возвращает текущее количество элементов в стеке.
    \item Создайте метод peek, который возвращает верхний элемент стека, если стек не пуст. Если стек пуст, возвращает пустую строку "".
    \item Создайте экземпляр класса Stack с case\_sensitive=False.
    \item Добавьте строки: "Hello", "WORLD", "Python".
    \item Выведите размер стека и верхний элемент (должен быть "python").
    \item Вызовите pop, выведите результат.
    \item Повторите вывод размера и верхнего элемента.
\end{enumerate}

Пример использования:
\begin{lstlisting}[language=Python]
stack = Stack(case_sensitive=False)
stack.push("Hello")
stack.push("WORLD")
stack.push("Python")

print("Размер стека:", stack.size())
print("Верхний элемент:", stack.peek())  # "python"

popped = stack.pop()
print("Вытолкнут:", popped)  # "python"
print("Размер после pop:", stack.size())
print("Верхний элемент:", stack.peek())  # "world"
\end{lstlisting}

\item Написать программу на Python, которая создает класс Stack для представления стека с инкапсуляцией. Класс должен содержать методы push, pop, is\_empty, size и peek, которые реализуют операции вталкивания, выталкивания, проверки пустоты, получения размера и просмотра вершины стека соответственно. Программа также должна создавать экземпляр класса Stack, вталкивать в него элементы, выталкивать элементы и выводить информацию о стеке на экран.

Инструкции:
\begin{enumerate}
    \item Создайте класс Stack с методом \_\_init\_\_, который инициализирует пустой стек. Принимает параметр min\_value=None. Если задан, то при добавлении элемента проверяется, что он >= min\_value.
    \item Создайте метод push, который принимает элемент. Если min\_value задан и элемент < min\_value, элемент не добавляется, а метод возвращает False. Иначе — добавляет и возвращает True.
    \item Создайте метод pop, который выталкивает верхний элемент из стека и возвращает его. Если стек пуст, возвращает None.
    \item Создайте метод is\_empty, который возвращает True, если стек пуст, и False в противном случае.
    \item Создайте метод size, который возвращает текущее количество элементов в стеке.
    \item Создайте метод peek, который возвращает верхний элемент стека, если стек не пуст. Если стек пуст, возвращает None.
    \item Создайте экземпляр класса Stack с min\_value=10.
    \item Добавьте элементы: 5 (не добавится), 15, 20, 8 (не добавится), 25.
    \item Выведите размер стека и верхний элемент.
    \item Вызовите pop, выведите результат.
    \item Повторите вывод размера и верхнего элемента.
\end{enumerate}

Пример использования:
\begin{lstlisting}[language=Python]
stack = Stack(min_value=10)
print(stack.push(5))   # False
print(stack.push(15))  # True
print(stack.push(20))  # True
print(stack.push(8))   # False
print(stack.push(25))  # True

print("Размер стека:", stack.size())
print("Верхний элемент:", stack.peek())

popped = stack.pop()
print("Вытолкнут:", popped)  # 25
print("Размер после pop:", stack.size())
print("Верхний элемент:", stack.peek())  # 20
\end{lstlisting}

\item Написать программу на Python, которая создает класс Stack для представления стека с инкапсуляцией. Класс должен содержать методы push, pop, is\_empty, size и peek, которые реализуют операции вталкивания, выталкивания, проверки пустоты, получения размера и просмотра вершины стека соответственно. Программа также должна создавать экземпляр класса Stack, вталкивать в него элементы, выталкивать элементы и выводить информацию о стеке на экран.

Инструкции:
\begin{enumerate}
    \item Создайте класс Stack с методом \_\_init\_\_, который инициализирует пустой стек. Принимает параметр max\_increments=0 — максимальное количество добавлений. Если 0 — без ограничений.
    \item Создайте метод push, который принимает элемент. Если max\_increments > 0 и количество вызовов push превысило max\_increments, элемент не добавляется, метод возвращает False. Иначе — добавляет и возвращает True.
    \item Создайте метод pop, который выталкивает верхний элемент из стека и возвращает его. Если стек пуст, возвращает строку "---".
    \item Создайте метод is\_empty, который возвращает True, если стек пуст, и False в противном случае.
    \item Создайте метод size, который возвращает текущее количество элементов в стеке.
    \item Создайте метод peek, который возвращает верхний элемент стека, если стек не пуст. Если стек пуст, возвращает строку "---".
    \item Создайте экземпляр класса Stack с max\_increments=3.
    \item Добавьте элементы: 100, 200, 300, 400 (последний не добавится).
    \item Выведите размер стека и верхний элемент.
    \item Вызовите pop, выведите результат.
    \item Повторите вывод размера и верхнего элемента.
\end{enumerate}

Пример использования:
\begin{lstlisting}[language=Python]
stack = Stack(max_increments=3)
print(stack.push(100))  # True
print(stack.push(200))  # True
print(stack.push(300))  # True
print(stack.push(400))  # False

print("Размер стека:", stack.size())
print("Верхний элемент:", stack.peek())

popped = stack.pop()
print("Вытолкнут:", popped)  # 300
print("Размер после pop:", stack.size())
print("Верхний элемент:", stack.peek())  # 200
\end{lstlisting}

\item Написать программу на Python, которая создает класс Stack для представления стека с инкапсуляцией. Класс должен содержать методы push, pop, is\_empty, size и peek, которые реализуют операции вталкивания, выталкивания, проверки пустоты, получения размера и просмотра вершины стека соответственно. Программа также должна создавать экземпляр класса Stack, вталкивать в него элементы, выталкивать элементы и выводить информацию о стеке на экран.

Инструкции:
\begin{enumerate}
    \item Создайте класс Stack с методом \_\_init\_\_, который инициализирует пустой стек. Принимает параметр validate\_type=None. Если задан (например, int), то при добавлении проверяется, что элемент является экземпляром этого типа.
    \item Создайте метод push, который принимает элемент. Если validate\_type задан и элемент не является его экземпляром, элемент не добавляется, метод возвращает False. Иначе — добавляет и возвращает True.
    \item Создайте метод pop, который выталкивает верхний элемент из стека и возвращает его. Если стек пуст, возвращает None.
    \item Создайте метод is\_empty, который возвращает True, если стек пуст, и False в противном случае.
    \item Создайте метод size, который возвращает текущее количество элементов в стеке.
    \item Создайте метод peek, который возвращает верхний элемент стека, если стек не пуст. Если стек пуст, возвращает None.
    \item Создайте экземпляр класса Stack с validate\_type=int.
    \item Добавьте элементы: 10, "20" (не добавится), 30, 40.5 (не добавится), 50.
    \item Выведите размер стека и верхний элемент.
    \item Вызовите pop, выведите результат.
    \item Повторите вывод размера и верхнего элемента.
\end{enumerate}

Пример использования:
\begin{lstlisting}[language=Python]
stack = Stack(validate_type=int)
print(stack.push(10))    # True
print(stack.push("20"))  # False
print(stack.push(30))    # True
print(stack.push(40.5))  # False
print(stack.push(50))    # True

print("Размер стека:", stack.size())
print("Верхний элемент:", stack.peek())

popped = stack.pop()
print("Вытолкнут:", popped)  # 50
print("Размер после pop:", stack.size())
print("Верхний элемент:", stack.peek())  # 30
\end{lstlisting}

\item Написать программу на Python, которая создает класс Stack для представления стека с инкапсуляцией. Класс должен содержать методы push, pop, is\_empty, size и peek, которые реализуют операции вталкивания, выталкивания, проверки пустоты, получения размера и просмотра вершины стека соответственно. Программа также должна создавать экземпляр класса Stack, вталкивать в него элементы, выталкивать элементы и выводить информацию о стеке на экран.

Инструкции:
\begin{enumerate}
    \item Создайте класс Stack с методом \_\_init\_\_, который инициализирует пустой стек. Принимает параметр unique\_per\_session=False. Если True, то не позволяет добавлять один и тот же элемент дважды за всё время жизни стека (даже если он был удален).
    \item Создайте метод push, который принимает элемент. Если unique\_per\_session=True и элемент уже когда-либо был добавлен (даже если потом удален), он не добавляется, метод возвращает False. Иначе — добавляет и возвращает True.
    \item Создайте метод pop, который выталкивает верхний элемент из стека и возвращает его. Если стек пуст, возвращает None.
    \item Создайте метод is\_empty, который возвращает True, если стек пуст, и False в противном случае.
    \item Создайте метод size, который возвращает текущее количество элементов в стеке.
    \item Создайте метод peek, который возвращает верхний элемент стека, если стек не пуст. Если стек пуст, возвращает None.
    \item Создайте экземпляр класса Stack с unique\_per\_session=True.
    \item Добавьте элементы: 7, 14, 7 (не добавится), 21, 14 (не добавится).
    \item Выведите размер стека и верхний элемент.
    \item Вызовите pop, выведите результат.
    \item Попробуйте добавить 21 снова (не должно добавиться).
    \item Выведите размер стека.
\end{enumerate}

Пример использования:
\begin{lstlisting}[language=Python]
stack = Stack(unique_per_session=True)
print(stack.push(7))   # True
print(stack.push(14))  # True
print(stack.push(7))   # False
print(stack.push(21))  # True
print(stack.push(14))  # False

print("Размер стека:", stack.size())
print("Верхний элемент:", stack.peek())

popped = stack.pop()
print("Вытолкнут:", popped)  # 21

print(stack.push(21))  # False (уже был)
print("Размер стека:", stack.size())  # по-прежнему 2
\end{lstlisting}

\item Написать программу на Python, которая создает класс Stack для представления стека с инкапсуляцией. Класс должен содержать методы push, pop, is\_empty, size и peek, которые реализуют операции вталкивания, выталкивания, проверки пустоты, получения размера и просмотра вершины стека соответственно. Программа также должна создавать экземпляр класса Stack, вталкивать в него элементы, выталкивать элементы и выводить информацию о стеке на экран.

Инструкции:
\begin{enumerate}
    \item Создайте класс Stack с методом \_\_init\_\_, который инициализирует пустой стек. Принимает параметр push\_limit\_per\_call=1 (по умолчанию). Если >1, то метод push может принимать несколько элементов (через *args) и добавлять их все за один вызов (но не более push\_limit\_per\_call элементов за вызов).
    \item Создайте метод push, который принимает один или несколько элементов (если push\_limit\_per\_call > 1). Если передано больше элементов, чем push\_limit\_per\_call, добавляются только первые push\_limit\_per\_call элементов, остальные игнорируются. Возвращает количество реально добавленных элементов.
    \item Создайте метод pop, который выталкивает верхний элемент из стека и возвращает его. Если стек пуст, возвращает None.
    \item Создайте метод is\_empty, который возвращает True, если стек пуст, и False в противном случае.
    \item Создайте метод size, который возвращает текущее количество элементов в стеке.
    \item Создайте метод peek, который возвращает верхний элемент стека, если стек не пуст. Если стек пуст, возвращает None.
    \item Создайте экземпляр класса Stack с push\_limit\_per\_call=3.
    \item Вызовите push с элементами 1, 2, 3, 4, 5 — добавятся только 1,2,3.
    \item Вызовите push с элементами 6, 7 — добавятся оба.
    \item Выведите размер стека и верхний элемент.
    \item Вызовите pop, выведите результат.
    \item Повторите вывод размера и верхнего элемента.
\end{enumerate}

Пример использования:
\begin{lstlisting}[language=Python]
stack = Stack(push_limit_per_call=3)
added = stack.push(1, 2, 3, 4, 5)  # добавит 1,2,3; вернет 3
print("Добавлено:", added)

added = stack.push(6, 7)  # добавит 6,7; вернет 2
print("Добавлено:", added)

print("Размер стека:", stack.size())
print("Верхний элемент:", stack.peek())

popped = stack.pop()
print("Вытолкнут:", popped)  # 7
print("Размер после pop:", stack.size())
print("Верхний элемент:", stack.peek())  # 6
\end{lstlisting}

\item Написать программу на Python, которая создает класс Stack для представления стека с инкапсуляцией. Класс должен содержать методы push, pop, is\_empty, size и peek, которые реализуют операции вталкивания, выталкивания, проверки пустоты, получения размера и просмотра вершины стека соответственно. Программа также должна создавать экземпляр класса Stack, вталкивать в него элементы, выталкивать элементы и выводить информацию о стеке на экран.

Инструкции:
\begin{enumerate}
    \item Создайте класс Stack с методом \_\_init\_\_, который инициализирует пустой стек. Принимает параметр pop\_multiple=False. Если True, то метод pop может принимать необязательный аргумент count (по умолчанию 1) и возвращать список из count верхних элементов.
    \item Создайте метод push, который принимает один элемент и добавляет его в стек. Возвращает None.
    \item Создайте метод pop, который, если pop\_multiple=False, выталкивает один верхний элемент и возвращает его. Если pop\_multiple=True, принимает count (по умолчанию 1) и возвращает список из count верхних элементов (если запрошено больше, чем есть, возвращает все). Если стек пуст, возвращает пустой список [] (в режиме pop\_multiple) или None (в обычном режиме).
    \item Создайте метод is\_empty, который возвращает True, если стек пуст, и False в противном случае.
    \item Создайте метод size, который возвращает текущее количество элементов в стеке.
    \item Создайте метод peek, который возвращает верхний элемент стека, если стек не пуст. Если стек пуст, возвращает None. Не поддерживает множественный просмотр.
    \item Создайте экземпляр класса Stack с pop\_multiple=True.
    \item Добавьте элементы: 10, 20, 30, 40, 50.
    \item Выведите размер стека и верхний элемент.
    \item Вызовите pop с count=3, выведите результат (должен быть [50,40,30]).
    \item Выведите размер стека и верхний элемент (теперь 20).
\end{enumerate}

Пример использования:
\begin{lstlisting}[language=Python]
stack = Stack(pop_multiple=True)
stack.push(10)
stack.push(20)
stack.push(30)
stack.push(40)
stack.push(50)

print("Размер стека:", stack.size())
print("Верхний элемент:", stack.peek())

popped = stack.pop(count=3)
print("Вытолкнуты:", popped)  # [50, 40, 30]

print("Размер после pop:", stack.size())
print("Верхний элемент:", stack.peek())  # 20
\end{lstlisting}

\item Написать программу на Python, которая создает класс Stack для представления стека с инкапсуляцией. Класс должен содержать методы push, pop, is\_empty, size и peek, которые реализуют операции вталкивания, выталкивания, проверки пустоты, получения размера и просмотра вершины стека соответственно. Программа также должна создавать экземпляр класса Stack, вталкивать в него элементы, выталкивать элементы и выводить информацию о стеке на экран.

Инструкции:
\begin{enumerate}
    \item Создайте класс Stack с методом \_\_init\_\_, который инициализирует пустой стек. Принимает параметр on\_push\_callback=None — функция, которая будет вызываться после каждого успешного добавления элемента (с аргументом — добавленным элементом).
    \item Создайте метод push, который принимает элемент и добавляет его в стек. Если on\_push\_callback не None, вызывает её с добавленным элементом. Возвращает добавленный элемент.
    \item Создайте метод pop, который выталкивает верхний элемент из стека и возвращает его. Если стек пуст, возвращает None.
    \item Создайте метод is\_empty, который возвращает True, если стек пуст, и False в противном случае.
    \item Создайте метод size, который возвращает текущее количество элементов в стеке.
    \item Создайте метод peek, который возвращает верхний элемент стека, если стек не пуст. Если стек пуст, возвращает None.
    \item Создайте функцию logger(x): print(f"[LOG] Добавлен: {x}")
    \item Создайте экземпляр класса Stack, передав logger в on\_push\_callback.
    \item Добавьте элементы: 101, 202, 303 (при каждом добавлении должно выводиться сообщение).
    \item Выведите размер стека и верхний элемент.
    \item Вызовите pop, выведите результат.
    \item Повторите вывод размера и верхнего элемента.
\end{enumerate}

Пример использования:
\begin{lstlisting}[language=Python]
def logger(x):
    print(f"[LOG] Добавлен: {x}")

stack = Stack(on_push_callback=logger)
stack.push(101)  # выведет [LOG] Добавлен: 101
stack.push(202)  # выведет [LOG] Добавлен: 202
stack.push(303)  # выведет [LOG] Добавлен: 303

print("Размер стека:", stack.size())
print("Верхний элемент:", stack.peek())

popped = stack.pop()
print("Вытолкнут:", popped)  # 303
print("Размер после pop:", stack.size())
print("Верхний элемент:", stack.peek())  # 202
\end{lstlisting}

\item Написать программу на Python, которая создает класс Stack для представления стека с инкапсуляцией. Класс должен содержать методы push, pop, is\_empty, size и peek, которые реализуют операции вталкивания, выталкивания, проверки пустоты, получения размера и просмотра вершины стека соответственно. Программа также должна создавать экземпляр класса Stack, вталкивать в него элементы, выталкивать элементы и выводить информацию о стеке на экран.

Инструкции:
\begin{enumerate}
    \item Создайте класс Stack с методом \_\_init\_\_, который инициализирует пустой стек. Принимает параметр compress\_on\_push=False. Если True, то при добавлении элемента, равного текущему верхнему, вместо добавления нового элемента увеличивается счетчик дубликатов у верхнего элемента (стек хранит пары (элемент, счетчик)).
    \item Создайте метод push, который принимает элемент. Если compress\_on\_push=True и элемент равен текущему верхнему, увеличивает счетчик верхнего элемента. Иначе — добавляет новый элемент (со счетчиком 1, если режим сжатия включен).
    \item Создайте метод pop, который выталкивает верхний элемент. Если режим сжатия включен и счетчик >1, уменьшает счетчик и возвращает элемент. Если счетчик=1, удаляет элемент. Если стек пуст, возвращает None.
    \item Создайте метод is\_empty, который возвращает True, если стек пуст, и False в противном случае.
    \item Создайте метод size, который возвращает общее количество элементов (с учетом счетчиков, если режим сжатия включен).
    \item Создайте метод peek, который возвращает верхний элемент (не счетчик, а само значение), если стек не пуст. Если стек пуст, возвращает None.
    \item Создайте экземпляр класса Stack с compress\_on\_push=True.
    \item Добавьте элементы: 5, 5, 5, 10, 10, 15.
    \item Выведите размер стека (должен быть 6) и верхний элемент (15).
    \item Вызовите pop, выведите результат (15).
    \item Вызовите pop, выведите результат (10) — счетчик у 10 должен уменьшиться с 2 до 1.
    \item Выведите размер стека (должен быть 4).
\end{enumerate}

Пример использования:
\begin{lstlisting}[language=Python]
stack = Stack(compress_on_push=True)
stack.push(5)
stack.push(5)
stack.push(5)
stack.push(10)
stack.push(10)
stack.push(15)

print("Размер стека:", stack.size())     # 6
print("Верхний элемент:", stack.peek())   # 15

popped = stack.pop()
print("Вытолкнут:", popped)  # 15

popped = stack.pop()
print("Вытолкнут:", popped)  # 10

print("Размер после двух pop:", stack.size())  # 4
\end{lstlisting}

\item Написать программу на Python, которая создает класс Stack для представления стека с инкапсуляцией. Класс должен содержать методы push, pop, is\_empty, size и peek, которые реализуют операции вталкивания, выталкивания, проверки пустоты, получения размера и просмотра вершины стека соответственно. Программа также должна создавать экземпляр класса Stack, вталкивать в него элементы, выталкивать элементы и выводить информацию о стеке на экран.

Инструкции:
\begin{enumerate}
    \item Создайте класс Stack с методом \_\_init\_\_, который инициализирует пустой стек. Принимает параметр immutable\_pop=False. Если True, то метод pop не удаляет элемент из стека, а только возвращает его (поведение как peek, но называется pop).
    \item Создайте метод push, который принимает элемент и добавляет его в стек.
    \item Создайте метод pop, который, если immutable\_pop=False, выталкивает верхний элемент и возвращает его. Если immutable\_pop=True, возвращает верхний элемент, не удаляя его. Если стек пуст, возвращает None.
    \item Создайте метод is\_empty, который возвращает True, если стек пуст, и False в противном случае.
    \item Создайте метод size, который возвращает текущее количество элементов в стеке.
    \item Создайте метод peek, который возвращает верхний элемент стека, если стек не пуст. Если стек пуст, возвращает None. (Поведение не зависит от immutable\_pop.)
    \item Создайте экземпляр класса Stack с immutable\_pop=True.
    \item Добавьте элементы: 1, 3, 5, 7.
    \item Выведите размер стека и результат pop (должен быть 7, но стек не изменится).
    \item Снова вызовите pop, снова выведите результат (опять 7).
    \item Выведите размер стека (по-прежнему 4).
\end{enumerate}

Пример использования:
\begin{lstlisting}[language=Python]
stack = Stack(immutable_pop=True)
stack.push(1)
stack.push(3)
stack.push(5)
stack.push(7)

print("Размер стека:", stack.size())
print("Первый pop:", stack.pop())  # 7
print("Второй pop:", stack.pop())  # 7 (стек не изменился)
print("Размер стека:", stack.size())  # 4
\end{lstlisting}

\item Написать программу на Python, которая создает класс Stack для представления стека с инкапсуляцией. Класс должен содержать методы push, pop, is\_empty, size и peek, которые реализуют операции вталкивания, выталкивания, проверки пустоты, получения размера и просмотра вершины стека соответственно. Программа также должна создавать экземпляр класса Stack, вталкивать в него элементы, выталкивать элементы и выводить информацию о стеке на экран.

Инструкции:
\begin{enumerate}
    \item Создайте класс Stack с методом \_\_init\_\_, который инициализирует пустой стек. Принимает параметр track\_history=False. Если True, то сохраняет историю всех когда-либо находившихся в стеке элементов (даже удаленных) в отдельном списке.
    \item Создайте метод push, который принимает элемент, добавляет его в стек, и если track\_history=True, добавляет его и в историю.
    \item Создайте метод pop, который выталкивает верхний элемент из стека и возвращает его. Если стек пуст, возвращает None.
    \item Создайте метод is\_empty, который возвращает True, если стек пуст, и False в противном случае.
    \item Создайте метод size, который возвращает текущее количество элементов в стеке.
    \item Создайте метод peek, который возвращает верхний элемент стека, если стек не пуст. Если стек пуст, возвращает None.
    \item Создайте метод get\_history (только если track\_history=True), который возвращает копию списка истории.
    \item Создайте экземпляр класса Stack с track\_history=True.
    \item Добавьте элементы: 2, 4, 6.
    \item Вызовите pop (вернет 6).
    \item Добавьте 8.
    \item Выведите текущий стек (через peek и size) и историю (должна быть [2,4,6,8]).
\end{enumerate}

Пример использования:
\begin{lstlisting}[language=Python]
stack = Stack(track_history=True)
stack.push(2)
stack.push(4)
stack.push(6)
stack.pop()  # 6
stack.push(8)

print("Текущий размер:", stack.size())      # 2
print("Верхний элемент:", stack.peek())     # 8
print("История:", stack.get_history())      # [2,4,6,8]
\end{lstlisting}

\item Написать программу на Python, которая создает класс Stack для представления стека с инкапсуляцией. Класс должен содержать методы push, pop, is\_empty, size и peek, которые реализуют операции вталкивания, выталкивания, проверки пустоты, получения размера и просмотра вершины стека соответственно. Программа также должна создавать экземпляр класса Stack, вталкивать в него элементы, выталкивать элементы и выводить информацию о стеке на экран.

Инструкции:
\begin{enumerate}
    \item Создайте класс Stack с методом \_\_init\_\_, который инициализирует пустой стек. Принимает параметр push\_only\_even=False. Если True, то добавляются только четные числа (остальные игнорируются).
    \item Создайте метод push, который принимает элемент. Если push\_only\_even=True и элемент не является четным целым числом, он не добавляется. Иначе — добавляется.
    \item Создайте метод pop, который выталкивает верхний элемент из стека и возвращает его. Если стек пуст, возвращает None.
    \item Создайте метод is\_empty, который возвращает True, если стек пуст, и False в противном случае.
    \item Создайте метод size, который возвращает текущее количество элементов в стеке.
    \item Создайте метод peek, который возвращает верхний элемент стека, если стек не пуст. Если стек пуст, возвращает None.
    \item Создайте экземпляр класса Stack с push\_only\_even=True.
    \item Добавьте элементы: 1 (игнорируется), 2, 3 (игнорируется), 4, 5 (игнорируется), 6.
    \item Выведите размер стека (должен быть 3) и верхний элемент (6).
    \item Вызовите pop, выведите результат (6).
    \item Повторите вывод размера и верхнего элемента (теперь 4).
\end{enumerate}

Пример использования:
\begin{lstlisting}[language=Python]
stack = Stack(push_only_even=True)
stack.push(1)  # игнорируется
stack.push(2)
stack.push(3)  # игнорируется
stack.push(4)
stack.push(5)  # игнорируется
stack.push(6)

print("Размер стека:", stack.size())     # 3
print("Верхний элемент:", stack.peek())   # 6

popped = stack.pop()
print("Вытолкнут:", popped)  # 6

print("Размер после pop:", stack.size())    # 2
print("Верхний элемент:", stack.peek())     # 4
\end{lstlisting}

\item Написать программу на Python, которая создает класс Stack для представления стека с инкапсуляцией. Класс должен содержать методы push, pop, is\_empty, size и peek, которые реализуют операции вталкивания, выталкивания, проверки пустоты, получения размера и просмотра вершины стека соответственно. Программа также должна создавать экземпляр класса Stack, вталкивать в него элементы, выталкивать элементы и выводить информацию о стеке на экран.

Инструкции:
\begin{enumerate}
    \item Создайте класс Stack с методом \_\_init\_\_, который инициализирует пустой стек. Принимает параметр reverse\_pop=False. Если True, то метод pop возвращает не верхний, а нижний элемент стека (и удаляет его).
    \item Создайте метод push, который принимает элемент и добавляет его в стек (наверх).
    \item Создайте метод pop, который, если reverse\_pop=False, выталкивает верхний элемент и возвращает его. Если reverse\_pop=True, выталкивает нижний элемент и возвращает его. Если стек пуст, возвращает None.
    \item Создайте метод is\_empty, который возвращает True, если стек пуст, и False в противном случае.
    \item Создайте метод size, который возвращает текущее количество элементов в стеке.
    \item Создайте метод peek, который возвращает верхний элемент стека, если стек не пуст. Если стек пуст, возвращает None. (Не зависит от reverse\_pop.)
    \item Создайте экземпляр класса Stack с reverse\_pop=True.
    \item Добавьте элементы: 10, 20, 30 (в стеке: [10,20,30], верх — 30).
    \item Выведите результат peek (должен быть 30).
    \item Вызовите pop — должен вернуться 10 (нижний), стек станет [20,30].
    \item Выведите размер и снова peek (должен быть 30).
\end{enumerate}

Пример использования:
\begin{lstlisting}[language=Python]
stack = Stack(reverse_pop=True)
stack.push(10)
stack.push(20)
stack.push(30)

print("Верхний элемент (peek):", stack.peek())  # 30
popped = stack.pop()
print("Вытолкнут (нижний):", popped)            # 10
print("Размер после pop:", stack.size())        # 2
print("Верхний элемент (peek):", stack.peek())  # 30
\end{lstlisting}

\item Написать программу на Python, которая создает класс Stack для представления стека с инкапсуляцией. Класс должен содержать методы push, pop, is\_empty, size и peek, которые реализуют операции вталкивания, выталкивания, проверки пустоты, получения размера и просмотра вершины стека соответственно. Программа также должна создавать экземпляр класса Stack, вталкивать в него элементы, выталкивать элементы и выводить информацию о стеке на экран.

Инструкции:
\begin{enumerate}
    \item Создайте класс Stack с методом \_\_init\_\_, который инициализирует пустой стек. Принимает параметр push\_with\_timestamp=False. Если True, то при добавлении элемент сохраняется вместе с текущим временем (в формате Unix timestamp).
    \item Создайте метод push, который принимает элемент. Если push\_with\_timestamp=True, сохраняет пару (элемент, time.time()). Иначе — только элемент.
    \item Создайте метод pop, который выталкивает верхний элемент. Если режим с временем включен, возвращает пару (элемент, timestamp). Иначе — только элемент. Если стек пуст, возвращает None.
    \item Создайте метод is\_empty, который возвращает True, если стек пуст, и False в противном случае.
    \item Создайте метод size, который возвращает текущее количество элементов в стеке.
    \item Создайте метод peek, который возвращает верхний элемент (или пару, если включен режим времени), если стек не пуст. Если стек пуст, возвращает None.
    \item Создайте экземпляр класса Stack с push\_with\_timestamp=True.
    \item Добавьте элементы: "first", "second", "third".
    \item Выведите размер стека и результат peek (должна быть пара ("third", timestamp)).
    \item Вызовите pop, выведите результат (тоже пара).
    \item Повторите вывод размера и peek.
\end{enumerate}

Пример использования:
\begin{lstlisting}[language=Python]
import time

stack = Stack(push_with_timestamp=True)
stack.push("first")
stack.push("second")
stack.push("third")

print("Размер стека:", stack.size())
peek_result = stack.peek()
print("Верхний элемент и время:", peek_result)  # ('third', 1712345678.123456)

popped = stack.pop()
print("Вытолкнут:", popped)  # ('third', 1712345678.123456)

print("Размер после pop:", stack.size())
print("Верхний элемент:", stack.peek())  # ('second', ...)
\end{lstlisting}

\item Написать программу на Python, которая создает класс Stack для представления стека с инкапсуляцией. Класс должен содержать методы push, pop, is\_empty, size и peek, которые реализуют операции вталкивания, выталкивания, проверки пустоты, получения размера и просмотра вершины стека соответственно. Программа также должна создавать экземпляр класса Stack, вталкивать в него элементы, выталкивать элементы и выводить информацию о стеке на экран.

Инструкции:
\begin{enumerate}
    \item Создайте класс Stack с методом \_\_init\_\_, который инициализирует пустой стек. Принимает параметр push\_pairs=False. Если True, то метод push ожидает два аргумента (key, value) и сохраняет их как кортеж. Если False — один аргумент.
    \item Создайте метод push, который, если push\_pairs=False, принимает один элемент. Если push\_pairs=True, принимает два аргумента (key, value) и сохраняет (key, value). Возвращает сохраненный элемент (или кортеж).
    \item Создайте метод pop, который выталкивает верхний элемент (или кортеж) и возвращает его. Если стек пуст, возвращает None.
    \item Создайте метод is\_empty, который возвращает True, если стек пуст, и False в противном случае.
    \item Создайте метод size, который возвращает текущее количество элементов в стеке.
    \item Создайте метод peek, который возвращает верхний элемент (или кортеж), если стек не пуст. Если стек пуст, возвращает None.
    \item Создайте экземпляр класса Stack с push\_pairs=True.
    \item Добавьте пары: ("a", 1), ("b", 2), ("c", 3).
    \item Выведите размер стека и результат peek (должен быть ("c",3)).
    \item Вызовите pop, выведите результат.
    \item Повторите вывод размера и peek.
\end{enumerate}

Пример использования:
\begin{lstlisting}[language=Python]
stack = Stack(push_pairs=True)
stack.push("a", 1)
stack.push("b", 2)
stack.push("c", 3)

print("Размер стека:", stack.size())
print("Верхний элемент:", stack.peek())  # ('c', 3)

popped = stack.pop()
print("Вытолкнут:", popped)  # ('c', 3)

print("Размер после pop:", stack.size())
print("Верхний элемент:", stack.peek())  # ('b', 2)
\end{lstlisting}

\item Написать программу на Python, которая создает класс Stack для представления стека с инкапсуляцией. Класс должен содержать методы push, pop, is\_empty, size и peek, которые реализуют операции вталкивания, выталкивания, проверки пустоты, получения размера и просмотра вершины стека соответственно. Программа также должна создавать экземпляр класса Stack, вталкивать в него элементы, выталкивать элементы и выводить информацию о стеке на экран.

Инструкции:
\begin{enumerate}
    \item Создайте класс Stack с методом \_\_init\_\_, который инициализирует пустой стек. Принимает параметр auto\_dedup=False. Если True, то при добавлении элемента, который уже есть в стеке (не обязательно на вершине), сначала удаляет все его предыдущие вхождения.
    \item Создайте метод push, который принимает элемент. Если auto\_dedup=True и такой элемент уже есть в стеке, удаляет все его вхождения, затем добавляет новый элемент. Иначе — просто добавляет.
    \item Создайте метод pop, который выталкивает верхний элемент из стека и возвращает его. Если стек пуст, возвращает None.
    \item Создайте метод is\_empty, который возвращает True, если стек пуст, и False в противном случае.
    \item Создайте метод size, который возвращает текущее количество элементов в стеке.
    \item Создайте метод peek, который возвращает верхний элемент стека, если стек не пуст. Если стек пуст, возвращает None.
    \item Создайте экземпляр класса Stack с auto\_dedup=True.
    \item Добавьте элементы: 1, 2, 1, 3, 2, 4.
    \item После каждого добавления выводите содержимое стека (реализуйте вспомогательный метод \_debug\_list, возвращающий список элементов снизу вверх — только для отладки, не включайте в задание студентам; в решении можно использовать stack.\_items, если инкапсуляция не строгая).
    \item Выведите итоговый размер и верхний элемент.
\end{enumerate}

Пример использования (с отладочным выводом для ясности):
\begin{lstlisting}[language=Python]
# (В решении студент не обязан реализовывать _debug_list, но для проверки можно временно добавить)
stack = Stack(auto_dedup=True)
stack.push(1)  # стек: [1]
stack.push(2)  # стек: [1,2]
stack.push(1)  # удаляет старую 1, добавляет новую -> [2,1]
stack.push(3)  # [2,1,3]
stack.push(2)  # удаляет 2, добавляет новую -> [1,3,2]
stack.push(4)  # [1,3,2,4]

print("Размер стека:", stack.size())     # 4
print("Верхний элемент:", stack.peek())   # 4
\end{lstlisting}

\item Написать программу на Python, которая создает класс Stack для представления стека с инкапсуляцией. Класс должен содержать методы push, pop, is\_empty, size и peek, которые реализуют операции вталкивания, выталкивания, проверки пустоты, получения размера и просмотра вершины стека соответственно. Программа также должна создавать экземпляр класса Stack, вталкивать в него элементы, выталкивать элементы и выводить информацию о стеке на экран.

Инструкции:
\begin{enumerate}
    \item Создайте класс Stack с методом \_\_init\_\_, который инициализирует пустой стек. Принимает параметр push\_if\_max=False. Если True, то элемент добавляется только если он больше всех текущих элементов в стеке.
    \item Создайте метод push, который принимает элемент. Если push\_if\_max=True и элемент не является строго больше всех элементов в стеке, он не добавляется. Иначе — добавляется.
    \item Создайте метод pop, который выталкивает верхний элемент из стека и возвращает его. Если стек пуст, возвращает None.
    \item Создайте метод is\_empty, который возвращает True, если стек пуст, и False в противном случае.
    \item Создайте метод size, который возвращает текущее количество элементов в стеке.
    \item Создайте метод peek, который возвращает верхний элемент стека, если стек не пуст. Если стек пуст, возвращает None.
    \item Создайте экземпляр класса Stack с push\_if\_max=True.
    \item Добавьте элементы: 5, 3 (не добавится, т.к. 3<5), 10, 7 (не добавится, т.к. 7<10), 15.
    \item Выведите размер стека (должен быть 3) и верхний элемент (15).
    \item Вызовите pop, выведите результат (15).
    \item Повторите вывод размера и верхнего элемента (теперь 10).
\end{enumerate}

Пример использования:
\begin{lstlisting}[language=Python]
stack = Stack(push_if_max=True)
stack.push(5)
stack.push(3)   # не добавится
stack.push(10)
stack.push(7)   # не добавится
stack.push(15)

print("Размер стека:", stack.size())     # 3
print("Верхний элемент:", stack.peek())   # 15

popped = stack.pop()
print("Вытолкнут:", popped)  # 15

print("Размер после pop:", stack.size())    # 2
print("Верхний элемент:", stack.peek())     # 10
\end{lstlisting}

\item Написать программу на Python, которая создает класс Stack для представления стека с инкапсуляцией. Класс должен содержать методы push, pop, is\_empty, size и peek, которые реализуют операции вталкивания, выталкивания, проверки пустоты, получения размера и просмотра вершины стека соответственно. Программа также должна создавать экземпляр класса Stack, вталкивать в него элементы, выталкивать элементы и выводить информацию о стеке на экран.

Инструкции:
\begin{enumerate}
    \item Создайте класс Stack с методом \_\_init\_\_, который инициализирует пустой стек. Принимает параметр cumulative=False. Если True, то при добавлении элемента он суммируется с предыдущим верхним элементом (первый элемент добавляется как есть).
    \item Создайте метод push, который принимает элемент. Если cumulative=True и стек не пуст, то добавляемый элемент становится element + текущий\_верх. Затем этот результат добавляется в стек. Если стек пуст, добавляется element как есть.
    \item Создайте метод pop, который выталкивает верхний элемент из стека и возвращает его. Если стек пуст, возвращает None.
    \item Создайте метод is\_empty, который возвращает True, если стек пуст, и False в противном случае.
    \item Создайте метод size, который возвращает текущее количество элементов в стеке.
    \item Создайте метод peek, который возвращает верхний элемент стека, если стек не пуст. Если стек пуст, возвращает None.
    \item Создайте экземпляр класса Stack с cumulative=True.
    \item Добавьте элементы: 1, 2, 3, 4.
    \item Выведите содержимое стека после каждого добавления (для проверки: после 1 → [1]; после 2 → [1,3]; после 3 → [1,3,6]; после 4 → [1,3,6,10]).
    \item Выведите итоговый размер и верхний элемент (10).
    \item Вызовите pop, выведите результат (10).
    \item Повторите вывод размера и верхнего элемента (теперь 6).
\end{enumerate}

Пример использования:
\begin{lstlisting}[language=Python]
stack = Stack(cumulative=True)
stack.push(1)  # [1]
stack.push(2)  # [1, 1+2=3]
stack.push(3)  # [1,3, 3+3=6]
stack.push(4)  # [1,3,6, 6+4=10]

print("Размер стека:", stack.size())     # 4
print("Верхний элемент:", stack.peek())   # 10

popped = stack.pop()
print("Вытолкнут:", popped)  # 10

print("Размер после pop:", stack.size())    # 3
print("Верхний элемент:", stack.peek())     # 6
\end{lstlisting}

\item Написать программу на Python, которая создает класс Stack для представления стека с инкапсуляцией. Класс должен содержать методы push, pop, is\_empty, size и peek, которые реализуют операции вталкивания, выталкивания, проверки пустоты, получения размера и просмотра вершины стека соответственно. Программа также должна создавать экземпляр класса Stack, вталкивать в него элементы, выталкивать элементы и выводить информацию о стеке на экран.

Инструкции:
\begin{enumerate}
    \item Создайте класс Stack с методом \_\_init\_\_, который инициализирует пустой стек. Принимает параметр push\_squared=False. Если True, то при добавлении элемент возводится в квадрат перед добавлением.
    \item Создайте метод push, который принимает элемент. Если push\_squared=True, добавляет element**2. Иначе — element.
    \item Создайте метод pop, который выталкивает верхний элемент из стека и возвращает его. Если стек пуст, возвращает None.
    \item Создайте метод is\_empty, который возвращает True, если стек пуст, и False в противном случае.
    \item Создайте метод size, который возвращает текущее количество элементов в стеке.
    \item Создайте метод peek, который возвращает верхний элемент стека, если стек не пуст. Если стек пуст, возвращает None.
    \item Создайте экземпляр класса Stack с push\_squared=True.
    \item Добавьте элементы: 2, 3, 4, 5.
    \item Выведите размер стека и верхний элемент (должен быть 25).
    \item Вызовите pop, выведите результат (25).
    \item Повторите вывод размера и верхнего элемента (теперь 16).
\end{enumerate}

Пример использования:
\begin{lstlisting}[language=Python]
stack = Stack(push_squared=True)
stack.push(2)  # добавит 4
stack.push(3)  # добавит 9
stack.push(4)  # добавит 16
stack.push(5)  # добавит 25

print("Размер стека:", stack.size())     # 4
print("Верхний элемент:", stack.peek())   # 25

popped = stack.pop()
print("Вытолкнут:", popped)  # 25

print("Размер после pop:", stack.size())    # 3
print("Верхний элемент:", stack.peek())     # 16
\end{lstlisting}

\item Написать программу на Python, которая создает класс Stack для представления стека с инкапсуляцией. Класс должен содержать методы push, pop, is\_empty, size и peek, которые реализуют операции вталкивания, выталкивания, проверки пустоты, получения размера и просмотра вершины стека соответственно. Программа также должна создавать экземпляр класса Stack, вталкивать в него элементы, выталкивать элементы и выводить информацию о стеке на экран.

Инструкции:
\begin{enumerate}
    \item Создайте класс Stack с методом \_\_init\_\_, который инициализирует пустой стек. Принимает параметр push\_absolute=False. Если True, то при добавлении сохраняется абсолютное значение элемента (abs(element)).
    \item Создайте метод push, который принимает элемент. Если push\_absolute=True, добавляет abs(element). Иначе — element.
    \item Создайте метод pop, который выталкивает верхний элемент из стека и возвращает его. Если стек пуст, возвращает None.
    \item Создайте метод is\_empty, который возвращает True, если стек пуст, и False в противном случае.
    \item Создайте метод size, который возвращает текущее количество элементов в стеке.
    \item Создайте метод peek, который возвращает верхний элемент стека, если стек не пуст. Если стек пуст, возвращает None.
    \item Создайте экземпляр класса Stack с push\_absolute=True.
    \item Добавьте элементы: -5, 3, -8, 2.
    \item Выведите размер стека и верхний элемент (должен быть 2).
    \item Вызовите pop, выведите результат (2).
    \item Повторите вывод размера и верхнего элемента (теперь 8).
\end{enumerate}

Пример использования:
\begin{lstlisting}[language=Python]
stack = Stack(push_absolute=True)
stack.push(-5)  # добавит 5
stack.push(3)   # добавит 3
stack.push(-8)  # добавит 8
stack.push(2)   # добавит 2

print("Размер стека:", stack.size())     # 4
print("Верхний элемент:", stack.peek())   # 2

popped = stack.pop()
print("Вытолкнут:", popped)  # 2

print("Размер после pop:", stack.size())    # 3
print("Верхний элемент:", stack.peek())     # 8
\end{lstlisting}

\item Написать программу на Python, которая создает класс Stack для представления стека с инкапсуляцией. Класс должен содержать методы push, pop, is\_empty, size и peek, которые реализуют операции вталкивания, выталкивания, проверки пустоты, получения размера и просмотра вершины стека соответственно. Программа также должна создавать экземпляр класса Stack, вталкивать в него элементы, выталкивать элементы и выводить информацию о стеке на экран.

Инструкции:
\begin{enumerate}
    \item Создайте класс Stack с методом \_\_init\_\_, который инициализирует пустой стек. Принимает параметр push\_rounded=False. Если True, то при добавлении элемент округляется до целого числа (round(element)).
    \item Создайте метод push, который принимает элемент. Если push\_rounded=True, добавляет round(element). Иначе — element.
    \item Создайте метод pop, который выталкивает верхний элемент из стека и возвращает его. Если стек пуст, возвращает None.
    \item Создайте метод is\_empty, который возвращает True, если стек пуст, и False в противном случае.
    \item Создайте метод size, который возвращает текущее количество элементов в стеке.
    \item Создайте метод peek, который возвращает верхний элемент стека, если стек не пуст. Если стек пуст, возвращает None.
    \item Создайте экземпляр класса Stack с push\_rounded=True.
    \item Добавьте элементы: 3.2, 4.7, 5.1, 6.9.
    \item Выведите размер стека и верхний элемент (должен быть 7).
    \item Вызовите pop, выведите результат (7).
    \item Повторите вывод размера и верхнего элемента (теперь 5).
\end{enumerate}

Пример использования:
\begin{lstlisting}[language=Python]
stack = Stack(push_rounded=True)
stack.push(3.2)  # 3
stack.push(4.7)  # 5
stack.push(5.1)  # 5
stack.push(6.9)  # 7

print("Размер стека:", stack.size())     # 4
print("Верхний элемент:", stack.peek())   # 7

popped = stack.pop()
print("Вытолкнут:", popped)  # 7

print("Размер после pop:", stack.size())    # 3
print("Верхний элемент:", stack.peek())     # 5
\end{lstlisting}

\item Написать программу на Python, которая создает класс Stack для представления стека с инкапсуляцией. Класс должен содержать методы push, pop, is\_empty, size и peek, которые реализуют операции вталкивания, выталкивания, проверки пустоты, получения размера и просмотра вершины стека соответственно. Программа также должна создавать экземпляр класса Stack, вталкивать в него элементы, выталкивать элементы и выводить информацию о стеке на экран.

Инструкции:
\begin{enumerate}
    \item Создайте класс Stack с методом \_\_init\_\_, который инициализирует пустой стек. Принимает параметр push\_negated=False. Если True, то при добавлении элемент сохраняется с обратным знаком (-element).
    \item Создайте метод push, который принимает элемент. Если push\_negated=True, добавляет -element. Иначе — element.
    \item Создайте метод pop, который выталкивает верхний элемент из стека и возвращает его. Если стек пуст, возвращает None.
    \item Создайте метод is\_empty, который возвращает True, если стек пуст, и False в противном случае.
    \item Создайте метод size, который возвращает текущее количество элементов в стеке.
    \item Создайте метод peek, который возвращает верхний элемент стека, если стек не пуст. Если стек пуст, возвращает None.
    \item Создайте экземпляр класса Stack с push\_negated=True.
    \item Добавьте элементы: 10, 20, 30, 40.
    \item Выведите размер стека и верхний элемент (должен быть -40).
    \item Вызовите pop, выведите результат (-40).
    \item Повторите вывод размера и верхнего элемента (теперь -30).
\end{enumerate}

Пример использования:
\begin{lstlisting}[language=Python]
stack = Stack(push_negated=True)
stack.push(10)  # -10
stack.push(20)  # -20
stack.push(30)  # -30
stack.push(40)  # -40

print("Размер стека:", stack.size())     # 4
print("Верхний элемент:", stack.peek())   # -40

popped = stack.pop()
print("Вытолкнут:", popped)  # -40

print("Размер после pop:", stack.size())    # 3
print("Верхний элемент:", stack.peek())     # -30
\end{lstlisting}

\item Написать программу на Python, которая создает класс Stack для представления стека с инкапсуляцией. Класс должен содержать методы push, pop, is\_empty, size и peek, которые реализуют операции вталкивания, выталкивания, проверки пустоты, получения размера и просмотра вершины стека соответственно. Программа также должна создавать экземпляр класса Stack, вталкивать в него элементы, выталкивать элементы и выводить информацию о стеке на экран.

Инструкции:
\begin{enumerate}
    \item Создайте класс Stack с методом \_\_init\_\_, который инициализирует пустой стек. Принимает параметр push\_doubled=False. Если True, то при добавлении элемент умножается на 2.
    \item Создайте метод push, который принимает элемент. Если push\_doubled=True, добавляет element * 2. Иначе — element.
    \item Создайте метод pop, который выталкивает верхний элемент из стека и возвращает его. Если стек пуст, возвращает None.
    \item Создайте метод is\_empty, который возвращает True, если стек пуст, и False в противном случае.
    \item Создайте метод size, который возвращает текущее количество элементов в стеке.
    \item Создайте метод peek, который возвращает верхний элемент стека, если стек не пуст. Если стек пуст, возвращает None.
    \item Создайте экземпляр класса Stack с push\_doubled=True.
    \item Добавьте элементы: 1, 2, 3, 4.
    \item Выведите размер стека и верхний элемент (должен быть 8).
    \item Вызовите pop, выведите результат (8).
    \item Повторите вывод размера и верхнего элемента (теперь 6).
\end{enumerate}

Пример использования:
\begin{lstlisting}[language=Python]
stack = Stack(push_doubled=True)
stack.push(1)  # 2
stack.push(2)  # 4
stack.push(3)  # 6
stack.push(4)  # 8

print("Размер стека:", stack.size())     # 4
print("Верхний элемент:", stack.peek())   # 8

popped = stack.pop()
print("Вытолкнут:", popped)  # 8

print("Размер после pop:", stack.size())    # 3
print("Верхний элемент:", stack.peek())     # 6
\end{lstlisting}

\item Написать программу на Python, которая создает класс Stack для представления стека с инкапсуляцией. Класс должен содержать методы push, pop, is\_empty, size и peek, которые реализуют операции вталкивания, выталкивания, проверки пустоты, получения размера и просмотра вершины стека соответственно. Программа также должна создавать экземпляр класса Stack, вталкивать в него элементы, выталкивать элементы и выводить информацию о стеке на экран.

Инструкции:
\begin{enumerate}
    \item Создайте класс Stack с методом \_\_init\_\_, который инициализирует пустой стек. Принимает параметр push\_halved=False. Если True, то при добавлении элемент делится на 2.0.
    \item Создайте метод push, который принимает элемент. Если push\_halved=True, добавляет element / 2.0. Иначе — element.
    \item Создайте метод pop, который выталкивает верхний элемент из стека и возвращает его. Если стек пуст, возвращает None.
    \item Создайте метод is\_empty, который возвращает True, если стек пуст, и False в противном случае.
    \item Создайте метод size, который возвращает текущее количество элементов в стеке.
    \item Создайте метод peek, который возвращает верхний элемент стека, если стек не пуст. Если стек пуст, возвращает None.
    \item Создайте экземпляр класса Stack с push\_halved=True.
    \item Добавьте элементы: 4, 8, 12, 16.
    \item Выведите размер стека и верхний элемент (должен быть 8.0).
    \item Вызовите pop, выведите результат (8.0).
    \item Повторите вывод размера и верхнего элемента (теперь 6.0).
\end{enumerate}

Пример использования:
\begin{lstlisting}[language=Python]
stack = Stack(push_halved=True)
stack.push(4)   # 2.0
stack.push(8)   # 4.0
stack.push(12)  # 6.0
stack.push(16)  # 8.0

print("Размер стека:", stack.size())     # 4
print("Верхний элемент:", stack.peek())   # 8.0

popped = stack.pop()
print("Вытолкнут:", popped)  # 8.0

print("Размер после pop:", stack.size())    # 3
print("Верхний элемент:", stack.peek())     # 6.0
\end{lstlisting}

\item Написать программу на Python, которая создает класс Stack для представления стека с инкапсуляцией. Класс должен содержать методы push, pop, is\_empty, size и peek, которые реализуют операции вталкивания, выталкивания, проверки пустоты, получения размера и просмотра вершины стека соответственно. Программа также должна создавать экземпляр класса Stack, вталкивать в него элементы, выталкивать элементы и выводить информацию о стеке на экран.

Инструкции:
\begin{enumerate}
    \item Создайте класс Stack с методом \_\_init\_\_, который инициализирует пустой стек. Принимает параметр push\_as\_string=False. Если True, то при добавлении элемент преобразуется в строку str(element).
    \item Создайте метод push, который принимает элемент. Если push\_as\_string=True, добавляет str(element). Иначе — element.
    \item Создайте метод pop, который выталкивает верхний элемент из стека и возвращает его. Если стек пуст, возвращает None.
    \item Создайте метод is\_empty, который возвращает True, если стек пуст, и False в противном случае.
    \item Создайте метод size, который возвращает текущее количество элементов в стеке.
    \item Создайте метод peek, который возвращает верхний элемент стека, если стек не пуст. Если стек пуст, возвращает None.
    \item Создайте экземпляр класса Stack с push\_as\_string=True.
    \item Добавьте элементы: 100, 200, 300, 400.
    \item Выведите размер стека и верхний элемент (должен быть "400").
    \item Вызовите pop, выведите результат ("400").
    \item Повторите вывод размера и верхнего элемента (теперь "300").
\end{enumerate}

Пример использования:
\begin{lstlisting}[language=Python]
stack = Stack(push_as_string=True)
stack.push(100)  # "100"
stack.push(200)  # "200"
stack.push(300)  # "300"
stack.push(400)  # "400"

print("Размер стека:", stack.size())     # 4
print("Верхний элемент:", stack.peek())   # "400"

popped = stack.pop()
print("Вытолкнут:", popped)  # "400"

print("Размер после pop:", stack.size())    # 3
print("Верхний элемент:", stack.peek())     # "300"
\end{lstlisting}

\item Написать программу на Python, которая создает класс Stack для представления стека с инкапсуляцией. Класс должен содержать методы push, pop, is\_empty, size и peek, которые реализуют операции вталкивания, выталкивания, проверки пустоты, получения размера и просмотра вершины стека соответственно. Программа также должна создавать экземпляр класса Stack, вталкивать в него элементы, выталкивать элементы и выводить информацию о стеке на экран.

Инструкции:
\begin{enumerate}
    \item Создайте класс Stack с методом \_\_init\_\_, который инициализирует пустой стек. Принимает параметр push\_with\_index=False. Если True, то при добавлении сохраняется кортеж (элемент, порядковый\_номер\_добавления).
    \item Создайте метод push, который принимает элемент. Если push\_with\_index=True, добавляет (element, self.\_counter), где \_counter — внутренний счетчик, увеличивающийся при каждом добавлении. Иначе — element.
    \item Создайте метод pop, который выталкивает верхний элемент (или кортеж) и возвращает его. Если стек пуст, возвращает None.
    \item Создайте метод is\_empty, который возвращает True, если стек пуст, и False в противном случае.
    \item Создайте метод size, который возвращает текущее количество элементов в стеке.
    \item Создайте метод peek, который возвращает верхний элемент (или кортеж), если стек не пуст. Если стек пуст, возвращает None.
    \item Создайте экземпляр класса Stack с push\_with\_index=True.
    \item Добавьте элементы: "alpha", "beta", "gamma".
    \item Выведите размер стека и результат peek (должен быть ("gamma", 2) — если считать с 0).
    \item Вызовите pop, выведите результат.
    \item Повторите вывод размера и peek.
\end{enumerate}

Пример использования:
\begin{lstlisting}[language=Python]
stack = Stack(push_with_index=True)
stack.push("alpha")  # ("alpha", 0)
stack.push("beta")   # ("beta", 1)
stack.push("gamma")  # ("gamma", 2)

print("Размер стека:", stack.size())
print("Верхний элемент:", stack.peek())  # ('gamma', 2)

popped = stack.pop()
print("Вытолкнут:", popped)  # ('gamma', 2)

print("Размер после pop:", stack.size())
print("Верхний элемент:", stack.peek())  # ('beta', 1)
\end{lstlisting}

\item Написать программу на Python, которая создает класс Stack для представления стека с инкапсуляцией. Класс должен содержать методы push, pop, is\_empty, size и peek, которые реализуют операции вталкивания, выталкивания, проверки пустоты, получения размера и просмотра вершины стека соответственно. Программа также должна создавать экземпляр класса Stack, вталкивать в него элементы, выталкивать элементы и выводить информацию о стеке на экран.

Инструкции:
\begin{enumerate}
    \item Создайте класс Stack с методом \_\_init\_\_, который инициализирует пустой стек. Принимает параметр push\_unique\_top=False. Если True, то при добавлении, если элемент равен текущему верхнему, он не добавляется.
    \item Создайте метод push, который принимает элемент. Если push\_unique\_top=True и стек не пуст и element == текущий\_верх, то элемент не добавляется. Иначе — добавляется.
    \item Создайте метод pop, который выталкивает верхний элемент из стека и возвращает его. Если стек пуст, возвращает None.
    \item Создайте метод is\_empty, который возвращает True, если стек пуст, и False в противном случае.
    \item Создайте метод size, который возвращает текущее количество элементов в стеке.
    \item Создайте метод peek, который возвращает верхний элемент стека, если стек не пуст. Если стек пуст, возвращает None.
    \item Создайте экземпляр класса Stack с push\_unique\_top=True.
    \item Добавьте элементы: 1, 2, 2, 3, 3, 3, 4.
    \item Выведите размер стека (должен быть 4) и верхний элемент (4).
    \item Вызовите pop, выведите результат (4).
    \item Повторите вывод размера и верхнего элемента (теперь 3).
\end{enumerate}

Пример использования:
\begin{lstlisting}[language=Python]
stack = Stack(push_unique_top=True)
stack.push(1)
stack.push(2)
stack.push(2)  # не добавится
stack.push(3)
stack.push(3)  # не добавится
stack.push(3)  # не добавится
stack.push(4)

print("Размер стека:", stack.size())     # 4
print("Верхний элемент:", stack.peek())   # 4

popped = stack.pop()
print("Вытолкнут:", popped)  # 4

print("Размер после pop:", stack.size())    # 3
print("Верхний элемент:", stack.peek())     # 3
\end{lstlisting}

\item Написать программу на Python, которая создает класс Stack для представления стека с инкапсуляцией. Класс должен содержать методы push, pop, is\_empty, size и peek, которые реализуют операции вталкивания, выталкивания, проверки пустоты, получения размера и просмотра вершины стека соответственно. Программа также должна создавать экземпляр класса Stack, вталкивать в него элементы, выталкивать элементы и выводить информацию о стеке на экран.

Инструкции:
\begin{enumerate}
    \item Создайте класс Stack с методом \_\_init\_\_, который инициализирует пустой стек. Принимает параметр push\_even\_only=False. Если True, то добавляются только четные числа.
    \item Создайте метод push, который принимает элемент. Если push\_even\_only=True и element \% 2 != 0, элемент не добавляется. Иначе — добавляется.
    \item Создайте метод pop, который выталкивает верхний элемент из стека и возвращает его. Если стек пуст, возвращает None.
    \item Создайте метод is\_empty, который возвращает True, если стек пуст, и False в противном случае.
    \item Создайте метод size, который возвращает текущее количество элементов в стеке.
    \item Создайте метод peek, который возвращает верхний элемент стека, если стек не пуст. Если стек пуст, возвращает None.
    \item Создайте экземпляр класса Stack с push\_even\_only=True.
    \item Добавьте элементы: 1 (не добавится), 2, 3 (не добавится), 4, 5 (не добавится), 6.
    \item Выведите размер стека (должен быть 3) и верхний элемент (6).
    \item Вызовите pop, выведите результат (6).
    \item Повторите вывод размера и верхнего элемента (теперь 4).
\end{enumerate}

Пример использования:
\begin{lstlisting}[language=Python]
stack = Stack(push_even_only=True)
stack.push(1)  # нет
stack.push(2)
stack.push(3)  # нет
stack.push(4)
stack.push(5)  # нет
stack.push(6)

print("Размер стека:", stack.size())     # 3
print("Верхний элемент:", stack.peek())   # 6

popped = stack.pop()
print("Вытолкнут:", popped)  # 6

print("Размер после pop:", stack.size())    # 2
print("Верхний элемент:", stack.peek())     # 4
\end{lstlisting}

\item Написать программу на Python, которая создает класс Stack для представления стека с инкапсуляцией. Класс должен содержать методы push, pop, is\_empty, size и peek, которые реализуют операции вталкивания, выталкивания, проверки пустоты, получения размера и просмотра вершины стека соответственно. Программа также должна создавать экземпляр класса Stack, вталкивать в него элементы, выталкивать элементы и выводить информацию о стеке на экран.

Инструкции:
\begin{enumerate}
    \item Создайте класс Stack с методом \_\_init\_\_, который инициализирует пустой стек. Принимает параметр push\_odd\_only=False. Если True, то добавляются только нечетные числа.
    \item Создайте метод push, который принимает элемент. Если push\_odd\_only=True и element \% 2 == 0, элемент не добавляется. Иначе — добавляется.
    \item Создайте метод pop, который выталкивает верхний элемент из стека и возвращает его. Если стек пуст, возвращает None.
    \item Создайте метод is\_empty, который возвращает True, если стек пуст, и False в противном случае.
    \item Создайте метод size, который возвращает текущее количество элементов в стеке.
    \item Создайте метод peek, который возвращает верхний элемент стека, если стек не пуст. Если стек пуст, возвращает None.
    \item Создайте экземпляр класса Stack с push\_odd\_only=True.
    \item Добавьте элементы: 2 (не добавится), 1, 4 (не добавится), 3, 6 (не добавится), 5.
    \item Выведите размер стека (должен быть 3) и верхний элемент (5).
    \item Вызовите pop, выведите результат (5).
    \item Повторите вывод размера и верхнего элемента (теперь 3).
\end{enumerate}

Пример использования:
\begin{lstlisting}[language=Python]
stack = Stack(push_odd_only=True)
stack.push(2)  # нет
stack.push(1)
stack.push(4)  # нет
stack.push(3)
stack.push(6)  # нет
stack.push(5)

print("Размер стека:", stack.size())     # 3
print("Верхний элемент:", stack.peek())   # 5

popped = stack.pop()
print("Вытолкнут:", popped)  # 5

print("Размер после pop:", stack.size())    # 2
print("Верхний элемент:", stack.peek())     # 3
\end{lstlisting}

\end{enumerate}