\documentclass{article}
\usepackage[utf8]{inputenc}
\usepackage[russian]{babel}
\usepackage[T2A]{fontenc}
\usepackage{hyperref}
\usepackage{amssymb}
\usepackage[a4paper]{geometry}
\usepackage{listingsutf8}
\usepackage{tikz}
\lstset{
    inputencoding=utf8,
    language=Python,
    basicstyle=\ttfamily\footnotesize,
    breaklines=true,
    keepspaces=true,
    showstringspaces=false,
	literate={а}{{\selectfont\char224}}1
             {б}{{\selectfont\char225}}1
             {в}{{\selectfont\char226}}1
             {г}{{\selectfont\char227}}1
             {д}{{\selectfont\char228}}1
             {е}{{\selectfont\char229}}1
             {ё}{{\"e}}1
             {ж}{{\selectfont\char230}}1
             {з}{{\selectfont\char231}}1
             {и}{{\selectfont\char232}}1
             {й}{{\selectfont\char233}}1
             {к}{{\selectfont\char234}}1
             {л}{{\selectfont\char235}}1
             {м}{{\selectfont\char236}}1
             {н}{{\selectfont\char237}}1
             {о}{{\selectfont\char238}}1
             {п}{{\selectfont\char239}}1
             {р}{{\selectfont\char240}}1
             {с}{{\selectfont\char241}}1
             {т}{{\selectfont\char242}}1
             {у}{{\selectfont\char243}}1
             {ф}{{\selectfont\char244}}1
             {х}{{\selectfont\char245}}1
             {ц}{{\selectfont\char246}}1
             {ч}{{\selectfont\char247}}1
             {ш}{{\selectfont\char248}}1
             {щ}{{\selectfont\char249}}1
             {ъ}{{\selectfont\char250}}1
             {ы}{{\selectfont\char251}}1
             {ь}{{\selectfont\char252}}1
             {э}{{\selectfont\char253}}1
             {ю}{{\selectfont\char254}}1
             {я}{{\selectfont\char255}}1
             {А}{{\selectfont\char192}}1
             {Б}{{\selectfont\char193}}1
             {В}{{\selectfont\char194}}1
             {Г}{{\selectfont\char195}}1
             {Д}{{\selectfont\char196}}1
             {Е}{{\selectfont\char197}}1
             {Ё}{{\"E}}1
             {Ж}{{\selectfont\char198}}1
             {З}{{\selectfont\char199}}1
             {И}{{\selectfont\char200}}1
             {Й}{{\selectfont\char201}}1
             {К}{{\selectfont\char202}}1
             {Л}{{\selectfont\char203}}1
             {М}{{\selectfont\char204}}1
             {Н}{{\selectfont\char205}}1
             {О}{{\selectfont\char206}}1
             {П}{{\selectfont\char207}}1
             {Р}{{\selectfont\char208}}1
             {С}{{\selectfont\char209}}1
             {Т}{{\selectfont\char210}}1
             {У}{{\selectfont\char211}}1
             {Ф}{{\selectfont\char212}}1
             {Х}{{\selectfont\char213}}1
             {Ц}{{\selectfont\char214}}1
             {Ч}{{\selectfont\char215}}1
             {Ш}{{\selectfont\char216}}1
             {Щ}{{\selectfont\char217}}1
             {Ъ}{{\selectfont\char218}}1
             {Ы}{{\selectfont\char219}}1
             {Ь}{{\selectfont\char220}}1
             {Э}{{\selectfont\char221}}1
             {Ю}{{\selectfont\char222}}1
             {Я}{{\selectfont\char223}}1
}

\begin{document}
{
	\thispagestyle{empty}
	\vspace*{\fill}
	\centering 

	\Large Сборник заданий для семинарских занятий \\
	по курсу \\
	<<Объектно-ориентированное программирование на Python>>\\
	\large
	\vspace{40pt}
	\vspace*{\fill}
	\newpage
}

\tableofcontents
\newpage
\section{Общие сведения}

Сборник содержит задания для семинарских занятий 
по курсу <<Объектно-ориентированное программирование на Python>> 
(32 часа).

Задачник находится в процессе наполнения и новые задания 
появляются перед проведением нового семинара.

Возможна сдача другого кода (например, выполненного в ходе
проектной деятельности), еслои они полностью покрывают материал семинара.

\section{Задания}

\subsection{Семинар <<Правила формирования класса для программирования в IDE PyCharm. Отработка навыков создания простых классов и объектов класса>> 
(2 очных часа)}

В ходе работы создайте 5 классов с соответствующими методами, описанными в индивидуальном задании. 
Предполагается, что пользователь класса не имеет права обращаться к свойствам напрямую 
(соблюдая принцип инкапсуляции), а должен использовать методы. Важно: в задании не всегда указаны 
все необходимые методы и свойства, при необходимости вам надо самостоятельно их добавить.
Продемонстрируйте работоспособность всех методов (из задания) посредством создания запускаемых файлов, где осуществляется вызов методов для разных ситуаций (без ручного ввода, но с выводом результатов в консоль). 
Каждый класс должен сохраняться в отдельном исходном файле. Необходимо соблюдать все стандартные требования к качеству кода (отступы, именования переменных, классов, методов, проверка корректности входных данных).
Для каждого класса создайте отдельный запускаемый файл для проверки всех его методов 
(допускается использование других классов в этих тестах).

Все предлагаемые классы в заданиях упрощенные; для использования в production-окружении они требуют серьезной доработки. Суть задания — в отработке базовых навыков, а не в идеальном моделировании предложенных ситуаций.

Для сдачи работы будьте готовы пояснить или аналогично заданию модифицировать любую часть кода, а также ответить на вопросы:
\begin{enumerate}
    \item Кратко опишите парадигму объектно-ориентированного программирования (ООП).
    \item Что такое класс в парадигме ООП?
    \item Что такое объект (экземпляр) в парадигме ООП?
    \item Что обозначает свойство инкапсуляции в парадигме ООП?
    \item Синтаксис классов в Python (в рамках выполненной работы), создание и работа с объектами в Python.
\end{enumerate}

При выполнении задания предполагается самое простое базовое описание классов, соответствующее следующему 
примеру (вы можете использовать то, что вы ЗНАЕТЕ дополнительно, но это остается на ваше усмотрение):

Если вы нашли в задачнике ошибки, опечатки и другие недостатки, то вы можете сделать pull-request. 

\begin{lstlisting}
class Worker:
    def set_last_name(self, last_name):
        self.last_name = last_name

    def print_last_name(self):
        print (f"Фамилия: {self.last_name}")

    def get_last_name(self):
        return last_name

worker = Worker()
worker.set_last_name(self,"Иванов")
worker.print_last_name()
print(worker.get_last_name())
\end{lstlisting}

\textbf{Срок сдачи работы (начала сдачи):} следующее занятие после его выдачи. В последующие сроки оценка будет снижаться (при отсутствии оправдывающих документов).

\begin{enumerate}

\item
\textbf{Описание ситуации:}
Рассмотрим работу грузовой железнодорожной станции. На станции есть несколько путей, по которым поезда могут прибывать и отправляться. Каждый путь имеет свой номер и может вместить несколько поездов. Поезда формируются из вагонов, каждый из которых может перевозить разные грузы. Работники станции отвечают за диспетчерское управление маневровыми локомотивами, осмотр вагонов, выполнение погрузочно-разгрузочных работ, прием груза к перевозке, ремонт путей, обеспечение безопасности и т.п. Они используют радиостанции для связи друг с другом и для отслеживания положения поездов и передвижения вагонов.

\textbf{Создаваемые классы:} `Путь`, `Поезд`, `Вагон`, `Станция`, `РаботникСтанции`.

Для классов реализовать следующие простые методы (ниже приведен не исчерпывающий список методов; для демонстрации работы классов вам потребуются дополнительные методы, позволяющие отследить состояние объектов), используя для хранения данных списки (`[]`) Python:
\begin{enumerate}
    \item \textbf{Путь:} добавить поезд на путь, убрать поезд с пути, получить список поездов на конкретном пути.
    \item \textbf{Поезд:} прицепить вагон к поезду, отцепить вагон от поезда, получить (распечатать) список вагонов в поезде, вывести информацию о грузе в поезде.
    \item \textbf{Вагон:} добавить номер поезда, в который включался конкретный вагон, удалить номер поезда из истории, отобразить историю поездов для конкретного вагона.
    \item \textbf{РаботникСтанции:} класс, представляющий отдельного работника на станции, имеющий идентификатор, информацию о персональной радиостанции, список закрепленных за ним поездов для осмотра, ФИО, должность.
    \item \textbf{Станция:} добавить станционный путь, добавить поезд на станцию, нанять работника станции, вывести информацию о всех путях, поездах, работниках, удалить путь, удалить поезд, уволить работника.
\end{enumerate}

\item
\textbf{Описание ситуации:}
Рассмотрим работу крупного логистического терминала для обработки грузовых автомобилей. На терминале есть несколько доков (рамп), куда фуры прибывают для проведения погрузочно-разгрузочных работ. Каждый док имеет свой номер и может одновременно обслуживать одну машину. Грузовики перевозят паллеты, каждая из которых содержит определенный товар. Сотрудники терминала отвечают за прием грузовиков, управление погрузочной техникой, проверку сопроводительных документов, приемку и отгрузку товара, а также техническое обслуживание доков. Они используют портативные рации для координации действий и отслеживания статуса обработки автомобилей.

\textbf{Создаваемые классы:} `Док`, `Грузовик`, `Паллета`, `Терминал`, `Сотрудник`.

Для классов реализовать следующие простые методы, используя для хранения данных списки (`[]`) Python:
\begin{enumerate}
    \item \textbf{Док:} занять док конкретным грузовиком, освободить док, получить информацию о грузовике, который сейчас находится на доке.
    \item \textbf{Грузовик:} добавить паллету в грузовик, выгрузить паллету из грузовика, получить (распечатать) список паллет в грузовике, вывести информацию о товарах в грузовике.
    \item \textbf{Паллета:} добавить номер грузовика, в который загружалась конкретная паллета, удалить номер грузовика из истории, отобразить историю перевозок (номера грузовиков) для конкретной паллеты.
    \item \textbf{Сотрудник:} класс, представляющий отдельного сотрудника терминала, имеющий идентификатор, номер рации, список доков, за которые он отвечает, ФИО, должность.
    \item \textbf{Терминал:} добавить новый док на терминале, зарегистрировать прибытие грузовика, нанять нового сотрудника, вывести список всех доков, грузовиков на территории, сотрудников, удалить док, удалить грузовик, уволить сотрудника.
\end{enumerate}

\item
\textbf{Описание ситуации:}
Рассмотрим работу аэропорта. В аэропорту есть несколько взлетно-посадочных полос (ВПП), которые принимают и отправляют рейсы. Каждая ВПП имеет свой номер, длину и статус доступности. Самолеты перевозят пассажиров и их ручную кладь, размещенную в салоне. Авиадиспетчеры управляют движением самолетов, назначают полосы для взлета и посадки, следят за воздушной обстановкой и координируют действия с помощью радиосвязи.

\textbf{Создаваемые классы:} `ВПП`, `Самолет`, `Пассажир`, `Аэропорт`, `Авиадиспетчер`.

Для классов реализовать следующие простые методы, используя для хранения данных списки (`[]`) Python:
\begin{enumerate}
    \item \textbf{ВПП:} занять полосу для взлета/посадки, освободить полосу, получить список рейсов, использовавших полосу.
    \item \textbf{Самолет:} добавить пассажира на борт (включая вес его ручной клади), высадить пассажира, получить (распечатать) список пассажиров на борту, рассчитать общий вес ручной клади.
    \item \textbf{Пассажир:} добавить рейс в историю перелетов пассажира, удалить рейс из истории (ошибка бронирования), отобразить всю историю перелетов.
    \item \textbf{Авиадиспетчер:} класс, представляющий диспетчера, имеющий идентификатор, рабочую частоту, график работы (список интервалов времени в сутках), ФИО.
    \item \textbf{Аэропорт:} добавить новую ВПП, зарегистрировать прибытие самолета, нанять диспетчера, вывести список всех ВПП, самолетов в аэропорту, диспетчеров, удалить ВПП (на ремонт), списать самолет, уволить диспетчера.
\end{enumerate}

\item
\textbf{Описание ситуации:}
Рассмотрим работу речного порта. В порту есть несколько причалов для швартовки грузовых барж и буксиров. Каждый причал имеет уникальный номер и максимальную глубину, определяющую осадку судов, которые могут к нему подойти. Баржи перевозят контейнеры с различными грузами. 
Их характеризуют вес судна, максимальная грузоподъемность и осадка (как без груза, так и с максимальным грузом). Портовые рабочие отвечают за швартовку судов, управление портовыми кранами для погрузки/разгрузки контейнеров, оформление документов и поддержание порядка на территории.

\textbf{Создаваемые классы:} `Причал`, `Баржа`, `Контейнер`, `Порт`, `ПортовыйРабочий`.

Для классов реализовать следующие простые методы, используя для хранения данных списки (`[]`) Python:
\begin{enumerate}
    \item \textbf{Причал:} пришвартовать баржу к причалу, отшвартовать баржу, получить список барж, находящихся у причала.
    \item \textbf{Баржа:} загрузить контейнер на баржу (с указанием веса контейнера), разгрузить контейнер с баржи,
     получить (распечатать) список контейнеров на барже, рассчитать текущую осадку судна 
     (предполагается линейная зависимость осадки от суммарного веса груза и баржи).
    \item \textbf{Контейнер:} добавить номер баржи, на которую погрузили контейнер, удалить номер баржи, отобразить историю перемещений контейнера между баржами.
    \item \textbf{ПортовыйРабочий:} класс, представляющий рабочего, имеющий идентификатор, допуск к работе с краном, список закрепленных причалов, ФИО, должность.
    \item \textbf{Порт:} ввести новый причал в эксплуатацию, принять баржу в акваторию порта, принять на работу рабочего, вывести список причалов, барж в акватории, рабочих, списать причал, отправить баржу, уволить рабочего.
\end{enumerate}

\item
\textbf{Описание ситуации:}
Рассмотрим работу автобусного парка. В парке есть несколько маршрутов, которые обслуживаются автобусами. Каждый маршрут имеет номер и список остановок. Автобусы имеют государственный номер, количество мест и текущий пробег. Водители закреплены за автобусами и маршрутами. Диспетчеры автопарка составляют расписание, следят за выходами автобусов на линию, учетом пробега и техническим состоянием.

\textbf{Создаваемые классы:} `Маршрут`, `Автобус`, `Остановка`, `Автопарк`, `Водитель`.

Для классов реализовать следующие простые методы, используя для хранения данных списки (`[]`) Python:
\begin{enumerate}
    \item \textbf{Маршрут:} добавить остановку в маршрут, удалить остановку из маршрута, получить список всех остановок на маршруте.
    \item \textbf{Автобус:} назначить автобус на маршрут, снять с маршрута, увеличить пробег на заданное значение, получить текущий пробег.
    \item \textbf{Остановка:} добавить маршрут, проходящий через остановку, удалить маршрут, отобразить список всех маршрутов, проходящих через данную остановку.
    \item \textbf{Водитель:} класс, представляющий водителя, имеющий идентификатор, права категории, закрепленный автобус, ФИО, график работы.
    \item \textbf{Автопарк:} добавить новый маршрут, приобрести новый автобус, принять на работу водителя, вывести список маршрутов, автобусов (с указанием их состояния), водителей, списать автобус, уволить водителя.
\end{enumerate}

\item
\textbf{Описание ситуации:}
Рассмотрим работу метрополитена. В метро есть линии, состоящие из станций и тоннелей между ними. Составы из вагонов перемещаются по линиям. Каждая станция имеет название и может быть точкой пересадки на другие линии. Машинисты управляют поездами. Дежурные по станции следят за порядком на платформах и работой оборудования. Управление метрополитеном координирует движение составов.

\textbf{Создаваемые классы:} `ЛинияМетро`, `ПоездМетро`, `Станция`, `УправлениеМетрополитеном`, `Машинист`.

Для классов реализовать следующие простые методы, используя для хранения данных списки (`[]`) Python:
\begin{enumerate}
    \item \textbf{ЛинияМетро:} добавить станцию на линию, получить список станций на линии, получить список поездов на линии.
    \item \textbf{ПоездМетро:} добавить вагон в состав, отцепить вагон, назначить машиниста на поезд.
    \item \textbf{Станция:} добавить линию, проходящую через станцию (для моделирования пересадочных узлов), получить список линий на станции.
    \item \textbf{Машинист:} класс, представляющий машиниста, имеющий идентификатор, допуск к управлению, закрепленный поезд, ФИО, стаж.
    \item \textbf{УправлениеМетрополитеном:} открыть новую линию, ввести новый поезд в эксплуатацию, принять на работу машиниста, вывести список линий, поездов (в депо и на линиях), машинистов, закрыть линию на техобслуживание, списать поезд, вывести полную схему метро (в текстовом виде).
\end{enumerate}

\item
\textbf{Описание ситуации:}
Рассмотрим работу службы доставки пиццы. В службе есть несколько филиалов. Каждый филиал обслуживает определенный район и имеет курьеров. Заказы формируются из позиций меню. Курьеры используют скутеры для доставки. Менеджеры филиалов принимают заказы, назначают курьеров и следят за выполнением заказов.

\textbf{Создаваемые классы:} `Филиал`, `Заказ`, `Курьер`, `Скутер`, `Менеджер`.

Для классов реализовать следующие простые методы, используя для хранения данных списки (`[]`) Python:
\begin{enumerate}
    \item \textbf{Филиал:} добавить курьера в филиал, уволить курьера, получить список активных заказов филиала.
    \item \textbf{Заказ:} добавить позицию в заказ (название + цена), удалить позицию, рассчитать стоимость заказа, изменить статус заказа (принят, готовится, в пути, доставлен).
    \item \textbf{Курьер:} назначить заказ курьеру, завершить доставку заказа, получить список доставленных заказов за смену, закрепить скутер за курьером.
    \item \textbf{Менеджер:} класс, представляющий менеджера, имеющий идентификатор, закрепленный филиал, ФИО, смену.
    \item \textbf{Скутер:} отправить на зарядку, вернуть в строй, увеличить пробег, получить текущий пробег.
\end{enumerate}

\textbf{Описание ситуации:}
Рассмотрим работу трамвайного депо. В депо есть несколько маршрутов, обслуживаемых трамвайными вагонами. Каждый трамвайный вагон имеет бортовой номер, вместимость и текущий пробег. Маршруты состоят из остановок и имеют определенный график движения. Водители трамваев закреплены за конкретными вагонами и маршрутами. Диспетчеры управляют выпуском трамваев на линию и ведут учет технического состояния.

\textbf{Создаваемые классы:} Маршрут, Трамвай, Остановка, Депо, Водитель.

Для классов реализовать следующие простые методы, используя для хранения данных списки ([]) Python:
\begin{enumerate}
\item \textbf{Маршрут:} добавить остановку в маршрут, удалить остановку из маршрута, получить список всех остановок на маршруте.
\item \textbf{Трамвай:} назначить трамвай на маршрут, снять с маршрута, увеличить пробег на заданное значение, получить текущий пробег.
\item \textbf{Остановка:} добавить маршрут, проходящий через остановку, удалить маршрут, отобразить список всех маршрутов, проходящих через данную остановку.
\item \textbf{Водитель:} класс, представляющий водителя, имеющий идентификатор, права категории, закрепленный трамвай, ФИО, график работы.
\item \textbf{Депо:} добавить новый маршрут, принять новый трамвай в депо, принять на работу водителя, выполнить вывод списка маршрутов, трамваев (с указанием их состояния), водителей, списать трамвай, уволить водителя.
\end{enumerate}

\item
\textbf{Описание ситуации:}
Рассмотрим работу морского порта для приёма пассажирских паромов. 
В порту есть несколько причалов, каждый из которых обслуживает один паром за раз. 
Паромы перевозят пассажиров и автомобили. 
Пассажиры покупают билеты, автомобили записываются в список грузовой палубы. 
Сотрудники порта координируют погрузку, проверку билетов и безопасность.
\textbf{Создаваемые классы:} Причал, Паром, Пассажир, Автомобиль, СотрудникПорта.
\begin{enumerate}
    \item \textbf{Причал:} пришвартовать паром, освободить причал, получить информацию о пароме у причала.
    \item \textbf{Паром:} добавить пассажира, добавить автомобиль, высадить пассажира, выгрузить автомобиль.
    \item \textbf{Пассажир:} добавить рейс в историю поездок, удалить рейс из истории, 
    вывести историю поездок.
    \item \textbf{Автомобиль:} зарегистрировать номер парома, удалить номер парома, вывести историю перевозок.
    \item \textbf{СотрудникПорта:} идентификатор, должность, ФИО, список закреплённых причалов.
\end{enumerate}

\item
\textbf{Описание ситуации:}
Рассмотрим работу пригородной электрички. В системе есть станции, 
между которыми курсируют электрички. У каждой электрички есть номер, 
список вагонов и машинист. Пассажиры покупают билеты и 
занимают места в вагонах. Диспетчеры контролируют движение электричек.
\textbf{Создаваемые классы:} Станция, Электричка, Вагон, Пассажир, Диспетчер.
\begin{enumerate}
    \item \textbf{Станция:} принять электричку, отправить электричку, вывести список электричек на станции.
    \item \textbf{Электричка:} добавить вагон, отцепить вагон, получить список вагонов.
    \item \textbf{Вагон:} посадить пассажира, высадить пассажира, вывести список пассажиров.
    \item \textbf{Пассажир:} добавить поездку в историю, удалить поездку, показать историю поездок.
    \item \textbf{Диспетчер:} идентификатор, ФИО, рабочая смена, список контролируемых электричек.
\end{enumerate}

\item
\textbf{Описание ситуации:}
Рассмотрим работу таксопарка. В таксопарке есть автомобили, 
водители и диспетчеры. Автомобиль закрепляется за водителем. 
Диспетчеры принимают заказы и назначают их водителям. Пассажиры совершают поездки.
\textbf{Создаваемые классы:} Таксопарк, Автомобиль, Водитель, Заказ, Диспетчер.
\begin{enumerate}
    \item \textbf{Таксопарк:} добавить автомобиль, принять водителя, вывести список машин и водителей, 
    уволить водителя.
    \item \textbf{Автомобиль:} назначить водителя, снять водителя, увеличить пробег, получить пробег.
    \item \textbf{Водитель:} назначить заказ, завершить заказ, 
    вывести список выполненных заказов.
    \item \textbf{Заказ:} назначить пассажира, завершить поездку, вывести информацию о заказе.
    \item \textbf{Диспетчер:} идентификатор, ФИО, список назначенных заказов.
\end{enumerate}

\item
\textbf{Описание ситуации:}
Рассмотрим работу грузового аэропорта. Самолёты перевозят контейнеры. 
В аэропорту есть ангары для хранения самолётов и площадки для погрузки. 
Работники аэропорта координируют загрузку и выгрузку контейнеров.
\textbf{Создаваемые классы:} Самолёт, Контейнер, Ангар, РаботникАэропорта, Аэропорт.
\begin{enumerate}
    \item \textbf{Самолёт:} загрузить контейнер, выгрузить контейнер, вывести список контейнеров.
    \item \textbf{Контейнер:} добавить номер самолёта, удалить номер самолёта, вывести историю перевозок.
    \item \textbf{Ангар:} принять самолёт, вывести список самолётов, освободить ангар.
    \item \textbf{РаботникАэропорта:} идентификатор, ФИО, должность, список самолётов в обслуживании.
    \item \textbf{Аэропорт:} принять самолёт, убрать самолёт, принять раотника, уволить работника, 
    вывести список самолётов и работников.
\end{enumerate}

\item
\textbf{Описание ситуации:}
Рассмотрим работу велопроката. В прокате есть велосипеды, станции для их хранения, 
клиенты и сотрудники. Клиенты арендуют велосипеды и возвращают их на станцию.
\textbf{Создаваемые классы:} Велосипед, СтанцияПроката, Клиент, Сотрудник, Прокат.
\begin{enumerate}
    \item \textbf{Велосипед:} выдать в аренду, вернуть на станцию, получить пробег.
    \item \textbf{СтанцияПроката:} добавить велосипед, убрать велосипед, вывести список велосипедов.
    \item \textbf{Клиент:} арендовать велосипед, вернуть велосипед, вывести историю аренд.
    \item \textbf{Сотрудник:} идентификатор, ФИО, должность, список закреплённых станций.
    \item \textbf{Прокат:} добавить станцию, демонтировать станцию, 
    вывести список станций и велосипедов, уволить сотрудника, нанять сотрудника, вывести список сотрудников.
\end{enumerate}

\item
\textbf{Описание ситуации:}
Рассмотрим работу речных теплоходов. У каждого теплохода есть рейсы и список пассажиров. 
Пассажиры покупают билеты. Работники пристани обслуживают теплоходы.
\textbf{Создаваемые классы:} Теплоход, Рейс, Пассажир, Пристань, РаботникПристани.
\begin{enumerate}
    \item \textbf{Теплоход:} добавить рейс, убрать рейс, вывести список рейсов.
    \item \textbf{Рейс:} добавить пассажира, удалить пассажира, вывести список пассажиров.
    \item \textbf{Пассажир:} добавить рейс в историю, удалить рейс, вывести историю.
    \item \textbf{Пристань:} принять теплоход, отправить теплоход, вывести список теплоходов.
    \item \textbf{РаботникПристани:} идентификатор, ФИО, должность, закреплённые рейсы.
\end{enumerate}

\item
\textbf{Описание ситуации:}
Рассмотрим работу каршеринга. В системе есть автомобили, клиенты и диспетчеры. 
Автомобили бронируются клиентами и возвращаются после поездки. Диспетчеры контролируют состояние машин.
\textbf{Создаваемые классы:} Автомобиль, Клиент, Диспетчер, Заказ, Каршеринг.
\begin{enumerate}
    \item \textbf{Автомобиль:} выдать клиенту, вернуть, увеличить пробег, вывести пробег.
    \item \textbf{Клиент:} арендовать автомобиль, завершить аренду, вывести историю аренд.
    \item \textbf{Диспетчер:} идентификатор, ФИО, список автомобилей под контролем.
    \item \textbf{Заказ:} назначить автомобиль, завершить поездку, вывести данные заказа.
    \item \textbf{Каршеринг:} добавить автомобиль, списать автомобиль, добавить клиента, удалить клиента, 
    добавить диспетчера, удалить диспетчера,
    вывести список клиентов, диспетчеров и машин.
\end{enumerate}

\item
\textbf{Описание ситуации:}
Рассмотрим работу железнодорожного музея. 
В музее есть экспонаты (локомотивы и вагоны), 
экскурсии и экскурсоводы. Посетители записываются на экскурсии.
\textbf{Создаваемые классы:} Экспонат, Экскурсия, Экскурсовод, Посетитель, Музей.
\begin{enumerate}
    \item \textbf{Экспонат:} добавить к экскурсии, убрать, вывести список экскурсий.
    \item \textbf{Экскурсия:} записать посетителя, удалить, вывести список посетителей.
    \item \textbf{Экскурсовод:} идентификатор, ФИО, список экскурсий.
    \item \textbf{Посетитель:} записаться на экскурсию, отменить запись, вывести историю.
    \item \textbf{Музей:} добавить экспонат, списать экспонат, добавить экскурсовода, уволить экскурсовода, 
    провести экскурсию, вывести список всех экскурсий и экскурсоводов.
\end{enumerate}

\item
\textbf{Описание ситуации:}
Рассмотрим работу автозаправочной станции. На станции есть топливо, 
колонки и операторы. Автомобили приезжают заправляться.
\textbf{Создаваемые классы:} Колонка, Автомобиль, Оператор, Топливо, АЗС.
\begin{enumerate}
    \item \textbf{Колонка:} заправить автомобиль, освободить колонку, вывести статус.
    \item \textbf{Автомобиль:} получить заправку, вывести историю заправок.
    \item \textbf{Оператор:} идентификатор, ФИО, список закреплённых колонок.
    \item \textbf{Топливо:} уменьшить количество, увеличить количество, вывести остаток.
    \item \textbf{АЗС:} добавить колонку, нанять оператора, уволить оператора, демонтировать колонку, 
    вывести список машин, операторов и колонок.
\end{enumerate}


\item \textbf{Описание ситуации:} Рассмотрим работу сортировочного центра курьерской службы. 
В центре есть зоны обработки посылок, конвейерные линии и сотрудники. 
Каждая посылка имеет трек-номер и проходит через несколько этапов обработки. 
Сотрудники сканируют посылки, сортируют их по направлениям и загружают в 
транспортировочные контейнеры. Менеджеры контролируют процесс сортировки и работу оборудования.

\textbf{Создаваемые классы:} `ЗонаОбработки`, `Посылка`, `Конвейер`, `СотрудникЦентра`, `СортировочныйЦентр`.

Для классов реализовать следующие простые методы, используя для хранения данных списки (`[]`) Python:
\begin{enumerate}
    \item \textbf{ЗонаОбработки:} добавить посылку в зону, удалить посылку из зоны, 
    получить список посылок в зоне.
    \item \textbf{Посылка:} добавить статус обработки (принята, сортируется, отправлена), 
    удалить ошибочный статус, отобразить историю статусов обработки.
    \item \textbf{Конвейер:} запустить конвейерную ленту, остановить конвейер, 
    добавить посылку на конвейер, снять посылку с конвейера.
    \item \textbf{СотрудникЦентра:} класс, представляющий сотрудника, имеющий идентификатор, 
    смену, список закрепленных зон обработки, ФИО, должность.
    \item \textbf{СортировочныйЦентр:} добавить новую зону обработки, 
    ввести в эксплуатацию конвейер, нанять сотрудника, вывести список всех зон, конвейеров, 
    сотрудников, удалить зону, вывести из эксплуатации конвейер, уволить сотрудника.
\end{enumerate}

\item \textbf{Описание ситуации:} Рассмотрим работу диспетчерской службы 
городского пассажирского транспорта. 
Диспетчеры отслеживают движение автобусов, троллейбусов и трамваев 
на маршрутах, регулируют интервалы движения, фиксируют отклонения от 
графика. Транспортные средства оснащены GPS-трекерами для передачи местоположения.

\textbf{Создаваемые классы:} `Маршрут`, `ТранспортноеСредство`, `Диспетчер`, `Остановка`, `ДиспетчерскаяСлужба`.

Для классов реализовать следующие простые методы, используя для хранения данных списки (`[]`) Python:
\begin{enumerate}
    \item \textbf{Маршрут:} добавить транспортное средство на маршрут, 
    снять с маршрута, получить список транспорта на маршруте.
    \item \textbf{ТранспортноеСредство:} обновить местоположение (координаты), 
    получить текущее местоположение, добавить информацию о задержке/опережении графика.
    \item \textbf{Диспетчер:} класс, представляющий диспетчера, 
    имеющий идентификатор, смену, список контролируемых маршрутов, ФИО.
    \item \textbf{Остановка:} добавить маршрут, проходящий через остановку, 
    удалить маршрут, получить список маршрутов на остановке.
    \item \textbf{ДиспетчерскаяСлужба:} добавить новый маршрут, зарегистрировать транспортное средство, 
    нанять диспетчера, вывести информацию о всех маршрутах, транспорте, 
    диспетчерах, удалить маршрут, списать транспорт, уволить диспетчера.
\end{enumerate}

\item \textbf{Описание ситуации:} Рассмотрим работу центра технического обслуживания 
городского транспорта. В центре есть ремонтные зоны для разных видов 
транспорта, запасы запчастей и бригады механиков. Транспортные средства 
проходят плановое ТО и внеплановый ремонт.

\textbf{Создаваемые классы:} `РемонтнаяЗона`, `ТранспортноеСредство`, `Запчасть`, `Механик`, `ЦентрТехОбслуживания`.

Для классов реализовать следующие простые методы, используя для хранения данных списки (`[]`) Python:
\begin{enumerate}
    \item \textbf{РемонтнаяЗона:} поставить транспорт на ремонт, 
    завершить ремонт, получить список транспорта в ремонте.
    \item \textbf{ТранспортноеСредство:} добавить запись о ремонте (дата, вид работ), 
    удалить ошибочную запись, отобразить историю ремонтов.
    \item \textbf{Запчасть:} уменьшить количество на складе, увеличить количество, 
    получить текущий остаток.
    \item \textbf{Механик:} класс, представляющий механика, имеющий идентификатор, 
    квалификацию, список закрепленных ремонтных зон, ФИО.
    \item \textbf{ЦентрТехОбслуживания:} добавить ремонтную зону, закупить запчасти, 
    нанять механика, вывести информацию о зонах, запчастях, механиках, удалить зону, уволить механика.
\end{enumerate}

\item \textbf{Описание ситуации:}
Рассмотрим работу логистического центра междугородных автобусных перевозок. 
Автобусы совершают рейсы между городами по определенным маршрутам, 
перевозя пассажиров и их багаж. Диспетчеры формируют расписание, 
продают билеты и контролируют отправление автобусов.

\textbf{Создаваемые классы:} `Автобус`, `Маршрут`, `Пассажир`, `Диспетчер`, `ЛогистическийЦентр`.

Для классов реализовать следующие простые методы, используя для хранения данных списки (`[]`) Python:
\begin{enumerate}
    \item \textbf{Автобус:} назначить на маршрут, снять с маршрута, 
    добавить пассажира, высадить пассажира, получить список пассажиров.
    \item \textbf{Маршрут:} добавить город в маршрут, удалить город, 
    получить список всех городов на маршруте.
    \item \textbf{Пассажир:} купить билет (добавить маршрут в историю), 
    сдать билет (удалить маршрут), показать историю поездок.
    \item \textbf{Диспетчер:} класс, представляющий диспетчера, 
    имеющий идентификатор, список закрепленных маршрутов, ФИО, график работы.
    \item \textbf{ЛогистическийЦентр:} добавить автобус в парк, 
    добавить маршрут, нанять диспетчера, вывести список автобусов, 
    маршрутов и диспетчеров, списать автобус, уволить диспетчера.
\end{enumerate}

\item \textbf{Описание ситуации:} Рассмотрим работу центра управления интеллектуальной 
транспортной системой города. Система включает в себя управление светофорами, 
камеры видеонаблюдения, датчики транспортного потока. Операторы следят 
за дорожной ситуацией и оперативно реагируют на инциденты.

\textbf{Создаваемые классы:} `Перекресток`, `Светофор`, `КамераНаблюдения`, `ОператорИТС`, `ЦентрУправления`.

Для классов реализовать следующие простые методы, используя для хранения данных списки (`[]`) Python:
\begin{enumerate}
    \item \textbf{Перекресток:} добавить светофор к перекрестку, удалить светофор, 
    получить список светофоров на перекрестке.
    \item \textbf{Светофор:} изменить режим работы (красный/желтый/зеленый), 
    получить текущий режим, добавить информацию о неисправности, вывести список неисправностей.
    \item \textbf{КамераНаблюдения:} включить запись, выключить запись, 
    получить статус работы, зафиксировать нарушение ПДД, вывести список нарушений.
    \item \textbf{ОператорИТС:} класс, представляющий оператора, 
    имеющий идентификатор, смену, список контролируемых перекрестков, ФИО.
    \item \textbf{ЦентрУправления:} добавить новый перекресток в систему, 
    установить светофор, установить камеру, нанять оператора, вывести информацию о перекрестках, 
    светофорах, камерах, операторах, удалить перекресток, уволить оператора, снять камеру, снять светофор.
\end{enumerate}

\item \textbf{Описание ситуации:} Рассмотрим работу службы эвакуации аварийных транспортных средств. 
Эвакуаторы дежурят на специальных парковках и выезжают по вызову на места ДТП 
или поломок. Диспетчеры принимают вызовы и направляют ближайший свободный эвакуатор.

\textbf{Создаваемые классы:} `Эвакуатор`, `Вызов`, `ПарковкаЭвакуаторов`, `ДиспетчерЭвакуации`, `СлужбаЭвакуации`.

Для классов реализовать следующие простые методы, используя для хранения данных списки (`[]`) Python:
\begin{enumerate}
    \item \textbf{Эвакуатор:} принять вызов, завершить вызов, 
    получить текущий статус (свободен/занят), обновить местоположение.
    \item \textbf{Вызов:} зафиксировать время принятия, время выполнения, 
    получить статус выполнения.
    \item \textbf{ПарковкаЭвакуаторов:} принять эвакуатор на парковку, 
    выпустить эвакуатор с парковки, 
    получить список эвакуаторов на парковке.
    \item \textbf{ДиспетчерЭвакуации:} класс, представляющий диспетчера, 
    имеющий идентификатор, смену, список обработанных вызовов, ФИО.
    \item \textbf{СлужбаЭвакуации:} добавить эвакуатор в парк, 
    списать эвакуатор, 
    нанять диспетчера, вывести информацию о эвакуаторах, вызовах, диспетчерах, уволить диспетчера.
\end{enumerate}

\item \textbf{Описание ситуации:} Рассмотрим работу центра контроля коммерческих грузоперевозок. 
Система отслеживает движение грузовых автомобилей, контролирует соблюдение маршрутов, 
норм труда водителей и расход топлива. Менеджеры по логистике планируют маршруты 
и анализируют отчеты.

\textbf{Создаваемые классы:} `ГрузовойАвтомобиль`, `МаршрутПеревозки`, `Водитель`, `Рейс`, `МенеджерЛогистики`.

Для классов реализовать следующие простые методы, используя для хранения данных списки (`[]`) Python:
\begin{enumerate}
    \item \textbf{ГрузовойАвтомобиль:} начать рейс, завершить рейс, получить текущий статус, 
    зафиксировать расход топлива.
    \item \textbf{МаршрутПеревозки:} добавить точку маршрута (город, склад), 
    удалить точку, получить полный маршрут.
    \item \textbf{Водитель:} класс, представляющий водителя, 
    имеющий идентификатор, права, график работы, ФИО, стаж.
    \item \textbf{Рейс:} закрепить автомобиль за рейсом, закрепить водителя за рейсом, 
    открепить автомобиль, снять водителя, получить информацию о рейсе.
    \item \textbf{МенеджерЛогистики:} класс, представляющий менеджера, 
    имеющий идентификатор, список контролируемых маршрутов, ФИО.
\end{enumerate}

\item \textbf{Описание ситуации:} Рассмотрим работу службы парковки аэропорта. 
На территории аэропорта есть несколько парковочных зон для разных типов 
транспорта (краткосрочная, долгосрочная, VIP). 
Операторы контролируют занятость мест, прием оплаты и работу шлагбаумов.

\textbf{Создаваемые классы:} `ПарковочнаяЗона`, `ПарковочноеМесто`, `Автомобиль`, `ОператорПарковки`, `СлужбаПарковки`.

Для классов реализовать следующие простые методы, используя для хранения данных списки (`[]`) Python:
\begin{enumerate}
    \item \textbf{ПарковочнаяЗона:} добавить парковочное место, 
    удалить место, получить список мест в зоне, получить список всех автомобилей. Так же парковочной зоне 
    соответсвует стоимость часа стоянки.
    \item \textbf{ПарковочноеМесто:} занять место автомобилем, 
    освободить место, получить текущий статус (свободно/занято).
    \item \textbf{Автомобиль:} зафиксировать время въезда, время выезда + 
    рассчитать стоимость парковки (с учетом стоимости часа), получить историю.
    \item \textbf{ОператорПарковки:} класс, представляющий оператора, 
    имеющий идентификатор, смену, список контролируемых зон, ФИО.
    \item \textbf{СлужбаПарковки:} добавить новую парковочную зону, 
    нанять оператора, вывести информацию о зонах, местах, операторах, удалить зону, уволить оператора.
\end{enumerate}

\item \textbf{Описание ситуации:} Рассмотрим работу центра управления речным судоходством. 
Диспетчеры следят за движением судов по фарватеру, 
распределяют шлюзы, контролируют соблюдение графика движения 
и обеспечивают безопасность судоходства.

\textbf{Создаваемые классы:} `Судно`, `Шлюз`, `Фарватер`, `ДиспетчерСудоходства`, `ЦентрУправления`.

Для классов реализовать следующие простые методы, используя для хранения данных списки (`[]`) Python:
\begin{enumerate}
    \item \textbf{Судно:} начать движение по фарватеру, завершить движение, 
    получить текущее местоположение, зафиксировать прохождение шлюза.
    \item \textbf{Шлюз:} принять судно для шлюзования, 
    завершить шлюзование, получить текущий статус (свободен/занят).
    \item \textbf{Фарватер:} добавить участок фарватера, 
    удалить участок, получить список судов на фарватере.
    \item \textbf{ДиспетчерСудоходства:} класс, представляющий диспетчера, 
    имеющий идентификатор, смену, список контролируемых шлюзов, ФИО.
    \item \textbf{ЦентрУправления:} добавить шлюз в систему, 
    зарегистрировать судно, нанять диспетчера, 
    вывести информацию о шлюзах, фарватерах, судах, диспетчерах, удалить шлюз, уволить диспетчера.
\end{enumerate}

\item \textbf{Описание ситуации:} Рассмотрим работу службы технического контроля метрополитена. 
Инспекторы проверяют состояние путей, тоннелей, подвижного состава и оборудования станций. 
Дефекты фиксируются в системе для оперативного устранения ремонтными бригадами.

\textbf{Создаваемые классы:} `УчастокПути`, `ПодвижнойСостав`, `Инспектор`, `Дефект`, `СлужбаКонтроля`.

Для классов реализовать следующие простые методы, использующие для хранения данных списки (`[]`) Python:
\begin{enumerate}
    \item \textbf{УчастокПути:} добавить информацию о дефекте, 
    получить список неустраненных дефектов на участке.
    \item \textbf{ПодвижнойСостав:} добавить запись о техническом осмотре, 
    удалить ошибочную запись, отобразить историю осмотров.
    \item \textbf{Инспектор:} класс, представляющий инспектора, 
    имеющий идентификатор, квалификацию, список закрепленных участков, ФИО.
    \item \textbf{Дефект:} зафиксировать время обнаружения, 
    время устранения, получить статус устранения.
    \item \textbf{СлужбаКонтроля:} добавить участок пути в систему, 
    зарегистрировать подвижной состав, нанять инспектора, 
    вывести информацию об участках, составе, инспекторах, дефектах, удалить участок, уволить инспектора, 
    снять с эксплуатации подвижной состав.
\end{enumerate}

\item
\textbf{Описание ситуации:}
Рассмотрим работу центра управления умными светофорами на перекрестках. 
Умные светофоры адаптивно меняют режим работы в зависимости от транспортного потока, 
приоритизируя общественный транспорт и спецтранспорт. 
Система анализирует данные с датчиков и камер, оптимизируя пропускную способность перекрестков.

\textbf{Создаваемые классы:} УмныйСветофор, Перекресток, ДатчикТранспортногоПотока, ИнженерАТС, ЦентрУправленияСветофорами.

Для классов реализовать следующие простые методы, используя для хранения данных списки ([]) Python:
\begin{enumerate}
\item \textbf{УмныйСветофор:} изменить длительность фаз (красный/зеленый), 
перейти в аварийный режим, получить текущий режим работы.
\item \textbf{Перекресток:} добавить светофор к перекрестку, удалить светофор, 
получить список всех светофоров перекрестка.
\item \textbf{ДатчикТранспортногоПотока:} установить текущие данные о интенсивности движения, 
получить текущие показания, получить историю показаний.
\item \textbf{ИнженерАТС:} класс, представляющий инженера автоматизированной транспортной системы, 
имеющий идентификатор, квалификацию, список закрепленных перекрестков, ФИО.
\item \textbf{ЦентрУправленияСветофорами:} добавить новый перекресток в систему, 
установить умный светофор, нанять инженера, вывести информацию о перекрестках, светофорах, инженерах, 
удалить перекресток, уволить инженера, снять умный светофор.
\end{enumerate}

\item
\textbf{Описание ситуации:}
Рассмотрим работу монорельсовой транспортной системы. 
Монорельс движется по эстакаде, состоящей из станций и перегонов. 
Составы имеют фиксированное количество вагонов. 
Операторы управляют движением составов, следят за соблюдением графика и безопасностью пассажиров.

\textbf{Создаваемые классы:} СтанцияМонорельса, СоставМонорельса, ВагонМонорельса, ОператорСистемы, УправлениеМонорельсом.

Для классов реализовать следующие простые методы, используя для хранения данных списки ([]) Python:
\begin{enumerate}
\item \textbf{СтанцияМонорельса:} принять состав, отправить состав, 
получить список составов на станции.
\item \textbf{СоставМонорельса:} добавить вагон в состав (при техническом обслуживании), 
удалить вагон, получить список вагонов.
\item \textbf{ВагонМонорельса:} зафиксировать текущий пробег, 
провести техническое обслуживание, получить историю обслуживаний.
\item \textbf{ОператорСистемы:} класс, представляющий оператора, 
имеющий идентификатор, смену, список закрепленных станций, ФИО.
\item \textbf{УправлениеМонорельсом:} добавить новую станцию, ввести состав в эксплуатацию, 
нанять оператора, вывести информацию о станциях, составах, операторах, закрыть станцию на ремонт, 
списать состав, уволить оператора.
\end{enumerate}

\item
\textbf{Описание ситуации:}
Рассмотрим работу канатной дороги. Канатная дорога состоит из линий с опорами и кабинок, 
перемещающихся между станциями. 
Кабинки имеют ограниченную вместимость. Техники обслуживают механизмы и следят за безопасностью.

\textbf{Создаваемые классы:} ЛинияКанатнойДороги, Кабинка, СтанцияКанатнойДороги, Техник, УправлениеКанатнойДорогой.

Для классов реализовать следующие простые методы, используя для хранения данных списки ([]) Python:
\begin{enumerate}
\item \textbf{ЛинияКанатнойДороги:} добавить кабинку на линию, снять кабинку, 
получить список кабинок на линии.
\item \textbf{Кабинка:} запустить в движение, остановить для посадки/высадки, 
получить текущий статус (движется/стоит).
\item \textbf{СтанцияКанатнойДороги:} принять кабинку, отправить кабинку, 
получить список кабинок на станции.
\item \textbf{Техник:} класс, представляющий техника, имеющий идентификатор, 
квалификацию, список закрепленных линий, ФИО.
\item \textbf{УправлениеКанатнойДорогой:} добавить новую линию, 
ввести кабинку в эксплуатацию, нанять техника, вывести информацию о линиях, кабинках, техниках, 
закрыть линию на обслуживание, списать кабинку, уволить техника.
\end{enumerate}

\item
\textbf{Описание ситуации:}
Рассмотрим работу службы доставки с использованием дронов. 
Дроны осуществляют доставку небольших грузов между пунктами выдачи. 
Каждый дрон имеет грузоподъемность и дальность полета. 
Операторы управляют полетами дронов и обслуживают пункты выдачи.

\textbf{Создаваемые классы:} ПунктВыдачи, Дрон, Груз, ОператорДронов, СлужбаДоставки.

Для классов реализовать следующие простые методы, используя для хранения данных списки ([]) Python:
\begin{enumerate}
\item \textbf{ПунктВыдачи:} принять дрон с грузом, отправить дрон, получить список дронов в пункте.
\item \textbf{Дрон:} загрузить груз, выгрузить груз, 
начать полет, завершить полет, получить текущий статус (в полете/на земле).
\item \textbf{Груз:} зарегистрировать отправку, зарегистрировать доставку, 
получить историю перемещений.
\item \textbf{ОператорДронов:} класс, представляющий оператора, 
имеющий идентификатор, смену, список закрепленных пунктов выдачи, ФИО.
\item \textbf{СлужбаДоставки:} добавить новый пункт выдачи, 
ввести дрон в эксплуатацию, нанять оператора, вывести информацию о пунктах, 
дронах, операторах, закрыть пункт, списать дрон, уволить оператора.
\end{enumerate}

\end{enumerate}
\subsection{Семинар <<Конструкторы, наследование и полиморфизм. 1 часть>>  
(2 часа)}


В ходе работы решите 4 задачи. 
Предполагается, что пользователь класса не имеет права обращаться к свойствам напрямую 
(соблюдая принцип инкапсуляции), а должен использовать методы. 

Продемонстрируйте работоспособность всех методов (из задания) 
посредством создания запускаемых файлов, где осуществляется 
вызов методов для разных ситуаций 
(без ручного ввода, но с выводом результатов в консоль). 

Каждый класс должен сохраняться в отдельном исходном файле. 
Необходимо соблюдать все стандартные требования к качеству кода 
(отступы, именования переменных, классов, методов, 
проверка корректности входных данных).
Для каждого класса создайте отдельный запускаемый файл для проверки всех его методов 
(допускается использование других классов в этих тестах).

Все предлагаемые классы в заданиях упрощенные; для использования в production-окружении они требуют серьезной доработки. Суть задания — в отработке базовых навыков, а не в идеальном моделировании предложенных ситуаций.

Для сдачи работы будьте готовы пояснить или аналогично заданию модифицировать любую часть кода, а также ответить на вопросы:
\begin{enumerate}
    \item Что обозначает свойство наследования в парадигме ООП?
    \item Что обозначает свойство полиморфизма в парадигме ООП?
    \item Опишите реализацию наследования в Python
    \item Как создать конструктор в Python
    \item Как реализовать абстрактный класс в Python (и что это значит)
    \item Как реализовать абстрактные методы в Python (и что это значит)
\end{enumerate}

Если вы нашли в задачнике ошибки, опечатки и другие недостатки, то вы можете сделать pull-request. 

\textbf{Срок сдачи работы (начала сдачи):} через одно занятие после его выдачи. В последующие сроки оценка будет снижаться (при отсутствии оправдывающих документов).

\textbf{Задача 1}

\begin{enumerate}
    \item 

Напишите программу, которая создаёт класс
\texttt{Circle} с методами для вычисления площади
и длины окружности (периметра). Программа должна запрашивать у пользователя радиус
и выводить вычисленные площадь и длину окружности.

\subsection*{Инструкции:}
\begin{enumerate}
\item Создайте класс \texttt{Circle} с методом
\texttt{\_\_init\_\_}, который принимает радиус окружности в
качестве аргумента и сохраняет его в атрибуте \texttt{self.\_\_radius}.

\item Создайте метод \texttt{calculate\_circle\_area},
без аргументов, который вычисляет площадь круга по формуле:
\[
\pi \cdot \texttt{\_\_radius}^2
\]

\item Создайте метод \texttt{calculate\_circle\_perimeter} без аргументов,
который вычисляет длину окружности по формуле:
\[
2 \cdot \pi \cdot \texttt{\_\_radius}
\]

\item Напишите цикл, который повторяется 10 раз. В каждой итерации программа должна:
\begin{enumerate}
\item запрашивать у пользователя радиус окружности,
\item создавать экземпляр класса \texttt{Circle} с этим радиусом,
\item вычислять площадь и длину окружности с помощью соответствующих методов,
\item выводить результаты на экран.
\end{enumerate}
\end{enumerate}

\subsection*{Пример использования:}
\begin{verbatim}
radius = 3
circle = Circle(radius)
area = circle.calculate_circle_area()
perimeter = circle.calculate_circle_perimeter()
print(f"Площадь окружности равна: {area}")
print(f"Периметр окружности равен: {perimeter}")
\end{verbatim}

\textbf{Вывод:}
\begin{verbatim}
Площадь окружности равна: 28.274333882308138
Периметр окружности равен: 18.84955592153876
\end{verbatim}

\item 
Напишите программу, которая создаёт класс \texttt{Square} с методами для вычисления площади
и периметра. Программа должна запрашивать у пользователя длину стороны
и выводить вычисленные площадь и периметр.

\subsection*{Инструкции:}
\begin{enumerate}
\item Создайте класс \texttt{Square} с методом
\texttt{\_\_init\_\_}, который принимает длину стороны квадрата в
качестве аргумента и сохраняет её в атрибуте \texttt{self.\_\_side}.

\item Создайте метод \texttt{calculate\_area},
без аргументов, который вычисляет площадь квадрата по формуле:
\[
\texttt{\_\_side}^2
\]

\item Создайте метод \texttt{calculate\_perimeter} без аргументов,
который вычисляет периметр квадрата по формуле:
\[
4 \cdot \texttt{\_\_side}
\]

\item Напишите цикл, который повторяется 10 раз. В каждой итерации программа должна:
\begin{enumerate}
\item запрашивать у пользователя длину стороны квадрата,
\item создавать экземпляр класса \texttt{Square} с этой длиной,
\item вычислять площадь и периметр с помощью соответствующих методов,
\item выводить результаты на экран.
\end{enumerate}
\end{enumerate}

\subsection*{Пример использования:}
\begin{verbatim}
side = 5
square = Square(side)
area = square.calculate_area()
perimeter = square.calculate_perimeter()
print(f"Площадь квадрата равна: {area}")
print(f"Периметр квадрата равен: {perimeter}")
\end{verbatim}

\textbf{Вывод:}
\begin{verbatim}
Площадь квадрата равна: 25
Периметр квадрата равен: 20
\end{verbatim}

\item
Напишите программу, которая создаёт класс \texttt{Rectangle} с методами для вычисления площади
и периметра. Программа должна запрашивать у пользователя ширину прямоугольника
(при соотношении сторон 1:2) и выводить вычисленные площадь и периметр.

\subsection*{Инструкции:}
\begin{enumerate}
\item Создайте класс \texttt{Rectangle} с методом
\texttt{\_\_init\_\_}, который принимает ширину прямоугольника в
качестве аргумента и сохраняет её в атрибуте \texttt{self.\_\_width}.
Высота прямоугольника должна быть в два раза больше ширины.

\item Создайте метод \texttt{calculate\_area},
без аргументов, который вычисляет площадь прямоугольника по формуле:
\[
\texttt{\_\_width} \cdot (2 \cdot \texttt{\_\_width})
\]

\item Создайте метод \texttt{calculate\_perimeter} без аргументов,
который вычисляет периметр прямоугольника по формуле:
\[
2 \cdot (\texttt{\_\_width} + 2 \cdot \texttt{\_\_width})
\]

\item Напишите цикл, который повторяется 10 раз. В каждой итерации программа должна:
\begin{enumerate}
\item запрашивать у пользователя ширину прямоугольника,
\item создавать экземпляр класса \texttt{Rectangle} с этой шириной,
\item вычислять площадь и периметр с помощью соответствующих методов,
\item выводить результаты на экран.
\end{enumerate}
\end{enumerate}

\subsection*{Пример использования:}
\begin{verbatim}
width = 3
rectangle = Rectangle(width)
area = rectangle.calculate_area()
perimeter = rectangle.calculate_perimeter()
print(f"Площадь прямоугольника равна: {area}")
print(f"Периметр прямоугольника равен: {perimeter}")
\end{verbatim}

\textbf{Вывод:}
\begin{verbatim}
Площадь прямоугольника равна: 18
Периметр прямоугольника равен: 18
\end{verbatim}

\item
Напишите программу, которая создаёт класс \texttt{Triangle} с методами для вычисления площади
и периметра. Программа должна запрашивать у пользователя длину стороны
равностороннего треугольника и выводить вычисленные площадь и периметр.

\subsection*{Инструкции:}
\begin{enumerate}
\item Создайте класс \texttt{Triangle} с методом
\texttt{\_\_init\_\_}, который принимает длину стороны треугольника в
качестве аргумента и сохраняет её в атрибуте \texttt{self.\_\_side}.

\item Создайте метод \texttt{calculate\_area},
без аргументов, который вычисляет площадь равностороннего треугольника по формуле:
\[
\frac{\sqrt{3}}{4} \cdot \texttt{\_\_side}^2
\]

\item Создайте метод \texttt{calculate\_perimeter} без аргументов,
который вычисляет периметр треугольника по формуле:
\[
3 \cdot \texttt{\_\_side}
\]

\item Напишите цикл, который повторяется 10 раз. В каждой итерации программа должна:
\begin{enumerate}
\item запрашивать у пользователя длину стороны треугольника,
\item создавать экземпляр класса \texttt{Triangle} с этой длиной,
\item вычислять площадь и периметр с помощью соответствующих методов,
\item выводить результаты на экран.
\end{enumerate}
\end{enumerate}

\subsection*{Пример использования:}
\begin{verbatim}
side = 4
triangle = Triangle(side)
area = triangle.calculate_area()
perimeter = triangle.calculate_perimeter()
print(f"Площадь треугольника равна: {area}")
print(f"Периметр треугольника равен: {perimeter}")
\end{verbatim}

\textbf{Вывод:}
\begin{verbatim}
Площадь треугольника равна: 6.928203230275509
Периметр треугольника равен: 12
\end{verbatim}

\item
Напишите программу, которая создаёт класс \texttt{Sphere} с методами для вычисления площади поверхности
и объёма. Программа должна запрашивать у пользователя радиус сферы
и выводить вычисленные площадь поверхности и объём.

\subsection*{Инструкции:}
\begin{enumerate}
\item Создайте класс \texttt{Sphere} с методом
\texttt{\_\_init\_\_}, который принимает радиус сферы в
качестве аргумента и сохраняет его в атрибуте \texttt{self.\_\_radius}.

\item Создайте метод \texttt{calculate\_surface\_area},
без аргументов, который вычисляет площадь поверхности сферы по формуле:
\[
4 \cdot \pi \cdot \texttt{\_\_radius}^2
\]

\item Создайте метод \texttt{calculate\_volume} без аргументов,
который вычисляет объём сферы по формуле:
\[
\frac{4}{3} \cdot \pi \cdot \texttt{\_\_radius}^3
\]

\item Напишите цикл, который повторяется 10 раз. В каждой итерации программа должна:
\begin{enumerate}
\item запрашивать у пользователя радиус сферы,
\item создавать экземпляр класса \texttt{Sphere} с этим радиусом,
\item вычислять площадь поверхности и объём с помощью соответствующих методов,
\item выводить результаты на экран.
\end{enumerate}
\end{enumerate}

\subsection*{Пример использования:}
\begin{verbatim}
radius = 2
sphere = Sphere(radius)
surface_area = sphere.calculate_surface_area()
volume = sphere.calculate_volume()
print(f"Площадь поверхности сферы равна: {surface_area}")
print(f"Объём сферы равен: {volume}")
\end{verbatim}

\textbf{Вывод:}
\begin{verbatim}
Площадь поверхности сферы равна: 50.26548245743669
Объём сферы равен: 33.510321638291124
\end{verbatim}

\item
Напишите программу, которая создаёт класс \texttt{Cylinder} с методами для вычисления объёма
и площади боковой поверхности. Программа должна запрашивать у пользователя радиус основания
и выводить вычисленные объём и площадь боковой поверхности (высота цилиндра фиксирована и равна 5).

\subsection*{Инструкции:}
\begin{enumerate}
\item Создайте класс \texttt{Cylinder} с методом
\texttt{\_\_init\_\_}, который принимает радиус основания цилиндра в
качестве аргумента и сохраняет его в атрибуте \texttt{self.\_\_radius}.
Высота цилиндра фиксирована и равна 5.

\item Создайте метод \texttt{calculate\_volume},
без аргументов, который вычисляет объём цилиндра по формуле:
\[
\pi \cdot \texttt{\_\_radius}^2 \cdot 5
\]

\item Создайте метод \texttt{calculate\_lateral\_area} без аргументов,
который вычисляет площадь боковой поверхности цилиндра по формуле:
\[
2 \cdot \pi \cdot \texttt{\_\_radius} \cdot 5
\]

\item Напишите цикл, который повторяется 10 раз. В каждой итерации программа должна:
\begin{enumerate}
\item запрашивать у пользователя радиус основания цилиндра,
\item создавать экземпляр класса \texttt{Cylinder} с этим радиусом,
\item вычислять объём и площадь боковой поверхности с помощью соответствующих методов,
\item выводить результаты на экран.
\end{enumerate}
\end{enumerate}

\subsection*{Пример использования:}
\begin{verbatim}
radius = 3
cylinder = Cylinder(radius)
volume = cylinder.calculate_volume()
lateral_area = cylinder.calculate_lateral_area()
print(f"Объём цилиндра равен: {volume}")
print(f"Площадь боковой поверхности равна: {lateral_area}")
\end{verbatim}

\textbf{Вывод:}
\begin{verbatim}
Объём цилиндра равен: 141.3716694115407
Площадь боковой поверхности равна: 94.24777960769379
\end{verbatim}

\item
Напишите программу, которая создаёт класс \texttt{Cone} с методами для вычисления объёма
и площади боковой поверхности. Программа должна запрашивать у пользователя радиус основания
и выводить вычисленные объём и площадь боковой поверхности (высота конуса фиксирована и равна 10).

\subsection*{Инструкции:}
\begin{enumerate}
\item Создайте класс \texttt{Cone} с методом
\texttt{\_\_init\_\_}, который принимает радиус основания конуса в
качестве аргумента и сохраняет его в атрибуте \texttt{self.\_\_radius}.
Высота конуса фиксирована и равна 10.

\item Создайте метод \texttt{calculate\_volume},
без аргументов, который вычисляет объём конуса по формуле:
\[
\frac{1}{3} \cdot \pi \cdot \texttt{\_\_radius}^2 \cdot 10
\]

\item Создайте метод \texttt{calculate\_lateral\_area} без аргументов,
который вычисляет площадь боковой поверхности конуса по формуле:
\[
\pi \cdot \texttt{\_\_radius} \cdot \sqrt{\texttt{\_\_radius}^2 + 10^2}
\]

\item Напишите цикл, который повторяется 10 раз. В каждой итерации программа должна:
\begin{enumerate}
\item запрашивать у пользователя радиус основания конуса,
\item создавать экземпляр класса \texttt{Cone} с этим радиусом,
\item вычислять объём и площадь боковой поверхности с помощью соответствующих методов,
\item выводить результаты на экран.
\end{enumerate}
\end{enumerate}

\subsection*{Пример использования:}
\begin{verbatim}
radius = 3
cone = Cone(radius)
volume = cone.calculate_volume()
lateral_area = cone.calculate_lateral_area()
print(f"Объём конуса равен: {volume}")
print(f"Площадь боковой поверхности равна: {lateral_area}")
\end{verbatim}

\textbf{Вывод:}
\begin{verbatim}
Объём конуса равен: 94.24777960769379
Площадь боковой поверхности равна: 94.86832980505137
\end{verbatim}

\item
Напишите программу, которая создаёт класс \texttt{Cube} с методами для вычисления объёма
и площади полной поверхности. Программа должна запрашивать у пользователя длину ребра куба
и выводить вычисленные объём и площадь.

\subsection*{Инструкции:}
\begin{enumerate}
\item Создайте класс \texttt{Cube} с методом
\texttt{\_\_init\_\_}, который принимает длину ребра куба в
качестве аргумента и сохраняет её в атрибуте \texttt{self.\_\_side}.

\item Создайте метод \texttt{calculate\_volume},
без аргументов, который вычисляет объём куба по формуле:
\[
\texttt{\_\_side}^3
\]

\item Создайте метод \texttt{calculate\_surface\_area} без аргументов,
который вычисляет площадь полной поверхности куба по формуле:
\[
6 \cdot \texttt{\_\_side}^2
\]

\item Напишите цикл, который повторяется 10 раз. В каждой итерации программа должна:
\begin{enumerate}
\item запрашивать у пользователя длину ребра куба,
\item создавать экземпляр класса \texttt{Cube} с этой длиной,
\item вычислять объём и площадь полной поверхности с помощью соответствующих методов,
\item выводить результаты на экран.
\end{enumerate}
\end{enumerate}

\subsection*{Пример использования:}
\begin{verbatim}
side = 4
cube = Cube(side)
volume = cube.calculate_volume()
surface_area = cube.calculate_surface_area()
print(f"Объём куба равен: {volume}")
print(f"Площадь полной поверхности равна: {surface_area}")
\end{verbatim}

\textbf{Вывод:}
\begin{verbatim}
Объём куба равен: 64
Площадь полной поверхности равна: 96
\end{verbatim}

\item
Напишите программу, которая создаёт класс \texttt{Parallelogram} с методами для вычисления площади
и периметра. Программа должна запрашивать у пользователя длину основания параллелограмма
и выводить вычисленные площадь и периметр (высота параллелограмма фиксирована и равна 8, 
а боковая сторона равна 6).

\subsection*{Инструкции:}
\begin{enumerate}
\item Создайте класс \texttt{Parallelogram} с методом
\texttt{\_\_init\_\_}, который принимает длину основания параллелограмма в
качестве аргумента и сохраняет её в атрибуте \texttt{self.\_\_base}.
Высота параллелограмма фиксирована и равна 8, а боковая сторона равна 6.

\item Создайте метод \texttt{calculate\_area},
без аргументов, который вычисляет площадь параллелограмма по формуле:
\[
\texttt{\_\_base} \cdot 8
\]

\item Создайте метод \texttt{calculate\_perimeter} без аргументов,
который вычисляет периметр параллелограмма по формуле:
\[
2 \cdot (\texttt{\_\_base} + 6)
\]

\item Напишите цикл, который повторяется 10 раз. В каждой итерации программа должна:
\begin{enumerate}
\item запрашивать у пользователя длину основания параллелограмма,
\item создавать экземпляр класса \texttt{Parallelogram} с этой длиной,
\item вычислять площадь и периметр с помощью соответствующих методов,
\item выводить результаты на экран.
\end{enumerate}
\end{enumerate}

\subsection*{Пример использования:}
\begin{verbatim}
base = 5
parallelogram = Parallelogram(base)
area = parallelogram.calculate_area()
perimeter = parallelogram.calculate_perimeter()
print(f"Площадь параллелограмма равна: {area}")
print(f"Периметр параллелограмма равен: {perimeter}")
\end{verbatim}

\textbf{Вывод:}
\begin{verbatim}
Площадь параллелограмма равна: 40
Периметр параллелограмма равен: 22
\end{verbatim}


\item
Напишите программу, которая создаёт класс \texttt{Ellipse} с методами для вычисления площади
и приближённого значения периметра. Программа должна запрашивать у пользователя длину большой полуоси
и выводить вычисленные площадь и периметр (длина малой полуоси фиксирована и равна 3).

\subsection*{Инструкции:}
\begin{enumerate}
\item Создайте класс \texttt{Ellipse} с методом
\texttt{\_\_init\_\_}, который принимает длину большой полуоси эллипса в
качестве аргумента и сохраняет её в атрибуте \texttt{self.\_\_major\_axis}.
Длина малой полуоси фиксирована и равна 3.

\item Создайте метод \texttt{calculate\_area},
без аргументов, который вычисляет площадь эллипса по формуле:
\[
\pi \cdot \texttt{\_\_major\_axis} \cdot 3
\]

\item Создайте метод \texttt{calculate\_perimeter} без аргументов,
который вычисляет приближённое значение периметра эллипса по формуле Рамануджана:
\[
\pi \cdot \left(3(\texttt{\_\_major\_axis} + 3) - \sqrt{(3\texttt{\_\_major\_axis} + 3)(\texttt{\_\_major\_axis} + 9)}\right)
\]

\item Напишите цикл, который повторяется 10 раз. В каждой итерации программа должна:
\begin{enumerate}
\item запрашивать у пользователя длину большой полуоси эллипса,
\item создавать экземпляр класса \texttt{Ellipse} с этой длиной,
\item вычислять площадь и периметр с помощью соответствующих методов,
\item выводить результаты на экран.
\end{enumerate}
\end{enumerate}

\subsection*{Пример использования:}
\begin{verbatim}
major_axis = 5
ellipse = Ellipse(major_axis)
area = ellipse.calculate_area()
perimeter = ellipse.calculate_perimeter()
print(f"Площадь эллипса равна: {area}")
print(f"Периметр эллипса равен: {perimeter}")
\end{verbatim}

\textbf{Вывод:}
\begin{verbatim}
Площадь эллипса равна: 47.12388980384689
Периметр эллипса равен: 25.74488980384689
\end{verbatim}

\item
Напишите программу, которая создаёт класс \texttt{BankAccount} с методами для вычисления начисленных процентов
и суммы налога на доход. Программа должна запрашивать у пользователя начальный баланс счёта
и выводить вычисленные проценты и налог (процентная ставка фиксирована и равна 5\%, 
налоговая ставка на доход фиксирована и равна 13\%).

\subsection*{Инструкции:}
\begin{enumerate}
\item Создайте класс \texttt{BankAccount} с методом
\texttt{\_\_init\_\_}, который принимает начальный баланс счёта в
качестве аргумента и сохраняет его в атрибуте \texttt{self.\_\_balance}.

\item Создайте метод \texttt{calculate\_interest},
без аргументов, который вычисляет начисленные проценты по формуле:
\[
\texttt{\_\_balance} \cdot 0.05
\]

\item Создайте метод \texttt{calculate\_tax} без аргументов,
который вычисляет сумму налога на полученный доход (проценты) по формуле:
\[
(\texttt{\_\_balance} \cdot 0.05) \cdot 0.13
\]

\item Напишите цикл, который повторяется 10 раз. В каждой итерации программа должна:
\begin{enumerate}
\item запрашивать у пользователя начальный баланс счёта,
\item создавать экземпляр класса \texttt{BankAccount} с этим балансом,
\item вычислять начисленные проценты и сумму налога с помощью соответствующих методов,
\item выводить результаты на экран.
\end{enumerate}
\end{enumerate}

\subsection*{Пример использования:}
\begin{verbatim}
balance = 1000
account = BankAccount(balance)
interest = account.calculate_interest()
tax = account.calculate_tax()
print(f"Начисленные проценты: {interest}")
print(f"Сумма налога на доход: {tax}")
\end{verbatim}

\textbf{Вывод:}
\begin{verbatim}
Начисленные проценты: 50.0
Сумма налога на доход: 6.5
\end{verbatim}

\item
Напишите программу, которая создаёт класс \texttt{TemperatureConverter} с методами для преобразования температуры
из градусов Цельсия в Фаренгейты и Кельвины. Программа должна запрашивать у пользователя температуру в Цельсиях
и выводить преобразованные значения.

\subsection*{Инструкции:}
\begin{enumerate}
\item Создайте класс \texttt{TemperatureConverter} с методом
\texttt{\_\_init\_\_}, который принимает температуру в градусах Цельсия в
качестве аргумента и сохраняет её в атрибуте \texttt{self.\_\_celsius}.

\item Создайте метод \texttt{to\_fahrenheit},
без аргументов, который преобразует температуру в Фаренгейты по формуле:
\[
(\texttt{\_\_celsius} \times \frac{9}{5}) + 32
\]

\item Создайте метод \texttt{to\_kelvin} без аргументов,
который преобразует температуру в Кельвины по формуле:
\[
\texttt{\_\_celsius} + 273.15
\]

\item Напишите цикл, который повторяется 10 раз. В каждой итерации программа должна:
\begin{enumerate}
\item запрашивать у пользователя температуру в градусах Цельсия,
\item создавать экземпляр класса \texttt{TemperatureConverter} с этим значением,
\item вычислять температуру в Фаренгейтах и Кельвинах с помощью соответствующих методов,
\item выводить результаты на экран.
\end{enumerate}
\end{enumerate}

\subsection*{Пример использования:}
\begin{verbatim}
celsius = 25
converter = TemperatureConverter(celsius)
fahrenheit = converter.to_fahrenheit()
kelvin = converter.to_kelvin()
print(f"Температура в Фаренгейтах: {fahrenheit}")
print(f"Температура в Кельвинах: {kelvin}")
\end{verbatim}

\textbf{Вывод:}
\begin{verbatim}
Температура в Фаренгейтах: 77.0
Температура в Кельвинах: 298.15
\end{verbatim}

\item
Напишите программу, которая создаёт класс \texttt{DistanceConverter} с методами для преобразования расстояния
из метров в километры и мили. Программа должна запрашивать у пользователя расстояние в метрах
и выводить преобразованные значения.

\subsection*{Инструкции:}
\begin{enumerate}
\item Создайте класс \texttt{DistanceConverter} с методом
\texttt{\_\_init\_\_}, который принимает расстояние в метрах в
качестве аргумента и сохраняет его в атрибуте \texttt{self.\_\_meters}.

\item Создайте метод \texttt{to\_kilometers},
без аргументов, который преобразует расстояние в километры по формуле:
\[
\texttt{\_\_meters} \div 1000
\]

\item Создайте метод \texttt{to\_miles} без аргументов,
который преобразует расстояние в мили по формуле:
\[
\texttt{\_\_meters} \div 1609.344
\]

\item Напишите цикл, который повторяется 10 раз. В каждой итерации программа должна:
\begin{enumerate}
\item запрашивать у пользователя расстояние в метрах,
\item создавать экземпляр класса \texttt{DistanceConverter} с этим значением,
\item вычислять расстояние в километрах и милях с помощью соответствующих методов,
\item выводить результаты на экран.
\end{enumerate}
\end{enumerate}

\subsection*{Пример использования:}
\begin{verbatim}
meters = 1609.344
converter = DistanceConverter(meters)
kilometers = converter.to_kilometers()
miles = converter.to_miles()
print(f"Расстояние в километрах: {kilometers}")
print(f"Расстояние в милях: {miles}")
\end{verbatim}

\textbf{Вывод:}
\begin{verbatim}
Расстояние в километрах: 1.609344
Расстояние в милях: 1.0
\end{verbatim}

\item
Напишите программу, которая создаёт класс \texttt{WeightConverter} с методами для преобразования массы
из килограммов в граммы и фунты. Программа должна запрашивать у пользователя массу в килограммах
и выводить преобразованные значения.

\subsection*{Инструкции:}
\begin{enumerate}
\item Создайте класс \texttt{WeightConverter} с методом
\texttt{\_\_init\_\_}, который принимает массу в килограммах в
качестве аргумента и сохраняет её в атрибуте \texttt{self.\_\_kg}.

\item Создайте метод \texttt{to\_grams},
без аргументов, который преобразует массу в граммы по формуле:
\[
\texttt{\_\_kg} \times 1000
\]

\item Создайте метод \texttt{to\_pounds} без аргументов,
который преобразует массу в фунты по формуле:
\[
\texttt{\_\_kg} \times 2.20462
\]

\item Напишите цикл, который повторяется 10 раз. В каждой итерации программа должна:
\begin{enumerate}
\item запрашивать у пользователя массу в килограммах,
\item создавать экземпляр класса \texttt{WeightConverter} с этим значением,
\item вычислять массу в граммах и фунтах с помощью соответствующих методов,
\item выводить результаты на экран.
\end{enumerate}
\end{enumerate}

\subsection*{Пример использования:}
\begin{verbatim}
kg = 2.5
converter = WeightConverter(kg)
grams = converter.to_grams()
pounds = converter.to_pounds()
print(f"Масса в граммах: {grams}")
print(f"Масса в фунтах: {pounds}")
\end{verbatim}

\textbf{Вывод:}
\begin{verbatim}
Масса в граммах: 2500.0
Масса в фунтах: 5.51155
\end{verbatim}

\item 


Напишите программу, которая создаёт класс \texttt{TimeConverter} с методами для преобразования времени
из секунд в минуты и часы. Программа должна запрашивать у пользователя время в секундах
и выводить преобразованные значения.

\subsection*{Инструкции:}
\begin{enumerate}
\item Создайте класс \texttt{TimeConverter} с методом
\texttt{\_\_init\_\_}, который принимает время в секундах в
качестве аргумента и сохраняет его в атрибуте \texttt{self.\_\_seconds}.

\item Создайте метод \texttt{to\_minutes},
без аргументов, который преобразует время в минуты по формуле:
\[
\texttt{\_\_seconds} \div 60
\]

\item Создайте метод \texttt{to\_hours} без аргументов,
который преобразует время в часы по формуле:
\[
\texttt{\_\_seconds} \div 3600
\]

\item Напишите цикл, который повторяется 10 раз. В каждой итерации программа должна:
\begin{enumerate}
\item запрашивать у пользователя время в секундах,
\item создавать экземпляр класса \texttt{TimeConverter} с этим значением,
\item вычислять время в минутах и часах с помощью соответствующих методов,
\item выводить результаты на экран.
\end{enumerate}
\end{enumerate}

\subsection*{Пример использования:}
\begin{verbatim}
seconds = 7200
converter = TimeConverter(seconds)
minutes = converter.to_minutes()
hours = converter.to_hours()
print(f"Время в минутах: {minutes}")
print(f"Время в часах: {hours}")
\end{verbatim}

\textbf{Вывод:}
\begin{verbatim}
Время в минутах: 120.0
Время в часах: 2.0
\end{verbatim}

\item


Напишите программу, которая создаёт класс \texttt{SpeedConverter} с методами для преобразования скорости
из километров в час в метры в секунду и мили в час. Программа должна запрашивать у пользователя скорость в км/ч
и выводить преобразованные значения.

\subsection*{Инструкции:}
\begin{enumerate}
\item Создайте класс \texttt{SpeedConverter} с методом
\texttt{\_\_init\_\_}, который принимает скорость в км/ч в
качестве аргумента и сохраняет её в атрибуте \texttt{self.\_\_kmh}.

\item Создайте метод \texttt{to\_ms},
без аргументов, который преобразует скорость в м/с по формуле:
\[
\texttt{\_\_kmh} \times \frac{1000}{3600}
\]

\item Создайте метод \texttt{to\_mph} без аргументов,
который преобразует скорость в мили/ч по формуле:
\[
\texttt{\_\_kmh} \div 1.60934
\]

\item Напишите цикл, который повторяется 10 раз. В каждой итерации программа должна:
\begin{enumerate}
\item запрашивать у пользователя скорость в км/ч,
\item создавать экземпляр класса \texttt{SpeedConverter} с этим значением,
\item вычислять скорость в м/с и милях/ч с помощью соответствующих методов,
\item выводить результаты на экран.
\end{enumerate}
\end{enumerate}

\subsection*{Пример использования:}
\begin{verbatim}
kmh = 100
converter = SpeedConverter(kmh)
ms = converter.to_ms()
mph = converter.to_mph()
print(f"Скорость в м/с: {ms}")
print(f"Скорость в милях/ч: {mph}")
\end{verbatim}

\textbf{Вывод:}
\begin{verbatim}
Скорость в м/с: 27.77777777777778
Скорость в милях/ч: 62.13727366498068
\end{verbatim}

\item 

Напишите программу, которая создаёт класс \texttt{AreaConverter} с методами для преобразования площади
из квадратных метров в гектары и акры. Программа должна запрашивать у пользователя площадь в м²
и выводить преобразованные значения.

\subsection*{Инструкции:}
\begin{enumerate}
\item Создайте класс \texttt{AreaConverter} с методом
\texttt{\_\_init\_\_}, который принимает площадь в м² в
качестве аргумента и сохраняет её в атрибуте \texttt{self.\_\_sq\_meters}.

\item Создайте метод \texttt{to\_hectares},
без аргументов, который преобразует площадь в гектары по формуле:
\[
\texttt{\_\_sq\_meters} \div 10000
\]

\item Создайте метод \texttt{to\_acres} без аргументов,
который преобразует площадь в акры по формуле:
\[
\texttt{\_\_sq\_meters} \div 4046.86
\]

\item Напишите цикл, который повторяется 10 раз. В каждой итерации программа должна:
\begin{enumerate}
\item запрашивать у пользователя площадь в м²,
\item создавать экземпляр класса \texttt{AreaConverter} с этим значением,
\item вычислять площадь в гектарах и акрах с помощью соответствующих методов,
\item выводить результаты на экран.
\end{enumerate}
\end{enumerate}

\subsection*{Пример использования:}
\begin{verbatim}
sq_meters = 10000
converter = AreaConverter(sq_meters)
hectares = converter.to_hectares()
acres = converter.to_acres()
print(f"Площадь в гектарах: {hectares}")
print(f"Площадь в акрах: {acres}")
\end{verbatim}

\textbf{Вывод:}
\begin{verbatim}
Площадь в гектары: 1.0
Площадь в акрах: 2.4710514233241505
\end{verbatim}

\item 

Напишите программу, которая создаёт класс \texttt{VolumeConverter} с методами для преобразования объёма
из литров в галлоны и кубические метры. Программа должна запрашивать у пользователя объём в литрах
и выводить преобразованные значения.

\subsection*{Инструкции:}
\begin{enumerate}
\item Создайте класс \texttt{VolumeConverter} с методом
\texttt{\_\_init\_\_}, который принимает объём в литрах в
качестве аргумента и сохраняет его в атрибуте \texttt{self.\_\_liters}.

\item Создайте метод \texttt{to\_gallons},
без аргументов, который преобразует объём в галлоны по формуле:
\[
\texttt{\_\_liters} \div 3.78541
\]

\item Создайте метод \texttt{to\_cubic\_meters} без аргументов,
который преобразует объём в кубические метры по формуле:
\[
\texttt{\_\_liters} \div 1000
\]

\item Напишите цикл, который повторяется 10 раз. В каждой итерации программа должна:
\begin{enumerate}
\item запрашивать у пользователя объём в литрах,
\item создавать экземпляр класса \texttt{VolumeConverter} с этим значением,
\item вычислять объём в галлонах и кубических метрах с помощью соответствующих методов,
\item выводить результаты на экран.
\end{enumerate}
\end{enumerate}

\subsection*{Пример использования:}
\begin{verbatim}
liters = 10
converter = VolumeConverter(liters)
gallons = converter.to_gallons()
cubic_meters = converter.to_cubic_meters()
print(f"Объём в галлонах: {gallons}")
print(f"Объём в кубических метрах: {cubic_meters}")
\end{verbatim}

\textbf{Вывод:}
\begin{verbatim}
Объём в галлонах: 2.641720523581484
Объём в кубических метрах: 0.01
\end{verbatim}

\item

Напишите программу, которая создаёт класс \texttt{EnergyConverter} с методами для преобразования энергии
из джоулей в калории и киловатт-часы. Программа должна запрашивать у пользователя энергию в джоулях
и выводить преобразованные значения.

\subsection*{Инструкции:}
\begin{enumerate}
\item Создайте класс \texttt{EnergyConverter} с методом
\texttt{\_\_init\_\_}, который принимает энергию в джоулях в
качестве аргумента и сохраняет её в атрибуте \texttt{self.\_\_joules}.

\item Создайте метод \texttt{to\_calories},
без аргументов, который преобразует энергию в калории по формуле:
\[
\texttt{\_\_joules} \div 4.184
\]

\item Создайте метод \texttt{to\_kwh} без аргументов,
который преобразует энергию в киловатт-часы по формуле:
\[
\texttt{\_\_joules} \div 3.6 \times 10^6
\]

\item Напишите цикл, который повторяется 10 раз. В каждой итерации программа должна:
\begin{enumerate}
\item запрашивать у пользователя энергию в джоулях,
\item создавать экземпляр класса \texttt{EnergyConverter} с этим значением,
\item вычислять энергию в калориях и киловатт-часах с помощью соответствующих методов,
\item выводить результаты на экран.
\end{enumerate}
\end{enumerate}

\subsection*{Пример использования:}
\begin{verbatim}
joules = 10000
converter = EnergyConverter(joules)
calories = converter.to_calories()
kwh = converter.to_kwh()
print(f"Энергия в калориях: {calories}")
print(f"Энергия в киловатт-часах: {kwh}")
\end{verbatim}

\textbf{Вывод:}
\begin{verbatim}
Энергия в калориях: 2390.057361376673
Энергия в киловатт-часах: 0.002777777777777778
\end{verbatim}

\item 

Напишите программу, которая создаёт класс \texttt{PowerConverter} с методами для преобразования мощности
из ватт в лошадиные силы и киловатты. Программа должна запрашивать у пользователя мощность в ваттах
и выводить преобразованные значения.

\subsection*{Инструкции:}
\begin{enumerate}
\item Создайте класс \texttt{PowerConverter} с методом
\texttt{\_\_init\_\_}, который принимает мощность в ваттах в
качестве аргумента и сохраняет её в атрибуте \texttt{self.\_\_watts}.

\item Создайте метод \texttt{to\_horsepower},
без аргументов, который преобразует мощность в лошадиные силы по формуле:
\[
\texttt{\_\_watts} \div 745.7
\]

\item Создайте метод \texttt{to\_kilowatts} без аргументов,
который преобразует мощность в киловатты по формуле:
\[
\texttt{\_\_watts} \div 1000
\]

\item Напишите цикл, который повторяется 10 раз. В каждой итерации программа должна:
\begin{enumerate}
\item запрашивать у пользователя мощность в ваттах,
\item создавать экземпляр класса \texttt{PowerConverter} с этим значением,
\item вычислять мощность в л.с. и киловаттах с помощью соответствующих методов,
\item выводить результаты на экран.
\end{enumerate}
\end{enumerate}

\subsection*{Пример использования:}
\begin{verbatim}
watts = 1000
converter = PowerConverter(watts)
horsepower = converter.to_horsepower()
kilowatts = converter.to_kilowatts()
print(f"Мощность в л.с.: {horsepower}")
print(f"Мощность в киловаттах: {kilowatts}")
\end{verbatim}

\textbf{Вывод:}
\begin{verbatim}
Мощность в л.с.: 1.3410220903956017
Мощность в киловаттах: 1.0
\end{verbatim}


\item

Напишите программу, которая создаёт класс \texttt{PressureConverter} с методами для преобразования давления
из паскалей в атмосферы и бары. Программа должна запрашивать у пользователя давление в паскалях
и выводить преобразованные значения.

\subsection*{Инструкции:}
\begin{enumerate}
\item Создайте класс \texttt{PressureConverter} с методом
\texttt{\_\_init\_\_}, который принимает давление в паскалях в
качестве аргумента и сохраняет его в атрибуте \texttt{self.\_\_pascals}.

\item Создайте метод \texttt{to\_atm},
без аргументов, который преобразует давление в атмосферы по формуле:
\[
\texttt{\_\_pascals} \div 101325
\]

\item Создайте метод \texttt{to\_bar} без аргументов,
который преобразует давление в бары по формуле:
\[
\texttt{\_\_pascals} \div 100000
\]

\item Напишите цикл, который повторяется 10 раз. В каждой итерации программа должна:
\begin{enumerate}
\item запрашивать у пользователя давление в паскалях,
\item создавать экземпляр класса \texttt{PressureConverter} с этим значением,
\item вычислять давление в атмосферах и барах с помощью соответствующих методов,
\item выводить результаты на экран.
\end{enumerate}
\end{enumerate}

\subsection*{Пример использования:}
\begin{verbatim}
pascals = 101325
converter = PressureConverter(pascals)
atm = converter.to_atm()
bar = converter.to_bar()
print(f"Давление в атмосферах: {atm}")
print(f"Давление в барах: {bar}")
\end{verbatim}

\textbf{Вывод:}
\begin{verbatim}
Давление в атмосферах: 1.0
Давление в барах: 1.01325
\end{verbatim}

\item 

Напишите программу, которая создаёт класс \texttt{ForceConverter} с методами для преобразования силы
из ньютонов в дины и фунты-силы. Программа должна запрашивать у пользователя силу в ньютонах
и выводить преобразованные значения.

\subsection*{Инструкции:}
\begin{enumerate}
\item Создайте класс \texttt{ForceConverter} с методом
\texttt{\_\_init\_\_}, который принимает силу в ньютонах в
качестве аргумента и сохраняет её в атрибуте \texttt{self.\_\_newtons}.

\item Создайте метод \texttt{to\_dyne},
без аргументов, который преобразует силу в дины по формуле:
\[
\texttt{\_\_newtons} \times 100000
\]

\item Создайте метод \texttt{to\_pound\_force} без аргументов,
который преобразует силу в фунты-силы по формуле:
\[
\texttt{\_\_newtons} \div 4.44822
\]

\item Напишите цикл, который повторяется 10 раз. В каждой итерации программа должна:
\begin{enumerate}
\item запрашивать у пользователя силу в ньютонах,
\item создавать экземпляр класса \texttt{ForceConverter} с этим значением,
\item вычислять силу в динах и фунтах-силы с помощью соответствующих методов,
\item выводить результаты на экран.
\end{enumerate}
\end{enumerate}

\subsection*{Пример использования:}
\begin{verbatim}
newtons = 10
converter = ForceConverter(newtons)
dyne = converter.to_dyne()
pound_force = converter.to_pound_force()
print(f"Сила в динах: {dyne}")
print(f"Сила в фунтах-силы: {pound_force}")
\end{verbatim}

\textbf{Вывод:}
\begin{verbatim}
Сила в динах: 1000000.0
Сила в фунтах-силы: 2.248089430997145
\end{verbatim}

\item 

\subsection*{Задание: Конвертер силы}
Напишите программу, которая создаёт класс \texttt{ForceConverter} с методами для преобразования силы
из ньютонов в дины и фунты-силы. Программа должна запрашивать у пользователя силу в ньютонах
и выводить преобразованные значения.

\subsection*{Инструкции:}
\begin{enumerate}
\item Создайте класс \texttt{ForceConverter} с методом
\texttt{\_\_init\_\_}, который принимает силу в ньютонах в
качестве аргумента и сохраняет её в атрибуте \texttt{self.\_\_newtons}.

\item Создайте метод \texttt{to\_dyne},
без аргументов, который преобразует силу в дины по формуле:
\[
\texttt{\_\_newtons} \times 100000
\]

\item Создайте метод \texttt{to\_pound\_force} без аргументов,
который преобразует силу в фунты-силы по формуле:
\[
\texttt{\_\_newtons} \div 4.44822
\]

\item Напишите цикл, который повторяется 10 раз. В каждой итерации программа должна:
\begin{enumerate}
\item запрашивать у пользователя силу в ньютонах,
\item создавать экземпляр класса \texttt{ForceConverter} с этим значением,
\item вычислять силу в динах и фунтах-силы с помощью соответствующих методов,
\item выводить результаты на экран.
\end{enumerate}
\end{enumerate}

\subsection*{Пример использования:}
\begin{verbatim}
newtons = 10
converter = ForceConverter(newtons)
dyne = converter.to_dyne()
pound_force = converter.to_pound_force()
print(f"Сила в динах: {dyne}")
print(f"Сила в фунтах-силы: {pound_force}")
\end{verbatim}

\textbf{Вывод:}
\begin{verbatim}
Сила в динах: 1000000.0
Сила в фунтах-силы: 2.248089430997145
\end{verbatim}

\item

Напишите программу, которая создаёт класс \texttt{ResistanceConverter} с методами для преобразования электрического сопротивления
из омов в килоомы и мегаомы. Программа должна запрашивать у пользователя сопротивление в омах
и выводить преобразованные значения.

\subsection*{Инструкции:}
\begin{enumerate}
\item Создайте класс \texttt{ResistanceConverter} с методом
\texttt{\_\_init\_\_}, который принимает сопротивление в омах в
качестве аргумента и сохраняет его в атрибуте \texttt{self.\_\_ohms}.

\item Создайте метод \texttt{to\_kiloohms},
без аргументов, который преобразует сопротивление в килоомы по формуле:
\[
\texttt{\_\_ohms} \div 1000
\]

\item Создайте метод \texttt{to\_megaohms} без аргументов,
который преобразует сопротивление в мегаомы по формуле:
\[
\texttt{\_\_ohms} \div 1000000
\]

\item Напишите цикл, который повторяется 10 раз. В каждой итерации программа должна:
\begin{enumerate}
\item запрашивать у пользователя сопротивление в омах,
\item создавать экземпляр класса \texttt{ResistanceConverter} с этим значением,
\item вычислять сопротивление в килоомах и мегаомах с помощью соответствующих методов,
\item выводить результаты на экран.
\end{enumerate}
\end{enumerate}

\subsection*{Пример использования:}
\begin{verbatim}
ohms = 10000
converter = ResistanceConverter(ohms)
kiloohms = converter.to_kiloohms()
megaohms = converter.to_megaohms()
print(f"Сопротивление в килоомах: {kiloohms}")
print(f"Сопротивление в мегаомах: {megaohms}")
\end{verbatim}

\textbf{Вывод:}
\begin{verbatim}
Сопротивление в килоомах: 10.0
Сопротивление в мегаомах: 0.01
\end{verbatim}

\item 


\section*{Дополнительные задания}

\item
Напишите программу, которая создаёт класс \texttt{Pentagon} с методами для вычисления площади
и периметра правильного пятиугольника. Программа должна запрашивать у пользователя длину сторону
и выводить вычисленные площадь и периметр.

\subsection*{Инструкции:}
\begin{enumerate}
\item Создайте класс \texttt{Pentagon} с методом
\texttt{\_\_init\_\_}, который принимает длину стороны пятиугольника в
качестве аргумента и сохраняет её в атрибуте \texttt{self.\_\_side}.

\item Создайте метод \texttt{calculate\_area},
без аргументов, который вычисляет площадь правильного пятиугольника по формуле:
\[
\frac{1}{4} \sqrt{5(5 + 2\sqrt{5})} \cdot \texttt{\_\_side}^2
\]

\item Создайте метод \texttt{calculate\_perimeter} без аргументов,
который вычисляет периметр пятиугольника по формуле:
\[
5 \cdot \texttt{\_\_side}
\]

\item Напишите цикл, который повторяется 10 раз. В каждой итерации программа должна:
\begin{enumerate}
\item запрашивать у пользователя длину стороны пятиугольника,
\item создавать экземпляр класса \texttt{Pentagon} с этой длиной,
\item вычислять площадь и периметр с помощью соответствующих методов,
\item выводить результаты на экран.
\end{enumerate}
\end{enumerate}

\subsection*{Пример использования:}
\begin{verbatim}
side = 5
pentagon = Pentagon(side)
area = pentagon.calculate_area()
perimeter = pentagon.calculate_perimeter()
print(f"Площадь пятиугольника: {area}")
print(f"Периметр пятиугольника: {perimeter}")
\end{verbatim}

\textbf{Вывод:}
\begin{verbatim}
Площадь пятиугольника: 43.01193501472417
Периметр пятиугольника: 25
\end{verbatim}

\item
Напишите программу, которая создаёт класс \texttt{Hexagon} с методами для вычисления площади
и периметра правильного шестиугольника. Программа должна запрашивать у пользователя длину стороны
и выводить вычисленные площадь и периметр.

\subsection*{Инструкции:}
\begin{enumerate}
\item Создайте класс \texttt{Hexagon} с методом
\texttt{\_\_init\_\_}, который принимает длину стороны шестиугольника в
качестве аргумента и сохраняет её в атрибуте \texttt{self.\_\_side}.

\item Создайте метод \texttt{calculate\_area},
без аргументов, который вычисляет площадь правильного шестиугольника по формуле:
\[
\frac{3\sqrt{3}}{2} \cdot \texttt{\_\_side}^2
\]

\item Создайте метод \texttt{calculate\_perimeter} без аргументов,
который вычисляет периметр шестиугольника по формуле:
\[
6 \cdot \texttt{\_\_side}
\]

\item Напишите цикл, который повторяется 10 раз. В каждой итерации программа должна:
\begin{enumerate}
\item запрашивать у пользователя длину стороны шестиугольника,
\item создавать экземпляр класса \texttt{Hexagon} с этой длиной,
\item вычислять площадь и периметр с помощью соответствующих методов,
\item выводить результаты на экран.
\end{enumerate}
\end{enumerate}

\subsection*{Пример использования:}
\begin{verbatim}
side = 4
hexagon = Hexagon(side)
area = hexagon.calculate_area()
perimeter = hexagon.calculate_perimeter()
print(f"Площадь шестиугольника: {area}")
print(f"Периметр шестиугольника: {perimeter}")
\end{verbatim}

\textbf{Вывод:}
\begin{verbatim}
Площадь шестиугольника: 41.569219381653056
Периметр шестиугольника: 24
\end{verbatim}

\item
Напишите программу, которая создаёт класс \texttt{AngleConverter} с методами для преобразования углов
из градусов в радианы и грады. Программа должна запрашивать у пользователя угол в градусах
и выводить преобразованные значения.

\subsection*{Инструкции:}
\begin{enumerate}
\item Создайте класс \texttt{AngleConverter} с методом
\texttt{\_\_init\_\_}, который принимает угол в градусах в
качестве аргумента и сохраняет его в атрибуте \texttt{self.\_\_degrees}.

\item Создайте метод \texttt{to\_radians},
без аргументов, который преобразует угол в радианы по формуле:
\[
\texttt{\_\_degrees} \times \frac{\pi}{180}
\]

\item Создайте метод \texttt{to\_gradians} без аргументов,
который преобразует угол в грады по формуле:
\[
\texttt{\_\_degrees} \times \frac{10}{9}
\]

\item Напишите цикл, который повторяется 10 раз. В каждой итерации программа должна:
\begin{enumerate}
\item запрашивать у пользователя угол в градусах,
\item создавать экземпляр класса \texttt{AngleConverter} с этим значением,
\item вычислять угол в радианах и градах с помощью соответствующих методов,
\item выводить результаты на экран.
\end{enumerate}
\end{enumerate}

\subsection*{Пример использования:}
\begin{verbatim}
degrees = 90
converter = AngleConverter(degrees)
radians = converter.to_radians()
gradians = converter.to_gradians()
print(f"Угол в радианах: {radians}")
print(f"Угол в градах: {gradians}")
\end{verbatim}

\textbf{Вывод:}
\begin{verbatim}
Угол в радианах: 1.5707963267948966
Угол в градах: 100.0
\end{verbatim}

\item
Напишите программу, которая создаёт класс \texttt{Tetrahedron} с методами для вычисления объёма
и площади поверхности правильного тетраэдра. Программа должна запрашивать у пользователя длину ребра
и выводить вычисленные объём и площадь поверхности.

\subsection*{Инструкции:}
\begin{enumerate}
\item Создайте класс \texttt{Tetrahedron} с методом
\texttt{\_\_init\_\_}, который принимает длину ребра тетраэдра в
качестве аргумента и сохраняет её в атрибуте \texttt{self.\_\_edge}.

\item Создайте метод \texttt{calculate\_volume},
без аргументов, который вычисляет объём тетраэдра по формуле:
\[
\frac{\texttt{\_\_edge}^3}{6\sqrt{2}}
\]

\item Создайте метод \texttt{calculate\_surface\_area} без аргументов,
который вычисляет площадь поверхности тетраэдра по формуле:
\[
\sqrt{3} \cdot \texttt{\_\_edge}^2
\]

\item Напишите цикл, который повторяется 10 раз. В каждой итерации программа должна:
\begin{enumerate}
\item запрашивать у пользователя длину ребра тетраэдра,
\item создавать экземпляр класса \texttt{Tetrahedron} с этой длиной,
\item вычислять объём и площадь поверхности с помощью соответствующих методов,
\item выводить результаты на экран.
\end{enumerate}
\end{enumerate}

\subsection*{Пример использования:}
\begin{verbatim}
edge = 3
tetrahedron = Tetrahedron(edge)
volume = tetrahedron.calculate_volume()
surface_area = tetrahedron.calculate_surface_area()
print(f"Объём тетраэдра: {volume}")
print(f"Площадь поверхности: {surface_area}")
\end{verbatim}

\textbf{Вывод:}
\begin{verbatim}
Объём тетраэдра: 3.181980515339464
Площадь поверхности: 15.588457268119896
\end{verbatim}

\item
Напишите программу, которая создаёт класс \texttt{CubicMeterConverter} с методами для преобразования объёма
из кубических метров в литры и кубические футы. Программа должна запрашивать у пользователя объём в кубометрах
и выводить преобразованные значения.

\subsection*{Инструкции:}
\begin{enumerate}
\item Создайте класс \texttt{CubicMeterConverter} с методом
\texttt{\_\_init\_\_}, который принимает объём в кубических метрах в
качестве аргумента и сохраняет его в атрибуте \texttt{self.\_\_cubic\_meters}.

\item Создайте метод \texttt{to\_liters},
без аргументов, который преобразует объём в литры по формуле:
\[
\texttt{\_\_cubic\_meters} \times 1000
\]

\item Создайте метод \texttt{\_\_cubic\_feet} без аргументов,
который преобразует объём в кубические футы по формуле:
\[
\texttt{\_\_cubic\_meters} \times 35.3147
\]

\item Напишите цикл, который повторяется 10 раз. В каждой итерации программа должна:
\begin{enumerate}
\item запрашивать у пользователя объём в кубических метрах,
\item создавать экземпляр класса \texttt{CubicMeterConverter} с этим значением,
\item вычислять объём в литрах и кубических футах с помощью соответствующих методов,
\item выводить результаты на экран.
\end{enumerate}
\end{enumerate}

\subsection*{Пример использования:}
\begin{verbatim}
cubic_meters = 2
converter = CubicMeterConverter(cubic_meters)
liters = converter.to_liters()
cubic_feet = converter.to_cubic_feet()
print(f"Объём в литрах: {liters}")
print(f"Объём в кубических футах: {cubic_feet}")
\end{verbatim}

\textbf{Вывод:}
\begin{verbatim}
Объём в литрах: 2000.0
Объём в кубических футах: 70.6294
\end{verbatim}

\item
Напишите программу, которая создаёт класс \texttt{RightTriangle} с методами для вычисления гипотенузы
и площади прямоугольного треугольника. Программа должна запрашивать у пользователя длину одного катета
(второй катет фиксирован и равен 4) и выводить вычисленные гипотенузу и площадь.

\subsection*{Инструкции:}
\begin{enumerate}
\item Создайте класс \texttt{RightTriangle} с методом
\texttt{\_\_init\_\_}, который принимает длину первого катета в
качестве аргумента и сохраняет его в атрибуте \texttt{self.\_\_cathetus}.
Второй катет фиксирован и равен 4.

\item Создайте метод \texttt{calculate\_hypotenuse},
без аргументов, который вычисляет гипотенузу по формуле:
\[
\sqrt{\texttt{\_\_cathetus}^2 + 4^2}
\]

\item Создайте метод \texttt{calculate\_area} без аргументов,
который вычисляет площадь треугольника по формуле:
\[
\frac{\texttt{\_\_cathetus} \times 4}{2}
\]

\item Напишите цикл, который повторяется 10 раз. В каждой итерации программа должна:
\begin{enumerate}
\item запрашивать у пользователя длину катета,
\item создавать экземпляр класса \texttt{RightTriangle} с этой длиной,
\item вычислять гипотенузу и площадь с помощью соответствующих методов,
\item выводить результаты на экран.
\end{enumerate}
\end{enumerate}

\subsection*{Пример использования:}
\begin{verbatim}
cathetus = 3
triangle = RightTriangle(cathetus)
hypotenuse = triangle.calculate_hypotenuse()
area = triangle.calculate_area()
print(f"Гипотенуза: {hypotenuse}")
print(f"Площадь: {area}")
\end{verbatim}

\textbf{Вывод:}
\begin{verbatim}
Гипотенуза: 5.0
Площадь: 6.0
\end{verbatim}

\end{enumerate}

\textbf{Задача 2}

\begin{enumerate}
    \item

Написать программу, которая создаёт класс \texttt{LeapYearChecker} 
для определения високосного года. В классе должен быть статический метод
\texttt{is\_leap\_year} и возвращать \texttt{True}, если год високосный, 
и \texttt{False} в противном случае. 
Программа также должна использовать цикл для проверки каждого года от 
2000 до 2099 и вывода результата на экран.

\subsection*{Инструкции:}
\begin{enumerate}
    \item Создайте класс \texttt{LeapYearChecker}.
    \item Создайте \textbf{статический} метод \texttt{is\_leap\_year}, который принимает год в качестве аргумента и проверяет, является ли год високосным. Если год делится на 4 без остатка и не делится на 100 без остатка, или делится на 400 без остатка, то возвращает \texttt{True}. В противном случае возвращает \texttt{False}.
    \item Используйте цикл для проверки каждого года от 2000 до 2099 (включительно), вызывая статический метод \texttt{is\_leap\_year} и выводя результат на экран.
\end{enumerate}

\subsection*{Пример использования:}
\begin{lstlisting}[language=Python]
    v = LeapYearChecker.is_leap_year(1999)
\end{lstlisting}
Вывод (первые и последние строки):
\begin{verbatim}
2000 True
2001 False
...
2098 False
2099 False
\end{verbatim}

\item
Написать программу, которая создаёт класс \texttt{PrimeChecker} 
для определения простого числа. В классе должен быть статический метод
\texttt{is\_prime} и возвращать \texttt{True}, если число простое, 
и \texttt{False} в противном случае. 
Программа также должна использовать цикл для проверки каждого числа от 
1 до 100 и вывода результата на экран.

\subsection*{Инструкции:}
\begin{enumerate}
    \item Создайте класс \texttt{PrimeChecker}.
    \item Создайте \textbf{статический} метод \texttt{is\_prime}, который принимает число в качестве аргумента и проверяет, является ли число простым. Простое число делится только на 1 и на само себя.
    \item Используйте цикл для проверки каждого числа от 1 до 100 (включительно), вызывая статический метод \texttt{is\_prime} и выводя результат на экран.
\end{enumerate}

\subsection*{Пример использования:}
\begin{lstlisting}[language=Python]
    v = PrimeChecker.is_prime(17)
\end{lstlisting}
Вывод (первые и последние строки):
\begin{verbatim}
1 False
2 True
3 True
...
98 False
99 False
100 False
\end{verbatim}

\item
Написать программу, которая создаёт класс \texttt{EvenChecker} 
для определения чётности числа. В классе должен быть статический метод
\texttt{is\_even} и возвращать \texttt{True}, если число чётное, 
и \texttt{False} в противном случае. 
Программа также должна использовать цикл для проверки каждого числа от 
1 до 50 и вывода результата на экран.

\subsection*{Инструкции:}
\begin{enumerate}
    \item Создайте класс \texttt{EvenChecker}.
    \item Создайте \textbf{статический} метод \texttt{is\_even}, который принимает число в качестве аргумента и проверяет, является ли число чётным.
    \item Используйте цикл для проверки каждого числа от 1 до 50 (включительно), вызывая статический метод \texttt{is\_even} и выводя результат на экран.
\end{enumerate}

\subsection*{Пример использования:}
\begin{lstlisting}[language=Python]
    v = EvenChecker.is_even(25)
\end{lstlisting}
Вывод (первые и последние строки):
\begin{verbatim}
1 False
2 True
3 False
...
48 True
49 False
50 True
\end{verbatim}

\item
Написать программу, которая создаёт класс \texttt{SquareChecker} 
для определения квадратного числа. В классе должен быть статический метод
\texttt{is\_square} и возвращать \texttt{True}, если число является квадратом целого числа, 
и \texttt{False} в противном случае. 
Программа также должна использовать цикл для проверки каждого числа от 
1 до 100 и вывода результата на экран.

\subsection*{Инструкции:}
\begin{enumerate}
    \item Создайте класс \texttt{SquareChecker}.
    \item Создайте \textbf{статический} метод \texttt{is\_square}, который принимает число в качестве аргумента и проверяет, является ли число квадратом целого числа.
    \item Используйте цикл для проверки каждого числа от 1 до 100 (включительно), вызывая статический метод \texttt{is\_square} и выводя результат на экран.
\end{enumerate}

\subsection*{Пример использования:}
\begin{lstlisting}[language=Python]
    v = SquareChecker.is_square(36)
\end{lstlisting}
Вывод (первые и последние строки):
\begin{verbatim}
1 True
2 False
3 False
...
99 False
100 True
\end{verbatim}

\item
Написать программу, которая создаёт класс \texttt{FactorialCalculator} 
для вычисления факториала числа. В классе должен быть статический метод
\texttt{factorial} и возвращать факториал числа. 
Программа также должна использовать цикл для вычисления факториала каждого числа от 
1 до 10 и вывода результата на экран.

\subsection*{Инструкции:}
\begin{enumerate}
    \item Создайте класс \texttt{FactorialCalculator}.
    \item Создайте \textbf{статический} метод \texttt{factorial}, который принимает число в качестве аргумента и возвращает его факториал.
    \item Используйте цикл для вычисления факториала каждого числа от 1 до 10 (включительно), вызывая статический метод \texttt{factorial} и выводя результат на экран.
\end{enumerate}

\subsection*{Пример использования:}
\begin{lstlisting}[language=Python]
    v = FactorialCalculator.factorial(5)
\end{lstlisting}
Вывод (первые и последние строки):
\begin{verbatim}
1 1
2 2
3 6
...
9 362880
10 3628800
\end{verbatim}

\item
Написать программу, которая создаёт класс \texttt{PalindromeChecker} 
для определения палиндрома числа. В классе должен быть статический метод
\texttt{is\_palindrome} и возвращать \texttt{True}, если число является палиндромом, 
и \texttt{False} в противном случае. 
Программа также должна использовать цикл для проверки каждого числа от 
100 до 200 и вывода результата на экран.

\subsection*{Инструкции:}
\begin{enumerate}
    \item Создайте класс \texttt{PalindromeChecker}.
    \item Создайте \textbf{статический} метод \texttt{is\_palindrome}, который принимает число в качестве аргумента и проверяет, является ли число палиндромом (читается одинаково слева направо и справа налево).
    \item Используйте цикл для проверки каждого числа от 100 до 200 (включительно), вызывая статический метод \texttt{is\_palindrome} и выводя результат на экран.
\end{enumerate}

\subsection*{Пример использования:}
\begin{lstlisting}[language=Python]
    v = PalindromeChecker.is_palindrome(121)
\end{lstlisting}
Вывод (первые и последние строки):
\begin{verbatim}
100 False
101 True
102 False
...
199 False
200 False
\end{verbatim}

\item
Написать программу, которая создаёт класс \texttt{ArmstrongChecker} 
для определения числа Армстронга. В классе должен быть статический метод
\texttt{is\_armstrong} и возвращать \texttt{True}, если число является числом Армстронга, 
и \texttt{False} в противном случае. 
Программа также должна использовать цикл для проверки каждого числа от 
100 до 500 и вывода результата на экран.

\subsection*{Инструкции:}
\begin{enumerate}
    \item Создайте класс \texttt{ArmstrongChecker}.
    \item Создайте \textbf{статический} метод \texttt{is\_armstrong}, который принимает число в качестве аргумента и проверяет, является ли число числом Армстронга (сумма цифр в степени, равной количеству цифр, равна самому числу).
    \item Используйте цикл для проверки каждого числа от 100 до 500 (включительно), вызывая статический метод \texttt{is\_armstrong} и выводя результат на экран.
\end{enumerate}

\subsection*{Пример использования:}
\begin{lstlisting}[language=Python]
    v = ArmstrongChecker.is_armstrong(153)
\end{lstlisting}
Вывод (первые и последние строки):
\begin{verbatim}
100 False
101 False
102 False
...
499 False
500 False
\end{verbatim}

\item
Написать программу, которая создаёт класс \texttt{PerfectNumberChecker} 
для определения совершенного числа. В классе должен быть статический метод
\texttt{is\_perfect} и возвращать \texttt{True}, если число является совершенным, 
и \texttt{False} в противном случае. 
Программа также должна использовать цикл для проверки каждого числа от 
1 до 1000 и вывода результата на экран.

\subsection*{Инструкции:}
\begin{enumerate}
    \item Создайте класс \texttt{PerfectNumberChecker}.
    \item Создайте \textbf{статический} метод \texttt{is\_perfect}, который принимает число в качестве аргумента и проверяет, является ли число совершенным (сумма делителей равна числу).
    \item Используйте цикл для проверки каждого числа от 1 до 1000 (включительно), вызывая статический метод \texttt{is\_perfect} и выводя результат на экран.
\end{enumerate}

\subsection*{Пример использования:}
\begin{lstlisting}[language=Python]
    v = PerfectNumberChecker.is_perfect(28)
\end{lstlisting}
Вывод (первые и последние строки):
\begin{verbatim}
1 False
2 False
3 False
...
998 False
999 False
1000 False
\end{verbatim}

\item
Написать программу, которая создаёт класс \texttt{FibonacciChecker} 
для проверки числа Фибоначчи. В классе должен быть статический метод
\texttt{is\_fibonacci} и возвращать \texttt{True}, если число является числом Фибоначчи, 
и \texttt{False} в противном случае. 
Программа также должна использовать цикл для проверки каждого числа от 
1 до 100 и вывода результата на экран.

\subsection*{Инструкции:}
\begin{enumerate}
    \item Создайте класс \texttt{FibonacciChecker}.
    \item Создайте \textbf{статический} метод \texttt{is\_fibonacci}, который принимает число в качестве аргумента и проверяет, является ли число числом Фибоначчи.
    \item Используйте цикл для проверки каждого числа от 1 до 100 (включительно), вызывая статический метод \texttt{is\_fibonacci} и выводя результат на экран.
\end{enumerate}

\subsection*{Пример использования:}
\begin{lstlisting}[language=Python]
    v = FibonacciChecker.is_fibonacci(21)
\end{lstlisting}
Вывод (первые и последние строки):
\begin{verbatim}
1 True
2 True
3 True
...
98 False
99 False
100 False
\end{verbatim}

\item
Написать программу, которая создаёт класс \texttt{PowerOfTwoChecker} 
для проверки степени двойки. В классе должен быть статический метод
\texttt{is\_power\_of\_two} и возвращать \texttt{True}, если число является степенью двойки, 
и \texttt{False} в противном случае. 
Программа также должна использовать цикл для проверки каждого числа от 
1 до 128 и вывода результата на экран.

\subsection*{Инструкции:}
\begin{enumerate}
    \item Создайте класс \texttt{PowerOfTwoChecker}.
    \item Создайте \textbf{статический} метод \texttt{is\_power\_of\_two}, который принимает число в качестве аргумента и проверяет, является ли число степенью двойки.
    \item Используйте цикл для проверки каждого числа от 1 до 128 (включительно), вызывая статический метод \texttt{is\_power\_of\_two} и выводя результат на экран.
\end{enumerate}

\subsection*{Пример использования:}
\begin{lstlisting}[language=Python]
    v = PowerOfTwoChecker.is_power_of_two(64)
\end{lstlisting}
Вывод (первые и последние строки):
\begin{verbatim}
1 True
2 True
3 False
...
127 False
128 True
\end{verbatim}

\item
Написать программу, которая создаёт класс \texttt{SumOfDigitsCalculator} 
для вычисления суммы цифр числа. В классе должен быть статический метод
\texttt{sum\_of\_digits} и возвращать сумму цифр. 
Программа также должна использовать цикл для вычисления суммы цифр каждого числа от 
1 до 50 и вывода результата на экран.

\subsection*{Инструкции:}
\begin{enumerate}
    \item Создайте класс \texttt{SumOfDigitsCalculator}.
    \item Создайте \textbf{статический} метод \texttt{sum\_of\_digits}, который принимает число в качестве аргумента и возвращает сумму его цифр.
    \item Используйте цикл для вычисления суммы цифр каждого числа от 1 до 50 (включительно), вызывая статический метод \texttt{sum\_of\_digits} и выводя результат на экран.
\end{enumerate}

\subsection*{Пример использования:}
\begin{lstlisting}[language=Python]
    v = SumOfDigitsCalculator.sum_of_digits(123)
\end{lstlisting}
Вывод (первые и последние строки):
\begin{verbatim}
1 1
2 2
3 3
...
49 13
50 5
\end{verbatim}

\item
Написать программу, которая создаёт класс \texttt{PrimeSumCalculator} 
для вычисления суммы простых чисел в диапазоне. В классе должен быть статический метод
\texttt{sum\_of\_primes} и возвращать сумму простых чисел в заданном диапазоне. 
Программа также должна использовать цикл для вычисления суммы простых чисел от 
1 до 100 и вывода результата на экран.

\subsection*{Инструкции:}
\begin{enumerate}
    \item Создайте класс \texttt{PrimeSumCalculator}.
    \item Создайте \textbf{статический} метод \texttt{sum\_of\_primes}, который принимает два аргумента (начало и конец диапазона) и возвращает сумму простых чисел в этом диапазоне.
    \item Используйте метод для вычисления суммы простых чисел от 1 до 100 и выведите результат.
\end{enumerate}

\subsection*{Пример использования:}
\begin{lstlisting}[language=Python]
    v = PrimeSumCalculator.sum_of_primes(1, 10)
\end{lstlisting}
Вывод:
\begin{verbatim}
Сумма простых чисел от 1 до 100: 1060
\end{verbatim}

\item
Написать программу, которая создаёт класс \texttt{DigitCountCalculator} 
для подсчёта количества цифр в числе. В классе должен быть статический метод
\texttt{digit\_count} и возвращать количество цифр. 
Программа также должна использовать цикл для подсчёта цифр каждого числа от 
1 до 100 и вывода результата на экран.

\subsection*{Инструкции:}
\begin{enumerate}
    \item Создайте класс \texttt{DigitCountCalculator}.
    \item Создайте \textbf{статический} метод \texttt{digit\_count}, который принимает число в качестве аргумента и возвращает количество его цифр.
    \item Используйте цикл для подсчёта цифр каждого числа от 1 до 100 (включительно), вызывая статический метод \texttt{digit\_count} и выводя результат на экран.
\end{enumerate}

\subsection*{Пример использования:}
\begin{lstlisting}[language=Python]
    v = DigitCountCalculator.digit_count(12345)
\end{lstlisting}
Вывод (первые и последние строки):
\begin{verbatim}
1 1
2 1
3 1
...
99 2
100 3
\end{verbatim}

\item
Написать программу, которая создаёт класс \texttt{BinaryConverter} 
для преобразования числа в двоичное представление. В классе должен быть статический метод
\texttt{to\_binary} и возвращать строку с двоичным представлением числа. 
Программа также должна использовать цикл для преобразования каждого числа от 
1 до 16 и вывода результата на экран.

\subsection*{Инструкции:}
\begin{enumerate}
    \item Создайте класс \texttt{BinaryConverter}.
    \item Создайте \textbf{статический} метод \texttt{to\_binary}, который принимает число в качестве аргумента и возвращает его двоичное представление в виде строки.
    \item Используйте цикл для преобразования каждого числа от 1 до 16 (включительно), вызывая статический метод \texttt{to\_binary} и выводя результат на экран.
\end{enumerate}

\subsection*{Пример использования:}
\begin{lstlisting}[language=Python]
    v = BinaryConverter.to_binary(10)
\end{lstlisting}
Вывод (первые и последние строки):
\begin{verbatim}
1 1
2 10
3 11
...
15 1111
16 10000
\end{verbatim}

\item
Написать программу, которая создаёт класс \texttt{HexConverter} 
для преобразования числа в шестнадцатеричное представление. В классе должен быть статический метод
\texttt{to\_hex} и возвращать строку с шестнадцатеричным представлением числа. 
Программа также должна использовать цикл для преобразования каждого числа от 
1 до 20 и вывода результата на экран.

\subsection*{Инструкции:}
\begin{enumerate}
    \item Создайте класс \texttt{HexConverter}.
    \item Создайте \textbf{статический} метод \texttt{to\_hex}, который принимает число в качестве аргумента и возвращает его шестнадцатеричное представление в виде строки.
    \item Используйте цикл для преобразования каждого числа от 1 до 20 (включительно), вызывая статический метод \texttt{to\_hex} и выводя результат на экран.
\end{enumerate}

\subsection*{Пример использования:}
\begin{lstlisting}[language=Python]
    v = HexConverter.to_hex(255)
\end{lstlisting}
Вывод (первые и последние строки):
\begin{verbatim}
1 1
2 2
3 3
...
19 13
20 14
\end{verbatim}

\item
Написать программу, которая создаёт класс \texttt{DivisorChecker} 
для проверки делителей числа. В классе должен быть статический метод
\texttt{get\_divisors} и возвращать список делителей числа. 
Программа также должна использовать цикл для вывода делителей каждого числа от 
1 до 20 и вывода результата на экран.

\subsection*{Инструкции:}
\begin{enumerate}
    \item Создайте класс \texttt{DivisorChecker}.
    \item Создайте \textbf{статический} метод \texttt{get\_divisors}, который принимает число в качестве аргумента и возвращает список его делителей.
    \item Используйте цикл для вывода делителей каждого числа от 1 до 20 (включительно), вызывая статический метод \texttt{get\_divisors} и выводя результат на экран.
\end{enumerate}

\subsection*{Пример использования:}
\begin{lstlisting}[language=Python]
    v = DivisorChecker.get_divisors(12)
\end{lstlisting}
Вывод (первые и последние строки):
\begin{verbatim}
1 [1]
2 [1, 2]
3 [1, 3]
...
19 [1, 19]
20 [1, 2, 4, 5, 10, 20]
\end{verbatim}

\item
Написать программу, которая создаёт класс \texttt{Multiplier} 
для создания таблицы умножения. В классе должен быть статический метод
\texttt{multiply\_table} и выводить таблицу умножения для заданного числа. 
Программа также должна использовать цикл для вывода таблицы умножения для чисел от 
1 до 10 и вывода результата на экран.

\subsection*{Инструкции:}
\begin{enumerate}
    \item Создайте класс \texttt{Multiplier}.
    \item Создайте \textbf{статический} метод \texttt{multiply\_table}, который принимает число в качестве аргумента и выводит таблицу умножения для этого числа от 1 до 10.
    \item Используйте цикл для вывода таблицы умножения для чисел от 1 до 10 (включительно), вызывая статический метод \texttt{multiply\_table} и выводя результат на экран.
\end{enumerate}

\subsection*{Пример использования:}
\begin{lstlisting}[language=Python]
    Multiplier.multiply_table(5)
\end{lstlisting}
Вывод (для числа 5):
\begin{verbatim}
5 * 1 = 5
5 * 2 = 10
...
5 * 10 = 50
\end{verbatim}

\item
Написать программу, которая создаёт класс \texttt{GCDCalculator} 
для вычисления НОД двух чисел. В классе должен быть статический метод
\texttt{gcd} и возвращать наибольший общий делитель. 
Программа также должна использовать цикл для вычисления НОД чисел 
(1,1), (2,4), (3,9), ..., (10,100) и вывода результата на экран.

\subsection*{Инструкции:}
\begin{enumerate}
    \item Создайте класс \texttt{GCDCalculator}.
    \item Создайте \textbf{статический} метод \texttt{gcd}, который принимает два числа в качестве аргументов и возвращает их наибольший общий делитель.
    \item Используйте цикл для вычисления НОД пар чисел (1,1), (2,4), (3,9), ..., (10,100), вызывая статический метод \texttt{gcd} и выводя результат на экран.
\end{enumerate}

\subsection*{Пример использования:}
\begin{lstlisting}[language=Python]
    v = GCDCalculator.gcd(48, 18)
\end{lstlisting}
Вывод:
\begin{verbatim}
НОД(1, 1) = 1
НОД(2, 4) = 2
НОД(3, 9) = 3
...
НОД(10, 100) = 10
\end{verbatim}

\item
Написать программу, которая создаёт класс \texttt{LCMCalculator} 
для вычисления НОК двух чисел. В классе должен быть статический метод
\texttt{lcm} и возвращать наименьшее общее кратное. 
Программа также должна использовать цикл для вычисления НОК чисел 
(1,1), (2,3), (3,5), ..., (10,11) и вывода результата на экран.

\subsection*{Инструкции:}
\begin{enumerate}
    \item Создайте класс \texttt{LCMCalculator}.
    \item Создайте \textbf{статический} метод \texttt{lcm}, который принимает два числа в качестве аргументов и возвращает их наименьшее общее кратное.
    \item Используйте цикл для вычисления НОК пар чисел (1,1), (2,3), (3,5), ..., (10,11), вызывая статический метод \texttt{lcm} и выводя результат на экран.
\end{enumerate}

\subsection*{Пример использования:}
\begin{lstlisting}[language=Python]
    v = LCMCalculator.lcm(4, 6)
\end{lstlisting}
Вывод:
\begin{verbatim}
НОК(1, 1) = 1
НОК(2, 3) = 6
НОК(3, 5) = 15
...
НОК(10, 11) = 110
\end{verbatim}

\item
Написать программу, которая создаёт класс \texttt{DigitReverse} 
для разворота цифр числа. В классе должен быть статический метод
\texttt{reverse\_digits} и возвращать число с обратным порядком цифр. 
Программа также должна использовать цикл для разворота каждого числа от 
10 до 20 и вывода результата на экран.

\subsection*{Инструкции:}
\begin{enumerate}
    \item Создайте класс \texttt{DigitReverse}.
    \item Создайте \textbf{статический} метод \texttt{reverse\_digits}, который принимает число в качестве аргумента и возвращает число с обратным порядком цифр.
    \item Используйте цикл для разворота каждого числа от 10 до 20 (включительно), вызывая статический метод \texttt{reverse\_digits} и выводя результат на экран.
\end{enumerate}

\subsection*{Пример использования:}
\begin{lstlisting}[language=Python]
    v = DigitReverse.reverse_digits(123)
\end{lstlisting}
Вывод:
\begin{verbatim}
10 1
11 11
12 21
13 31
...
19 91
20 2
\end{verbatim}

\item
Написать программу, которая создаёт класс \texttt{NumberTypeChecker} 
для определения типа числа (положительное/отрицательное/ноль). В классе должен быть статический метод
\texttt{check\_number\_type} и возвращать строку с типом числа. 
Программа также должна использовать цикл для проверки чисел 
[-5, -4, -3, -2, -1, 0, 1, 2, 3, 4, 5] и вывода результата на экран.

\subsection*{Инструкции:}
\begin{enumerate}
    \item Создайте класс \texttt{NumberTypeChecker}.
    \item Создайте \textbf{статический} метод \texttt{check\_number\_type}, который принимает число в качестве аргумента и возвращает строку "positive", "negative" или "zero".
    \item Используйте цикл для проверки чисел [-5, -4, -3, -2, -1, 0, 1, 2, 3, 4, 5], вызывая статический метод \texttt{check\_number\_type} и выводя результат на экран.
\end{enumerate}

\subsection*{Пример использования:}
\begin{lstlisting}[language=Python]
    v = NumberTypeChecker.check_number_type(-7)
\end{lstlisting}
Вывод:
\begin{verbatim}
-5 negative
-4 negative
-3 negative
-2 negative
-1 negative
0 zero
1 positive
2 positive
3 positive
4 positive
5 positive
\end{verbatim}

\item
Написать программу, которая создаёт класс \texttt{FactorialChecker} 
для проверки факториала числа. В классе должен быть статический метод
\texttt{is\_factorial} и возвращать \texttt{True}, если число является факториалом какого-либо числа, 
и \texttt{False} в противном случае. 
Программа также должна использовать цикл для проверки каждого числа от 
1 до 120 и вывода результата на экран.

\subsection*{Инструкции:}
\begin{enumerate}
    \item Создайте класс \texttt{FactorialChecker}.
    \item Создайте \textbf{статический} метод \texttt{is\_factorial}, который принимает число в качестве аргумента и проверяет, является ли число факториалом какого-либо числа.
    \item Используйте цикл для проверки каждого числа от 1 до 120 (включительно), вызывая статический метод \texttt{is\_factorial} и выводя результат на экран.
\end{enumerate}

\subsection*{Пример использования:}
\begin{lstlisting}[language=Python]
    v = FactorialChecker.is_factorial(24)
\end{lstlisting}
Вывод (первые и последние строки):
\begin{verbatim}
1 True
2 True
3 False
...
119 False
120 True
\end{verbatim}

\item
Написать программу, которая создаёт класс \texttt{PowerChecker} 
для проверки степени числа. В классе должен быть статический метод
\texttt{is\_power} и возвращать \texttt{True}, если число является степенью заданного основания, 
и \texttt{False} в противном случае. 
Программа также должна использовать цикл для проверки каждого числа от 
1 до 100 относительно основания 3 и вывода результата на экран.

\subsection*{Инструкции:}
\begin{enumerate}
    \item Создайте класс \texttt{PowerChecker}.
    \item Создайте \textbf{статический} метод \texttt{is\_power}, который принимает число и основание в качестве аргументов и проверяет, является ли число степенью основания.
    \item Используйте цикл для проверки каждого числа от 1 до 100 (включительно) относительно основания 3, вызывая статический метод \texttt{is\_power} и выводя результат на экран.
\end{enumerate}

\subsection*{Пример использования:}
\begin{lstlisting}[language=Python]
    v = PowerChecker.is_power(81, 3)
\end{lstlisting}
Вывод (первые и последние строки):
\begin{verbatim}
1 True
2 False
3 True
...
99 False
100 False
\end{verbatim}

\item
Написать программу, которая создаёт класс \texttt{DigitProductCalculator} 
для вычисления произведения цифр числа. В классе должен быть статический метод
\texttt{digit\_product} и возвращать произведение цифр. 
Программа также должна использовать цикл для вычисления произведения цифр каждого числа от 
1 до 50 и вывода результата на экран.

\subsection*{Инструкции:}
\begin{enumerate}
    \item Создайте класс \texttt{DigitProductCalculator}.
    \item Создайте \textbf{статический} метод \texttt{digit\_product}, который принимает число в качестве аргумента и возвращает произведение его цифр.
    \item Используйте цикл для вычисления произведения цифр каждого числа от 1 до 50 (включительно), вызывая статический метод \texttt{digit\_product} и выводя результат на экран.
\end{enumerate}

\subsection*{Пример использования:}
\begin{lstlisting}[language=Python]
    v = DigitProductCalculator.digit_product(123)
\end{lstlisting}
Вывод (первые и последние строки):
\begin{verbatim}
1 1
2 2
3 3
...
49 36
50 0
\end{verbatim}

\item
Написать программу, которая создаёт класс \texttt{NumberLengthChecker} 
для проверки длины числа. В классе должен быть статический метод
\texttt{get\_length} и возвращать количество цифр в числе. 
Программа также должна использовать цикл для проверки длины каждого числа от 
1 до 1000 с шагом 100 и вывода результата на экран.

\subsection*{Инструкции:}
\begin{enumerate}
    \item Создайте класс \texttt{NumberLengthChecker}.
    \item Создайте \textbf{статический} метод \texttt{get\_length}, который принимает число в качестве аргумента и возвращает количество его цифр.
    \item Используйте цикл для проверки длины чисел 1, 100, 200, 300, 400, 500, 600, 700, 800, 900, 1000, вызывая статический метод \texttt{get\_length} и выводя результат на экран.
\end{enumerate}

\subsection*{Пример использования:}
\begin{lstlisting}[language=Python]
    v = NumberLengthChecker.get_length(12345)
\end{lstlisting}
Вывод:
\begin{verbatim}
1 1
100 3
200 3
300 3
400 3
500 3
600 3
700 3
800 3
900 3
1000 4
\end{verbatim}

\item
Написать программу, которая создаёт класс \texttt{NumberSquareSumCalculator} 
для вычисления суммы квадратов чисел. В классе должен быть статический метод
\texttt{square\_sum} и возвращать сумму квадратов чисел в диапазоне. 
Программа также должна использовать метод для вычисления суммы квадратов чисел от 
1 до 10 и вывода результата на экран.

\subsection*{Инструкции:}
\begin{enumerate}
    \item Создайте класс \texttt{NumberSquareSumCalculator}.
    \item Создайте \textbf{статический} метод \texttt{square\_sum}, который принимает два аргумента (начало и конец диапазона) и возвращает сумму квадратов чисел в этом диапазоне.
    \item Используйте метод для вычисления суммы квадратов чисел от 1 до 10 и выведите результат.
\end{enumerate}

\subsection*{Пример использования:}
\begin{lstlisting}[language=Python]
    v = NumberSquareSumCalculator.square_sum(1, 3)
\end{lstlisting}
Вывод:
\begin{verbatim}
Сумма квадратов чисел от 1 до 10: 385
\end{verbatim}

\item
Написать программу, которая создаёт класс \texttt{NumberCubeSumCalculator} 
для вычисления суммы кубов чисел. В классе должен быть статический метод
\texttt{cube\_sum} и возвращать сумму кубов чисел в диапазоне. 
Программа также должна использовать метод для вычисления суммы кубов чисел от 
1 до 10 и вывода результата на экран.

\subsection*{Инструкции:}
\begin{enumerate}
    \item Создайте класс \texttt{NumberCubeSumCalculator}.
    \item Создайте \textbf{статический} метод \texttt{cube\_sum}, который принимает два аргумента (начало и конец диапазона) и возвращает сумму кубов чисел в этом диапазоне.
    \item Используйте метод для вычисления суммы кубов чисел от 1 до 10 и выведите результат.
\end{enumerate}

\subsection*{Пример использования:}
\begin{lstlisting}[language=Python]
    v = NumberCubeSumCalculator.cube_sum(1, 3)
\end{lstlisting}
Вывод:
\begin{verbatim}
Сумма кубов чисел от 1 до 10: 3025
\end{verbatim}

\item
Написать программу, которая создаёт класс \texttt{NumberRangeChecker} 
для проверки числа на принадлежность диапазону. В классе должен быть статический метод
\texttt{in\_range} и возвращать \texttt{True}, если число находится в заданном диапазоне, 
и \texttt{False} в противном случае. 
Программа также должна использовать цикл для проверки чисел от 
-5 до 5 на принадлежность диапазону [0, 10] и вывода результата на экран.

\subsection*{Инструкции:}
\begin{enumerate}
    \item Создайте класс \texttt{NumberRangeChecker}.
    \item Создайте \textbf{статический} метод \texttt{in\_range}, который принимает число, начало и конец диапазона и проверяет, находится ли число в этом диапазоне.
    \item Используйте цикл для проверки чисел от -5 до 5 (включительно) на принадлежность диапазону [0, 10], вызывая статический метод \texttt{in\_range} и выводя результат на экран.
\end{enumerate}

\subsection*{Пример использования:}
\begin{lstlisting}[language=Python]
    v = NumberRangeChecker.in_range(5, 0, 10)
\end{lstlisting}
Вывод:
\begin{verbatim}
-5 False
-4 False
-3 False
-2 False
-1 False
0 True
1 True
2 True
3 True
4 True
5 True
\end{verbatim}

\item
Написать программу, которая создаёт класс \texttt{NumberSignChecker} 
для проверки знака числа. В классе должен быть статический метод
\texttt{get\_sign} и возвращать строку с знаком числа (+, - или 0). 
Программа также должна использовать цикл для проверки чисел 
[-5, -4, -3, -2, -1, 0, 1, 2, 3, 4, 5] и вывода результата на экран.

\subsection*{Инструкции:}
\begin{enumerate}
    \item Создайте класс \texttt{NumberSignChecker}.
    \item Создайте \textbf{статический} метод \texttt{get\_sign}, который принимает число в качестве аргумента и возвращает строку с его знаком (+, - или 0).
    \item Используйте цикл для проверки чисел [-5, -4, -3, -2, -1, 0, 1, 2, 3, 4, 5], вызывая статический метод \texttt{get\_sign} и выводя результат на экран.
\end{enumerate}

\subsection*{Пример использования:}
\begin{lstlisting}[language=Python]
    v = NumberSignChecker.get_sign(-7)
\end{lstlisting}
Вывод:
\begin{verbatim}
-5 -
-4 -
-3 -
-2 -
-1 -
0 0
1 +
2 +
3 +
4 +
5 +
\end{verbatim}

\item
Написать программу, которая создаёт класс \texttt{NumberPalindromeChecker} 
для проверки палиндрома числа. В классе должен быть статический метод
\texttt{is\_palindrome} и возвращать \texttt{True}, если число является палиндромом, 
и \texttt{False} в противном случае. 
Программа также должна использовать цикл для проверки каждого числа от 
100 до 150 и вывода результата на экран.

\subsection*{Инструкции:}
\begin{enumerate}
    \item Создайте класс \texttt{NumberPalindromeChecker}.
    \item Создайте \textbf{статический} метод \texttt{is\_palindrome}, который принимает число в качестве аргумента и проверяет, является ли число палиндромом.
    \item Используйте цикл для проверки каждого числа от 100 до 150 (включительно), вызывая статический метод \texttt{is\_palindrome} и выводя результат на экран.
\end{enumerate}

\subsection*{Пример использования:}
\begin{lstlisting}[language=Python]
    v = NumberPalindromeChecker.is_palindrome(121)
\end{lstlisting}
Вывод (первые и последние строки):
\begin{verbatim}
100 False
101 True
102 False
...
149 False
150 False
\end{verbatim}

\item
Написать программу, которая создаёт класс \texttt{NumberAscendingChecker} 
для проверки, что цифры числа идут в порядке возрастания. В классе должен быть статический метод
\texttt{is\_ascending} и возвращать \texttt{True}, если цифры числа идут в порядке возрастания, 
и \texttt{False} в противном случае. 
Программа также должна использовать цикл для проверки каждого числа от 
10 до 100 и вывода результата на экран.

\subsection*{Инструкции:}
\begin{enumerate}
    \item Создайте класс \texttt{NumberAscendingChecker}.
    \item Создайте \textbf{статический} метод \texttt{is\_ascending}, который принимает число в качестве аргумента и проверяет, идут ли его цифры в порядке возрастания.
    \item Используйте цикл для проверки каждого числа от 10 до 100 (включительно), вызывая статический метод \texttt{is\_ascending} и выводя результат на экран.
\end{enumerate}

\subsection*{Пример использования:}
\begin{lstlisting}[language=Python]
    v = NumberAscendingChecker.is_ascending(123)
\end{lstlisting}
Вывод (первые и последние строки):
\begin{verbatim}
10 False
11 False
12 True
13 True
...
98 False
99 False
100 False
\end{verbatim}

\item
Написать программу, которая создаёт класс \texttt{NumberDescendingChecker} 
для проверки, что цифры числа идут в порядке убывания. В классе должен быть статический метод
\texttt{is\_descending} и возвращать \texttt{True}, если цифры числа идут в порядке убывания, 
и \texttt{False} в противном случае. 
Программа также должна использовать цикл для проверки каждого числа от 
10 до 100 и вывода результата на экран.

\subsection*{Инструкции:}
\begin{enumerate}
    \item Создайте класс \texttt{NumberDescendingChecker}.
    \item Создайте \textbf{статический} метод \texttt{is\_descending}, который принимает число в качестве аргумента и проверяет, идут ли его цифры в порядке убывания.
    \item Используйте цикл для проверки каждого числа от 10 до 100 (включительно), вызывая статический метод \texttt{is\_descending} и выводя результат на экран.
\end{enumerate}

\subsection*{Пример использования:}
\begin{lstlisting}[language=Python]
    v = NumberDescendingChecker.is_descending(321)
\end{lstlisting}
Вывод (первые и последние строки):
\begin{verbatim}
10 False
11 False
12 False
13 False
...
98 True
99 True
100 False
\end{verbatim}

\item
Написать программу, которая создаёт класс \texttt{NumberPrimeDigitChecker} 
для проверки, что все цифры числа простые. В классе должен быть статический метод
\texttt{all\_digits\_prime} и возвращать \texttt{True}, если все цифры числа простые, 
и \texttt{False} в противном случае. 
Программа также должна использовать цикл для проверки каждого числа от 
10 до 100 и вывода результата на экран.

\subsection*{Инструкции:}
\begin{enumerate}
    \item Создайте класс \texttt{NumberPrimeDigitChecker}.
    \item Создайте \textbf{статический} метод \texttt{all\_digits\_prime}, который принимает число в качестве аргумента и проверяет, являются ли все его цифры простыми числами.
    \item Используйте цикл для проверки каждого числа от 10 до 100 (включительно), вызывая статический метод \texttt{all\_digits\_prime} и выводя результат на экран.
\end{enumerate}

\subsection*{Пример использования:}
\begin{lstlisting}[language=Python]
    v = NumberPrimeDigitChecker.all_digits_prime(23)
\end{lstlisting}
Вывод (первые и последние строки):
\begin{verbatim}
10 False
11 False
12 False
13 False
...
98 False
99 False
100 False
\end{verbatim}

\item
Написать программу, которая создаёт класс \texttt{NumberEvenDigitChecker} 
для проверки, что все цифры числа чётные. В классе должен быть статический метод
\texttt{all\_digits\_even} и возвравать \texttt{True}, если все цифры числа чётные, 
и \texttt{False} в противном случае. 
Программа также должна использовать цикл для проверки каждого числа от 
10 до 100 и вывода результата на экран.

\subsection*{Инструкции:}
\begin{enumerate}
    \item Создайте класс \texttt{NumberEvenDigitChecker}.
    \item Создайте \textbf{статический} метод \texttt{all\_digits\_even}, который принимает число в качестве аргумента и проверяет, являются ли все его цифры чётными.
    \item Используйте цикл для проверки каждого числа от 10 до 100 (включительно), вызывая статический метод \texttt{all\_digits\_even} и выводя результат на экран.
\end{enumerate}

\subsection*{Пример использования:}
\begin{lstlisting}[language=Python]
    v = NumberEvenDigitChecker.all_digits_even(24)
\end{lstlisting}
Вывод (первые и последние строки):
\begin{verbatim}
10 False
11 False
12 False
13 False
...
98 False
99 False
100 False
\end{verbatim}

\item
Написать программу, которая создаёт класс \texttt{NumberOddDigitChecker} 
для проверки, что все цифры числа нечётные. В классе должен быть статический метод
\texttt{all\_digits\_odd} и возвращать \texttt{True}, если все цифры числа нечётные, 
и \texttt{False} в противном случае. 
Программа также должна использовать цикл для проверки каждого числа от 
10 до 100 и вывода результата на экран.

\subsection*{Инструкции:}
\begin{enumerate}
    \item Создайте класс \texttt{NumberOddDigitChecker}.
    \item Создайте \textbf{статический} метод \texttt{all\_digits\_odd}, который принимает число в качестве аргумента и проверяет, являются ли все его цифры нечётными.
    \item Используйте цикл для проверки каждого числа от 10 до 100 (включительно), вызывая статический метод \texttt{all\_digits\_odd} и выводя результат на экран.
\end{enumerate}

\subsection*{Пример использования:}
\begin{lstlisting}[language=Python]
    v = NumberOddDigitChecker.all_digits_odd(135)
\end{lstlisting}
Вывод (первые и последние строки):
\begin{verbatim}
10 False
11 True
12 False
13 True
...
98 False
99 True
100 False
\end{verbatim}


\end{enumerate}

\textbf{Задача 3}

\begin{enumerate}

\item
Написать программу на Python, которая создает класс \texttt{Person} для представления сотрудника персонала. Класс должен содержать закрытые атрибуты \texttt{\_\_name}, \texttt{\_\_country}, \texttt{\_\_date\_of\_birth} и метод \texttt{calculate\_age}. Доступ к атрибутам только через методы-геттеры. Создать экземпляры и вывести информацию о каждом человеке.

\subsection*{Инструкции:}
\begin{enumerate}
    \item Создайте класс \texttt{Person} с методом \texttt{\_\_init\_\_}, который принимает имя, страну и дату рождения.
    \item Создайте методы-геттеры: \texttt{get\_name()}, \texttt{get\_country()}, \texttt{get\_date\_of\_birth()}.
    \item Создайте метод \texttt{calculate\_age()} для вычисления возраста.
    \item Создайте несколько экземпляров класса \texttt{Person}.
    \item Выведите данные каждого человека через методы класса.
\end{enumerate}

\subsection*{Пример использования:}
\begin{lstlisting}[caption=Пример кода]
from datetime import date

person1 = Person("Иванов Иван Иванович", "Россия", date(1946, 8, 15))
person2 = Person("Петров Сергей Александрович", "Белоруссия", date(1982, 10, 22))

print("Персона 1:")
print("Имя: ", person1.get_name())
print("Страна: ", person1.get_country())
print("Дата рождения: ", person1.get_date_of_birth())
print("Возраст: ", person1.calculate_age())

print("Персона 2:")
print("Имя: ", person2.get_name())
print("Страна: ", person2.get_country())
print("Дата рождения: ", person2.get_date_of_birth())
print("Возраст: ", person2.calculate_age())
\end{lstlisting}

\subsection*{Вывод:}
\begin{lstlisting}[caption=Ожидаемый вывод]
Персона 1:
Имя:  Иванов Иван Иванович
Страна:  Россия
Дата рождения:  1946-08-15
Возраст:  77
Персона 2:
Имя:  Петров Сергей Александрович
Страна:  Белоруссия
Дата рождения:  1982-10-22
Возраст:  41
\end{lstlisting}

% ================= Вариант 2 =================
\item
Создайте класс \texttt{Student} с закрытыми атрибутами \texttt{\_\_full\_name}, \texttt{\_\_enrollment\_date}, \texttt{\_\_major}. Реализуйте методы-геттеры и метод \texttt{study\_duration()} для вычисления количества лет с момента зачисления.

\subsection*{Инструкции:}
\begin{enumerate}
    \item Создайте класс \texttt{Student} с методом \texttt{\_\_init\_\_}.
    \item Методы-геттеры: \texttt{get\_full\_name()}, \texttt{get\_enrollment\_date()}, \texttt{get\_major()}.
    \item Метод \texttt{study\_duration()} вычисляет количество лет с зачисления.
    \item Создайте несколько экземпляров класса.
    \item Выведите данные каждого студента.
\end{enumerate}

\subsection*{Пример использования:}
\begin{lstlisting}[caption=Пример кода]
from datetime import date

student1 = Student("Сидоров Алексей", date(2018, 9, 1), "Математика")
student2 = Student("Иванова Мария", date(2021, 9, 1), "Физика")

print("Студент 1:")
print("Имя: ", student1.get_full_name())
print("Направление: ", student1.get_major())
print("Дата зачисления: ", student1.get_enrollment_date())
print("Стаж учёбы: ", student1.study_duration())

print("Студент 2:")
print("Имя: ", student2.get_full_name())
print("Направление: ", student2.get_major())
print("Дата зачисления: ", student2.get_enrollment_date())
print("Стаж учёбы: ", student2.study_duration())
\end{lstlisting}

\subsection*{Вывод:}
\begin{lstlisting}[caption=Ожидаемый вывод]
Студент 1:
Имя:  Сидоров Алексей
Направление:  Математика
Дата зачисления:  2018-09-01
Стаж учёбы:  5
Студент 2:
Имя:  Иванова Мария
Направление:  Физика
Дата зачисления:  2021-09-01
Стаж учёбы:  2
\end{lstlisting}

\item
Создайте класс \texttt{Employee} с закрытыми атрибутами \texttt{\_\_name}, \texttt{\_\_position}, \texttt{\_\_hire\_date}. Реализуйте методы-геттеры и метод \texttt{work\_experience()} для вычисления количества лет работы.

\subsection*{Инструкции:}
\begin{enumerate}
    \item Создайте класс \texttt{Employee} с методом \texttt{\_\_init\_\_}.
    \item Методы-геттеры: \texttt{get\_name()}, \texttt{get\_position()}, \texttt{get\_hire\_date()}.
    \item Метод \texttt{work\_experience()} вычисляет стаж в годах.
    \item Создайте несколько экземпляров класса.
    \item Выведите данные каждого сотрудника.
\end{enumerate}

\subsection*{Пример использования:}
\begin{lstlisting}[caption=Пример кода]
from datetime import date

emp1 = Employee("Кузнецов Дмитрий", "Инженер", date(2010, 5, 10))
emp2 = Employee("Смирнова Ольга", "Менеджер", date(2015, 8, 1))

print("Сотрудник 1:")
print("Имя: ", emp1.get_name())
print("Должность: ", emp1.get_position())
print("Дата приёма: ", emp1.get_hire_date())
print("Стаж: ", emp1.work_experience())

print("Сотрудник 2:")
print("Имя: ", emp2.get_name())
print("Должность: ", emp2.get_position())
print("Дата приёма: ", emp2.get_hire_date())
print("Стаж: ", emp2.work_experience())
\end{lstlisting}

\subsection*{Вывод:}
\begin{lstlisting}[caption=Ожидаемый вывод]
Сотрудник 1:
Имя:  Кузнецов Дмитрий
Должность:  Инженер
Дата приёма:  2010-05-10
Стаж:  17
Сотрудник 2:
Имя:  Смирнова Ольга
Должность:  Менеджер
Дата приёма:  2015-08-01
Стаж:  8
\end{lstlisting}

% ================= Вариант 4 =================
\item
Создайте класс \texttt{Book} с закрытыми атрибутами \texttt{\_\_title}, \texttt{\_\_author}, \texttt{\_\_publish\_date}. Реализуйте геттеры и метод \texttt{book\_age()} для вычисления возраста книги.

\subsection*{Инструкции:}
\begin{enumerate}
    \item Создайте класс \texttt{Book}.
    \item Методы-геттеры: \texttt{get\_title()}, \texttt{get\_author()}, \texttt{get\_publish\_date()}.
    \item Метод \texttt{book\_age()} вычисляет возраст книги.
    \item Создайте экземпляры класса.
    \item Выведите данные каждой книги.
\end{enumerate}

\subsection*{Пример использования:}
\begin{lstlisting}[caption=Пример кода]
from datetime import date

book1 = Book("Программирование на Python", "Иванов И.И.", date(2015, 3, 10))
book2 = Book("Алгебра", "Петров П.П.", date(2000, 9, 1))

print("Книга 1:")
print("Название: ", book1.get_title())
print("Автор: ", book1.get_author())
print("Дата публикации: ", book1.get_publish_date())
print("Возраст книги: ", book1.book_age())

print("Книга 2:")
print("Название: ", book2.get_title())
print("Автор: ", book2.get_author())
print("Дата публикации: ", book2.get_publish_date())
print("Возраст книги: ", book2.book_age())
\end{lstlisting}

\subsection*{Вывод:}
\begin{lstlisting}[caption=Ожидаемый вывод]
Книга 1:
Название:  Программирование на Python
Автор:  Иванов И.И.
Дата публикации:  2015-03-10
Возраст книги:  8
Книга 2:
Название:  Алгебра
Автор:  Петров П.П.
Дата публикации:  2000-09-01
Возраст книги:  23
\end{lstlisting}

% ================= Вариант 5 =================
\item
Создайте класс \texttt{Car} с закрытыми атрибутами \texttt{\_\_model}, \texttt{\_\_manufacturer}, \texttt{\_\_production\_date}. Геттеры и метод \texttt{car\_age()} для вычисления возраста автомобиля.

\subsection*{Инструкции:}
\begin{enumerate}
    \item Создайте класс \texttt{Car}.
    \item Методы-геттеры: \texttt{get\_model()}, \texttt{get\_manufacturer()}, \texttt{get\_production\_date()}.
    \item Метод \texttt{car\_age()} вычисляет возраст автомобиля.
    \item Создайте экземпляры класса.
    \item Выведите данные каждого автомобиля.
\end{enumerate}

\subsection*{Пример использования:}
\begin{lstlisting}[caption=Пример кода]
from datetime import date

car1 = Car("Camry", "Toyota", date(2012, 6, 15))
car2 = Car("Focus", "Ford", date(2018, 4, 20))

print("Автомобиль 1:")
print("Модель: ", car1.get_model())
print("Производитель: ", car1.get_manufacturer())
print("Дата выпуска: ", car1.get_production_date())
print("Возраст авто: ", car1.car_age())

print("Автомобиль 2:")
print("Модель: ", car2.get_model())
print("Производитель: ", car2.get_manufacturer())
print("Дата выпуска: ", car2.get_production_date())
print("Возраст авто: ", car2.car_age())
\end{lstlisting}

\subsection*{Вывод:}
\begin{lstlisting}[caption=Ожидаемый вывод]
Автомобиль 1:
Модель:  Camry
Производитель:  Toyota
Дата выпуска:  2012-06-15
Возраст авто:  11
Автомобиль 2:
Модель:  Focus
Производитель:  Ford
Дата выпуска:  2018-04-20
Возраст авто:  5
\end{lstlisting}

% ================= Вариант 6 =================
\item
Создайте класс \texttt{Pet} с закрытыми атрибутами \texttt{\_\_name}, \texttt{\_\_species}, \texttt{\_\_birth\_date}. Реализуйте методы-геттеры и метод \texttt{pet\_age()} для вычисления возраста питомца. Создайте несколько экземпляров и выведите их данные.

\subsection*{Инструкции:}
\begin{enumerate}
    \item Создайте класс \texttt{Pet} с методом \texttt{\_\_init\_\_}.
    \item Методы-геттеры: \texttt{get\_name()}, \texttt{get\_species()}, \texttt{get\_birth\_date()}.
    \item Метод \texttt{pet\_age()} вычисляет возраст питомца в годах.
    \item Создайте несколько экземпляров класса.
    \item Выведите данные каждого питомца через методы класса.
\end{enumerate}

\subsection*{Пример использования:}
\begin{lstlisting}[caption=Пример кода]
from datetime import date

pet1 = Pet("Барсик", "Кошка", date(2018, 5, 12))
pet2 = Pet("Рекс", "Собака", date(2015, 8, 1))

print("Питомец 1:")
print("Имя: ", pet1.get_name())
print("Вид: ", pet1.get_species())
print("Дата рождения: ", pet1.get_birth_date())
print("Возраст: ", pet1.pet_age())

print("Питомец 2:")
print("Имя: ", pet2.get_name())
print("Вид: ", pet2.get_species())
print("Дата рождения: ", pet2.get_birth_date())
print("Возраст: ", pet2.pet_age())
\end{lstlisting}

\subsection*{Вывод:}
\begin{lstlisting}[caption=Ожидаемый вывод]
Питомец 1:
Имя:  Барсик
Вид:  Кошка
Дата рождения:  2018-05-12
Возраст:  7
Питомец 2:
Имя:  Рекс
Вид:  Собака
Дата рождения:  2015-08-01
Возраст:  10
\end{lstlisting}

% ================= Вариант 7 =================
\item
Создайте класс \texttt{Membership} с закрытыми атрибутами \texttt{\_\_member\_name}, \texttt{\_\_membership\_type}, \texttt{\_\_join\_date}. Реализуйте методы-геттеры и метод \texttt{membership\_duration()} для вычисления длительности членства в годах.

\subsection*{Инструкции:}
\begin{enumerate}
    \item Создайте класс \texttt{Membership}.
    \item Методы-геттеры: \texttt{get\_member\_name()}, \texttt{get\_membership\_type()}, \texttt{get\_join\_date()}.
    \item Метод \texttt{membership\_duration()} вычисляет длительность членства в годах.
    \item Создайте несколько экземпляров.
    \item Выведите данные каждого участника.
\end{enumerate}

\subsection*{Пример использования:}
\begin{lstlisting}[caption=Пример кода]
from datetime import date

member1 = Membership("Иванов Иван", "Золотой", date(2018, 3, 15))
member2 = Membership("Петров Петр", "Серебряный", date(2020, 6, 1))

print("Член 1:")
print("Имя: ", member1.get_member_name())
print("Тип членства: ", member1.get_membership_type())
print("Дата вступления: ", member1.get_join_date())
print("Длительность членства: ", member1.membership_duration())

print("Член 2:")
print("Имя: ", member2.get_member_name())
print("Тип членства: ", member2.get_membership_type())
print("Дата вступления: ", member2.get_join_date())
print("Длительность членства: ", member2.membership_duration())
\end{lstlisting}

\subsection*{Вывод:}
\begin{lstlisting}[caption=Ожидаемый вывод]
Член 1:
Имя:  Иванов Иван
Тип членства:  Золотой
Дата вступления:  2018-03-15
Длительность членства:  5
Член 2:
Имя:  Петров Петр
Тип членства:  Серебряный
Дата вступления:  2020-06-01
Длительность членства:  3
\end{lstlisting}

% ================= Вариант 8 =================
\item
Создайте класс \texttt{Event} с закрытыми атрибутами \texttt{\_\_event\_name}, \texttt{\_\_location}, \texttt{\_\_event\_date}. Реализуйте методы-геттеры и метод \texttt{days\_until\_event()} для вычисления количества дней до события.

\subsection*{Инструкции:}
\begin{enumerate}
    \item Создайте класс \texttt{Event}.
    \item Методы-геттеры: \texttt{get\_event\_name()}, \texttt{get\_location()}, \texttt{get\_event\_date()}.
    \item Метод \texttt{days\_until\_event()} вычисляет дни до события.
    \item Создайте несколько экземпляров.
    \item Выведите данные каждого события.
\end{enumerate}

\subsection*{Пример использования:}
\begin{lstlisting}[caption=Пример кода]
from datetime import date

event1 = Event("Концерт", "Стадион", date(2025, 12, 1))
event2 = Event("Выставка", "Музей", date(2025, 11, 20))

print("Событие 1:")
print("Название: ", event1.get_event_name())
print("Место: ", event1.get_location())
print("Дата: ", event1.get_event_date())
print("Дней до события: ", event1.days_until_event())

print("Событие 2:")
print("Название: ", event2.get_event_name())
print("Место: ", event2.get_location())
print("Дата: ", event2.get_event_date())
print("Дней до события: ", event2.days_until_event())
\end{lstlisting}

\subsection*{Вывод:}
\begin{lstlisting}[caption=Ожидаемый вывод]
Событие 1:
Название:  Концерт
Место:  Стадион
Дата:  2025-12-01
Дней до события:  112
Событие 2:
Название:  Выставка
Место:  Музей
Дата:  2025-11-20
Дней до события:  101
\end{lstlisting}

% ================= Вариант 9 =================
\item
Создайте класс \texttt{Course} с закрытыми атрибутами \texttt{\_\_course\_name}, \texttt{\_\_start\_date}, \texttt{\_\_duration\_weeks}. Реализуйте методы-геттеры и метод \texttt{weeks\_elapsed()} для вычисления прошедших недель с начала курса.

\subsection*{Инструкции:}
\begin{enumerate}
    \item Создайте класс \texttt{Course}.
    \item Методы-геттеры: \texttt{get\_course\_name()}, \texttt{get\_start\_date()}, \texttt{get\_duration\_weeks()}.
    \item Метод \texttt{weeks\_elapsed()} вычисляет количество прошедших недель.
    \item Создайте несколько экземпляров.
    \item Выведите данные каждого курса.
\end{enumerate}

\subsection*{Пример использования:}
\begin{lstlisting}[caption=Пример кода]
from datetime import date

course1 = Course("Python", date(2025, 1, 1), 12)
course2 = Course("Алгебра", date(2025, 2, 1), 10)

print("Курс 1:")
print("Название: ", course1.get_course_name())
print("Дата начала: ", course1.get_start_date())
print("Продолжительность (недель): ", course1.get_duration_weeks())
print("Прошло недель: ", course1.weeks_elapsed())

print("Курс 2:")
print("Название: ", course2.get_course_name())
print("Дата начала: ", course2.get_start_date())
print("Продолжительность (недель): ", course2.get_duration_weeks())
print("Прошло недель: ", course2.weeks_elapsed())
\end{lstlisting}

\subsection*{Вывод:}
\begin{lstlisting}[caption=Ожидаемый вывод]
Курс 1:
Название:  Python
Дата начала:  2025-01-01
Продолжительность (недель):  12
Прошло недель:  36
Курс 2:
Название:  Алгебра
Дата начала:  2025-02-01
Продолжительность (недель):  10
Прошло недель:  31
\end{lstlisting}

% ================= Вариант 10 =================
\item
Создайте класс \texttt{Subscription} с закрытыми атрибутами \texttt{\_\_user}, \texttt{\_\_plan}, \texttt{\_\_start\_date}. Реализуйте методы-геттеры и метод \texttt{subscription\_age()} для вычисления возраста подписки в годах.

\subsection*{Инструкции:}
\begin{enumerate}
    \item Создайте класс \texttt{Subscription}.
    \item Методы-геттеры: \texttt{get\_user()}, \texttt{get\_plan()}, \texttt{get\_start\_date()}.
    \item Метод \texttt{subscription\_age()} вычисляет возраст подписки.
    \item Создайте несколько экземпляров.
    \item Выведите данные каждой подписки.
\end{enumerate}

\subsection*{Пример использования:}
\begin{lstlisting}[caption=Пример кода]
from datetime import date

sub1 = Subscription("Иванов И.", "Premium", date(2021, 3, 1))
sub2 = Subscription("Петров П.", "Basic", date(2022, 7, 15))

print("Подписка 1:")
print("Пользователь: ", sub1.get_user())
print("План: ", sub1.get_plan())
print("Дата начала: ", sub1.get_start_date())
print("Возраст подписки: ", sub1.subscription_age())

print("Подписка 2:")
print("Пользователь: ", sub2.get_user())
print("План: ", sub2.get_plan())
print("Дата начала: ", sub2.get_start_date())
print("Возраст подписки: ", sub2.subscription_age())
\end{lstlisting}

\subsection*{Вывод:}
\begin{lstlisting}[caption=Ожидаемый вывод]
Подписка 1:
Пользователь:  Иванов И.
План:  Premium
Дата начала:  2021-03-01
Возраст подписки:  4
Подписка 2:
Пользователь:  Петров П.
План:  Basic
Дата начала:  2022-07-15
Возраст подписки:  3
\end{lstlisting}

% ================= Вариант 11 =================
\item
Создайте класс \texttt{Flight} с закрытыми атрибутами \texttt{\_\_flight\_number}, \texttt{\_\_departure\_date}, \texttt{\_\_destination}. Реализуйте методы-геттеры и метод \texttt{days\_until\_departure()} для вычисления количества дней до вылета.

\subsection*{Инструкции:}
\begin{enumerate}
    \item Создайте класс \texttt{Flight}.
    \item Методы-геттеры: \texttt{get\_flight\_number()}, \texttt{get\_departure\_date()}, \texttt{get\_destination()}.
    \item Метод \texttt{days\_until\_departure()} вычисляет количество дней до вылета.
    \item Создайте несколько экземпляров.
    \item Выведите данные каждого рейса.
\end{enumerate}

\subsection*{Пример использования:}
\begin{lstlisting}[caption=Пример кода]
from datetime import date

flight1 = Flight("SU123", date(2025, 10, 15), "Москва")
flight2 = Flight("AF456", date(2025, 11, 1), "Париж")

print("Рейс 1:")
print("Номер: ", flight1.get_flight_number())
print("Дата вылета: ", flight1.get_departure_date())
print("Пункт назначения: ", flight1.get_destination())
print("Дней до вылета: ", flight1.days_until_departure())

print("Рейс 2:")
print("Номер: ", flight2.get_flight_number())
print("Дата вылета: ", flight2.get_departure_date())
print("Пункт назначения: ", flight2.get_destination())
print("Дней до вылета: ", flight2.days_until_departure())
\end{lstlisting}

\subsection*{Вывод:}
\begin{lstlisting}[caption=Ожидаемый вывод]
Рейс 1:
Номер:  SU123
Дата вылета:  2025-10-15
Пункт назначения:  Москва
Дней до вылета:  54
Рейс 2:
Номер:  AF456
Дата вылета:  2025-11-01
Пункт назначения:  Париж
Дней до вылета:  71
\end{lstlisting}

% ================= Вариант 12 =================
\item
Создайте класс \texttt{Project} с закрытыми атрибутами \texttt{\_\_project\_name}, \texttt{\_\_start\_date}, \texttt{\_\_deadline}. Реализуйте методы-геттеры и метод \texttt{days\_remaining()} для вычисления количества дней до завершения проекта.

\subsection*{Инструкции:}
\begin{enumerate}
    \item Создайте класс \texttt{Project}.
    \item Методы-геттеры: \texttt{get\_project\_name()}, \texttt{get\_start\_date()}, \texttt{get\_deadline()}.
    \item Метод \texttt{days\_remaining()} вычисляет дни до дедлайна.
    \item Создайте несколько экземпляров.
    \item Выведите данные каждого проекта.
\end{enumerate}

\subsection*{Пример использования:}
\begin{lstlisting}[caption=Пример кода]
from datetime import date

project1 = Project("Разработка сайта", date(2025, 9, 1), date(2025, 12, 1))
project2 = Project("Анализ данных", date(2025, 10, 1), date(2025, 11, 30))

print("Проект 1:")
print("Название: ", project1.get_project_name())
print("Дата начала: ", project1.get_start_date())
print("Дедлайн: ", project1.get_deadline())
print("Дней до завершения: ", project1.days_remaining())

print("Проект 2:")
print("Название: ", project2.get_project_name())
print("Дата начала: ", project2.get_start_date())
print("Дедлайн: ", project2.get_deadline())
print("Дней до завершения: ", project2.days_remaining())
\end{lstlisting}

\subsection*{Вывод:}
\begin{lstlisting}[caption=Ожидаемый вывод]
Проект 1:
Название:  Разработка сайта
Дата начала:  2025-09-01
Дедлайн:  2025-12-01
Дней до завершения:  101
Проект 2:
Название:  Анализ данных
Дата начала:  2025-10-01
Дедлайн:  2025-11-30
Дней до завершения:  91
\end{lstlisting}

% ================= Вариант 13 =================
\item
Создайте класс \texttt{Doctor} с закрытыми атрибутами \texttt{\_\_full\_name}, \texttt{\_\_specialty}, \texttt{\_\_birth\_date}. Реализуйте методы-геттеры и метод \texttt{calculate\_age()} для вычисления возраста врача.

\subsection*{Инструкции:}
\begin{enumerate}
    \item Создайте класс \texttt{Doctor}.
    \item Методы-геттеры: \texttt{get\_full\_name()}, \texttt{get\_specialty()}, \texttt{get\_birth\_date()}.
    \item Метод \texttt{calculate\_age()} вычисляет возраст.
    \item Создайте несколько экземпляров.
    \item Выведите данные каждого врача.
\end{enumerate}

\subsection*{Пример использования:}
\begin{lstlisting}[caption=Пример кода]
from datetime import date

doc1 = Doctor("Иванов И.И.", "Терапевт", date(1980, 5, 12))
doc2 = Doctor("Петров П.П.", "Хирург", date(1975, 8, 1))

print("Врач 1:")
print("Имя: ", doc1.get_full_name())
print("Специальность: ", doc1.get_specialty())
print("Дата рождения: ", doc1.get_birth_date())
print("Возраст: ", doc1.calculate_age())

print("Врач 2:")
print("Имя: ", doc2.get_full_name())
print("Специальность: ", doc2.get_specialty())
print("Дата рождения: ", doc2.get_birth_date())
print("Возраст: ", doc2.calculate_age())
\end{lstlisting}

\subsection*{Вывод:}
\begin{lstlisting}[caption=Ожидаемый вывод]
Врач 1:
Имя:  Иванов И.И.
Специальность:  Терапевт
Дата рождения:  1980-05-12
Возраст:  45
Врач 2:
Имя:  Петров П.П.
Специальность:  Хирург
Дата рождения:  1975-08-01
Возраст:  50
\end{lstlisting}

% ================= Вариант 14 =================
\item
Создайте класс \texttt{Patient} с закрытыми атрибутами \texttt{\_\_full\_name}, \texttt{\_\_admission\_date}, \texttt{\_\_diagnosis}. Реализуйте методы-геттеры и метод \texttt{hospital\_stay()} для вычисления количества дней пребывания в больнице.

\subsection*{Инструкции:}
\begin{enumerate}
    \item Создайте класс \texttt{Patient}.
    \item Методы-геттеры: \texttt{get\_full\_name()}, \texttt{get\_admission\_date()}, \texttt{get\_diagnosis()}.
    \item Метод \texttt{hospital\_stay()} вычисляет дни пребывания.
    \item Создайте несколько экземпляров.
    \item Выведите данные каждого пациента.
\end{enumerate}

\subsection*{Пример использования:}
\begin{lstlisting}[caption=Пример кода]
from datetime import date

patient1 = Patient("Сидоров С.С.", date(2025, 9, 1), "ОРВИ")
patient2 = Patient("Кузнецов К.К.", date(2025, 8, 28), "Грипп")

print("Пациент 1:")
print("Имя: ", patient1.get_full_name())
print("Дата госпитализации: ", patient1.get_admission_date())
print("Диагноз: ", patient1.get_diagnosis())
print("Дней в больнице: ", patient1.hospital_stay())

print("Пациент 2:")
print("Имя: ", patient2.get_full_name())
print("Дата госпитализации: ", patient2.get_admission_date())
print("Диагноз: ", patient2.get_diagnosis())
print("Дней в больнице: ", patient2.hospital_stay())
\end{lstlisting}

\subsection*{Вывод:}
\begin{lstlisting}[caption=Ожидаемый вывод]
Пациент 1:
Имя:  Сидоров С.С.
Дата госпитализации:  2025-09-01
Диагноз:  ОРВИ
Дней в больнице:  15
Пациент 2:
Имя:  Кузнецов К.К.
Дата госпитализации:  2025-08-28
Диагноз:  Грипп
Дней в больнице:  19
\end{lstlisting}

% ================= Вариант 15 =================
\item
Создайте класс \texttt{Concert} с закрытыми атрибутами \texttt{\_\_artist}, \texttt{\_\_venue}, \texttt{\_\_concert\_date}. Реализуйте методы-геттеры и метод \texttt{days\_until\_concert()}.

\subsection*{Инструкции:}
\begin{enumerate}
    \item Создайте класс \texttt{Concert}.
    \item Методы-геттеры: \texttt{get\_artist()}, \texttt{get\_venue()}, \texttt{get\_concert\_date()}.
    \item Метод \texttt{days\_until\_concert()} вычисляет дни до концерта.
    \item Создайте несколько экземпляров.
    \item Выведите данные каждого концерта.
\end{enumerate}

\subsection*{Пример использования:}
\begin{lstlisting}[caption=Пример кода]
from datetime import date

concert1 = Concert("Imagine Dragons", "Лужники", date(2025, 10, 10))
concert2 = Concert("Coldplay", "O2 Arena", date(2025, 11, 5))

print("Концерт 1:")
print("Исполнитель: ", concert1.get_artist())
print("Место: ", concert1.get_venue())
print("Дата: ", concert1.get_concert_date())
print("Дней до концерта: ", concert1.days_until_concert())

print("Концерт 2:")
print("Исполнитель: ", concert2.get_artist())
print("Место: ", concert2.get_venue())
print("Дата: ", concert2.get_concert_date())
print("Дней до концерта: ", concert2.days_until_concert())
\end{lstlisting}

\subsection*{Вывод:}
\begin{lstlisting}[caption=Ожидаемый вывод]
Концерт 1:
Исполнитель:  Imagine Dragons
Место:  Лужники
Дата:  2025-10-10
Дней до концерта:  49
Концерт 2:
Исполнитель:  Coldplay
Место:  O2 Arena
Дата:  2025-11-05
Дней до концерта:  75
\end{lstlisting}

% ================= Вариант 16 =================
\item
Создайте класс \texttt{Holiday} с закрытыми атрибутами \texttt{\_\_name}, \texttt{\_\_country}, \texttt{\_\_holiday\_date}. Реализуйте методы-геттеры и метод \texttt{days\_until\_holiday()}.

\subsection*{Инструкции:}
\begin{enumerate}
    \item Создайте класс \texttt{Holiday}.
    \item Методы-геттеры: \texttt{get\_name()}, \texttt{get\_country()}, \texttt{get\_holiday\_date()}.
    \item Метод \texttt{days\_until\_holiday()} вычисляет дни до праздника.
    \item Создайте несколько экземпляров.
    \item Выведите данные каждого праздника.
\end{enumerate}

\subsection*{Пример использования:}
\begin{lstlisting}[caption=Пример кода]
from datetime import date

holiday1 = Holiday("Новый Год", "Россия", date(2026, 1, 1))
holiday2 = Holiday("Рождество", "Германия", date(2025, 12, 25))

print("Праздник 1:")
print("Название: ", holiday1.get_name())
print("Страна: ", holiday1.get_country())
print("Дата: ", holiday1.get_holiday_date())
print("Дней до праздника: ", holiday1.days_until_holiday())

print("Праздник 2:")
print("Название: ", holiday2.get_name())
print("Страна: ", holiday2.get_country())
print("Дата: ", holiday2.get_holiday_date())
print("Дней до праздника: ", holiday2.days_until_holiday())
\end{lstlisting}

\subsection*{Вывод:}
\begin{lstlisting}[caption=Ожидаемый вывод]
Праздник 1:
Название:  Новый Год
Страна:  Россия
Дата:  2026-01-01
Дней до праздника:  83
Праздник 2:
Название:  Рождество
Страна:  Германия
Дата:  2025-12-25
Дней до праздника:  67
\end{lstlisting}
% ================= Вариант 17 =================
\item
Создайте класс \texttt{Employee} с закрытыми атрибутами \texttt{\_\_full\_name}, \texttt{\_\_position}, \texttt{\_\_hire\_date}. Реализуйте методы-геттеры и метод \texttt{years\_worked()} для вычисления стажа работы в годах.

\subsection*{Инструкции:}
\begin{enumerate}
    \item Создайте класс \texttt{Employee}.
    \item Методы-геттеры: \texttt{get\_full\_name()}, \texttt{get\_position()}, \texttt{get\_hire\_date()}.
    \item Метод \texttt{years\_worked()} вычисляет стаж в годах.
    \item Создайте несколько экземпляров.
    \item Выведите данные каждого сотрудника.
\end{enumerate}

\subsection*{Пример использования:}
\begin{lstlisting}[caption=Пример кода]
from datetime import date

emp1 = Employee("Иванов И.И.", "Менеджер", date(2015, 4, 1))
emp2 = Employee("Петров П.П.", "Разработчик", date(2018, 7, 15))

print("Сотрудник 1:")
print("Имя: ", emp1.get_full_name())
print("Должность: ", emp1.get_position())
print("Дата приема: ", emp1.get_hire_date())
print("Стаж: ", emp1.years_worked())

print("Сотрудник 2:")
print("Имя: ", emp2.get_full_name())
print("Должность: ", emp2.get_position())
print("Дата приема: ", emp2.get_hire_date())
print("Стаж: ", emp2.years_worked())
\end{lstlisting}

\subsection*{Вывод:}
\begin{lstlisting}[caption=Ожидаемый вывод]
Сотрудник 1:
Имя:  Иванов И.И.
Должность:  Менеджер
Дата приема:  2015-04-01
Стаж:  10
Сотрудник 2:
Имя:  Петров П.П.
Должность:  Разработчик
Дата приема:  2018-07-15
Стаж:  7
\end{lstlisting}

% ================= Вариант 18 =================
\item
Создайте класс \texttt{LibraryBook} с закрытыми атрибутами \texttt{\_\_title}, \texttt{\_\_author}, \texttt{\_\_publication\_date}. Реализуйте методы-геттеры и метод \texttt{book\_age()} для вычисления возраста книги.

\subsection*{Инструкции:}
\begin{enumerate}
    \item Создайте класс \texttt{LibraryBook}.
    \item Методы-геттеры: \texttt{get\_title()}, \texttt{get\_author()}, \texttt{get\_publication\_date()}.
    \item Метод \texttt{book\_age()} вычисляет возраст книги в годах.
    \item Создайте несколько экземпляров.
    \item Выведите данные каждой книги.
\end{enumerate}

\subsection*{Пример использования:}
\begin{lstlisting}[caption=Пример кода]
from datetime import date

book1 = LibraryBook("Война и мир", "Толстой", date(1869, 1, 1))
book2 = LibraryBook("Мастер и Маргарита", "Булгаков", date(1967, 5, 1))

print("Книга 1:")
print("Название: ", book1.get_title())
print("Автор: ", book1.get_author())
print("Дата публикации: ", book1.get_publication_date())
print("Возраст книги: ", book1.book_age())

print("Книга 2:")
print("Название: ", book2.get_title())
print("Автор: ", book2.get_author())
print("Дата публикации: ", book2.get_publication_date())
print("Возраст книги: ", book2.book_age())
\end{lstlisting}

\subsection*{Вывод:}
\begin{lstlisting}[caption=Ожидаемый вывод]
Книга 1:
Название:  Война и мир
Автор:  Толстой
Дата публикации:  1869-01-01
Возраст книги:  156
Книга 2:
Название:  Мастер и Маргарита
Автор:  Булгаков
Дата публикации:  1967-05-01
Возраст книги:  59
\end{lstlisting}

% ================= Вариант 19 =================
\item
Создайте класс \texttt{Vehicle} с закрытыми атрибутами \texttt{\_\_brand}, \texttt{\_\_model}, \texttt{\_\_manufacture\_date}. Реализуйте методы-геттеры и метод \texttt{vehicle\_age()}.

\subsection*{Инструкции:}
\begin{enumerate}
    \item Создайте класс \texttt{Vehicle}.
    \item Методы-геттеры: \texttt{get\_brand()}, \texttt{get\_model()}, \texttt{get\_manufacture\_date()}.
    \item Метод \texttt{vehicle\_age()} вычисляет возраст транспортного средства.
    \item Создайте несколько экземпляров.
    \item Выведите данные каждого транспортного средства.
\end{enumerate}

\subsection*{Пример использования:}
\begin{lstlisting}[caption=Пример кода]
from datetime import date

vehicle1 = Vehicle("Toyota", "Camry", date(2015, 5, 1))
vehicle2 = Vehicle("BMW", "X5", date(2018, 3, 10))

print("Транспорт 1:")
print("Марка: ", vehicle1.get_brand())
print("Модель: ", vehicle1.get_model())
print("Дата производства: ", vehicle1.get_manufacture_date())
print("Возраст: ", vehicle1.vehicle_age())

print("Транспорт 2:")
print("Марка: ", vehicle2.get_brand())
print("Модель: ", vehicle2.get_model())
print("Дата производства: ", vehicle2.get_manufacture_date())
print("Возраст: ", vehicle2.vehicle_age())
\end{lstlisting}

\subsection*{Вывод:}
\begin{lstlisting}[caption=Ожидаемый вывод]
Транспорт 1:
Марка:  Toyota
Модель:  Camry
Дата производства:  2015-05-01
Возраст:  10
Транспорт 2:
Марка:  BMW
Модель:  X5
Дата производства:  2018-03-10
Возраст:  7
\end{lstlisting}

% ================= Вариант 20 =================
\item
Создайте класс \texttt{Student} с закрытыми атрибутами \texttt{\_\_full\_name}, \texttt{\_\_enrollment\_date}, \texttt{\_\_major}. Реализуйте методы-геттеры и метод \texttt{study\_years()}.

\subsection*{Инструкции:}
\begin{enumerate}
    \item Создайте класс \texttt{Student}.
    \item Методы-геттеры: \texttt{get\_full\_name()}, \texttt{get\_enrollment\_date()}, \texttt{get\_major()}.
    \item Метод \texttt{study\_years()} вычисляет количество лет учебы.
    \item Создайте несколько экземпляров.
    \item Выведите данные каждого студента.
\end{enumerate}

\subsection*{Пример использования:}
\begin{lstlisting}[caption=Пример кода]
from datetime import date

student1 = Student("Иванов И.И.", date(2020, 9, 1), "Математика")
student2 = Student("Петров П.П.", date(2021, 9, 1), "Физика")

print("Студент 1:")
print("Имя: ", student1.get_full_name())
print("Дата зачисления: ", student1.get_enrollment_date())
print("Специальность: ", student1.get_major())
print("Лет учебы: ", student1.study_years())

print("Студент 2:")
print("Имя: ", student2.get_full_name())
print("Дата зачисления: ", student2.get_enrollment_date())
print("Специальность: ", student2.get_major())
print("Лет учебы: ", student2.study_years())
\end{lstlisting}

\subsection*{Вывод:}
\begin{lstlisting}[caption=Ожидаемый вывод]
Студент 1:
Имя:  Иванов И.И.
Дата зачисления:  2020-09-01
Специальность:  Математика
Лет учебы:  5
Студент 2:
Имя:  Петров П.П.
Дата зачисления:  2021-09-01
Специальность:  Физика
Лет учебы:  4
\end{lstlisting}

% ================= Вариант 21 =================
\item
Создайте класс \texttt{Ticket} с закрытыми атрибутами \texttt{\_\_ticket\_number}, \texttt{\_\_issue\_date}, \texttt{\_\_valid\_until}. Реализуйте методы-геттеры и метод \texttt{days\_valid()}.

\subsection*{Инструкции:}
\begin{enumerate}
    \item Создайте класс \texttt{Ticket}.
    \item Методы-геттеры: \texttt{get\_ticket\_number()}, \texttt{get\_issue\_date()}, \texttt{get\_valid\_until()}.
    \item Метод \texttt{days\_valid()} вычисляет дни до окончания действия билета.
    \item Создайте несколько экземпляров.
    \item Выведите данные каждого билета.
\end{enumerate}

\subsection*{Пример использования:}
\begin{lstlisting}[caption=Пример кода]
from datetime import date

ticket1 = Ticket("A123", date(2025, 9, 1), date(2025, 12, 1))
ticket2 = Ticket("B456", date(2025, 10, 1), date(2026, 1, 1))

print("Билет 1:")
print("Номер: ", ticket1.get_ticket_number())
print("Дата выдачи: ", ticket1.get_issue_date())
print("Действителен до: ", ticket1.get_valid_until())
print("Дней до окончания: ", ticket1.days_valid())

print("Билет 2:")
print("Номер: ", ticket2.get_ticket_number())
print("Дата выдачи: ", ticket2.get_issue_date())
print("Действителен до: ", ticket2.get_valid_until())
print("Дней до окончания: ", ticket2.days_valid())
\end{lstlisting}

\subsection*{Вывод:}
\begin{lstlisting}[caption=Ожидаемый вывод]
Билет 1:
Номер:  A123
Дата выдачи:  2025-09-01
Действителен до:  2025-12-01
Дней до окончания:  91
Билет 2:
Номер:  B456
Дата выдачи:  2025-10-01
Действителен до:  2026-01-01
Дней до окончания:  92
\end{lstlisting}

% ================= Вариант 22 =================
\item
Создайте класс \texttt{Appointment} с закрытыми атрибутами \texttt{\_\_client}, \texttt{\_\_service}, \texttt{\_\_appointment\_date}. Реализуйте методы-геттеры и метод \texttt{days\_until\_appointment()}.
\subsection*{Инструкции:}
\begin{enumerate}
    \item Создайте класс \texttt{Appointment}.
    \item Методы-геттеры: \texttt{get\_client()}, \texttt{get\_service()}, \texttt{get\_appointment\_date()}.
    \item Метод \texttt{days\_until\_appointment()} вычисляет дни до приёма.
    \item Создайте несколько экземпляров.
    \item Выведите данные каждого приёма.
\end{enumerate}

\subsection*{Пример использования:}
\begin{lstlisting}[caption=Пример кода]
from datetime import date

app1 = Appointment("Иванов И.", "Массаж", date(2025, 10, 5))
app2 = Appointment("Петров П.", "Стрижка", date(2025, 10, 15))

print("Приём 1:")
print("Клиент: ", app1.get_client())
print("Услуга: ", app1.get_service())
print("Дата: ", app1.get_appointment_date())
print("Дней до приёма: ", app1.days_until_appointment())

print("Приём 2:")
print("Клиент: ", app2.get_client())
print("Услуга: ", app2.get_service())
print("Дата: ", app2.get_appointment_date())
print("Дней до приёма: ", app2.days_until_appointment())
\end{lstlisting}

\subsection*{Вывод:}
\begin{lstlisting}[caption=Ожидаемый вывод]
Приём 1:
Клиент:  Иванов И.
Услуга:  Массаж
Дата:  2025-10-05
Дней до приёма:  44
Приём 2:
Клиент:  Петров П.
Услуга:  Стрижка
Дата:  2025-10-15
Дней до приёма:  54
\end{lstlisting}
% ================= Вариант 23 =================
\item
Создайте класс \texttt{Subscription} с закрытыми атрибутами \texttt{\_\_subscriber}, \texttt{\_\_start\_date}, \texttt{\_\_end\_date}. Реализуйте методы-геттеры и метод \texttt{days\_remaining()}.

\subsection*{Инструкции:}
\begin{enumerate}
    \item Создайте класс \texttt{Subscription}.
    \item Методы-геттеры: \texttt{get\_subscriber()}, \texttt{get\_start\_date()}, \texttt{get\_end\_date()}.
    \item Метод \texttt{days\_remaining()} вычисляет дни до окончания подписки.
    \item Создайте несколько экземпляров.
    \item Выведите данные каждой подписки.
\end{enumerate}

\subsection*{Пример использования:}
\begin{lstlisting}[caption=Пример кода]
from datetime import date

sub1 = Subscription("Иванов И.", date(2025, 1, 1), date(2025, 12, 31))
sub2 = Subscription("Петров П.", date(2025, 6, 1), date(2026, 5, 31))

print("Подписка 1:")
print("Абонент: ", sub1.get_subscriber())
print("Дата начала: ", sub1.get_start_date())
print("Дата окончания: ", sub1.get_end_date())
print("Дней до окончания: ", sub1.days_remaining())

print("Подписка 2:")
print("Абонент: ", sub2.get_subscriber())
print("Дата начала: ", sub2.get_start_date())
print("Дата окончания: ", sub2.get_end_date())
print("Дней до окончания: ", sub2.days_remaining())
\end{lstlisting}

\subsection*{Вывод:}
\begin{lstlisting}[caption=Ожидаемый вывод]
Подписка 1:
Абонент:  Иванов И.
Дата начала:  2025-01-01
Дата окончания:  2025-12-31
Дней до окончания:  113
Подписка 2:
Абонент:  Петров П.
Дата начала:  2025-06-01
Дата окончания:  2026-05-31
Дней до окончания:  245
\end{lstlisting}

% ================= Вариант 24 =================
\item
Создайте класс \texttt{MembershipCard} с закрытыми атрибутами \texttt{\_\_owner}, \texttt{\_\_issue\_date}, \texttt{\_\_expiry\_date}. Реализуйте методы-геттеры и метод \texttt{days\_until\_expiry()}.

\subsection*{Инструкции:}
\begin{enumerate}
    \item Создайте класс \texttt{MembershipCard}.
    \item Методы-геттеры: \texttt{get\_owner()}, \texttt{get\_issue\_date()}, \texttt{get\_expiry\_date()}.
    \item Метод \texttt{days\_until\_expiry()} вычисляет дни до истечения действия карты.
    \item Создайте несколько экземпляров.
    \item Выведите данные каждой карты.
\end{enumerate}

\subsection*{Пример использования:}
\begin{lstlisting}[caption=Пример кода]
from datetime import date

card1 = MembershipCard("Иванов И.", date(2025, 1, 1), date(2026, 1, 1))
card2 = MembershipCard("Петров П.", date(2025, 5, 1), date(2026, 5, 1))

print("Карта 1:")
print("Владелец: ", card1.get_owner())
print("Дата выдачи: ", card1.get_issue_date())
print("Срок действия: ", card1.get_expiry_date())
print("Дней до окончания: ", card1.days_until_expiry())

print("Карта 2:")
print("Владелец: ", card2.get_owner())
print("Дата выдачи: ", card2.get_issue_date())
print("Срок действия: ", card2.get_expiry_date())
print("Дней до окончания: ", card2.days_until_expiry())
\end{lstlisting}

\subsection*{Вывод:}
\begin{lstlisting}[caption=Ожидаемый вывод]
Карта 1:
Владелец:  Иванов И.
Дата выдачи:  2025-01-01
Срок действия:  2026-01-01
Дней до окончания:  113
Карта 2:
Владелец:  Петров П.
Дата выдачи:  2025-05-01
Срок действия:  2026-05-01
Дней до окончания:  204
\end{lstlisting}

% ================= Вариант 25 =================
\item
Создайте класс \texttt{Event} с закрытыми атрибутами \texttt{\_\_title}, \texttt{\_\_location}, \texttt{\_\_event\_date}. Реализуйте методы-геттеры и метод \texttt{days\_until\_event()}.

\subsection*{Инструкции:}
\begin{enumerate}
    \item Создайте класс \texttt{Event}.
    \item Методы-геттеры: \texttt{get\_title()}, \texttt{get\_location()}, \texttt{get\_event\_date()}.
    \item Метод \texttt{days\_until\_event()} вычисляет дни до события.
    \item Создайте несколько экземпляров.
    \item Выведите данные каждого события.
\end{enumerate}

\subsection*{Пример использования:}
\begin{lstlisting}[caption=Пример кода]
from datetime import date

event1 = Event("Фестиваль науки", "Москва", date(2025, 10, 20))
event2 = Event("Конференция IT", "Санкт-Петербург", date(2025, 11, 10))

print("Событие 1:")
print("Название: ", event1.get_title())
print("Место: ", event1.get_location())
print("Дата: ", event1.get_event_date())
print("Дней до события: ", event1.days_until_event())

print("Событие 2:")
print("Название: ", event2.get_title())
print("Место: ", event2.get_location())
print("Дата: ", event2.get_event_date())
print("Дней до события: ", event2.days_until_event())
\end{lstlisting}

\subsection*{Вывод:}
\begin{lstlisting}[caption=Ожидаемый вывод]
Событие 1:
Название:  Фестиваль науки
Место:  Москва
Дата:  2025-10-20
Дней до события:  59
Событие 2:
Название:  Конференция IT
Место:  Санкт-Петербург
Дата:  2025-11-10
Дней до события:  80
\end{lstlisting}

% ================= Вариант 26 =================
\item
Создайте класс \texttt{CarRental} с закрытыми атрибутами \texttt{\_\_client}, \texttt{\_\_rental\_date}, \texttt{\_\_return\_date}. Реализуйте методы-геттеры и метод \texttt{rental\_duration()}.

\subsection*{Инструкции:}
\begin{enumerate}
    \item Создайте класс \texttt{CarRental}.
    \item Методы-геттеры: \texttt{get\_client()}, \texttt{get\_rental\_date()}, \texttt{get\_return\_date()}.
    \item Метод \texttt{rental\_duration()} вычисляет длительность аренды в днях.
    \item Создайте несколько экземпляров.
    \item Выведите данные каждой аренды.
\end{enumerate}

\subsection*{Пример использования:}
\begin{lstlisting}[caption=Пример кода]
from datetime import date

rental1 = CarRental("Иванов И.", date(2025, 10, 1), date(2025, 10, 10))
rental2 = CarRental("Петров П.", date(2025, 11, 1), date(2025, 11, 5))

print("Аренда 1:")
print("Клиент: ", rental1.get_client())
print("Дата аренды: ", rental1.get_rental_date())
print("Дата возврата: ", rental1.get_return_date())
print("Длительность аренды: ", rental1.rental_duration())

print("Аренда 2:")
print("Клиент: ", rental2.get_client())
print("Дата аренды: ", rental2.get_rental_date())
print("Дата возврата: ", rental2.get_return_date())
print("Длительность аренды: ", rental2.rental_duration())
\end{lstlisting}

\subsection*{Вывод:}
\begin{lstlisting}[caption=Ожидаемый вывод]
Аренда 1:
Клиент:  Иванов И.
Дата аренды:  2025-10-01
Дата возврата:  2025-10-10
Длительность аренды:  9
Аренда 2:
Клиент:  Петров П.
Дата аренды:  2025-11-01
Дата возврата:  2025-11-05
Длительность аренды:  4
\end{lstlisting}

% ================= Вариант 27 =================
\item
Создайте класс \texttt{Visa} с закрытыми атрибутами \texttt{\_\_holder}, \texttt{\_\_issue\_date}, \texttt{\_\_expiry\_date}. Реализуйте методы-геттеры и метод \texttt{days\_until\_expiry()}.

\subsection*{Инструкции:}
\begin{enumerate}
    \item Создайте класс \texttt{Visa}.
    \item Методы-геттеры: \texttt{get\_holder()}, \texttt{get\_issue\_date()}, \texttt{get\_expiry\_date()}.
    \item Метод \texttt{days\_until\_expiry()} вычисляет дни до окончания визы.
    \item Создайте несколько экземпляров.
    \item Выведите данные каждой визы.
\end{enumerate}

\subsection*{Пример использования:}
\begin{lstlisting}[caption=Пример кода]
from datetime import date

visa1 = Visa("Иванов И.", date(2025, 1, 1), date(2026, 1, 1))
visa2 = Visa("Петров П.", date(2025, 6, 1), date(2026, 6, 1))

print("Виза 1:")
print("Держатель: ", visa1.get_holder())
print("Дата выдачи: ", visa1.get_issue_date())
print("Дата окончания: ", visa1.get_expiry_date())
print("Дней до окончания: ", visa1.days_until_expiry())

print("Виза 2:")
print("Держатель: ", visa2.get_holder())
print("Дата выдачи: ", visa2.get_issue_date())
print("Дата окончания: ", visa2.get_expiry_date())
print("Дней до окончания: ", visa2.days_until_expiry())
\end{lstlisting}

\subsection*{Вывод:}
\begin{lstlisting}[caption=Ожидаемый вывод]
Виза 1:
Держатель:  Иванов И.
Дата выдачи:  2025-01-01
Дата окончания:  2026-01-01
Дней до окончания:  113
Виза 2:
Держатель:  Петров П.
Дата выдачи:  2025-06-01
Дата окончания:  2026-06-01
Дней до окончания:  204
\end{lstlisting}

% ================= Вариант 28 =================
\item
Создайте класс \texttt{Reservation} с закрытыми атрибутами \texttt{\_\_guest}, \texttt{\_\_checkin\_date}, \texttt{\_\_checkout\_date}. Реализуйте методы-геттеры и метод \texttt{stay\_duration()}.

\subsection*{Инструкции:}
\begin{enumerate}
    \item Создайте класс \texttt{Reservation}.
    \item Методы-геттеры: \texttt{get\_guest()}, \texttt{get\_checkin\_date()}, \texttt{get\_checkout\_date()}.
    \item Метод \texttt{stay\_duration()} вычисляет продолжительность пребывания в днях.
    \item Создайте несколько экземпляров.
    \item Выведите данные каждой брони.
\end{enumerate}

\subsection*{Пример использования:}
\begin{lstlisting}[caption=Пример кода]
from datetime import date

res1 = Reservation("Иванов И.", date(2025, 10, 1), date(2025, 10, 7))
res2 = Reservation("Петров П.", date(2025, 11, 5), date(2025, 11, 12))

print("Бронь 1:")
print("Гость: ", res1.get_guest())
print("Дата заезда: ", res1.get_checkin_date())
print("Дата выезда: ", res1.get_checkout_date())
print("Продолжительность пребывания: ", res1.stay_duration())

print("Бронь 2:")
print("Гость: ", res2.get_guest())
print("Дата заезда: ", res2.get_checkin_date())
print("Дата выезда: ", res2.get_checkout_date())
print("Продолжительность пребывания: ", res2.stay_duration())
\end{lstlisting}

\subsection*{Вывод:}
\begin{lstlisting}[caption=Ожидаемый вывод]
Бронь 1:
Гость:  Иванов И.
Дата заезда:  2025-10-01
Дата выезда:  2025-10-07
Продолжительность пребывания:  6
Бронь 2:
Гость:  Петров П.
Дата заезда:  2025-11-05
Дата выезда:  2025-11-12
Продолжительность пребывания:  7
\end{lstlisting}

% ================= Вариант 29 =================
\item
Создайте класс \texttt{Conference} с закрытыми атрибутами \texttt{\_\_name}, \texttt{\_\_city}, \texttt{\_\_start\_date}. Реализуйте методы-геттеры и метод \texttt{days\_until\_start()}.

\subsection*{Инструкции:}
\begin{enumerate}
    \item Создайте класс \texttt{Conference}.
    \item Методы-геттеры: \texttt{get\_name()}, \texttt{get\_city()}, \texttt{get\_start\_date()}.
    \item Метод \texttt{days\_until\_start()} вычисляет дни до начала конференции.
    \item Создайте несколько экземпляров.
    \item Выведите данные каждой конференции.
\end{enumerate}

\subsection*{Пример использования:}
\begin{lstlisting}[caption=Пример кода]
from datetime import date

conf1 = Conference("PythonConf", "Москва", date(2025, 10, 20))
conf2 = Conference("DataScience Summit", "Санкт-Петербург", date(2025, 11, 15))

print("Конференция 1:")
print("Название: ", conf1.get_name())
print("Город: ", conf1.get_city())
print("Дата начала: ", conf1.get_start_date())
print("Дней до начала: ", conf1.days_until_start())

print("Конференция 2:")
print("Название: ", conf2.get_name())
print("Город: ", conf2.get_city())
print("Дата начала: ", conf2.get_start_date())
print("Дней до начала: ", conf2.days_until_start())
\end{lstlisting}

\subsection*{Вывод:}
\begin{lstlisting}[caption=Ожидаемый вывод]
Конференция 1:
Название:  PythonConf
Город:  Москва
Дата начала:  2025-10-20
Дней до начала:  59
Конференция 2:
Название:  DataScience Summit
Город:  Санкт-Петербург
Дата начала:  2025-11-15
Дней до начала:  85
\end{lstlisting}

% ================= Вариант 30 =================
\item
Создайте класс \texttt{Medication} с закрытыми атрибутами \texttt{\_\_name}, \texttt{\_\_manufacturer}, \texttt{\_\_expiry\_date}. Реализуйте методы-геттеры и метод \texttt{days\_until\_expiry()}.

\subsection*{Инструкции:}
\begin{enumerate}
    \item Создайте класс \texttt{Medication}.
    \item Методы-геттеры: \texttt{get\_name()}, \texttt{get\_manufacturer()}, \texttt{get\_expiry\_date()}.
    \item Метод \texttt{days\_until\_expiry()} вычисляет дни до окончания срока годности.
    \item Создайте несколько экземпляров.
    \item Выведите данные каждого лекарства.
\end{enumerate}

\subsection*{Пример использования:}
\begin{lstlisting}[caption=Пример кода]
from datetime import date

med1 = Medication("Парацетамол", "Фармком", date(2026, 1, 1))
med2 = Medication("Ибупрофен", "БиоФарм", date(2025, 12, 1))

print("Лекарство 1:")
print("Название: ", med1.get_name())
print("Производитель: ", med1.get_manufacturer())
print("Срок годности: ", med1.get_expiry_date())
print("Дней до окончания: ", med1.days_until_expiry())

print("Лекарство 2:")
print("Название: ", med2.get_name())
print("Производитель: ", med2.get_manufacturer())
print("Срок годности: ", med2.get_expiry_date())
print("Дней до окончания: ", med2.days_until_expiry())
\end{lstlisting}

\subsection*{Вывод:}
\begin{lstlisting}[caption=Ожидаемый вывод]
Лекарство 1:
Название:  Парацетамол
Производитель:  Фармком
Срок годности:  2026-01-01
Дней до окончания:  113
Лекарство 2:
Название:  Ибупрофен
Производитель:  БиоФарм
Срок годности:  2025-12-01
Дней до окончания:  92
\end{lstlisting}

% ================= Вариант 31 =================
\item
Создайте класс \texttt{Project} с закрытыми атрибутами \texttt{\_\_title}, \texttt{\_\_start\_date}, \texttt{\_\_deadline}. Реализуйте методы-геттеры и метод \texttt{days\_until\_deadline()}.

\subsection*{Инструкции:}
\begin{enumerate}
    \item Создайте класс \texttt{Project}.
    \item Методы-геттеры: \texttt{get\_title()}, \texttt{get\_start\_date()}, \texttt{get\_deadline()}.
    \item Метод \texttt{days\_until\_deadline()} вычисляет дни до дедлайна.
    \item Создайте несколько экземпляров.
    \item Выведите данные каждого проекта.
\end{enumerate}

\subsection*{Пример использования:}
\begin{lstlisting}[caption=Пример кода]
from datetime import date

proj1 = Project("Разработка сайта", date(2025, 9, 1), date(2025, 12, 1))
proj2 = Project("Мобильное приложение", date(2025, 10, 1), date(2026, 1, 15))

print("Проект 1:")
print("Название: ", proj1.get_title())
print("Дата начала: ", proj1.get_start_date())
print("Дедлайн: ", proj1.get_deadline())
print("Дней до дедлайна: ", proj1.days_until_deadline())

print("Проект 2:")
print("Название: ", proj2.get_title())
print("Дата начала: ", proj2.get_start_date())
print("Дедлайн: ", proj2.get_deadline())
print("Дней до дедлайна: ", proj2.days_until_deadline())
\end{lstlisting}

\subsection*{Вывод:}
\begin{lstlisting}[caption=Ожидаемый вывод]
Проект 1:
Название:  Разработка сайта
Дата начала:  2025-09-01
Дедлайн:  2025-12-01
Дней до дедлайна:  91
Проект 2:
Название:  Мобильное приложение
Дата начала:  2025-10-01
Дедлайн:  2026-01-15
Дней до дедлайна:  106
\end{lstlisting}


\end{enumerate}


\textbf{Задача 4}

\input{problem_inheritence_task4}
\subsection{Семинар <<Структуры данных в ООП-реализации>>  
(2 часа)}


В ходе работы решите 4 задачи. 
Предполагается, что пользователь класса не имеет права обращаться к свойствам напрямую 
(соблюдая принцип инкапсуляции), а должен использовать методы. 

Продемонстрируйте работоспособность всех методов (из задания) 
посредством создания запускаемых файлов, где осуществляется 
вызов методов для разных ситуаций 
(без ручного ввода, но с выводом результатов в консоль). 

Каждый класс должен сохраняться в отдельном исходном файле. 
Необходимо соблюдать все стандартные требования к качеству кода 
(отступы, именования переменных, классов, методов, 
проверка корректности входных данных).

Задания этого семинара предназначены для освоения не только ООП, но и структур данных, поэтому
требуется структуры формировать вручную без использования библиотечных вариантов.

Для сдачи работы будьте готовы пояснить или аналогично заданию модифицировать любую часть кода, а также ответить на вопросы:
\begin{enumerate}
    \item Что обозначает свойство наследования в парадигме ООП?
    \item Что обозначает свойство полиморфизма в парадигме ООП?
    \item Опишите реализацию наследования в Python
    \item Как создать конструктор в Python
    \item Как реализовать абстрактный класс в Python (и что это значит)
    \item Как реализовать абстрактные методы в Python (и что это значит)
    \item Опишите бинарное дерево
    \item Как вставить элемент в бинарное дерево
    \item Как найи элемент в бинарном дереве
    \item Опишите, что такое стек
    \item Опишите, что такое очередь
    \item Опишите двусвязный список
    \item Сравните стек, очередь и двусвязный список
\end{enumerate}

Если вы нашли в задачнике ошибки, опечатки и другие недостатки, то вы можете сделать pull-request. 

\textbf{Срок сдачи работы (начала сдачи):} через одно занятие после его выдачи. В последующие сроки оценка будет снижаться (при отсутствии оправдывающих документов).


\subsubsection{Задача 1 (дерево)} 


\begin{enumerate}
\item Написать программу на Python, которая реализует бинарное дерево поиска с инкапсуляцией внутренней структуры. Программа должна создавать экземпляры класса TreeNode, которые представляют узлы дерева, и класса SearchTree, который представляет дерево поиска. Класс SearchTree должен содержать методы для добавления, поиска и удаления элементов из дерева, при этом все вспомогательные методы должны быть приватными. Программа также должна создавать дерево поиска, вставлять в него случайные числа и выполнять поиск элементов в дереве.

Инструкции:
\begin{enumerate}
    \item Создайте класс TreeNode с методом \_\_init\_\_, который принимает значение в качестве аргумента и сохраняет его в атрибуте self.data. Атрибуты left и right должны быть инициализированы как None.
    \item Создайте класс SearchTree с методом \_\_init\_\_, который инициализирует корневой узел дерева как None.
    \item Создайте публичный метод add в классе SearchTree, который добавляет значение в дерево. Если корневой узел отсутствует, создайте новый узел с добавляемым значением. В противном случае, вызовите приватный метод \_add\_helper, передав ему корень и значение.
    \item Создайте приватный метод \_add\_helper в классе SearchTree, который рекурсивно добавляет значение в дерево. Если значение меньше или равно значению текущего узла, добавьте его в левое поддерево. Если значение строго больше значения текущего узла, добавьте его в правое поддерево.
    \item Создайте публичный метод locate в классе SearchTree, который ищет значение в дереве. Если дерево пустое, верните None. В противном случае, вызовите приватный метод \_locate\_helper, передав ему корень и искомое значение.
    \item Создайте приватный метод \_locate\_helper в классе SearchTree, который рекурсивно ищет значение в дереве. Если текущий узел равен None или значение текущего узла равно искомому значению, верните текущий узел. В противном случае, рекурсивно вызывайте метод \_locate\_helper для поиска значения в левом поддереве (если искомое значение меньше или равно значению текущего узла) или в правом поддереве (если искомое значение больше).
    \item Создайте экземпляр класса SearchTree и вставьте в него 15 случайных чисел от 1 до 30.
    \item Выполните поиск элементов в дереве и выведите результаты на экран.
\end{enumerate}

Пример использования:
\begin{lstlisting}[language=Python]
tree = SearchTree()
for i in range(15):
    tree.add(random.randint(1, 30))

print("Поиск элементов:")
print(tree.locate(7))   # Обнаружено, возвращен узел (7)
print(tree.locate(25))  # Не обнаружено, возвращено None
print(tree.locate(15))  # Обнаружено, возвращен узел (15)
\end{lstlisting}

\begin{figure}[h]
\centering
\begin{tikzpicture}[level distance=1.5cm,
  level 1/.style={sibling distance=3cm},
  level 2/.style={sibling distance=1.5cm}]
  \node {10}
    child {node {5}
      child {node {2}}
      child {node {8}}
    }
    child {node {15}
      child {node {12}}
      child {node {20}}
    };
\end{tikzpicture}
\caption{Пример бинарного дерева поиска}
\end{figure}

\item Написать программу на Python, которая реализует бинарное дерево поиска с соблюдением принципов инкапсуляции. Программа должна создавать экземпляры класса Vertex, которые представляют узлы дерева, и класса BinaryTree, который представляет дерево поиска. Класс BinaryTree должен содержать методы для вставки, поиска и удаления элементов, при этом все рекурсивные вспомогательные методы должны быть скрыты от внешнего доступа. Программа также должна создавать дерево поиска, вставлять в него случайные числа и выполнять поиск элементов в дереве.

Инструкции:
\begin{enumerate}
    \item Создайте класс Vertex с методом \_\_init\_\_, который принимает значение value и сохраняет его в атрибуте self.key. Атрибуты self.left\_child и self.right\_child должны быть инициализированы как None.
    \item Создайте класс BinaryTree с методом \_\_init\_\_, который инициализирует атрибут self.top как None.
    \item Создайте публичный метод put в классе BinaryTree, который вставляет значение в дерево. Если self.top отсутствует, создайте новый узел с вставляемым значением. В противном случае, вызовите приватный метод \_put\_recursively, передав ему self.top и значение.
    \item Создайте приватный метод \_put\_recursively в классе BinaryTree, который рекурсивно вставляет значение в дерево. Если значение строго меньше значения текущего узла, вставьте его в левое поддерево. Если значение больше или равно значению текущего узла, вставьте его в правое поддерево.
    \item Создайте публичный метод find в классе BinaryTree, который ищет значение в дереве. Если дерево пустое, верните None. В противном случае, вызовите приватный метод \_find\_recursively, передав ему self.top и искомое значение.
    \item Создайте приватный метод \_find\_recursively в классе BinaryTree, который рекурсивно ищет значение в дереве. Если текущий узел равен None или значение текущего узла равно искомому значению, верните текущий узел. В противном случае, рекурсивно вызывайте метод \_find\_recursively для поиска значения в левом поддереве (если искомое значение меньше текущего) или в правом поддереве (если искомое значение больше или равно текущему).
    \item Создайте экземпляр класса BinaryTree и вставьте в него 18 случайных чисел от 5 до 35.
    \item Выполните поиск элементов в дереве и выведите результаты на экран.
\end{enumerate}

Пример использования:
\begin{lstlisting}[language=Python]
bt = BinaryTree()
for i in range(18):
    bt.put(random.randint(5, 35))

print("Поиск элементов:")
print(bt.find(10))  # Обнаружено, возвращен узел (10)
print(bt.find(40))  # Не обнаружено, возвращено None
print(bt.find(22))  # Обнаружено, возвращен узел (22)
\end{lstlisting}

\begin{figure}[h]
\centering
\begin{tikzpicture}[level distance=1.5cm,
  level 1/.style={sibling distance=3cm},
  level 2/.style={sibling distance=1.5cm}]
  \node {18}
    child {node {9}
      child {node {4}}
      child {node {14}}
    }
    child {node {25}
      child {node {21}}
      child {node {30}}
    };
\end{tikzpicture}
\caption{Пример бинарного дерева поиска}
\end{figure}

\item Написать программу на Python, которая реализует бинарное дерево поиска с инкапсуляцией внутренней логики. Программа должна создавать экземпляры класса BNode, которые представляют узлы дерева, и класса BSTree, который представляет дерево поиска. Класс BSTree должен содержать методы для вставки, поиска и удаления элементов, при этом все рекурсивные функции должны быть приватными. Программа также должна создавать дерево поиска, вставлять в него случайные числа и выполнять поиск элементов в дереве.

Инструкции:
\begin{enumerate}
    \item Создайте класс BNode с методом \_\_init\_\_, который принимает параметр item и сохраняет его в атрибуте self.element. Атрибуты self.left\_branch и self.right\_branch должны быть инициализированы как None.
    \item Создайте класс BSTree с методом \_\_init\_\_, который инициализирует атрибут self.root\_node как None.
    \item Создайте публичный метод insert\_value в классе BSTree, который вставляет значение в дерево. Если self.root\_node отсутствует, создайте новый узел с вставляемым значением. В противном случае, вызовите приватный метод \_recursive\_insert, передав ему self.root\_node и значение.
    \item Создайте приватный метод \_recursive\_insert в классе BSTree, который рекурсивно вставляет значение в дерево. Если значение меньше или равно значению текущего узла, вставьте его в левое поддерево. Если значение строго больше значения текущего узла, вставьте его в правое поддерево.
    \item Создайте публичный метод retrieve в классе BSTree, который ищет значение в дереве. Если дерево пустое, верните None. В противном случае, вызовите приватный метод \_recursive\_retrieve, передав ему self.root\_node и искомое значение.
    \item Создайте приватный метод \_recursive\_retrieve в классе BSTree, который рекурсивно ищет значение в дереве. Если текущий узел равен None или значение текущего узла равно искомому значению, верните текущий узел. В противном случае, рекурсивно вызывайте метод \_recursive\_retrieve для поиска значения в левом поддереве (если искомое значение меньше или равно текущему) или в правом поддереве (если искомое значение больше).
    \item Создайте экземпляр класса BSTree и вставьте в него 20 случайных чисел от 1 до 40.
    \item Выполните поиск элементов в дереве и выведите результаты на экран.
\end{enumerate}

Пример использования:
\begin{lstlisting}[language=Python]
bst = BSTree()
for i in range(20):
    bst.insert_value(random.randint(1, 40))

print("Поиск элементов:")
print(bst.retrieve(12))  # Обнаружено, возвращен узел (12)
print(bst.retrieve(50))  # Не обнаружено, возвращено None
print(bst.retrieve(33))  # Обнаружено, возвращен узел (33)
\end{lstlisting}

\begin{figure}[h]
\centering
\begin{tikzpicture}[level distance=1.5cm,
  level 1/.style={sibling distance=3cm},
  level 2/.style={sibling distance=1.5cm}]
  \node {20}
    child {node {10}
      child {node {5}}
      child {node {15}}
    }
    child {node {30}
      child {node {25}}
      child {node {35}}
    };
\end{tikzpicture}
\caption{Пример бинарного дерева поиска}
\end{figure}

\item Написать программу на Python, которая реализует бинарное дерево поиска с инкапсуляцией. Программа должна создавать экземпляры класса ElementNode, которые представляют узлы дерева, и класса OrderedTree, который представляет дерево поиска. Класс OrderedTree должен содержать методы для вставки, поиска и удаления элементов, при этом все вспомогательные методы должны быть приватными. Программа также должна создавать дерево поиска, вставлять в него случайные числа и выполнять поиск элементов в дереве.

Инструкции:
\begin{enumerate}
    \item Создайте класс ElementNode с методом \_\_init\_\_, который принимает параметр content и сохраняет его в атрибуте self.payload. Атрибуты self.left и self.right должны быть инициализированы как None.
    \item Создайте класс OrderedTree с методом \_\_init\_\_, который инициализирует атрибут self.head как None.
    \item Создайте публичный метод store в классе OrderedTree, который вставляет значение в дерево. Если self.head отсутствует, создайте новый узел с вставляемым значением. В противном случае, вызовите приватный метод \_store\_recursive, передав ему self.head и значение.
    \item Создайте приватный метод \_store\_recursive в классе OrderedTree, который рекурсивно вставляет значение в дерево. Если значение строго меньше значения текущего узла, вставьте его в левое поддерево. Если значение больше или равно значению текущего узла, вставьте его в правое поддерево.
    \item Создайте публичный метод query в классе OrderedTree, который ищет значение в дереве. Если дерево пустое, верните None. В противном случае, вызовите приватный метод \_query\_recursive, передав ему self.head и искомое значение.
    \item Создайте приватный метод \_query\_recursive в классе OrderedTree, который рекурсивно ищет значение в дереве. Если текущий узел равен None или значение текущего узла равно искомому значению, верните текущий узел. В противном случае, рекурсивно вызывайте метод \_query\_recursive для поиска значения в левом поддереве (если искомое значение меньше текущего) или в правом поддереве (если искомое значение больше или равно текущему).
    \item Создайте экземпляр класса OrderedTree и вставьте в него 17 случайных чисел от 3 до 33.
    \item Выполните поиск элементов в дереве и выведите результаты на экран.
\end{enumerate}

Пример использования:
\begin{lstlisting}[language=Python]
ot = OrderedTree()
for i in range(17):
    ot.store(random.randint(3, 33))

print("Поиск элементов:")
print(ot.query(8))   # Обнаружено, возвращен узел (8)
print(ot.query(45))  # Не обнаружено, возвращено None
print(ot.query(27))  # Обнаружено, возвращен узел (27)
\end{lstlisting}

\begin{figure}[h]
\centering
\begin{tikzpicture}[level distance=1.5cm,
  level 1/.style={sibling distance=3cm},
  level 2/.style={sibling distance=1.5cm}]
  \node {17}
    child {node {8}
      child {node {3}}
      child {node {12}}
    }
    child {node {27}
      child {node {22}}
      child {node {32}}
    };
\end{tikzpicture}
\caption{Пример бинарного дерева поиска}
\end{figure}

\item Написать программу на Python, которая реализует бинарное дерево поиска с инкапсуляцией внутренней структуры. Программа должна создавать экземпляры класса DataNode, которые представляют узлы дерева, и класса SortedTree, который представляет дерево поиска. Класс SortedTree должен содержать методы для вставки, поиска и удаления элементов, при этом все рекурсивные методы должны быть скрыты. Программа также должна создавать дерево поиска, вставлять в него случайные числа и выполнять поиск элементов в дереве.

Инструкции:
\begin{enumerate}
    \item Создайте класс DataNode с методом \_\_init\_\_, который принимает параметр val и сохраняет его в атрибуте self.entry. Атрибуты self.left и self.right должны быть инициализированы как None.
    \item Создайте класс SortedTree с методом \_\_init\_\_, который инициализирует атрибут self.first\_node как None.
    \item Создайте публичный метод enqueue в классе SortedTree, который вставляет значение в дерево. Если self.first\_node отсутствует, создайте новый узел с вставляемым значением. В противном случае, вызовите приватный метод \_enqueue\_helper, передав ему self.first\_node и значение.
    \item Создайте приватный метод \_enqueue\_helper в классе SortedTree, который рекурсивно вставляет значение в дерево. Если значение меньше или равно значению текущего узла, вставьте его в левое поддерево. Если значение строго больше значения текущего узла, вставьте его в правое поддерево.
    \item Создайте публичный метод lookup в классе SortedTree, который ищет значение в дереве. Если дерево пустое, верните None. В противном случае, вызовите приватный метод \_lookup\_helper, передав ему self.first\_node и искомое значение.
    \item Создайте приватный метод \_lookup\_helper в классе SortedTree, который рекурсивно ищет значение в дереве. Если текущий узел равен None или значение текущего узла равно искомому значению, верните текущий узел. В противном случае, рекурсивно вызывайте метод \_lookup\_helper для поиска значения в левом поддереве (если искомое значение меньше или равно текущему) или в правом поддереве (если искомое значение больше).
    \item Создайте экземпляр класса SortedTree и вставьте в него 16 случайных чисел от 2 до 28.
    \item Выполните поиск элементов в дереве и выведите результаты на экран.
\end{enumerate}

Пример использования:
\begin{lstlisting}[language=Python]
st = SortedTree()
for i in range(16):
    st.enqueue(random.randint(2, 28))

print("Поиск элементов:")
print(st.lookup(6))   # Обнаружено, возвращен узел (6)
print(st.lookup(35))  # Не обнаружено, возвращено None
print(st.lookup(19))  # Обнаружено, возвращен узел (19)
\end{lstlisting}

\begin{figure}[h]
\centering
\begin{tikzpicture}[level distance=1.5cm,
  level 1/.style={sibling distance=3cm},
  level 2/.style={sibling distance=1.5cm}]
  \node {14}
    child {node {6}
      child {node {2}}
      child {node {10}}
    }
    child {node {22}
      child {node {18}}
      child {node {26}}
    };
\end{tikzpicture}
\caption{Пример бинарного дерева поиска}
\end{figure}

\item Написать программу на Python, которая реализует бинарное дерево поиска с инкапсуляцией. Программа должна создавать экземпляры класса BinNode, которые представляют узлы дерева, и класса LookupTree, который представляет дерево поиска. Класс LookupTree должен содержать методы для вставки, поиска и удаления элементов, при этом все вспомогательные методы должны быть приватными. Программа также должна создавать дерево поиска, вставлять в него случайные числа и выполнять поиск элементов в дереве.

Инструкции:
\begin{enumerate}
    \item Создайте класс BinNode с методом \_\_init\_\_, который принимает параметр num и сохраняет его в атрибуте self.number. Атрибуты self.left и self.right должны быть инициализированы как None.
    \item Создайте класс LookupTree с методом \_\_init\_\_, который инициализирует атрибут self.initial\_node как None.
    \item Создайте публичный метод add\_entry в классе LookupTree, который вставляет значение в дерево. Если self.initial\_node отсутствует, создайте новый узел с вставляемым значением. В противном случае, вызовите приватный метод \_add\_entry\_rec, передав ему self.initial\_node и значение.
    \item Создайте приватный метод \_add\_entry\_rec в классе LookupTree, который рекурсивно вставляет значение в дерево. Если значение строго меньше значения текущего узла, вставьте его в левое поддерево. Если значение больше или равно значению текущего узла, вставьте его в правое поддерево.
    \item Создайте публичный метод fetch в классе LookupTree, который ищет значение в дереве. Если дерево пустое, верните None. В противном случае, вызовите приватный метод \_fetch\_rec, передав ему self.initial\_node и искомое значение.
    \item Создайте приватный метод \_fetch\_rec в классе LookupTree, который рекурсивно ищет значение в дереве. Если текущий узел равен None или значение текущего узла равно искомому значению, верните текущий узел. В противном случае, рекурсивно вызывайте метод \_fetch\_rec для поиска значения в левом поддереве (если искомое значение меньше текущего) или в правом поддереве (если искомое значение больше или равно текущему).
    \item Создайте экземпляр класса LookupTree и вставьте в него 19 случайных чисел от 4 до 34.
    \item Выполните поиск элементов в дереве и выведите результаты на экран.
\end{enumerate}

Пример использования:
\begin{lstlisting}[language=Python]
lt = LookupTree()
for i in range(19):
    lt.add_entry(random.randint(4, 34))

print("Поиск элементов:")
print(lt.fetch(9))   # Обнаружено, возвращен узел (9)
print(lt.fetch(40))  # Не обнаружено, возвращено None
print(lt.fetch(24))  # Обнаружено, возвращен узел (24)
\end{lstlisting}

\begin{figure}[h]
\centering
\begin{tikzpicture}[level distance=1.5cm,
  level 1/.style={sibling distance=3cm},
  level 2/.style={sibling distance=1.5cm}]
  \node {19}
    child {node {9}
      child {node {4}}
      child {node {14}}
    }
    child {node {29}
      child {node {24}}
      child {node {34}}
    };
\end{tikzpicture}
\caption{Пример бинарного дерева поиска}
\end{figure}

\item Написать программу на Python, которая реализует бинарное дерево поиска с инкапсуляцией. Программа должна создавать экземпляры класса NodeItem, которые представляют узлы дерева, и класса BinaryTreeSearch, который представляет дерево поиска. Класс BinaryTreeSearch должен содержать методы для вставки, поиска и удаления элементов, при этом все рекурсивные методы должны быть приватными. Программа также должна создавать дерево поиска, вставлять в него случайные числа и выполнять поиск элементов в дереве.

Инструкции:
\begin{enumerate}
    \item Создайте класс NodeItem с методом \_\_init\_\_, который принимает параметр item\_value и сохраняет его в атрибуте self.val. Атрибуты self.left и self.right должны быть инициализированы как None.
    \item Создайте класс BinaryTreeSearch с методом \_\_init\_\_, который инициализирует атрибут self.start\_node как None.
    \item Создайте публичный метод insert\_item в классе BinaryTreeSearch, который вставляет значение в дерево. Если self.start\_node отсутствует, создайте новый узел с вставляемым значением. В противном случае, вызовите приватный метод \_insert\_item\_helper, передав ему self.start\_node и значение.
    \item Создайте приватный метод \_insert\_item\_helper в классе BinaryTreeSearch, который рекурсивно вставляет значение в дерево. Если значение меньше или равно значению текущего узла, вставьте его в левое поддерево. Если значение строго больше значения текущего узла, вставьте его в правое поддерево.
    \item Создайте публичный метод find\_item в классе BinaryTreeSearch, который ищет значение в дереве. Если дерево пустое, верните None. В противном случае, вызовите приватный метод \_find\_item\_helper, передав ему self.start\_node и искомое значение.
    \item Создайте приватный метод \_find\_item\_helper в классе BinaryTreeSearch, который рекурсивно ищет значение в дереве. Если текущий узел равен None или значение текущего узла равно искомому значению, верните текущий узел. В противном случае, рекурсивно вызывайте метод \_find\_item\_helper для поиска значения в левом поддереве (если искомое значение меньше или равно текущему) или в правом поддереве (если искомое значение больше).
    \item Создайте экземпляр класса BinaryTreeSearch и вставьте в него 21 случайное число от 1 до 38.
    \item Выполните поиск элементов в дереве и выведите результаты на экран.
\end{enumerate}

Пример использования:
\begin{lstlisting}[language=Python]
bts = BinaryTreeSearch()
for i in range(21):
    bts.insert_item(random.randint(1, 38))

print("Поиск элементов:")
print(bts.find_item(11))  # Обнаружено, возвращен узел (11)
print(bts.find_item(50))  # Не обнаружено, возвращено None
print(bts.find_item(29))  # Обнаружено, возвращен узел (29)
\end{lstlisting}

\begin{figure}[h]
\centering
\begin{tikzpicture}[level distance=1.5cm,
  level 1/.style={sibling distance=3cm},
  level 2/.style={sibling distance=1.5cm}]
  \node {21}
    child {node {11}
      child {node {5}}
      child {node {16}}
    }
    child {node {31}
      child {node {26}}
      child {node {36}}
    };
\end{tikzpicture}
\caption{Пример бинарного дерева поиска}
\end{figure}

\item Написать программу на Python, которая реализует бинарное дерево поиска с инкапсуляцией. Программа должна создавать экземпляры класса TreeVertex, которые представляют узлы дерева, и класса SearchBinTree, который представляет дерево поиска. Класс SearchBinTree должен содержать методы для вставки, поиска и удаления элементов, при этом все вспомогательные методы должны быть приватными. Программа также должна создавать дерево поиска, вставлять в него случайные числа и выполнять поиск элементов в дереве.

Инструкции:
\begin{enumerate}
    \item Создайте класс TreeVertex с методом \_\_init\_\_, который принимает параметр vertex\_data и сохраняет его в атрибуте self.info. Атрибуты self.left и self.right должны быть инициализированы как None.
    \item Создайте класс SearchBinTree с методом \_\_init\_\_, который инициализирует атрибут self.root\_vertex как None.
    \item Создайте публичный метод insert\_data в классе SearchBinTree, который вставляет значение в дерево. Если self.root\_vertex отсутствует, создайте новый узел с вставляемым значением. В противном случае, вызовите приватный метод \_insert\_data\_rec, передав ему self.root\_vertex и значение.
    \item Создайте приватный метод \_insert\_data\_rec в классе SearchBinTree, который рекурсивно вставляет значение в дерево. Если значение строго меньше значения текущего узла, вставьте его в левое поддерево. Если значение больше или равно значению текущего узла, вставьте его в правое поддерево.
    \item Создайте публичный метод search\_data в классе SearchBinTree, который ищет значение в дереве. Если дерево пустое, верните None. В противном случае, вызовите приватный метод \_search\_data\_rec, передав ему self.root\_vertex и искомое значение.
    \item Создайте приватный метод \_search\_data\_rec в классе SearchBinTree, который рекурсивно ищет значение в дереве. Если текущий узел равен None или значение текущего узла равно искомому значению, верните текущий узел. В противном случае, рекурсивно вызывайте метод \_search\_data\_rec для поиска значения в левом поддереве (если искомое значение меньше текущего) или в правом поддереве (если искомое значение больше или равно текущему).
    \item Создайте экземпляр класса SearchBinTree и вставьте в него 14 случайных чисел от 6 до 36.
    \item Выполните поиск элементов в дереве и выведите результаты на экран.
\end{enumerate}

Пример использования:
\begin{lstlisting}[language=Python]
sbt = SearchBinTree()
for i in range(14):
    sbt.insert_data(random.randint(6, 36))

print("Поиск элементов:")
print(sbt.search_data(13))  # Обнаружено, возвращен узел (13)
print(sbt.search_data(42))  # Не обнаружено, возвращено None
print(sbt.search_data(28))  # Обнаружено, возвращен узел (28)
\end{lstlisting}

\begin{figure}[h]
\centering
\begin{tikzpicture}[level distance=1.5cm,
  level 1/.style={sibling distance=3cm},
  level 2/.style={sibling distance=1.5cm}]
  \node {23}
    child {node {13}
      child {node {8}}
      child {node {18}}
    }
    child {node {33}
      child {node {28}}
      child {node {38}}
    };
\end{tikzpicture}
\caption{Пример бинарного дерева поиска}
\end{figure}

\item Написать программу на Python, которая реализует бинарное дерево поиска с инкапсуляцией. Программа должна создавать экземпляры класса BranchNode, которые представляют узлы дерева, и класса BinaryTreeLookup, который представляет дерево поиска. Класс BinaryTreeLookup должен содержать методы для вставки, поиска и удаления элементов, при этом все рекурсивные методы должны быть приватными. Программа также должна создавать дерево поиска, вставлять в него случайные числа и выполнять поиск элементов в дереве.

Инструкции:
\begin{enumerate}
    \item Создайте класс BranchNode с методом \_\_init\_\_, который принимает параметр node\_val и сохраняет его в атрибуте self.data\_point. Атрибуты self.left\_link и self.right\_link должны быть инициализированы как None.
    \item Создайте класс BinaryTreeLookup с методом \_\_init\_\_, который инициализирует атрибут self.base\_node как None.
    \item Создайте публичный метод add\_point в классе BinaryTreeLookup, который вставляет значение в дерево. Если self.base\_node отсутствует, создайте новый узел с вставляемым значением. В противном случае, вызовите приватный метод \_add\_point\_recursive, передав ему self.base\_node и значение.
    \item Создайте приватный метод \_add\_point\_recursive в классе BinaryTreeLookup, который рекурсивно вставляет значение в дерево. Если значение меньше или равно значению текущего узла, вставьте его в левое поддерево. Если значение строго больше значения текущего узла, вставьте его в правое поддерево.
    \item Создайте публичный метод locate\_point в классе BinaryTreeLookup, который ищет значение в дереве. Если дерево пустое, верните None. В противном случае, вызовите приватный метод \_locate\_point\_recursive, передав ему self.base\_node и искомое значение.
    \item Создайте приватный метод \_locate\_point\_recursive в классе BinaryTreeLookup, который рекурсивно ищет значение в дереве. Если текущий узел равен None или значение текущего узла равно искомому значению, верните текущий узел. В противном случае, рекурсивно вызывайте метод \_locate\_point\_recursive для поиска значения в левом поддереве (если искомое значение меньше или равно текущему) или в правом поддереве (если искомое значение больше).
    \item Создайте экземпляр класса BinaryTreeLookup и вставьте в него 13 случайных чисел от 7 до 37.
    \item Выполните поиск элементов в дереве и выведите результаты на экран.
\end{enumerate}

Пример использования:
\begin{lstlisting}[language=Python]
btl = BinaryTreeLookup()
for i in range(13):
    btl.add_point(random.randint(7, 37))

print("Поиск элементов:")
print(btl.locate_point(14))  # Обнаружено, возвращен узел (14)
print(btl.locate_point(45))  # Не обнаружено, возвращено None
print(btl.locate_point(27))  # Обнаружено, возвращен узел (27)
\end{lstlisting}

\begin{figure}[h]
\centering
\begin{tikzpicture}[level distance=1.5cm,
  level 1/.style={sibling distance=3cm},
  level 2/.style={sibling distance=1.5cm}]
  \node {24}
    child {node {14}
      child {node {9}}
      child {node {19}}
    }
    child {node {34}
      child {node {29}}
      child {node {39}}
    };
\end{tikzpicture}
\caption{Пример бинарного дерева поиска}
\end{figure}

\item Написать программу на Python, которая реализует бинарное дерево поиска с инкапсуляцией. Программа должна создавать экземпляры класса TreeNodeStruct, которые представляют узлы дерева, и класса BinSearchStructure, который представляет дерево поиска. Класс BinSearchStructure должен содержать методы для вставки, поиска и удаления элементов, при этом все вспомогательные методы должны быть приватными. Программа также должна создавать дерево поиска, вставлять в него случайные числа и выполнять поиск элементов в дереве.

Инструкции:
\begin{enumerate}
    \item Создайте класс TreeNodeStruct с методом \_\_init\_\_, который принимает параметр struct\_value и сохраняет его в атрибуте self.node\_value. Атрибуты self.left\_sub и self.right\_sub должны быть инициализированы как None.
    \item Создайте класс BinSearchStructure с методом \_\_init\_\_, который инициализирует атрибут self.top\_element как None.
    \item Создайте публичный метод insert\_struct в классе BinSearchStructure, который вставляет значение в дерево. Если self.top\_element отсутствует, создайте новый узел с вставляемым значением. В противном случае, вызовите приватный метод \_insert\_struct\_helper, передав ему self.top\_element и значение.
    \item Создайте приватный метод \_insert\_struct\_helper в классе BinSearchStructure, который рекурсивно вставляет значение в дерево. Если значение строго меньше значения текущего узла, вставьте его в левое поддерево. Если значение больше или равно значению текущего узла, вставьте его в правое поддерево.
    \item Создайте публичный метод find\_struct в классе BinSearchStructure, который ищет значение в дереве. Если дерево пустое, верните None. В противном случае, вызовите приватный метод \_find\_struct\_helper, передав ему self.top\_element и искомое значение.
    \item Создайте приватный метод \_find\_struct\_helper в классе BinSearchStructure, который рекурсивно ищет значение в дереве. Если текущий узел равен None или значение текущего узла равно искомому значению, верните текущий узел. В противном случае, рекурсивно вызывайте метод \_find\_struct\_helper для поиска значения в левом поддереве (если искомое значение меньше текущего) или в правом поддереве (если искомое значение больше или равно текущему).
    \item Создайте экземпляр класса BinSearchStructure и вставьте в него 22 случайных числа от 2 до 42.
    \item Выполните поиск элементов в дереве и выведите результаты на экран.
\end{enumerate}

Пример использования:
\begin{lstlisting}[language=Python]
bss = BinSearchStructure()
for i in range(22):
    bss.insert_struct(random.randint(2, 42))

print("Поиск элементов:")
print(bss.find_struct(15))  # Обнаружено, возвращен узел (15)
print(bss.find_struct(55))  # Не обнаружено, возвращено None
print(bss.find_struct(35))  # Обнаружено, возвращен узел (35)
\end{lstlisting}

\begin{figure}[h]
\centering
\begin{tikzpicture}[level distance=1.5cm,
  level 1/.style={sibling distance=3cm},
  level 2/.style={sibling distance=1.5cm}]
  \node {25}
    child {node {15}
      child {node {10}}
      child {node {20}}
    }
    child {node {35}
      child {node {30}}
      child {node {40}}
    };
\end{tikzpicture}
\caption{Пример бинарного дерева поиска}
\end{figure}

\item Написать программу на Python, которая реализует бинарное дерево поиска с инкапсуляцией. Программа должна создавать экземпляры класса NodeElement, которые представляют узлы дерева, и класса TreeIndex, который представляет дерево поиска. Класс TreeIndex должен содержать методы для вставки, поиска и удаления элементов, при этом все рекурсивные методы должны быть приватными. Программа также должна создавать дерево поиска, вставлять в него случайные числа и выполнять поиск элементов в дереве.

Инструкции:
\begin{enumerate}
    \item Создайте класс NodeElement с методом \_\_init\_\_, который принимает параметр elem\_value и сохраняет его в атрибуте self.index\_key. Атрибуты self.left и self.right должны быть инициализированы как None.
    \item Создайте класс TreeIndex с методом \_\_init\_\_, который инициализирует атрибут self.root\_elem как None.
    \item Создайте публичный метод add\_key в классе TreeIndex, который вставляет значение в дерево. Если self.root\_elem отсутствует, создайте новый узел с вставляемым значением. В противном случае, вызовите приватный метод \_add\_key\_rec, передав ему self.root\_elem и значение.
    \item Создайте приватный метод \_add\_key\_rec в классе TreeIndex, который рекурсивно вставляет значение в дерево. Если значение меньше или равно значению текущего узла, вставьте его в левое поддерево. Если значение строго больше значения текущего узла, вставьте его в правое поддерево.
    \item Создайте публичный метод get\_key в классе TreeIndex, который ищет значение в дереве. Если дерево пустое, верните None. В противном случае, вызовите приватный метод \_get\_key\_rec, передав ему self.root\_elem и искомое значение.
    \item Создайте приватный метод \_get\_key\_rec в классе TreeIndex, который рекурсивно ищет значение в дереве. Если текущий узел равен None или значение текущего узла равно искомому значению, верните текущий узел. В противном случае, рекурсивно вызывайте метод \_get\_key\_rec для поиска значения в левом поддереве (если искомое значение меньше или равно текущему) или в правом поддереве (если искомое значение больше).
    \item Создайте экземпляр класса TreeIndex и вставьте в него 23 случайных числа от 3 до 43.
    \item Выполните поиск элементов в дереве и выведите результаты на экран.
\end{enumerate}

Пример использования:
\begin{lstlisting}[language=Python]
ti = TreeIndex()
for i in range(23):
    ti.add_key(random.randint(3, 43))

print("Поиск элементов:")
print(ti.get_key(16))  # Обнаружено, возвращен узел (16)
print(ti.get_key(56))  # Не обнаружено, возвращено None
print(ti.get_key(36))  # Обнаружено, возвращен узел (36)
\end{lstlisting}

\begin{figure}[h]
\centering
\begin{tikzpicture}[level distance=1.5cm,
  level 1/.style={sibling distance=3cm},
  level 2/.style={sibling distance=1.5cm}]
  \node {26}
    child {node {16}
      child {node {11}}
      child {node {21}}
    }
    child {node {36}
      child {node {31}}
      child {node {41}}
    };
\end{tikzpicture}
\caption{Пример бинарного дерева поиска}
\end{figure}

\item Написать программу на Python, которая реализует бинарное дерево поиска с инкапсуляцией. Программа должна создавать экземпляры класса BinElement, которые представляют узлы дерева, и класса IndexTree, который представляет дерево поиска. Класс IndexTree должен содержать методы для вставки, поиска и удаления элементов, при этом все вспомогательные методы должны быть приватными. Программа также должна создавать дерево поиска, вставлять в него случайные числа и выполнять поиск элементов в дереве.

Инструкции:
\begin{enumerate}
    \item Создайте класс BinElement с методом \_\_init\_\_, который принимает параметр bin\_val и сохраняет его в атрибуте self.key\_value. Атрибуты self.left\_node и self.right\_node должны быть инициализированы как None.
    \item Создайте класс IndexTree с методом \_\_init\_\_, который инициализирует атрибут self.first\_element как None.
    \item Создайте публичный метод insert\_key в классе IndexTree, который вставляет значение в дерево. Если self.first\_element отсутствует, создайте новый узел с вставляемым значением. В противном случае, вызовите приватный метод \_insert\_key\_helper, передав ему self.first\_element и значение.
    \item Создайте приватный метод \_insert\_key\_helper в классе IndexTree, который рекурсивно вставляет значение в дерево. Если значение строго меньше значения текущего узла, вставьте его в левое поддерево. Если значение больше или равно значению текущего узла, вставьте его в правое поддерево.
    \item Создайте публичный метод search\_key в классе IndexTree, который ищет значение в дереве. Если дерево пустое, верните None. В противном случае, вызовите приватный метод \_search\_key\_helper, передав ему self.first\_element и искомое значение.
    \item Создайте приватный метод \_search\_key\_helper в классе IndexTree, который рекурсивно ищет значение в дереве. Если текущий узел равен None или значение текущего узла равно искомому значению, верните текущий узел. В противном случае, рекурсивно вызывайте метод \_search\_key\_helper для поиска значения в левом поддереве (если искомое значение меньше текущего) или в правом поддереве (если искомое значение больше или равно текущему).
    \item Создайте экземпляр класса IndexTree и вставьте в него 24 случайных числа от 4 до 44.
    \item Выполните поиск элементов в дереве и выведите результаты на экран.
\end{enumerate}

Пример использования:
\begin{lstlisting}[language=Python]
it = IndexTree()
for i in range(24):
    it.insert_key(random.randint(4, 44))

print("Поиск элементов:")
print(it.search_key(17))  # Обнаружено, возвращен узел (17)
print(it.search_key(57))  # Не обнаружено, возвращено None
print(it.search_key(37))  # Обнаружено, возвращен узел (37)
\end{lstlisting}

\begin{figure}[h]
\centering
\begin{tikzpicture}[level distance=1.5cm,
  level 1/.style={sibling distance=3cm},
  level 2/.style={sibling distance=1.5cm}]
  \node {27}
    child {node {17}
      child {node {12}}
      child {node {22}}
    }
    child {node {37}
      child {node {32}}
      child {node {42}}
    };
\end{tikzpicture}
\caption{Пример бинарного дерева поиска}
\end{figure}

\item Написать программу на Python, которая реализует бинарное дерево поиска с инкапсуляцией. Программа должна создавать экземпляры класса SearchNode, которые представляют узлы дерева, и класса BinaryTreeIndex, который представляет дерево поиска. Класс BinaryTreeIndex должен содержать методы для вставки, поиска и удаления элементов, при этом все рекурсивные методы должны быть приватными. Программа также должна создавать дерево поиска, вставлять в него случайные числа и выполнять поиск элементов в дереве.

Инструкции:
\begin{enumerate}
    \item Создайте класс SearchNode с методом \_\_init\_\_, который принимает параметр search\_val и сохраняет его в атрибуте self.node\_key. Атрибуты self.left\_child и self.right\_child должны быть инициализированы как None.
    \item Создайте класс BinaryTreeIndex с методом \_\_init\_\_, который инициализирует атрибут self.initial\_element как None.
    \item Создайте публичный метод add\_node в классе BinaryTreeIndex, который вставляет значение в дерево. Если self.initial\_element отсутствует, создайте новый узел с вставляемым значением. В противном случае, вызовите приватный метод \_add\_node\_recursive, передав ему self.initial\_element и значение.
    \item Создайте приватный метод \_add\_node\_recursive в классе BinaryTreeIndex, который рекурсивно вставляет значение в дерево. Если значение меньше или равно значению текущего узла, вставьте его в левое поддерево. Если значение строго больше значения текущего узла, вставьте его в правое поддерево.
    \item Создайте публичный метод find\_node в классе BinaryTreeIndex, который ищет значение в дереве. Если дерево пустое, верните None. В противном случае, вызовите приватный метод \_find\_node\_recursive, передав ему self.initial\_element и искомое значение.
    \item Создайте приватный метод \_find\_node\_recursive в классе BinaryTreeIndex, который рекурсивно ищет значение в дереве. Если текущий узел равен None или значение текущего узла равно искомому значению, верните текущий узел. В противном случае, рекурсивно вызывайте метод \_find\_node\_recursive для поиска значения в левом поддереве (если искомое значение меньше или равно текущему) или в правом поддереве (если искомое значение больше).
    \item Создайте экземпляр класса BinaryTreeIndex и вставьте в него 25 случайных чисел от 5 до 45.
    \item Выполните поиск элементов в дереве и выведите результаты на экран.
\end{enumerate}

Пример использования:
\begin{lstlisting}[language=Python]
bti = BinaryTreeIndex()
for i in range(25):
    bti.add_node(random.randint(5, 45))

print("Поиск элементов:")
print(bti.find_node(18))  # Обнаружено, возвращен узел (18)
print(bti.find_node(58))  # Не обнаружено, возвращено None
print(bti.find_node(38))  # Обнаружено, возвращен узел (38)
\end{lstlisting}

\begin{figure}[h]
\centering
\begin{tikzpicture}[level distance=1.5cm,
  level 1/.style={sibling distance=3cm},
  level 2/.style={sibling distance=1.5cm}]
  \node {28}
    child {node {18}
      child {node {13}}
      child {node {23}}
    }
    child {node {38}
      child {node {33}}
      child {node {43}}
    };
\end{tikzpicture}
\caption{Пример бинарного дерева поиска}
\end{figure}

\item Написать программу на Python, которая реализует бинарное дерево поиска с инкапсуляцией. Программа должна создавать экземпляры класса IndexNode, которые представляют узлы дерева, и класса SearchStructure, который представляет дерево поиска. Класс SearchStructure должен содержать методы для вставки, поиска и удаления элементов, при этом все вспомогательные методы должны быть приватными. Программа также должна создавать дерево поиска, вставлять в него случайные числа и выполнять поиск элементов в дереве.

Инструкции:
\begin{enumerate}
    \item Создайте класс IndexNode с методом \_\_init\_\_, который принимает параметр idx\_value и сохраняет его в атрибуте self.element\_key. Атрибуты self.left\_elem и self.right\_elem должны быть инициализированы как None.
    \item Создайте класс SearchStructure с методом \_\_init\_\_, который инициализирует атрибут self.start\_element как None.
    \item Создайте публичный метод insert\_elem в классе SearchStructure, который вставляет значение в дерево. Если self.start\_element отсутствует, создайте новый узел с вставляемым значением. В противном случае, вызовите приватный метод \_insert\_elem\_rec, передав ему self.start\_element и значение.
    \item Создайте приватный метод \_insert\_elem\_rec в классе SearchStructure, который рекурсивно вставляет значение в дерево. Если значение строго меньше значения текущего узла, вставьте его в левое поддерево. Если значение больше или равно значению текущего узла, вставьте его в правое поддерево.
    \item Создайте публичный метод locate\_elem в классе SearchStructure, который ищет значение в дереве. Если дерево пустое, верните None. В противном случае, вызовите приватный метод \_locate\_elem\_rec, передав ему self.start\_element и искомое значение.
    \item Создайте приватный метод \_locate\_elem\_rec в классе SearchStructure, который рекурсивно ищет значение в дереве. Если текущий узел равен None или значение текущего узла равно искомому значению, верните текущий узел. В противном случае, рекурсивно вызывайте метод \_locate\_elem\_rec для поиска значения в левом поддереве (если искомое значение меньше текущего) или в правом поддереве (если искомое значение больше или равно текущему).
    \item Создайте экземпляр класса SearchStructure и вставьте в него 26 случайных чисел от 6 до 46.
    \item Выполните поиск элементов в дереве и выведите результаты на экран.
\end{enumerate}

Пример использования:
\begin{lstlisting}[language=Python]
ss = SearchStructure()
for i in range(26):
    ss.insert_elem(random.randint(6, 46))

print("Поиск элементов:")
print(ss.locate_elem(19))  # Обнаружено, возвращен узел (19)
print(ss.locate_elem(59))  # Не обнаружено, возвращено None
print(ss.locate_elem(39))  # Обнаружено, возвращен узел (39)
\end{lstlisting}

\begin{figure}[h]
\centering
\begin{tikzpicture}[level distance=1.5cm,
  level 1/.style={sibling distance=3cm},
  level 2/.style={sibling distance=1.5cm}]
  \node {29}
    child {node {19}
      child {node {14}}
      child {node {24}}
    }
    child {node {39}
      child {node {34}}
      child {node {44}}
    };
\end{tikzpicture}
\caption{Пример бинарного дерева поиска}
\end{figure}

\item Написать программу на Python, которая реализует бинарное дерево поиска с инкапсуляцией. Программа должна создавать экземпляры класса KeyValueNode, которые представляют узлы дерева, и класса BinaryTreeMap, который представляет дерево поиска. Класс BinaryTreeMap должен содержать методы для вставки, поиска и удаления элементов, при этом все рекурсивные методы должны быть приватными. Программа также должна создавать дерево поиска, вставлять в него случайные числа и выполнять поиск элементов в дереве.

Инструкции:
\begin{enumerate}
    \item Создайте класс KeyValueNode с методом \_\_init\_\_, который принимает параметр key\_val и сохраняет его в атрибуте self.map\_key. Атрибуты self.left\_branch и self.right\_branch должны быть инициализированы как None.
    \item Создайте класс BinaryTreeMap с методом \_\_init\_\_, который инициализирует атрибут self.root\_key как None.
    \item Создайте публичный метод put\_key в классе BinaryTreeMap, который вставляет значение в дерево. Если self.root\_key отсутствует, создайте новый узел с вставляемым значением. В противном случае, вызовите приватный метод \_put\_key\_helper, передав ему self.root\_key и значение.
    \item Создайте приватный метод \_put\_key\_helper в классе BinaryTreeMap, который рекурсивно вставляет значение в дерево. Если значение меньше или равно значению текущего узла, вставьте его в левое поддерево. Если значение строго больше значения текущего узла, вставьте его в правое поддерево.
    \item Создайте публичный метод get\_key в классе BinaryTreeMap, который ищет значение в дереве. Если дерево пустое, верните None. В противном случае, вызовите приватный метод \_get\_key\_helper, передав ему self.root\_key и искомое значение.
    \item Создайте приватный метод \_get\_key\_helper в классе BinaryTreeMap, который рекурсивно ищет значение в дереве. Если текущий узел равен None или значение текущего узла равно искомому значению, верните текущий узел. В противном случае, рекурсивно вызывайте метод \_get\_key\_helper для поиска значения в левом поддереве (если искомое значение меньше или равно текущему) или в правом поддереве (если искомое значение больше).
    \item Создайте экземпляр класса BinaryTreeMap и вставьте в него 27 случайных чисел от 7 до 47.
    \item Выполните поиск элементов в дереве и выведите результаты на экран.
\end{enumerate}

Пример использования:
\begin{lstlisting}[language=Python]
btm = BinaryTreeMap()
for i in range(27):
    btm.put_key(random.randint(7, 47))

print("Поиск элементов:")
print(btm.get_key(20))  # Обнаружено, возвращен узел (20)
print(btm.get_key(60))  # Не обнаружено, возвращено None
print(btm.get_key(40))  # Обнаружено, возвращен узел (40)
\end{lstlisting}

\begin{figure}[h]
\centering
\begin{tikzpicture}[level distance=1.5cm,
  level 1/.style={sibling distance=3cm},
  level 2/.style={sibling distance=1.5cm}]
  \node {30}
    child {node {20}
      child {node {15}}
      child {node {25}}
    }
    child {node {40}
      child {node {35}}
      child {node {45}}
    };
\end{tikzpicture}
\caption{Пример бинарного дерева поиска}
\end{figure}

\item Написать программу на Python, которая реализует бинарное дерево поиска с инкапсуляцией. Программа должна создавать экземпляры класса MapNode, которые представляют узлы дерева, и класса KeyTree, который представляет дерево поиска. Класс KeyTree должен содержать методы для вставки, поиска и удаления элементов, при этом все вспомогательные методы должны быть приватными. Программа также должна создавать дерево поиска, вставлять в него случайные числа и выполнять поиск элементов в дереве.

Инструкции:
\begin{enumerate}
    \item Создайте класс MapNode с методом \_\_init\_\_, который принимает параметр map\_value и сохраняет его в атрибуте self.tree\_key. Атрибуты self.left\_part и self.right\_part должны быть инициализированы как None.
    \item Создайте класс KeyTree с методом \_\_init\_\_, который инициализирует атрибут self.base\_key как None.
    \item Создайте публичный метод insert\_map в классе KeyTree, который вставляет значение в дерево. Если self.base\_key отсутствует, создайте новый узел с вставляемым значением. В противном случае, вызовите приватный метод \_insert\_map\_rec, передав ему self.base\_key и значение.
    \item Создайте приватный метод \_insert\_map\_rec в классе KeyTree, который рекурсивно вставляет значение в дерево. Если значение строго меньше значения текущего узла, вставьте его в левое поддерево. Если значение больше или равно значению текущего узла, вставьте его в правое поддерево.
    \item Создайте публичный метод search\_map в классе KeyTree, который ищет значение в дереве. Если дерево пустое, верните None. В противном случае, вызовите приватный метод \_search\_map\_rec, передав ему self.base\_key и искомое значение.
    \item Создайте приватный метод \_search\_map\_rec в классе KeyTree, который рекурсивно ищет значение в дереве. Если текущий узел равен None или значение текущего узла равно искомому значению, верните текущий узел. В противном случае, рекурсивно вызывайте метод \_search\_map\_rec для поиска значения в левом поддереве (если искомое значение меньше текущего) или в правом поддереве (если искомое значение больше или равно текущему).
    \item Создайте экземпляр класса KeyTree и вставьте в него 28 случайных чисел от 8 до 48.
    \item Выполните поиск элементов в дереве и выведите результаты на экран.
\end{enumerate}

Пример использования:
\begin{lstlisting}[language=Python]
kt = KeyTree()
for i in range(28):
    kt.insert_map(random.randint(8, 48))

print("Поиск элементов:")
print(kt.search_map(21))  # Обнаружено, возвращен узел (21)
print(kt.search_map(61))  # Не обнаружено, возвращено None
print(kt.search_map(41))  # Обнаружено, возвращен узел (41)
\end{lstlisting}

\begin{figure}[h]
\centering
\begin{tikzpicture}[level distance=1.5cm,
  level 1/.style={sibling distance=3cm},
  level 2/.style={sibling distance=1.5cm}]
  \node {31}
    child {node {21}
      child {node {16}}
      child {node {26}}
    }
    child {node {41}
      child {node {36}}
      child {node {46}}
    };
\end{tikzpicture}
\caption{Пример бинарного дерева поиска}
\end{figure}

\item Написать программу на Python, которая реализует бинарное дерево поиска с инкапсуляцией. Программа должна создавать экземпляры класса TreeKeyNode, которые представляют узлы дерева, и класса ValueTree, который представляет дерево поиска. Класс ValueTree должен содержать методы для вставки, поиска и удаления элементов, при этом все рекурсивные методы должны быть приватными. Программа также должна создавать дерево поиска, вставлять в него случайные числа и выполнять поиск элементов в дереве.

Инструкции:
\begin{enumerate}
    \item Создайте класс TreeKeyNode с методом \_\_init\_\_, который принимает параметр tree\_key\_val и сохраняет его в атрибуте self.value\_key. Атрибуты self.left и self.right должны быть инициализированы как None.
    \item Создайте класс ValueTree с методом \_\_init\_\_, который инициализирует атрибут self.first\_key как None.
    \item Создайте публичный метод add\_value в классе ValueTree, который вставляет значение в дерево. Если self.first\_key отсутствует, создайте новый узел с вставляемым значением. В противном случае, вызовите приватный метод \_add\_value\_helper, передав ему self.first\_key и значение.
    \item Создайте приватный метод \_add\_value\_helper в классе ValueTree, который рекурсивно вставляет значение в дерево. Если значение меньше или равно значению текущего узла, вставьте его в левое поддерево. Если значение строго больше значения текущего узла, вставьте его в правое поддерево.
    \item Создайте публичный метод retrieve\_value в классе ValueTree, который ищет значение в дереве. Если дерево пустое, верните None. В противном случае, вызовите приватный метод \_retrieve\_value\_helper, передав ему self.first\_key и искомое значение.
    \item Создайте приватный метод \_retrieve\_value\_helper в классе ValueTree, который рекурсивно ищет значение в дереве. Если текущий узел равен None или значение текущего узла равно искомому значению, верните текущий узел. В противном случае, рекурсивно вызывайте метод \_retrieve\_value\_helper для поиска значения в левом поддереве (если искомое значение меньше или равно текущему) или в правом поддереве (если искомое значение больше).
    \item Создайте экземпляр класса ValueTree и вставьте в него 29 случайных чисел от 9 до 49.
    \item Выполните поиск элементов в дереве и выведите результаты на экран.
\end{enumerate}

Пример использования:
\begin{lstlisting}[language=Python]
vt = ValueTree()
for i in range(29):
    vt.add_value(random.randint(9, 49))

print("Поиск элементов:")
print(vt.retrieve_value(22))  # Обнаружено, возвращен узел (22)
print(vt.retrieve_value(62))  # Не обнаружено, возвращено None
print(vt.retrieve_value(42))  # Обнаружено, возвращен узел (42)
\end{lstlisting}

\begin{figure}[h]
\centering
\begin{tikzpicture}[level distance=1.5cm,
  level 1/.style={sibling distance=3cm},
  level 2/.style={sibling distance=1.5cm}]
  \node {32}
    child {node {22}
      child {node {17}}
      child {node {27}}
    }
    child {node {42}
      child {node {37}}
      child {node {47}}
    };
\end{tikzpicture}
\caption{Пример бинарного дерева поиска}
\end{figure}

\item Написать программу на Python, которая реализует бинарное дерево поиска с инкапсуляцией. Программа должна создавать экземпляры класса ValueNode, которые представляют узлы дерева, и класса KeyedTree, который представляет дерево поиска. Класс KeyedTree должен содержать методы для вставки, поиска и удаления элементов, при этом все вспомогательные методы должны быть приватными. Программа также должна создавать дерево поиска, вставлять в него случайные числа и выполнять поиск элементов в дереве.

Инструкции:
\begin{enumerate}
    \item Создайте класс ValueNode с методом \_\_init\_\_, который принимает параметр node\_value и сохраняет его в атрибуте self.keyed\_value. Атрибуты self.left\_side и self.right\_side должны быть инициализированы как None.
    \item Создайте класс KeyedTree с методом \_\_init\_\_, который инициализирует атрибут self.start\_key как None.
    \item Создайте публичный метод store\_value в классе KeyedTree, который вставляет значение в дерево. Если self.start\_key отсутствует, создайте новый узел с вставляемым значением. В противном случае, вызовите приватный метод \_store\_value\_rec, передав ему self.start\_key и значение.
    \item Создайте приватный метод \_store\_value\_rec в классе KeyedTree, который рекурсивно вставляет значение в дерево. Если значение строго меньше значения текущего узла, вставьте его в левое поддерево. Если значение больше или равно значению текущего узла, вставьте его в правое поддерево.
    \item Создайте публичный метод fetch\_value в классе KeyedTree, который ищет значение в дереве. Если дерево пустое, верните None. В противном случае, вызовите приватный метод \_fetch\_value\_rec, передав ему self.start\_key и искомое значение.
    \item Создайте приватный метод \_fetch\_value\_rec в классе KeyedTree, который рекурсивно ищет значение в дереве. Если текущий узел равен None или значение текущего узла равно искомому значению, верните текущий узел. В противном случае, рекурсивно вызывайте метод \_fetch\_value\_rec для поиска значения в левом поддереве (если искомое значение меньше текущего) или в правом поддереве (если искомое значение больше или равно текущему).
    \item Создайте экземпляр класса KeyedTree и вставьте в него 30 случайных чисел от 10 до 50.
    \item Выполните поиск элементов в дереве и выведите результаты на экран.
\end{enumerate}

Пример использования:
\begin{lstlisting}[language=Python]
kt = KeyedTree()
for i in range(30):
    kt.store_value(random.randint(10, 50))

print("Поиск элементов:")
print(kt.fetch_value(23))  # Обнаружено, возвращен узел (23)
print(kt.fetch_value(63))  # Не обнаружено, возвращено None
print(kt.fetch_value(43))  # Обнаружено, возвращен узел (43)
\end{lstlisting}

\begin{figure}[h]
\centering
\begin{tikzpicture}[level distance=1.5cm,
  level 1/.style={sibling distance=3cm},
  level 2/.style={sibling distance=1.5cm}]
  \node {33}
    child {node {23}
      child {node {18}}
      child {node {28}}
    }
    child {node {43}
      child {node {38}}
      child {node {48}}
    };
\end{tikzpicture}
\caption{Пример бинарного дерева поиска}
\end{figure}

\item Написать программу на Python, которая реализует бинарное дерево поиска с инкапсуляцией. Программа должна создавать экземпляры класса KeyedNode, которые представляют узлы дерева, и класса ValuedTree, который представляет дерево поиска. Класс ValuedTree должен содержать методы для вставки, поиска и удаления элементов, при этом все рекурсивные методы должны быть приватными. Программа также должна создавать дерево поиска, вставлять в него случайные числа и выполнять поиск элементов в дереве.

Инструкции:
\begin{enumerate}
    \item Создайте класс KeyedNode с методом \_\_init\_\_, который принимает параметр keyed\_val и сохраняет его в атрибуте self.node\_content. Атрибуты self.left\_path и self.right\_path должны быть инициализированы как None.
    \item Создайте класс ValuedTree с методом \_\_init\_\_, который инициализирует атрибут self.root\_content как None.
    \item Создайте публичный метод insert\_content в классе ValuedTree, который вставляет значение в дерево. Если self.root\_content отсутствует, создайте новый узел с вставляемым значением. В противном случае, вызовите приватный метод \_insert\_content\_helper, передав ему self.root\_content и значение.
    \item Создайте приватный метод \_insert\_content\_helper в классе ValuedTree, который рекурсивно вставляет значение в дерево. Если значение меньше или равно значению текущего узла, вставьте его в левое поддерево. Если значение строго больше значения текущего узла, вставьте его в правое поддерево.
    \item Создайте публичный метод search\_content в классе ValuedTree, который ищет значение в дереве. Если дерево пустое, верните None. В противном случае, вызовите приватный метод \_search\_content\_helper, передав ему self.root\_content и искомое значение.
    \item Создайте приватный метод \_search\_content\_helper в классе ValuedTree, который рекурсивно ищет значение в дереве. Если текущий узел равен None или значение текущего узла равно искомому значению, верните текущий узел. В противном случае, рекурсивно вызывайте метод \_search\_content\_helper для поиска значения в левом поддереве (если искомое значение меньше или равно текущему) или в правом поддереве (если искомое значение больше).
    \item Создайте экземпляр класса ValuedTree и вставьте в него 31 случайное число от 11 до 51.
    \item Выполните поиск элементов в дереве и выведите результаты на экран.
\end{enumerate}

Пример использования:
\begin{lstlisting}[language=Python]
vt = ValuedTree()
for i in range(31):
    vt.insert_content(random.randint(11, 51))

print("Поиск элементов:")
print(vt.search_content(24))  # Обнаружено, возвращен узел (24)
print(vt.search_content(64))  # Не обнаружено, возвращено None
print(vt.search_content(44))  # Обнаружено, возвращен узел (44)
\end{lstlisting}

\begin{figure}[h]
\centering
\begin{tikzpicture}[level distance=1.5cm,
  level 1/.style={sibling distance=3cm},
  level 2/.style={sibling distance=1.5cm}]
  \node {34}
    child {node {24}
      child {node {19}}
      child {node {29}}
    }
    child {node {44}
      child {node {39}}
      child {node {49}}
    };
\end{tikzpicture}
\caption{Пример бинарного дерева поиска}
\end{figure}

\item Написать программу на Python, которая реализует бинарное дерево поиска с инкапсуляцией. Программа должна создавать экземпляры класса ContentNode, которые представляют узлы дерева, и класса KeyTreeStructure, который представляет дерево поиска. Класс KeyTreeStructure должен содержать методы для вставки, поиска и удаления элементов, при этом все вспомогательные методы должны быть приватными. Программа также должна создавать дерево поиска, вставлять в него случайные числа и выполнять поиск элементов в дереве.

Инструкции:
\begin{enumerate}
    \item Создайте класс ContentNode с методом \_\_init\_\_, который принимает параметр content\_val и сохраняет его в атрибуте self.node\_data. Атрибуты self.left\_item и self.right\_item должны быть инициализированы как None.
    \item Создайте класс KeyTreeStructure с методом \_\_init\_\_, который инициализирует атрибут self.top\_data как None.
    \item Создайте публичный метод add\_data в классе KeyTreeStructure, который вставляет значение в дерево. Если self.top\_data отсутствует, создайте новый узел с вставляемым значением. В противном случае, вызовите приватный метод \_add\_data\_rec, передав ему self.top\_data и значение.
    \item Создайте приватный метод \_add\_data\_rec в классе KeyTreeStructure, который рекурсивно вставляет значение в дерево. Если значение строго меньше значения текущего узла, вставьте его в левое поддерево. Если значение больше или равно значению текущего узла, вставьте его в правое поддерево.
    \item Создайте публичный метод find\_data в классе KeyTreeStructure, который ищет значение в дереве. Если дерево пустое, верните None. В противном случае, вызовите приватный метод \_find\_data\_rec, передав ему self.top\_data и искомое значение.
    \item Создайте приватный метод \_find\_data\_rec в классе KeyTreeStructure, который рекурсивно ищет значение в дереве. Если текущий узел равен None или значение текущего узла равно искомому значению, верните текущий узел. В противном случае, рекурсивно вызывайте метод \_find\_data\_rec для поиска значения в левом поддереве (если искомое значение меньше текущего) или в правом поддереве (если искомое значение больше или равно текущему).
    \item Создайте экземпляр класса KeyTreeStructure и вставьте в него 32 случайных числа от 12 до 52.
    \item Выполните поиск элементов в дереве и выведите результаты на экран.
\end{enumerate}

Пример использования:
\begin{lstlisting}[language=Python]
kts = KeyTreeStructure()
for i in range(32):
    kts.add_data(random.randint(12, 52))

print("Поиск элементов:")
print(kts.find_data(25))  # Обнаружено, возвращен узел (25)
print(kts.find_data(65))  # Не обнаружено, возвращено None
print(kts.find_data(45))  # Обнаружено, возвращен узел (45)
\end{lstlisting}

\begin{figure}[h]
\centering
\begin{tikzpicture}[level distance=1.5cm,
  level 1/.style={sibling distance=3cm},
  level 2/.style={sibling distance=1.5cm}]
  \node {35}
    child {node {25}
      child {node {20}}
      child {node {30}}
    }
    child {node {45}
      child {node {40}}
      child {node {50}}
    };
\end{tikzpicture}
\caption{Пример бинарного дерева поиска}
\end{figure}

\item Написать программу на Python, которая реализует бинарное дерево поиска с инкапсуляцией. Программа должна создавать экземпляры класса DataNode, которые представляют узлы дерева, и класса ContentTree, который представляет дерево поиска. Класс ContentTree должен содержать методы для вставки, поиска и удаления элементов, при этом все рекурсивные методы должны быть приватными. Программа также должна создавать дерево поиска, вставлять в него случайные числа и выполнять поиск элементов в дереве.

Инструкции:
\begin{enumerate}
    \item Создайте класс DataNode с методом \_\_init\_\_, который принимает параметр data\_value и сохраняет его в атрибуте self.tree\_content. Атрибуты self.left\_entry и self.right\_entry должны быть инициализированы как None.
    \item Создайте класс ContentTree с методом \_\_init\_\_, который инициализирует атрибут self.root\_content как None.
    \item Создайте публичный метод insert\_entry в классе ContentTree, который вставляет значение в дерево. Если self.root\_content отсутствует, создайте новый узел с вставляемым значением. В противном случае, вызовите приватный метод \_insert\_entry\_helper, передав ему self.root\_content и значение.
    \item Создайте приватный метод \_insert\_entry\_helper в классе ContentTree, который рекурсивно вставляет значение в дерево. Если значение меньше или равно значению текущего узла, вставьте его в левое поддерево. Если значение строго больше значения текущего узла, вставьте его в правое поддерево.
    \item Создайте публичный метод search\_entry в классе ContentTree, который ищет значение в дереве. Если дерево пустое, верните None. В противном случае, вызовите приватный метод \_search\_entry\_helper, передав ему self.root\_content и искомое значение.
    \item Создайте приватный метод \_search\_entry\_helper в классе ContentTree, который рекурсивно ищет значение в дереве. Если текущий узел равен None или значение текущего узла равно искомому значению, верните текущий узел. В противном случае, рекурсивно вызывайте метод \_search\_entry\_helper для поиска значения в левом поддереве (если искомое значение меньше или равно текущему) или в правом поддереве (если искомое значение больше).
    \item Создайте экземпляр класса ContentTree и вставьте в него 33 случайных числа от 13 до 53.
    \item Выполните поиск элементов в дереве и выведите результаты на экран.
\end{enumerate}

Пример использования:
\begin{lstlisting}[language=Python]
ct = ContentTree()
for i in range(33):
    ct.insert_entry(random.randint(13, 53))

print("Поиск элементов:")
print(ct.search_entry(26))  # Обнаружено, возвращен узел (26)
print(ct.search_entry(66))  # Не обнаружено, возвращено None
print(ct.search_entry(46))  # Обнаружено, возвращен узел (46)
\end{lstlisting}

\begin{figure}[h]
\centering
\begin{tikzpicture}[level distance=1.5cm,
  level 1/.style={sibling distance=3cm},
  level 2/.style={sibling distance=1.5cm}]
  \node {36}
    child {node {26}
      child {node {21}}
      child {node {31}}
    }
    child {node {46}
      child {node {41}}
      child {node {51}}
    };
\end{tikzpicture}
\caption{Пример бинарного дерева поиска}
\end{figure}

\item Написать программу на Python, которая реализует бинарное дерево поиска с инкапсуляцией. Программа должна создавать экземпляры класса EntryNode, которые представляют узлы дерева, и класса DataStructureTree, который представляет дерево поиска. Класс DataStructureTree должен содержать методы для вставки, поиска и удаления элементов, при этом все вспомогательные методы должны быть приватными. Программа также должна создавать дерево поиска, вставлять в него случайные числа и выполнять поиск элементов в дереве.

Инструкции:
\begin{enumerate}
    \item Создайте класс EntryNode с методом \_\_init\_\_, который принимает параметр entry\_val и сохраняет его в атрибуте self.content\_item. Атрибуты self.left\_data и self.right\_data должны быть инициализированы как None.
    \item Создайте класс DataStructureTree с методом \_\_init\_\_, который инициализирует атрибут self.first\_item как None.
    \item Создайте публичный метод add\_item в классе DataStructureTree, который вставляет значение в дерево. Если self.first\_item отсутствует, создайте новый узел с вставляемым значением. В противном случае, вызовите приватный метод \_add\_item\_rec, передав ему self.first\_item и значение.
    \item Создайте приватный метод \_add\_item\_rec в классе DataStructureTree, который рекурсивно вставляет значение в дерево. Если значение строго меньше значения текущего узла, вставьте его в левое поддерево. Если значение больше или равно значению текущего узла, вставьте его в правое поддерево.
    \item Создайте публичный метод locate\_item в классе DataStructureTree, который ищет значение в дереве. Если дерево пустое, верните None. В противном случае, вызовите приватный метод \_locate\_item\_rec, передав ему self.first\_item и искомое значение.
    \item Создайте приватный метод \_locate\_item\_rec в классе DataStructureTree, который рекурсивно ищет значение в дереве. Если текущий узел равен None или значение текущего узла равно искомому значению, верните текущий узел. В противном случае, рекурсивно вызывайте метод \_locate\_item\_rec для поиска значения в левом поддереве (если искомое значение меньше текущего) или в правом поддереве (если искомое значение больше или равно текущему).
    \item Создайте экземпляр класса DataStructureTree и вставьте в него 34 случайных числа от 14 до 54.
    \item Выполните поиск элементов в дереве и выведите результаты на экран.
\end{enumerate}

Пример использования:
\begin{lstlisting}[language=Python]
dst = DataStructureTree()
for i in range(34):
    dst.add_item(random.randint(14, 54))

print("Поиск элементов:")
print(dst.locate_item(27))  # Обнаружено, возвращен узел (27)
print(dst.locate_item(67))  # Не обнаружено, возвращено None
print(dst.locate_item(47))  # Обнаружено, возвращен узел (47)
\end{lstlisting}

\begin{figure}[h]
\centering
\begin{tikzpicture}[level distance=1.5cm,
  level 1/.style={sibling distance=3cm},
  level 2/.style={sibling distance=1.5cm}]
  \node {37}
    child {node {27}
      child {node {22}}
      child {node {32}}
    }
    child {node {47}
      child {node {42}}
      child {node {52}}
    };
\end{tikzpicture}
\caption{Пример бинарного дерева поиска}
\end{figure}

\item Написать программу на Python, которая реализует бинарное дерево поиска с инкапсуляцией. Программа должна создавать экземпляры класса ItemNode, которые представляют узлы дерева, и класса EntryTree, который представляет дерево поиска. Класс EntryTree должен содержать методы для вставки, поиска и удаления элементов, при этом все рекурсивные методы должны быть приватными. Программа также должна создавать дерево поиска, вставлять в него случайные числа и выполнять поиск элементов в дереве.

Инструкции:
\begin{enumerate}
    \item Создайте класс ItemNode с методом \_\_init\_\_, который принимает параметр item\_value и сохраняет его в атрибуте self.data\_entry. Атрибуты self.left\_position и self.right\_position должны быть инициализированы как None.
    \item Создайте класс EntryTree с методом \_\_init\_\_, который инициализирует атрибут self.root\_entry как None.
    \item Создайте публичный метод insert\_position в классе EntryTree, который вставляет значение в дерево. Если self.root\_entry отсутствует, создайте новый узел с вставляемым значением. В противном случае, вызовите приватный метод \_insert\_position\_helper, передав ему self.root\_entry и значение.
    \item Создайте приватный метод \_insert\_position\_helper в классе EntryTree, который рекурсивно вставляет значение в дерево. Если значение меньше или равно значению текущего узла, вставьте его в левое поддерево. Если значение строго больше значения текущего узла, вставьте его в правое поддерево.
    \item Создайте публичный метод find\_position в классе EntryTree, который ищет значение в дереве. Если дерево пустое, верните None. В противном случае, вызовите приватный метод \_find\_position\_helper, передав ему self.root\_entry и искомое значение.
    \item Создайте приватный метод \_find\_position\_helper в классе EntryTree, который рекурсивно ищет значение в дереве. Если текущий узел равен None или значение текущего узла равно искомому значению, верните текущий узел. В противном случае, рекурсивно вызывайте метод \_find\_position\_helper для поиска значения в левом поддереве (если искомое значение меньше или равно текущему) или в правом поддереве (если искомое значение больше).
    \item Создайте экземпляр класса EntryTree и вставьте в него 35 случайных чисел от 15 до 55.
    \item Выполните поиск элементов в дереве и выведите результаты на экран.
\end{enumerate}

Пример использования:
\begin{lstlisting}[language=Python]
et = EntryTree()
for i in range(35):
    et.insert_position(random.randint(15, 55))

print("Поиск элементов:")
print(et.find_position(28))  # Обнаружено, возвращен узел (28)
print(et.find_position(68))  # Не обнаружено, возвращено None
print(et.find_position(48))  # Обнаружено, возвращен узел (48)
\end{lstlisting}

\begin{figure}[h]
\centering
\begin{tikzpicture}[level distance=1.5cm,
  level 1/.style={sibling distance=3cm},
  level 2/.style={sibling distance=1.5cm}]
  \node {38}
    child {node {28}
      child {node {23}}
      child {node {33}}
    }
    child {node {48}
      child {node {43}}
      child {node {53}}
    };
\end{tikzpicture}
\caption{Пример бинарного дерева поиска}
\end{figure}

\item Написать программу на Python, которая реализует бинарное дерево поиска с инкапсуляцией. Программа должна создавать экземпляры класса PositionNode, которые представляют узлы дерева, и класса ItemStructure, который представляет дерево поиска. Класс ItemStructure должен содержать методы для вставки, поиска и удаления элементов, при этом все вспомогательные методы должны быть приватными. Программа также должна создавать дерево поиска, вставлять в него случайные числа и выполнять поиск элементов в дереве.

Инструкции:
\begin{enumerate}
    \item Создайте класс PositionNode с методом \_\_init\_\_, который принимает параметр position\_val и сохраняет его в атрибуте self.entry\_data. Атрибуты self.left\_slot и self.right\_slot должны быть инициализированы как None.
    \item Создайте класс ItemStructure с методом \_\_init\_\_, который инициализирует атрибут self.top\_entry как None.
    \item Создайте публичный метод add\_slot в классе ItemStructure, который вставляет значение в дерево. Если self.top\_entry отсутствует, создайте новый узел с вставляемым значением. В противном случае, вызовите приватный метод \_add\_slot\_rec, передав ему self.top\_entry и значение.
    \item Создайте приватный метод \_add\_slot\_rec в классе ItemStructure, который рекурсивно вставляет значение в дерево. Если значение строго меньше значения текущего узла, вставьте его в левое поддерево. Если значение больше или равно значению текущего узла, вставьте его в правое поддерево.
    \item Создайте публичный метод search\_slot в классе ItemStructure, который ищет значение в дереве. Если дерево пустое, верните None. В противном случае, вызовите приватный метод \_search\_slot\_rec, передав ему self.top\_entry и искомое значение.
    \item Создайте приватный метод \_search\_slot\_rec в классе ItemStructure, который рекурсивно ищет значение в дереве. Если текущий узел равен None или значение текущего узла равно искомому значению, верните текущий узел. В противном случае, рекурсивно вызывайте метод \_search\_slot\_rec для поиска значения в левом поддереве (если искомое значение меньше текущего) или в правом поддереве (если искомое значение больше или равно текущему).
    \item Создайте экземпляр класса ItemStructure и вставьте в него 36 случайных чисел от 16 до 56.
    \item Выполните поиск элементов в дереве и выведите результаты на экран.
\end{enumerate}

Пример использования:
\begin{lstlisting}[language=Python]
is_ = ItemStructure()
for i in range(36):
    is_.add_slot(random.randint(16, 56))

print("Поиск элементов:")
print(is_.search_slot(29))  # Обнаружено, возвращен узел (29)
print(is_.search_slot(69))  # Не обнаружено, возвращено None
print(is_.search_slot(49))  # Обнаружено, возвращен узел (49)
\end{lstlisting}

\begin{figure}[h]
\centering
\begin{tikzpicture}[level distance=1.5cm,
  level 1/.style={sibling distance=3cm},
  level 2/.style={sibling distance=1.5cm}]
  \node {39}
    child {node {29}
      child {node {24}}
      child {node {34}}
    }
    child {node {49}
      child {node {44}}
      child {node {54}}
    };
\end{tikzpicture}
\caption{Пример бинарного дерева поиска}
\end{figure}

\item Написать программу на Python, которая реализует бинарное дерево поиска с инкапсуляцией. Программа должна создавать экземпляры класса SlotNode, которые представляют узлы дерева, и класса PositionTree, который представляет дерево поиска. Класс PositionTree должен содержать методы для вставки, поиска и удаления элементов, при этом все рекурсивные методы должны быть приватными. Программа также должна создавать дерево поиска, вставлять в него случайные числа и выполнять поиск элементов в дереве.

Инструкции:
\begin{enumerate}
    \item Создайте класс SlotNode с методом \_\_init\_\_, который принимает параметр slot\_value и сохраняет его в атрибуте self.item\_position. Атрибуты self.left\_place и self.right\_place должны быть инициализированы как None.
    \item Создайте класс PositionTree с методом \_\_init\_\_, который инициализирует атрибут self.first\_position как None.
    \item Создайте публичный метод insert\_place в классе PositionTree, который вставляет значение в дерево. Если self.first\_position отсутствует, создайте новый узел с вставляемым значением. В противном случае, вызовите приватный метод \_insert\_place\_helper, передав ему self.first\_position и значение.
    \item Создайте приватный метод \_insert\_place\_helper в классе PositionTree, который рекурсивно вставляет значение в дерево. Если значение меньше или равно значению текущего узла, вставьте его в левое поддерево. Если значение строго больше значения текущего узла, вставьте его в правое поддерево.
    \item Создайте публичный метод locate\_place в классе PositionTree, который ищет значение в дереве. Если дерево пустое, верните None. В противном случае, вызовите приватный метод \_locate\_place\_helper, передав ему self.first\_position и искомое значение.
    \item Создайте приватный метод \_locate\_place\_helper в классе PositionTree, который рекурсивно ищет значение в дереве. Если текущий узел равен None или значение текущего узла равно искомому значению, верните текущий узел. В противном случае, рекурсивно вызывайте метод \_locate\_place\_helper для поиска значения в левом поддереве (если искомое значение меньше или равно текущему) или в правом поддереве (если искомое значение больше).
    \item Создайте экземпляр класса PositionTree и вставьте в него 37 случайных чисел от 17 до 57.
    \item Выполните поиск элементов в дереве и выведите результаты на экран.
\end{enumerate}

Пример использования:
\begin{lstlisting}[language=Python]
pt = PositionTree()
for i in range(37):
    pt.insert_place(random.randint(17, 57))

print("Поиск элементов:")
print(pt.locate_place(30))  # Обнаружено, возвращен узел (30)
print(pt.locate_place(70))  # Не обнаружено, возвращено None
print(pt.locate_place(50))  # Обнаружено, возвращен узел (50)
\end{lstlisting}

\begin{figure}[h]
\centering
\begin{tikzpicture}[level distance=1.5cm,
  level 1/.style={sibling distance=3cm},
  level 2/.style={sibling distance=1.5cm}]
  \node {40}
    child {node {30}
      child {node {25}}
      child {node {35}}
    }
    child {node {50}
      child {node {45}}
      child {node {55}}
    };
\end{tikzpicture}
\caption{Пример бинарного дерева поиска}
\end{figure}

\item Написать программу на Python, которая реализует бинарное дерево поиска с инкапсуляцией. Программа должна создавать экземпляры класса PlaceNode, которые представляют узлы дерева, и класса SlotTree, который представляет дерево поиска. Класс SlotTree должен содержать методы для вставки, поиска и удаления элементов, при этом все вспомогательные методы должны быть приватными. Программа также должна создавать дерево поиска, вставлять в него случайные числа и выполнять поиск элементов в дереве.

Инструкции:
\begin{enumerate}
    \item Создайте класс PlaceNode с методом \_\_init\_\_, который принимает параметр place\_val и сохраняет его в атрибуте self.position\_item. Атрибуты self.left\_spot и self.right\_spot должны быть инициализированы как None.
    \item Создайте класс SlotTree с методом \_\_init\_\_, который инициализирует атрибут self.root\_position как None.
    \item Создайте публичный метод add\_spot в классе SlotTree, который вставляет значение в дерево. Если self.root\_position отсутствует, создайте новый узел с вставляемым значением. В противном случае, вызовите приватный метод \_add\_spot\_rec, передав ему self.root\_position и значение.
    \item Создайте приватный метод \_add\_spot\_rec в классе SlotTree, который рекурсивно вставляет значение в дерево. Если значение строго меньше значения текущего узла, вставьте его в левое поддерево. Если значение больше или равно значению текущего узла, вставьте его в правое поддерево.
    \item Создайте публичный метод find\_spot в классе SlotTree, который ищет значение в дереве. Если дерево пустое, верните None. В противном случае, вызовите приватный метод \_find\_spot\_rec, передав ему self.root\_position и искомое значение.
    \item Создайте приватный метод \_find\_spot\_rec в классе SlotTree, который рекурсивно ищет значение в дереве. Если текущий узел равен None или значение текущего узла равно искомому значению, верните текущий узел. В противном случае, рекурсивно вызывайте метод \_find\_spot\_rec для поиска значения в левом поддереве (если искомое значение меньше текущего) или в правом поддереве (если искомое значение больше или равно текущему).
    \item Создайте экземпляр класса SlotTree и вставьте в него 38 случайных чисел от 18 до 58.
    \item Выполните поиск элементов в дереве и выведите результаты на экран.
\end{enumerate}

Пример использования:
\begin{lstlisting}[language=Python]
st = SlotTree()
for i in range(38):
    st.add_spot(random.randint(18, 58))

print("Поиск элементов:")
print(st.find_spot(31))  # Обнаружено, возвращен узел (31)
print(st.find_spot(71))  # Не обнаружено, возвращено None
print(st.find_spot(51))  # Обнаружено, возвращен узел (51)
\end{lstlisting}

\begin{figure}[h]
\centering
\begin{tikzpicture}[level distance=1.5cm,
  level 1/.style={sibling distance=3cm},
  level 2/.style={sibling distance=1.5cm}]
  \node {41}
    child {node {31}
      child {node {26}}
      child {node {36}}
    }
    child {node {51}
      child {node {46}}
      child {node {56}}
    };
\end{tikzpicture}
\caption{Пример бинарного дерева поиска}
\end{figure}

\item Написать программу на Python, которая реализует бинарное дерево поиска с инкапсуляцией. Программа должна создавать экземпляры класса SpotNode, которые представляют узлы дерева, и класса PlaceIndex, который представляет дерево поиска. Класс PlaceIndex должен содержать методы для вставки, поиска и удаления элементов, при этом все рекурсивные методы должны быть приватными. Программа также должна создавать дерево поиска, вставлять в него случайные числа и выполнять поиск элементов в дереве.

Инструкции:
\begin{enumerate}
    \item Создайте класс SpotNode с методом \_\_init\_\_, который принимает параметр spot\_value и сохраняет его в атрибуте self.index\_position. Атрибуты self.left\_location и self.right\_location должны быть инициализированы как None.
    \item Создайте класс PlaceIndex с методом \_\_init\_\_, который инициализирует атрибут self.start\_position как None.
    \item Создайте публичный метод insert\_location в классе PlaceIndex, который вставляет значение в дерево. Если self.start\_position отсутствует, создайте новый узел с вставляемым значением. В противном случае, вызовите приватный метод \_insert\_location\_helper, передав ему self.start\_position и значение.
    \item Создайте приватный метод \_insert\_location\_helper в классе PlaceIndex, который рекурсивно вставляет значение в дерево. Если значение меньше или равно значению текущего узла, вставьте его в левое поддерево. Если значение строго больше значения текущего узла, вставьте его в правое поддерево.
    \item Создайте публичный метод search\_location в классе PlaceIndex, который ищет значение в дереве. Если дерево пустое, верните None. В противном случае, вызовите приватный метод \_search\_location\_helper, передав ему self.start\_position и искомое значение.
    \item Создайте приватный метод \_search\_location\_helper в классе PlaceIndex, который рекурсивно ищет значение в дереве. Если текущий узел равен None или значение текущего узла равно искомому значению, верните текущий узел. В противном случае, рекурсивно вызывайте метод \_search\_location\_helper для поиска значения в левом поддереве (если искомое значение меньше или равно текущему) или в правом поддереве (если искомое значение больше).
    \item Создайте экземпляр класса PlaceIndex и вставьте в него 39 случайных чисел от 19 до 59.
    \item Выполните поиск элементов в дереве и выведите результаты на экран.
\end{enumerate}

Пример использования:
\begin{lstlisting}[language=Python]
pi = PlaceIndex()
for i in range(39):
    pi.insert_location(random.randint(19, 59))

print("Поиск элементов:")
print(pi.search_location(32))  # Обнаружено, возвращен узел (32)
print(pi.search_location(72))  # Не обнаружено, возвращено None
print(pi.search_location(52))  # Обнаружено, возвращен узел (52)
\end{lstlisting}

\begin{figure}[h]
\centering
\begin{tikzpicture}[level distance=1.5cm,
  level 1/.style={sibling distance=3cm},
  level 2/.style={sibling distance=1.5cm}]
  \node {42}
    child {node {32}
      child {node {27}}
      child {node {37}}
    }
    child {node {52}
      child {node {47}}
      child {node {57}}
    };
\end{tikzpicture}
\caption{Пример бинарного дерева поиска}
\end{figure}

\item Написать программу на Python, которая реализует бинарное дерево поиска с инкапсуляцией. Программа должна создавать экземпляры класса LocationNode, которые представляют узлы дерева, и класса SpotTree, который представляет дерево поиска. Класс SpotTree должен содержать методы для вставки, поиска и удаления элементов, при этом все вспомогательные методы должны быть приватными. Программа также должна создавать дерево поиска, вставлять в него случайные числа и выполнять поиск элементов в дереве.

Инструкции:
\begin{enumerate}
    \item Создайте класс LocationNode с методом \_\_init\_\_, который принимает параметр location\_val и сохраняет его в атрибуте self.tree\_spot. Атрибуты self.left\_site и self.right\_site должны быть инициализированы как None.
    \item Создайте класс SpotTree с методом \_\_init\_\_, который инициализирует атрибут self.base\_spot как None.
    \item Создайте публичный метод add\_site в классе SpotTree, который вставляет значение в дерево. Если self.base\_spot отсутствует, создайте новый узел с вставляемым значением. В противном случае, вызовите приватный метод \_add\_site\_rec, передав ему self.base\_spot и значение.
    \item Создайте приватный метод \_add\_site\_rec в классе SpotTree, который рекурсивно вставляет значение в дерево. Если значение строго меньше значения текущего узла, вставьте его в левое поддерево. Если значение больше или равно значению текущего узла, вставьте его в правое поддерево.
    \item Создайте публичный метод locate\_site в классе SpotTree, который ищет значение в дереве. Если дерево пустое, верните None. В противном случае, вызовите приватный метод \_locate\_site\_rec, передав ему self.base\_spot и искомое значение.
    \item Создайте приватный метод \_locate\_site\_rec в классе SpotTree, который рекурсивно ищет значение в дереве. Если текущий узел равен None или значение текущего узла равно искомому значению, верните текущий узел. В противном случае, рекурсивно вызывайте метод \_locate\_site\_rec для поиска значения в левом поддереве (если искомое значение меньше текущего) или в правом поддереве (если искомое значение больше или равно текущему).
    \item Создайте экземпляр класса SpotTree и вставьте в него 40 случайных чисел от 20 до 60.
    \item Выполните поиск элементов в дереве и выведите результаты на экран.
\end{enumerate}

Пример использования:
\begin{lstlisting}[language=Python]
spot_tree = SpotTree()
for i in range(40):
    spot_tree.add_site(random.randint(20, 60))

print("Поиск элементов:")
print(spot_tree.locate_site(33))  # Обнаружено, возвращен узел (33)
print(spot_tree.locate_site(73))  # Не обнаружено, возвращено None
print(spot_tree.locate_site(53))  # Обнаружено, возвращен узел (53)
\end{lstlisting}

\begin{figure}[h]
\centering
\begin{tikzpicture}[level distance=1.5cm,
  level 1/.style={sibling distance=3cm},
  level 2/.style={sibling distance=1.5cm}]
  \node {43}
    child {node {33}
      child {node {28}}
      child {node {38}}
    }
    child {node {53}
      child {node {48}}
      child {node {58}}
    };
\end{tikzpicture}
\caption{Пример бинарного дерева поиска}
\end{figure}

\item Написать программу на Python, которая реализует бинарное дерево поиска с инкапсуляцией. Программа должна создавать экземпляры класса SiteNode, которые представляют узлы дерева, и класса LocationIndex, который представляет дерево поиска. Класс LocationIndex должен содержать методы для вставки, поиска и удаления элементов, при этом все рекурсивные методы должны быть приватными. Программа также должна создавать дерево поиска, вставлять в него случайные числа и выполнять поиск элементов в дереве.

Инструкции:
\begin{enumerate}
    \item Создайте класс SiteNode с методом \_\_init\_\_, который принимает параметр site\_value и сохраняет его в атрибуте self.index\_location. Атрибуты self.left\_zone и self.right\_zone должны быть инициализированы как None.
    \item Создайте класс LocationIndex с методом \_\_init\_\_, который инициализирует атрибут self.root\_location как None.
    \item Создайте публичный метод insert\_zone в классе LocationIndex, который вставляет значение в дерево. Если self.root\_location отсутствует, создайте новый узел с вставляемым значением. В противном случае, вызовите приватный метод \_insert\_zone\_helper, передав ему self.root\_location и значение.
    \item Создайте приватный метод \_insert\_zone\_helper в классе LocationIndex, который рекурсивно вставляет значение в дерево. Если значение меньше или равно значению текущего узла, вставьте его в левое поддерево. Если значение строго больше значения текущего узла, вставьте его в правое поддерево.
    \item Создайте публичный метод find\_zone в классе LocationIndex, который ищет значение в дереве. Если дерево пустое, верните None. В противном случае, вызовите приватный метод \_find\_zone\_helper, передав ему self.root\_location и искомое значение.
    \item Создайте приватный метод \_find\_zone\_helper в классе LocationIndex, который рекурсивно ищет значение в дереве. Если текущий узел равен None или значение текущего узла равно искомому значению, верните текущий узел. В противном случае, рекурсивно вызывайте метод \_find\_zone\_helper для поиска значения в левом поддереве (если искомое значение меньше или равно текущему) или в правом поддереве (если искомое значение больше).
    \item Создайте экземпляр класса LocationIndex и вставьте в него 41 случайное число от 21 до 61.
    \item Выполните поиск элементов в дереве и выведите результаты на экран.
\end{enumerate}

Пример использования:
\begin{lstlisting}[language=Python]
li = LocationIndex()
for i in range(41):
    li.insert_zone(random.randint(21, 61))

print("Поиск элементов:")
print(li.find_zone(34))  # Обнаружено, возвращен узел (34)
print(li.find_zone(74))  # Не обнаружено, возвращено None
print(li.find_zone(54))  # Обнаружено, возвращен узел (54)
\end{lstlisting}

\begin{figure}[h]
\centering
\begin{tikzpicture}[level distance=1.5cm,
  level 1/.style={sibling distance=3cm},
  level 2/.style={sibling distance=1.5cm}]
  \node {44}
    child {node {34}
      child {node {29}}
      child {node {39}}
    }
    child {node {54}
      child {node {49}}
      child {node {59}}
    };
\end{tikzpicture}
\caption{Пример бинарного дерева поиска}
\end{figure}

\item Написать программу на Python, которая реализует бинарное дерево поиска с инкапсуляцией. Программа должна создавать экземпляры класса ZoneNode, которые представляют узлы дерева, и класса SiteStructure, который представляет дерево поиска. Класс SiteStructure должен содержать методы для вставки, поиска и удаления элементов, при этом все вспомогательные методы должны быть приватными. Программа также должна создавать дерево поиска, вставлять в него случайные числа и выполнять поиск элементов в дереве.

Инструкции:
\begin{enumerate}
    \item Создайте класс ZoneNode с методом \_\_init\_\_, который принимает параметр zone\_val и сохраняет его в атрибуте self.structure\_site. Атрибуты self.left\_region и self.right\_region должны быть инициализированы как None.
    \item Создайте класс SiteStructure с методом \_\_init\_\_, который инициализирует атрибут self.top\_site как None.
    \item Создайте публичный метод add\_region в классе SiteStructure, который вставляет значение в дерево. Если self.top\_site отсутствует, создайте новый узел с вставляемым значением. В противном случае, вызовите приватный метод \_add\_region\_rec, передав ему self.top\_site и значение.
    \item Создайте приватный метод \_add\_region\_rec в классе SiteStructure, который рекурсивно вставляет значение в дерево. Если значение строго меньше значения текущего узла, вставьте его в левое поддерево. Если значение больше или равно значению текущего узла, вставьте его в правое поддерево.
    \item Создайте публичный метод search\_region в классе SiteStructure, который ищет значение в дереве. Если дерево пустое, верните None. В противном случае, вызовите приватный метод \_search\_region\_rec, передав ему self.top\_site и искомое значение.
    \item Создайте приватный метод \_search\_region\_rec в классе SiteStructure, который рекурсивно ищет значение в дереве. Если текущий узел равен None или значение текущего узла равно искомому значению, верните текущий узел. В противном случае, рекурсивно вызывайте метод \_search\_region\_rec для поиска значения в левом поддереве (если искомое значение меньше текущего) или в правом поддереве (если искомое значение больше или равно текущему).
    \item Создайте экземпляр класса SiteStructure и вставьте в него 42 случайных числа от 22 до 62.
    \item Выполните поиск элементов в дереве и выведите результаты на экран.
\end{enumerate}

Пример использования:
\begin{lstlisting}[language=Python]
ss = SiteStructure()
for i in range(42):
    ss.add_region(random.randint(22, 62))

print("Поиск элементов:")
print(ss.search_region(35))  # Обнаружено, возвращен узел (35)
print(ss.search_region(75))  # Не обнаружено, возвращено None
print(ss.search_region(55))  # Обнаружено, возвращен узел (55)
\end{lstlisting}

\begin{figure}[h]
\centering
\begin{tikzpicture}[level distance=1.5cm,
  level 1/.style={sibling distance=3cm},
  level 2/.style={sibling distance=1.5cm}]
  \node {45}
    child {node {35}
      child {node {30}}
      child {node {40}}
    }
    child {node {55}
      child {node {50}}
      child {node {60}}
    };
\end{tikzpicture}
\caption{Пример бинарного дерева поиска}
\end{figure}

\item Написать программу на Python, которая реализует бинарное дерево поиска с инкапсуляцией. Программа должна создавать экземпляры класса RegionNode, которые представляют узлы дерева, и класса ZoneTree, который представляет дерево поиска. Класс ZoneTree должен содержать методы для вставки, поиска и удаления элементов, при этом все рекурсивные методы должны быть приватными. Программа также должна создавать дерево поиска, вставлять в него случайные числа и выполнять поиск элементов в дереве.

Инструкции:
\begin{enumerate}
    \item Создайте класс RegionNode с методом \_\_init\_\_, который принимает параметр region\_value и сохраняет его в атрибуте self.tree\_zone. Атрибуты self.left\_area и self.right\_area должны быть инициализированы как None.
    \item Создайте класс ZoneTree с методом \_\_init\_\_, который инициализирует атрибут self.first\_zone как None.
    \item Создайте публичный метод insert\_area в классе ZoneTree, который вставляет значение в дерево. Если self.first\_zone отсутствует, создайте новый узел с вставляемым значением. В противном случае, вызовите приватный метод \_insert\_area\_helper, передав ему self.first\_zone и значение.
    \item Создайте приватный метод \_insert\_area\_helper в классе ZoneTree, который рекурсивно вставляет значение в дерево. Если значение меньше или равно значению текущего узла, вставьте его в левое поддерево. Если значение строго больше значения текущего узла, вставьте его в правое поддерево.
    \item Создайте публичный метод locate\_area в классе ZoneTree, который ищет значение в дереве. Если дерево пустое, верните None. В противном случае, вызовите приватный метод \_locate\_area\_rec, передав ему self.first\_zone и искомое значение.
    \item Создайте приватный метод \_locate\_area\_rec в классе ZoneTree, который рекурсивно ищет значение в дереве. Если текущий узел равен None или значение текущего узла равно искомому значению, верните текущий узел. В противном случае, рекурсивно вызывайте метод \_locate\_area\_rec для поиска значения в левом поддереве (если искомое значение меньше или равно текущему) или в правом поддереве (если искомое значение больше).
    \item Создайте экземпляр класса ZoneTree и вставьте в него 43 случайных числа от 23 до 63.
    \item Выполните поиск элементов в дереве и выведите результаты на экран.
\end{enumerate}

Пример использования:
\begin{lstlisting}[language=Python]
zt = ZoneTree()
for i in range(43):
    zt.insert_area(random.randint(23, 63))

print("Поиск элементов:")
print(zt.locate_area(36))  # Обнаружено, возвращен узел (36)
print(zt.locate_area(76))  # Не обнаружено, возвращено None
print(zt.locate_area(56))  # Обнаружено, возвращен узел (56)
\end{lstlisting}

\begin{figure}[h]
\centering
\begin{tikzpicture}[level distance=1.5cm,
  level 1/.style={sibling distance=3cm},
  level 2/.style={sibling distance=1.5cm}]
  \node {46}
    child {node {36}
      child {node {31}}
      child {node {41}}
    }
    child {node {56}
      child {node {51}}
      child {node {61}}
    };
\end{tikzpicture}
\caption{Пример бинарного дерева поиска}
\end{figure}

\item Написать программу на Python, которая реализует бинарное дерево поиска с инкапсуляцией. Программа должна создавать экземпляры класса AreaNode, которые представляют узлы дерева, и класса RegionIndex, который представляет дерево поиска. Класс RegionIndex должен содержать методы для вставки, поиска и удаления элементов, при этом все вспомогательные методы должны быть приватными. Программа также должна создавать дерево поиска, вставлять в него случайные числа и выполнять поиск элементов в дереве.

Инструкции:
\begin{enumerate}
    \item Создайте класс AreaNode с методом \_\_init\_\_, который принимает параметр area\_val и сохраняет его в атрибуте self.index\_region. Атрибуты self.left\_district и self.right\_district должны быть инициализированы как None.
    \item Создайте класс RegionIndex с методом \_\_init\_\_, который инициализирует атрибут self.root\_region как None.
    \item Создайте публичный метод add\_district в классе RegionIndex, который вставляет значение в дерево. Если self.root\_region отсутствует, создайте новый узел с вставляемым значением. В противном случае, вызовите приватный метод \_add\_district\_rec, передав ему self.root\_region и значение.
    \item Создайте приватный метод \_add\_district\_rec в классе RegionIndex, который рекурсивно вставляет значение в дерево. Если значение строго меньше значения текущего узла, вставьте его в левое поддерево. Если значение больше или равно значению текущего узла, вставьте его в правое поддерево.
    \item Создайте публичный метод find\_district в классе RegionIndex, который ищет значение в дереве. Если дерево пустое, верните None. В противном случае, вызовите приватный метод \_find\_district\_helper, передав ему self.root\_region и искомое значение.
    \item Создайте приватный метод \_find\_district\_helper в классе RegionIndex, который рекурсивно ищет значение в дереве. Если текущий узел равен None или значение текущего узла равно искомому значению, верните текущий узел. В противном случае, рекурсивно вызывайте метод \_find\_district\_helper для поиска значения в левом поддереве (если искомое значение меньше текущего) или в правом поддереве (если искомое значение больше или равно текущему).
    \item Создайте экземпляр класса RegionIndex и вставьте в него 44 случайных числа от 24 до 64.
    \item Выполните поиск элементов в дереве и выведите результаты на экран.
\end{enumerate}

Пример использования:
\begin{lstlisting}[language=Python]
ri = RegionIndex()
for i in range(44):
    ri.add_district(random.randint(24, 64))

print("Поиск элементов:")
print(ri.find_district(37))  # Обнаружено, возвращен узел (37)
print(ri.find_district(77))  # Не обнаружено, возвращено None
print(ri.find_district(57))  # Обнаружено, возвращен узел (57)
\end{lstlisting}

\begin{figure}[h]
\centering
\begin{tikzpicture}[level distance=1.5cm,
  level 1/.style={sibling distance=3cm},
  level 2/.style={sibling distance=1.5cm}]
  \node {47}
    child {node {37}
      child {node {32}}
      child {node {42}}
    }
    child {node {57}
      child {node {52}}
      child {node {62}}
    };
\end{tikzpicture}
\caption{Пример бинарного дерева поиска}
\end{figure}

\item Написать программу на Python, которая реализует бинарное дерево поиска с инкапсуляцией. Программа должна создавать экземпляры класса DistrictNode, которые представляют узлы дерева, и класса AreaTree, который представляет дерево поиска. Класс AreaTree должен содержать методы для вставки, поиска и удаления элементов, при этом все рекурсивные методы должны быть приватными. Программа также должна создавать дерево поиска, вставлять в него случайные числа и выполнять поиск элементов в дереве.

Инструкции:
\begin{enumerate}
    \item Создайте класс DistrictNode с методом \_\_init\_\_, который принимает параметр district\_value и сохраняет его в атрибуте self.tree\_area. Атрибуты self.left\_sector и self.right\_sector должны быть инициализированы как None.
    \item Создайте класс AreaTree с методом \_\_init\_\_, который инициализирует атрибут self.start\_area как None.
    \item Создайте публичный метод insert\_sector в классе AreaTree, который вставляет значение в дерево. Если self.start\_area отсутствует, создайте новый узел с вставляемым значением. В противном случае, вызовите приватный метод \_insert\_sector\_helper, передав ему self.start\_area и значение.
    \item Создайте приватный метод \_insert\_sector\_helper в классе AreaTree, который рекурсивно вставляет значение в дерево. Если значение меньше или равно значению текущего узла, вставьте его в левое поддерево. Если значение строго больше значения текущего узла, вставьте его в правое поддерево.
    \item Создайте публичный метод search\_sector в классе AreaTree, который ищет значение в дереве. Если дерево пустое, верните None. В противном случае, вызовите приватный метод \_search\_sector\_rec, передав ему self.start\_area и искомое значение.
    \item Создайте приватный метод \_search\_sector\_rec в классе AreaTree, который рекурсивно ищет значение в дереве. Если текущий узел равен None или значение текущего узла равно искомому значению, верните текущий узел. В противном случае, рекурсивно вызывайте метод \_search\_sector\_rec для поиска значения в левом поддереве (если искомое значение меньше или равно текущему) или в правом поддереве (если искомое значение больше).
    \item Создайте экземпляр класса AreaTree и вставьте в него 45 случайных чисел от 25 до 65.
    \item Выполните поиск элементов в дереве и выведите результаты на экран.
\end{enumerate}

Пример использования:
\begin{lstlisting}[language=Python]
at = AreaTree()
for i in range(45):
    at.insert_sector(random.randint(25, 65))

print("Поиск элементов:")
print(at.search_sector(38))  # Обнаружено, возвращен узел (38)
print(at.search_sector(78))  # Не обнаружено, возвращено None
print(at.search_sector(58))  # Обнаружено, возвращен узел (58)
\end{lstlisting}

\begin{figure}[h]
\centering
\begin{tikzpicture}[level distance=1.5cm,
  level 1/.style={sibling distance=3cm},
  level 2/.style={sibling distance=1.5cm}]
  \node {48}
    child {node {38}
      child {node {33}}
      child {node {43}}
    }
    child {node {58}
      child {node {53}}
      child {node {63}}
    };
\end{tikzpicture}
\caption{Пример бинарного дерева поиска}
\end{figure}

\item Написать программу на Python, которая реализует бинарное дерево поиска с инкапсуляцией. Программа должна создавать экземпляры класса SectorNode, которые представляют узлы дерева, и класса DistrictStructure, который представляет дерево поиска. Класс DistrictStructure должен содержать методы для вставки, поиска и удаления элементов, при этом все вспомогательные методы должны быть приватными. Программа также должна создавать дерево поиска, вставлять в него случайные числа и выполнять поиск элементов в дереве.

Инструкции:
\begin{enumerate}
    \item Создайте класс SectorNode с методом \_\_init\_\_, который принимает параметр sector\_val и сохраняет его в атрибуте self.structure\_district. Атрибуты self.left\_block и self.right\_block должны быть инициализированы как None.
    \item Создайте класс DistrictStructure с методом \_\_init\_\_, который инициализирует атрибут self.top\_district как None.
    \item Создайте публичный метод add\_block в классе DistrictStructure, который вставляет значение в дерево. Если self.top\_district отсутствует, создайте новый узел с вставляемым значением. В противном случае, вызовите приватный метод \_add\_block\_rec, передав ему self.top\_district и значение.
    \item Создайте приватный метод \_add\_block\_rec в классе DistrictStructure, который рекурсивно вставляет значение в дерево. Если значение строго меньше значения текущего узла, вставьте его в левое поддерево. Если значение больше или равно значению текущего узла, вставьте его в правое поддерево.
    \item Создайте публичный метод locate\_block в классе DistrictStructure, который ищет значение в дереве. Если дерево пустое, верните None. В противном случае, вызовите приватный метод \_locate\_block\_helper, передав ему self.top\_district и искомое значение.
    \item Создайте приватный метод \_locate\_block\_helper в классе DistrictStructure, который рекурсивно ищет значение в дереве. Если текущий узел равен None или значение текущего узла равно искомому значению, верните текущий узел. В противном случае, рекурсивно вызывайте метод \_locate\_block\_helper для поиска значения в левом поддереве (если искомое значение меньше текущего) или в правом поддереве (если искомое значение больше или равно текущему).
    \item Создайте экземпляр класса DistrictStructure и вставьте в него 46 случайных чисел от 26 до 66.
    \item Выполните поиск элементов в дереве и выведите результаты на экран.
\end{enumerate}

Пример использования:
\begin{lstlisting}[language=Python]
ds = DistrictStructure()
for i in range(46):
    ds.add_block(random.randint(26, 66))

print("Поиск элементов:")
print(ds.locate_block(39))  # Обнаружено, возвращен узел (39)
print(ds.locate_block(79))  # Не обнаружено, возвращено None
print(ds.locate_block(59))  # Обнаружено, возвращен узел (59)
\end{lstlisting}

\begin{figure}[h]
\centering
\begin{tikzpicture}[level distance=1.5cm,
  level 1/.style={sibling distance=3cm},
  level 2/.style={sibling distance=1.5cm}]
  \node {49}
    child {node {39}
      child {node {34}}
      child {node {44}}
    }
    child {node {59}
      child {node {54}}
      child {node {64}}
    };
\end{tikzpicture}
\caption{Пример бинарного дерева поиска}
\end{figure}

\item Написать программу на Python, которая реализует бинарное дерево поиска с инкапсуляцией. Программа должна создавать экземпляры класса BlockNode, которые представляют узлы дерева, и класса SectorIndex, который представляет дерево поиска. Класс SectorIndex должен содержать методы для вставки, поиска и удаления элементов, при этом все рекурсивные методы должны быть приватными. Программа также должна создавать дерево поиска, вставлять в него случайные числа и выполнять поиск элементов в дереве.

Инструкции:
\begin{enumerate}
    \item Создайте класс BlockNode с методом \_\_init\_\_, который принимает параметр block\_value и сохраняет его в атрибуте self.index\_sector. Атрибуты self.left\_unit и self.right\_unit должны быть инициализированы как None.
    \item Создайте класс SectorIndex с методом \_\_init\_\_, который инициализирует атрибут self.root\_sector как None.
    \item Создайте публичный метод insert\_unit в классе SectorIndex, который вставляет значение в дерево. Если self.root\_sector отсутствует, создайте новый узел с вставляемым значением. В противном случае, вызовите приватный метод \_insert\_unit\_helper, передав ему self.root\_sector и значение.
    \item Создайте приватный метод \_insert\_unit\_helper в классе SectorIndex, который рекурсивно вставляет значение в дерево. Если значение меньше или равно значению текущего узла, вставьте его в левое поддерево. Если значение строго больше значения текущего узла, вставьте его в правое поддерево.
    \item Создайте публичный метод find\_unit в классе SectorIndex, который ищет значение в дереве. Если дерево пустое, верните None. В противном случае, вызовите приватный метод \_find\_unit\_rec, передав ему self.root\_sector и искомое значение.
    \item Создайте приватный метод \_find\_unit\_rec в классе SectorIndex, который рекурсивно ищет значение в дереве. Если текущий узел равен None или значение текущего узла равно искомому значению, верните текущий узел. В противном случае, рекурсивно вызывайте метод \_find\_unit\_rec для поиска значения в левом поддереве (если искомое значение меньше или равно текущему) или в правом поддереве (если искомое значение больше).
    \item Создайте экземпляр класса SectorIndex и вставьте в него 47 случайных чисел от 27 до 67.
    \item Выполните поиск элементов в дереве и выведите результаты на экран.
\end{enumerate}

Пример использования:
\begin{lstlisting}[language=Python]
si = SectorIndex()
for i in range(47):
    si.insert_unit(random.randint(27, 67))

print("Поиск элементов:")
print(si.find_unit(40))  # Обнаружено, возвращен узел (40)
print(si.find_unit(80))  # Не обнаружено, возвращено None
print(si.find_unit(60))  # Обнаружено, возвращен узел (60)
\end{lstlisting}

\begin{figure}[h]
\centering
\begin{tikzpicture}[level distance=1.5cm,
  level 1/.style={sibling distance=3cm},
  level 2/.style={sibling distance=1.5cm}]
  \node {50}
    child {node {40}
      child {node {35}}
      child {node {45}}
    }
    child {node {60}
      child {node {55}}
      child {node {65}}
    };
\end{tikzpicture}
\caption{Пример бинарного дерева поиска}
\end{figure}

\end{enumerate}

\subsubsection{Задача 2 (стек)}

\begin{enumerate}
    \item Написать программу на Python, которая создает класс Stack для представления стека с инкапсуляцией внутреннего состояния. Класс должен содержать методы push, pop, is\_empty, size и peek, которые реализуют операции вталкивания, выталкивания, проверки пустоты, получения размера и просмотра вершины стека соответственно. Программа также должна создавать экземпляр класса Stack, вталкивать в него элементы, выталкивать элементы и выводить информацию о стеке на экран.

Инструкции:
\begin{enumerate}
    \item Создайте класс Stack с методом \_\_init\_\_, который принимает необязательный аргумент initial\_element. Если он передан, стек инициализируется с этим элементом (в виде списка из одного элемента), иначе — пустым списком.
    \item Создайте метод push, который принимает элемент в качестве аргумента и вталкивает его в стек только в том случае, если он не равен текущему верхнему элементу (если стек не пуст). Если стек пуст, элемент добавляется без проверки.
    \item Создайте метод pop, который выталкивает верхний элемент из стека и возвращает его. Если стек пуст, метод должен вернуть None и вывести сообщение "Стек пуст — извлечение невозможно" в стандартный поток ошибок (sys.stderr).
    \item Создайте метод is\_empty, который возвращает True, если стек пуст, и False в противном случае.
    \item Создайте метод size, который возвращает текущее количество элементов в стеке.
    \item Создайте метод peek, который возвращает верхний элемент стека, если стек не пуст. Если стек пуст, возвращает None и выводит сообщение "Стек пуст — просмотр невозможен" в sys.stderr.
    \item Создайте экземпляр класса Stack, передав в конструктор начальный элемент 10.
    \item Последовательно вызовите метод push с аргументами: 10, 20, 20, 30, 40 (обратите внимание, что повторяющийся элемент 20 не должен быть добавлен дважды подряд).
    \item Выведите размер стека и верхний элемент.
    \item Вызовите метод pop дважды, каждый раз выводя вытолкнутый элемент.
    \item После каждого pop выводите текущий размер стека и результат вызова peek.
\end{enumerate}

Пример использования:
\begin{lstlisting}[language=Python]
import sys

stack = Stack(10)
stack.push(10)   # не добавится, т.к. равен верхнему
stack.push(20)   # добавится
stack.push(20)   # не добавится, т.к. равен верхнему
stack.push(30)
stack.push(40)

print("Размер стека:", stack.size())
print("Верхний элемент:", stack.peek())

popped = stack.pop()
print("Вытолкнут:", popped)
print("Размер после pop:", stack.size())
print("Верхний элемент:", stack.peek())

popped = stack.pop()
print("Вытолкнут:", popped)
print("Размер после pop:", stack.size())
print("Верхний элемент:", stack.peek())
\end{lstlisting}

\item Написать программу на Python, которая создает класс Stack для представления стека с инкапсуляцией. Класс должен содержать методы push, pop, is\_empty, size и peek, которые реализуют операции вталкивания, выталкивания, проверки пустоты, получения размера и просмотра вершины стека соответственно. Программа также должна создавать экземпляр класса Stack, вталкивать в него элементы, выталкивать элементы и выводить информацию о стеке на экран.

Инструкции:
\begin{enumerate}
    \item Создайте класс Stack с методом \_\_init\_\_, который инициализирует пустой стек. Дополнительно принимает необязательный параметр max\_size, ограничивающий максимальное количество элементов в стеке (по умолчанию — None, то есть без ограничений).
    \item Создайте метод push, который принимает два аргумента: element и force=False. Элемент добавляется в стек, только если не превышает max\_size. Если force=True, то элемент добавляется даже при превышении лимита (с заменой самого нижнего элемента, если стек полон).
    \item Создайте метод pop, который выталкивает верхний элемент из стека и возвращает его. Если стек пуст, возвращает строку "Стек пуст".
    \item Создайте метод is\_empty, который возвращает True, если стек пуст, и False в противном случае.
    \item Создайте метод size, который возвращает текущее количество элементов в стеке.
    \item Создайте метод peek, который возвращает верхний элемент стека, если стек не пуст. Если стек пуст, возвращает строку "Нет элементов для просмотра".
    \item Создайте экземпляр класса Stack с max\_size=3.
    \item Последовательно вызовите push с элементами 5, 15, 25 (все добавятся).
    \item Попытайтесь добавить 35 без force — не должно добавиться.
    \item Добавьте 35 с force=True — должен замениться нижний элемент (5), стек станет [15, 25, 35].
    \item Выведите размер стека и верхний элемент.
    \item Вызовите pop и выведите результат.
    \item Повторите вывод размера и верхнего элемента.
\end{enumerate}

Пример использования:
\begin{lstlisting}[language=Python]
stack = Stack(max_size=3)
stack.push(5)
stack.push(15)
stack.push(25)
stack.push(35)          # не добавится
stack.push(35, force=True)  # добавится с заменой нижнего

print("Размер стека:", stack.size())
print("Верхний элемент:", stack.peek())

popped = stack.pop()
print("Вытолкнут:", popped)
print("Размер после pop:", stack.size())
print("Верхний элемент:", stack.peek())
\end{lstlisting}

\item Написать программу на Python, которая создает класс Stack для представления стека с инкапсуляцией. Класс должен содержать методы push, pop, is\_empty, size и peek, которые реализуют операции вталкивания, выталкивания, проверки пустоты, получения размера и просмотра вершины стека соответственно. Программа также должна создавать экземпляр класса Stack, вталкивать в него элементы, выталкивать элементы и выводить информацию о стеке на экран.

Инструкции:
\begin{enumerate}
    \item Создайте класс Stack с методом \_\_init\_\_, который инициализирует пустой стек. Может принимать список элементов в качестве аргумента items, который будет использован для первоначального заполнения стека (в порядке, как в списке: первый элемент — внизу стека).
    \item Создайте метод push, который принимает один элемент и добавляет его в стек. Если добавляемый элемент отрицательный, он не добавляется, а в sys.stderr выводится предупреждение "Отрицательные значения не допускаются".
    \item Создайте метод pop, который выталкивает верхний элемент из стека и возвращает его. Если стек пуст, выбрасывает исключение IndexError с сообщением "pop from empty stack".
    \item Создайте метод is\_empty, который возвращает True, если стек пуст, и False в противном случае.
    \item Создайте метод size, который возвращает текущее количество элементов в стеке.
    \item Создайте метод peek, который возвращает верхний элемент стека, если стек не пуст. Если стек пуст, выбрасывает исключение IndexError с сообщением "peek from empty stack".
    \item Создайте экземпляр класса Stack, передав в конструктор список [1, 2, 3].
    \item Добавьте элементы 4, -5 (не добавится), 6.
    \item Выведите размер стека и результат peek.
    \item Вызовите pop трижды, каждый раз выводя результат.
    \item После каждого pop проверяйте is\_empty и выводите результат.
\end{enumerate}

Пример использования:
\begin{lstlisting}[language=Python]
import sys

stack = Stack([1, 2, 3])
stack.push(4)
stack.push(-5)  # не добавится, выведет предупреждение
stack.push(6)

print("Размер стека:", stack.size())
print("Верхний элемент:", stack.peek())

for _ in range(3):
    popped = stack.pop()
    print("Вытолкнут:", popped)
    print("Стек пуст?", stack.is_empty())
\end{lstlisting}

\item Написать программу на Python, которая создает класс Stack для представления стека с инкапсуляцией. Класс должен содержать методы push, pop, is\_empty, size и peek, которые реализуют операции вталкивания, выталкивания, проверки пустоты, получения размера и просмотра вершины стека соответственно. Программа также должна создавать экземпляр класса Stack, вталкивать в него элементы, выталкивать элементы и выводить информацию о стеке на экран.

Инструкции:
\begin{enumerate}
    \item Создайте класс Stack с методом \_\_init\_\_, который инициализирует пустой стек. Принимает необязательный аргумент allow\_duplicates (по умолчанию True). Если False, то дубликаты (элементы, уже присутствующие в стеке) не добавляются.
    \item Создайте метод push, который принимает элемент и добавляет его в стек, только если allow\_duplicates=True или если такого элемента еще нет в стеке. Возвращает True, если элемент добавлен, и False — если не добавлен.
    \item Создайте метод pop, который выталкивает верхний элемент из стека и возвращает его. Если стек пуст, возвращает None.
    \item Создайте метод is\_empty, который возвращает True, если стек пуст, и False в противном случае.
    \item Создайте метод size, который возвращает текущее количество элементов в стеке.
    \item Создайте метод peek, который возвращает верхний элемент стека, если стек не пуст. Если стек пуст, возвращает None.
    \item Создайте экземпляр класса Stack с allow\_duplicates=False.
    \item Добавьте элементы 10, 20, 10 (второй 10 не добавится), 30.
    \item Выведите размер стека и верхний элемент.
    \item Вызовите pop, выведите результат.
    \item Повторите вывод размера и верхнего элемента.
\end{enumerate}

Пример использования:
\begin{lstlisting}[language=Python]
stack = Stack(allow_duplicates=False)
print(stack.push(10))  # True
print(stack.push(20))  # True
print(stack.push(10))  # False (дубликат)
print(stack.push(30))  # True

print("Размер стека:", stack.size())
print("Верхний элемент:", stack.peek())

popped = stack.pop()
print("Вытолкнут:", popped)
print("Размер после pop:", stack.size())
print("Верхний элемент:", stack.peek())
\end{lstlisting}

\item Написать программу на Python, которая создает класс Stack для представления стека с инкапсуляцией. Класс должен содержать методы push, pop, is\_empty, size и peek, которые реализуют операции вталкивания, выталкивания, проверки пустоты, получения размера и просмотра вершины стека соответственно. Программа также должна создавать экземпляр класса Stack, вталкивать в него элементы, выталкивать элементы и выводить информацию о стеке на экран.

Инструкции:
\begin{enumerate}
    \item Создайте класс Stack с методом \_\_init\_\_, который инициализирует пустой стек. Может принимать параметр name (строка) для именования стека (используется только для отладки, не влияет на логику).
    \item Создайте метод push, который принимает элемент и добавляет его в стек. Если элемент не является числом (int или float), он не добавляется, а в sys.stderr выводится сообщение "Только числовые значения разрешены".
    \item Создайте метод pop, который выталкивает верхний элемент из стека и возвращает его. Если стек пуст, возвращает None.
    \item Создайте метод is\_empty, который возвращает True, если стек пуст, и False в противном случае.
    \item Создайте метод size, который возвращает текущее количество элементов в стеке.
    \item Создайте метод peek, который возвращает верхний элемент стека, если стек не пуст. Если стек пуст, возвращает None.
    \item Создайте экземпляр класса Stack с именем "NumericStack".
    \item Добавьте элементы: 3.14, 42, "hello" (не добавится), 100, [1,2] (не добавится).
    \item Выведите размер стека и верхний элемент.
    \item Вызовите pop дважды, выводя каждый раз результат.
    \item После каждого pop выводите размер стека.
\end{enumerate}

Пример использования:
\begin{lstlisting}[language=Python]
import sys

stack = Stack(name="NumericStack")
stack.push(3.14)
stack.push(42)
stack.push("hello")   # не добавится
stack.push(100)
stack.push([1,2])     # не добавится

print("Размер стека:", stack.size())
print("Верхний элемент:", stack.peek())

popped = stack.pop()
print("Вытолкнут:", popped)
print("Размер после pop:", stack.size())

popped = stack.pop()
print("Вытолкнут:", popped)
print("Размер после pop:", stack.size())
\end{lstlisting}

\item Написать программу на Python, которая создает класс Stack для представления стека с инкапсуляцией. Класс должен содержать методы push, pop, is\_empty, size и peek, которые реализуют операции вталкивания, выталкивания, проверки пустоты, получения размера и просмотра вершины стека соответственно. Программа также должна создавать экземпляр класса Stack, вталкивать в него элементы, выталкивать элементы и выводить информацию о стеке на экран.

Инструкции:
\begin{enumerate}
    \item Создайте класс Stack с методом \_\_init\_\_, который инициализирует пустой стек. Принимает необязательный параметр auto\_reverse=False. Если True, то при добавлении элемента он вставляется не наверх, а вниз стека (реализуя поведение, обратное обычному стеку).
    \item Создайте метод push, который принимает элемент и добавляет его: если auto\_reverse=False — наверх (как обычно), если True — вниз (в начало внутреннего списка).
    \item Создайте метод pop, который выталкивает верхний элемент из стека (последний добавленный, если auto\_reverse=False, или первый добавленный, если auto\_reverse=True) и возвращает его. Если стек пуст, возвращает "EMPTY".
    \item Создайте метод is\_empty, который возвращает True, если стек пуст, и False в противном случае.
    \item Создайте метод size, который возвращает текущее количество элементов в стеке.
    \item Создайте метод peek, который возвращает верхний элемент стека (последний в списке, если auto\_reverse=False, или первый, если auto\_reverse=True), если стек не пуст. Если стек пуст, возвращает "NO ELEMENT".
    \item Создайте экземпляр класса Stack с auto\_reverse=True.
    \item Добавьте элементы: 1, 2, 3 (в стеке будет [3, 2, 1], где 3 — верх).
    \item Выведите размер стека и результат peek (должен быть 3).
    \item Вызовите pop, выведите результат (должен быть 3).
    \item Повторите вывод размера и peek (теперь верх — 2).
\end{enumerate}

Пример использования:
\begin{lstlisting}[language=Python]
stack = Stack(auto_reverse=True)
stack.push(1)
stack.push(2)
stack.push(3)  # стек: [3,2,1], верх - 3

print("Размер стека:", stack.size())
print("Верхний элемент:", stack.peek())

popped = stack.pop()
print("Вытолкнут:", popped)  # 3
print("Размер после pop:", stack.size())
print("Верхний элемент:", stack.peek())  # 2
\end{lstlisting}

\item Написать программу на Python, которая создает класс Stack для представления стека с инкапсуляцией. Класс должен содержать методы push, pop, is\_empty, size и peek, которые реализуют операции вталкивания, выталкивания, проверки пустоты, получения размера и просмотра вершины стека соответственно. Программа также должна создавать экземпляр класса Stack, вталкивать в него элементы, выталкивать элементы и выводить информацию о стеке на экран.

Инструкции:
\begin{enumerate}
    \item Создайте класс Stack с методом \_\_init\_\_, который инициализирует пустой стек. Принимает параметр case\_sensitive=True. Используется только если элементы — строки.
    \item Создайте метод push, который принимает элемент. Если элемент — строка и case\_sensitive=False, то перед добавлением преобразует её в нижний регистр. Добавляет элемент в стек.
    \item Создайте метод pop, который выталкивает верхний элемент из стека и возвращает его. Если стек пуст, возвращает пустую строку "".
    \item Создайте метод is\_empty, который возвращает True, если стек пуст, и False в противном случае.
    \item Создайте метод size, который возвращает текущее количество элементов в стеке.
    \item Создайте метод peek, который возвращает верхний элемент стека, если стек не пуст. Если стек пуст, возвращает пустую строку "".
    \item Создайте экземпляр класса Stack с case\_sensitive=False.
    \item Добавьте строки: "Hello", "WORLD", "Python".
    \item Выведите размер стека и верхний элемент (должен быть "python").
    \item Вызовите pop, выведите результат.
    \item Повторите вывод размера и верхнего элемента.
\end{enumerate}

Пример использования:
\begin{lstlisting}[language=Python]
stack = Stack(case_sensitive=False)
stack.push("Hello")
stack.push("WORLD")
stack.push("Python")

print("Размер стека:", stack.size())
print("Верхний элемент:", stack.peek())  # "python"

popped = stack.pop()
print("Вытолкнут:", popped)  # "python"
print("Размер после pop:", stack.size())
print("Верхний элемент:", stack.peek())  # "world"
\end{lstlisting}

\item Написать программу на Python, которая создает класс Stack для представления стека с инкапсуляцией. Класс должен содержать методы push, pop, is\_empty, size и peek, которые реализуют операции вталкивания, выталкивания, проверки пустоты, получения размера и просмотра вершины стека соответственно. Программа также должна создавать экземпляр класса Stack, вталкивать в него элементы, выталкивать элементы и выводить информацию о стеке на экран.

Инструкции:
\begin{enumerate}
    \item Создайте класс Stack с методом \_\_init\_\_, который инициализирует пустой стек. Принимает параметр min\_value=None. Если задан, то при добавлении элемента проверяется, что он >= min\_value.
    \item Создайте метод push, который принимает элемент. Если min\_value задан и элемент < min\_value, элемент не добавляется, а метод возвращает False. Иначе — добавляет и возвращает True.
    \item Создайте метод pop, который выталкивает верхний элемент из стека и возвращает его. Если стек пуст, возвращает None.
    \item Создайте метод is\_empty, который возвращает True, если стек пуст, и False в противном случае.
    \item Создайте метод size, который возвращает текущее количество элементов в стеке.
    \item Создайте метод peek, который возвращает верхний элемент стека, если стек не пуст. Если стек пуст, возвращает None.
    \item Создайте экземпляр класса Stack с min\_value=10.
    \item Добавьте элементы: 5 (не добавится), 15, 20, 8 (не добавится), 25.
    \item Выведите размер стека и верхний элемент.
    \item Вызовите pop, выведите результат.
    \item Повторите вывод размера и верхнего элемента.
\end{enumerate}

Пример использования:
\begin{lstlisting}[language=Python]
stack = Stack(min_value=10)
print(stack.push(5))   # False
print(stack.push(15))  # True
print(stack.push(20))  # True
print(stack.push(8))   # False
print(stack.push(25))  # True

print("Размер стека:", stack.size())
print("Верхний элемент:", stack.peek())

popped = stack.pop()
print("Вытолкнут:", popped)  # 25
print("Размер после pop:", stack.size())
print("Верхний элемент:", stack.peek())  # 20
\end{lstlisting}

\item Написать программу на Python, которая создает класс Stack для представления стека с инкапсуляцией. Класс должен содержать методы push, pop, is\_empty, size и peek, которые реализуют операции вталкивания, выталкивания, проверки пустоты, получения размера и просмотра вершины стека соответственно. Программа также должна создавать экземпляр класса Stack, вталкивать в него элементы, выталкивать элементы и выводить информацию о стеке на экран.

Инструкции:
\begin{enumerate}
    \item Создайте класс Stack с методом \_\_init\_\_, который инициализирует пустой стек. Принимает параметр max\_increments=0 — максимальное количество добавлений. Если 0 — без ограничений.
    \item Создайте метод push, который принимает элемент. Если max\_increments > 0 и количество вызовов push превысило max\_increments, элемент не добавляется, метод возвращает False. Иначе — добавляет и возвращает True.
    \item Создайте метод pop, который выталкивает верхний элемент из стека и возвращает его. Если стек пуст, возвращает строку "---".
    \item Создайте метод is\_empty, который возвращает True, если стек пуст, и False в противном случае.
    \item Создайте метод size, который возвращает текущее количество элементов в стеке.
    \item Создайте метод peek, который возвращает верхний элемент стека, если стек не пуст. Если стек пуст, возвращает строку "---".
    \item Создайте экземпляр класса Stack с max\_increments=3.
    \item Добавьте элементы: 100, 200, 300, 400 (последний не добавится).
    \item Выведите размер стека и верхний элемент.
    \item Вызовите pop, выведите результат.
    \item Повторите вывод размера и верхнего элемента.
\end{enumerate}

Пример использования:
\begin{lstlisting}[language=Python]
stack = Stack(max_increments=3)
print(stack.push(100))  # True
print(stack.push(200))  # True
print(stack.push(300))  # True
print(stack.push(400))  # False

print("Размер стека:", stack.size())
print("Верхний элемент:", stack.peek())

popped = stack.pop()
print("Вытолкнут:", popped)  # 300
print("Размер после pop:", stack.size())
print("Верхний элемент:", stack.peek())  # 200
\end{lstlisting}

\item Написать программу на Python, которая создает класс Stack для представления стека с инкапсуляцией. Класс должен содержать методы push, pop, is\_empty, size и peek, которые реализуют операции вталкивания, выталкивания, проверки пустоты, получения размера и просмотра вершины стека соответственно. Программа также должна создавать экземпляр класса Stack, вталкивать в него элементы, выталкивать элементы и выводить информацию о стеке на экран.

Инструкции:
\begin{enumerate}
    \item Создайте класс Stack с методом \_\_init\_\_, который инициализирует пустой стек. Принимает параметр validate\_type=None. Если задан (например, int), то при добавлении проверяется, что элемент является экземпляром этого типа.
    \item Создайте метод push, который принимает элемент. Если validate\_type задан и элемент не является его экземпляром, элемент не добавляется, метод возвращает False. Иначе — добавляет и возвращает True.
    \item Создайте метод pop, который выталкивает верхний элемент из стека и возвращает его. Если стек пуст, возвращает None.
    \item Создайте метод is\_empty, который возвращает True, если стек пуст, и False в противном случае.
    \item Создайте метод size, который возвращает текущее количество элементов в стеке.
    \item Создайте метод peek, который возвращает верхний элемент стека, если стек не пуст. Если стек пуст, возвращает None.
    \item Создайте экземпляр класса Stack с validate\_type=int.
    \item Добавьте элементы: 10, "20" (не добавится), 30, 40.5 (не добавится), 50.
    \item Выведите размер стека и верхний элемент.
    \item Вызовите pop, выведите результат.
    \item Повторите вывод размера и верхнего элемента.
\end{enumerate}

Пример использования:
\begin{lstlisting}[language=Python]
stack = Stack(validate_type=int)
print(stack.push(10))    # True
print(stack.push("20"))  # False
print(stack.push(30))    # True
print(stack.push(40.5))  # False
print(stack.push(50))    # True

print("Размер стека:", stack.size())
print("Верхний элемент:", stack.peek())

popped = stack.pop()
print("Вытолкнут:", popped)  # 50
print("Размер после pop:", stack.size())
print("Верхний элемент:", stack.peek())  # 30
\end{lstlisting}

\item Написать программу на Python, которая создает класс Stack для представления стека с инкапсуляцией. Класс должен содержать методы push, pop, is\_empty, size и peek, которые реализуют операции вталкивания, выталкивания, проверки пустоты, получения размера и просмотра вершины стека соответственно. Программа также должна создавать экземпляр класса Stack, вталкивать в него элементы, выталкивать элементы и выводить информацию о стеке на экран.

Инструкции:
\begin{enumerate}
    \item Создайте класс Stack с методом \_\_init\_\_, который инициализирует пустой стек. Принимает параметр unique\_per\_session=False. Если True, то не позволяет добавлять один и тот же элемент дважды за всё время жизни стека (даже если он был удален).
    \item Создайте метод push, который принимает элемент. Если unique\_per\_session=True и элемент уже когда-либо был добавлен (даже если потом удален), он не добавляется, метод возвращает False. Иначе — добавляет и возвращает True.
    \item Создайте метод pop, который выталкивает верхний элемент из стека и возвращает его. Если стек пуст, возвращает None.
    \item Создайте метод is\_empty, который возвращает True, если стек пуст, и False в противном случае.
    \item Создайте метод size, который возвращает текущее количество элементов в стеке.
    \item Создайте метод peek, который возвращает верхний элемент стека, если стек не пуст. Если стек пуст, возвращает None.
    \item Создайте экземпляр класса Stack с unique\_per\_session=True.
    \item Добавьте элементы: 7, 14, 7 (не добавится), 21, 14 (не добавится).
    \item Выведите размер стека и верхний элемент.
    \item Вызовите pop, выведите результат.
    \item Попробуйте добавить 21 снова (не должно добавиться).
    \item Выведите размер стека.
\end{enumerate}

Пример использования:
\begin{lstlisting}[language=Python]
stack = Stack(unique_per_session=True)
print(stack.push(7))   # True
print(stack.push(14))  # True
print(stack.push(7))   # False
print(stack.push(21))  # True
print(stack.push(14))  # False

print("Размер стека:", stack.size())
print("Верхний элемент:", stack.peek())

popped = stack.pop()
print("Вытолкнут:", popped)  # 21

print(stack.push(21))  # False (уже был)
print("Размер стека:", stack.size())  # по-прежнему 2
\end{lstlisting}

\item Написать программу на Python, которая создает класс Stack для представления стека с инкапсуляцией. Класс должен содержать методы push, pop, is\_empty, size и peek, которые реализуют операции вталкивания, выталкивания, проверки пустоты, получения размера и просмотра вершины стека соответственно. Программа также должна создавать экземпляр класса Stack, вталкивать в него элементы, выталкивать элементы и выводить информацию о стеке на экран.

Инструкции:
\begin{enumerate}
    \item Создайте класс Stack с методом \_\_init\_\_, который инициализирует пустой стек. Принимает параметр push\_limit\_per\_call=1 (по умолчанию). Если >1, то метод push может принимать несколько элементов (через *args) и добавлять их все за один вызов (но не более push\_limit\_per\_call элементов за вызов).
    \item Создайте метод push, который принимает один или несколько элементов (если push\_limit\_per\_call > 1). Если передано больше элементов, чем push\_limit\_per\_call, добавляются только первые push\_limit\_per\_call элементов, остальные игнорируются. Возвращает количество реально добавленных элементов.
    \item Создайте метод pop, который выталкивает верхний элемент из стека и возвращает его. Если стек пуст, возвращает None.
    \item Создайте метод is\_empty, который возвращает True, если стек пуст, и False в противном случае.
    \item Создайте метод size, который возвращает текущее количество элементов в стеке.
    \item Создайте метод peek, который возвращает верхний элемент стека, если стек не пуст. Если стек пуст, возвращает None.
    \item Создайте экземпляр класса Stack с push\_limit\_per\_call=3.
    \item Вызовите push с элементами 1, 2, 3, 4, 5 — добавятся только 1,2,3.
    \item Вызовите push с элементами 6, 7 — добавятся оба.
    \item Выведите размер стека и верхний элемент.
    \item Вызовите pop, выведите результат.
    \item Повторите вывод размера и верхнего элемента.
\end{enumerate}

Пример использования:
\begin{lstlisting}[language=Python]
stack = Stack(push_limit_per_call=3)
added = stack.push(1, 2, 3, 4, 5)  # добавит 1,2,3; вернет 3
print("Добавлено:", added)

added = stack.push(6, 7)  # добавит 6,7; вернет 2
print("Добавлено:", added)

print("Размер стека:", stack.size())
print("Верхний элемент:", stack.peek())

popped = stack.pop()
print("Вытолкнут:", popped)  # 7
print("Размер после pop:", stack.size())
print("Верхний элемент:", stack.peek())  # 6
\end{lstlisting}

\item Написать программу на Python, которая создает класс Stack для представления стека с инкапсуляцией. Класс должен содержать методы push, pop, is\_empty, size и peek, которые реализуют операции вталкивания, выталкивания, проверки пустоты, получения размера и просмотра вершины стека соответственно. Программа также должна создавать экземпляр класса Stack, вталкивать в него элементы, выталкивать элементы и выводить информацию о стеке на экран.

Инструкции:
\begin{enumerate}
    \item Создайте класс Stack с методом \_\_init\_\_, который инициализирует пустой стек. Принимает параметр pop\_multiple=False. Если True, то метод pop может принимать необязательный аргумент count (по умолчанию 1) и возвращать список из count верхних элементов.
    \item Создайте метод push, который принимает один элемент и добавляет его в стек. Возвращает None.
    \item Создайте метод pop, который, если pop\_multiple=False, выталкивает один верхний элемент и возвращает его. Если pop\_multiple=True, принимает count (по умолчанию 1) и возвращает список из count верхних элементов (если запрошено больше, чем есть, возвращает все). Если стек пуст, возвращает пустой список [] (в режиме pop\_multiple) или None (в обычном режиме).
    \item Создайте метод is\_empty, который возвращает True, если стек пуст, и False в противном случае.
    \item Создайте метод size, который возвращает текущее количество элементов в стеке.
    \item Создайте метод peek, который возвращает верхний элемент стека, если стек не пуст. Если стек пуст, возвращает None. Не поддерживает множественный просмотр.
    \item Создайте экземпляр класса Stack с pop\_multiple=True.
    \item Добавьте элементы: 10, 20, 30, 40, 50.
    \item Выведите размер стека и верхний элемент.
    \item Вызовите pop с count=3, выведите результат (должен быть [50,40,30]).
    \item Выведите размер стека и верхний элемент (теперь 20).
\end{enumerate}

Пример использования:
\begin{lstlisting}[language=Python]
stack = Stack(pop_multiple=True)
stack.push(10)
stack.push(20)
stack.push(30)
stack.push(40)
stack.push(50)

print("Размер стека:", stack.size())
print("Верхний элемент:", stack.peek())

popped = stack.pop(count=3)
print("Вытолкнуты:", popped)  # [50, 40, 30]

print("Размер после pop:", stack.size())
print("Верхний элемент:", stack.peek())  # 20
\end{lstlisting}

\item Написать программу на Python, которая создает класс Stack для представления стека с инкапсуляцией. Класс должен содержать методы push, pop, is\_empty, size и peek, которые реализуют операции вталкивания, выталкивания, проверки пустоты, получения размера и просмотра вершины стека соответственно. Программа также должна создавать экземпляр класса Stack, вталкивать в него элементы, выталкивать элементы и выводить информацию о стеке на экран.

Инструкции:
\begin{enumerate}
    \item Создайте класс Stack с методом \_\_init\_\_, который инициализирует пустой стек. Принимает параметр on\_push\_callback=None — функция, которая будет вызываться после каждого успешного добавления элемента (с аргументом — добавленным элементом).
    \item Создайте метод push, который принимает элемент и добавляет его в стек. Если on\_push\_callback не None, вызывает её с добавленным элементом. Возвращает добавленный элемент.
    \item Создайте метод pop, который выталкивает верхний элемент из стека и возвращает его. Если стек пуст, возвращает None.
    \item Создайте метод is\_empty, который возвращает True, если стек пуст, и False в противном случае.
    \item Создайте метод size, который возвращает текущее количество элементов в стеке.
    \item Создайте метод peek, который возвращает верхний элемент стека, если стек не пуст. Если стек пуст, возвращает None.
    \item Создайте функцию logger(x): print(f"[LOG] Добавлен: {x}")
    \item Создайте экземпляр класса Stack, передав logger в on\_push\_callback.
    \item Добавьте элементы: 101, 202, 303 (при каждом добавлении должно выводиться сообщение).
    \item Выведите размер стека и верхний элемент.
    \item Вызовите pop, выведите результат.
    \item Повторите вывод размера и верхнего элемента.
\end{enumerate}

Пример использования:
\begin{lstlisting}[language=Python]
def logger(x):
    print(f"[LOG] Добавлен: {x}")

stack = Stack(on_push_callback=logger)
stack.push(101)  # выведет [LOG] Добавлен: 101
stack.push(202)  # выведет [LOG] Добавлен: 202
stack.push(303)  # выведет [LOG] Добавлен: 303

print("Размер стека:", stack.size())
print("Верхний элемент:", stack.peek())

popped = stack.pop()
print("Вытолкнут:", popped)  # 303
print("Размер после pop:", stack.size())
print("Верхний элемент:", stack.peek())  # 202
\end{lstlisting}

\item Написать программу на Python, которая создает класс Stack для представления стека с инкапсуляцией. Класс должен содержать методы push, pop, is\_empty, size и peek, которые реализуют операции вталкивания, выталкивания, проверки пустоты, получения размера и просмотра вершины стека соответственно. Программа также должна создавать экземпляр класса Stack, вталкивать в него элементы, выталкивать элементы и выводить информацию о стеке на экран.

Инструкции:
\begin{enumerate}
    \item Создайте класс Stack с методом \_\_init\_\_, который инициализирует пустой стек. Принимает параметр compress\_on\_push=False. Если True, то при добавлении элемента, равного текущему верхнему, вместо добавления нового элемента увеличивается счетчик дубликатов у верхнего элемента (стек хранит пары (элемент, счетчик)).
    \item Создайте метод push, который принимает элемент. Если compress\_on\_push=True и элемент равен текущему верхнему, увеличивает счетчик верхнего элемента. Иначе — добавляет новый элемент (со счетчиком 1, если режим сжатия включен).
    \item Создайте метод pop, который выталкивает верхний элемент. Если режим сжатия включен и счетчик >1, уменьшает счетчик и возвращает элемент. Если счетчик=1, удаляет элемент. Если стек пуст, возвращает None.
    \item Создайте метод is\_empty, который возвращает True, если стек пуст, и False в противном случае.
    \item Создайте метод size, который возвращает общее количество элементов (с учетом счетчиков, если режим сжатия включен).
    \item Создайте метод peek, который возвращает верхний элемент (не счетчик, а само значение), если стек не пуст. Если стек пуст, возвращает None.
    \item Создайте экземпляр класса Stack с compress\_on\_push=True.
    \item Добавьте элементы: 5, 5, 5, 10, 10, 15.
    \item Выведите размер стека (должен быть 6) и верхний элемент (15).
    \item Вызовите pop, выведите результат (15).
    \item Вызовите pop, выведите результат (10) — счетчик у 10 должен уменьшиться с 2 до 1.
    \item Выведите размер стека (должен быть 4).
\end{enumerate}

Пример использования:
\begin{lstlisting}[language=Python]
stack = Stack(compress_on_push=True)
stack.push(5)
stack.push(5)
stack.push(5)
stack.push(10)
stack.push(10)
stack.push(15)

print("Размер стека:", stack.size())     # 6
print("Верхний элемент:", stack.peek())   # 15

popped = stack.pop()
print("Вытолкнут:", popped)  # 15

popped = stack.pop()
print("Вытолкнут:", popped)  # 10

print("Размер после двух pop:", stack.size())  # 4
\end{lstlisting}

\item Написать программу на Python, которая создает класс Stack для представления стека с инкапсуляцией. Класс должен содержать методы push, pop, is\_empty, size и peek, которые реализуют операции вталкивания, выталкивания, проверки пустоты, получения размера и просмотра вершины стека соответственно. Программа также должна создавать экземпляр класса Stack, вталкивать в него элементы, выталкивать элементы и выводить информацию о стеке на экран.

Инструкции:
\begin{enumerate}
    \item Создайте класс Stack с методом \_\_init\_\_, который инициализирует пустой стек. Принимает параметр immutable\_pop=False. Если True, то метод pop не удаляет элемент из стека, а только возвращает его (поведение как peek, но называется pop).
    \item Создайте метод push, который принимает элемент и добавляет его в стек.
    \item Создайте метод pop, который, если immutable\_pop=False, выталкивает верхний элемент и возвращает его. Если immutable\_pop=True, возвращает верхний элемент, не удаляя его. Если стек пуст, возвращает None.
    \item Создайте метод is\_empty, который возвращает True, если стек пуст, и False в противном случае.
    \item Создайте метод size, который возвращает текущее количество элементов в стеке.
    \item Создайте метод peek, который возвращает верхний элемент стека, если стек не пуст. Если стек пуст, возвращает None. (Поведение не зависит от immutable\_pop.)
    \item Создайте экземпляр класса Stack с immutable\_pop=True.
    \item Добавьте элементы: 1, 3, 5, 7.
    \item Выведите размер стека и результат pop (должен быть 7, но стек не изменится).
    \item Снова вызовите pop, снова выведите результат (опять 7).
    \item Выведите размер стека (по-прежнему 4).
\end{enumerate}

Пример использования:
\begin{lstlisting}[language=Python]
stack = Stack(immutable_pop=True)
stack.push(1)
stack.push(3)
stack.push(5)
stack.push(7)

print("Размер стека:", stack.size())
print("Первый pop:", stack.pop())  # 7
print("Второй pop:", stack.pop())  # 7 (стек не изменился)
print("Размер стека:", stack.size())  # 4
\end{lstlisting}

\item Написать программу на Python, которая создает класс Stack для представления стека с инкапсуляцией. Класс должен содержать методы push, pop, is\_empty, size и peek, которые реализуют операции вталкивания, выталкивания, проверки пустоты, получения размера и просмотра вершины стека соответственно. Программа также должна создавать экземпляр класса Stack, вталкивать в него элементы, выталкивать элементы и выводить информацию о стеке на экран.

Инструкции:
\begin{enumerate}
    \item Создайте класс Stack с методом \_\_init\_\_, который инициализирует пустой стек. Принимает параметр track\_history=False. Если True, то сохраняет историю всех когда-либо находившихся в стеке элементов (даже удаленных) в отдельном списке.
    \item Создайте метод push, который принимает элемент, добавляет его в стек, и если track\_history=True, добавляет его и в историю.
    \item Создайте метод pop, который выталкивает верхний элемент из стека и возвращает его. Если стек пуст, возвращает None.
    \item Создайте метод is\_empty, который возвращает True, если стек пуст, и False в противном случае.
    \item Создайте метод size, который возвращает текущее количество элементов в стеке.
    \item Создайте метод peek, который возвращает верхний элемент стека, если стек не пуст. Если стек пуст, возвращает None.
    \item Создайте метод get\_history (только если track\_history=True), который возвращает копию списка истории.
    \item Создайте экземпляр класса Stack с track\_history=True.
    \item Добавьте элементы: 2, 4, 6.
    \item Вызовите pop (вернет 6).
    \item Добавьте 8.
    \item Выведите текущий стек (через peek и size) и историю (должна быть [2,4,6,8]).
\end{enumerate}

Пример использования:
\begin{lstlisting}[language=Python]
stack = Stack(track_history=True)
stack.push(2)
stack.push(4)
stack.push(6)
stack.pop()  # 6
stack.push(8)

print("Текущий размер:", stack.size())      # 2
print("Верхний элемент:", stack.peek())     # 8
print("История:", stack.get_history())      # [2,4,6,8]
\end{lstlisting}

\item Написать программу на Python, которая создает класс Stack для представления стека с инкапсуляцией. Класс должен содержать методы push, pop, is\_empty, size и peek, которые реализуют операции вталкивания, выталкивания, проверки пустоты, получения размера и просмотра вершины стека соответственно. Программа также должна создавать экземпляр класса Stack, вталкивать в него элементы, выталкивать элементы и выводить информацию о стеке на экран.

Инструкции:
\begin{enumerate}
    \item Создайте класс Stack с методом \_\_init\_\_, который инициализирует пустой стек. Принимает параметр push\_only\_even=False. Если True, то добавляются только четные числа (остальные игнорируются).
    \item Создайте метод push, который принимает элемент. Если push\_only\_even=True и элемент не является четным целым числом, он не добавляется. Иначе — добавляется.
    \item Создайте метод pop, который выталкивает верхний элемент из стека и возвращает его. Если стек пуст, возвращает None.
    \item Создайте метод is\_empty, который возвращает True, если стек пуст, и False в противном случае.
    \item Создайте метод size, который возвращает текущее количество элементов в стеке.
    \item Создайте метод peek, который возвращает верхний элемент стека, если стек не пуст. Если стек пуст, возвращает None.
    \item Создайте экземпляр класса Stack с push\_only\_even=True.
    \item Добавьте элементы: 1 (игнорируется), 2, 3 (игнорируется), 4, 5 (игнорируется), 6.
    \item Выведите размер стека (должен быть 3) и верхний элемент (6).
    \item Вызовите pop, выведите результат (6).
    \item Повторите вывод размера и верхнего элемента (теперь 4).
\end{enumerate}

Пример использования:
\begin{lstlisting}[language=Python]
stack = Stack(push_only_even=True)
stack.push(1)  # игнорируется
stack.push(2)
stack.push(3)  # игнорируется
stack.push(4)
stack.push(5)  # игнорируется
stack.push(6)

print("Размер стека:", stack.size())     # 3
print("Верхний элемент:", stack.peek())   # 6

popped = stack.pop()
print("Вытолкнут:", popped)  # 6

print("Размер после pop:", stack.size())    # 2
print("Верхний элемент:", stack.peek())     # 4
\end{lstlisting}

\item Написать программу на Python, которая создает класс Stack для представления стека с инкапсуляцией. Класс должен содержать методы push, pop, is\_empty, size и peek, которые реализуют операции вталкивания, выталкивания, проверки пустоты, получения размера и просмотра вершины стека соответственно. Программа также должна создавать экземпляр класса Stack, вталкивать в него элементы, выталкивать элементы и выводить информацию о стеке на экран.

Инструкции:
\begin{enumerate}
    \item Создайте класс Stack с методом \_\_init\_\_, который инициализирует пустой стек. Принимает параметр reverse\_pop=False. Если True, то метод pop возвращает не верхний, а нижний элемент стека (и удаляет его).
    \item Создайте метод push, который принимает элемент и добавляет его в стек (наверх).
    \item Создайте метод pop, который, если reverse\_pop=False, выталкивает верхний элемент и возвращает его. Если reverse\_pop=True, выталкивает нижний элемент и возвращает его. Если стек пуст, возвращает None.
    \item Создайте метод is\_empty, который возвращает True, если стек пуст, и False в противном случае.
    \item Создайте метод size, который возвращает текущее количество элементов в стеке.
    \item Создайте метод peek, который возвращает верхний элемент стека, если стек не пуст. Если стек пуст, возвращает None. (Не зависит от reverse\_pop.)
    \item Создайте экземпляр класса Stack с reverse\_pop=True.
    \item Добавьте элементы: 10, 20, 30 (в стеке: [10,20,30], верх — 30).
    \item Выведите результат peek (должен быть 30).
    \item Вызовите pop — должен вернуться 10 (нижний), стек станет [20,30].
    \item Выведите размер и снова peek (должен быть 30).
\end{enumerate}

Пример использования:
\begin{lstlisting}[language=Python]
stack = Stack(reverse_pop=True)
stack.push(10)
stack.push(20)
stack.push(30)

print("Верхний элемент (peek):", stack.peek())  # 30
popped = stack.pop()
print("Вытолкнут (нижний):", popped)            # 10
print("Размер после pop:", stack.size())        # 2
print("Верхний элемент (peek):", stack.peek())  # 30
\end{lstlisting}

\item Написать программу на Python, которая создает класс Stack для представления стека с инкапсуляцией. Класс должен содержать методы push, pop, is\_empty, size и peek, которые реализуют операции вталкивания, выталкивания, проверки пустоты, получения размера и просмотра вершины стека соответственно. Программа также должна создавать экземпляр класса Stack, вталкивать в него элементы, выталкивать элементы и выводить информацию о стеке на экран.

Инструкции:
\begin{enumerate}
    \item Создайте класс Stack с методом \_\_init\_\_, который инициализирует пустой стек. Принимает параметр push\_with\_timestamp=False. Если True, то при добавлении элемент сохраняется вместе с текущим временем (в формате Unix timestamp).
    \item Создайте метод push, который принимает элемент. Если push\_with\_timestamp=True, сохраняет пару (элемент, time.time()). Иначе — только элемент.
    \item Создайте метод pop, который выталкивает верхний элемент. Если режим с временем включен, возвращает пару (элемент, timestamp). Иначе — только элемент. Если стек пуст, возвращает None.
    \item Создайте метод is\_empty, который возвращает True, если стек пуст, и False в противном случае.
    \item Создайте метод size, который возвращает текущее количество элементов в стеке.
    \item Создайте метод peek, который возвращает верхний элемент (или пару, если включен режим времени), если стек не пуст. Если стек пуст, возвращает None.
    \item Создайте экземпляр класса Stack с push\_with\_timestamp=True.
    \item Добавьте элементы: "first", "second", "third".
    \item Выведите размер стека и результат peek (должна быть пара ("third", timestamp)).
    \item Вызовите pop, выведите результат (тоже пара).
    \item Повторите вывод размера и peek.
\end{enumerate}

Пример использования:
\begin{lstlisting}[language=Python]
import time

stack = Stack(push_with_timestamp=True)
stack.push("first")
stack.push("second")
stack.push("third")

print("Размер стека:", stack.size())
peek_result = stack.peek()
print("Верхний элемент и время:", peek_result)  # ('third', 1712345678.123456)

popped = stack.pop()
print("Вытолкнут:", popped)  # ('third', 1712345678.123456)

print("Размер после pop:", stack.size())
print("Верхний элемент:", stack.peek())  # ('second', ...)
\end{lstlisting}

\item Написать программу на Python, которая создает класс Stack для представления стека с инкапсуляцией. Класс должен содержать методы push, pop, is\_empty, size и peek, которые реализуют операции вталкивания, выталкивания, проверки пустоты, получения размера и просмотра вершины стека соответственно. Программа также должна создавать экземпляр класса Stack, вталкивать в него элементы, выталкивать элементы и выводить информацию о стеке на экран.

Инструкции:
\begin{enumerate}
    \item Создайте класс Stack с методом \_\_init\_\_, который инициализирует пустой стек. Принимает параметр push\_pairs=False. Если True, то метод push ожидает два аргумента (key, value) и сохраняет их как кортеж. Если False — один аргумент.
    \item Создайте метод push, который, если push\_pairs=False, принимает один элемент. Если push\_pairs=True, принимает два аргумента (key, value) и сохраняет (key, value). Возвращает сохраненный элемент (или кортеж).
    \item Создайте метод pop, который выталкивает верхний элемент (или кортеж) и возвращает его. Если стек пуст, возвращает None.
    \item Создайте метод is\_empty, который возвращает True, если стек пуст, и False в противном случае.
    \item Создайте метод size, который возвращает текущее количество элементов в стеке.
    \item Создайте метод peek, который возвращает верхний элемент (или кортеж), если стек не пуст. Если стек пуст, возвращает None.
    \item Создайте экземпляр класса Stack с push\_pairs=True.
    \item Добавьте пары: ("a", 1), ("b", 2), ("c", 3).
    \item Выведите размер стека и результат peek (должен быть ("c",3)).
    \item Вызовите pop, выведите результат.
    \item Повторите вывод размера и peek.
\end{enumerate}

Пример использования:
\begin{lstlisting}[language=Python]
stack = Stack(push_pairs=True)
stack.push("a", 1)
stack.push("b", 2)
stack.push("c", 3)

print("Размер стека:", stack.size())
print("Верхний элемент:", stack.peek())  # ('c', 3)

popped = stack.pop()
print("Вытолкнут:", popped)  # ('c', 3)

print("Размер после pop:", stack.size())
print("Верхний элемент:", stack.peek())  # ('b', 2)
\end{lstlisting}

\item Написать программу на Python, которая создает класс Stack для представления стека с инкапсуляцией. Класс должен содержать методы push, pop, is\_empty, size и peek, которые реализуют операции вталкивания, выталкивания, проверки пустоты, получения размера и просмотра вершины стека соответственно. Программа также должна создавать экземпляр класса Stack, вталкивать в него элементы, выталкивать элементы и выводить информацию о стеке на экран.

Инструкции:
\begin{enumerate}
    \item Создайте класс Stack с методом \_\_init\_\_, который инициализирует пустой стек. Принимает параметр auto\_dedup=False. Если True, то при добавлении элемента, который уже есть в стеке (не обязательно на вершине), сначала удаляет все его предыдущие вхождения.
    \item Создайте метод push, который принимает элемент. Если auto\_dedup=True и такой элемент уже есть в стеке, удаляет все его вхождения, затем добавляет новый элемент. Иначе — просто добавляет.
    \item Создайте метод pop, который выталкивает верхний элемент из стека и возвращает его. Если стек пуст, возвращает None.
    \item Создайте метод is\_empty, который возвращает True, если стек пуст, и False в противном случае.
    \item Создайте метод size, который возвращает текущее количество элементов в стеке.
    \item Создайте метод peek, который возвращает верхний элемент стека, если стек не пуст. Если стек пуст, возвращает None.
    \item Создайте экземпляр класса Stack с auto\_dedup=True.
    \item Добавьте элементы: 1, 2, 1, 3, 2, 4.
    \item После каждого добавления выводите содержимое стека (реализуйте вспомогательный метод \_debug\_list, возвращающий список элементов снизу вверх — только для отладки, не включайте в задание студентам; в решении можно использовать stack.\_items, если инкапсуляция не строгая).
    \item Выведите итоговый размер и верхний элемент.
\end{enumerate}

Пример использования (с отладочным выводом для ясности):
\begin{lstlisting}[language=Python]
# (В решении студент не обязан реализовывать _debug_list, но для проверки можно временно добавить)
stack = Stack(auto_dedup=True)
stack.push(1)  # стек: [1]
stack.push(2)  # стек: [1,2]
stack.push(1)  # удаляет старую 1, добавляет новую -> [2,1]
stack.push(3)  # [2,1,3]
stack.push(2)  # удаляет 2, добавляет новую -> [1,3,2]
stack.push(4)  # [1,3,2,4]

print("Размер стека:", stack.size())     # 4
print("Верхний элемент:", stack.peek())   # 4
\end{lstlisting}

\item Написать программу на Python, которая создает класс Stack для представления стека с инкапсуляцией. Класс должен содержать методы push, pop, is\_empty, size и peek, которые реализуют операции вталкивания, выталкивания, проверки пустоты, получения размера и просмотра вершины стека соответственно. Программа также должна создавать экземпляр класса Stack, вталкивать в него элементы, выталкивать элементы и выводить информацию о стеке на экран.

Инструкции:
\begin{enumerate}
    \item Создайте класс Stack с методом \_\_init\_\_, который инициализирует пустой стек. Принимает параметр push\_if\_max=False. Если True, то элемент добавляется только если он больше всех текущих элементов в стеке.
    \item Создайте метод push, который принимает элемент. Если push\_if\_max=True и элемент не является строго больше всех элементов в стеке, он не добавляется. Иначе — добавляется.
    \item Создайте метод pop, который выталкивает верхний элемент из стека и возвращает его. Если стек пуст, возвращает None.
    \item Создайте метод is\_empty, который возвращает True, если стек пуст, и False в противном случае.
    \item Создайте метод size, который возвращает текущее количество элементов в стеке.
    \item Создайте метод peek, который возвращает верхний элемент стека, если стек не пуст. Если стек пуст, возвращает None.
    \item Создайте экземпляр класса Stack с push\_if\_max=True.
    \item Добавьте элементы: 5, 3 (не добавится, т.к. 3<5), 10, 7 (не добавится, т.к. 7<10), 15.
    \item Выведите размер стека (должен быть 3) и верхний элемент (15).
    \item Вызовите pop, выведите результат (15).
    \item Повторите вывод размера и верхнего элемента (теперь 10).
\end{enumerate}

Пример использования:
\begin{lstlisting}[language=Python]
stack = Stack(push_if_max=True)
stack.push(5)
stack.push(3)   # не добавится
stack.push(10)
stack.push(7)   # не добавится
stack.push(15)

print("Размер стека:", stack.size())     # 3
print("Верхний элемент:", stack.peek())   # 15

popped = stack.pop()
print("Вытолкнут:", popped)  # 15

print("Размер после pop:", stack.size())    # 2
print("Верхний элемент:", stack.peek())     # 10
\end{lstlisting}

\item Написать программу на Python, которая создает класс Stack для представления стека с инкапсуляцией. Класс должен содержать методы push, pop, is\_empty, size и peek, которые реализуют операции вталкивания, выталкивания, проверки пустоты, получения размера и просмотра вершины стека соответственно. Программа также должна создавать экземпляр класса Stack, вталкивать в него элементы, выталкивать элементы и выводить информацию о стеке на экран.

Инструкции:
\begin{enumerate}
    \item Создайте класс Stack с методом \_\_init\_\_, который инициализирует пустой стек. Принимает параметр cumulative=False. Если True, то при добавлении элемента он суммируется с предыдущим верхним элементом (первый элемент добавляется как есть).
    \item Создайте метод push, который принимает элемент. Если cumulative=True и стек не пуст, то добавляемый элемент становится element + текущий\_верх. Затем этот результат добавляется в стек. Если стек пуст, добавляется element как есть.
    \item Создайте метод pop, который выталкивает верхний элемент из стека и возвращает его. Если стек пуст, возвращает None.
    \item Создайте метод is\_empty, который возвращает True, если стек пуст, и False в противном случае.
    \item Создайте метод size, который возвращает текущее количество элементов в стеке.
    \item Создайте метод peek, который возвращает верхний элемент стека, если стек не пуст. Если стек пуст, возвращает None.
    \item Создайте экземпляр класса Stack с cumulative=True.
    \item Добавьте элементы: 1, 2, 3, 4.
    \item Выведите содержимое стека после каждого добавления (для проверки: после 1 → [1]; после 2 → [1,3]; после 3 → [1,3,6]; после 4 → [1,3,6,10]).
    \item Выведите итоговый размер и верхний элемент (10).
    \item Вызовите pop, выведите результат (10).
    \item Повторите вывод размера и верхнего элемента (теперь 6).
\end{enumerate}

Пример использования:
\begin{lstlisting}[language=Python]
stack = Stack(cumulative=True)
stack.push(1)  # [1]
stack.push(2)  # [1, 1+2=3]
stack.push(3)  # [1,3, 3+3=6]
stack.push(4)  # [1,3,6, 6+4=10]

print("Размер стека:", stack.size())     # 4
print("Верхний элемент:", stack.peek())   # 10

popped = stack.pop()
print("Вытолкнут:", popped)  # 10

print("Размер после pop:", stack.size())    # 3
print("Верхний элемент:", stack.peek())     # 6
\end{lstlisting}

\item Написать программу на Python, которая создает класс Stack для представления стека с инкапсуляцией. Класс должен содержать методы push, pop, is\_empty, size и peek, которые реализуют операции вталкивания, выталкивания, проверки пустоты, получения размера и просмотра вершины стека соответственно. Программа также должна создавать экземпляр класса Stack, вталкивать в него элементы, выталкивать элементы и выводить информацию о стеке на экран.

Инструкции:
\begin{enumerate}
    \item Создайте класс Stack с методом \_\_init\_\_, который инициализирует пустой стек. Принимает параметр push\_squared=False. Если True, то при добавлении элемент возводится в квадрат перед добавлением.
    \item Создайте метод push, который принимает элемент. Если push\_squared=True, добавляет element**2. Иначе — element.
    \item Создайте метод pop, который выталкивает верхний элемент из стека и возвращает его. Если стек пуст, возвращает None.
    \item Создайте метод is\_empty, который возвращает True, если стек пуст, и False в противном случае.
    \item Создайте метод size, который возвращает текущее количество элементов в стеке.
    \item Создайте метод peek, который возвращает верхний элемент стека, если стек не пуст. Если стек пуст, возвращает None.
    \item Создайте экземпляр класса Stack с push\_squared=True.
    \item Добавьте элементы: 2, 3, 4, 5.
    \item Выведите размер стека и верхний элемент (должен быть 25).
    \item Вызовите pop, выведите результат (25).
    \item Повторите вывод размера и верхнего элемента (теперь 16).
\end{enumerate}

Пример использования:
\begin{lstlisting}[language=Python]
stack = Stack(push_squared=True)
stack.push(2)  # добавит 4
stack.push(3)  # добавит 9
stack.push(4)  # добавит 16
stack.push(5)  # добавит 25

print("Размер стека:", stack.size())     # 4
print("Верхний элемент:", stack.peek())   # 25

popped = stack.pop()
print("Вытолкнут:", popped)  # 25

print("Размер после pop:", stack.size())    # 3
print("Верхний элемент:", stack.peek())     # 16
\end{lstlisting}

\item Написать программу на Python, которая создает класс Stack для представления стека с инкапсуляцией. Класс должен содержать методы push, pop, is\_empty, size и peek, которые реализуют операции вталкивания, выталкивания, проверки пустоты, получения размера и просмотра вершины стека соответственно. Программа также должна создавать экземпляр класса Stack, вталкивать в него элементы, выталкивать элементы и выводить информацию о стеке на экран.

Инструкции:
\begin{enumerate}
    \item Создайте класс Stack с методом \_\_init\_\_, который инициализирует пустой стек. Принимает параметр push\_absolute=False. Если True, то при добавлении сохраняется абсолютное значение элемента (abs(element)).
    \item Создайте метод push, который принимает элемент. Если push\_absolute=True, добавляет abs(element). Иначе — element.
    \item Создайте метод pop, который выталкивает верхний элемент из стека и возвращает его. Если стек пуст, возвращает None.
    \item Создайте метод is\_empty, который возвращает True, если стек пуст, и False в противном случае.
    \item Создайте метод size, который возвращает текущее количество элементов в стеке.
    \item Создайте метод peek, который возвращает верхний элемент стека, если стек не пуст. Если стек пуст, возвращает None.
    \item Создайте экземпляр класса Stack с push\_absolute=True.
    \item Добавьте элементы: -5, 3, -8, 2.
    \item Выведите размер стека и верхний элемент (должен быть 2).
    \item Вызовите pop, выведите результат (2).
    \item Повторите вывод размера и верхнего элемента (теперь 8).
\end{enumerate}

Пример использования:
\begin{lstlisting}[language=Python]
stack = Stack(push_absolute=True)
stack.push(-5)  # добавит 5
stack.push(3)   # добавит 3
stack.push(-8)  # добавит 8
stack.push(2)   # добавит 2

print("Размер стека:", stack.size())     # 4
print("Верхний элемент:", stack.peek())   # 2

popped = stack.pop()
print("Вытолкнут:", popped)  # 2

print("Размер после pop:", stack.size())    # 3
print("Верхний элемент:", stack.peek())     # 8
\end{lstlisting}

\item Написать программу на Python, которая создает класс Stack для представления стека с инкапсуляцией. Класс должен содержать методы push, pop, is\_empty, size и peek, которые реализуют операции вталкивания, выталкивания, проверки пустоты, получения размера и просмотра вершины стека соответственно. Программа также должна создавать экземпляр класса Stack, вталкивать в него элементы, выталкивать элементы и выводить информацию о стеке на экран.

Инструкции:
\begin{enumerate}
    \item Создайте класс Stack с методом \_\_init\_\_, который инициализирует пустой стек. Принимает параметр push\_rounded=False. Если True, то при добавлении элемент округляется до целого числа (round(element)).
    \item Создайте метод push, который принимает элемент. Если push\_rounded=True, добавляет round(element). Иначе — element.
    \item Создайте метод pop, который выталкивает верхний элемент из стека и возвращает его. Если стек пуст, возвращает None.
    \item Создайте метод is\_empty, который возвращает True, если стек пуст, и False в противном случае.
    \item Создайте метод size, который возвращает текущее количество элементов в стеке.
    \item Создайте метод peek, который возвращает верхний элемент стека, если стек не пуст. Если стек пуст, возвращает None.
    \item Создайте экземпляр класса Stack с push\_rounded=True.
    \item Добавьте элементы: 3.2, 4.7, 5.1, 6.9.
    \item Выведите размер стека и верхний элемент (должен быть 7).
    \item Вызовите pop, выведите результат (7).
    \item Повторите вывод размера и верхнего элемента (теперь 5).
\end{enumerate}

Пример использования:
\begin{lstlisting}[language=Python]
stack = Stack(push_rounded=True)
stack.push(3.2)  # 3
stack.push(4.7)  # 5
stack.push(5.1)  # 5
stack.push(6.9)  # 7

print("Размер стека:", stack.size())     # 4
print("Верхний элемент:", stack.peek())   # 7

popped = stack.pop()
print("Вытолкнут:", popped)  # 7

print("Размер после pop:", stack.size())    # 3
print("Верхний элемент:", stack.peek())     # 5
\end{lstlisting}

\item Написать программу на Python, которая создает класс Stack для представления стека с инкапсуляцией. Класс должен содержать методы push, pop, is\_empty, size и peek, которые реализуют операции вталкивания, выталкивания, проверки пустоты, получения размера и просмотра вершины стека соответственно. Программа также должна создавать экземпляр класса Stack, вталкивать в него элементы, выталкивать элементы и выводить информацию о стеке на экран.

Инструкции:
\begin{enumerate}
    \item Создайте класс Stack с методом \_\_init\_\_, который инициализирует пустой стек. Принимает параметр push\_negated=False. Если True, то при добавлении элемент сохраняется с обратным знаком (-element).
    \item Создайте метод push, который принимает элемент. Если push\_negated=True, добавляет -element. Иначе — element.
    \item Создайте метод pop, который выталкивает верхний элемент из стека и возвращает его. Если стек пуст, возвращает None.
    \item Создайте метод is\_empty, который возвращает True, если стек пуст, и False в противном случае.
    \item Создайте метод size, который возвращает текущее количество элементов в стеке.
    \item Создайте метод peek, который возвращает верхний элемент стека, если стек не пуст. Если стек пуст, возвращает None.
    \item Создайте экземпляр класса Stack с push\_negated=True.
    \item Добавьте элементы: 10, 20, 30, 40.
    \item Выведите размер стека и верхний элемент (должен быть -40).
    \item Вызовите pop, выведите результат (-40).
    \item Повторите вывод размера и верхнего элемента (теперь -30).
\end{enumerate}

Пример использования:
\begin{lstlisting}[language=Python]
stack = Stack(push_negated=True)
stack.push(10)  # -10
stack.push(20)  # -20
stack.push(30)  # -30
stack.push(40)  # -40

print("Размер стека:", stack.size())     # 4
print("Верхний элемент:", stack.peek())   # -40

popped = stack.pop()
print("Вытолкнут:", popped)  # -40

print("Размер после pop:", stack.size())    # 3
print("Верхний элемент:", stack.peek())     # -30
\end{lstlisting}

\item Написать программу на Python, которая создает класс Stack для представления стека с инкапсуляцией. Класс должен содержать методы push, pop, is\_empty, size и peek, которые реализуют операции вталкивания, выталкивания, проверки пустоты, получения размера и просмотра вершины стека соответственно. Программа также должна создавать экземпляр класса Stack, вталкивать в него элементы, выталкивать элементы и выводить информацию о стеке на экран.

Инструкции:
\begin{enumerate}
    \item Создайте класс Stack с методом \_\_init\_\_, который инициализирует пустой стек. Принимает параметр push\_doubled=False. Если True, то при добавлении элемент умножается на 2.
    \item Создайте метод push, который принимает элемент. Если push\_doubled=True, добавляет element * 2. Иначе — element.
    \item Создайте метод pop, который выталкивает верхний элемент из стека и возвращает его. Если стек пуст, возвращает None.
    \item Создайте метод is\_empty, который возвращает True, если стек пуст, и False в противном случае.
    \item Создайте метод size, который возвращает текущее количество элементов в стеке.
    \item Создайте метод peek, который возвращает верхний элемент стека, если стек не пуст. Если стек пуст, возвращает None.
    \item Создайте экземпляр класса Stack с push\_doubled=True.
    \item Добавьте элементы: 1, 2, 3, 4.
    \item Выведите размер стека и верхний элемент (должен быть 8).
    \item Вызовите pop, выведите результат (8).
    \item Повторите вывод размера и верхнего элемента (теперь 6).
\end{enumerate}

Пример использования:
\begin{lstlisting}[language=Python]
stack = Stack(push_doubled=True)
stack.push(1)  # 2
stack.push(2)  # 4
stack.push(3)  # 6
stack.push(4)  # 8

print("Размер стека:", stack.size())     # 4
print("Верхний элемент:", stack.peek())   # 8

popped = stack.pop()
print("Вытолкнут:", popped)  # 8

print("Размер после pop:", stack.size())    # 3
print("Верхний элемент:", stack.peek())     # 6
\end{lstlisting}

\item Написать программу на Python, которая создает класс Stack для представления стека с инкапсуляцией. Класс должен содержать методы push, pop, is\_empty, size и peek, которые реализуют операции вталкивания, выталкивания, проверки пустоты, получения размера и просмотра вершины стека соответственно. Программа также должна создавать экземпляр класса Stack, вталкивать в него элементы, выталкивать элементы и выводить информацию о стеке на экран.

Инструкции:
\begin{enumerate}
    \item Создайте класс Stack с методом \_\_init\_\_, который инициализирует пустой стек. Принимает параметр push\_halved=False. Если True, то при добавлении элемент делится на 2.0.
    \item Создайте метод push, который принимает элемент. Если push\_halved=True, добавляет element / 2.0. Иначе — element.
    \item Создайте метод pop, который выталкивает верхний элемент из стека и возвращает его. Если стек пуст, возвращает None.
    \item Создайте метод is\_empty, который возвращает True, если стек пуст, и False в противном случае.
    \item Создайте метод size, который возвращает текущее количество элементов в стеке.
    \item Создайте метод peek, который возвращает верхний элемент стека, если стек не пуст. Если стек пуст, возвращает None.
    \item Создайте экземпляр класса Stack с push\_halved=True.
    \item Добавьте элементы: 4, 8, 12, 16.
    \item Выведите размер стека и верхний элемент (должен быть 8.0).
    \item Вызовите pop, выведите результат (8.0).
    \item Повторите вывод размера и верхнего элемента (теперь 6.0).
\end{enumerate}

Пример использования:
\begin{lstlisting}[language=Python]
stack = Stack(push_halved=True)
stack.push(4)   # 2.0
stack.push(8)   # 4.0
stack.push(12)  # 6.0
stack.push(16)  # 8.0

print("Размер стека:", stack.size())     # 4
print("Верхний элемент:", stack.peek())   # 8.0

popped = stack.pop()
print("Вытолкнут:", popped)  # 8.0

print("Размер после pop:", stack.size())    # 3
print("Верхний элемент:", stack.peek())     # 6.0
\end{lstlisting}

\item Написать программу на Python, которая создает класс Stack для представления стека с инкапсуляцией. Класс должен содержать методы push, pop, is\_empty, size и peek, которые реализуют операции вталкивания, выталкивания, проверки пустоты, получения размера и просмотра вершины стека соответственно. Программа также должна создавать экземпляр класса Stack, вталкивать в него элементы, выталкивать элементы и выводить информацию о стеке на экран.

Инструкции:
\begin{enumerate}
    \item Создайте класс Stack с методом \_\_init\_\_, который инициализирует пустой стек. Принимает параметр push\_as\_string=False. Если True, то при добавлении элемент преобразуется в строку str(element).
    \item Создайте метод push, который принимает элемент. Если push\_as\_string=True, добавляет str(element). Иначе — element.
    \item Создайте метод pop, который выталкивает верхний элемент из стека и возвращает его. Если стек пуст, возвращает None.
    \item Создайте метод is\_empty, который возвращает True, если стек пуст, и False в противном случае.
    \item Создайте метод size, который возвращает текущее количество элементов в стеке.
    \item Создайте метод peek, который возвращает верхний элемент стека, если стек не пуст. Если стек пуст, возвращает None.
    \item Создайте экземпляр класса Stack с push\_as\_string=True.
    \item Добавьте элементы: 100, 200, 300, 400.
    \item Выведите размер стека и верхний элемент (должен быть "400").
    \item Вызовите pop, выведите результат ("400").
    \item Повторите вывод размера и верхнего элемента (теперь "300").
\end{enumerate}

Пример использования:
\begin{lstlisting}[language=Python]
stack = Stack(push_as_string=True)
stack.push(100)  # "100"
stack.push(200)  # "200"
stack.push(300)  # "300"
stack.push(400)  # "400"

print("Размер стека:", stack.size())     # 4
print("Верхний элемент:", stack.peek())   # "400"

popped = stack.pop()
print("Вытолкнут:", popped)  # "400"

print("Размер после pop:", stack.size())    # 3
print("Верхний элемент:", stack.peek())     # "300"
\end{lstlisting}

\item Написать программу на Python, которая создает класс Stack для представления стека с инкапсуляцией. Класс должен содержать методы push, pop, is\_empty, size и peek, которые реализуют операции вталкивания, выталкивания, проверки пустоты, получения размера и просмотра вершины стека соответственно. Программа также должна создавать экземпляр класса Stack, вталкивать в него элементы, выталкивать элементы и выводить информацию о стеке на экран.

Инструкции:
\begin{enumerate}
    \item Создайте класс Stack с методом \_\_init\_\_, который инициализирует пустой стек. Принимает параметр push\_with\_index=False. Если True, то при добавлении сохраняется кортеж (элемент, порядковый\_номер\_добавления).
    \item Создайте метод push, который принимает элемент. Если push\_with\_index=True, добавляет (element, self.\_counter), где \_counter — внутренний счетчик, увеличивающийся при каждом добавлении. Иначе — element.
    \item Создайте метод pop, который выталкивает верхний элемент (или кортеж) и возвращает его. Если стек пуст, возвращает None.
    \item Создайте метод is\_empty, который возвращает True, если стек пуст, и False в противном случае.
    \item Создайте метод size, который возвращает текущее количество элементов в стеке.
    \item Создайте метод peek, который возвращает верхний элемент (или кортеж), если стек не пуст. Если стек пуст, возвращает None.
    \item Создайте экземпляр класса Stack с push\_with\_index=True.
    \item Добавьте элементы: "alpha", "beta", "gamma".
    \item Выведите размер стека и результат peek (должен быть ("gamma", 2) — если считать с 0).
    \item Вызовите pop, выведите результат.
    \item Повторите вывод размера и peek.
\end{enumerate}

Пример использования:
\begin{lstlisting}[language=Python]
stack = Stack(push_with_index=True)
stack.push("alpha")  # ("alpha", 0)
stack.push("beta")   # ("beta", 1)
stack.push("gamma")  # ("gamma", 2)

print("Размер стека:", stack.size())
print("Верхний элемент:", stack.peek())  # ('gamma', 2)

popped = stack.pop()
print("Вытолкнут:", popped)  # ('gamma', 2)

print("Размер после pop:", stack.size())
print("Верхний элемент:", stack.peek())  # ('beta', 1)
\end{lstlisting}

\item Написать программу на Python, которая создает класс Stack для представления стека с инкапсуляцией. Класс должен содержать методы push, pop, is\_empty, size и peek, которые реализуют операции вталкивания, выталкивания, проверки пустоты, получения размера и просмотра вершины стека соответственно. Программа также должна создавать экземпляр класса Stack, вталкивать в него элементы, выталкивать элементы и выводить информацию о стеке на экран.

Инструкции:
\begin{enumerate}
    \item Создайте класс Stack с методом \_\_init\_\_, который инициализирует пустой стек. Принимает параметр push\_unique\_top=False. Если True, то при добавлении, если элемент равен текущему верхнему, он не добавляется.
    \item Создайте метод push, который принимает элемент. Если push\_unique\_top=True и стек не пуст и element == текущий\_верх, то элемент не добавляется. Иначе — добавляется.
    \item Создайте метод pop, который выталкивает верхний элемент из стека и возвращает его. Если стек пуст, возвращает None.
    \item Создайте метод is\_empty, который возвращает True, если стек пуст, и False в противном случае.
    \item Создайте метод size, который возвращает текущее количество элементов в стеке.
    \item Создайте метод peek, который возвращает верхний элемент стека, если стек не пуст. Если стек пуст, возвращает None.
    \item Создайте экземпляр класса Stack с push\_unique\_top=True.
    \item Добавьте элементы: 1, 2, 2, 3, 3, 3, 4.
    \item Выведите размер стека (должен быть 4) и верхний элемент (4).
    \item Вызовите pop, выведите результат (4).
    \item Повторите вывод размера и верхнего элемента (теперь 3).
\end{enumerate}

Пример использования:
\begin{lstlisting}[language=Python]
stack = Stack(push_unique_top=True)
stack.push(1)
stack.push(2)
stack.push(2)  # не добавится
stack.push(3)
stack.push(3)  # не добавится
stack.push(3)  # не добавится
stack.push(4)

print("Размер стека:", stack.size())     # 4
print("Верхний элемент:", stack.peek())   # 4

popped = stack.pop()
print("Вытолкнут:", popped)  # 4

print("Размер после pop:", stack.size())    # 3
print("Верхний элемент:", stack.peek())     # 3
\end{lstlisting}

\item Написать программу на Python, которая создает класс Stack для представления стека с инкапсуляцией. Класс должен содержать методы push, pop, is\_empty, size и peek, которые реализуют операции вталкивания, выталкивания, проверки пустоты, получения размера и просмотра вершины стека соответственно. Программа также должна создавать экземпляр класса Stack, вталкивать в него элементы, выталкивать элементы и выводить информацию о стеке на экран.

Инструкции:
\begin{enumerate}
    \item Создайте класс Stack с методом \_\_init\_\_, который инициализирует пустой стек. Принимает параметр push\_even\_only=False. Если True, то добавляются только четные числа.
    \item Создайте метод push, который принимает элемент. Если push\_even\_only=True и element \% 2 != 0, элемент не добавляется. Иначе — добавляется.
    \item Создайте метод pop, который выталкивает верхний элемент из стека и возвращает его. Если стек пуст, возвращает None.
    \item Создайте метод is\_empty, который возвращает True, если стек пуст, и False в противном случае.
    \item Создайте метод size, который возвращает текущее количество элементов в стеке.
    \item Создайте метод peek, который возвращает верхний элемент стека, если стек не пуст. Если стек пуст, возвращает None.
    \item Создайте экземпляр класса Stack с push\_even\_only=True.
    \item Добавьте элементы: 1 (не добавится), 2, 3 (не добавится), 4, 5 (не добавится), 6.
    \item Выведите размер стека (должен быть 3) и верхний элемент (6).
    \item Вызовите pop, выведите результат (6).
    \item Повторите вывод размера и верхнего элемента (теперь 4).
\end{enumerate}

Пример использования:
\begin{lstlisting}[language=Python]
stack = Stack(push_even_only=True)
stack.push(1)  # нет
stack.push(2)
stack.push(3)  # нет
stack.push(4)
stack.push(5)  # нет
stack.push(6)

print("Размер стека:", stack.size())     # 3
print("Верхний элемент:", stack.peek())   # 6

popped = stack.pop()
print("Вытолкнут:", popped)  # 6

print("Размер после pop:", stack.size())    # 2
print("Верхний элемент:", stack.peek())     # 4
\end{lstlisting}

\item Написать программу на Python, которая создает класс Stack для представления стека с инкапсуляцией. Класс должен содержать методы push, pop, is\_empty, size и peek, которые реализуют операции вталкивания, выталкивания, проверки пустоты, получения размера и просмотра вершины стека соответственно. Программа также должна создавать экземпляр класса Stack, вталкивать в него элементы, выталкивать элементы и выводить информацию о стеке на экран.

Инструкции:
\begin{enumerate}
    \item Создайте класс Stack с методом \_\_init\_\_, который инициализирует пустой стек. Принимает параметр push\_odd\_only=False. Если True, то добавляются только нечетные числа.
    \item Создайте метод push, который принимает элемент. Если push\_odd\_only=True и element \% 2 == 0, элемент не добавляется. Иначе — добавляется.
    \item Создайте метод pop, который выталкивает верхний элемент из стека и возвращает его. Если стек пуст, возвращает None.
    \item Создайте метод is\_empty, который возвращает True, если стек пуст, и False в противном случае.
    \item Создайте метод size, который возвращает текущее количество элементов в стеке.
    \item Создайте метод peek, который возвращает верхний элемент стека, если стек не пуст. Если стек пуст, возвращает None.
    \item Создайте экземпляр класса Stack с push\_odd\_only=True.
    \item Добавьте элементы: 2 (не добавится), 1, 4 (не добавится), 3, 6 (не добавится), 5.
    \item Выведите размер стека (должен быть 3) и верхний элемент (5).
    \item Вызовите pop, выведите результат (5).
    \item Повторите вывод размера и верхнего элемента (теперь 3).
\end{enumerate}

Пример использования:
\begin{lstlisting}[language=Python]
stack = Stack(push_odd_only=True)
stack.push(2)  # нет
stack.push(1)
stack.push(4)  # нет
stack.push(3)
stack.push(6)  # нет
stack.push(5)

print("Размер стека:", stack.size())     # 3
print("Верхний элемент:", stack.peek())   # 5

popped = stack.pop()
print("Вытолкнут:", popped)  # 5

print("Размер после pop:", stack.size())    # 2
print("Верхний элемент:", stack.peek())     # 3
\end{lstlisting}

\end{enumerate}

\subsubsection{Задача 3 (двусвязный список)}

\begin{enumerate}
\item Написать программу на Python, которая создает класс DoublyLinkedList, представляющий \textbf{двусвязный список} с инкапсуляцией внутренней структуры. Класс должен содержать методы для отображения данных, вставки и удаления узлов. Программа также должна создавать экземпляр класса, вставлять узлы и удалять узлы.

Инструкции:
\begin{enumerate}
    \item Создайте класс Node с методом \_\_init\_\_, который принимает данные data и сохраняет их в атрибуте self.\_data. Также инициализирует self.\_next и self.\_prev как None.
    \item Создайте класс DoublyLinkedList с методом \_\_init\_\_, который инициализирует self.\_head и self.\_tail как None.
    \item Создайте метод display в классе DoublyLinkedList, который выводит все элементы списка через пробел, двигаясь от головы к хвосту. Если список пуст — выводит "Список пуст".
    \item Создайте метод insert в классе DoublyLinkedList, который принимает значение и вставляет новый узел \textbf{в конец списка}. Обновляет self.\_tail и ссылки prev/next.
    \item Создайте метод delete в классе DoublyLinkedList, который принимает значение и удаляет \textbf{первое} вхождение узла с этим значением. Корректно обновляет соседние ссылки и self.\_head/self.\_tail при необходимости.
    \item Создайте экземпляр класса DoublyLinkedList.
    \item Вставьте узлы со значениями 10, 20, 30, 40.
    \item Вызовите display и выведите результат.
    \item Вставьте узел со значением 50.
    \item Снова вызовите display.
    \item Удалите узел со значением 20.
    \item Снова вызовите display.
\end{enumerate}

Пример использования:
\begin{lstlisting}[language=Python]
dll = DoublyLinkedList()
dll.insert(10)
dll.insert(20)
dll.insert(30)
dll.insert(40)

print("Initial Doubly Linked List:")
dll.display()

dll.insert(50)
print("After inserting 50:")
dll.display()

dll.delete(20)
print("After deleting 20:")
dll.display()
\end{lstlisting}

\item Написать программу на Python, которая создает класс DoublyLinkedList, представляющий \textbf{двусвязный список} с инкапсуляцией. Класс должен содержать методы для отображения данных, вставки и удаления узлов. Программа также должна создавать экземпляр класса, вставлять узлы и удалять узлы.

Инструкции:
\begin{enumerate}
    \item Создайте класс Node с методом \_\_init\_\_, который принимает item и сохраняет его в self.\_value. Инициализирует self.\_next и self.\_previous как None.
    \item Создайте класс DoublyLinkedList с методом \_\_init\_\_, который инициализирует self.\_first и self.\_last как None.
    \item Создайте метод display в классе DoublyLinkedList, который выводит элементы списка от первого к последнему, разделенные запятыми. Если список пуст — выводит "Нет элементов".
    \item Создайте метод insert в классе DoublyLinkedList, который принимает элемент и вставляет его \textbf{в начало списка}. Обновляет self.\_first и ссылки.
    \item Создайте метод delete в классе DoublyLinkedList, который принимает значение и удаляет \textbf{последнее} вхождение узла с этим значением. Корректно обновляет связи и границы списка.
    \item Создайте экземпляр класса DoublyLinkedList.
    \item Вставьте узлы: 5, 15, 25, 15.
    \item Вызовите display.
    \item Вставьте узел 35 в начало.
    \item Снова вызовите display.
    \item Удалите последнее вхождение 15.
    \item Снова вызовите display.
\end{enumerate}

Пример использования:
\begin{lstlisting}[language=Python]
dll = DoublyLinkedList()
dll.insert(5)
dll.insert(15)
dll.insert(25)
dll.insert(15)

print("Initial Doubly Linked List:")
dll.display()

dll.insert(35)
print("After inserting 35 at start:")
dll.display()

dll.delete(15)
print("After deleting last occurrence of 15:")
dll.display()
\end{lstlisting}

\item Написать программу на Python, которая создает класс DoublyLinkedList, представляющий \textbf{двусвязный список} с инкапсуляцией. Класс должен содержать методы для отображения данных, вставки и удаления узлов. Программа также должна создавать экземпляр класса, вставлять узлы и удалять узлы.

Инструкции:
\begin{enumerate}
    \item Создайте класс Node с методом \_\_init\_\_, который принимает content и сохраняет его в self.\_payload. Инициализирует self.\_forward и self.\_backward как None.
    \item Создайте класс DoublyLinkedList с методом \_\_init\_\_, который инициализирует self.\_root и self.\_end как None.
    \item Создайте метод display в классе DoublyLinkedList, который выводит элементы в формате "[элемент1] <-> [элемент2] <-> ...". Если пуст — "Пусто".
    \item Создайте метод insert в классе DoublyLinkedList, который принимает значение и вставляет его \textbf{после первого узла} (если список не пуст; если пуст — вставляет как первый).
    \item Создайте метод delete в классе DoublyLinkedList, который принимает значение и удаляет \textbf{все вхождения} этого значения. Обновляет ссылки и границы.
    \item Создайте экземпляр класса DoublyLinkedList.
    \item Вставьте узлы: 100, 200, 300.
    \item Вызовите display.
    \item Вставьте 150 после первого узла.
    \item Снова вызовите display.
    \item Удалите все вхождения 150.
    \item Снова вызовите display.
\end{enumerate}

Пример использования:
\begin{lstlisting}[language=Python]
dll = DoublyLinkedList()
dll.insert(100)
dll.insert(200)
dll.insert(300)

print("Initial Doubly Linked List:")
dll.display()

dll.insert(150)
print("After inserting 150 after first:")
dll.display()

dll.delete(150)
print("After deleting all 150s:")
dll.display()
\end{lstlisting}

\item Написать программу на Python, которая создает класс DoublyLinkedList, представляющий \textbf{двусвязный список} с инкапсуляцией. Класс должен содержать методы для отображения данных, вставки и удаления узлов. Программа также должна создавать экземпляр класса, вставлять узлы и удалять узлы.

Инструкции:
\begin{enumerate}
    \item Создайте класс Node с методом \_\_init\_\_, который принимает entry и сохраняет его в self.\_item. Инициализирует self.\_succ и self.\_pred как None.
    \item Создайте класс DoublyLinkedList с методом \_\_init\_\_, который инициализирует self.\_top и self.\_bottom как None.
    \item Создайте метод display в классе DoublyLinkedList, который выводит элементы в обратном порядке (от хвоста к голове), разделенные " | ". Если пуст — "Обратный просмотр: пусто".
    \item Создайте метод insert в классе DoublyLinkedList, который принимает значение и вставляет его \textbf{перед последним узлом} (если узлов >1; если 0 или 1 — вставляет в конец).
    \item Создайте метод delete в классе DoublyLinkedList, который принимает значение и удаляет первый найденный узел. Если узел — единственный, обнуляет self.\_top и self.\_bottom.
    \item Создайте экземпляр класса DoublyLinkedList.
    \item Вставьте узлы: 7, 14, 21.
    \item Вызовите display.
    \item Вставьте 18 перед последним узлом.
    \item Снова вызовите display.
    \item Удалите узел со значением 14.
    \item Снова вызовите display.
\end{enumerate}

Пример использования:
\begin{lstlisting}[language=Python]
dll = DoublyLinkedList()
dll.insert(7)
dll.insert(14)
dll.insert(21)

print("Initial Doubly Linked List (reversed):")
dll.display()

dll.insert(18)
print("After inserting 18 before last:")
dll.display()

dll.delete(14)
print("After deleting 14:")
dll.display()
\end{lstlisting}

\item Написать программу на Python, которая создает класс DoublyLinkedList, представляющий \textbf{двусвязный список} с инкапсуляцией. Класс должен содержать методы для отображения данных, вставки и удаления узлов. Программа также должна создавать экземпляр класса, вставлять узлы и удалять узлы.

Инструкции:
\begin{enumerate}
    \item Создайте класс Node с методом \_\_init\_\_, который принимает value и сохраняет его в self.\_key. Инициализирует self.\_link\_next и self.\_link\_prev как None.
    \item Создайте класс DoublyLinkedList с методом \_\_init\_\_, который инициализирует self.\_header и self.\_trailer как None.
    \item Создайте метод display в классе DoublyLinkedList, который выводит элементы в квадратных скобках через запятую: [a, b, c]. Если пуст — [].
    \item Создайте метод insert в классе DoublyLinkedList, который принимает значение и вставляет его \textbf{только если такого значения еще нет в списке}. Вставляет в конец.
    \item Создайте метод delete в классе DoublyLinkedList, который принимает значение и удаляет узел, если он существует. Если не существует — ничего не делает.
    \item Создайте экземпляр класса DoublyLinkedList.
    \item Вставьте узлы: 3, 6, 9, 6 (второй 6 не вставится).
    \item Вызовите display.
    \item Вставьте 12.
    \item Снова вызовите display.
    \item Удалите 6.
    \item Снова вызовите display.
\end{enumerate}

Пример использования:
\begin{lstlisting}[language=Python]
dll = DoublyLinkedList()
dll.insert(3)
dll.insert(6)
dll.insert(9)
dll.insert(6)  # игнорируется

print("Initial Doubly Linked List:")
dll.display()

dll.insert(12)
print("After inserting 12:")
dll.display()

dll.delete(6)
print("After deleting 6:")
dll.display()
\end{lstlisting}

\item Написать программу на Python, которая создает класс DoublyLinkedList, представляющий \textbf{двусвязный список} с инкапсуляцией. Класс должен содержать методы для отображения данных, вставки и удаления узлов. Программа также должна создавать экземпляр класса, вставлять узлы и удалять узлы.

Инструкции:
\begin{enumerate}
    \item Создайте класс Node с методом \_\_init\_\_, который принимает data\_point и сохраняет его в self.\_datum. Инициализирует self.\_next\_node и self.\_prev\_node как None.
    \item Создайте класс DoublyLinkedList с методом \_\_init\_\_, который инициализирует self.\_start и self.\_finish как None.
    \item Создайте метод display в классе DoublyLinkedList, который выводит элементы в формате "Элементы: val1 -> val2 -> val3", двигаясь от начала к концу. Если пуст — "Элементы: (нет)".
    \item Создайте метод insert в классе DoublyLinkedList, который принимает значение и вставляет его \textbf{только если оно больше последнего элемента} (если список не пуст). Если пуст — вставляет. Иначе — игнорирует.
    \item Создайте метод delete в классе DoublyLinkedList, который принимает значение и удаляет \textbf{первый узел}, если он равен значению. Не ищет дальше.
    \item Создайте экземпляр класса DoublyLinkedList.
    \item Вставьте узлы: 1, 5, 3 (игнорируется), 10.
    \item Вызовите display.
    \item Вставьте 7 (игнорируется, т.к. 7 < 10).
    \item Снова вызовите display.
    \item Удалите 5.
    \item Снова вызовите display.
\end{enumerate}

Пример использования:
\begin{lstlisting}[language=Python]
dll = DoublyLinkedList()
dll.insert(1)
dll.insert(5)
dll.insert(3)  # игнорируется
dll.insert(10)

print("Initial Doubly Linked List:")
dll.display()

dll.insert(7)  # игнорируется
print("After attempting to insert 7:")
dll.display()

dll.delete(5)
print("After deleting 5:")
dll.display()
\end{lstlisting}

\item Написать программу на Python, которая создает класс DoublyLinkedList, представляющий \textbf{двусвязный список} с инкапсуляцией. Класс должен содержать методы для отображения данных, вставки и удаления узлов. Программа также должна создавать экземпляр класса, вставлять узлы и удалять узлы.

Инструкции:
\begin{enumerate}
    \item Создайте класс Node с методом \_\_init\_\_, который принимает item\_value и сохраняет его в self.\_content. Инициализирует self.\_ptr\_next и self.\_ptr\_prev как None.
    \item Создайте класс DoublyLinkedList с методом \_\_init\_\_, который инициализирует self.\_head\_node и self.\_tail\_node как None.
    \item Создайте метод display в классе DoublyLinkedList, который выводит элементы в виде строки, разделенной точками: "a.b.c". Если пуст — "пусто".
    \item Создайте метод insert в классе DoublyLinkedList, который принимает значение и вставляет его \textbf{в середину списка} (если четное количество — после левой средней позиции; если нечетное — в центр). Если список пуст — вставляет как первый.
    \item Создайте метод delete в классе DoublyLinkedList, который принимает значение и удаляет \textbf{все узлы с этим значением}.
    \item Создайте экземпляр класса DoublyLinkedList.
    \item Вставьте узлы: 10, 20, 30.
    \item Вызовите display.
    \item Вставьте 25 в середину (между 20 и 30).
    \item Снова вызовите display.
    \item Удалите все вхождения 25.
    \item Снова вызовите display.
\end{enumerate}

Пример использования:
\begin{lstlisting}[language=Python]
dll = DoublyLinkedList()
dll.insert(10)
dll.insert(20)
dll.insert(30)

print("Initial Doubly Linked List:")
dll.display()

dll.insert(25)
print("After inserting 25 in middle:")
dll.display()

dll.delete(25)
print("After deleting 25:")
dll.display()
\end{lstlisting}

\item Написать программу на Python, которая создает класс DoublyLinkedList, представляющий \textbf{двусвязный список} с инкапсуляцией. Класс должен содержать методы для отображения данных, вставки и удаления узлов. Программа также должна создавать экземпляр класса, вставлять узлы и удалять узлы.

Инструкции:
\begin{enumerate}
    \item Создайте класс Node с методом \_\_init\_\_, который принимает node\_data и сохраняет его в self.\_info. Инициализирует self.\_nxt и self.\_prv как None.
    \item Создайте класс DoublyLinkedList с методом \_\_init\_\_, который инициализирует self.\_front и self.\_rear как None.
    \item Создайте метод display в классе DoublyLinkedList, который выводит элементы в формате "Front->Back: [значения]" и "Back->Front: [значения в обратном порядке]". Если пуст — "Список пуст в обоих направлениях".
    \item Создайте метод insert в классе DoublyLinkedList, который принимает значение и вставляет его \textbf{в начало, только если значение четное}. Если нечетное — вставляет в конец.
    \item Создайте метод delete в классе DoublyLinkedList, который принимает значение и удаляет \textbf{первое вхождение}.
    \item Создайте экземпляр класса DoublyLinkedList.
    \item Вставьте узлы: 4 (в начало), 7 (в конец), 6 (в начало), 9 (в конец).
    \item Вызовите display.
    \item Вставьте 8 (в начало).
    \item Снова вызовите display.
    \item Удалите 7.
    \item Снова вызовите display.
\end{enumerate}

Пример использования:
\begin{lstlisting}[language=Python]
dll = DoublyLinkedList()
dll.insert(4)
dll.insert(7)
dll.insert(6)
dll.insert(9)

print("Initial Doubly Linked List:")
dll.display()

dll.insert(8)
print("After inserting 8:")
dll.display()

dll.delete(7)
print("After deleting 7:")
dll.display()
\end{lstlisting}

\item Написать программу на Python, которая создает класс DoublyLinkedList, представляющий \textbf{двусвязный список} с инкапсуляцией. Класс должен содержать методы для отображения данных, вставки и удаления узлов. Программа также должна создавать экземпляр класса, вставлять узлы и удалять узлы.

Инструкции:
\begin{enumerate}
    \item Создайте класс Node с методом \_\_init\_\_, который принимает value и сохраняет его в self.\_element. Инициализирует self.\_next\_elem и self.\_prev\_elem как None.
    \item Создайте класс DoublyLinkedList с методом \_\_init\_\_, который инициализирует self.\_head\_elem и self.\_tail\_elem как None.
    \item Создайте метод display в классе DoublyLinkedList, который выводит элементы в виде: "HEAD <-> val1 <-> val2 <-> ... <-> TAIL". Если пуст — "HEAD <-> TAIL (пусто)".
    \item Создайте метод insert в классе DoublyLinkedList, который принимает значение и вставляет его \textbf{после узла с наименьшим значением} (если несколько — после первого). Если список пуст — вставляет как единственный.
    \item Создайте метод delete в классе DoublyLinkedList, который принимает значение и удаляет \textbf{последнее вхождение}.
    \item Создайте экземпляр класса DoublyLinkedList.
    \item Вставьте узлы: 50, 30, 40.
    \item Вызовите display.
    \item Вставьте 35 (после 30 — минимального).
    \item Снова вызовите display.
    \item Удалите последнее вхождение 40.
    \item Снова вызовите display.
\end{enumerate}

Пример использования:
\begin{lstlisting}[language=Python]
dll = DoublyLinkedList()
dll.insert(50)
dll.insert(30)
dll.insert(40)

print("Initial Doubly Linked List:")
dll.display()

dll.insert(35)
print("After inserting 35 after min:")
dll.display()

dll.delete(40)
print("After deleting last occurrence of 40:")
dll.display()
\end{lstlisting}

\item Написать программу на Python, которая создает класс DoublyLinkedList, представляющий \textbf{двусвязный список} с инкапсуляцией. Класс должен содержать методы для отображения данных, вставки и удаления узлов. Программа также должна создавать экземпляр класса, вставлять узлы и удалять узлы.

Инструкции:
\begin{enumerate}
    \item Создайте класс Node с методом \_\_init\_\_, который принимает data и сохраняет его в self.\_val. Инициализирует self.\_link\_f и self.\_link\_b как None.
    \item Создайте класс DoublyLinkedList с методом \_\_init\_\_, который инициализирует self.\_first\_item и self.\_last\_item как None.
    \item Создайте метод display в классе DoublyLinkedList, который выводит элементы в виде: "Элементы (прямой порядок): ...", а затем "Элементы (обратный порядок): ...". Если пуст — "Нет данных".
    \item Создайте метод insert в классе DoublyLinkedList, который принимает значение и вставляет его \textbf{перед узлом с наибольшим значением} (если несколько — перед первым). Если список пуст — вставляет как единственный.
    \item Создайте метод delete в классе DoublyLinkedList, который принимает значение и удаляет \textbf{все вхождения}.
    \item Создайте экземпляр класса DoublyLinkedList.
    \item Вставьте узлы: 5, 15, 10.
    \item Вызовите display.
    \item Вставьте 12 (перед 15 — максимальным).
    \item Снова вызовите display.
    \item Удалите все вхождения 10.
    \item Снова вызовите display.
\end{enumerate}

Пример использования:
\begin{lstlisting}[language=Python]
dll = DoublyLinkedList()
dll.insert(5)
dll.insert(15)
dll.insert(10)

print("Initial Doubly Linked List:")
dll.display()

dll.insert(12)
print("After inserting 12 before max:")
dll.display()

dll.delete(10)
print("After deleting all 10s:")
dll.display()
\end{lstlisting}

\item Написать программу на Python, которая создает класс DoublyLinkedList, представляющий \textbf{двусвязный список} с инкапсуляцией. Класс должен содержать методы для отображения данных, вставки и удаления узлов. Программа также должна создавать экземпляр класса, вставлять узлы и удалять узлы.

Инструкции:
\begin{enumerate}
    \item Создайте класс Node с методом \_\_init\_\_, который принимает item и сохраняет его в self.\_data\_field. Инициализирует self.\_next\_ref и self.\_prev\_ref как None.
    \item Создайте класс DoublyLinkedList с методом \_\_init\_\_, который инициализирует self.\_entry\_point и self.\_exit\_point как None.
    \item Создайте метод display в классе DoublyLinkedList, который выводит элементы в одну строку, разделенные " => ", и в конце добавляет " => None". Если пуст — "None".
    \item Создайте метод insert в классе DoublyLinkedList, который принимает значение и вставляет его \textbf{в позицию, равную значению по модулю длины списка} (если список не пуст; если пуст — вставляет как первый). Например, при длине 3 и значении 7: 7 \% 3 = 1 → вставка на позицию 1 (второй элемент).
    \item Создайте метод delete в классе DoublyLinkedList, который принимает значение и удаляет \textbf{первое вхождение}.
    \item Создайте экземпляр класса DoublyLinkedList.
    \item Вставьте узлы: 2, 4, 6.
    \item Вызовите display.
    \item Вставьте 5 (5 \% 3 = 2 → вставка на позицию 2, т.е. после 4, перед 6).
    \item Снова вызовите display.
    \item Удалите 4.
    \item Снова вызовите display.
\end{enumerate}

Пример использования:
\begin{lstlisting}[language=Python]
dll = DoublyLinkedList()
dll.insert(2)
dll.insert(4)
dll.insert(6)

print("Initial Doubly Linked List:")
dll.display()

dll.insert(5)
print("After inserting 5 at position 5 % 3 = 2:")
dll.display()

dll.delete(4)
print("After deleting 4:")
dll.display()
\end{lstlisting}

\item Написать программу на Python, которая создает класс DoublyLinkedList, представляющий \textbf{двусвязный список} с инкапсуляцией. Класс должен содержать методы для отображения данных, вставки и удаления узлов. Программа также должна создавать экземпляр класса, вставлять узлы и удалять узлы.

Инструкции:
\begin{enumerate}
    \item Создайте класс Node с методом \_\_init\_\_, который принимает content и сохраняет его в self.\_stored\_value. Инициализирует self.\_connection\_next и self.\_connection\_prev как None.
    \item Создайте класс DoublyLinkedList с методом \_\_init\_\_, который инициализирует self.\_input и self.\_output как None.
    \item Создайте метод display в классе DoublyLinkedList, который выводит элементы в формате: "List: [значения через пробел] (размер: N)". Если пуст — "List: [] (размер: 0)".
    \item Создайте метод insert в классе DoublyLinkedList, который принимает значение и вставляет его \textbf{только если оно не отрицательное}. Вставляет в конец.
    \item Создайте метод delete в классе DoublyLinkedList, который принимает значение и удаляет \textbf{первое вхождение, только если значение положительное}. Если значение <=0 — ничего не делает.
    \item Создайте экземпляр класса DoublyLinkedList.
    \item Вставьте узлы: -1 (игнорируется), 8, 0, 12, -5 (игнорируется).
    \item Вызовите display.
    \item Вставьте 10.
    \item Снова вызовите display.
    \item Удалите 0 (не удаляется, т.к. не положительное).
    \item Снова вызовите display.
\end{enumerate}

Пример использования:
\begin{lstlisting}[language=Python]
dll = DoublyLinkedList()
dll.insert(-1)  # игнорируется
dll.insert(8)
dll.insert(0)
dll.insert(12)
dll.insert(-5)  # игнорируется

print("Initial Doubly Linked List:")
dll.display()

dll.insert(10)
print("After inserting 10:")
dll.display()

dll.delete(0)  # не удаляется
print("After attempting to delete 0:")
dll.display()
\end{lstlisting}

\item Написать программу на Python, которая создает класс DoublyLinkedList, представляющий \textbf{двусвязный список} с инкапсуляцией. Класс должен содержать методы для отображения данных, вставки и удаления узлов. Программа также должна создавать экземпляр класса, вставлять узлы и удалять узлы.

Инструкции:
\begin{enumerate}
    \item Создайте класс Node с методом \_\_init\_\_, который принимает data и сохраняет его в self.\_record. Инициализирует self.\_next\_entry и self.\_prev\_entry как None.
    \item Создайте класс DoublyLinkedList с методом \_\_init\_\_, который инициализирует self.\_head\_record и self.\_tail\_record как None.
    \item Создайте метод display в классе DoublyLinkedList, который выводит элементы в виде: "Записи: val1, val2, ..., valN". Если пуст — "Записей нет".
    \item Создайте метод insert в классе DoublyLinkedList, который принимает значение и вставляет его \textbf{в начало, если значение нечетное, и в конец, если четное}.
    \item Создайте метод delete в классе DoublyLinkedList, который принимает значение и удаляет \textbf{все узлы с этим значением}.
    \item Создайте экземпляр класса DoublyLinkedList.
    \item Вставьте узлы: 3 (в начало), 4 (в конец), 5 (в начало), 6 (в конец).
    \item Вызовите display.
    \item Вставьте 7 (в начало).
    \item Снова вызовите display.
    \item Удалите все вхождения 4.
    \item Снова вызовите display.
\end{enumerate}

Пример использования:
\begin{lstlisting}[language=Python]
dll = DoublyLinkedList()
dll.insert(3)
dll.insert(4)
dll.insert(5)
dll.insert(6)

print("Initial Doubly Linked List:")
dll.display()

dll.insert(7)
print("After inserting 7:")
dll.display()

dll.delete(4)
print("After deleting all 4s:")
dll.display()
\end{lstlisting}

\item Написать программу на Python, которая создает класс DoublyLinkedList, представляющий \textbf{двусвязный список} с инкапсуляцией. Класс должен содержать методы для отображения данных, вставки и удаления узлов. Программа также должна создавать экземпляр класса, вставлять узлы и удалять узлы.

Инструкции:
\begin{enumerate}
    \item Создайте класс Node с методом \_\_init\_\_, который принимает value и сохраняет его в self.\_cell. Инициализирует self.\_cell\_next и self.\_cell\_prev как None.
    \item Создайте класс DoublyLinkedList с методом \_\_init\_\_, который инициализирует self.\_first\_cell и self.\_last\_cell как None.
    \item Создайте метод display в классе DoublyLinkedList, который выводит элементы в виде: "Ячейки: [значения]" и отдельно "Количество: N". Если пуст — "Список ячеек пуст".
    \item Создайте метод insert в классе DoublyLinkedList, который принимает значение и вставляет его \textbf{после каждого узла, значение которого кратно 3} (если таких нет — вставляет в конец).
    \item Создайте метод delete в классе DoublyLinkedList, который принимает значение и удаляет \textbf{первое вхождение}.
    \item Создайте экземпляр класса DoublyLinkedList.
    \item Вставьте узлы: 6, 9, 4.
    \item Вызовите display.
    \item Вставьте 12 (вставится после 6 и после 9 — но по условию вставляется только один узел; вставим после первого кратного 3, т.е. после 6).
    \item Снова вызовите display.
    \item Удалите 9.
    \item Снова вызовите display.
\end{enumerate}

Пример использования:
\begin{lstlisting}[language=Python]
dll = DoublyLinkedList()
dll.insert(6)
dll.insert(9)
dll.insert(4)

print("Initial Doubly Linked List:")
dll.display()

dll.insert(12)
print("After inserting 12 after first multiple of 3:")
dll.display()

dll.delete(9)
print("After deleting 9:")
dll.display()
\end{lstlisting}

\item Написать программу на Python, которая создает класс DoublyLinkedList, представляющий \textbf{двусвязный список} с инкапсуляцией. Класс должен содержать методы для отображения данных, вставки и удаления узлов. Программа также должна создавать экземпляр класса, вставлять узлы и удалять узлы.

Инструкции:
\begin{enumerate}
    \item Создайте класс Node с методом \_\_init\_\_, который принимает item и сохраняет его в self.\_slot. Инициализирует self.\_slot\_next и self.\_slot\_prev как None.
    \item Создайте класс DoublyLinkedList с методом \_\_init\_\_, который инициализирует self.\_start\_slot и self.\_end\_slot как None.
    \item Создайте метод display в классе DoublyLinkedList, который выводит элементы в виде: "Слоты: val1 | val2 | val3". Если пуст — "Слоты отсутствуют".
    \item Создайте метод insert в классе DoublyLinkedList, который принимает значение и вставляет его \textbf{перед каждым узлом, значение которого кратно 5} (если таких нет — вставляет в начало).
    \item Создайте метод delete в классе DoublyLinkedList, который принимает значение и удаляет \textbf{последнее вхождение}.
    \item Создайте экземпляр класса DoublyLinkedList.
    \item Вставьте узлы: 10, 15, 7.
    \item Вызовите display.
    \item Вставьте 5 (вставится перед 10 и перед 15 — но по условию вставляется только один узел; вставим перед первым кратным 5, т.е. перед 10).
    \item Снова вызовите display.
    \item Удалите последнее вхождение 15.
    \item Снова вызовите display.
\end{enumerate}

Пример использования:
\begin{lstlisting}[language=Python]
dll = DoublyLinkedList()
dll.insert(10)
dll.insert(15)
dll.insert(7)

print("Initial Doubly Linked List:")
dll.display()

dll.insert(5)
print("After inserting 5 before first multiple of 5:")
dll.display()

dll.delete(15)
print("After deleting last occurrence of 15:")
dll.display()
\end{lstlisting}

\item Написать программу на Python, которая создает класс DoublyLinkedList, представляющий \textbf{двусвязный список} с инкапсуляцией. Класс должен содержать методы для отображения данных, вставки и удаления узлов. Программа также должна создавать экземпляр класса, вставлять узлы и удалять узлы.

Инструкции:
\begin{enumerate}
    \item Создайте класс Node с методом \_\_init\_\_, который принимает data и сохраняет его в self.\_block. Инициализирует self.\_block\_next и self.\_block\_prev как None.
    \item Создайте класс DoublyLinkedList с методом \_\_init\_\_, который инициализирует self.\_head\_block и self.\_tail\_block как None.
    \item Создайте метод display в классе DoublyLinkedList, который выводит элементы в виде: "Блоки: [значения]" и "Обратный порядок: [значения в обратном порядке]". Если пуст — "Нет блоков".
    \item Создайте метод insert в классе DoublyLinkedList, который принимает значение и вставляет его \textbf{только если сумма цифр значения четная}. Вставляет в конец.
    \item Создайте метод delete в классе DoublyLinkedList, который принимает значение и удаляет \textbf{все вхождения}.
    \item Создайте экземпляр класса DoublyLinkedList.
    \item Вставьте узлы: 23 (2+3=5 — нечет, не вставляется), 24 (2+4=6 — чет, вставляется), 35 (3+5=8 — чет, вставляется), 13 (1+3=4 — чет, вставляется).
    \item Вызовите display.
    \item Вставьте 46 (4+6=10 — чет, вставляется).
    \item Снова вызовите display.
    \item Удалите все вхождения 24.
    \item Снова вызовите display.
\end{enumerate}

Пример использования:
\begin{lstlisting}[language=Python]
dll = DoublyLinkedList()
dll.insert(23)  # не вставляется
dll.insert(24)
dll.insert(35)
dll.insert(13)

print("Initial Doubly Linked List:")
dll.display()

dll.insert(46)
print("After inserting 46:")
dll.display()

dll.delete(24)
print("After deleting all 24s:")
dll.display()
\end{lstlisting}

\item Написать программу на Python, которая создает класс DoublyLinkedList, представляющий \textbf{двусвязный список} с инкапсуляцией. Класс должен содержать методы для отображения данных, вставки и удаления узлов. Программа также должна создавать экземпляр класса, вставлять узлы и удалять узлы.

Инструкции:
\begin{enumerate}
    \item Создайте класс Node с методом \_\_init\_\_, который принимает value и сохраняет его в self.\_unit. Инициализирует self.\_unit\_next и self.\_unit\_prev как None.
    \item Создайте класс DoublyLinkedList с методом \_\_init\_\_, который инициализирует self.\_first\_unit и self.\_last\_unit как None.
    \item Создайте метод display в классе DoublyLinkedList, который выводит элементы в виде: "Единицы: val1 → val2 → val3 → null". Если пуст — "null".
    \item Создайте метод insert в классе DoublyLinkedList, который принимает значение и вставляет его \textbf{только если оно простое число} (используйте вспомогательную функцию is\_prime). Вставляет в начало.
    \item Создайте метод delete в классе DoublyLinkedList, который принимает значение и удаляет \textbf{первое вхождение}.
    \item Создайте вспомогательную функцию is\_prime(n).
    \item Создайте экземпляр класса DoublyLinkedList.
    \item Вставьте узлы: 4 (не простое), 5 (простое), 6 (не простое), 7 (простое), 8 (не простое), 11 (простое).
    \item Вызовите display.
    \item Вставьте 13 (простое).
    \item Снова вызовите display.
    \item Удалите 7.
    \item Снова вызовите display.
\end{enumerate}

Пример использования:
\begin{lstlisting}[language=Python]
def is_prime(n):
    if n < 2:
        return False
    for i in range(2, int(n**0.5)+1):
        if n % i == 0:
            return False
    return True

dll = DoublyLinkedList()
dll.insert(4)   # нет
dll.insert(5)   # да
dll.insert(6)   # нет
dll.insert(7)   # да
dll.insert(8)   # нет
dll.insert(11)  # да

print("Initial Doubly Linked List:")
dll.display()

dll.insert(13)
print("After inserting 13:")
dll.display()

dll.delete(7)
print("After deleting 7:")
dll.display()
\end{lstlisting}

\item Написать программу на Python, которая создает класс DoublyLinkedList, представляющий \textbf{двусвязный список} с инкапсуляцией. Класс должен содержать методы для отображения данных, вставки и удаления узлов. Программа также должна создавать экземпляр класса, вставлять узлы и удалять узлы.

Инструкции:
\begin{enumerate}
    \item Создайте класс Node с методом \_\_init\_\_, который принимает item и сохраняет его в self.\_segment. Инициализирует self.\_seg\_next и self.\_seg\_prev как None.
    \item Создайте класс DoublyLinkedList с методом \_\_init\_\_, который инициализирует self.\_head\_seg и self.\_tail\_seg как None.
    \item Создайте метод display в классе DoublyLinkedList, который выводит элементы в виде: "Сегменты (вперед): ...", "Сегменты (назад): ...". Если пуст — "Список сегментов пуст".
    \item Создайте метод insert в классе DoublyLinkedList, который принимает значение и вставляет его \textbf{только если оно палиндром} (например, 121, 33). Вставляет в конец.
    \item Создайте метод delete в классе DoublyLinkedList, который принимает значение и удаляет \textbf{последнее вхождение}.
    \item Создайте экземпляр класса DoublyLinkedList.
    \item Вставьте узлы: 12 (не палиндром), 22 (палиндром), 34 (не палиндром), 55 (палиндром), 121 (палиндром).
    \item Вызовите display.
    \item Вставьте 33 (палиндром).
    \item Снова вызовите display.
    \item Удалите последнее вхождение 55.
    \item Снова вызовите display.
\end{enumerate}

Пример использования:
\begin{lstlisting}[language=Python]
dll = DoublyLinkedList()
dll.insert(12)  # нет
dll.insert(22)  # да
dll.insert(34)  # нет
dll.insert(55)  # да
dll.insert(121) # да

print("Initial Doubly Linked List:")
dll.display()

dll.insert(33)
print("After inserting 33:")
dll.display()

dll.delete(55)
print("After deleting last occurrence of 55:")
dll.display()
\end{lstlisting}

\item Написать программу на Python, которая создает класс DoublyLinkedList, представляющий \textbf{двусвязный список} с инкапсуляцией. Класс должен содержать методы для отображения данных, вставки и удаления узлов. Программа также должна создавать экземпляр класса, вставлять узлы и удалять узлы.

Инструкции:
\begin{enumerate}
    \item Создайте класс Node с методом \_\_init\_\_, который принимает data и сохраняет его в self.\_piece. Инициализирует self.\_piece\_next и self.\_piece\_prev как None.
    \item Создайте класс DoublyLinkedList с методом \_\_init\_\_, который инициализирует self.\_first\_piece и self.\_last\_piece как None.
    \item Создайте метод display в классе DoublyLinkedList, который выводит элементы в виде: "Части: val1 - val2 - val3". Если пуст — "Нет частей".
    \item Создайте метод insert в классе DoublyLinkedList, который принимает значение и вставляет его \textbf{только если оно степень двойки} (1,2,4,8,16...). Вставляет в начало.
    \item Создайте метод delete в классе DoublyLinkedList, который принимает значение и удаляет \textbf{все вхождения}.
    \item Создайте экземпляр класса DoublyLinkedList.
    \item Вставьте узлы: 3 (нет), 4 (да), 5 (нет), 8 (да), 9 (нет), 16 (да).
    \item Вызовите display.
    \item Вставьте 32 (да).
    \item Снова вызовите display.
    \item Удалите все вхождения 8.
    \item Снова вызовите display.
\end{enumerate}

Пример использования:
\begin{lstlisting}[language=Python]
dll = DoublyLinkedList()
dll.insert(3)   # нет
dll.insert(4)   # да
dll.insert(5)   # нет
dll.insert(8)   # да
dll.insert(9)   # нет
dll.insert(16)  # да

print("Initial Doubly Linked List:")
dll.display()

dll.insert(32)
print("After inserting 32:")
dll.display()

dll.delete(8)
print("After deleting all 8s:")
dll.display()
\end{lstlisting}

\item Написать программу на Python, которая создает класс DoublyLinkedList, представляющий \textbf{двусвязный список} с инкапсуляцией. Класс должен содержать методы для отображения данных, вставки и удаления узлов. Программа также должна создавать экземпляр класса, вставлять узлы и удалять узлы.

Инструкции:
\begin{enumerate}
    \item Создайте класс Node с методом \_\_init\_\_, который принимает value и сохраняет его в self.\_fragment. Инициализирует self.\_frag\_next и self.\_frag\_prev как None.
    \item Создайте класс DoublyLinkedList с методом \_\_init\_\_, который инициализирует self.\_start\_frag и self.\_end\_frag как None.
    \item Создайте метод display в классе DoublyLinkedList, который выводит элементы в виде: "Фрагменты → val1 → val2 → val3 → конец". Если пуст — "Фрагменты: конец".
    \item Создайте метод insert в классе DoublyLinkedList, который принимает значение и вставляет его \textbf{только если оно делится на 3 без остатка}. Вставляет в конец.
    \item Создайте метод delete в классе DoublyLinkedList, который принимает значение и удаляет \textbf{первое вхождение}.
    \item Создайте экземпляр класса DoublyLinkedList.
    \item Вставьте узлы: 1 (нет), 3 (да), 4 (нет), 6 (да), 7 (нет), 9 (да).
    \item Вызовите display.
    \item Вставьте 12 (да).
    \item Снова вызовите display.
    \item Удалите 6.
    \item Снова вызовите display.
\end{enumerate}

Пример использования:
\begin{lstlisting}[language=Python]
dll = DoublyLinkedList()
dll.insert(1)  # нет
dll.insert(3)  # да
dll.insert(4)  # нет
dll.insert(6)  # да
dll.insert(7)  # нет
dll.insert(9)  # да

print("Initial Doubly Linked List:")
dll.display()

dll.insert(12)
print("After inserting 12:")
dll.display()

dll.delete(6)
print("After deleting 6:")
dll.display()
\end{lstlisting}

\item Написать программу на Python, которая создает класс DoublyLinkedList, представляющий \textbf{двусвязный список} с инкапсуляцией. Класс должен содержать методы для отображения данных, вставки и удаления узлов. Программа также должна создавать экземпляр класса, вставлять узлы и удалять узлы.

Инструкции:
\begin{enumerate}
    \item Создайте класс Node с методом \_\_init\_\_, который принимает item и сохраняет его в self.\_chunk. Инициализирует self.\_chunk\_next и self.\_chunk\_prev как None.
    \item Создайте класс DoublyLinkedList с методом \_\_init\_\_, который инициализирует self.\_head\_chunk и self.\_tail\_chunk как None.
    \item Создайте метод display в классе DoublyLinkedList, который выводит элементы в виде: "Чанки: [значения]" и "Размер: N". Если пуст — "Чанков нет".
    \item Создайте метод insert в классе DoublyLinkedList, который принимает значение и вставляет его \textbf{только если оно не делится на 5}. Вставляет в начало.
    \item Создайте метод delete в классе DoublyLinkedList, который принимает значение и удаляет \textbf{последнее вхождение}.
    \item Создайте экземпляр класса DoublyLinkedList.
    \item Вставьте узлы: 10 (делится на 5 — не вставляется), 11 (не делится — вставляется), 15 (делится — не вставляется), 16 (не делится — вставляется), 20 (делится — не вставляется), 21 (не делится — вставляется).
    \item Вызовите display.
    \item Вставьте 26 (не делится — вставляется).
    \item Снова вызовите display.
    \item Удалите последнее вхождение 16.
    \item Снова вызовите display.
\end{enumerate}

Пример использования:
\begin{lstlisting}[language=Python]
dll = DoublyLinkedList()
dll.insert(10)  # нет
dll.insert(11)  # да
dll.insert(15)  # нет
dll.insert(16)  # да
dll.insert(20)  # нет
dll.insert(21)  # да

print("Initial Doubly Linked List:")
dll.display()

dll.insert(26)
print("After inserting 26:")
dll.display()

dll.delete(16)
print("After deleting last occurrence of 16:")
dll.display()
\end{lstlisting}

\item Написать программу на Python, которая создает класс DoublyLinkedList, представляющий \textbf{двусвязный список} с инкапсуляцией. Класс должен содержать методы для отображения данных, вставки и удаления узлов. Программа также должна создавать экземпляр класса, вставлять узлы и удалять узлы.

Инструкции:
\begin{enumerate}
    \item Создайте класс Node с методом \_\_init\_\_, который принимает data и сохраняет его в self.\_item\_data. Инициализирует self.\_next\_item и self.\_prev\_item как None.
    \item Создайте класс DoublyLinkedList с методом \_\_init\_\_, который инициализирует self.\_first\_data и self.\_last\_data как None.
    \item Создайте метод display в классе DoublyLinkedList, который выводит элементы в виде: "Данные (→): val1, val2, val3" и "Данные (←): val3, val2, val1". Если пуст — "Данные отсутствуют".
    \item Создайте метод insert в классе DoublyLinkedList, который принимает значение и вставляет его \textbf{только если оно больше 10}. Вставляет в конец.
    \item Создайте метод delete в классе DoublyLinkedList, который принимает значение и удаляет \textbf{все вхождения}.
    \item Создайте экземпляр класса DoublyLinkedList.
    \item Вставьте узлы: 5 (нет), 15 (да), 8 (нет), 20 (да), 12 (да).
    \item Вызовите display.
    \item Вставьте 25 (да).
    \item Снова вызовите display.
    \item Удалите все вхождения 20.
    \item Снова вызовите display.
\end{enumerate}

Пример использования:
\begin{lstlisting}[language=Python]
dll = DoublyLinkedList()
dll.insert(5)   # нет
dll.insert(15)  # да
dll.insert(8)   # нет
dll.insert(20)  # да
dll.insert(12)  # да

print("Initial Doubly Linked List:")
dll.display()

dll.insert(25)
print("After inserting 25:")
dll.display()

dll.delete(20)
print("After deleting all 20s:")
dll.display()
\end{lstlisting}

\item Написать программу на Python, которая создает класс DoublyLinkedList, представляющий \textbf{двусвязный список} с инкапсуляцией. Класс должен содержать методы для отображения данных, вставки и удаления узлов. Программа также должна создавать экземпляр класса, вставлять узлы и удалять узлы.

Инструкции:
\begin{enumerate}
    \item Создайте класс Node с методом \_\_init\_\_, который принимает value и сохраняет его в self.\_node\_value. Инициализирует self.\_node\_next и self.\_node\_prev как None.
    \item Создайте класс DoublyLinkedList с методом \_\_init\_\_, который инициализирует self.\_start\_node и self.\_end\_node как None.
    \item Создайте метод display в классе DoublyLinkedList, который выводит элементы в виде: "Узлы: val1 <-> val2 <-> val3". Если пуст — "Нет узлов".
    \item Создайте метод insert в классе DoublyLinkedList, который принимает значение и вставляет его \textbf{только если оно меньше 50}. Вставляет в начало.
    \item Создайте метод delete в классе DoublyLinkedList, который принимает значение и удаляет \textbf{первое вхождение}.
    \item Создайте экземпляр класса DoublyLinkedList.
    \item Вставьте узлы: 60 (нет), 30 (да), 70 (нет), 40 (да), 45 (да).
    \item Вызовите display.
    \item Вставьте 25 (да).
    \item Снова вызовите display.
    \item Удалите 40.
    \item Снова вызовите display.
\end{enumerate}

Пример использования:
\begin{lstlisting}[language=Python]
dll = DoublyLinkedList()
dll.insert(60)  # нет
dll.insert(30)  # да
dll.insert(70)  # нет
dll.insert(40)  # да
dll.insert(45)  # да

print("Initial Doubly Linked List:")
dll.display()

dll.insert(25)
print("After inserting 25:")
dll.display()

dll.delete(40)
print("After deleting 40:")
dll.display()
\end{lstlisting}

\item Написать программу на Python, которая создает класс DoublyLinkedList, представляющий \textbf{двусвязный список} с инкапсуляцией. Класс должен содержать методы для отображения данных, вставки и удаления узлов. Программа также должна создавать экземпляр класса, вставлять узлы и удалять узлы.

Инструкции:
\begin{enumerate}
    \item Создайте класс Node с методом \_\_init\_\_, который принимает item и сохраняет его в self.\_data\_item. Инициализирует self.\_item\_next и self.\_item\_prev как None.
    \item Создайте класс DoublyLinkedList с методом \_\_init\_\_, который инициализирует self.\_head\_item и self.\_tail\_item как None.
    \item Создайте метод display в классе DoublyLinkedList, который выводит элементы в виде: "Элементы списка: val1 val2 val3 (всего N)". Если пуст — "Список пуст".
    \item Создайте метод insert в классе DoublyLinkedList, который принимает значение и вставляет его \textbf{только если оно не равно 0}. Вставляет в конец.
    \item Создайте метод delete в классе DoublyLinkedList, который принимает значение и удаляет \textbf{последнее вхождение}.
    \item Создайте экземпляр класса DoublyLinkedList.
    \item Вставьте узлы: 0 (нет), 10 (да), 0 (нет), 20 (да), 30 (да).
    \item Вызовите display.
    \item Вставьте 40 (да).
    \item Снова вызовите display.
    \item Удалите последнее вхождение 20.
    \item Снова вызовите display.
\end{enumerate}

Пример использования:
\begin{lstlisting}[language=Python]
dll = DoublyLinkedList()
dll.insert(0)   # нет
dll.insert(10)  # да
dll.insert(0)   # нет
dll.insert(20)  # да
dll.insert(30)  # да

print("Initial Doubly Linked List:")
dll.display()

dll.insert(40)
print("After inserting 40:")
dll.display()

dll.delete(20)
print("After deleting last occurrence of 20:")
dll.display()
\end{lstlisting}

\item Написать программу на Python, которая создает класс DoublyLinkedList, представляющий \textbf{двусвязный список} с инкапсуляцией. Класс должен содержать методы для отображения данных, вставки и удаления узлов. Программа также должна создавать экземпляр класса, вставлять узлы и удалять узлы.

Инструкции:
\begin{enumerate}
    \item Создайте класс Node с методом \_\_init\_\_, который принимает data и сохраняет его в self.\_list\_data. Инициализирует self.\_data\_next и self.\_data\_prev как None.
    \item Создайте класс DoublyLinkedList с методом \_\_init\_\_, который инициализирует self.\_first\_list и self.\_last\_list как None.
    \item Создайте метод display в классе DoublyLinkedList, который выводит элементы в виде: "Список: val1 | val2 | val3 | ...". Если пуст — "Пустой список".
    \item Создайте метод insert в классе DoublyLinkedList, который принимает значение и вставляет его \textbf{только если оно положительное}. Вставляет в начало.
    \item Создайте метод delete в классе DoublyLinkedList, который принимает значение и удаляет \textbf{все вхождения}.
    \item Создайте экземпляр класса DoublyLinkedList.
    \item Вставьте узлы: -5 (нет), 15 (да), -3 (нет), 25 (да), 0 (нет, если считать 0 не положительным).
    \item Вызовите display.
    \item Вставьте 35 (да).
    \item Снова вызовите display.
    \item Удалите все вхождения 25.
    \item Снова вызовите display.
\end{enumerate}

Пример использования:
\begin{lstlisting}[language=Python]
dll = DoublyLinkedList()
dll.insert(-5)  # нет
dll.insert(15)  # да
dll.insert(-3)  # нет
dll.insert(25)  # да
dll.insert(0)   # нет

print("Initial Doubly Linked List:")
dll.display()

dll.insert(35)
print("After inserting 35:")
dll.display()

dll.delete(25)
print("After deleting all 25s:")
dll.display()
\end{lstlisting}

\item Написать программу на Python, которая создает класс DoublyLinkedList, представляющий \textbf{двусвязный список} с инкапсуляцией. Класс должен содержать методы для отображения данных, вставки и удаления узлов. Программа также должна создавать экземпляр класса, вставлять узлы и удалять узлы.

Инструкции:
\begin{enumerate}
    \item Создайте класс Node с методом \_\_init\_\_, который принимает value и сохраняет его в self.\_entry\_value. Инициализирует self.\_value\_next и self.\_value\_prev как None.
    \item Создайте класс DoublyLinkedList с методом \_\_init\_\_, который инициализирует self.\_head\_value и self.\_tail\_value как None.
    \item Создайте метод display в классе DoublyLinkedList, который выводит элементы в виде: "Значения → val1 → val2 → val3 → конец". Если пуст — "→ конец".
    \item Создайте метод insert в классе DoublyLinkedList, который принимает значение и вставляет его \textbf{только если оно нечетное}. Вставляет в конец.
    \item Создайте метод delete в классе DoublyLinkedList, который принимает значение и удаляет \textbf{первое вхождение}.
    \item Создайте экземпляр класса DoublyLinkedList.
    \item Вставьте узлы: 2 (нет), 3 (да), 4 (нет), 5 (да), 6 (нет), 7 (да).
    \item Вызовите display.
    \item Вставьте 9 (да).
    \item Снова вызовите display.
    \item Удалите 5.
    \item Снова вызовите display.
\end{enumerate}

Пример использования:
\begin{lstlisting}[language=Python]
dll = DoublyLinkedList()
dll.insert(2)  # нет
dll.insert(3)  # да
dll.insert(4)  # нет
dll.insert(5)  # да
dll.insert(6)  # нет
dll.insert(7)  # да

print("Initial Doubly Linked List:")
dll.display()

dll.insert(9)
print("After inserting 9:")
dll.display()

dll.delete(5)
print("After deleting 5:")
dll.display()
\end{lstlisting}

\item Написать программу на Python, которая создает класс DoublyLinkedList, представляющий \textbf{двусвязный список} с инкапсуляцией. Класс должен содержать методы для отображения данных, вставки и удаления узлов. Программа также должна создавать экземпляр класса, вставлять узлы и удалять узлы.

Инструкции:
\begin{enumerate}
    \item Создайте класс Node с методом \_\_init\_\_, который принимает item и сохраняет его в self.\_data\_point. Инициализирует self.\_point\_next и self.\_point\_prev как None.
    \item Создайте класс DoublyLinkedList с методом \_\_init\_\_, который инициализирует self.\_start\_point и self.\_end\_point как None.
    \item Создайте метод display в классе DoublyLinkedList, который выводит элементы в виде: "Точки: val1, val2, val3 (обратно: val3, val2, val1)". Если пуст — "Точек нет".
    \item Создайте метод insert в классе DoublyLinkedList, который принимает значение и вставляет его \textbf{только если оно четное}. Вставляет в начало.
    \item Создайте метод delete в классе DoublyLinkedList, который принимает значение и удаляет \textbf{последнее вхождение}.
    \item Создайте экземпляр класса DoublyLinkedList.
    \item Вставьте узлы: 1 (нет), 4 (да), 3 (нет), 6 (да), 5 (нет), 8 (да).
    \item Вызовите display.
    \item Вставьте 10 (да).
    \item Снова вызовите display.
    \item Удалите последнее вхождение 6.
    \item Снова вызовите display.
\end{enumerate}

Пример использования:
\begin{lstlisting}[language=Python]
dll = DoublyLinkedList()
dll.insert(1)  # нет
dll.insert(4)  # да
dll.insert(3)  # нет
dll.insert(6)  # да
dll.insert(5)  # нет
dll.insert(8)  # да

print("Initial Doubly Linked List:")
dll.display()

dll.insert(10)
print("After inserting 10:")
dll.display()

dll.delete(6)
print("After deleting last occurrence of 6:")
dll.display()
\end{lstlisting}

\item Написать программу на Python, которая создает класс DoublyLinkedList, представляющий \textbf{двусвязный список} с инкапсуляцией. Класс должен содержать методы для отображения данных, вставки и удаления узлов. Программа также должна создавать экземпляр класса, вставлять узлы и удалять узлы.

Инструкции:
\begin{enumerate}
    \item Создайте класс Node с методом \_\_init\_\_, который принимает data и сохраняет его в self.\_node\_data. Инициализирует self.\_data\_link\_next и self.\_data\_link\_prev как None.
    \item Создайте класс DoublyLinkedList с методом \_\_init\_\_, который инициализирует self.\_first\_link и self.\_last\_link как None.
    \item Создайте метод display в классе DoublyLinkedList, который выводит элементы в виде: "Связи: val1 <-> val2 <-> val3". Если пуст — "Связи отсутствуют".
    \item Создайте метод insert в классе DoublyLinkedList, который принимает значение и вставляет его \textbf{только если оно кратно 4}. Вставляет в конец.
    \item Создайте метод delete в классе DoublyLinkedList, который принимает значение и удаляет \textbf{все вхождения}.
    \item Создайте экземпляр класса DoublyLinkedList.
    \item Вставьте узлы: 2 (нет), 4 (да), 6 (нет), 8 (да), 10 (нет), 12 (да).
    \item Вызовите display.
    \item Вставьте 16 (да).
    \item Снова вызовите display.
    \item Удалите все вхождения 8.
    \item Снова вызовите display.
\end{enumerate}

Пример использования:
\begin{lstlisting}[language=Python]
dll = DoublyLinkedList()
dll.insert(2)   # нет
dll.insert(4)   # да
dll.insert(6)   # нет
dll.insert(8)   # да
dll.insert(10)  # нет
dll.insert(12)  # да

print("Initial Doubly Linked List:")
dll.display()

dll.insert(16)
print("After inserting 16:")
dll.display()

dll.delete(8)
print("After deleting all 8s:")
dll.display()
\end{lstlisting}

\item Написать программу на Python, которая создает класс DoublyLinkedList, представляющий \textbf{двусвязный список} с инкапсуляцией. Класс должен содержать методы для отображения данных, вставки и удаления узлов. Программа также должна создавать экземпляр класса, вставлять узлы и удалять узлы.

Инструкции:
\begin{enumerate}
    \item Создайте класс Node с методом \_\_init\_\_, который принимает value и сохраняет его в self.\_item\_val. Инициализирует self.\_val\_next и self.\_val\_prev как None.
    \item Создайте класс DoublyLinkedList с методом \_\_init\_\_, который инициализирует self.\_head\_val и self.\_tail\_val как None.
    \item Создайте метод display в классе DoublyLinkedList, который выводит элементы в виде: "Значения: val1 - val2 - val3 (размер N)". Если пуст — "Нет значений".
    \item Создайте метод insert в классе DoublyLinkedList, который принимает значение и вставляет его \textbf{только если оно заканчивается на 5}. Вставляет в начало.
    \item Создайте метод delete в классе DoublyLinkedList, который принимает значение и удаляет \textbf{первое вхождение}.
    \item Создайте экземпляр класса DoublyLinkedList.
    \item Вставьте узлы: 10 (нет), 15 (да), 20 (нет), 25 (да), 30 (нет), 35 (да).
    \item Вызовите display.
    \item Вставьте 45 (да).
    \item Снова вызовите display.
    \item Удалите 25.
    \item Снова вызовите display.
\end{enumerate}

Пример использования:
\begin{lstlisting}[language=Python]
dll = DoublyLinkedList()
dll.insert(10)  # нет
dll.insert(15)  # да
dll.insert(20)  # нет
dll.insert(25)  # да
dll.insert(30)  # нет
dll.insert(35)  # да

print("Initial Doubly Linked List:")
dll.display()

dll.insert(45)
print("After inserting 45:")
dll.display()

dll.delete(25)
print("After deleting 25:")
dll.display()
\end{lstlisting}

\item Написать программу на Python, которая создает класс DoublyLinkedList, представляющий \textbf{двусвязный список} с инкапсуляцией. Класс должен содержать методы для отображения данных, вставки и удаления узлов. Программа также должна создавать экземпляр класса, вставлять узлы и удалять узлы.

Инструкции:
\begin{enumerate}
    \item Создайте класс Node с методом \_\_init\_\_, который принимает item и сохраняет его в self.\_data\_field. Инициализирует self.\_field\_next и self.\_field\_prev как None.
    \item Создайте класс DoublyLinkedList с методом \_\_init\_\_, который инициализирует self.\_first\_field и self.\_last\_field как None.
    \item Создайте метод display в классе DoublyLinkedList, который выводит элементы в виде: "Поля: val1 → val2 → val3 → null". Если пуст — "null".
    \item Создайте метод insert в классе DoublyLinkedList, который принимает значение и вставляет его \textbf{только если первая цифра числа — 1}. Вставляет в конец.
    \item Создайте метод delete в классе DoublyLinkedList, который принимает значение и удаляет \textbf{последнее вхождение}.
    \item Создайте экземпляр класса DoublyLinkedList.
    \item Вставьте узлы: 5 (нет), 12 (да), 23 (нет), 18 (да), 31 (нет), 19 (да).
    \item Вызовите display.
    \item Вставьте 11 (да).
    \item Снова вызовите display.
    \item Удалите последнее вхождение 18.
    \item Снова вызовите display.
\end{enumerate}

Пример использования:
\begin{lstlisting}[language=Python]
dll = DoublyLinkedList()
dll.insert(5)   # нет
dll.insert(12)  # да
dll.insert(23)  # нет
dll.insert(18)  # да
dll.insert(31)  # нет
dll.insert(19)  # да

print("Initial Doubly Linked List:")
dll.display()

dll.insert(11)
print("After inserting 11:")
dll.display()

dll.delete(18)
print("After deleting last occurrence of 18:")
dll.display()
\end{lstlisting}

\item Написать программу на Python, которая создает класс DoublyLinkedList, представляющий \textbf{двусвязный список} с инкапсуляцией. Класс должен содержать методы для отображения данных, вставки и удаления узлов. Программа также должна создавать экземпляр класса, вставлять узлы и удалять узлы.

Инструкции:
\begin{enumerate}
    \item Создайте класс Node с методом \_\_init\_\_, который принимает data и сохраняет его в self.\_record\_data. Инициализирует self.\_data\_record\_next и self.\_data\_record\_prev как None.
    \item Создайте класс DoublyLinkedList с методом \_\_init\_\_, который инициализирует self.\_head\_record и self.\_tail\_record как None.
    \item Создайте метод display в классе DoublyLinkedList, который выводит элементы в виде: "Записи: [val1, val2, val3]". Если пуст — "[]".
    \item Создайте метод insert в классе DoublyLinkedList, который принимает значение и вставляет его \textbf{только если оно начинается с цифры 2}. Вставляет в начало.
    \item Создайте метод delete в классе DoublyLinkedList, который принимает значение и удаляет \textbf{все вхождения}.
    \item Создайте экземпляр класса DoublyLinkedList.
    \item Вставьте узлы: 15 (нет), 25 (да), 35 (нет), 28 (да), 45 (нет), 22 (да).
    \item Вызовите display.
    \item Вставьте 20 (да).
    \item Снова вызовите display.
    \item Удалите все вхождения 28.
    \item Снова вызовите display.
\end{enumerate}

Пример использования:
\begin{lstlisting}[language=Python]
dll = DoublyLinkedList()
dll.insert(15)  # нет
dll.insert(25)  # да
dll.insert(35)  # нет
dll.insert(28)  # да
dll.insert(45)  # нет
dll.insert(22)  # да

print("Initial Doubly Linked List:")
dll.display()

dll.insert(20)
print("After inserting 20:")
dll.display()

dll.delete(28)
print("After deleting all 28s:")
dll.display()
\end{lstlisting}

\item Написать программу на Python, которая создает класс DoublyLinkedList, представляющий \textbf{двусвязный список} с инкапсуляцией. Класс должен содержать методы для отображения данных, вставки и удаления узлов. Программа также должна создавать экземпляр класса, вставлять узлы и удалять узлы.

Инструкции:
\begin{enumerate}
    \item Создайте класс Node с методом \_\_init\_\_, который принимает value и сохраняет его в self.\_cell\_value. Инициализирует self.\_value\_cell\_next и self.\_value\_cell\_prev как None.
    \item Создайте класс DoublyLinkedList с методом \_\_init\_\_, который инициализирует self.\_first\_cell и self.\_last\_cell как None.
    \item Создайте метод display в классе DoublyLinkedList, который выводит элементы в виде: "Ячейки: val1 | val2 | val3 (всего N)". Если пуст — "Нет ячеек".
    \item Создайте метод insert в классе DoublyLinkedList, который принимает значение и вставляет его \textbf{только если сумма его цифр нечетная}. Вставляет в конец.
    \item Создайте метод delete в классе DoublyLinkedList, который принимает значение и удаляет \textbf{первое вхождение}.
    \item Создайте экземпляр класса DoublyLinkedList.
    \item Вставьте узлы: 12 (1+2=3 — нечет, да), 14 (1+4=5 — нечет, да), 16 (1+6=7 — нечет, да), 18 (1+8=9 — нечет, да), 20 (2+0=2 — чет, нет).
    \item Вызовите display.
    \item Вставьте 21 (2+1=3 — нечет, да).
    \item Снова вызовите display.
    \item Удалите 16.
    \item Снова вызовите display.
\end{enumerate}

Пример использования:
\begin{lstlisting}[language=Python]
dll = DoublyLinkedList()
dll.insert(12)  # да
dll.insert(14)  # да
dll.insert(16)  # да
dll.insert(18)  # да
dll.insert(20)  # нет

print("Initial Doubly Linked List:")
dll.display()

dll.insert(21)
print("After inserting 21:")
dll.display()

dll.delete(16)
print("After deleting 16:")
dll.display()
\end{lstlisting}

\item Написать программу на Python, которая создает класс DoublyLinkedList, представляющий \textbf{двусвязный список} с инкапсуляцией. Класс должен содержать методы для отображения данных, вставки и удаления узлов. Программа также должна создавать экземпляр класса, вставлять узлы и удалять узлы.

Инструкции:
\begin{enumerate}
    \item Создайте класс Node с методом \_\_init\_\_, который принимает item и сохраняет его в self.\_slot\_data. Инициализирует self.\_data\_slot\_next и self.\_data\_slot\_prev как None.
    \item Создайте класс DoublyLinkedList с методом \_\_init\_\_, который инициализирует self.\_head\_slot и self.\_tail\_slot как None.
    \item Создайте метод display в классе DoublyLinkedList, который выводит элементы в виде: "Слоты → val1 → val2 → val3 → конец". Если пуст — "→ конец".
    \item Создайте метод insert в классе DoublyLinkedList, который принимает значение и вставляет его \textbf{только если оно заканчивается на 0}. Вставляет в начало.
    \item Создайте метод delete в классе DoublyLinkedList, который принимает значение и удаляет \textbf{последнее вхождение}.
    \item Создайте экземпляр класса DoublyLinkedList.
    \item Вставьте узлы: 5 (нет), 10 (да), 15 (нет), 20 (да), 25 (нет), 30 (да).
    \item Вызовите display.
    \item Вставьте 40 (да).
    \item Снова вызовите display.
    \item Удалите последнее вхождение 20.
    \item Снова вызовите display.
\end{enumerate}

Пример использования:
\begin{lstlisting}[language=Python]
dll = DoublyLinkedList()
dll.insert(5)   # нет
dll.insert(10)  # да
dll.insert(15)  # нет
dll.insert(20)  # да
dll.insert(25)  # нет
dll.insert(30)  # да

print("Initial Doubly Linked List:")
dll.display()

dll.insert(40)
print("After inserting 40:")
dll.display()

dll.delete(20)
print("After deleting last occurrence of 20:")
dll.display()
\end{lstlisting}

\item Написать программу на Python, которая создает класс DoublyLinkedList, представляющий \textbf{двусвязный список} с инкапсуляцией. Класс должен содержать методы для отображения данных, вставки и удаления узлов. Программа также должна создавать экземпляр класса, вставлять узлы и удалять узлы.

Инструкции:
\begin{enumerate}
    \item Создайте класс Node с методом \_\_init\_\_, который принимает data и сохраняет его в self.\_block\_data. Инициализирует self.\_data\_block\_next и self.\_data\_block\_prev как None.
    \item Создайте класс DoublyLinkedList с методом \_\_init\_\_, который инициализирует self.\_first\_block и self.\_last\_block как None.
    \item Создайте метод display в классе DoublyLinkedList, который выводит элементы в виде: "Блоки: val1, val2, val3 (обратный: val3, val2, val1)". Если пуст — "Пусто".
    \item Создайте метод insert в классе DoublyLinkedList, который принимает значение и вставляет его \textbf{только если оно простое и больше 10}. Вставляет в конец.
    \item Создайте метод delete в классе DoublyLinkedList, который принимает значение и удаляет \textbf{все вхождения}.
    \item Создайте экземпляр класса DoublyLinkedList.
    \item Вставьте узлы: 7 (простое, но <=10 — нет), 11 (да), 13 (да), 15 (нет), 17 (да), 9 (нет).
    \item Вызовите display.
    \item Вставьте 19 (да).
    \item Снова вызовите display.
    \item Удалите все вхождения 13.
    \item Снова вызовите display.
\end{enumerate}

Пример использования:
\begin{lstlisting}[language=Python]
def is_prime(n):
    if n < 2:
        return False
    for i in range(2, int(n**0.5)+1):
        if n % i == 0:
            return False
    return True

dll = DoublyLinkedList()
dll.insert(7)   # нет
dll.insert(11)  # да
dll.insert(13)  # да
dll.insert(15)  # нет
dll.insert(17)  # да
dll.insert(9)   # нет

print("Initial Doubly Linked List:")
dll.display()

dll.insert(19)
print("After inserting 19:")
dll.display()

dll.delete(13)
print("After deleting all 13s:")
dll.display()
\end{lstlisting}

\item Написать программу на Python, которая создает класс DoublyLinkedList, представляющий \textbf{двусвязный список} с инкапсуляцией. Класс должен содержать методы для отображения данных, вставки и удаления узлов. Программа также должна создавать экземпляр класса, вставлять узлы и удалять узлы.

Инструкции:
\begin{enumerate}
    \item Создайте класс Node с методом \_\_init\_\_, который принимает value и сохраняет его в self.\_unit\_value. Инициализирует self.\_value\_unit\_next и self.\_value\_unit\_prev как None.
    \item Создайте класс DoublyLinkedList с методом \_\_init\_\_, который инициализирует self.\_head\_unit и self.\_tail\_unit как None.
    \item Создайте метод display в классе DoublyLinkedList, который выводит элементы в виде: "Единицы: val1 <-> val2 <-> val3". Если пуст — "Нет данных".
    \item Создайте метод insert в классе DoublyLinkedList, который принимает значение и вставляет его \textbf{только если оно палиндром и двузначное}. Вставляет в начало.
    \item Создайте метод delete в классе DoublyLinkedList, который принимает значение и удаляет \textbf{первое вхождение}.
    \item Создайте экземпляр класса DoublyLinkedList.
    \item Вставьте узлы: 121 (трехзначное — нет), 22 (да), 34 (нет), 55 (да), 5 (однозначное — нет), 66 (да).
    \item Вызовите display.
    \item Вставьте 77 (да).
    \item Снова вызовите display.
    \item Удалите 55.
    \item Снова вызовите display.
\end{enumerate}

Пример использования:
\begin{lstlisting}[language=Python]
dll = DoublyLinkedList()
dll.insert(121)  # нет
dll.insert(22)   # да
dll.insert(34)   # нет
dll.insert(55)   # да
dll.insert(5)    # нет
dll.insert(66)   # да

print("Initial Doubly Linked List:")
dll.display()

dll.insert(77)
print("After inserting 77:")
dll.display()

dll.delete(55)
print("After deleting 55:")
dll.display()
\end{lstlisting}

\end{enumerate}

\subsubsection{Задача 4 (очередь)}

\begin{enumerate}
\item Написать программу на Python, которая создает класс Queue для представления структуры данных очереди с инкапсуляцией. Класс должен содержать методы enqueue, dequeue и is\_empty, которые реализуют операции добавления элементов в очередь, удаления элементов из очереди и проверки пустоты очереди соответственно. Программа также должна создавать экземпляр класса Queue, добавлять элементы в очередь, удалять элементы из очереди и выводить информацию о состоянии очереди на экран.

Инструкции:
\begin{enumerate}
    \item Создайте класс Queue с методом \_\_init\_\_, который инициализирует пустую очередь (внутренний список \_elements). Принимает необязательный параметр max\_size (по умолчанию None — без ограничений).
    \item Создайте метод enqueue, который принимает элемент и добавляет его в конец очереди, только если не превышает max\_size. Если превышает — выбрасывает ValueError("Очередь переполнена").
    \item Создайте метод dequeue, который удаляет и возвращает элемент из начала очереди. Если очередь пуста — выбрасывает IndexError("Очередь пуста").
    \item Создайте метод is\_empty, который возвращает True, если очередь пуста, и False в противном случае.
    \item Создайте приватный метод \_debug\_list (только для отладки, не включайте в задание студентам; в решении можно использовать queue.\_elements) для вывода внутреннего состояния.
    \item Создайте экземпляр класса Queue с max\_size=5.
    \item Добавьте элементы: 100, 200, 300, 400, 500.
    \item Попытайтесь добавить 600 — должно вызвать исключение (перехватите его и выведите сообщение).
    \item Выведите текущее состояние очереди.
    \item Вызовите dequeue дважды, выводя каждый раз удаленный элемент.
    \item Выведите обновленное состояние очереди.
\end{enumerate}

Пример использования:
\begin{lstlisting}[language=Python]
queue = Queue(max_size=5)
queue.enqueue(100)
queue.enqueue(200)
queue.enqueue(300)
queue.enqueue(400)
queue.enqueue(500)

try:
    queue.enqueue(600)
except ValueError as e:
    print("Ошибка:", e)

print("Current Queue:", queue._elements)  # только для проверки

dequeued_item = queue.dequeue()
print("Dequeued item:", dequeued_item)

dequeued_item = queue.dequeue()
print("Dequeued item:", dequeued_item)

print("Updated Queue:", queue._elements)
\end{lstlisting}

\item Написать программу на Python, которая создает класс Queue для представления структуры данных очереди с инкапсуляцией. Класс должен содержать методы enqueue, dequeue и is\_empty, которые реализуют операции добавления элементов в очередь, удаления элементов из очереди и проверки пустоты очереди соответственно. Программа также должна создавать экземпляр класса Queue, добавлять элементы в очередь, удалять элементы из очереди и выводить информацию о состоянии очереди на экран.

Инструкции:
\begin{enumerate}
    \item Создайте класс Queue с методом \_\_init\_\_, который инициализирует пустую очередь (список \_items). Принимает параметр allow\_duplicates=True. Если False, то не добавляет элемент, если он уже есть в очереди.
    \item Создайте метод enqueue, который принимает элемент. Если allow\_duplicates=False и элемент уже есть в очереди — не добавляет и возвращает False. Иначе — добавляет в конец и возвращает True.
    \item Создайте метод dequeue, который удаляет и возвращает первый элемент. Если очередь пуста — возвращает None (не выбрасывает исключение).
    \item Создайте метод is\_empty, который возвращает True, если очередь пуста, и False в противном случае.
    \item Создайте экземпляр класса Queue с allow\_duplicates=False.
    \item Добавьте элементы: 10, 20, 10 (не добавится), 30, 20 (не добавится), 40.
    \item Выведите текущее состояние очереди.
    \item Вызовите dequeue трижды, выводя каждый раз удаленный элемент.
    \item Выведите обновленное состояние очереди.
\end{enumerate}

Пример использования:
\begin{lstlisting}[language=Python]
queue = Queue(allow_duplicates=False)
print(queue.enqueue(10))  # True
print(queue.enqueue(20))  # True
print(queue.enqueue(10))  # False
print(queue.enqueue(30))  # True
print(queue.enqueue(20))  # False
print(queue.enqueue(40))  # True

print("Current Queue:", queue._items)

for _ in range(3):
    dequeued_item = queue.dequeue()
    print("Dequeued item:", dequeued_item)

print("Updated Queue:", queue._items)
\end{lstlisting}

\item Написать программу на Python, которая создает класс Queue для представления структуры данных очереди с инкапсуляцией. Класс должен содержать методы enqueue, dequeue и is\_empty, которые реализуют операции добавления элементов в очередь, удаления элементов из очереди и проверки пустоты очереди соответственно. Программа также должна создавать экземпляр класса Queue, добавлять элементы в очередь, удалять элементы из очереди и выводить информацию о состоянии очереди на экран.

Инструкции:
\begin{enumerate}
    \item Создайте класс Queue с методом \_\_init\_\_, который инициализирует пустую очередь (список \_data). Принимает параметр auto\_reverse=False. Если True, то enqueue добавляет в начало, а dequeue удаляет с конца (поведение стека, но интерфейс очереди).
    \item Создайте метод enqueue, который добавляет элемент: если auto\_reverse=False — в конец, если True — в начало.
    \item Создайте метод dequeue, который удаляет и возвращает элемент: если auto\_reverse=False — из начала, если True — из конца. Если очередь пуста — выбрасывает IndexError("Пусто").
    \item Создайте метод is\_empty, который возвращает True, если очередь пуста, и False в противном случае.
    \item Создайте экземпляр класса Queue с auto\_reverse=True.
    \item Добавьте элементы: 1, 2, 3, 4, 5.
    \item Выведите текущее состояние очереди.
    \item Вызовите dequeue дважды, выводя каждый раз удаленный элемент.
    \item Выведите обновленное состояние очереди.
\end{enumerate}

Пример использования:
\begin{lstlisting}[language=Python]
queue = Queue(auto_reverse=True)
queue.enqueue(1)
queue.enqueue(2)
queue.enqueue(3)
queue.enqueue(4)
queue.enqueue(5)

print("Current Queue:", queue._data)  # [5,4,3,2,1]

dequeued_item = queue.dequeue()  # удаляет 1
print("Dequeued item:", dequeued_item)

dequeued_item = queue.dequeue()  # удаляет 2
print("Dequeued item:", dequeued_item)

print("Updated Queue:", queue._data)  # [5,4,3]
\end{lstlisting}

\item Написать программу на Python, которая создает класс Queue для представления структуры данных очереди с инкапсуляцией. Класс должен содержать методы enqueue, dequeue и is\_empty, которые реализуют операции добавления элементов в очередь, удаления элементов из очереди и проверки пустоты очереди соответственно. Программа также должна создавать экземпляр класса Queue, добавлять элементы в очередь, удалять элементы из очереди и выводить информацию о состоянии очереди на экран.

Инструкции:
\begin{enumerate}
    \item Создайте класс Queue с методом \_\_init\_\_, который инициализирует пустую очередь (список \_buffer). Принимает параметр dequeue\_all\_at\_once=False. Если True, то dequeue возвращает список всех элементов и очищает очередь.
    \item Создайте метод enqueue, который добавляет элемент в конец очереди.
    \item Создайте метод dequeue, который, если dequeue\_all\_at\_once=False, удаляет и возвращает первый элемент. Если True — возвращает список всех элементов и очищает очередь. Если очередь пуста — возвращает пустой список [].
    \item Создайте метод is\_empty, который возвращает True, если очередь пуста, и False в противном случае.
    \item Создайте экземпляр класса Queue с dequeue\_all\_at\_once=True.
    \item Добавьте элементы: 5, 15, 25, 35.
    \item Выведите текущее состояние очереди.
    \item Вызовите dequeue (вернет [5,15,25,35] и очистит очередь).
    \item Выведите результат dequeue и состояние очереди после вызова.
\end{enumerate}

Пример использования:
\begin{lstlisting}[language=Python]
queue = Queue(dequeue_all_at_once=True)
queue.enqueue(5)
queue.enqueue(15)
queue.enqueue(25)
queue.enqueue(35)

print("Current Queue:", queue._buffer)

dequeued_items = queue.dequeue()
print("Dequeued items:", dequeued_items)  # [5,15,25,35]
print("Updated Queue:", queue._buffer)    # []
\end{lstlisting}

\item Написать программу на Python, которая создает класс Queue для представления структуры данных очереди с инкапсуляцией. Класс должен содержать методы enqueue, dequeue и is\_empty, которые реализуют операции добавления элементов в очередь, удаления элементов из очереди и проверки пустоты очереди соответственно. Программа также должна создавать экземпляр класса Queue, добавлять элементы в очередь, удалять элементы из очереди и выводить информацию о состоянии очереди на экран.

Инструкции:
\begin{enumerate}
    \item Создайте класс Queue с методом \_\_init\_\_, который инициализирует пустую очередь (список \_store). Принимает параметр on\_enqueue\_callback=None — функция, вызываемая при каждом добавлении (с аргументом — добавленным элементом).
    \item Создайте метод enqueue, который добавляет элемент в конец и, если on\_enqueue\_callback не None, вызывает её с элементом.
    \item Создайте метод dequeue, который удаляет и возвращает первый элемент. Если очередь пуста — выбрасывает IndexError("Нельзя извлечь из пустой очереди").
    \item Создайте метод is\_empty, который возвращает True, если очередь пуста, и False в противном случае.
    \item Создайте функцию printer(x): print(f"[+] Добавлен: {x}")
    \item Создайте экземпляр класса Queue, передав printer в on\_enqueue\_callback.
    \item Добавьте элементы: 101, 202, 303.
    \item Выведите текущее состояние очереди.
    \item Вызовите dequeue, выведите удаленный элемент.
    \item Выведите обновленное состояние очереди.
\end{enumerate}

Пример использования:
\begin{lstlisting}[language=Python]
def printer(x):
    print(f"[+] Добавлен: {x}")

queue = Queue(on_enqueue_callback=printer)
queue.enqueue(101)  # [+] Добавлен: 101
queue.enqueue(202)  # [+] Добавлен: 202
queue.enqueue(303)  # [+] Добавлен: 303

print("Current Queue:", queue._store)

dequeued_item = queue.dequeue()
print("Dequeued item:", dequeued_item)

print("Updated Queue:", queue._store)
\end{lstlisting}

\item Написать программу на Python, которая создает класс Queue для представления структуры данных очереди с инкапсуляцией. Класс должен содержать методы enqueue, dequeue и is\_empty, которые реализуют операции добавления элементов в очередь, удаления элементов из очереди и проверки пустоты очереди соответственно. Программа также должна создавать экземпляр класса Queue, добавлять элементы в очередь, удалять элементы из очереди и выводить информацию о состоянии очереди на экран.

Инструкции:
\begin{enumerate}
    \item Создайте класс Queue с методом \_\_init\_\_, который инициализирует пустую очередь (список \_pool). Принимает параметр compress\_on\_enqueue=False. Если True, то при добавлении элемента, равного последнему в очереди, вместо добавления увеличивает счетчик дубликатов у последнего элемента (хранит пары (элемент, счетчик)).
    \item Создайте метод enqueue, который, если compress\_on\_enqueue=True и очередь не пуста и элемент == последний\_элемент, увеличивает счетчик последнего элемента. Иначе — добавляет новый элемент (со счетчиком 1, если режим сжатия включен).
    \item Создайте метод dequeue, который удаляет первый элемент. Если режим сжатия включен и счетчик >1, уменьшает счетчик и возвращает элемент. Если счетчик=1, удаляет элемент. Если очередь пуста — выбрасывает IndexError("Очередь пуста").
    \item Создайте метод is\_empty, который возвращает True, если очередь пуста, и False в противном случае.
    \item Создайте экземпляр класса Queue с compress\_on\_enqueue=True.
    \item Добавьте элементы: 7, 7, 7, 14, 14, 21.
    \item Выведите текущее состояние очереди (внутреннее представление).
    \item Вызовите dequeue трижды, выводя каждый раз удаленный элемент.
    \item Выведите обновленное состояние очереди.
\end{enumerate}

Пример использования:
\begin{lstlisting}[language=Python]
queue = Queue(compress_on_enqueue=True)
queue.enqueue(7)
queue.enqueue(7)
queue.enqueue(7)
queue.enqueue(14)
queue.enqueue(14)
queue.enqueue(21)

print("Current Queue:", queue._pool)  # [(7,3), (14,2), (21,1)]

for _ in range(3):
    dequeued_item = queue.dequeue()
    print("Dequeued item:", dequeued_item)  # 7, 7, 7

print("Updated Queue:", queue._pool)  # [(14,2), (21,1)]
\end{lstlisting}

\item Написать программу на Python, которая создает класс Queue для представления структуры данных очереди с инкапсуляцией. Класс должен содержать методы enqueue, dequeue и is\_empty, которые реализуют операции добавления элементов в очередь, удаления элементов из очереди и проверки пустоты очереди соответственно. Программа также должна создавать экземпляр класса Queue, добавлять элементы в очередь, удалять элементы из очереди и выводить информацию о состоянии очереди на экран.

Инструкции:
\begin{enumerate}
    \item Создайте класс Queue с методом \_\_init\_\_, который инициализирует пустую очередь (список \_line). Принимает параметр immutable\_dequeue=False. Если True, то dequeue возвращает первый элемент, но не удаляет его.
    \item Создайте метод enqueue, который добавляет элемент в конец очереди.
    \item Создайте метод dequeue, который, если immutable\_dequeue=False, удаляет и возвращает первый элемент. Если True — возвращает первый элемент, не удаляя его. Если очередь пуста — возвращает None.
    \item Создайте метод is\_empty, который возвращает True, если очередь пуста, и False в противном случае.
    \item Создайте экземпляр класса Queue с immutable\_dequeue=True.
    \item Добавьте элементы: 1, 3, 5.
    \item Выведите текущее состояние очереди.
    \item Вызовите dequeue дважды, выводя каждый раз результат (должен быть 1 оба раза).
    \item Выведите состояние очереди (не должно измениться).
\end{enumerate}

Пример использования:
\begin{lstlisting}[language=Python]
queue = Queue(immutable_dequeue=True)
queue.enqueue(1)
queue.enqueue(3)
queue.enqueue(5)

print("Current Queue:", queue._line)

print("Dequeued item:", queue.dequeue())  # 1
print("Dequeued item:", queue.dequeue())  # 1 (не удалилось)

print("Updated Queue:", queue._line)  # [1,3,5]
\end{lstlisting}

\item Написать программу на Python, которая создает класс Queue для представления структуры данных очереди с инкапсуляцией. Класс должен содержать методы enqueue, dequeue и is\_empty, которые реализуют операции добавления элементов в очередь, удаления элементов из очереди и проверки пустоты очереди соответственно. Программа также должна создавать экземпляр класса Queue, добавлять элементы в очередь, удалять элементы из очереди и выводить информацию о состоянии очереди на экран.

Инструкции:
\begin{enumerate}
    \item Создайте класс Queue с методом \_\_init\_\_, который инициализирует пустую очередь (список \_stream). Принимает параметр track\_history=False. Если True, сохраняет историю всех когда-либо добавленных элементов (даже удаленных) в отдельном списке \_history.
    \item Создайте метод enqueue, который добавляет элемент в конец \_stream и, если track\_history=True, добавляет его в \_history.
    \item Создайте метод dequeue, который удаляет и возвращает первый элемент из \_stream. Если очередь пуста — выбрасывает IndexError("Пусто").
    \item Создайте метод is\_empty, который возвращает True, если \_stream пуст, и False в противном случае.
    \item Создайте метод get\_history (только если track\_history=True), возвращающий копию \_history.
    \item Создайте экземпляр класса Queue с track\_history=True.
    \item Добавьте элементы: 2, 4, 6.
    \item Вызовите dequeue (вернет 2).
    \item Добавьте 8.
    \item Выведите текущую очередь и историю.
\end{enumerate}

Пример использования:
\begin{lstlisting}[language=Python]
queue = Queue(track_history=True)
queue.enqueue(2)
queue.enqueue(4)
queue.enqueue(6)
queue.dequeue()  # 2
queue.enqueue(8)

print("Current Queue:", queue._stream)    # [4,6,8]
print("History:", queue.get_history())    # [2,4,6,8]
\end{lstlisting}

\item Написать программу на Python, которая создает класс Queue для представления структуры данных очереди с инкапсуляцией. Класс должен содержать методы enqueue, dequeue и is\_empty, которые реализуют операции добавления элементов в очередь, удаления элементов из очереди и проверки пустоты очереди соответственно. Программа также должна создавать экземпляр класса Queue, добавлять элементы в очередь, удалять элементы из очереди и выводить информацию о состоянии очереди на экран.

Инструкции:
\begin{enumerate}
    \item Создайте класс Queue с методом \_\_init\_\_, который инициализирует пустую очередь (список \_flow). Принимает параметр enqueue\_only\_even=False. Если True, то добавляются только четные числа.
    \item Создайте метод enqueue, который добавляет элемент в конец, только если enqueue\_only\_even=False или элемент четный.
    \item Создайте метод dequeue, который удаляет и возвращает первый элемент. Если очередь пуста — выбрасывает IndexError("Очередь пуста").
    \item Создайте метод is\_empty, который возвращает True, если очередь пуста, и False в противном случае.
    \item Создайте экземпляр класса Queue с enqueue\_only\_even=True.
    \item Добавьте элементы: 1 (игнорируется), 2, 3 (игнорируется), 4, 5 (игнорируется), 6.
    \item Выведите текущее состояние очереди.
    \item Вызовите dequeue, выведите удаленный элемент.
    \item Выведите обновленное состояние очереди.
\end{enumerate}

Пример использования:
\begin{lstlisting}[language=Python]
queue = Queue(enqueue_only_even=True)
queue.enqueue(1)  # игнорируется
queue.enqueue(2)
queue.enqueue(3)  # игнорируется
queue.enqueue(4)
queue.enqueue(5)  # игнорируется
queue.enqueue(6)

print("Current Queue:", queue._flow)  # [2,4,6]

dequeued_item = queue.dequeue()
print("Dequeued item:", dequeued_item)  # 2

print("Updated Queue:", queue._flow)  # [4,6]
\end{lstlisting}

\item Написать программу на Python, которая создает класс Queue для представления структуры данных очереди с инкапсуляцией. Класс должен содержать методы enqueue, dequeue и is\_empty, которые реализуют операции добавления элементов в очередь, удаления элементов из очереди и проверки пустоты очереди соответственно. Программа также должна создавать экземпляр класса Queue, добавлять элементы в очередь, удалять элементы из очереди и выводить информацию о состоянии очереди на экран.

Инструкции:
\begin{enumerate}
    \item Создайте класс Queue с методом \_\_init\_\_, который инициализирует пустую очередь (список \_pipe). Принимает параметр reverse\_dequeue=False. Если True, то dequeue удаляет и возвращает не первый, а последний элемент.
    \item Создайте метод enqueue, который добавляет элемент в конец очереди.
    \item Создайте метод dequeue, который, если reverse\_dequeue=False, удаляет и возвращает первый элемент. Если True — удаляет и возвращает последний элемент. Если очередь пуста — выбрасывает IndexError("Пусто").
    \item Создайте метод is\_empty, который возвращает True, если очередь пуста, и False в противном случае.
    \item Создайте экземпляр класса Queue с reverse\_dequeue=True.
    \item Добавьте элементы: 10, 20, 30.
    \item Выведите текущее состояние очереди.
    \item Вызовите dequeue — должен вернуться 30 (последний).
    \item Выведите обновленное состояние очереди.
\end{enumerate}

Пример использования:
\begin{lstlisting}[language=Python]
queue = Queue(reverse_dequeue=True)
queue.enqueue(10)
queue.enqueue(20)
queue.enqueue(30)

print("Current Queue:", queue._pipe)  # [10,20,30]

dequeued_item = queue.dequeue()  # 30
print("Dequeued item:", dequeued_item)

print("Updated Queue:", queue._pipe)  # [10,20]
\end{lstlisting}

\item Написать программу на Python, которая создает класс Queue для представления структуры данных очереди с инкапсуляцией. Класс должен содержать методы enqueue, dequeue и is\_empty, которые реализуют операции добавления элементов в очередь, удаления элементов из очереди и проверки пустоты очереди соответственно. Программа также должна создавать экземпляр класса Queue, добавлять элементы в очередь, удалять элементы из очереди и выводить информацию о состоянии очереди на экран.

Инструкции:
\begin{enumerate}
    \item Создайте класс Queue с методом \_\_init\_\_, который инициализирует пустую очередь (список \_channel). Принимает параметр enqueue\_with\_timestamp=False. Если True, то при добавлении сохраняет пару (элемент, time.time()).
    \item Создайте метод enqueue, который, если enqueue\_with\_timestamp=True, добавляет (элемент, timestamp). Иначе — элемент.
    \item Создайте метод dequeue, который удаляет и возвращает первый элемент (или пару). Если очередь пуста — выбрасывает IndexError("Очередь пуста").
    \item Создайте метод is\_empty, который возвращает True, если очередь пуста, и False в противном случае.
    \item Создайте экземпляр класса Queue с enqueue\_with\_timestamp=True.
    \item Добавьте элементы: "first", "second", "third".
    \item Выведите текущее состояние очереди.
    \item Вызовите dequeue, выведите результат (пару).
    \item Выведите обновленное состояние очереди.
\end{enumerate}

Пример использования:
\begin{lstlisting}[language=Python]
import time

queue = Queue(enqueue_with_timestamp=True)
queue.enqueue("first")
queue.enqueue("second")
queue.enqueue("third")

print("Current Queue:", queue._channel)

dequeued_item = queue.dequeue()
print("Dequeued item:", dequeued_item)  # ('first', timestamp)

print("Updated Queue:", queue._channel)
\end{lstlisting}

\item Написать программу на Python, которая создает класс Queue для представления структуры данных очереди с инкапсуляцией. Класс должен содержать методы enqueue, dequeue и is\_empty, которые реализуют операции добавления элементов в очередь, удаления элементов из очереди и проверки пустоты очереди соответственно. Программа также должна создавать экземпляр класса Queue, добавлять элементы в очередь, удалять элементы из очереди и выводить информацию о состоянии очереди на экран.

Инструкции:
\begin{enumerate}
    \item Создайте класс Queue с методом \_\_init\_\_, который инициализирует пустую очередь (список \_tube). Принимает параметр enqueue\_pairs=False. Если True, то enqueue принимает два аргумента (key, value) и сохраняет кортеж (key, value).
    \item Создайте метод enqueue, который, если enqueue\_pairs=False, принимает один элемент. Если True — два аргумента и сохраняет кортеж.
    \item Создайте метод dequeue, который удаляет и возвращает первый элемент (или кортеж). Если очередь пуста — выбрасывает IndexError("Пусто").
    \item Создайте метод is\_empty, который возвращает True, если очередь пуста, и False в противном случае.
    \item Создайте экземпляр класса Queue с enqueue\_pairs=True.
    \item Добавьте пары: ("a", 1), ("b", 2), ("c", 3).
    \item Выведите текущее состояние очереди.
    \item Вызовите dequeue, выведите результат.
    \item Выведите обновленное состояние очереди.
\end{enumerate}

Пример использования:
\begin{lstlisting}[language=Python]
queue = Queue(enqueue_pairs=True)
queue.enqueue("a", 1)
queue.enqueue("b", 2)
queue.enqueue("c", 3)

print("Current Queue:", queue._tube)

dequeued_item = queue.dequeue()
print("Dequeued item:", dequeued_item)  # ('a', 1)

print("Updated Queue:", queue._tube)
\end{lstlisting}

\item Написать программу на Python, которая создает класс Queue для представления структуры данных очереди с инкапсуляцией. Класс должен содержать методы enqueue, dequeue и is\_empty, которые реализуют операции добавления элементов в очередь, удаления элементов из очереди и проверки пустоты очереди соответственно. Программа также должна создавать экземпляр класса Queue, добавлять элементы в очередь, удалять элементы из очереди и выводить информацию о состоянии очереди на экран.

Инструкции:
\begin{enumerate}
    \item Создайте класс Queue с методом \_\_init\_\_, который инициализирует пустую очередь (список \_conduit). Принимает параметр auto\_dedup=False. Если True, то при добавлении, если элемент уже есть в очереди, сначала удаляет все его предыдущие вхождения.
    \item Создайте метод enqueue, который, если auto\_dedup=True и элемент уже есть, удаляет все его вхождения, затем добавляет в конец. Иначе — просто добавляет.
    \item Создайте метод dequeue, который удаляет и возвращает первый элемент. Если очередь пуста — выбрасывает IndexError("Очередь пуста").
    \item Создайте метод is\_empty, который возвращает True, если очередь пуста, и False в противном случае.
    \item Создайте экземпляр класса Queue с auto\_dedup=True.
    \item Добавьте элементы: 1, 2, 1, 3, 2, 4.
    \item Выведите текущее состояние очереди.
    \item Вызовите dequeue, выведите удаленный элемент.
    \item Выведите обновленное состояние очереди.
\end{enumerate}

Пример использования:
\begin{lstlisting}[language=Python]
queue = Queue(auto_dedup=True)
queue.enqueue(1)  # [1]
queue.enqueue(2)  # [1,2]
queue.enqueue(1)  # удаляет старую 1 -> [2,1]
queue.enqueue(3)  # [2,1,3]
queue.enqueue(2)  # удаляет 2 -> [1,3,2]
queue.enqueue(4)  # [1,3,2,4]

print("Current Queue:", queue._conduit)

dequeued_item = queue.dequeue()
print("Dequeued item:", dequeued_item)  # 1

print("Updated Queue:", queue._conduit)  # [3,2,4]
\end{lstlisting}

\item Написать программу на Python, которая создает класс Queue для представления структуры данных очереди с инкапсуляцией. Класс должен содержать методы enqueue, dequeue и is\_empty, которые реализуют операции добавления элементов в очередь, удаления элементов из очереди и проверки пустоты очереди соответственно. Программа также должна создавать экземпляр класса Queue, добавлять элементы в очередь, удалять элементы из очереди и выводить информацию о состоянии очереди на экран.

Инструкции:
\begin{enumerate}
    \item Создайте класс Queue с методом \_\_init\_\_, который инициализирует пустую очередь (список \_duct). Принимает параметр enqueue\_if\_max=False. Если True, то элемент добавляется только если он больше всех текущих элементов в очереди.
    \item Создайте метод enqueue, который добавляет элемент, только если enqueue\_if\_max=False или элемент > всех элементов в очереди.
    \item Создайте метод dequeue, который удаляет и возвращает первый элемент. Если очередь пуста — выбрасывает IndexError("Пусто").
    \item Создайте метод is\_empty, который возвращает True, если очередь пуста, и False в противном случае.
    \item Создайте экземпляр класса Queue с enqueue\_if\_max=True.
    \item Добавьте элементы: 5, 3 (не добавится), 10, 7 (не добавится), 15.
    \item Выведите текущее состояние очереди.
    \item Вызовите dequeue, выведите удаленный элемент.
    \item Выведите обновленное состояние очереди.
\end{enumerate}

Пример использования:
\begin{lstlisting}[language=Python]
queue = Queue(enqueue_if_max=True)
queue.enqueue(5)
queue.enqueue(3)   # не добавится
queue.enqueue(10)
queue.enqueue(7)   # не добавится
queue.enqueue(15)

print("Current Queue:", queue._duct)  # [5,10,15]

dequeued_item = queue.dequeue()
print("Dequeued item:", dequeued_item)  # 5

print("Updated Queue:", queue._duct)  # [10,15]
\end{lstlisting}

\item Написать программу на Python, которая создает класс Queue для представления структуры данных очереди с инкапсуляцией. Класс должен содержать методы enqueue, dequeue и is\_empty, которые реализуют операции добавления элементов в очередь, удаления элементов из очереди и проверки пустоты очереди соответственно. Программа также должна создавать экземпляр класса Queue, добавлять элементы в очередь, удалять элементы из очереди и выводить информацию о состоянии очереди на экран.

Инструкции:
\begin{enumerate}
    \item Создайте класс Queue с методом \_\_init\_\_, который инициализирует пустую очередь (список \_pipe). Принимает параметр cumulative=False. Если True, то при добавлении элемент становится element + последний\_элемент (если очередь не пуста). Первый элемент добавляется как есть.
    \item Создайте метод enqueue, который, если cumulative=True и очередь не пуста, добавляет element + последний\_элемент. Иначе — element.
    \item Создайте метод dequeue, который удаляет и возвращает первый элемент. Если очередь пуста — выбрасывает IndexError("Очередь пуста").
    \item Создайте метод is\_empty, который возвращает True, если очередь пуста, и False в противном случае.
    \item Создайте экземпляр класса Queue с cumulative=True.
    \item Добавьте элементы: 1, 2, 3, 4.
    \item Выведите текущее состояние очереди.
    \item Вызовите dequeue, выведите удаленный элемент.
    \item Выведите обновленное состояние очереди.
\end{enumerate}

Пример использования:
\begin{lstlisting}[language=Python]
queue = Queue(cumulative=True)
queue.enqueue(1)  # [1]
queue.enqueue(2)  # [1, 1+2=3]
queue.enqueue(3)  # [1,3, 3+3=6]
queue.enqueue(4)  # [1,3,6, 6+4=10]

print("Current Queue:", queue._pipe)

dequeued_item = queue.dequeue()
print("Dequeued item:", dequeued_item)  # 1

print("Updated Queue:", queue._pipe)  # [3,6,10]
\end{lstlisting}

\item Написать программу на Python, которая создает класс Queue для представления структуры данных очереди с инкапсуляцией. Класс должен содержать методы enqueue, dequeue и is\_empty, которые реализуют операции добавления элементов в очередь, удаления элементов из очереди и проверки пустоты очереди соответственно. Программа также должна создавать экземпляр класса Queue, добавлять элементы в очередь, удалять элементы из очереди и выводить информацию о состоянии очереди на экран.

Инструкции:
\begin{enumerate}
    \item Создайте класс Queue с методом \_\_init\_\_, который инициализирует пустую очередь (список \_line). Принимает параметр enqueue\_squared=False. Если True, то при добавлении сохраняется element**2.
    \item Создайте метод enqueue, который добавляет element**2, если enqueue\_squared=True, иначе — element.
    \item Создайте метод dequeue, который удаляет и возвращает первый элемент. Если очередь пуста — выбрасывает IndexError("Пусто").
    \item Создайте метод is\_empty, который возвращает True, если очередь пуста, и False в противном случае.
    \item Создайте экземпляр класса Queue с enqueue\_squared=True.
    \item Добавьте элементы: 2, 3, 4, 5.
    \item Выведите текущее состояние очереди.
    \item Вызовите dequeue, выведите удаленный элемент.
    \item Выведите обновленное состояние очереди.
\end{enumerate}

Пример использования:
\begin{lstlisting}[language=Python]
queue = Queue(enqueue_squared=True)
queue.enqueue(2)  # 4
queue.enqueue(3)  # 9
queue.enqueue(4)  # 16
queue.enqueue(5)  # 25

print("Current Queue:", queue._line)

dequeued_item = queue.dequeue()
print("Dequeued item:", dequeued_item)  # 4

print("Updated Queue:", queue._line)  # [9,16,25]
\end{lstlisting}

\item Написать программу на Python, которая создает класс Queue для представления структуры данных очереди с инкапсуляцией. Класс должен содержать методы enqueue, dequeue и is\_empty, которые реализуют операции добавления элементов в очередь, удаления элементов из очереди и проверки пустоты очереди соответственно. Программа также должна создавать экземпляр класса Queue, добавлять элементы в очередь, удалять элементы из очереди и выводить информацию о состоянии очереди на экран.

Инструкции:
\begin{enumerate}
    \item Создайте класс Queue с методом \_\_init\_\_, который инициализирует пустую очередь (список \_stream). Принимает параметр enqueue\_absolute=False. Если True, то при добавлении сохраняется abs(element).
    \item Создайте метод enqueue, который добавляет abs(element), если enqueue\_absolute=True, иначе — element.
    \item Создайте метод dequeue, который удаляет и возвращает первый элемент. Если очередь пуста — выбрасывает IndexError("Очередь пуста").
    \item Создайте метод is\_empty, который возвращает True, если очередь пуста, и False в противном случае.
    \item Создайте экземпляр класса Queue с enqueue\_absolute=True.
    \item Добавьте элементы: -5, 3, -8, 2.
    \item Выведите текущее состояние очереди.
    \item Вызовите dequeue, выведите удаленный элемент.
    \item Выведите обновленное состояние очереди.
\end{enumerate}

Пример использования:
\begin{lstlisting}[language=Python]
queue = Queue(enqueue_absolute=True)
queue.enqueue(-5)  # 5
queue.enqueue(3)   # 3
queue.enqueue(-8)  # 8
queue.enqueue(2)   # 2

print("Current Queue:", queue._stream)

dequeued_item = queue.dequeue()
print("Dequeued item:", dequeued_item)  # 5

print("Updated Queue:", queue._stream)  # [3,8,2]
\end{lstlisting}

\item Написать программу на Python, которая создает класс Queue для представления структуры данных очереди с инкапсуляцией. Класс должен содержать методы enqueue, dequeue и is\_empty, которые реализуют операции добавления элементов в очередь, удаления элементов из очереди и проверки пустоты очереди соответственно. Программа также должна создавать экземпляр класса Queue, добавлять элементы в очередь, удалять элементы из очереди и выводить информацию о состоянии очереди на экран.

Инструкции:
\begin{enumerate}
    \item Создайте класс Queue с методом \_\_init\_\_, который инициализирует пустую очередь (список \_buffer). Принимает параметр enqueue\_rounded=False. Если True, то при добавлении сохраняется round(element).
    \item Создайте метод enqueue, который добавляет round(element), если enqueue\_rounded=True, иначе — element.
    \item Создайте метод dequeue, который удаляет и возвращает первый элемент. Если очередь пуста — выбрасывает IndexError("Пусто").
    \item Создайте метод is\_empty, который возвращает True, если очередь пуста, и False в противном случае.
    \item Создайте экземпляр класса Queue с enqueue\_rounded=True.
    \item Добавьте элементы: 3.2, 4.7, 5.1, 6.9.
    \item Выведите текущее состояние очереди.
    \item Вызовите dequeue, выведите удаленный элемент.
    \item Выведите обновленное состояние очереди.
\end{enumerate}

Пример использования:
\begin{lstlisting}[language=Python]
queue = Queue(enqueue_rounded=True)
queue.enqueue(3.2)  # 3
queue.enqueue(4.7)  # 5
queue.enqueue(5.1)  # 5
queue.enqueue(6.9)  # 7

print("Current Queue:", queue._buffer)

dequeued_item = queue.dequeue()
print("Dequeued item:", dequeued_item)  # 3

print("Updated Queue:", queue._buffer)  # [5,5,7]
\end{lstlisting}

\item Написать программу на Python, которая создает класс Queue для представления структуры данных очереди с инкапсуляцией. Класс должен содержать методы enqueue, dequeue и is\_empty, которые реализуют операции добавления элементов в очередь, удаления элементов из очереди и проверки пустоты очереди соответственно. Программа также должна создавать экземпляр класса Queue, добавлять элементы в очередь, удалять элементы из очереди и выводить информацию о состоянии очереди на экран.

Инструкции:
\begin{enumerate}
    \item Создайте класс Queue с методом \_\_init\_\_, который инициализирует пустую очередь (список \_store). Принимает параметр enqueue\_negated=False. Если True, то при добавлении сохраняется -element.
    \item Создайте метод enqueue, который добавляет -element, если enqueue\_negated=True, иначе — element.
    \item Создайте метод dequeue, который удаляет и возвращает первый элемент. Если очередь пуста — выбрасывает IndexError("Очередь пуста").
    \item Создайте метод is\_empty, который возвращает True, если очередь пуста, и False в противном случае.
    \item Создайте экземпляр класса Queue с enqueue\_negated=True.
    \item Добавьте элементы: 10, 20, 30, 40.
    \item Выведите текущее состояние очереди.
    \item Вызовите dequeue, выведите удаленный элемент.
    \item Выведите обновленное состояние очереди.
\end{enumerate}

Пример использования:
\begin{lstlisting}[language=Python]
queue = Queue(enqueue_negated=True)
queue.enqueue(10)  # -10
queue.enqueue(20)  # -20
queue.enqueue(30)  # -30
queue.enqueue(40)  # -40

print("Current Queue:", queue._store)

dequeued_item = queue.dequeue()
print("Dequeued item:", dequeued_item)  # -10

print("Updated Queue:", queue._store)  # [-20,-30,-40]
\end{lstlisting}

\item Написать программу на Python, которая создает класс Queue для представления структуры данных очереди с инкапсуляцией. Класс должен содержать методы enqueue, dequeue и is\_empty, которые реализуют операции добавления элементов в очередь, удаления элементов из очереди и проверки пустоты очереди соответственно. Программа также должна создавать экземпляр класса Queue, добавлять элементы в очередь, удалять элементы из очереди и выводить информацию о состоянии очереди на экран.

Инструкции:
\begin{enumerate}
    \item Создайте класс Queue с методом \_\_init\_\_, который инициализирует пустую очередь (список \_pool). Принимает параметр enqueue\_doubled=False. Если True, то при добавлении сохраняется element * 2.
    \item Создайте метод enqueue, который добавляет element * 2, если enqueue\_doubled=True, иначе — element.
    \item Создайте метод dequeue, который удаляет и возвращает первый элемент. Если очередь пуста — выбрасывает IndexError("Пусто").
    \item Создайте метод is\_empty, который возвращает True, если очередь пуста, и False в противном случае.
    \item Создайте экземпляр класса Queue с enqueue\_doubled=True.
    \item Добавьте элементы: 1, 2, 3, 4.
    \item Выведите текущее состояние очереди.
    \item Вызовите dequeue, выведите удаленный элемент.
    \item Выведите обновленное состояние очереди.
\end{enumerate}

Пример использования:
\begin{lstlisting}[language=Python]
queue = Queue(enqueue_doubled=True)
queue.enqueue(1)  # 2
queue.enqueue(2)  # 4
queue.enqueue(3)  # 6
queue.enqueue(4)  # 8

print("Current Queue:", queue._pool)

dequeued_item = queue.dequeue()
print("Dequeued item:", dequeued_item)  # 2

print("Updated Queue:", queue._pool)  # [4,6,8]
\end{lstlisting}

\item Написать программу на Python, которая создает класс Queue для представления структуры данных очереди с инкапсуляцией. Класс должен содержать методы enqueue, dequeue и is\_empty, которые реализуют операции добавления элементов в очередь, удаления элементов из очереди и проверки пустоты очереди соответственно. Программа также должна создавать экземпляр класса Queue, добавлять элементы в очередь, удалять элементы из очереди и выводить информацию о состоянии очереди на экран.

Инструкции:
\begin{enumerate}
    \item Создайте класс Queue с методом \_\_init\_\_, который инициализирует пустую очередь (список \_reservoir). Принимает параметр enqueue\_halved=False. Если True, то при добавлении сохраняется element / 2.0.
    \item Создайте метод enqueue, который добавляет element / 2.0, если enqueue\_halved=True, иначе — element.
    \item Создайте метод dequeue, который удаляет и возвращает первый элемент. Если очередь пуста — выбрасывает IndexError("Очередь пуста").
    \item Создайте метод is\_empty, который возвращает True, если очередь пуста, и False в противном случае.
    \item Создайте экземпляр класса Queue с enqueue\_halved=True.
    \item Добавьте элементы: 4, 8, 12, 16.
    \item Выведите текущее состояние очереди.
    \item Вызовите dequeue, выведите удаленный элемент.
    \item Выведите обновленное состояние очереди.
\end{enumerate}

Пример использования:
\begin{lstlisting}[language=Python]
queue = Queue(enqueue_halved=True)
queue.enqueue(4)   # 2.0
queue.enqueue(8)   # 4.0
queue.enqueue(12)  # 6.0
queue.enqueue(16)  # 8.0

print("Current Queue:", queue._reservoir)

dequeued_item = queue.dequeue()
print("Dequeued item:", dequeued_item)  # 2.0

print("Updated Queue:", queue._reservoir)  # [4.0,6.0,8.0]
\end{lstlisting}

\item Написать программу на Python, которая создает класс Queue для представления структуры данных очереди с инкапсуляцией. Класс должен содержать методы enqueue, dequeue и is\_empty, которые реализуют операции добавления элементов в очередь, удаления элементов из очереди и проверки пустоты очереди соответственно. Программа также должна создавать экземпляр класса Queue, добавлять элементы в очередь, удалять элементы из очереди и выводить информацию о состоянии очереди на экран.

Инструкции:
\begin{enumerate}
    \item Создайте класс Queue с методом \_\_init\_\_, который инициализирует пустую очередь (список \_tank). Принимает параметр enqueue\_as\_string=False. Если True, то при добавлении сохраняется str(element).
    \item Создайте метод enqueue, который добавляет str(element), если enqueue\_as\_string=True, иначе — element.
    \item Создайте метод dequeue, который удаляет и возвращает первый элемент. Если очередь пуста — выбрасывает IndexError("Пусто").
    \item Создайте метод is\_empty, который возвращает True, если очередь пуста, и False в противном случае.
    \item Создайте экземпляр класса Queue с enqueue\_as\_string=True.
    \item Добавьте элементы: 100, 200, 300, 400.
    \item Выведите текущее состояние очереди.
    \item Вызовите dequeue, выведите удаленный элемент.
    \item Выведите обновленное состояние очереди.
\end{enumerate}

Пример использования:
\begin{lstlisting}[language=Python]
queue = Queue(enqueue_as_string=True)
queue.enqueue(100)  # "100"
queue.enqueue(200)  # "200"
queue.enqueue(300)  # "300"
queue.enqueue(400)  # "400"

print("Current Queue:", queue._tank)

dequeued_item = queue.dequeue()
print("Dequeued item:", dequeued_item)  # "100"

print("Updated Queue:", queue._tank)  # ["200","300","400"]
\end{lstlisting}

\item Написать программу на Python, которая создает класс Queue для представления структуры данных очереди с инкапсуляцией. Класс должен содержать методы enqueue, dequeue и is\_empty, которые реализуют операции добавления элементов в очередь, удаления элементов из очереди и проверки пустоты очереди соответственно. Программа также должна создавать экземпляр класса Queue, добавлять элементы в очередь, удалять элементы из очереди и выводить информацию о состоянии очереди на экран.

Инструкции:
\begin{enumerate}
    \item Создайте класс Queue с методом \_\_init\_\_, который инициализирует пустую очередь (список \_container). Принимает параметр enqueue\_with\_index=False. Если True, то при добавлении сохраняется кортеж (element, порядковый\_номер\_добавления).
    \item Создайте метод enqueue, который добавляет (element, self.\_counter), где \_counter — внутренний счетчик, увеличивающийся при каждом добавлении. Иначе — element.
    \item Создайте метод dequeue, который удаляет и возвращает первый элемент (или кортеж). Если очередь пуста — выбрасывает IndexError("Очередь пуста").
    \item Создайте метод is\_empty, который возвращает True, если очередь пуста, и False в противном случае.
    \item Создайте экземпляр класса Queue с enqueue\_with\_index=True.
    \item Добавьте элементы: "alpha", "beta", "gamma".
    \item Выведите текущее состояние очереди.
    \item Вызовите dequeue, выведите удаленный элемент.
    \item Выведите обновленное состояние очереди.
\end{enumerate}

Пример использования:
\begin{lstlisting}[language=Python]
queue = Queue(enqueue_with_index=True)
queue.enqueue("alpha")  # ("alpha", 0)
queue.enqueue("beta")   # ("beta", 1)
queue.enqueue("gamma")  # ("gamma", 2)

print("Current Queue:", queue._container)

dequeued_item = queue.dequeue()
print("Dequeued item:", dequeued_item)  # ('alpha', 0)

print("Updated Queue:", queue._container)  # [('beta',1), ('gamma',2)]
\end{lstlisting}

\item Написать программу на Python, которая создает класс Queue для представления структуры данных очереди с инкапсуляцией. Класс должен содержать методы enqueue, dequeue и is\_empty, которые реализуют операции добавления элементов в очередь, удаления элементов из очереди и проверки пустоты очереди соответственно. Программа также должна создавать экземпляр класса Queue, добавлять элементы в очередь, удалять элементы из очереди и выводить информацию о состоянии очереди на экран.

Инструкции:
\begin{enumerate}
    \item Создайте класс Queue с методом \_\_init\_\_, который инициализирует пустую очередь (список \_vessel). Принимает параметр enqueue\_unique\_rear=False. Если True, то при добавлении, если элемент равен текущему последнему, он не добавляется.
    \item Создайте метод enqueue, который добавляет элемент, только если enqueue\_unique\_rear=False или очередь пуста или element != последний\_элемент.
    \item Создайте метод dequeue, который удаляет и возвращает первый элемент. Если очередь пуста — выбрасывает IndexError("Пусто").
    \item Создайте метод is\_empty, который возвращает True, если очередь пуста, и False в противном случае.
    \item Создайте экземпляр класса Queue с enqueue\_unique\_rear=True.
    \item Добавьте элементы: 1, 2, 2, 3, 3, 3, 4.
    \item Выведите текущее состояние очереди.
    \item Вызовите dequeue, выведите удаленный элемент.
    \item Выведите обновленное состояние очереди.
\end{enumerate}

Пример использования:
\begin{lstlisting}[language=Python]
queue = Queue(enqueue_unique_rear=True)
queue.enqueue(1)
queue.enqueue(2)
queue.enqueue(2)  # не добавится
queue.enqueue(3)
queue.enqueue(3)  # не добавится
queue.enqueue(3)  # не добавится
queue.enqueue(4)

print("Current Queue:", queue._vessel)  # [1,2,3,4]

dequeued_item = queue.dequeue()
print("Dequeued item:", dequeued_item)  # 1

print("Updated Queue:", queue._vessel)  # [2,3,4]
\end{lstlisting}

\item Написать программу на Python, которая создает класс Queue для представления структуры данных очереди с инкапсуляцией. Класс должен содержать методы enqueue, dequeue и is\_empty, которые реализуют операции добавления элементов в очередь, удаления элементов из очереди и проверки пустоты очереди соответственно. Программа также должна создавать экземпляр класса Queue, добавлять элементы в очередь, удалять элементы из очереди и выводить информацию о состоянии очереди на экран.

Инструкции:
\begin{enumerate}
    \item Создайте класс Queue с методом \_\_init\_\_, который инициализирует пустую очередь (список \_bin). Принимает параметр enqueue\_even\_only=False. Если True, то добавляются только четные числа.
    \item Создайте метод enqueue, который добавляет элемент, только если enqueue\_even\_only=False или element \% 2 == 0.
    \item Создайте метод dequeue, который удаляет и возвращает первый элемент. Если очередь пуста — выбрасывает IndexError("Очередь пуста").
    \item Создайте метод is\_empty, который возвращает True, если очередь пуста, и False в противном случае.
    \item Создайте экземпляр класса Queue с enqueue\_even\_only=True.
    \item Добавьте элементы: 1 (не добавится), 2, 3 (не добавится), 4, 5 (не добавится), 6.
    \item Выведите текущее состояние очереди.
    \item Вызовите dequeue, выведите удаленный элемент.
    \item Выведите обновленное состояние очереди.
\end{enumerate}

Пример использования:
\begin{lstlisting}[language=Python]
queue = Queue(enqueue_even_only=True)
queue.enqueue(1)  # нет
queue.enqueue(2)
queue.enqueue(3)  # нет
queue.enqueue(4)
queue.enqueue(5)  # нет
queue.enqueue(6)

print("Current Queue:", queue._bin)  # [2,4,6]

dequeued_item = queue.dequeue()
print("Dequeued item:", dequeued_item)  # 2

print("Updated Queue:", queue._bin)  # [4,6]
\end{lstlisting}

\item Написать программу на Python, которая создает класс Queue для представления структуры данных очереди с инкапсуляцией. Класс должен содержать методы enqueue, dequeue и is\_empty, которые реализуют операции добавления элементов в очередь, удаления элементов из очереди и проверки пустоты очереди соответственно. Программа также должна создавать экземпляр класса Queue, добавлять элементы в очередь, удалять элементы из очереди и выводить информацию о состоянии очереди на экран.

Инструкции:
\begin{enumerate}
    \item Создайте класс Queue с методом \_\_init\_\_, который инициализирует пустую очередь (список \_box). Принимает параметр enqueue\_odd\_only=False. Если True, то добавляются только нечетные числа.
    \item Создайте метод enqueue, который добавляет элемент, только если enqueue\_odd\_only=False или element \% 2 != 0.
    \item Создайте метод dequeue, который удаляет и возвращает первый элемент. Если очередь пуста — выбрасывает IndexError("Пусто").
    \item Создайте метод is\_empty, который возвращает True, если очередь пуста, и False в противном случае.
    \item Создайте экземпляр класса Queue с enqueue\_odd\_only=True.
    \item Добавьте элементы: 2 (не добавится), 1, 4 (не добавится), 3, 6 (не добавится), 5.
    \item Выведите текущее состояние очереди.
    \item Вызовите dequeue, выведите удаленный элемент.
    \item Выведите обновленное состояние очереди.
\end{enumerate}

Пример использования:
\begin{lstlisting}[language=Python]
queue = Queue(enqueue_odd_only=True)
queue.enqueue(2)  # нет
queue.enqueue(1)
queue.enqueue(4)  # нет
queue.enqueue(3)
queue.enqueue(6)  # нет
queue.enqueue(5)

print("Current Queue:", queue._box)  # [1,3,5]

dequeued_item = queue.dequeue()
print("Dequeued item:", dequeued_item)  # 1

print("Updated Queue:", queue._box)  # [3,5]
\end{lstlisting}

\item Написать программу на Python, которая создает класс Queue для представления структуры данных очереди с инкапсуляцией. Класс должен содержать методы enqueue, dequeue и is\_empty, которые реализуют операции добавления элементов в очередь, удаления элементов из очереди и проверки пустоты очереди соответственно. Программа также должна создавать экземпляр класса Queue, добавлять элементы в очередь, удалять элементы из очереди и выводить информацию о состоянии очереди на экран.

Инструкции:
\begin{enumerate}
    \item Создайте класс Queue с методом \_\_init\_\_, который инициализирует пустую очередь (список \_crate). Принимает параметр enqueue\_positive\_only=False. Если True, то добавляются только положительные числа (>0).
    \item Создайте метод enqueue, который добавляет элемент, только если enqueue\_positive\_only=False или element > 0.
    \item Создайте метод dequeue, который удаляет и возвращает первый элемент. Если очередь пуста — выбрасывает IndexError("Очередь пуста").
    \item Создайте метод is\_empty, который возвращает True, если очередь пуста, и False в противном случае.
    \item Создайте экземпляр класса Queue с enqueue\_positive\_only=True.
    \item Добавьте элементы: -1 (не добавится), 0 (не добавится), 1, 2, -5 (не добавится), 3.
    \item Выведите текущее состояние очереди.
    \item Вызовите dequeue, выведите удаленный элемент.
    \item Выведите обновленное состояние очереди.
\end{enumerate}

Пример использования:
\begin{lstlisting}[language=Python]
queue = Queue(enqueue_positive_only=True)
queue.enqueue(-1)  # нет
queue.enqueue(0)   # нет
queue.enqueue(1)
queue.enqueue(2)
queue.enqueue(-5)  # нет
queue.enqueue(3)

print("Current Queue:", queue._crate)  # [1,2,3]

dequeued_item = queue.dequeue()
print("Dequeued item:", dequeued_item)  # 1

print("Updated Queue:", queue._crate)  # [2,3]
\end{lstlisting}

\item Написать программу на Python, которая создает класс Queue для представления структуры данных очереди с инкапсуляцией. Класс должен содержать методы enqueue, dequeue и is\_empty, которые реализуют операции добавления элементов в очередь, удаления элементов из очереди и проверки пустоты очереди соответственно. Программа также должна создавать экземпляр класса Queue, добавлять элементы в очередь, удалять элементы из очереди и выводить информацию о состоянии очереди на экран.

Инструкции:
\begin{enumerate}
    \item Создайте класс Queue с методом \_\_init\_\_, который инициализирует пустую очередь (список \_carton). Принимает параметр enqueue\_nonzero\_only=False. Если True, то добавляются только ненулевые числа.
    \item Создайте метод enqueue, который добавляет элемент, только если enqueue\_nonzero\_only=False или element != 0.
    \item Создайте метод dequeue, который удаляет и возвращает первый элемент. Если очередь пуста — выбрасывает IndexError("Пусто").
    \item Создайте метод is\_empty, который возвращает True, если очередь пуста, и False в противном случае.
    \item Создайте экземпляр класса Queue с enqueue\_nonzero\_only=True.
    \item Добавьте элементы: 0 (не добавится), 5, 0 (не добавится), 10, 15.
    \item Выведите текущее состояние очереди.
    \item Вызовите dequeue, выведите удаленный элемент.
    \item Выведите обновленное состояние очереди.
\end{enumerate}

Пример использования:
\begin{lstlisting}[language=Python]
queue = Queue(enqueue_nonzero_only=True)
queue.enqueue(0)   # нет
queue.enqueue(5)
queue.enqueue(0)   # нет
queue.enqueue(10)
queue.enqueue(15)

print("Current Queue:", queue._carton)  # [5,10,15]

dequeued_item = queue.dequeue()
print("Dequeued item:", dequeued_item)  # 5

print("Updated Queue:", queue._carton)  # [10,15]
\end{lstlisting}

\item Написать программу на Python, которая создает класс Queue для представления структуры данных очереди с инкапсуляцией. Класс должен содержать методы enqueue, dequeue и is\_empty, которые реализуют операции добавления элементов в очередь, удаления элементов из очереди и проверки пустоты очереди соответственно. Программа также должна создавать экземпляр класса Queue, добавлять элементы в очередь, удалять элементы из очереди и выводить информацию о состоянии очереди на экран.

Инструкции:
\begin{enumerate}
    \item Создайте класс Queue с методом \_\_init\_\_, который инициализирует пустую очередь (список \_package). Принимает параметр enqueue\_prime\_only=False. Если True, то добавляются только простые числа (реализуйте простую проверку).
    \item Создайте метод enqueue, который добавляет элемент, только если enqueue\_prime\_only=False или element — простое число.
    \item Создайте метод dequeue, который удаляет и возвращает первый элемент. Если очередь пуста — выбрасывает IndexError("Очередь пуста").
    \item Создайте метод is\_empty, который возвращает True, если очередь пуста, и False в противном случае.
    \item Создайте вспомогательную функцию is\_prime(n) (вне класса).
    \item Создайте экземпляр класса Queue с enqueue\_prime\_only=True.
    \item Добавьте элементы: 4 (не простое), 5 (простое), 6 (не простое), 7 (простое), 8 (не простое), 11 (простое).
    \item Выведите текущее состояние очереди.
    \item Вызовите dequeue, выведите удаленный элемент.
    \item Выведите обновленное состояние очереди.
\end{enumerate}

Пример использования:
\begin{lstlisting}[language=Python]
def is_prime(n):
    if n < 2:
        return False
    for i in range(2, int(n**0.5)+1):
        if n % i == 0:
            return False
    return True

queue = Queue(enqueue_prime_only=True)
queue.enqueue(4)   # нет
queue.enqueue(5)   # да
queue.enqueue(6)   # нет
queue.enqueue(7)   # да
queue.enqueue(8)   # нет
queue.enqueue(11)  # да

print("Current Queue:", queue._package)  # [5,7,11]

dequeued_item = queue.dequeue()
print("Dequeued item:", dequeued_item)  # 5

print("Updated Queue:", queue._package)  # [7,11]
\end{lstlisting}

\item Написать программу на Python, которая создает класс Queue для представления структуры данных очереди с инкапсуляцией. Класс должен содержать методы enqueue, dequeue и is\_empty, которые реализуют операции добавления элементов в очередь, удаления элементов из очереди и проверки пустоты очереди соответственно. Программа также должна создавать экземпляр класса Queue, добавлять элементы в очередь, удалять элементы из очереди и выводить информацию о состоянии очереди на экран.

Инструкции:
\begin{enumerate}
    \item Создайте класс Queue с методом \_\_init\_\_, который инициализирует пустую очередь (список \_parcel). Принимает параметр enqueue\_fibonacci\_only=False. Если True, то добавляются только числа Фибоначчи (до 100: 0,1,1,2,3,5,8,13,21,34,55,89).
    \item Создайте метод enqueue, который добавляет элемент, только если enqueue\_fibonacci\_only=False или element входит в FIB\_SET.
    \item Создайте метод dequeue, который удаляет и возвращает первый элемент. Если очередь пуста — выбрасывает IndexError("Пусто").
    \item Создайте метод is\_empty, который возвращает True, если очередь пуста, и False в противном случае.
    \item Создайте экземпляр класса Queue с enqueue\_fibonacci\_only=True.
    \item Добавьте элементы: 4 (не Фибоначчи), 5 (Фибоначчи), 6 (не Фибоначчи), 8 (Фибоначчи), 7 (не Фибоначчи), 13 (Фибоначчи).
    \item Выведите текущее состояние очереди.
    \item Вызовите dequeue, выведите удаленный элемент.
    \item Выведите обновленное состояние очереди.
\end{enumerate}

Пример использования:
\begin{lstlisting}[language=Python]
FIB_SET = {0, 1, 2, 3, 5, 8, 13, 21, 34, 55, 89}

queue = Queue(enqueue_fibonacci_only=True)
queue.enqueue(4)   # нет
queue.enqueue(5)   # да
queue.enqueue(6)   # нет
queue.enqueue(8)   # да
queue.enqueue(7)   # нет
queue.enqueue(13)  # да

print("Current Queue:", queue._parcel)  # [5,8,13]

dequeued_item = queue.dequeue()
print("Dequeued item:", dequeued_item)  # 5

print("Updated Queue:", queue._parcel)  # [8,13]
\end{lstlisting}

\item Написать программу на Python, которая создает класс Queue для представления структуры данных очереди с инкапсуляцией. Класс должен содержать методы enqueue, dequeue и is\_empty, которые реализуют операции добавления элементов в очередь, удаления элементов из очереди и проверки пустоты очереди соответственно. Программа также должна создавать экземпляр класса Queue, добавлять элементы в очередь, удалять элементы из очереди и выводить информацию о состоянии очереди на экран.

Инструкции:
\begin{enumerate}
    \item Создайте класс Queue с методом \_\_init\_\_, который инициализирует пустую очередь (список \_sack). Принимает параметр enqueue\_palindrome\_only=False. Если True, то добавляются только числа-палиндромы.
    \item Создайте метод enqueue, который добавляет элемент, только если enqueue\_palindrome\_only=False или element — палиндром (str(element) == str(element)[::-1]).
    \item Создайте метод dequeue, который удаляет и возвращает первый элемент. Если очередь пуста — выбрасывает IndexError("Очередь пуста").
    \item Создайте метод is\_empty, который возвращает True, если очередь пуста, и False в противном случае.
    \item Создайте экземпляр класса Queue с enqueue\_palindrome\_only=True.
    \item Добавьте элементы: 12 (не палиндром), 22 (палиндром), 34 (не палиндром), 55 (палиндром), 123 (не палиндром), 121 (палиндром).
    \item Выведите текущее состояние очереди.
    \item Вызовите dequeue, выведите удаленный элемент.
    \item Выведите обновленное состояние очереди.
\end{enumerate}

Пример использования:
\begin{lstlisting}[language=Python]
queue = Queue(enqueue_palindrome_only=True)
queue.enqueue(12)   # нет
queue.enqueue(22)   # да
queue.enqueue(34)   # нет
queue.enqueue(55)   # да
queue.enqueue(123)  # нет
queue.enqueue(121)  # да

print("Current Queue:", queue._sack)  # [22,55,121]

dequeued_item = queue.dequeue()
print("Dequeued item:", dequeued_item)  # 22

print("Updated Queue:", queue._sack)  # [55,121]
\end{lstlisting}

\item Написать программу на Python, которая создает класс Queue для представления структуры данных очереди с инкапсуляцией. Класс должен содержать методы enqueue, dequeue и is\_empty, которые реализуют операции добавления элементов в очередь, удаления элементов из очереди и проверки пустоты очереди соответственно. Программа также должна создавать экземпляр класса Queue, добавлять элементы в очередь, удалять элементы из очереди и выводить информацию о состоянии очереди на экран.

Инструкции:
\begin{enumerate}
    \item Создайте класс Queue с методом \_\_init\_\_, который инициализирует пустую очередь (список \_bag). Принимает параметр enqueue\_power\_of\_two=False. Если True, то добавляются только степени двойки.
    \item Создайте метод enqueue, который добавляет элемент, только если enqueue\_power\_of\_two=False или element > 0 и (element \& (element-1)) == 0.
    \item Создайте метод dequeue, который удаляет и возвращает первый элемент. Если очередь пуста — выбрасывает IndexError("Пусто").
    \item Создайте метод is\_empty, который возвращает True, если очередь пуста, и False в противном случае.
    \item Создайте экземпляр класса Queue с enqueue\_power\_of\_two=True.
    \item Добавьте элементы: 3 (не степень), 4 (степень), 5 (не степень), 8 (степень), 9 (не степень), 16 (степень).
    \item Выведите текущее состояние очереди.
    \item Вызовите dequeue, выведите удаленный элемент.
    \item Выведите обновленное состояние очереди.
\end{enumerate}

Пример использования:
\begin{lstlisting}[language=Python]
queue = Queue(enqueue_power_of_two=True)
queue.enqueue(3)   # нет
queue.enqueue(4)   # да
queue.enqueue(5)   # нет
queue.enqueue(8)   # да
queue.enqueue(9)   # нет
queue.enqueue(16)  # да

print("Current Queue:", queue._bag)  # [4,8,16]

dequeued_item = queue.dequeue()
print("Dequeued item:", dequeued_item)  # 4

print("Updated Queue:", queue._bag)  # [8,16]
\end{lstlisting}

\item Написать программу на Python, которая создает класс Queue для представления структуры данных очереди с инкапсуляцией. Класс должен содержать методы enqueue, dequeue и is\_empty, которые реализуют операции добавления элементов в очередь, удаления элементов из очереди и проверки пустоты очереди соответственно. Программа также должна создавать экземпляр класса Queue, добавлять элементы в очередь, удалять элементы из очереди и выводить информацию о состоянии очереди на экран.

Инструкции:
\begin{enumerate}
    \item Создайте класс Queue с методом \_\_init\_\_, который инициализирует пустую очередь (список \_suitcase). Принимает параметр enqueue\_divisible\_by\_three=False. Если True, то добавляются только числа, делящиеся на 3.
    \item Создайте метод enqueue, который добавляет элемент, только если enqueue\_divisible\_by\_three=False или element \% 3 == 0.
    \item Создайте метод dequeue, который удаляет и возвращает первый элемент. Если очередь пуста — выбрасывает IndexError("Очередь пуста").
    \item Создайте метод is\_empty, который возвращает True, если очередь пуста, и False в противном случае.
    \item Создайте экземпляр класса Queue с enqueue\_divisible\_by\_three=True.
    \item Добавьте элементы: 1 (нет), 3 (да), 4 (нет), 6 (да), 7 (нет), 9 (да).
    \item Выведите текущее состояние очереди.
    \item Вызовите dequeue, выведите удаленный элемент.
    \item Выведите обновленное состояние очереди.
\end{enumerate}

Пример использования:
\begin{lstlisting}[language=Python]
queue = Queue(enqueue_divisible_by_three=True)
queue.enqueue(1)  # нет
queue.enqueue(3)  # да
queue.enqueue(4)  # нет
queue.enqueue(6)  # да
queue.enqueue(7)  # нет
queue.enqueue(9)  # да

print("Current Queue:", queue._suitcase)  # [3,6,9]

dequeued_item = queue.dequeue()
print("Dequeued item:", dequeued_item)  # 3

print("Updated Queue:", queue._suitcase)  # [6,9]
\end{lstlisting}

\item Написать программу на Python, которая создает класс Queue для представления структуры данных очереди с инкапсуляцией. Класс должен содержать методы enqueue, dequeue и is\_empty, которые реализуют операции добавления элементов в очередь, удаления элементов из очереди и проверки пустоты очереди соответственно. Программа также должна создавать экземпляр класса Queue, добавлять элементы в очередь, удалять элементы из очереди и выводить информацию о состоянии очереди на экран.

Инструкции:
\begin{enumerate}
    \item Создайте класс Queue с методом \_\_init\_\_, который инициализирует пустую очередь (список \_luggage). Принимает параметр enqueue\_greater\_than\_prev=False. Если True, то элемент добавляется только если он строго больше предыдущего добавленного элемента (первый — всегда).
    \item Создайте метод enqueue, который добавляет элемент, только если enqueue\_greater\_than\_prev=False или очередь пуста или element > последний\_элемент.
    \item Создайте метод dequeue, который удаляет и возвращает первый элемент. Если очередь пуста — выбрасывает IndexError("Пусто").
    \item Создайте метод is\_empty, который возвращает True, если очередь пуста, и False в противном случае.
    \item Создайте экземпляр класса Queue с enqueue\_greater\_than\_prev=True.
    \item Добавьте элементы: 5, 3 (не добавится), 7, 6 (не добавится), 10, 8 (не добавится).
    \item Выведите текущее состояние очереди.
    \item Вызовите dequeue, выведите удаленный элемент.
    \item Выведите обновленное состояние очереди.
\end{enumerate}

Пример использования:
\begin{lstlisting}[language=Python]
queue = Queue(enqueue_greater_than_prev=True)
queue.enqueue(5)
queue.enqueue(3)  # нет
queue.enqueue(7)
queue.enqueue(6)  # нет
queue.enqueue(10)
queue.enqueue(8)  # нет

print("Current Queue:", queue._luggage)  # [5,7,10]

dequeued_item = queue.dequeue()
print("Dequeued item:", dequeued_item)  # 5

print("Updated Queue:", queue._luggage)  # [7,10]
\end{lstlisting}

\item Написать программу на Python, которая создает класс Queue для представления структуры данных очереди с инкапсуляцией. Класс должен содержать методы enqueue, dequeue и is\_empty, которые реализуют операции добавления элементов в очередь, удаления элементов из очереди и проверки пустоты очереди соответственно. Программа также должна создавать экземпляр класса Queue, добавлять элементы в очередь, удалять элементы из очереди и выводить информацию о состоянии очереди на экран.

Инструкции:
\begin{enumerate}
    \item Создайте класс Queue с методом \_\_init\_\_, который инициализирует пустую очередь (список \_trunk). Принимает параметр enqueue\_less\_than\_prev=False. Если True, то элемент добавляется только если он строго меньше предыдущего добавленного элемента (первый — всегда).
    \item Создайте метод enqueue, который добавляет элемент, только если enqueue\_less\_than\_prev=False или очередь пуста или element < последний\_элемент.
    \item Создайте метод dequeue, который удаляет и возвращает первый элемент. Если очередь пуста — выбрасывает IndexError("Очередь пуста").
    \item Создайте метод is\_empty, который возвращает True, если очередь пуста, и False в противном случае.
    \item Создайте экземпляр класса Queue с enqueue\_less\_than\_prev=True.
    \item Добавьте элементы: 10, 15 (не добавится), 8, 9 (не добавится), 5, 7 (не добавится).
    \item Выведите текущее состояние очереди.
    \item Вызовите dequeue, выведите удаленный элемент.
    \item Выведите обновленное состояние очереди.
\end{enumerate}

Пример использования:
\begin{lstlisting}[language=Python]
queue = Queue(enqueue_less_than_prev=True)
queue.enqueue(10)
queue.enqueue(15)  # нет
queue.enqueue(8)
queue.enqueue(9)   # нет
queue.enqueue(5)
queue.enqueue(7)   # нет

print("Current Queue:", queue._trunk)  # [10,8,5]

dequeued_item = queue.dequeue()
print("Dequeued item:", dequeued_item)  # 10

print("Updated Queue:", queue._trunk)  # [8,5]
\end{lstlisting}

\end{enumerate}
\subsection{Семинар <<Структуры данных (закрепление) и \texttt{\_\_new\_\_}>>  
(2 часа)}

При создании подкласса неизменяемого встроенного типа данных 
(например, \texttt{float}, \texttt{str}, \texttt{int}) 
возникает проблема: значение объекта 
устанавливается \textit{в момент его создания}, 
и метод \texttt{\_\_init\_\_} вызывается уже 
\textit{после} этого, когда изменить базовое значение невозможно.

Кроме того, конструктор родительского неизменяемого типа 
(например, \texttt{float.\_\_new\_\_()}) часто не принимает 
дополнительные аргументы так же гибко, 
как \texttt{object.\_\_new\_\_()}, что приводит к ошибкам.

\textbf{Решение:} Использовать метод \texttt{\_\_new\_\_} для 
инициализации объекта \textit{в момент его создания}.

\begin{lstlisting}[language=Python, caption=Пример: Класс Distance с использованием \_\_new\_\_]
class Distance(float):
    def __new__(cls, value, unit):
        # 1. Создаем новый экземпляр float с заданным значением
        instance = super().__new__(cls, value)
        # 2. Настраиваем экземпляр, добавляя изменяемый атрибут
        instance.unit = unit
        # 3. Возвращаем настроенный экземпляр
        return instance

# Использование:
d = Distance(10.5, "km")
print(d)      # 10.5
print(d.unit) # km
d.unit = "m"  # Атрибут unit изменяем!
print(d.unit) # m
\end{lstlisting}

В этом примере \texttt{\_\_new\_\_} выполняет три шага:
\begin{enumerate}
    \item Создает новый экземпляр текущего класса \texttt{cls}, вызывая \texttt{super().\_\_new\_\_(cls, value)}. Это обращение к \texttt{float.\_\_new\_\_()}, который создает и инициализирует новый экземпляр \texttt{float}.
    \item Настраивает новый экземпляр, добавляя к нему изменяемый атрибут \texttt{unit}.
    \item Возвращает новый, настроенный экземпляр.
\end{enumerate}

Теперь класс \texttt{Distance} работает корректно, позволяя хранить единицы измерения в изменяемом атрибуте \texttt{unit}.

\textbf{Замечение}: для упрщения мы не применяли свойство ООП 
\textit{инкапсуляция} 
в примере.

\subsubsection {Задача 1 (Singleton)}

Реализуйте задание согласно своему варианту. Обратите внимание, что мы
не реализуем логику работы сложных вещей, а только её имитируем во всех 
вариантах. 

\textbf{Замечание}: Singleton -- это антипаттерн, в production его использовать
не стоит, но для учебных целей он хорош и, кроме того, знание его
сущности обязательно для разработчика.

\begin{enumerate}

\item Написать программу на Python, которая создает класс `DataBase` с использованием метода `\_\_new\_\_` для реализации паттерна Singleton (один экземпляр). Программа должна принимать параметры при создании и выводить сообщение при подключении.

Инструкции:
\begin{enumerate}
    \item Создайте класс `DataBase`.
    \item Добавьте приватный атрибут класса `\_instance` и инициализируйте его значением `None`.
    \item Переопределите метод `\_\_new\_\_`, чтобы он проверял, существует ли уже экземпляр. Если нет — создает новый с помощью `super().\_\_new\_\_(cls)` и присваивает его `\_instance`. Возвращает `\_instance`.
    \item Переопределите метод `\_\_init\_\_`, принимающий `user`, `psw`, `port`. Устанавливает эти атрибуты экземпляра, но только если они еще не были установлены (чтобы не перезаписывать при повторном "создании").
    \item Добавьте метод `connect`, который выводит сообщение "Подключение к БД: \{user\}, \{psw\}, \{port\}".
    \item Добавьте метод `\_\_del\_\_`, который выводит "Закрытие соединения с БД".
    \item Добавьте метод `get\_data`, который возвращает строку "Данные получены".
    \item Добавьте метод `set\_data`, который принимает `data` и выводит "Данные '\{data\}' записаны".
    \item Создайте два экземпляра `db1` и `db2` с разными параметрами.
    \item Вызовите `connect` для `db1`, затем для `db2`.
    \item Выведите `id(db1)` и `id(db2)` — они должны совпадать.
\end{enumerate}

Пример использования:
\begin{lstlisting}[language=Python]
db1 = DataBase("admin", "secret", 5432)
db2 = DataBase("user", "12345", 3306)  # Это тот же объект, что и db1!

db1.connect()
db2.connect()  # Выведет те же параметры, что и db1

print("ID db1:", id(db1))
print("ID db2:", id(db2))  # ID будут одинаковыми
\end{lstlisting}

\item Написать программу на Python, которая создает класс `ConnectionManager` с использованием метода `\_\_new\_\_` для реализации паттерна Singleton. Программа должна принимать параметры `host`, `username`, `timeout` при создании экземпляра.

Инструкции:
\begin{enumerate}
    \item Создайте класс `ConnectionManager`.
    \item Добавьте приватный атрибут класса `\_shared\_instance` и инициализируйте его значением `None`.
    \item Переопределите метод `\_\_new\_\_`, чтобы он возвращал существующий экземпляр, если он есть, или создавал новый.
    \item Переопределите метод `\_\_init\_\_`, принимающий `host`, `username`, `timeout`. Устанавливает атрибуты, только если они еще не заданы.
    \item Добавьте метод `establish`, который выводит "Соединение установлено с \{host\} под пользователем \{username\} (таймаут: \{timeout\})".
    \item Добавьте метод `\_\_del\_\_`, который выводит "Соединение разорвано".
    \item Добавьте метод `fetch`, который возвращает "Запрос выполнен".
    \item Добавьте метод `commit`, который принимает `transaction` и выводит "Транзакция '\{transaction\}' зафиксирована".
    \item Создайте два экземпляра `cm1` и `cm2` с разными параметрами.
    \item Вызовите `establish` для `cm1`, затем для `cm2`.
    \item Выведите `cm1 is cm2` — должно быть `True`.
\end{enumerate}

Пример использования:
\begin{lstlisting}[language=Python]
cm1 = ConnectionManager("localhost", "root", 30)
cm2 = ConnectionManager("remote.server", "guest", 60)

cm1.establish()
cm2.establish()  # Параметры будут от cm1

print("cm1 is cm2:", cm1 is cm2)  # True
\end{lstlisting}

\item Написать программу на Python, которая создает класс `ConfigLoader` с использованием метода `\_\_new\_\_` для реализации паттерна Singleton. Программа должна принимать параметры `config\_file`, `env`, `debug` при создании экземпляра.

Инструкции:
\begin{enumerate}
    \item Создайте класс `ConfigLoader`.
    \item Добавьте приватный атрибут класса `\_instance\_ref` и инициализируйте его значением `None`.
    \item Переопределите метод `\_\_new\_\_`, чтобы он обеспечивал единственный экземпляр.
    \item Переопределите метод `\_\_init\_\_`, принимающий `config\_file`, `env`, `debug`. Устанавливает атрибуты, только если они еще не заданы.
    \item Добавьте метод `load`, который выводит "Конфигурация загружена из '\{config\_file\}' для среды '\{env\}' (debug=\{debug\})".
    \item Добавьте метод `\_\_del\_\_`, который выводит "Конфигурация выгружена".
    \item Добавьте метод `get\_setting`, который принимает `key` и возвращает "Значение для \{key\}".
    \item Добавьте метод `set\_setting`, который принимает `key`, `value` и выводит "Настройка \{key\} установлена в \{value\}".
    \item Создайте два экземпляра `cfg1` и `cfg2` с разными параметрами.
    \item Вызовите `load` для `cfg1`, затем для `cfg2`.
    \item Проверьте, что `cfg1.debug == cfg2.debug` (должно быть `True`, если первый был создан с `debug=True`).
\end{enumerate}

Пример использования:
\begin{lstlisting}[language=Python]
cfg1 = ConfigLoader("app.yaml", "prod", True)
cfg2 = ConfigLoader("dev.yaml", "dev", False)

cfg1.load()
cfg2.load()  # Параметры будут от cfg1

print("Debug mode (cfg1):", cfg1.debug)
print("Debug mode (cfg2):", cfg2.debug)  # Будет True, как у cfg1
\end{lstlisting}

\item Написать программу на Python, которая создает класс `Logger` с использованием метода `\_\_new\_\_` для реализации паттерна Singleton. Программа должна принимать параметры `log\_level`, `output\_file`, `rotate` при создании экземпляра.

Инструкции:
\begin{enumerate}
    \item Создайте класс `Logger`.
    \item Добавьте приватный атрибут класса `\_the\_logger` и инициализируйте его значением `None`.
    \item Переопределите метод `\_\_new\_\_`, чтобы он возвращал единственный экземпляр.
    \item Переопределите метод `\_\_init\_\_`, принимающий `log\_level`, `output\_file`, `rotate`. Устанавливает атрибуты, только если они еще не заданы.
    \item Добавьте метод `log`, который принимает `message` и выводит "[\{log\_level\}] \{message\} -> \{output\_file\}".
    \item Добавьте метод `\_\_del\_\_`, который выводит "Логгер остановлен".
    \item Добавьте метод `set\_level`, который принимает `level` и устанавливает `self.log\_level = level`.
    \item Добавьте метод `flush`, который выводит "Буфер логов сброшен".
    \item Создайте два экземпляра `log1` и `log2` с разными параметрами.
    \item Вызовите `log` для `log1`, затем `set\_level("ERROR")` для `log2`.
    \item Вызовите `log` для `log1` снова — уровень должен измениться.
\end{enumerate}

Пример использования:
\begin{lstlisting}[language=Python]
log1 = Logger("INFO", "app.log", True)
log2 = Logger("DEBUG", "debug.log", False)

log1.log("Старт приложения")
log2.set\_level("ERROR")  # Меняет уровень для log1 тоже!
log1.log("Ошибка!")  # Выведет [ERROR] Ошибка! -> app.log
\end{lstlisting}

\item Написать программу на Python, которая создает класс `Cache` с использованием метода `\_\_new\_\_` для реализации паттерна Singleton. Программа должна принимать параметры `max\_size`, `ttl`, `strategy` при создании экземпляра.

Инструкции:
\begin{enumerate}
    \item Создайте класс `Cache`.
    \item Добавьте приватный атрибут класса `\_cache\_instance` и инициализируйте его значением `None`.
    \item Переопределите метод `\_\_new\_\_`, чтобы он обеспечивал единственный экземпляр.
    \item Переопределите метод `\_\_init\_\_`, принимающий `max\_size`, `ttl`, `strategy`. Устанавливает атрибуты, только если они еще не заданы.
    \item Добавьте метод `put`, который принимает `key`, `value` и выводит "Ключ '\{key\}' закеширован (стратегия: \{strategy\})".
    \item Добавьте метод `\_\_del\_\_`, который выводит "Кеш очищен".
    \item Добавьте метод `get`, который принимает `key` и возвращает "Значение для \{key\}".
    \item Добавьте метод `clear`, который выводит "Кеш принудительно очищен".
    \item Создайте два экземпляра `cache1` и `cache2` с разными параметрами.
    \item Вызовите `put` для `cache1`, затем `clear` для `cache2`.
    \item Проверьте, что `cache1.max\_size == cache2.max\_size`.
\end{enumerate}

Пример использования:
\begin{lstlisting}[language=Python]
cache1 = Cache(1000, 3600, "LRU")
cache2 = Cache(500, 1800, "FIFO")

cache1.put("user\_123", \{"name": "Alice"\})
cache2.clear()  # Очищает кеш cache1 тоже

print("Max size:", cache1.max\_size)  # 1000 (от первого вызова)
\end{lstlisting}

\item Написать программу на Python, которая создает класс `SessionHandler` с использованием метода `\_\_new\_\_` для реализации паттерна Singleton. Программа должна принимать параметры `session\_id`, `timeout`, `secure` при создании экземпляра.

Инструкции:
\begin{enumerate}
    \item Создайте класс `SessionHandler`.
    \item Добавьте приватный атрибут класса `\_handler` и инициализируйте его значением `None`.
    \item Переопределите метод `\_\_new\_\_`, чтобы он возвращал единственный экземпляр.
    \item Переопределите метод `\_\_init\_\_`, принимающий `session\_id`, `timeout`, `secure`. Устанавливает атрибуты, только если они еще не заданы.
    \item Добавьте метод `start`, который выводит "Сессия \{session\_id\} начата (timeout=\{timeout\}, secure=\{secure\})".
    \item Добавьте метод `\_\_del\_\_`, который выводит "Сессия завершена".
    \item Добавьте метод `get\_session\_data`, который возвращает "Данные сессии".
    \item Добавьте метод `invalidate`, который выводит "Сессия аннулирована".
    \item Создайте два экземпляра `sh1` и `sh2` с разными параметрами.
    \item Вызовите `start` для `sh1`, затем `invalidate` для `sh2`.
    \item Выведите `sh1.session\_id` и `sh2.session\_id` — они должны быть одинаковыми.
\end{enumerate}

Пример использования:
\begin{lstlisting}[language=Python]
sh1 = SessionHandler("SID-001", 1800, True)
sh2 = SessionHandler("SID-999", 600, False)

sh1.start()
sh2.invalidate()  # Аннулирует сессию sh1

print("Session ID sh1:", sh1.session\_id)  # SID-001
print("Session ID sh2:", sh2.session\_id)  # SID-001
\end{lstlisting}

\item Написать программу на Python, которая создает класс `ResourceManager` с использованием метода `\_\_new\_\_` для реализации паттерна Singleton. Программа должна принимать параметры `resource\_type`, `capacity`, `priority` при создании экземпляра.

Инструкции:
\begin{enumerate}
    \item Создайте класс `ResourceManager`.
    \item Добавьте приватный атрибут класса `\_manager` и инициализируйте его значением `None`.
    \item Переопределите метод `\_\_new\_\_`, чтобы он возвращал единственный экземпляр.
    \item Переопределите метод `\_\_init\_\_`, принимающий `resource\_type`, `capacity`, `priority`. Устанавливает атрибуты, только если они еще не заданы.
    \item Добавьте метод `allocate`, который выводит "Выделено \{capacity\} ресурсов типа \{resource\_type\} (приоритет: \{priority\})".
    \item Добавьте метод `\_\_del\_\_`, который выводит "Освобождение ресурсов".
    \item Добавьте метод `status`, который возвращает "Ресурсы доступны".
    \item Добавьте метод `release`, который выводит "Ресурсы освобождены".
    \item Создайте два экземпляра `rm1` и `rm2` с разными параметрами.
    \item Вызовите `allocate` для `rm1`, затем `release` для `rm2`.
    \item Проверьте, что `rm1.capacity == rm2.capacity`.
\end{enumerate}

Пример использования:
\begin{lstlisting}[language=Python]
rm1 = ResourceManager("CPU", 4, 1)
rm2 = ResourceManager("GPU", 2, 2)

rm1.allocate()
rm2.release()  # Освобождает ресурсы rm1

print("Capacity:", rm1.capacity)  # 4
\end{lstlisting}

\item Написать программу на Python, которая создает класс `PrinterPool` с использованием метода `\_\_new\_\_` для реализации паттерна Singleton. Программа должна принимать параметры `printer\_id`, `speed`, `color` при создании экземпляра.

Инструкции:
\begin{enumerate}
    \item Создайте класс `PrinterPool`.
    \item Добавьте приватный атрибут класса `\_pool` и инициализируйте его значением `None`.
    \item Переопределите метод `\_\_new\_\_`, чтобы он возвращал единственный экземпляр.
    \item Переопределите метод `\_\_init\_\_`, принимающий `printer\_id`, `speed`, `color`. Устанавливает атрибуты, только если они еще не заданы.
    \item Добавьте метод `print`, который выводит "Печать документа на принтере \{printer\_id\} (скорость: \{speed\}, цвет: \{color\})".
    \item Добавьте метод `\_\_del\_\_`, который выводит "Принтер выключен".
    \item Добавьте метод `get\_status`, который возвращает "Готов к печати".
    \item Добавьте метод `add\_job`, который принимает `job` и выводит "Добавлено задание: \{job\}".
    \item Создайте два экземпляра `pp1` и `pp2` с разными параметрами.
    \item Вызовите `print` для `pp1`, затем `add\_job` для `pp2`.
    \item Проверьте, что `pp1.speed == pp2.speed`.
\end{enumerate}

Пример использования:
\begin{lstlisting}[language=Python]
pp1 = PrinterPool("P100", 10, "Yes")
pp2 = PrinterPool("P200", 8, "No")

pp1.print()
pp2.add\_job("Report.pdf")  # Добавляет задание для pp1

print("Speed:", pp1.speed)  # 10
\end{lstlisting}

\item Написать программу на Python, которая создает класс `NetworkInterface` с использованием метода `\_\_new\_\_` для реализации паттерна Singleton. Программа должна принимать параметры `interface\_name`, `ip`, `mac` при создании экземпляра.

Инструкции:
\begin{enumerate}
    \item Создайте класс `NetworkInterface`.
    \item Добавьте приватный атрибут класса `\_interface` и инициализируйте его значением `None`.
    \item Переопределите метод `\_\_new\_\_`, чтобы он возвращал единственный экземпляр.
    \item Переопределите метод `\_\_init\_\_`, принимающий `interface\_name`, `ip`, `mac`. Устанавливает атрибуты, только если они еще не заданы.
    \item Добавьте метод `connect`, который выводит "Подключение к сети через \{interface\_name\} (\{ip\}, \{mac\})".
    \item Добавьте метод `\_\_del\_\_`, который выводит "Отключение от сети".
    \item Добавьте метод `ping`, который возвращает "Пинг успешен".
    \item Добавьте метод `configure`, который принимает `new\_ip` и выводит "IP изменен на \{new\_ip\}".
    \item Создайте два экземпляра `ni1` и `ni2` с разными параметрами.
    \item Вызовите `connect` для `ni1`, затем `configure` для `ni2`.
    \item Проверьте, что `ni1.ip == ni2.ip`.
\end{enumerate}

Пример использования:
\begin{lstlisting}[language=Python]
ni1 = NetworkInterface("eth0", "192.168.1.100", "AA:BB:CC:DD:EE:FF")
ni2 = NetworkInterface("wlan0", "192.168.1.101", "11:22:33:44:55:66")

ni1.connect()
ni2.configure("192.168.1.102")  # Изменяет IP для ni1

print("IP:", ni1.ip)  # 192.168.1.102
\end{lstlisting}

\item Написать программу на Python, которая создает класс `FileManager` с использованием метода `\_\_new\_\_` для реализации паттерна Singleton. Программа должна принимать параметры `path`, `mode`, `buffered` при создании экземпляра.

Инструкции:
\begin{enumerate}
    \item Создайте класс `FileManager`.
    \item Добавьте приватный атрибут класса `\_manager` и инициализируйте его значением `None`.
    \item Переопределите метод `\_\_new\_\_`, чтобы он возвращал единственный экземпляр.
    \item Переопределите метод `\_\_init\_\_`, принимающий `path`, `mode`, `buffered`. Устанавливает атрибуты, только если они еще не заданы.
    \item Добавьте метод `open`, который выводит "Открытие файла '\{path\}' в режиме '\{mode\}' (буферизация: \{buffered\})".
    \item Добавьте метод `\_\_del\_\_`, который выводит "Файл закрыт".
    \item Добавьте метод `read`, который возвращает "Данные прочитаны".
    \item Добавьте метод `write`, который принимает `data` и выводит "Данные '\{data\}' записаны".
    \item Создайте два экземпляра `fm1` и `fm2` с разными параметрами.
    \item Вызовите `open` для `fm1`, затем `write` для `fm2`.
    \item Проверьте, что `fm1.mode == fm2.mode`.
\end{enumerate}

Пример использования:
\begin{lstlisting}[language=Python]
fm1 = FileManager("data.txt", "r", True)
fm2 = FileManager("log.txt", "w", False)

fm1.open()
fm2.write("Hello")  # Записывает в файл fm1

print("Mode:", fm1.mode)  # r
\end{lstlisting}

\item Написать программу на Python, которая создает класс `DatabaseConnector` с использованием метода `\_\_new\_\_` для реализации паттерна Singleton. Программа должна принимать параметры `dbname`, `host`, `port` при создании экземпляра.

Инструкции:
\begin{enumerate}
    \item Создайте класс `DatabaseConnector`.
    \item Добавьте приватный атрибут класса `\_connector` и инициализируйте его значением `None`.
    \item Переопределите метод `\_\_new\_\_`, чтобы он возвращал единственный экземпляр.
    \item Переопределите метод `\_\_init\_\_`, принимающий `dbname`, `host`, `port`. Устанавливает атрибуты, только если они еще не заданы.
    \item Добавьте метод `connect`, который выводит "Подключение к базе данных '\{dbname\}' на \{host\}:\{port\}".
    \item Добавьте метод `\_\_del\_\_`, который выводит "Отключение от базы данных".
    \item Добавьте метод `query`, который возвращает "Запрос выполнен".
    \item Добавьте метод `disconnect`, который выводит "Разрыв соединения".
    \item Создайте два экземпляра `dc1` и `dc2` с разными параметрами.
    \item Вызовите `connect` для `dc1`, затем `disconnect` для `dc2`.
    \item Проверьте, что `dc1.port == dc2.port`.
\end{enumerate}

Пример использования:
\begin{lstlisting}[language=Python]
dc1 = DatabaseConnector("users", "localhost", 5432)
dc2 = DatabaseConnector("products", "db.example.com", 5432)

dc1.connect()
dc2.disconnect()  # Разрывает соединение dc1

print("Port:", dc1.port)  # 5432
\end{lstlisting}

\item Написать программу на Python, которая создает класс `MessageQueue` с использованием метода `\_\_new\_\_` для реализации паттерна Singleton. Программа должна принимать параметры `queue\_name`, `max\_messages`, `timeout` при создании экземпляра.

Инструкции:
\begin{enumerate}
    \item Создайте класс `MessageQueue`.
    \item Добавьте приватный атрибут класса `\_queue` и инициализируйте его значением `None`.
    \item Переопределите метод `\_\_new\_\_`, чтобы он возвращал единственный экземпляр.
    \item Переопределите метод `\_\_init\_\_`, принимающий `queue\_name`, `max\_messages`, `timeout`. Устанавливает атрибуты, только если они еще не заданы.
    \item Добавьте метод `send`, который принимает `message` и выводит "Отправка сообщения '\{message\}' в очередь \{queue\_name\}".
    \item Добавьте метод `\_\_del\_\_`, который выводит "Очередь закрыта".
    \item Добавьте метод `receive`, который возвращает "Сообщение получено".
    \item Добавьте метод `clear`, который выводит "Очередь очищена".
    \item Создайте два экземпляра `mq1` и `mq2` с разными параметрами.
    \item Вызовите `send` для `mq1`, затем `clear` для `mq2`.
    \item Проверьте, что `mq1.max\_messages == mq2.max\_messages`.
\end{enumerate}

Пример использования:
\begin{lstlisting}[language=Python]
mq1 = MessageQueue("orders", 100, 30)
mq2 = MessageQueue("notifications", 50, 60)

mq1.send("New order")
mq2.clear()  # Очищает очередь mq1

print("Max messages:", mq1.max\_messages)  # 100
\end{lstlisting}

\item Написать программу на Python, которая создает класс `StorageDevice` с использованием метода `\_\_new\_\_` для реализации паттерна Singleton. Программа должна принимать параметры `device\_id`, `capacity`, `type` при создании экземпляра.

Инструкции:
\begin{enumerate}
    \item Создайте класс `StorageDevice`.
    \item Добавьте приватный атрибут класса `\_device` и инициализируйте его значением `None`.
    \item Переопределите метод `\_\_new\_\_`, чтобы он возвращал единственный экземпляр.
    \item Переопределите метод `\_\_init\_\_`, принимающий `device\_id`, `capacity`, `type`. Устанавливает атрибуты, только если они еще не заданы.
    \item Добавьте метод `mount`, который выводит "Подключение устройства \{device\_id\} (тип: \{type\}, емкость: \{capacity\})".
    \item Добавьте метод `\_\_del\_\_`, который выводит "Отключение устройства".
    \item Добавьте метод `read`, который возвращает "Чтение данных".
    \item Добавьте метод `write`, который принимает `data` и выводит "Запись данных '\{data\}'".
    \item Создайте два экземпляра `sd1` и `sd2` с разными параметрами.
    \item Вызовите `mount` для `sd1`, затем `write` для `sd2`.
    \item Проверьте, что `sd1.capacity == sd2.capacity`.
\end{enumerate}

Пример использования:
\begin{lstlisting}[language=Python]
sd1 = StorageDevice("SSD-001", 512, "SSD")
sd2 = StorageDevice("HDD-002", 1024, "HDD")

sd1.mount()
sd2.write("File.txt")  # Записывает в устройство sd1

print("Capacity:", sd1.capacity)  # 512
\end{lstlisting}

\item Написать программу на Python, которая создает класс `APIGateway` с использованием метода `\_\_new\_\_` для реализации паттерна Singleton. Программа должна принимать параметры `api\_url`, `token`, `version` при создании экземпляра.

Инструкции:
\begin{enumerate}
    \item Создайте класс `APIGateway`.
    \item Добавьте приватный атрибут класса `\_gateway` и инициализируйте его значением `None`.
    \item Переопределите метод `\_\_new\_\_`, чтобы он возвращал единственный экземпляр.
    \item Переопределите метод `\_\_init\_\_`, принимающий `api\_url`, `token`, `version`. Устанавливает атрибуты, только если они еще не заданы.
    \item Добавьте метод `call`, который принимает `endpoint` и выводит "Вызов API \{endpoint\} на \{api\_url\} (версия: \{version\})".
    \item Добавьте метод `\_\_del\_\_`, который выводит "API шлюз отключен".
    \item Добавьте метод `get`, который возвращает "Данные получены".
    \item Добавьте метод `post`, который принимает `data` и выводит "Отправлено: \{data\}".
    \item Создайте два экземпляра `ag1` и `ag2` с разными параметрами.
    \item Вызовите `call` для `ag1`, затем `post` для `ag2`.
    \item Проверьте, что `ag1.version == ag2.version`.
\end{enumerate}

Пример использования:
\begin{lstlisting}[language=Python]
ag1 = APIGateway("https://api.example.com", "abc123", "v1")
ag2 = APIGateway("https://api.test.com", "def456", "v2")

ag1.call("/users")
ag2.post("Hello")  # Отправляет данные через ag1

print("Version:", ag1.version)  # v2
\end{lstlisting}

\item Написать программу на Python, которая создает класс `TaskScheduler` с использованием метода `\_\_new\_\_` для реализации паттерна Singleton. Программа должна принимать параметры `scheduler\_id`, `interval`, `enabled` при создании экземпляра.

Инструкции:
\begin{enumerate}
    \item Создайте класс `TaskScheduler`.
    \item Добавьте приватный атрибут класса `\_scheduler` и инициализируйте его значением `None`.
    \item Переопределите метод `\_\_new\_\_`, чтобы он возвращал единственный экземпляр.
    \item Переопределите метод `\_\_init\_\_`, принимающий `scheduler\_id`, `interval`, `enabled`. Устанавливает атрибуты, только если они еще не заданы.
    \item Добавьте метод `start`, который выводит "Запуск планировщика \{scheduler\_id\} (интервал: \{interval\}, включен: \{enabled\})".
    \item Добавьте метод `\_\_del\_\_`, который выводит "Планировщик остановлен".
    \item Добавьте метод `schedule`, который принимает `task` и выводит "Запланирована задача: \{task\}".
    \item Добавьте метод `stop`, который выводит "Остановка планировщика".
    \item Создайте два экземпляра `ts1` и `ts2` с разными параметрами.
    \item Вызовите `start` для `ts1`, затем `schedule` для `ts2`.
    \item Проверьте, что `ts1.interval == ts2.interval`.
\end{enumerate}

Пример использования:
\begin{lstlisting}[language=Python]
ts1 = TaskScheduler("daily", 3600, True)
ts2 = TaskScheduler("hourly", 300, False)

ts1.start()
ts2.schedule("Backup")  # Запланирована задача для ts1

print("Interval:", ts1.interval)  # 3600
\end{lstlisting}

\item Написать программу на Python, которая создает класс `ServiceMonitor` с использованием метода `\_\_new\_\_` для реализации паттерна Singleton. Программа должна принимать параметры `service\_name`, `check\_interval`, `threshold` при создании экземпляра.

Инструкции:
\begin{enumerate}
    \item Создайте класс `ServiceMonitor`.
    \item Добавьте приватный атрибут класса `\_monitor` и инициализируйте его значением `None`.
    \item Переопределите метод `\_\_new\_\_`, чтобы он возвращал единственный экземпляр.
    \item Переопределите метод `\_\_init\_\_`, принимающий `service\_name`, `check\_interval`, `threshold`. Устанавливает атрибуты, только если они еще не заданы.
    \item Добавьте метод `start`, который выводит "Мониторинг службы \{service\_name\} запущен (интервал: \{check\_interval\}, порог: \{threshold\})".
    \item Добавьте метод `\_\_del\_\_`, который выводит "Мониторинг остановлен".
    \item Добавьте метод `check`, который возвращает "Проверка завершена".
    \item Добавьте метод `alert`, который принимает `message` и выводит "Алерт: \{message\}".
    \item Создайте два экземпляра `sm1` и `sm2` с разными параметрами.
    \item Вызовите `start` для `sm1`, затем `alert` для `sm2`.
    \item Проверьте, что `sm1.check\_interval == sm2.check\_interval`.
\end{enumerate}

Пример использования:
\begin{lstlisting}[language=Python]
sm1 = ServiceMonitor("web", 60, 0.9)
sm2 = ServiceMonitor("db", 30, 0.8)

sm1.start()
sm2.alert("High load")  # Алерт для sm1

print("Check interval:", sm1.check\_interval)  # 60
\end{lstlisting}

\item Написать программу на Python, которая создает класс `EventBus` с использованием метода `\_\_new\_\_` для реализации паттерна Singleton. Программа должна принимать параметры `bus\_id`, `topic`, `max\_listeners` при создании экземпляра.

Инструкции:
\begin{enumerate}
    \item Создайте класс `EventBus`.
    \item Добавьте приватный атрибут класса `\_bus` и инициализируйте его значением `None`.
    \item Переопределите метод `\_\_new\_\_`, чтобы он возвращал единственный экземпляр.
    \item Переопределите метод `\_\_init\_\_`, принимающий `bus\_id`, `topic`, `max\_listeners`. Устанавливает атрибуты, только если они еще не заданы.
    \item Добавьте метод `publish`, который принимает `event` и выводит "Опубликовано событие '\{event\}' в топик \{topic\}".
    \item Добавьте метод `\_\_del\_\_`, который выводит "Шина событий закрыта".
    \item Добавьте метод `subscribe`, который принимает `listener` и выводит "Подписан слушатель: \{listener\}".
    \item Добавьте метод `unsubscribe`, который принимает `listener` и выводит "Отписка слушателя: \{listener\}".
    \item Создайте два экземпляра `eb1` и `eb2` с разными параметрами.
    \item Вызовите `publish` для `eb1`, затем `subscribe` для `eb2`.
    \item Проверьте, что `eb1.max\_listeners == eb2.max\_listeners`.
\end{enumerate}

Пример использования:
\begin{lstlisting}[language=Python]
eb1 = EventBus("main", "system", 10)
eb2 = EventBus("backup", "alerts", 5)

eb1.publish("Start")
eb2.subscribe("User")  # Подписка на eb1

print("Max listeners:", eb1.max\_listeners)  # 10
\end{lstlisting}

\item Написать программу на Python, которая создает класс `SignalProcessor` с использованием метода `\_\_new\_\_` для реализации паттерна Singleton. Программа должна принимать параметры `processor\_id`, `sample\_rate`, `filter\_type` при создании экземпляра.

Инструкции:
\begin{enumerate}
    \item Создайте класс `SignalProcessor`.
    \item Добавьте приватный атрибут класса `\_processor` и инициализируйте его значением `None`.
    \item Переопределите метод `\_\_new\_\_`, чтобы он возвращал единственный экземпляр.
    \item Переопределите метод `\_\_init\_\_`, принимающий `processor\_id`, `sample\_rate`, `filter\_type`. Устанавливает атрибуты, только если они еще не заданы.
    \item Добавьте метод `process`, который принимает `signal` и выводит "Обработка сигнала '\{signal\}' (частота: \{sample\_rate\}, фильтр: \{filter\_type\})".
    \item Добавьте метод `\_\_del\_\_`, который выводит "Процессор сигналов остановлен".
    \item Добавьте метод `analyze`, который возвращает "Анализ завершен".
    \item Добавьте метод `apply\_filter`, который принимает `filter\_params` и выводит "Применён фильтр с параметрами: \{filter\_params\}".
    \item Создайте два экземпляра `sp1` и `sp2` с разными параметрами.
    \item Вызовите `process` для `sp1`, затем `apply\_filter` для `sp2`.
    \item Проверьте, что `sp1.sample\_rate == sp2.sample\_rate`.
\end{enumerate}

Пример использования:
\begin{lstlisting}[language=Python]
sp1 = SignalProcessor("audio", 44100, "lowpass")
sp2 = SignalProcessor("video", 30000, "bandpass")

sp1.process("sound")
sp2.apply\_filter(\{"cutoff": 1000\})  # Применяет фильтр для sp1

print("Sample rate:", sp1.sample\_rate)  # 44100
\end{lstlisting}

\item Написать программу на Python, которая создает класс `DataPipeline` с использованием метода `\_\_new\_\_` для реализации паттерна Singleton. Программа должна принимать параметры `pipeline\_id`, `source`, `destination` при создании экземпляра.

Инструкции:
\begin{enumerate}
    \item Создайте класс `DataPipeline`.
    \item Добавьте приватный атрибут класса `\_pipeline` и инициализируйте его значением `None`.
    \item Переопределите метод `\_\_new\_\_`, чтобы он возвращал единственный экземпляр.
    \item Переопределите метод `\_\_init\_\_`, принимающий `pipeline\_id`, `source`, `destination`. Устанавливает атрибуты, только если они еще не заданы.
    \item Добавьте метод `start`, который выводит "Запуск потока данных \{pipeline\_id\} (\{source\} → \{destination\})".
    \item Добавьте метод `\_\_del\_\_`, который выводит "Поток данных остановлен".
    \item Добавьте метод `transform`, который возвращает "Трансформация завершена".
    \item Добавьте метод `transfer`, который принимает `data` и выводит "Передача данных '\{data\}'".
    \item Создайте два экземпляра `dp1` и `dp2` с разными параметрами.
    \item Вызовите `start` для `dp1`, затем `transfer` для `dp2`.
    \item Проверьте, что `dp1.destination == dp2.destination`.
\end{enumerate}

Пример использования:
\begin{lstlisting}[language=Python]
dp1 = DataPipeline("etl", "db", "cloud")
dp2 = DataPipeline("backup", "local", "cloud")

dp1.start()
dp2.transfer("records")  # Передача данных через dp1

print("Destination:", dp1.destination)  # cloud
\end{lstlisting}

\item Написать программу на Python, которая создает класс `SecurityGuard` с использованием метода `\_\_new\_\_` для реализации паттерна Singleton. Программа должна принимать параметры `guard\_id`, `access\_level`, `rules` при создании экземпляра.

Инструкции:
\begin{enumerate}
    \item Создайте класс `SecurityGuard`.
    \item Добавьте приватный атрибут класса `\_guard` и инициализируйте его значением `None`.
    \item Переопределите метод `\_\_new\_\_`, чтобы он возвращал единственный экземпляр.
    \item Переопределите метод `\_\_init\_\_`, принимающий `guard\_id`, `access\_level`, `rules`. Устанавливает атрибуты, только если они еще не заданы.
    \item Добавьте метод `authorize`, который принимает `request` и выводит "Авторизация запроса '\{request\}' (уровень: \{access\_level\})".
    \item Добавьте метод `\_\_del\_\_`, который выводит "Система безопасности отключена".
    \item Добавьте метод `audit`, который возвращает "Аудит завершен".
    \item Добавьте метод `block`, который принимает `entity` и выводит "Блокировка сущности: \{entity\}".
    \item Создайте два экземпляра `sg1` и `sg2` с разными параметрами.
    \item Вызовите `authorize` для `sg1`, затем `block` для `sg2`.
    \item Проверьте, что `sg1.access\_level == sg2.access\_level`.
\end{enumerate}

Пример использования:
\begin{lstlisting}[language=Python]
sg1 = SecurityGuard("main", "admin", ["rule1"])
sg2 = SecurityGuard("backup", "user", ["rule2"])

sg1.authorize("login")
sg2.block("malware")  # Блокировка для sg1

print("Access level:", sg1.access\_level)  # admin
\end{lstlisting}

\item Написать программу на Python, которая создает класс `SystemTray` с использованием метода `\_\_new\_\_` для реализации паттерна Singleton. Программа должна принимать параметры `tray\_id`, `icon`, `tooltip` при создании экземпляра.

Инструкции:
\begin{enumerate}
    \item Создайте класс `SystemTray`.
    \item Добавьте приватный атрибут класса `\_tray` и инициализируйте его значением `None`.
    \item Переопределите метод `\_\_new\_\_`, чтобы он возвращал единственный экземпляр.
    \item Переопределите метод `\_\_init\_\_`, принимающий `tray\_id`, `icon`, `tooltip`. Устанавливает атрибуты, только если они еще не заданы.
    \item Добавьте метод `show`, который выводит "Показать значок \{icon\} в трее (подсказка: \{tooltip\})".
    \item Добавьте метод `\_\_del\_\_`, который выводит "Значок скрыт".
    \item Добавьте метод `hide`, который выводит "Скрыть значок".
    \item Добавьте метод `notify`, который принимает `message` и выводит "Уведомление: \{message\}".
    \item Создайте два экземпляра `st1` и `st2` с разными параметрами.
    \item Вызовите `show` для `st1`, затем `notify` для `st2`.
    \item Проверьте, что `st1.icon == st2.icon`.
\end{enumerate}

Пример использования:
\begin{lstlisting}[language=Python]
st1 = SystemTray("app", "app.ico", "My App")
st2 = SystemTray("tool", "tool.ico", "My Tool")

st1.show()
st2.notify("Update available")  # Уведомление для st1

print("Icon:", st1.icon)  # app.ico
\end{lstlisting}

\item Написать программу на Python, которая создает класс `ApplicationLauncher` с использованием метода `\_\_new\_\_` для реализации паттерна Singleton. Программа должна принимать параметры `launcher\_id`, `apps`, `auto\_start` при создании экземпляра.

Инструкции:
\begin{enumerate}
    \item Создайте класс `ApplicationLauncher`.
    \item Добавьте приватный атрибут класса `\_launcher` и инициализируйте его значением `None`.
    \item Переопределите метод `\_\_new\_\_`, чтобы он возвращал единственный экземпляр.
    \item Переопределите метод `\_\_init\_\_`, принимающий `launcher\_id`, `apps`, `auto\_start`. Устанавливает атрибуты, только если они еще не заданы.
    \item Добавьте метод `launch`, который принимает `app` и выводит "Запуск приложения '\{app\}' (автозапуск: \{auto\_start\})".
    \item Добавьте метод `\_\_del\_\_`, который выводит "Запускатор остановлен".
    \item Добавьте метод `list\_apps`, который возвращает "Список приложений: \{apps\}".
    \item Добавьте метод `add\_app`, который принимает `app` и выводит "Добавлено приложение: \{app\}".
    \item Создайте два экземпляра `al1` и `al2` с разными параметрами.
    \item Вызовите `launch` для `al1`, затем `add\_app` для `al2`.
    \item Проверьте, что `al1.apps == al2.apps`.
\end{enumerate}

Пример использования:
\begin{lstlisting}[language=Python]
al1 = ApplicationLauncher("main", ["browser", "editor"], True)
al2 = ApplicationLauncher("backup", ["backup", "sync"], False)

al1.launch("browser")
al2.add\_app("calc")  # Добавляет приложение в al1

print("Apps:", al1.apps)  # ['browser', 'editor', 'calc']
\end{lstlisting}

\item Написать программу на Python, которая создает класс `NetworkScanner` с использованием метода `\_\_new\_\_` для реализации паттерна Singleton. Программа должна принимать параметры `scanner\_id`, `target`, `timeout` при создании экземпляра.

Инструкции:
\begin{enumerate}
    \item Создайте класс `NetworkScanner`.
    \item Добавьте приватный атрибут класса `\_scanner` и инициализируйте его значением `None`.
    \item Переопределите метод `\_\_new\_\_`, чтобы он возвращал единственный экземпляр.
    \item Переопределите метод `\_\_init\_\_`, принимающий `scanner\_id`, `target`, `timeout`. Устанавливает атрибуты, только если они еще не заданы.
    \item Добавьте метод `scan`, который выводит "Сканирование сети \{target\} (таймаут: \{timeout\})".
    \item Добавьте метод `\_\_del\_\_`, который выводит "Сканер остановлен".
    \item Добавьте метод `report`, который возвращает "Отчет о сканировании".
    \item Добавьте метод `ping`, который принимает `host` и выводит "Проверка доступности \{host\}".
    \item Создайте два экземпляра `ns1` и `ns2` с разными параметрами.
    \item Вызовите `scan` для `ns1`, затем `ping` для `ns2`.
    \item Проверьте, что `ns1.timeout == ns2.timeout`.
\end{enumerate}

Пример использования:
\begin{lstlisting}[language=Python]
ns1 = NetworkScanner("fast", "192.168.1.0/24", 1)
ns2 = NetworkScanner("slow", "10.0.0.0/8", 5)

ns1.scan()
ns2.ping("192.168.1.1")  # Проверка для ns1

print("Timeout:", ns1.timeout)  # 1
\end{lstlisting}

\item Написать программу на Python, которая создает класс `HealthChecker` с использованием метода `\_\_new\_\_` для реализации паттерна Singleton. Программа должна принимать параметры `checker\_id`, `services`, `frequency` при создании экземпляра.

Инструкции:
\begin{enumerate}
    \item Создайте класс `HealthChecker`.
    \item Добавьте приватный атрибут класса `\_checker` и инициализируйте его значением `None`.
    \item Переопределите метод `\_\_new\_\_`, чтобы он возвращал единственный экземпляр.
    \item Переопределите метод `\_\_init\_\_`, принимающий `checker\_id`, `services`, `frequency`. Устанавливает атрибуты, только если они еще не заданы.
    \item Добавьте метод `check`, который выводит "Проверка состояния сервисов \{services\} (частота: \{frequency\})".
    \item Добавьте метод `\_\_del\_\_`, который выводит "Проверка здоровья остановлена".
    \item Добавьте метод `status`, который возвращает "Состояние: OK".
    \item Добавьте метод `alert`, который принимает `service` и выводит "Алерт: \{service\} не отвечает".
    \item Создайте два экземпляра `h1` и `h2` с разными параметрами.
    \item Вызовите `check` для `h1`, затем `alert` для `h2`.
    \item Проверьте, что `h1.frequency == h2.frequency`.
\end{enumerate}

Пример использования:
\begin{lstlisting}[language=Python]
h1 = HealthChecker("main", ["web", "db"], 60)
h2 = HealthChecker("backup", ["cache", "redis"], 30)

h1.check()
h2.alert("redis")  # Алерт для h1

print("Frequency:", h1.frequency)  # 60
\end{lstlisting}

\item Написать программу на Python, которая создает класс `PerformanceMonitor` с использованием метода `\_\_new\_\_` для реализации паттерна Singleton. Программа должна принимать параметры `monitor\_id`, `metrics`, `interval` при создании экземпляра.

Инструкции:
\begin{enumerate}
    \item Создайте класс `PerformanceMonitor`.
    \item Добавьте приватный атрибут класса `\_monitor` и инициализируйте его значением `None`.
    \item Переопределите метод `\_\_new\_\_`, чтобы он возвращал единственный экземпляр.
    \item Переопределите метод `\_\_init\_\_`, принимающий `monitor\_id`, `metrics`, `interval`. Устанавливает атрибуты, только если они еще не заданы.
    \item Добавьте метод `start`, который выводит "Мониторинг производительности \{monitor\_id\} запущен (метрики: \{metrics\}, интервал: \{interval\})".
    \item Добавьте метод `\_\_del\_\_`, который выводит "Мониторинг остановлен".
    \item Добавьте метод `collect`, который возвращает "Сбор метрик завершен".
    \item Добавьте метод `report`, который принимает `data` и выводит "Отчет: \{data\}".
    \item Создайте два экземпляра `pm1` и `pm2` с разными параметрами.
    \item Вызовите `start` для `pm1`, затем `report` для `pm2`.
    \item Проверьте, что `pm1.interval == pm2.interval`.
\end{enumerate}

Пример использования:
\begin{lstlisting}[language=Python]
pm1 = PerformanceMonitor("cpu", ["usage", "temp"], 10)
pm2 = PerformanceMonitor("memory", ["ram", "swap"], 5)

pm1.start()
pm2.report("High CPU load")  # Отчет для pm1

print("Interval:", pm1.interval)  # 10
\end{lstlisting}

\item Написать программу на Python, которая создает класс `LogAggregator` с использованием метода `\_\_new\_\_` для реализации паттерна Singleton. Программа должна принимать параметры `aggregator\_id`, `sources`, `format` при создании экземпляра.

Инструкции:
\begin{enumerate}
    \item Создайте класс `LogAggregator`.
    \item Добавьте приватный атрибут класса `\_aggregator` и инициализируйте его значением `None`.
    \item Переопределите метод `\_\_new\_\_`, чтобы он возвращал единственный экземпляр.
    \item Переопределите метод `\_\_init\_\_`, принимающий `aggregator\_id`, `sources`, `format`. Устанавливает атрибуты, только если они еще не заданы.
    \item Добавьте метод `aggregate`, который выводит "Агрегация логов из \{sources\} (формат: \{format\})".
    \item Добавьте метод `\_\_del\_\_`, который выводит "Агрегатор остановлен".
    \item Добавьте метод `forward`, который принимает `logs` и выводит "Передача логов: \{logs\}".
    \item Добавьте метод `filter`, который принимает `criteria` и выводит "Фильтрация по критерию: \{criteria\}".
    \item Создайте два экземпляра `la1` и `la2` с разными параметрами.
    \item Вызовите `aggregate` для `la1`, затем `forward` для `la2`.
    \item Проверьте, что `la1.format == la2.format`.
\end{enumerate}

Пример использования:
\begin{lstlisting}[language=Python]
la1 = LogAggregator("main", ["app", "db"], "json")
la2 = LogAggregator("backup", ["web", "api"], "text")

la1.aggregate()
la2.forward("error logs")  # Передача для la1

print("Format:", la1.format)  # json
\end{lstlisting}

\item Написать программу на Python, которая создает класс `ResourceTracker` с использованием метода `\_\_new\_\_` для реализации паттерна Singleton. Программа должна принимать параметры `tracker\_id`, `resources`, `threshold` при создании экземпляра.

Инструкции:
\begin{enumerate}
    \item Создайте класс `ResourceTracker`.
    \item Добавьте приватный атрибут класса `\_tracker` и инициализируйте его значением `None`.
    \item Переопределите метод `\_\_new\_\_`, чтобы он возвращал единственный экземпляр.
    \item Переопределите метод `\_\_init\_\_`, принимающий `tracker\_id`, `resources`, `threshold`. Устанавливает атрибуты, только если они еще не заданы.
    \item Добавьте метод `track`, который выводит "Отслеживание ресурсов \{resources\} (порог: \{threshold\})".
    \item Добавьте метод `\_\_del\_\_`, который выводит "Отслеживание остановлено".
    \item Добавьте метод `update`, который принимает `data` и выводит "Обновление данных: \{data\}".
    \item Добавьте метод `alarm`, который принимает `resource` и выводит "Авария: \{resource\} исчерпан".
    \item Создайте два экземпляра `rt1` и `rt2` с разными параметрами.
    \item Вызовите `track` для `rt1`, затем `alarm` для `rt2`.
    \item Проверьте, что `rt1.threshold == rt2.threshold`.
\end{enumerate}

Пример использования:
\begin{lstlisting}[language=Python]
rt1 = ResourceTracker("cpu", ["cores", "freq"], 0.8)
rt2 = ResourceTracker("memory", ["ram", "swap"], 0.9)

rt1.track()
rt2.alarm("ram")  # Авария для rt1

print("Threshold:", rt1.threshold)  # 0.8
\end{lstlisting}

\item Написать программу на Python, которая создает класс `NotificationCenter` с использованием метода `\_\_new\_\_` для реализации паттерна Singleton. Программа должна принимать параметры `center\_id`, `channels`, `priority` при создании экземпляра.

Инструкции:
\begin{enumerate}
    \item Создайте класс `NotificationCenter`.
    \item Добавьте приватный атрибут класса `\_center` и инициализируйте его значением `None`.
    \item Переопределите метод `\_\_new\_\_`, чтобы он возвращал единственный экземпляр.
    \item Переопределите метод `\_\_init\_\_`, принимающий `center\_id`, `channels`, `priority`. Устанавливает атрибуты, только если они еще не заданы.
    \item Добавьте метод `notify`, который принимает `message` и выводит "Уведомление: \{message\} (каналы: \{channels\}, приоритет: \{priority\})".
    \item Добавьте метод `\_\_del\_\_`, который выводит "Центр уведомлений остановлен".
    \item Добавьте метод `subscribe`, который принимает `channel` и выводит "Подписан на канал: \{channel\}".
    \item Добавьте метод `unsubcribe`, который принимает `channel` и выводит "Отписка от канала: \{channel\}".
    \item Создайте два экземпляра `nc1` и `nc2` с разными параметрами.
    \item Вызовите `notify` для `nc1`, затем `subscribe` для `nc2`.
    \item Проверьте, что `nc1.priority == nc2.priority`.
\end{enumerate}

Пример использования:
\begin{lstlisting}[language=Python]
nc1 = NotificationCenter("main", ["email", "push"], 1)
nc2 = NotificationCenter("backup", ["sms", "phone"], 2)

nc1.notify("Update ready")
nc2.subscribe("email")  # Подписка на nc1

print("Priority:", nc1.priority)  # 1
\end{lstlisting}

\item Написать программу на Python, которая создает класс `ConfigurationManager` с использованием метода `\_\_new\_\_` для реализации паттерна Singleton. Программа должна принимать параметры `manager\_id`, `config\_files`, `reload\_on\_change` при создании экземпляра.

Инструкции:
\begin{enumerate}
    \item Создайте класс `ConfigurationManager`.
    \item Добавьте приватный атрибут класса `\_manager` и инициализируйте его значением `None`.
    \item Переопределите метод `\_\_new\_\_`, чтобы он возвращал единственный экземпляр.
    \item Переопределите метод `\_\_init\_\_`, принимающий `manager\_id`, `config\_files`, `reload\_on\_change`. Устанавливает атрибуты, только если они еще не заданы.
    \item Добавьте метод `load`, который выводит "Загрузка конфигураций из \{config\_files\} (перезагрузка при изменении: \{reload\_on\_change\})".
    \item Добавьте метод `\_\_del\_\_`, который выводит "Конфигурации выгружены".
    \item Добавьте метод `get`, который принимает `key` и возвращает "Значение для \{key\}".
    \item Добавьте метод `set`, который принимает `key`, `value` и выводит "Настройка \{key\} установлена в \{value\}".
    \item Создайте два экземпляра `cm1` и `cm2` с разными параметрами.
    \item Вызовите `load` для `cm1`, затем `set` для `cm2`.
    \item Проверьте, что `cm1.reload\_on\_change == cm2.reload\_on\_change`.
\end{enumerate}

Пример использования:
\begin{lstlisting}[language=Python]
cm1 = ConfigurationManager("app", ["config.yaml"], True)
cm2 = ConfigurationManager("test", ["test.conf"], False)

cm1.load()
cm2.set("debug", True)  # Установка для cm1

print("Reload on change:", cm1.reload\_on\_change)  # True
\end{lstlisting}

\item Написать программу на Python, которая создает класс `JobScheduler` с использованием метода `\_\_new\_\_` для реализации паттерна Singleton. Программа должна принимать параметры `scheduler\_id`, `jobs`, `time\_zone` при создании экземпляра.

Инструкции:
\begin{enumerate}
    \item Создайте класс `JobScheduler`.
    \item Добавьте приватный атрибут класса `\_scheduler` и инициализируйте его значением `None`.
    \item Переопределите метод `\_\_new\_\_`, чтобы он возвращал единственный экземпляр.
    \item Переопределите метод `\_\_init\_\_`, принимающий `scheduler\_id`, `jobs`, `time\_zone`. Устанавливает атрибуты, только если они еще не заданы.
    \item Добавьте метод `schedule`, который выводит "Запланированы задания \{jobs\} (часовой пояс: \{time\_zone\})".
    \item Добавьте метод `\_\_del\_\_`, который выводит "Планировщик остановлен".
    \item Добавьте метод `run`, который возвращает "Выполнение заданий".
    \item Добавьте метод `cancel`, который принимает `job` и выводит "Отмена задания: \{job\}".
    \item Создайте два экземпляра `js1` и `js2` с разными параметрами.
    \item Вызовите `schedule` для `js1`, затем `cancel` для `js2`.
    \item Проверьте, что `js1.time\_zone == js2.time\_zone`.
\end{enumerate}

Пример использования:
\begin{lstlisting}[language=Python]
js1 = JobScheduler("daily", ["backup", "cleanup"], "UTC")
js2 = JobScheduler("weekly", ["report", "archive"], "Europe/Moscow")

js1.schedule()
js2.cancel("report")  # Отмена для js1

print("Time zone:", js1.time\_zone)  # UTC
\end{lstlisting}

\item Написать программу на Python, которая создает класс `AnalyticsEngine` с использованием метода `\_\_new\_\_` для реализации паттерна Singleton. Программа должна принимать параметры `engine\_id`, `datasets`, `model` при создании экземпляра.

Инструкции:
\begin{enumerate}
    \item Создайте класс `AnalyticsEngine`.
    \item Добавьте приватный атрибут класса `\_engine` и инициализируйте его значением `None`.
    \item Переопределите метод `\_\_new\_\_`, чтобы он возвращал единственный экземпляр.
    \item Переопределите метод `\_\_init\_\_`, принимающий `engine\_id`, `datasets`, `model`. Устанавливает атрибуты, только если они еще не заданы.
    \item Добавьте метод `analyze`, который выводит "Анализ данных \{datasets\} с моделью \{model\}".
    \item Добавьте метод `\_\_del\_\_`, который выводит "Аналитический движок остановлен".
    \item Добавьте метод `train`, который возвращает "Обучение модели завершено".
    \item Добавьте метод `predict`, который принимает `data` и выводит "Прогноз на основе данных: \{data\}".
    \item Создайте два экземпляра `ae1` и `ae2` с разными параметрами.
    \item Вызовите `analyze` для `ae1`, затем `predict` для `ae2`.
    \item Проверьте, что `ae1.model == ae2.model`.
\end{enumerate}

Пример использования:
\begin{lstlisting}[language=Python]
ae1 = AnalyticsEngine("sales", ["orders", "customers"], "linear")
ae2 = AnalyticsEngine("marketing", ["ads", "clicks"], "neural")

ae1.analyze()
ae2.predict("next month")  # Прогноз для ae1

print("Model:", ae1.model)  # linear
\end{lstlisting}

\item Написать программу на Python, которая создает класс `AuditTrail` с использованием метода `\_\_new\_\_` для реализации паттерна Singleton. Программа должна принимать параметры `trail\_id`, `events`, `retention` при создании экземпляра.

Инструкции:
\begin{enumerate}
    \item Создайте класс `AuditTrail`.
    \item Добавьте приватный атрибут класса `\_trail` и инициализируйте его значением `None`.
    \item Переопределите метод `\_\_new\_\_`, чтобы он возвращал единственный экземпляр.
    \item Переопределите метод `\_\_init\_\_`, принимающий `trail\_id`, `events`, `retention`. Устанавливает атрибуты, только если они еще не заданы.
    \item Добавьте метод `log`, который принимает `action` и выводит "Запись действия '\{action\}' в журнал (события: \{events\}, срок хранения: \{retention\})".
    \item Добавьте метод `\_\_del\_\_`, который выводит "Журнал аудита закрыт".
    \item Добавьте метод `search`, который принимает `query` и возвращает "Результаты поиска: \{query\}".
    \item Добавьте метод `purge`, который выводит "Очистка журнала".
    \item Создайте два экземпляра `at1` и `at2` с разными параметрами.
    \item Вызовите `log` для `at1`, затем `search` для `at2`.
    \item Проверьте, что `at1.retention == at2.retention`.
\end{enumerate}

Пример использования:
\begin{lstlisting}[language=Python]
at1 = AuditTrail("security", ["login", "logout"], 365)
at2 = AuditTrail("operations", ["start", "stop"], 90)

at1.log("User login")
at2.search("logout")  # Поиск в at1

print("Retention:", at1.retention)  # 365
\end{lstlisting}

\item Написать программу на Python, которая создает класс `ContentFilter` с использованием метода `\_\_new\_\_` для реализации паттерна Singleton. Программа должна принимать параметры `filter\_id`, `rules`, `strict` при создании экземпляра.

Инструкции:
\begin{enumerate}
    \item Создайте класс `ContentFilter`.
    \item Добавьте приватный атрибут класса `\_filter` и инициализируйте его значением `None`.
    \item Переопределите метод `\_\_new\_\_`, чтобы он возвращал единственный экземпляр.
    \item Переопределите метод `\_\_init\_\_`, принимающий `filter\_id`, `rules`, `strict`. Устанавливает атрибуты, только если они еще не заданы.
    \item Добавьте метод `filter`, который принимает `content` и выводит "Фильтрация контента '\{content\}' (правила: \{rules\}, строгий режим: \{strict\})".
    \item Добавьте метод `\_\_del\_\_`, который выводит "Фильтр отключен".
    \item Добавьте метод `get\_rules`, который возвращает "Правила: \{rules\}".
    \item Добавьте метод `add\_rule`, который принимает `rule` и выводит "Добавлено правило: \{rule\}".
    \item Создайте два экземпляра `cf1` и `cf2` с разными параметрами.
    \item Вызовите `filter` для `cf1`, затем `add\_rule` для `cf2`.
    \item Проверьте, что `cf1.strict == cf2.strict`.
\end{enumerate}

Пример использования:
\begin{lstlisting}[language=Python]
cf1 = ContentFilter("main", ["bad", "spam"], True)
cf2 = ContentFilter("backup", ["offensive", "inappropriate"], False)

cf1.filter("This is spam")
cf2.add\_rule("hate")  # Добавление правила для cf1

print("Strict:", cf1.strict)  # True
\end{lstlisting}

\item Написать программу на Python, которая создает класс `RateLimiter` с использованием метода `\_\_new\_\_` для реализации паттерна Singleton. Программа должна принимать параметры `limiter\_id`, `rate`, `burst` при создании экземпляра.

Инструкции:
\begin{enumerate}
    \item Создайте класс `RateLimiter`.
    \item Добавьте приватный атрибут класса `\_limiter` и инициализируйте его значением `None`.
    \item Переопределите метод `\_\_new\_\_`, чтобы он возвращал единственный экземпляр.
    \item Переопределите метод `\_\_init\_\_`, принимающий `limiter\_id`, `rate`, `burst`. Устанавливает атрибуты, только если они еще не заданы.
    \item Добавьте метод `limit`, который принимает `request` и выводит "Ограничение запроса '\{request\}' (скорость: \{rate\}, burst: \{burst\})".
    \item Добавьте метод `\_\_del\_\_`, который выводит "Ограничитель отключен".
    \item Добавьте метод `allow`, который возвращает "Запрос разрешен".
    \item Добавьте метод `deny`, который принимает `reason` и выводит "Запрос отклонен: \{reason\}".
    \item Создайте два экземпляра `rl1` и `rl2` с разными параметрами.
    \item Вызовите `limit` для `rl1`, затем `deny` для `rl2`.
    \item Проверьте, что `rl1.rate == rl2.rate`.
\end{enumerate}

Пример использования:
\begin{lstlisting}[language=Python]
rl1 = RateLimiter("api", 10, 5)
rl2 = RateLimiter("web", 5, 3)

rl1.limit("GET /users")
rl2.deny("Too many requests")  # Отклонение для rl1

print("Rate:", rl1.rate)  # 10
\end{lstlisting}

\item Написать программу на Python, которая создает класс `CacheManager` с использованием метода `\_\_new\_\_` для реализации паттерна Singleton. Программа должна принимать параметры `manager\_id`, `cache\_size`, `eviction\_policy` при создании экземпляра.

Инструкции:
\begin{enumerate}
    \item Создайте класс `CacheManager`.
    \item Добавьте приватный атрибут класса `\_manager` и инициализируйте его значением `None`.
    \item Переопределите метод `\_\_new\_\_`, чтобы он возвращал единственный экземпляр.
    \item Переопределите метод `\_\_init\_\_`, принимающий `manager\_id`, `cache\_size`, `eviction\_policy`. Устанавливает атрибуты, только если они еще не заданы.
    \item Добавьте метод `put`, который принимает `key`, `value` и выводит "Кэширование ключа '\{key\}' (размер: \{cache\_size\}, политика: \{eviction\_policy\})".
    \item Добавьте метод `\_\_del\_\_`, который выводит "Кэш очищен".
    \item Добавьте метод `get`, который принимает `key` и возвращает "Значение для \{key\}".
    \item Добавьте метод `evict`, который выводит "Освобождение места в кэше".
    \item Создайте два экземпляра `cm1` и `cm2` с разными параметрами.
    \item Вызовите `put` для `cm1`, затем `evict` для `cm2`.
    \item Проверьте, что `cm1.cache\_size == cm2.cache\_size`.
\end{enumerate}

Пример использования:
\begin{lstlisting}[language=Python]
cm1 = CacheManager("main", 1000, "LRU")
cm2 = CacheManager("backup", 500, "FIFO")

cm1.put("user\_123", \{"name": "Alice"\})
cm2.evict()  # Освобождение для cm1

print("Cache size:", cm1.cache\_size)  # 1000
\end{lstlisting}

\end{enumerate}
\subsubsection{Задача 2 (ограничение количества экземпляров)}
\begin{enumerate}
\item Написать программу на Python, которая создает класс `LimitedInstances` с использованием метода `\_\_new\_\_` для ограничения количества создаваемых экземпляров до 5.

Инструкции:
\begin{enumerate}
    \item Создайте класс `LimitedInstances`.
    \item Добавьте атрибут класса `\_instances` и инициализируйте его пустым списком.
    \item Добавьте атрибут класса `\_limit` и инициализируйте его значением 5.
    \item Переопределите метод `\_\_new\_\_`. Если `len(\_instances) >= \_limit`, выбросьте `RuntimeError("Превышен лимит объектов: 5")`. Иначе, создайте экземпляр с помощью `super().\_\_new\_\_(cls)`, добавьте его в `\_instances` и верните.
    \item Переопределите метод `\_\_del\_\_`, чтобы он удалял `self` из `\_instances` при уничтожении объекта.
    \item Переопределите метод `\_\_init\_\_`, который принимает `name` и устанавливает `self.name = name`.
    \item Создайте 5 экземпляров класса.
    \item Попытайтесь создать 6-й экземпляр - должно возникнуть исключение `RuntimeError`.
    \item Удалите один из первых 5 экземпляров (например, `del obj1`).
    \item Создайте 6-й экземпляр - теперь это должно сработать.
\end{enumerate}

Пример использования:
\begin{lstlisting}[language=Python]
# Создаем 5 объектов
objs = [LimitedInstances(f"Obj{i}") for i in range(1, 6)]

# Попытка создать 6-й - вызовет ошибку
try:
    obj6 = LimitedInstances("Obj6")
except RuntimeError as e:
    print(e)

# Удаляем один объект
del objs[0]

# Теперь можно создать 6-й
obj6 = LimitedInstances("Obj6")
print("Успешно создан 6-й объект:", obj6.name)
\end{lstlisting}

\item Написать программу на Python, которая создает класс `BoundedObjects` с использованием метода `\_\_new\_\_` для ограничения количества создаваемых экземпляров до 3.

Инструкции:
\begin{enumerate}
    \item Создайте класс `BoundedObjects`.
    \item Добавьте атрибут класса `\_pool` и инициализируйте его пустым списком.
    \item Добавьте атрибут класса `MAX\_OBJECTS` и инициализируйте его значением 3.
    \item Переопределите метод `\_\_new\_\_`. Если `len(\_pool) >= MAX\_OBJECTS`, выбросьте `RuntimeError("Максимум 3 объекта!")`. Иначе, создайте экземпляр, добавьте в `\_pool`, верните.
    \item Переопределите метод `\_\_del\_\_`, чтобы он удалял `self` из `\_pool`.
    \item Переопределите метод `\_\_init\_\_`, который принимает `id` и устанавливает `self.object\_id = id`.
    \item Создайте 3 экземпляра.
    \item Попытайтесь создать 4-й - поймайте и выведите исключение.
    \item Удалите один экземпляр.
    \item Создайте 4-й экземпляр - должно сработать.
\end{enumerate}

Пример использования:
\begin{lstlisting}[language=Python]
# Создаем 3 объекта
obj1 = BoundedObjects(1)
obj2 = BoundedObjects(2)
obj3 = BoundedObjects(3)

# Попытка создать 4-й
try:
    obj4 = BoundedObjects(4)
except RuntimeError as e:
    print("Ошибка:", e)

# Удаляем один
del obj1

# Создаем 4-й - успешно
obj4 = BoundedObjects(4)
print("ID нового объекта:", obj4.object_id)
\end{lstlisting}

\item Написать программу на Python, которая создает класс `ResourcePool` с использованием метода `\_\_new\_\_` для ограничения количества создаваемых экземпляров до 10.

Инструкции:
\begin{enumerate}
    \item Создайте класс `ResourcePool`.
    \item Добавьте атрибут класса `\_allocated` и инициализируйте его пустым списком.
    \item Добавьте атрибут класса `CAPACITY` и инициализируйте его значением 10.
    \item Переопределите метод `\_\_new\_\_`. Если `len(\_allocated) >= CAPACITY`, выбросьте `RuntimeError("Ресурсы исчерпаны!")`. Иначе, создайте экземпляр, добавьте в `\_allocated`, верните.
    \item Переопределите метод `\_\_del\_\_`, чтобы он удалял `self` из `\_allocated`.
    \item Переопределите метод `\_\_init\_\_`, который принимает `resource\_type` и устанавливает `self.type = resource\_type`.
    \item Создайте 10 экземпляров.
    \item Попытайтесь создать 11-й - поймайте и выведите исключение.
    \item Удалите два экземпляра.
    \item Создайте 11-й и 12-й экземпляры - должно сработать.
\end{enumerate}

Пример использования:
\begin{lstlisting}[language=Python]
# Создаем 10 объектов
resources = [ResourcePool(f"Type{i}") for i in range(10)]

# Попытка создать 11-й
try:
    r11 = ResourcePool("Type11")
except RuntimeError as e:
    print(e)

# Удаляем два
del resources[0], resources[1]

# Создаем 11-й и 12-й - успешно
r11 = ResourcePool("Type11")
r12 = ResourcePool("Type12")
print("Созданы:", r11.type, r12.type)
\end{lstlisting}

\item Написать программу на Python, которая создает класс `CarFleet` с использованием метода `\_\_new\_\_` для ограничения количества создаваемых экземпляров до 7.

Инструкции:
\begin{enumerate}
    \item Создайте класс `CarFleet`.
    \item Добавьте атрибут класса `\_cars` и инициализируйте его пустым списком.
    \item Добавьте атрибут класса `FLEET\_SIZE` и инициализируйте его значением 7.
    \item Переопределите метод `\_\_new\_\_`. Если `len(\_cars) >= FLEET\_SIZE`, выбросьте `RuntimeError("Автопарк переполнен!")`. Иначе, создайте экземпляр, добавьте в `\_cars`, верните.
    \item Переопределите метод `\_\_del\_\_`, чтобы он удалял `self` из `\_cars`.
    \item Переопределите метод `\_\_init\_\_`, который принимает `model` и устанавливает `self.model = model`.
    \item Создайте 7 экземпляров.
    \item Попытайтесь создать 8-й - поймайте и выведите исключение.
    \item Удалите три экземпляра.
    \item Создайте 8-й, 9-й и 10-й экземпляры - должно сработать.
\end{enumerate}

Пример использования:
\begin{lstlisting}[language=Python]
# Создаем 7 машин
fleet = [CarFleet(f"Model{i}") for i in range(7)]

# Попытка создать 8-ю
try:
    car8 = CarFleet("Model8")
except RuntimeError as e:
    print("Ошибка:", e)

# Удаляем три
del fleet[0], fleet[1], fleet[2]

# Создаем 8-ю, 9-ю, 10-ю - успешно
car8 = CarFleet("Model8")
car9 = CarFleet("Model9")
car10 = CarFleet("Model10")
print("Новые модели:", car8.model, car9.model, car10.model)
\end{lstlisting}

\item Написать программу на Python, которая создает класс `StudentGroup` с использованием метода `\_\_new\_\_` для ограничения количества создаваемых экземпляров до 30.

Инструкции:
\begin{enumerate}
    \item Создайте класс `StudentGroup`.
    \item Добавьте атрибут класса `\_students` и инициализируйте его пустым списком.
    \item Добавьте атрибут класса `GROUP\_MAX` и инициализируйте его значением 30.
    \item Переопределите метод `\_\_new\_\_`. Если `len(\_students) >= GROUP\_MAX`, выбросьте `RuntimeError("Группа заполнена!")`. Иначе, создайте экземпляр, добавьте в `\_students`, верните.
    \item Переопределите метод `\_\_del\_\_`, чтобы он удалял `self` из `\_students`.
    \item Переопределите метод `\_\_init\_\_`, который принимает `student\_name` и устанавливает `self.name = student\_name`.
    \item Создайте 30 экземпляров.
    \item Попытайтесь создать 31-й - поймайте и выведите исключение.
    \item Удалите пять экземпляров.
    \item Создайте 31-й, 32-й, 33-й, 34-й, 35-й экземпляры - должно сработать.
\end{enumerate}

Пример использования:
\begin{lstlisting}[language=Python]
# Создаем 30 студентов
students = [StudentGroup(f"Student{i}") for i in range(30)]

# Попытка создать 31-го
try:
    s31 = StudentGroup("Alice")
except RuntimeError as e:
    print("Ошибка:", e)

# Удаляем пять
for i in range(5):
    del students[0]

# Создаем 31-го, 32-го, 33-го, 34-го, 35-го - успешно
new_students = [StudentGroup(f"New{i}") for i in range(31, 36)]
for s in new_students:
    print("Добавлен:", s.name)
\end{lstlisting}

\item Написать программу на Python, которая создает класс `TaskQueue` с использованием метода `\_\_new\_\_` для ограничения количества создаваемых экземпляров до 100.

Инструкции:
\begin{enumerate}
    \item Создайте класс `TaskQueue`.
    \item Добавьте атрибут класса `\_tasks` и инициализируйте его пустым списком.
    \item Добавьте атрибут класса `QUEUE\_LIMIT` и инициализируйте его значением 100.
    \item Переопределите метод `\_\_new\_\_`. Если `len(\_tasks) >= QUEUE\_LIMIT`, выбросьте `RuntimeError("Очередь задач переполнена!")`. Иначе, создайте экземпляр, добавьте в `\_tasks`, верните.
    \item Переопределите метод `\_\_del\_\_`, чтобы он удалял `self` из `\_tasks`.
    \item Переопределите метод `\_\_init\_\_`, который принимает `task\_name` и устанавливает `self.task = task\_name`.
    \item Создайте 100 экземпляров.
    \item Попытайтесь создать 101-й - поймайте и выведите исключение.
    \item Удалите десять экземпляров.
    \item Создайте 101-й, 102-й, ..., 110-й экземпляры - должно сработать.
\end{enumerate}

Пример использования:
\begin{lstlisting}[language=Python]
# Создаем 100 задач
tasks = [TaskQueue(f"Task{i}") for i in range(100)]

# Попытка создать 101-ю
try:
    t101 = TaskQueue("FinalTask")
except RuntimeError as e:
    print("Ошибка:", e)

# Удаляем 10 задач
for i in range(10):
    del tasks[0]

# Создаем 101-ю, 102-ю, ..., 110-ю - успешно
new_tasks = [TaskQueue(f"NewTask{i}") for i in range(101, 111)]
for t in new_tasks:
    print("Добавлена задача:", t.task)
\end{lstlisting}

\item Написать программу на Python, которая создает класс `ConnectionPool` с использованием метода `\_\_new\_\_` для ограничения количества создаваемых экземпляров до 8.

Инструкции:
\begin{enumerate}
    \item Создайте класс `ConnectionPool`.
    \item Добавьте атрибут класса `\_connections` и инициализируйте его пустым списком.
    \item Добавьте атрибут класса `POOL\_SIZE` и инициализируйте его значением 8.
    \item Переопределите метод `\_\_new\_\_`. Если `len(\_connections) >= POOL\_SIZE`, выбросьте `RuntimeError("Пул соединений полон!")`. Иначе, создайте экземпляр, добавьте в `\_connections`, верните.
    \item Переопределите метод `\_\_del\_\_`, чтобы он удалял `self` из `\_connections`.
    \item Переопределите метод `\_\_init\_\_`, который принимает `connection\_id` и устанавливает `self.id = connection\_id`.
    \item Создайте 8 экземпляров.
    \item Попытайтесь создать 9-й - поймайте и выведите исключение.
    \item Удалите четыре экземпляра.
    \item Создайте 9-й, 10-й, 11-й, 12-й экземпляры - должно сработать.
\end{enumerate}

Пример использования:
\begin{lstlisting}[language=Python]
# Создаем 8 соединений
pool = [ConnectionPool(f"Conn{i}") for i in range(8)]

# Попытка создать 9-е
try:
    conn9 = ConnectionPool("Conn9")
except RuntimeError as e:
    print("Ошибка:", e)

# Удаляем 4
del pool[0], pool[1], pool[2], pool[3]

# Создаем 9-е, 10-е, 11-е, 12-е - успешно
new_conns = [ConnectionPool(f"Conn{i}") for i in range(9, 13)]
for c in new_conns:
    print("Создано соединение:", c.id)
\end{lstlisting}

\item Написать программу на Python, которая создает класс `DeviceManager` с использованием метода `\_\_new\_\_` для ограничения количества создаваемых экземпляров до 15.

Инструкции:
\begin{enumerate}
    \item Создайте класс `DeviceManager`.
    \item Добавьте атрибут класса `\_devices` и инициализируйте его пустым списком.
    \item Добавьте атрибут класса `MANAGER\_LIMIT` и инициализируйте его значением 15.
    \item Переопределите метод `\_\_new\_\_`. Если `len(\_devices) >= MANAGER\_LIMIT`, выбросьте `RuntimeError("Менеджер устройств перегружен!")`. Иначе, создайте экземпляр, добавьте в `\_devices`, верните.
    \item Переопределите метод `\_\_del\_\_`, чтобы он удалял `self` из `\_devices`.
    \item Переопределите метод `\_\_init\_\_`, который принимает `device\_name` и устанавливает `self.device = device\_name`.
    \item Создайте 15 экземпляров.
    \item Попытайтесь создать 16-й - поймайте и выведите исключение.
    \item Удалите семь экземпляров.
    \item Создайте 16-й, 17-й, ..., 22-й экземпляры - должно сработать.
\end{enumerate}

Пример использования:
\begin{lstlisting}[language=Python]
# Создаем 15 устройств
devices = [DeviceManager(f"Device{i}") for i in range(15)]

# Попытка создать 16-е
try:
    d16 = DeviceManager("NewDevice")
except RuntimeError as e:
    print("Ошибка:", e)

# Удаляем 7
for i in range(7):
    del devices[0]

# Создаем 16-е, 17-е, ..., 22-е - успешно
new_devices = [DeviceManager(f"Device{i}") for i in range(16, 23)]
for d in new_devices:
    print("Добавлено устройство:", d.device)
\end{lstlisting}

\item Написать программу на Python, которая создает класс `SessionPool` с использованием метода `\_\_new\_\_` для ограничения количества создаваемых экземпляров до 6.

Инструкции:
\begin{enumerate}
    \item Создайте класс `SessionPool`.
    \item Добавьте атрибут класса `\_sessions` и инициализируйте его пустым списком.
    \item Добавьте атрибут класса `SESSION\_LIMIT` и инициализируйте его значением 6.
    \item Переопределите метод `\_\_new\_\_`. Если `len(\_sessions) >= SESSION\_LIMIT`, выбросьте `RuntimeError("Пул сессий исчерпан!")`. Иначе, создайте экземпляр, добавьте в `\_sessions`, верните.
    \item Переопределите метод `\_\_del\_\_`, чтобы он удалял `self` из `\_sessions`.
    \item Переопределите метод `\_\_init\_\_`, который принимает `session\_token` и устанавливает `self.token = session\_token`.
    \item Создайте 6 экземпляров.
    \item Попытайтесь создать 7-й - поймайте и выведите исключение.
    \item Удалите два экземпляра.
    \item Создайте 7-й и 8-й экземпляры - должно сработать.
\end{enumerate}

Пример использования:
\begin{lstlisting}[language=Python]
# Создаем 6 сессий
sessions = [SessionPool(f"Token{i}") for i in range(6)]

# Попытка создать 7-ю
try:
    s7 = SessionPool("Token7")
except RuntimeError as e:
    print("Ошибка:", e)

# Удаляем 2
del sessions[0], sessions[1]

# Создаем 7-ю и 8-ю - успешно
s7 = SessionPool("Token7")
s8 = SessionPool("Token8")
print("Созданы токены:", s7.token, s8.token)
\end{lstlisting}

\item Написать программу на Python, которая создает класс `ThreadPool` с использованием метода `\_\_new\_\_` для ограничения количества создаваемых экземпляров до 12.

Инструкции:
\begin{enumerate}
    \item Создайте класс `ThreadPool`.
    \item Добавьте атрибут класса `\_threads` и инициализируйте его пустым списком.
    \item Добавьте атрибут класса `THREAD\_MAX` и инициализируйте его значением 12.
    \item Переопределите метод `\_\_new\_\_`. Если `len(\_threads) >= THREAD\_MAX`, выбросьте `RuntimeError("Достигнут лимит потоков!")`. Иначе, создайте экземпляр, добавьте в `\_threads`, верните.
    \item Переопределите метод `\_\_del\_\_`, чтобы он удалял `self` из `\_threads`.
    \item Переопределите метод `\_\_init\_\_`, который принимает `thread\_id` и устанавливает `self.thread = thread\_id`.
    \item Создайте 12 экземпляров.
    \item Попытайтесь создать 13-й - поймайте и выведите исключение.
    \item Удалите три экземпляра.
    \item Создайте 13-й, 14-й, 15-й экземпляры - должно сработать.
\end{enumerate}

Пример использования:
\begin{lstlisting}[language=Python]
# Создаем 12 потоков
threads = [ThreadPool(f"Thread{i}") for i in range(12)]

# Попытка создать 13-й
try:
    t13 = ThreadPool("Thread13")
except RuntimeError as e:
    print("Ошибка:", e)

# Удаляем 3
del threads[0], threads[1], threads[2]

# Создаем 13-й, 14-й, 15-й - успешно
t13 = ThreadPool("Thread13")
t14 = ThreadPool("Thread14")
t15 = ThreadPool("Thread15")
print("Созданы потоки:", t13.thread, t14.thread, t15.thread)
\end{lstlisting}

\item Написать программу на Python, которая создает класс `CachePool` с использованием метода `\_\_new\_\_` для ограничения количества создаваемых экземпляров до 20.

Инструкции:
\begin{enumerate}
    \item Создайте класс `CachePool`.
    \item Добавьте атрибут класса `\_caches` и инициализируйте его пустым списком.
    \item Добавьте атрибут класса `CACHE\_LIMIT` и инициализируйте его значением 20.
    \item Переопределите метод `\_\_new\_\_`. Если `len(\_caches) >= CACHE\_LIMIT`, выбросьте `RuntimeError("Кэш-пул переполнен!")`. Иначе, создайте экземпляр, добавьте в `\_caches`, верните.
    \item Переопределите метод `\_\_del\_\_`, чтобы он удалял `self` из `\_caches`.
    \item Переопределите метод `\_\_init\_\_`, который принимает `cache\_key` и устанавливает `self.key = cache\_key`.
    \item Создайте 20 экземпляров.
    \item Попытайтесь создать 21-й - поймайте и выведите исключение.
    \item Удалите пять экземпляров.
    \item Создайте 21-й, 22-й, ..., 25-й экземпляры - должно сработать.
\end{enumerate}

Пример использования:
\begin{lstlisting}[language=Python]
# Создаем 20 кэшей
caches = [CachePool(f"Key{i}") for i in range(20)]

# Попытка создать 21-й
try:
    c21 = CachePool("Key21")
except RuntimeError as e:
    print("Ошибка:", e)

# Удаляем 5
for i in range(5):
    del caches[0]

# Создаем 21-й, 22-й, ..., 25-й - успешно
new_caches = [CachePool(f"Key{i}") for i in range(21, 26)]
for c in new_caches:
    print("Создан ключ:", c.key)
\end{lstlisting}

\item Написать программу на Python, которая создает класс `DatabasePool` с использованием метода `\_\_new\_\_` для ограничения количества создаваемых экземпляров до 4.

Инструкции:
\begin{enumerate}
    \item Создайте класс `DatabasePool`.
    \item Добавьте атрибут класса `\_databases` и инициализируйте его пустым списком.
    \item Добавьте атрибут класса `DB\_LIMIT` и инициализируйте его значением 4.
    \item Переопределите метод `\_\_new\_\_`. Если `len(\_databases) >= DB\_LIMIT`, выбросьте `RuntimeError("Базы данных: лимит превышен!")`. Иначе, создайте экземпляр, добавьте в `\_databases`, верните.
    \item Переопределите метод `\_\_del\_\_`, чтобы он удалял `self` из `\_databases`.
    \item Переопределите метод `\_\_init\_\_`, который принимает `db\_name` и устанавливает `self.name = db\_name`.
    \item Создайте 4 экземпляра.
    \item Попытайтесь создать 5-й - поймайте и выведите исключение.
    \item Удалите один экземпляр.
    \item Создайте 5-й экземпляр - должно сработать.
\end{enumerate}

Пример использования:
\begin{lstlisting}[language=Python]
# Создаем 4 базы
dbs = [DatabasePool(f"DB{i}") for i in range(4)]

# Попытка создать 5-ю
try:
    db5 = DatabasePool("DB5")
except RuntimeError as e:
    print("Ошибка:", e)

# Удаляем одну
del dbs[0]

# Создаем 5-ю - успешно
db5 = DatabasePool("DB5")
print("Создана база:", db5.name)
\end{lstlisting}

\item Написать программу на Python, которая создает класс `FileHandlerPool` с использованием метода `\_\_new\_\_` для ограничения количества создаваемых экземпляров до 9.

Инструкции:
\begin{enumerate}
    \item Создайте класс `FileHandlerPool`.
    \item Добавьте атрибут класса `\_handlers` и инициализируйте его пустым списком.
    \item Добавьте атрибут класса `HANDLER\_MAX` и инициализируйте его значением 9.
    \item Переопределите метод `\_\_new\_\_`. Если `len(\_handlers) >= HANDLER\_MAX`, выбросьте `RuntimeError("Слишком много обработчиков файлов!")`. Иначе, создайте экземпляр, добавьте в `\_handlers`, верните.
    \item Переопределите метод `\_\_del\_\_`, чтобы он удалял `self` из `\_handlers`.
    \item Переопределите метод `\_\_init\_\_`, который принимает `file\_path` и устанавливает `self.path = file\_path`.
    \item Создайте 9 экземпляров.
    \item Попытайтесь создать 10-й - поймайте и выведите исключение.
    \item Удалите четыре экземпляра.
    \item Создайте 10-й, 11-й, 12-й, 13-й экземпляры - должно сработать.
\end{enumerate}

Пример использования:
\begin{lstlisting}[language=Python]
# Создаем 9 обработчиков
handlers = [FileHandlerPool(f"/path/to/file{i}.txt") for i in range(9)]

# Попытка создать 10-й
try:
    h10 = FileHandlerPool("/path/to/newfile.txt")
except RuntimeError as e:
    print("Ошибка:", e)

# Удаляем 4
del handlers[0], handlers[1], handlers[2], handlers[3]

# Создаем 10-й, 11-й, 12-й, 13-й - успешно
new_handlers = [FileHandlerPool(f"/path/to/newfile{i}.txt") for i in range(10, 14)]
for h in new_handlers:
    print("Обработчик для:", h.path)
\end{lstlisting}

\item Написать программу на Python, которая создает класс `NetworkPool` с использованием метода `\_\_new\_\_` для ограничения количества создаваемых экземпляров до 11.

Инструкции:
\begin{enumerate}
    \item Создайте класс `NetworkPool`.
    \item Добавьте атрибут класса `\_networks` и инициализируйте его пустым списком.
    \item Добавьте атрибут класса `NETWORK\_CAP` и инициализируйте его значением 11.
    \item Переопределите метод `\_\_new\_\_`. Если `len(\_networks) >= NETWORK\_CAP`, выбросьте `RuntimeError("Сеть: превышен лимит!")`. Иначе, создайте экземпляр, добавьте в `\_networks`, верните.
    \item Переопределите метод `\_\_del\_\_`, чтобы он удалял `self` из `\_networks`.
    \item Переопределите метод `\_\_init\_\_`, который принимает `network\_id` и устанавливает `self.net\_id = network\_id`.
    \item Создайте 11 экземпляров.
    \item Попытайтесь создать 12-й - поймайте и выведите исключение.
    \item Удалите шесть экземпляров.
    \item Создайте 12-й, 13-й, ..., 17-й экземпляры - должно сработать.
\end{enumerate}

Пример использования:
\begin{lstlisting}[language=Python]
# Создаем 11 сетей
networks = [NetworkPool(f"Net{i}") for i in range(11)]

# Попытка создать 12-ю
try:
    n12 = NetworkPool("Net12")
except RuntimeError as e:
    print("Ошибка:", e)

# Удаляем 6
for i in range(6):
    del networks[0]

# Создаем 12-ю, 13-ю, ..., 17-ю - успешно
new_networks = [NetworkPool(f"Net{i}") for i in range(12, 18)]
for n in new_networks:
    print("Создана сеть:", n.net_id)
\end{lstlisting}

\item Написать программу на Python, которая создает класс `MemoryPool` с использованием метода `\_\_new\_\_` для ограничения количества создаваемых экземпляров до 25.

Инструкции:
\begin{enumerate}
    \item Создайте класс `MemoryPool`.
    \item Добавьте атрибут класса `\_blocks` и инициализируйте его пустым списком.
    \item Добавьте атрибут класса `MEMORY\_LIMIT` и инициализируйте его значением 25.
    \item Переопределите метод `\_\_new\_\_`. Если `len(\_blocks) >= MEMORY\_LIMIT`, выбросьте `RuntimeError("Память: лимит блоков превышен!")`. Иначе, создайте экземпляр, добавьте в `\_blocks`, верните.
    \item Переопределите метод `\_\_del\_\_`, чтобы он удалял `self` из `\_blocks`.
    \item Переопределите метод `\_\_init\_\_`, который принимает `block\_size` и устанавливает `self.size = block\_size`.
    \item Создайте 25 экземпляров.
    \item Попытайтесь создать 26-й - поймайте и выведите исключение.
    \item Удалите десять экземпляров.
    \item Создайте 26-й, 27-й, ..., 35-й экземпляры - должно сработать.
\end{enumerate}

Пример использования:
\begin{lstlisting}[language=Python]
# Создаем 25 блоков
blocks = [MemoryPool(f"Size{i}") for i in range(25)]

# Попытка создать 26-й
try:
    b26 = MemoryPool("Size26")
except RuntimeError as e:
    print("Ошибка:", e)

# Удаляем 10
for i in range(10):
    del blocks[0]

# Создаем 26-й, 27-й, ..., 35-й - успешно
new_blocks = [MemoryPool(f"Size{i}") for i in range(26, 36)]
for b in new_blocks:
    print("Создан блок размером:", b.size)
\end{lstlisting}

\item Написать программу на Python, которая создает класс `ProcessPool` с использованием метода `\_\_new\_\_` для ограничения количества создаваемых экземпляров до 16.

Инструкции:
\begin{enumerate}
    \item Создайте класс `ProcessPool`.
    \item Добавьте атрибут класса `\_processes` и инициализируйте его пустым списком.
    \item Добавьте атрибут класса `PROCESS\_MAX` и инициализируйте его значением 16.
    \item Переопределите метод `\_\_new\_\_`. Если `len(\_processes) >= PROCESS\_MAX`, выбросьте `RuntimeError("Процессы: лимит превышен!")`. Иначе, создайте экземпляр, добавьте в `\_processes`, верните.
    \item Переопределите метод `\_\_del\_\_`, чтобы он удалял `self` из `\_processes`.
    \item Переопределите метод `\_\_init\_\_`, который принимает `process\_name` и устанавливает `self.name = process\_name`.
    \item Создайте 16 экземпляров.
    \item Попытайтесь создать 17-й - поймайте и выведите исключение.
    \item Удалите восемь экземпляров.
    \item Создайте 17-й, 18-й, ..., 24-й экземпляры - должно сработать.
\end{enumerate}

Пример использования:
\begin{lstlisting}[language=Python]
# Создаем 16 процессов
processes = [ProcessPool(f"Proc{i}") for i in range(16)]

# Попытка создать 17-й
try:
    p17 = ProcessPool("Proc17")
except RuntimeError as e:
    print("Ошибка:", e)

# Удаляем 8
for i in range(8):
    del processes[0]

# Создаем 17-й, 18-й, ..., 24-й - успешно
new_processes = [ProcessPool(f"Proc{i}") for i in range(17, 25)]
for p in new_processes:
    print("Запущен процесс:", p.name)
\end{lstlisting}

\item Написать программу на Python, которая создает класс `BufferPool` с использованием метода `\_\_new\_\_` для ограничения количества создаваемых экземпляров до 18.

Инструкции:
\begin{enumerate}
    \item Создайте класс `BufferPool`.
    \item Добавьте атрибут класса `\_buffers` и инициализируйте его пустым списком.
    \item Добавьте атрибут класса `BUFFER\_SIZE` и инициализируйте его значением 18.
    \item Переопределите метод `\_\_new\_\_`. Если `len(\_buffers) >= BUFFER\_SIZE`, выбросьте `RuntimeError("Буфер: переполнение!")`. Иначе, создайте экземпляр, добавьте в `\_buffers`, верните.
    \item Переопределите метод `\_\_del\_\_`, чтобы он удалял `self` из `\_buffers`.
    \item Переопределите метод `\_\_init\_\_`, который принимает `buffer\_id` и устанавливает `self.id = buffer\_id`.
    \item Создайте 18 экземпляров.
    \item Попытайтесь создать 19-й - поймайте и выведите исключение.
    \item Удалите девять экземпляров.
    \item Создайте 19-й, 20-й, ..., 27-й экземпляры - должно сработать.
\end{enumerate}

Пример использования:
\begin{lstlisting}[language=Python]
# Создаем 18 буферов
buffers = [BufferPool(f"Buf{i}") for i in range(18)]

# Попытка создать 19-й
try:
    b19 = BufferPool("Buf19")
except RuntimeError as e:
    print("Ошибка:", e)

# Удаляем 9
for i in range(9):
    del buffers[0]

# Создаем 19-й, 20-й, ..., 27-й - успешно
new_buffers = [BufferPool(f"Buf{i}") for i in range(19, 28)]
for b in new_buffers:
    print("Создан буфер:", b.id)
\end{lstlisting}

\item Написать программу на Python, которая создает класс `ChannelPool` с использованием метода `\_\_new\_\_` для ограничения количества создаваемых экземпляров до 13.

Инструкции:
\begin{enumerate}
    \item Создайте класс `ChannelPool`.
    \item Добавьте атрибут класса `\_channels` и инициализируйте его пустым списком.
    \item Добавьте атрибут класса `CHANNEL\_LIMIT` и инициализируйте его значением 13.
    \item Переопределите метод `\_\_new\_\_`. Если `len(\_channels) >= CHANNEL\_LIMIT`, выбросьте `RuntimeError("Каналы: лимит исчерпан!")`. Иначе, создайте экземпляр, добавьте в `\_channels`, верните.
    \item Переопределите метод `\_\_del\_\_`, чтобы он удалял `self` из `\_channels`.
    \item Переопределите метод `\_\_init\_\_`, который принимает `channel\_name` и устанавливает `self.name = channel\_name`.
    \item Создайте 13 экземпляров.
    \item Попытайтесь создать 14-й - поймайте и выведите исключение.
    \item Удалите три экземпляра.
    \item Создайте 14-й, 15-й, 16-й экземпляры - должно сработать.
\end{enumerate}

Пример использования:
\begin{lstlisting}[language=Python]
# Создаем 13 каналов
channels = [ChannelPool(f"Channel{i}") for i in range(13)]

# Попытка создать 14-й
try:
    c14 = ChannelPool("Channel14")
except RuntimeError as e:
    print("Ошибка:", e)

# Удаляем 3
del channels[0], channels[1], channels[2]

# Создаем 14-й, 15-й, 16-й - успешно
c14 = ChannelPool("Channel14")
c15 = ChannelPool("Channel15")
c16 = ChannelPool("Channel16")
print("Созданы каналы:", c14.name, c15.name, c16.name)
\end{lstlisting}

\item Написать программу на Python, которая создает класс `SocketPool` с использованием метода `\_\_new\_\_` для ограничения количества создаваемых экземпляров до 22.

Инструкции:
\begin{enumerate}
    \item Создайте класс `SocketPool`.
    \item Добавьте атрибут класса `\_sockets` и инициализируйте его пустым списком.
    \item Добавьте атрибут класса `SOCKET\_MAX` и инициализируйте его значением 22.
    \item Переопределите метод `\_\_new\_\_`. Если `len(\_sockets) >= SOCKET\_MAX`, выбросьте `RuntimeError("Сокеты: лимит превышен!")`. Иначе, создайте экземпляр, добавьте в `\_sockets`, верните.
    \item Переопределите метод `\_\_del\_\_`, чтобы он удалял `self` из `\_sockets`.
    \item Переопределите метод `\_\_init\_\_`, который принимает `socket\_port` и устанавливает `self.port = socket\_port`.
    \item Создайте 22 экземпляра.
    \item Попытайтесь создать 23-й - поймайте и выведите исключение.
    \item Удалите одиннадцать экземпляров.
    \item Создайте 23-й, 24-й, ..., 33-й экземпляры - должно сработать.
\end{enumerate}

Пример использования:
\begin{lstlisting}[language=Python]
# Создаем 22 сокета
sockets = [SocketPool(8000 + i) for i in range(22)]

# Попытка создать 23-й
try:
    s23 = SocketPool(8022)
except RuntimeError as e:
    print("Ошибка:", e)

# Удаляем 11
for i in range(11):
    del sockets[0]

# Создаем 23-й, 24-й, ..., 33-й - успешно
new_sockets = [SocketPool(8022 + i) for i in range(11)]
for s in new_sockets:
    print("Создан сокет на порту:", s.port)
\end{lstlisting}

\item Написать программу на Python, которая создает класс `LockPool` с использованием метода `\_\_new\_\_` для ограничения количества создаваемых экземпляров до 14.

Инструкции:
\begin{enumerate}
    \item Создайте класс `LockPool`.
    \item Добавьте атрибут класса `\_locks` и инициализируйте его пустым списком.
    \item Добавьте атрибут класса `LOCK\_COUNT` и инициализируйте его значением 14.
    \item Переопределите метод `\_\_new\_\_`. Если `len(\_locks) >= LOCK\_COUNT`, выбросьте `RuntimeError("Замки: все заняты!")`. Иначе, создайте экземпляр, добавьте в `\_locks`, верните.
    \item Переопределите метод `\_\_del\_\_`, чтобы он удалял `self` из `\_locks`.
    \item Переопределите метод `\_\_init\_\_`, который принимает `lock\_name` и устанавливает `self.name = lock\_name`.
    \item Создайте 14 экземпляров.
    \item Попытайтесь создать 15-й - поймайте и выведите исключение.
    \item Удалите семь экземпляров.
    \item Создайте 15-й, 16-й, ..., 21-й экземпляры - должно сработать.
\end{enumerate}

Пример использования:
\begin{lstlisting}[language=Python]
# Создаем 14 замков
locks = [LockPool(f"Lock{i}") for i in range(14)]

# Попытка создать 15-й
try:
    l15 = LockPool("Lock15")
except RuntimeError as e:
    print("Ошибка:", e)

# Удаляем 7
for i in range(7):
    del locks[0]

# Создаем 15-й, 16-й, ..., 21-й - успешно
new_locks = [LockPool(f"Lock{i}") for i in range(15, 22)]
for l in new_locks:
    print("Создан замок:", l.name)
\end{lstlisting}

\item Написать программу на Python, которая создает класс `QueuePool` с использованием метода `\_\_new\_\_` для ограничения количества создаваемых экземпляров до 19.

Инструкции:
\begin{enumerate}
    \item Создайте класс `QueuePool`.
    \item Добавьте атрибут класса `\_queues` и инициализируйте его пустым списком.
    \item Добавьте атрибут класса `QUEUE\_COUNT` и инициализируйте его значением 19.
    \item Переопределите метод `\_\_new\_\_`. Если `len(\_queues) >= QUEUE\_COUNT`, выбросьте `RuntimeError("Очереди: лимит достигнут!")`. Иначе, создайте экземпляр, добавьте в `\_queues`, верните.
    \item Переопределите метод `\_\_del\_\_`, чтобы он удалял `self` из `\_queues`.
    \item Переопределите метод `\_\_init\_\_`, который принимает `queue\_name` и устанавливает `self.name = queue\_name`.
    \item Создайте 19 экземпляров.
    \item Попытайтесь создать 20-й - поймайте и выведите исключение.
    \item Удалите десять экземпляров.
    \item Создайте 20-й, 21-й, ..., 29-й экземпляры - должно сработать.
\end{enumerate}

Пример использования:
\begin{lstlisting}[language=Python]
# Создаем 19 очередей
queues = [QueuePool(f"Queue{i}") for i in range(19)]

# Попытка создать 20-ю
try:
    q20 = QueuePool("Queue20")
except RuntimeError as e:
    print("Ошибка:", e)

# Удаляем 10
for i in range(10):
    del queues[0]

# Создаем 20-ю, 21-ю, ..., 29-ю - успешно
new_queues = [QueuePool(f"Queue{i}") for i in range(20, 30)]
for q in new_queues:
    print("Создана очередь:", q.name)
\end{lstlisting}

\item Написать программу на Python, которая создает класс `SemaphorePool` с использованием метода `\_\_new\_\_` для ограничения количества создаваемых экземпляров до 8.

Инструкции:
\begin{enumerate}
    \item Создайте класс `SemaphorePool`.
    \item Добавьте атрибут класса `\_semaphores` и инициализируйте его пустым списком.
    \item Добавьте атрибут класса `SEMA\_LIMIT` и инициализируйте его значением 8.
    \item Переопределите метод `\_\_new\_\_`. Если `len(\_semaphores) >= SEMA\_LIMIT`, выбросьте `RuntimeError("Семафоры: лимит превышен!")`. Иначе, создайте экземпляр, добавьте в `\_semaphores`, верните.
    \item Переопределите метод `\_\_del\_\_`, чтобы он удалял `self` из `\_semaphores`.
    \item Переопределите метод `\_\_init\_\_`, который принимает `sema\_id` и устанавливает `self.id = sema\_id`.
    \item Создайте 8 экземпляров.
    \item Попытайтесь создать 9-й - поймайте и выведите исключение.
    \item Удалите четыре экземпляра.
    \item Создайте 9-й, 10-й, 11-й, 12-й экземпляры - должно сработать.
\end{enumerate}

Пример использования:
\begin{lstlisting}[language=Python]
# Создаем 8 семафоров
semas = [SemaphorePool(f"Sema{i}") for i in range(8)]

# Попытка создать 9-й
try:
    s9 = SemaphorePool("Sema9")
except RuntimeError as e:
    print("Ошибка:", e)

# Удаляем 4
del semas[0], semas[1], semas[2], semas[3]

# Создаем 9-й, 10-й, 11-й, 12-й - успешно
s9 = SemaphorePool("Sema9")
s10 = SemaphorePool("Sema10")
s11 = SemaphorePool("Sema11")
s12 = SemaphorePool("Sema12")
print("Созданы семафоры:", s9.id, s10.id, s11.id, s12.id)
\end{lstlisting}

\item Написать программу на Python, которая создает класс `TimerPool` с использованием метода `\_\_new\_\_` для ограничения количества создаваемых экземпляров до 21.

Инструкции:
\begin{enumerate}
    \item Создайте класс `TimerPool`.
    \item Добавьте атрибут класса `\_timers` и инициализируйте его пустым списком.
    \item Добавьте атрибут класса `TIMER\_MAX` и инициализируйте его значением 21.
    \item Переопределите метод `\_\_new\_\_`. Если `len(\_timers) >= TIMER\_MAX`, выбросьте `RuntimeError("Таймеры: лимит исчерпан!")`. Иначе, создайте экземпляр, добавьте в `\_timers`, верните.
    \item Переопределите метод `\_\_del\_\_`, чтобы он удалял `self` из `\_timers`.
    \item Переопределите метод `\_\_init\_\_`, который принимает `timer\_duration` и устанавливает `self.duration = timer\_duration`.
    \item Создайте 21 экземпляр.
    \item Попытайтесь создать 22-й - поймайте и выведите исключение.
    \item Удалите одиннадцать экземпляров.
    \item Создайте 22-й, 23-й, ..., 32-й экземпляры - должно сработать.
\end{enumerate}

Пример использования:
\begin{lstlisting}[language=Python]
# Создаем 21 таймер
timers = [TimerPool(i * 10) for i in range(21)]

# Попытка создать 22-й
try:
    t22 = TimerPool(220)
except RuntimeError as e:
    print("Ошибка:", e)

# Удаляем 11
for i in range(11):
    del timers[0]

# Создаем 22-й, 23-й, ..., 32-й - успешно
new_timers = [TimerPool(i * 10) for i in range(22, 33)]
for t in new_timers:
    print("Создан таймер на:", t.duration, "сек")
\end{lstlisting}

\item Написать программу на Python, которая создает класс `WorkerPool` с использованием метода `\_\_new\_\_` для ограничения количества создаваемых экземпляров до 23.

Инструкции:
\begin{enumerate}
    \item Создайте класс `WorkerPool`.
    \item Добавьте атрибут класса `\_workers` и инициализируйте его пустым списком.
    \item Добавьте атрибут класса `WORKER\_LIMIT` и инициализируйте его значением 23.
    \item Переопределите метод `\_\_new\_\_`. Если `len(\_workers) >= WORKER\_LIMIT`, выбросьте `RuntimeError("Рабочие: лимит превышен!")`. Иначе, создайте экземпляр, добавьте в `\_workers`, верните.
    \item Переопределите метод `\_\_del\_\_`, чтобы он удалял `self` из `\_workers`.
    \item Переопределите метод `\_\_init\_\_`, который принимает `worker\_id` и устанавливает `self.id = worker\_id`.
    \item Создайте 23 экземпляра.
    \item Попытайтесь создать 24-й - поймайте и выведите исключение.
    \item Удалите двенадцать экземпляров.
    \item Создайте 24-й, 25-й, ..., 35-й экземпляры - должно сработать.
\end{enumerate}

Пример использования:
\begin{lstlisting}[language=Python]
# Создаем 23 рабочих
workers = [WorkerPool(f"Worker{i}") for i in range(23)]

# Попытка создать 24-го
try:
    w24 = WorkerPool("Worker24")
except RuntimeError as e:
    print("Ошибка:", e)

# Удаляем 12
for i in range(12):
    del workers[0]

# Создаем 24-го, 25-го, ..., 35-го - успешно
new_workers = [WorkerPool(f"Worker{i}") for i in range(24, 36)]
for w in new_workers:
    print("Создан рабочий:", w.id)
\end{lstlisting}

\item Написать программу на Python, которая создает класс `JobPool` с использованием метода `\_\_new\_\_` для ограничения количества создаваемых экземпляров до 26.

Инструкции:
\begin{enumerate}
    \item Создайте класс `JobPool`.
    \item Добавьте атрибут класса `\_jobs` и инициализируйте его пустым списком.
    \item Добавьте атрибут класса `JOB\_CAP` и инициализируйте его значением 26.
    \item Переопределите метод `\_\_new\_\_`. Если `len(\_jobs) >= JOB\_CAP`, выбросьте `RuntimeError("Задания: лимит превышен!")`. Иначе, создайте экземпляр, добавьте в `\_jobs`, верните.
    \item Переопределите метод `\_\_del\_\_`, чтобы он удалял `self` из `\_jobs`.
    \item Переопределите метод `\_\_init\_\_`, который принимает `job\_name` и устанавливает `self.name = job\_name`.
    \item Создайте 26 экземпляров.
    \item Попытайтесь создать 27-й - поймайте и выведите исключение.
    \item Удалите тринадцать экземпляров.
    \item Создайте 27-й, 28-й, ..., 39-й экземпляры - должно сработать.
\end{enumerate}

Пример использования:
\begin{lstlisting}[language=Python]
# Создаем 26 заданий
jobs = [JobPool(f"Job{i}") for i in range(26)]

# Попытка создать 27-е
try:
    j27 = JobPool("Job27")
except RuntimeError as e:
    print("Ошибка:", e)

# Удаляем 13
for i in range(13):
    del jobs[0]

# Создаем 27-е, 28-е, ..., 39-е - успешно
new_jobs = [JobPool(f"Job{i}") for i in range(27, 40)]
for j in new_jobs:
    print("Создано задание:", j.name)
\end{lstlisting}

\item Написать программу на Python, которая создает класс `RequestPool` с использованием метода `\_\_new\_\_` для ограничения количества создаваемых экземпляров до 27.

Инструкции:
\begin{enumerate}
    \item Создайте класс `RequestPool`.
    \item Добавьте атрибут класса `\_requests` и инициализируйте его пустым списком.
    \item Добавьте атрибут класса `REQUEST\_MAX` и инициализируйте его значением 27.
    \item Переопределите метод `\_\_new\_\_`. Если `len(\_requests) >= REQUEST\_MAX`, выбросьте `RuntimeError("Запросы: лимит превышен!")`. Иначе, создайте экземпляр, добавьте в `\_requests`, верните.
    \item Переопределите метод `\_\_del\_\_`, чтобы он удалял `self` из `\_requests`.
    \item Переопределите метод `\_\_init\_\_`, который принимает `request\_url` и устанавливает `self.url = request\_url`.
    \item Создайте 27 экземпляров.
    \item Попытайтесь создать 28-й - поймайте и выведите исключение.
    \item Удалите четырнадцать экземпляров.
    \item Создайте 28-й, 29-й, ..., 41-й экземпляры - должно сработать.
\end{enumerate}

Пример использования:
\begin{lstlisting}[language=Python]
# Создаем 27 запросов
requests = [RequestPool(f"http://site{i}.com") for i in range(27)]

# Попытка создать 28-й
try:
    r28 = RequestPool("http://newsite.com")
except RuntimeError as e:
    print("Ошибка:", e)

# Удаляем 14
for i in range(14):
    del requests[0]

# Создаем 28-й, 29-й, ..., 41-й - успешно
new_requests = [RequestPool(f"http://newsite{i}.com") for i in range(28, 42)]
for r in new_requests:
    print("Создан запрос к:", r.url)
\end{lstlisting}

\item Написать программу на Python, которая создает класс `EventPool` с использованием метода `\_\_new\_\_` для ограничения количества создаваемых экземпляров до 28.

Инструкции:
\begin{enumerate}
    \item Создайте класс `EventPool`.
    \item Добавьте атрибут класса `\_events` и инициализируйте его пустым списком.
    \item Добавьте атрибут класса `EVENT\_LIMIT` и инициализируйте его значением 28.
    \item Переопределите метод `\_\_new\_\_`. Если `len(\_events) >= EVENT\_LIMIT`, выбросьте `RuntimeError("События: лимит превышен!")`. Иначе, создайте экземпляр, добавьте в `\_events`, верните.
    \item Переопределите метод `\_\_del\_\_`, чтобы он удалял `self` из `\_events`.
    \item Переопределите метод `\_\_init\_\_`, который принимает `event\_type` и устанавливает `self.type = event\_type`.
    \item Создайте 28 экземпляров.
    \item Попытайтесь создать 29-е - поймайте и выведите исключение.
    \item Удалите пятнадцать экземпляров.
    \item Создайте 29-е, 30-е, ..., 43-е экземпляры - должно сработать.
\end{enumerate}

Пример использования:
\begin{lstlisting}[language=Python]
# Создаем 28 событий
events = [EventPool(f"Event{i}") for i in range(28)]

# Попытка создать 29-е
try:
    e29 = EventPool("Event29")
except RuntimeError as e:
    print("Ошибка:", e)

# Удаляем 15
for i in range(15):
    del events[0]

# Создаем 29-е, 30-е, ..., 43-е - успешно
new_events = [EventPool(f"Event{i}") for i in range(29, 44)]
for e in new_events:
    print("Создано событие типа:", e.type)
\end{lstlisting}

\item Написать программу на Python, которая создает класс `MessagePool` с использованием метода `\_\_new\_\_` для ограничения количества создаваемых экземпляров до 29.

Инструкции:
\begin{enumerate}
    \item Создайте класс `MessagePool`.
    \item Добавьте атрибут класса `\_messages` и инициализируйте его пустым списком.
    \item Добавьте атрибут класса `MSG\_MAX` и инициализируйте его значением 29.
    \item Переопределите метод `\_\_new\_\_`. Если `len(\_messages) >= MSG\_MAX`, выбросьте `RuntimeError("Сообщения: лимит превышен!")`. Иначе, создайте экземпляр, добавьте в `\_messages`, верните.
    \item Переопределите метод `\_\_del\_\_`, чтобы он удалял `self` из `\_messages`.
    \item Переопределите метод `\_\_init\_\_`, который принимает `message\_text` и устанавливает `self.text = message\_text`.
    \item Создайте 29 экземпляров.
    \item Попытайтесь создать 30-й - поймайте и выведите исключение.
    \item Удалите шестнадцать экземпляров.
    \item Создайте 30-й, 31-й, ..., 45-й экземпляры - должно сработать.
\end{enumerate}

Пример использования:
\begin{lstlisting}[language=Python]
# Создаем 29 сообщений
messages = [MessagePool(f"Message{i}") for i in range(29)]

# Попытка создать 30-е
try:
    m30 = MessagePool("Message30")
except RuntimeError as e:
    print("Ошибка:", e)

# Удаляем 16
for i in range(16):
    del messages[0]

# Создаем 30-е, 31-е, ..., 45-е - успешно
new_messages = [MessagePool(f"Message{i}") for i in range(30, 46)]
for m in new_messages:
    print("Создано сообщение:", m.text)
\end{lstlisting}

\item Написать программу на Python, которая создает класс `NotificationPool` с использованием метода `\_\_new\_\_` для ограничения количества создаваемых экземпляров до 31.

Инструкции:
\begin{enumerate}
    \item Создайте класс `NotificationPool`.
    \item Добавьте атрибут класса `\_notifications` и инициализируйте его пустым списком.
    \item Добавьте атрибут класса `NOTIF\_LIMIT` и инициализируйте его значением 31.
    \item Переопределите метод `\_\_new\_\_`. Если `len(\_notifications) >= NOTIF\_LIMIT`, выбросьте `RuntimeError("Уведомления: лимит превышен!")`. Иначе, создайте экземпляр, добавьте в `\_notifications`, верните.
    \item Переопределите метод `\_\_del\_\_`, чтобы он удалял `self` из `\_notifications`.
    \item Переопределите метод `\_\_init\_\_`, который принимает `notification\_title` и устанавливает `self.title = notification\_title`.
    \item Создайте 31 экземпляр.
    \item Попытайтесь создать 32-й - поймайте и выведите исключение.
    \item Удалите семнадцать экземпляров.
    \item Создайте 32-й, 33-й, ..., 48-й экземпляры - должно сработать.
\end{enumerate}

Пример использования:
\begin{lstlisting}[language=Python]
# Создаем 31 уведомление
notifications = [NotificationPool(f"Notif{i}") for i in range(31)]

# Попытка создать 32-е
try:
    n32 = NotificationPool("Notif32")
except RuntimeError as e:
    print("Ошибка:", e)

# Удаляем 17
for i in range(17):
    del notifications[0]

# Создаем 32-е, 33-е, ..., 48-е - успешно
new_notifications = [NotificationPool(f"Notif{i}") for i in range(32, 49)]
for n in new_notifications:
    print("Создано уведомление:", n.title)
\end{lstlisting}
\item Написать программу на Python, которая создает класс `LoggerPool` с использованием метода `\_\_new\_\_` для ограничения количества создаваемых экземпляров до 5.

Инструкции:
\begin{enumerate}
    \item Создайте класс `LoggerPool`.
    \item Добавьте атрибут класса `\_loggers` и инициализируйте его пустым списком.
    \item Добавьте атрибут класса `LOGGER\_LIMIT` и инициализируйте его значением 5.
    \item Переопределите метод `\_\_new\_\_`. Если `len(\_loggers) >= LOGGER\_LIMIT`, выбросьте `RuntimeError("Логгеры: лимит превышен!")`. Иначе, создайте экземпляр, добавьте в `\_loggers`, верните.
    \item Переопределите метод `\_\_del\_\_`, чтобы он удалял `self` из `\_loggers`.
    \item Переопределите метод `\_\_init\_\_`, который принимает `logger\_name` и устанавливает `self.name = logger\_name`.
    \item Создайте 5 экземпляров.
    \item Попытайтесь создать 6-й - поймайте и выведите исключение.
    \item Удалите два экземпляра.
    \item Создайте 6-й и 7-й экземпляры - должно сработать.
\end{enumerate}

Пример использования:
\begin{lstlisting}[language=Python]
# Создаем 5 логгеров
loggers = [LoggerPool(f"Logger{i}") for i in range(5)]

# Попытка создать 6-й
try:
    l6 = LoggerPool("Logger6")
except RuntimeError as e:
    print("Ошибка:", e)

# Удаляем 2
del loggers[0], loggers[1]

# Создаем 6-й и 7-й - успешно
l6 = LoggerPool("Logger6")
l7 = LoggerPool("Logger7")
print("Созданы логгеры:", l6.name, l7.name)
\end{lstlisting}

\item Написать программу на Python, которая создает класс `ConfigPool` с использованием метода `\_\_new\_\_` для ограничения количества создаваемых экземпляров до 12.

Инструкции:
\begin{enumerate}
    \item Создайте класс `ConfigPool`.
    \item Добавьте атрибут класса `\_configs` и инициализируйте его пустым списком.
    \item Добавьте атрибут класса `CONFIG\_MAX` и инициализируйте его значением 12.
    \item Переопределите метод `\_\_new\_\_`. Если `len(\_configs) >= CONFIG\_MAX`, выбросьте `RuntimeError("Конфигурации: лимит превышен!")`. Иначе, создайте экземпляр, добавьте в `\_configs`, верните.
    \item Переопределите метод `\_\_del\_\_`, чтобы он удалял `self` из `\_configs`.
    \item Переопределите метод `\_\_init\_\_`, который принимает `config\_name` и устанавливает `self.name = config\_name`.
    \item Создайте 12 экземпляров.
    \item Попытайтесь создать 13-й - поймайте и выведите исключение.
    \item Удалите шесть экземпляров.
    \item Создайте 13-й, 14-й, ..., 18-й экземпляры - должно сработать.
\end{enumerate}

Пример использования:
\begin{lstlisting}[language=Python]
# Создаем 12 конфигураций
configs = [ConfigPool(f"Config{i}") for i in range(12)]

# Попытка создать 13-ю
try:
    c13 = ConfigPool("Config13")
except RuntimeError as e:
    print("Ошибка:", e)

# Удаляем 6
for i in range(6):
    del configs[0]

# Создаем 13-ю, 14-ю, ..., 18-ю - успешно
new_configs = [ConfigPool(f"Config{i}") for i in range(13, 19)]
for c in new_configs:
    print("Создана конфигурация:", c.name)
\end{lstlisting}

\item Написать программу на Python, которая создает класс `PluginPool` с использованием метода `\_\_new\_\_` для ограничения количества создаваемых экземпляров до 10.

Инструкции:
\begin{enumerate}
    \item Создайте класс `PluginPool`.
    \item Добавьте атрибут класса `\_plugins` и инициализируйте его пустым списком.
    \item Добавьте атрибут класса `PLUGIN\_CAP` и инициализируйте его значением 10.
    \item Переопределите метод `\_\_new\_\_`. Если `len(\_plugins) >= PLUGIN\_CAP`, выбросьте `RuntimeError("Плагины: лимит исчерпан!")`. Иначе, создайте экземпляр, добавьте в `\_plugins`, верните.
    \item Переопределите метод `\_\_del\_\_`, чтобы он удалял `self` из `\_plugins`.
    \item Переопределите метод `\_\_init\_\_`, который принимает `plugin\_id` и устанавливает `self.id = plugin\_id`.
    \item Создайте 10 экземпляров.
    \item Попытайтесь создать 11-й - поймайте и выведите исключение.
    \item Удалите пять экземпляров.
    \item Создайте 11-й, 12-й, ..., 15-й экземпляры - должно сработать.
\end{enumerate}

Пример использования:
\begin{lstlisting}[language=Python]
# Создаем 10 плагинов
plugins = [PluginPool(f"Plugin{i}") for i in range(10)]

# Попытка создать 11-й
try:
    p11 = PluginPool("Plugin11")
except RuntimeError as e:
    print("Ошибка:", e)

# Удаляем 5
for i in range(5):
    del plugins[0]

# Создаем 11-й, 12-й, ..., 15-й - успешно
new_plugins = [PluginPool(f"Plugin{i}") for i in range(11, 16)]
for p in new_plugins:
    print("Создан плагин:", p.id)
\end{lstlisting}

\item Написать программу на Python, которая создает класс `ServicePool` с использованием метода `\_\_new\_\_` для ограничения количества создаваемых экземпляров до 8.

Инструкции:
\begin{enumerate}
    \item Создайте класс `ServicePool`.
    \item Добавьте атрибут класса `\_services` и инициализируйте его пустым списком.
    \item Добавьте атрибут класса `SERVICE\_LIMIT` и инициализируйте его значением 8.
    \item Переопределите метод `\_\_new\_\_`. Если `len(\_services) >= SERVICE\_LIMIT`, выбросьте `RuntimeError("Сервисы: лимит превышен!")`. Иначе, создайте экземпляр, добавьте в `\_services`, верните.
    \item Переопределите метод `\_\_del\_\_`, чтобы он удалял `self` из `\_services`.
    \item Переопределите метод `\_\_init\_\_`, который принимает `service\_name` и устанавливает `self.name = service\_name`.
    \item Создайте 8 экземпляров.
    \item Попытайтесь создать 9-й - поймайте и выведите исключение.
    \item Удалите четыре экземпляра.
    \item Создайте 9-й, 10-й, 11-й, 12-й экземпляры - должно сработать.
\end{enumerate}

Пример использования:
\begin{lstlisting}[language=Python]
# Создаем 8 сервисов
services = [ServicePool(f"Service{i}") for i in range(8)]

# Попытка создать 9-й
try:
    s9 = ServicePool("Service9")
except RuntimeError as e:
    print("Ошибка:", e)

# Удаляем 4
del services[0], services[1], services[2], services[3]

# Создаем 9-й, 10-й, 11-й, 12-й - успешно
s9 = ServicePool("Service9")
s10 = ServicePool("Service10")
s11 = ServicePool("Service11")
s12 = ServicePool("Service12")
print("Созданы сервисы:", s9.name, s10.name, s11.name, s12.name)
\end{lstlisting}

\item Написать программу на Python, которая создает класс `CacheEntryPool` с использованием метода `\_\_new\_\_` для ограничения количества создаваемых экземпляров до 15.

Инструкции:
\begin{enumerate}
    \item Создайте класс `CacheEntryPool`.
    \item Добавьте атрибут класса `\_entries` и инициализируйте его пустым списком.
    \item Добавьте атрибут класса `ENTRY\_MAX` и инициализируйте его значением 15.
    \item Переопределите метод `\_\_new\_\_`. Если `len(\_entries) >= ENTRY\_MAX`, выбросьте `RuntimeError("Кэш-записи: лимит превышен!")`. Иначе, создайте экземпляр, добавьте в `\_entries`, верните.
    \item Переопределите метод `\_\_del\_\_`, чтобы он удалял `self` из `\_entries`.
    \item Переопределите метод `\_\_init\_\_`, который принимает `entry\_key` и устанавливает `self.key = entry\_key`.
    \item Создайте 15 экземпляров.
    \item Попытайтесь создать 16-й - поймайте и выведите исключение.
    \item Удалите семь экземпляров.
    \item Создайте 16-й, 17-й, ..., 22-й экземпляры - должно сработать.
\end{enumerate}

Пример использования:
\begin{lstlisting}[language=Python]
# Создаем 15 записей
entries = [CacheEntryPool(f"Key{i}") for i in range(15)]

# Попытка создать 16-ю
try:
    e16 = CacheEntryPool("Key16")
except RuntimeError as e:
    print("Ошибка:", e)

# Удаляем 7
for i in range(7):
    del entries[0]

# Создаем 16-ю, 17-ю, ..., 22-ю - успешно
new_entries = [CacheEntryPool(f"Key{i}") for i in range(16, 23)]
for e in new_entries:
    print("Создана запись с ключом:", e.key)
\end{lstlisting}

\item Написать программу на Python, которая создает класс `ConnectionHandlerPool` с использованием метода `\_\_new\_\_` для ограничения количества создаваемых экземпляров до 20.

Инструкции:
\begin{enumerate}
    \item Создайте класс `ConnectionHandlerPool`.
    \item Добавьте атрибут класса `\_handlers` и инициализируйте его пустым списком.
    \item Добавьте атрибут класса `HANDLER\_LIMIT` и инициализируйте его значением 20.
    \item Переопределите метод `\_\_new\_\_`. Если `len(\_handlers) >= HANDLER\_LIMIT`, выбросьте `RuntimeError("Обработчики соединений: лимит превышен!")`. Иначе, создайте экземпляр, добавьте в `\_handlers`, верните.
    \item Переопределите метод `\_\_del\_\_`, чтобы он удалял `self` из `\_handlers`.
    \item Переопределите метод `\_\_init\_\_`, который принимает `handler\_id` и устанавливает `self.id = handler\_id`.
    \item Создайте 20 экземпляров.
    \item Попытайтесь создать 21-й - поймайте и выведите исключение.
    \item Удалите десять экземпляров.
    \item Создайте 21-й, 22-й, ..., 30-й экземпляры - должно сработать.
\end{enumerate}

Пример использования:
\begin{lstlisting}[language=Python]
# Создаем 20 обработчиков
handlers = [ConnectionHandlerPool(f"H{i}") for i in range(20)]

# Попытка создать 21-й
try:
    h21 = ConnectionHandlerPool("H21")
except RuntimeError as e:
    print("Ошибка:", e)

# Удаляем 10
for i in range(10):
    del handlers[0]

# Создаем 21-й, 22-й, ..., 30-й - успешно
new_handlers = [ConnectionHandlerPool(f"H{i}") for i in range(21, 31)]
for h in new_handlers:
    print("Создан обработчик:", h.id)
\end{lstlisting}

\end{enumerate}
\subsubsection{Задача 3 (именование)}

\begin{enumerate}
\item Написать программу на Python, которая создает класс `Vagon` с использованием метода `\_\_new\_\_` для контроля именования. Имена должны начинаться с "vagon\_". Метод `\_\_init\_\_` должен быть пустым.

Инструкции:
\begin{enumerate}
    \item Создайте класс `Vagon`.
    \item Добавьте атрибут класса `numbers` и инициализируйте его пустым словарем.
    \item Переопределите метод `\_\_new\_\_`, принимающий `cls`, `name`, `number`.
    \item В `\_\_new\_\_`: если `name` не начинается с "vagon\_", выбросьте `ValueError("Имя должно начинаться с 'vagon\_'")`.
    \item Извлеките номер вагона: `vagon\_number = name[6:]` (удаляем "vagon\_").
    \item Создайте экземпляр: `instance = super().\_\_new\_\_(cls)`.
    \item Добавьте номер в словарь: `cls.numbers[vagon\_number] = instance`.
    \item Установите атрибут экземпляра: `setattr(instance, f"v{vagon\_number}", number)`.
    \item Верните `instance`.
    \item Переопределите метод `\_\_init\_\_` как пустой: `def \_\_init\_\_(self, *args, **kwargs): pass`.
    \item Создайте объект `v1` с именем "vagon\_1" и номером 101.
    \item Создайте объект `v2` с именем "vagon\_2" и номером 102.
    \item Попытайтесь создать объект с именем "car\_3" — должно возникнуть исключение `ValueError`.
    \item Выведите `v1.v1` и `v2.v2`.
    \item Выведите `Vagon.numbers`.
\end{enumerate}

Пример использования:
\begin{lstlisting}[language=Python]
v1 = Vagon("vagon_1", 101)
v2 = Vagon("vagon_2", 102)

try:
    v3 = Vagon("car_3", 103)
except ValueError as e:
    print("Ошибка:", e)

print("v1.v1:", v1.v1)  # 101
print("v2.v2:", v2.v2)  # 102
print("Vagon.numbers:", Vagon.numbers)
\end{lstlisting}

\item Написать программу на Python, которая создает класс `Room` с использованием метода `\_\_new\_\_` для контроля именования. Имена должны начинаться с "room\_". Метод `\_\_init\_\_` должен быть пустым.

Инструкции:
\begin{enumerate}
    \item Создайте класс `Room`.
    \item Добавьте атрибут класса `registry` и инициализируйте его пустым словарем.
    \item Переопределите метод `\_\_new\_\_`, принимающий `cls`, `name`, `capacity`.
    \item В `\_\_new\_\_`: если `name` не начинается с "room\_", выбросьте `ValueError("Недопустимое имя комнаты")`.
    \item Извлеките номер комнаты: `room\_num = name[5:]`.
    \item Создайте экземпляр: `instance = super().\_\_new\_\_(cls)`.
    \item Добавьте номер в словарь: `cls.registry[room\_num] = instance`.
    \item Установите атрибут экземпляра: `setattr(instance, f"r{room\_num}", capacity)`.
    \item Верните `instance`.
    \item Переопределите метод `\_\_init\_\_` как пустой.
    \item Создайте объект `r1` с именем "room\_101" и вместимостью 50.
    \item Создайте объект `r2` с именем "room\_202" и вместимостью 30.
    \item Попытайтесь создать объект с именем "hall\_A" — поймайте исключение.
    \item Выведите `r1.r101` и `r2.r202`.
    \item Выведите `Room.registry`.
\end{enumerate}

Пример использования:
\begin{lstlisting}[language=Python]
r1 = Room("room_101", 50)
r2 = Room("room_202", 30)

try:
    r3 = Room("hall_A", 100)
except ValueError as e:
    print("Ошибка:", e)

print("r1.r101:", r1.r101)  # 50
print("r2.r202:", r2.r202)  # 30
print("Room.registry:", Room.registry)
\end{lstlisting}

\item Написать программу на Python, которая создает класс `Device` с использованием метода `\_\_new\_\_` для контроля именования. Имена должны начинаться с "dev\_". Метод `\_\_init\_\_` должен быть пустым.

Инструкции:
\begin{enumerate}
    \item Создайте класс `Device`.
    \item Добавьте атрибут класса `inventory` и инициализируйте его пустым словарем.
    \item Переопределите метод `\_\_new\_\_`, принимающий `cls`, `name`, `model`.
    \item В `\_\_new\_\_`: если `name` не начинается с "dev\_", выбросьте `ValueError("Неверный префикс устройства")`.
    \item Извлеките ID устройства: `dev\_id = name[4:]`.
    \item Создайте экземпляр: `instance = super().\_\_new\_\_(cls)`.
    \item Добавьте ID в словарь: `cls.inventory[dev\_id] = instance`.
    \item Установите атрибут экземпляра: `setattr(instance, f"d{dev\_id}", model)`.
    \item Верните `instance`.
    \item Переопределите метод `\_\_init\_\_` как пустой.
    \item Создайте объект `d1` с именем "dev\_001" и моделью "X1".
    \item Создайте объект `d2` с именем "dev\_002" и моделью "Y2".
    \item Попытайтесь создать объект с именем "sensor\_01" — поймайте исключение.
    \item Выведите `d1.d001` и `d2.d002`.
    \item Выведите `Device.inventory`.
\end{enumerate}

Пример использования:
\begin{lstlisting}[language=Python]
d1 = Device("dev_001", "X1")
d2 = Device("dev_002", "Y2")

try:
    d3 = Device("sensor_01", "Z3")
except ValueError as e:
    print("Ошибка:", e)

print("d1.d001:", d1.d001)  # X1
print("d2.d002:", d2.d002)  # Y2
print("Device.inventory:", Device.inventory)
\end{lstlisting}

\item Написать программу на Python, которая создает класс `Book` с использованием метода `\_\_new\_\_` для контроля именования. Имена должны начинаться с "book\_". Метод `\_\_init\_\_` должен быть пустым.

Инструкции:
\begin{enumerate}
    \item Создайте класс `Book`.
    \item Добавьте атрибут класса `catalog` и инициализируйте его пустым словарем.
    \item Переопределите метод `\_\_new\_\_`, принимающий `cls`, `name`, `author`.
    \item В `\_\_new\_\_`: если `name` не начинается с "book\_", выбросьте `ValueError("Книга должна иметь префикс 'book\_'")`.
    \item Извлеките ID книги: `book\_id = name[5:]`.
    \item Создайте экземпляр: `instance = super().\_\_new\_\_(cls)`.
    \item Добавьте ID в словарь: `cls.catalog[book\_id] = instance`.
    \item Установите атрибут экземпляра: `setattr(instance, f"b{book\_id}", author)`.
    \item Верните `instance`.
    \item Переопределите метод `\_\_init\_\_` как пустой.
    \item Создайте объект `b1` с именем "book\_001" и автором "Толстой".
    \item Создайте объект `b2` с именем "book\_002" и автором "Достоевский".
    \item Попытайтесь создать объект с именем "magazine\_01" — поймайте исключение.
    \item Выведите `b1.b001` и `b2.b002`.
    \item Выведите `Book.catalog`.
\end{enumerate}

Пример использования:
\begin{lstlisting}[language=Python]
b1 = Book("book_001", "Толстой")
b2 = Book("book_002", "Достоевский")

try:
    b3 = Book("magazine_01", "Пушкин")
except ValueError as e:
    print("Ошибка:", e)

print("b1.b001:", b1.b001)  # Толстой
print("b2.b002:", b2.b002)  # Достоевский
print("Book.catalog:", Book.catalog)
\end{lstlisting}

\item Написать программу на Python, которая создает класс `File` с использованием метода `\_\_new\_\_` для контроля именования. Имена должны начинаться с "file\_". Метод `\_\_init\_\_` должен быть пустым.

Инструкции:
\begin{enumerate}
    \item Создайте класс `File`.
    \item Добавьте атрибут класса `index` и инициализируйте его пустым словарем.
    \item Переопределите метод `\_\_new\_\_`, принимающий `cls`, `name`, `size`.
    \item В `\_\_new\_\_`: если `name` не начинается с "file\_", выбросьте `ValueError("Файл должен иметь префикс 'file\_'")`.
    \item Извлеките ID файла: `file\_id = name[5:]`.
    \item Создайте экземпляр: `instance = super().\_\_new\_\_(cls)`.
    \item Добавьте ID в словарь: `cls.index[file\_id] = instance`.
    \item Установите атрибут экземпляра: `setattr(instance, f"f{file\_id}", size)`.
    \item Верните `instance`.
    \item Переопределите метод `\_\_init\_\_` как пустой.
    \item Создайте объект `f1` с именем "file\_config" и размером 1024.
    \item Создайте объект `f2` с именем "file\_data" и размером 2048.
    \item Попытайтесь создать объект с именем "document\_1" — поймайте исключение.
    \item Выведите `f1.fconfig` и `f2.fdata`.
    \item Выведите `File.index`.
\end{enumerate}

Пример использования:
\begin{lstlisting}[language=Python]
f1 = File("file_config", 1024)
f2 = File("file_data", 2048)

try:
    f3 = File("document_1", 512)
except ValueError as e:
    print("Ошибка:", e)

print("f1.fconfig:", f1.fconfig)  # 1024
print("f2.fdata:", f2.fdata)      # 2048
print("File.index:", File.index)
\end{lstlisting}

\item Написать программу на Python, которая создает класс `User` с использованием метода `\_\_new\_\_` для контроля именования. Имена должны начинаться с "user\_". Метод `\_\_init\_\_` должен быть пустым.

Инструкции:
\begin{enumerate}
    \item Создайте класс `User`.
    \item Добавьте атрибут класса `directory` и инициализируйте его пустым словарем.
    \item Переопределите метод `\_\_new\_\_`, принимающий `cls`, `name`, `email`.
    \item В `\_\_new\_\_`: если `name` не начинается с "user\_", выбросьте `ValueError("Пользователь должен иметь префикс 'user\_'")`.
    \item Извлеките ID пользователя: `user\_id = name[5:]`.
    \item Создайте экземпляр: `instance = super().\_\_new\_\_(cls)`.
    \item Добавьте ID в словарь: `cls.directory[user\_id] = instance`.
    \item Установите атрибут экземпляра: `setattr(instance, f"u{user\_id}", email)`.
    \item Верните `instance`.
    \item Переопределите метод `\_\_init\_\_` как пустой.
    \item Создайте объект `u1` с именем "user\_alice" и email "alice@example.com".
    \item Создайте объект `u2` с именем "user\_bob" и email "bob@example.com".
    \item Попытайтесь создать объект с именем "admin\_john" — поймайте исключение.
    \item Выведите `u1.ualice` и `u2.ubob`.
    \item Выведите `User.directory`.
\end{enumerate}

Пример использования:
\begin{lstlisting}[language=Python]
u1 = User("user_alice", "alice@example.com")
u2 = User("user_bob", "bob@example.com")

try:
    u3 = User("admin_john", "john@example.com")
except ValueError as e:
    print("Ошибка:", e)

print("u1.ualice:", u1.ualice)  # alice@example.com
print("u2.ubob:", u2.ubob)      # bob@example.com
print("User.directory:", User.directory)
\end{lstlisting}

\item Написать программу на Python, которая создает класс `Product` с использованием метода `\_\_new\_\_` для контроля именования. Имена должны начинаться с "prod\_". Метод `\_\_init\_\_` должен быть пустым.

Инструкции:
\begin{enumerate}
    \item Создайте класс `Product`.
    \item Добавьте атрибут класса `warehouse` и инициализируйте его пустым словарем.
    \item Переопределите метод `\_\_new\_\_`, принимающий `cls`, `name`, `price`.
    \item В `\_\_new\_\_`: если `name` не начинается с "prod\_", выбросьте `ValueError("Продукт должен иметь префикс 'prod\_'")`.
    \item Извлеките ID продукта: `prod\_id = name[5:]`.
    \item Создайте экземпляр: `instance = super().\_\_new\_\_(cls)`.
    \item Добавьте ID в словарь: `cls.warehouse[prod\_id] = instance`.
    \item Установите атрибут экземпляра: `setattr(instance, f"p{prod\_id}", price)`.
    \item Верните `instance`.
    \item Переопределите метод `\_\_init\_\_` как пустой.
    \item Создайте объект `p1` с именем "prod\_laptop" и ценой 999.
    \item Создайте объект `p2` с именем "prod\_mouse" и ценой 25.
    \item Попытайтесь создать объект с именем "item\_keyboard" — поймайте исключение.
    \item Выведите `p1.plaptop` и `p2.pmouse`.
    \item Выведите `Product.warehouse`.
\end{enumerate}

Пример использования:
\begin{lstlisting}[language=Python]
p1 = Product("prod_laptop", 999)
p2 = Product("prod_mouse", 25)

try:
    p3 = Product("item_keyboard", 50)
except ValueError as e:
    print("Ошибка:", e)

print("p1.plaptop:", p1.plaptop)  # 999
print("p2.pmouse:", p2.pmouse)    # 25
print("Product.warehouse:", Product.warehouse)
\end{lstlisting}

\item Написать программу на Python, которая создает класс `Employee` с использованием метода `\_\_new\_\_` для контроля именования. Имена должны начинаться с "emp\_". Метод `\_\_init\_\_` должен быть пустым.

Инструкции:
\begin{enumerate}
    \item Создайте класс `Employee`.
    \item Добавьте атрибут класса `staff` и инициализируйте его пустым словарем.
    \item Переопределите метод `\_\_new\_\_`, принимающий `cls`, `name`, `department`.
    \item В `\_\_new\_\_`: если `name` не начинается с "emp\_", выбросьте `ValueError("Сотрудник должен иметь префикс 'emp\_'")`.
    \item Извлеките ID сотрудника: `emp\_id = name[4:]`.
    \item Создайте экземпляр: `instance = super().\_\_new\_\_(cls)`.
    \item Добавьте ID в словарь: `cls.staff[emp\_id] = instance`.
    \item Установите атрибут экземпляра: `setattr(instance, f"e{emp\_id}", department)`.
    \item Верните `instance`.
    \item Переопределите метод `\_\_init\_\_` как пустой.
    \item Создайте объект `e1` с именем "emp\_001" и отделом "IT".
    \item Создайте объект `e2` с именем "emp\_002" и отделом "HR".
    \item Попытайтесь создать объект с именем "worker\_003" — поймайте исключение.
    \item Выведите `e1.e001` и `e2.e002`.
    \item Выведите `Employee.staff`.
\end{enumerate}

Пример использования:
\begin{lstlisting}[language=Python]
e1 = Employee("emp_001", "IT")
e2 = Employee("emp_002", "HR")

try:
    e3 = Employee("worker_003", "Sales")
except ValueError as e:
    print("Ошибка:", e)

print("e1.e001:", e1.e001)  # IT
print("e2.e002:", e2.e002)  # HR
print("Employee.staff:", Employee.staff)
\end{lstlisting}

\item Написать программу на Python, которая создает класс `Order` с использованием метода `\_\_new\_\_` для контроля именования. Имена должны начинаться с "order\_". Метод `\_\_init\_\_` должен быть пустым.

Инструкции:
\begin{enumerate}
    \item Создайте класс `Order`.
    \item Добавьте атрибут класса `ledger` и инициализируйте его пустым словарем.
    \item Переопределите метод `\_\_new\_\_`, принимающий `cls`, `name`, `total`.
    \item В `\_\_new\_\_`: если `name` не начинается с "order\_", выбросьте `ValueError("Заказ должен иметь префикс 'order\_'")`.
    \item Извлеките ID заказа: `order\_id = name[6:]`.
    \item Создайте экземпляр: `instance = super().\_\_new\_\_(cls)`.
    \item Добавьте ID в словарь: `cls.ledger[order\_id] = instance`.
    \item Установите атрибут экземпляра: `setattr(instance, f"o{order\_id}", total)`.
    \item Верните `instance`.
    \item Переопределите метод `\_\_init\_\_` как пустой.
    \item Создайте объект `o1` с именем "order\_1001" и суммой 150.0.
    \item Создайте объект `o2` с именем "order\_1002" и суммой 89.99.
    \item Попытайтесь создать объект с именем "purchase\_1003" — поймайте исключение.
    \item Выведите `o1.o1001` и `o2.o1002`.
    \item Выведите `Order.ledger`.
\end{enumerate}

Пример использования:
\begin{lstlisting}[language=Python]
o1 = Order("order_1001", 150.0)
o2 = Order("order_1002", 89.99)

try:
    o3 = Order("purchase_1003", 200.0)
except ValueError as e:
    print("Ошибка:", e)

print("o1.o1001:", o1.o1001)  # 150.0
print("o2.o1002:", o2.o1002)  # 89.99
print("Order.ledger:", Order.ledger)
\end{lstlisting}

\item Написать программу на Python, которая создает класс `Ticket` с использованием метода `\_\_new\_\_` для контроля именования. Имена должны начинаться с "ticket\_". Метод `\_\_init\_\_` должен быть пустым.

Инструкции:
\begin{enumerate}
    \item Создайте класс `Ticket`.
    \item Добавьте атрибут класса `database` и инициализируйте его пустым словарем.
    \item Переопределите метод `\_\_new\_\_`, принимающий `cls`, `name`, `priority`.
    \item В `\_\_new\_\_`: если `name` не начинается с "ticket\_", выбросьте `ValueError("Тикет должен иметь префикс 'ticket\_'")`.
    \item Извлеките ID тикета: `ticket\_id = name[7:]`.
    \item Создайте экземпляр: `instance = super().\_\_new\_\_(cls)`.
    \item Добавьте ID в словарь: `cls.database[ticket\_id] = instance`.
    \item Установите атрибут экземпляра: `setattr(instance, f"t{ticket\_id}", priority)`.
    \item Верните `instance`.
    \item Переопределите метод `\_\_init\_\_` как пустой.
    \item Создайте объект `t1` с именем "ticket\_001" и приоритетом "High".
    \item Создайте объект `t2` с именем "ticket\_002" и приоритетом "Low".
    \item Попытайтесь создать объект с именем "issue\_003" — поймайте исключение.
    \item Выведите `t1.t001` и `t2.t002`.
    \item Выведите `Ticket.database`.
\end{enumerate}

Пример использования:
\begin{lstlisting}[language=Python]
t1 = Ticket("ticket_001", "High")
t2 = Ticket("ticket_002", "Low")

try:
    t3 = Ticket("issue_003", "Medium")
except ValueError as e:
    print("Ошибка:", e)

print("t1.t001:", t1.t001)  # High
print("t2.t002:", t2.t002)  # Low
print("Ticket.database:", Ticket.database)
\end{lstlisting}

\item Написать программу на Python, которая создает класс `Project` с использованием метода `\_\_new\_\_` для контроля именования. Имена должны начинаться с "proj\_". Метод `\_\_init\_\_` должен быть пустым.

Инструкции:
\begin{enumerate}
    \item Создайте класс `Project`.
    \item Добавьте атрибут класса `portfolio` и инициализируйте его пустым словарем.
    \item Переопределите метод `\_\_new\_\_`, принимающий `cls`, `name`, `status`.
    \item В `\_\_new\_\_`: если `name` не начинается с "proj\_", выбросьте `ValueError("Проект должен иметь префикс 'proj\_'")`.
    \item Извлеките ID проекта: `proj\_id = name[5:]`.
    \item Создайте экземпляр: `instance = super().\_\_new\_\_(cls)`.
    \item Добавьте ID в словарь: `cls.portfolio[proj\_id] = instance`.
    \item Установите атрибут экземпляра: `setattr(instance, f"pr{proj\_id}", status)`.
    \item Верните `instance`.
    \item Переопределите метод `\_\_init\_\_` как пустой.
    \item Создайте объект `pr1` с именем "proj\_alpha" и статусом "Active".
    \item Создайте объект `pr2` с именем "proj\_beta" и статусом "Inactive".
    \item Попытайтесь создать объект с именем "task\_gamma" — поймайте исключение.
    \item Выведите `pr1.pralpha` и `pr2.prbeta`.
    \item Выведите `Project.portfolio`.
\end{enumerate}

Пример использования:
\begin{lstlisting}[language=Python]
pr1 = Project("proj_alpha", "Active")
pr2 = Project("proj_beta", "Inactive")

try:
    pr3 = Project("task_gamma", "Pending")
except ValueError as e:
    print("Ошибка:", e)

print("pr1.pralpha:", pr1.pralpha)  # Active
print("pr2.prbeta:", pr2.prbeta)    # Inactive
print("Project.portfolio:", Project.portfolio)
\end{lstlisting}

\item Написать программу на Python, которая создает класс `Sensor` с использованием метода `\_\_new\_\_` для контроля именования. Имена должны начинаться с "sensor\_". Метод `\_\_init\_\_` должен быть пустым.

Инструкции:
\begin{enumerate}
    \item Создайте класс `Sensor`.
    \item Добавьте атрибут класса `registry` и инициализируйте его пустым словарем.
    \item Переопределите метод `\_\_new\_\_`, принимающий `cls`, `name`, `type`.
    \item В `\_\_new\_\_`: если `name` не начинается с "sensor\_", выбросьте `ValueError("Сенсор должен иметь префикс 'sensor\_'")`.
    \item Извлеките ID сенсора: `sensor\_id = name[7:]`.
    \item Создайте экземпляр: `instance = super().\_\_new\_\_(cls)`.
    \item Добавьте ID в словарь: `cls.registry[sensor\_id] = instance`.
    \item Установите атрибут экземпляра: `setattr(instance, f"s{sensor\_id}", type)`.
    \item Верните `instance`.
    \item Переопределите метод `\_\_init\_\_` как пустой.
    \item Создайте объект `s1` с именем "sensor\_temp" и типом "Temperature".
    \item Создайте объект `s2` с именем "sensor\_humid" и типом "Humidity".
    \item Попытайтесь создать объект с именем "device\_press" — поймайте исключение.
    \item Выведите `s1.stemp` и `s2.shumid`.
    \item Выведите `Sensor.registry`.
\end{enumerate}

Пример использования:
\begin{lstlisting}[language=Python]
s1 = Sensor("sensor_temp", "Temperature")
s2 = Sensor("sensor_humid", "Humidity")

try:
    s3 = Sensor("device_press", "Pressure")
except ValueError as e:
    print("Ошибка:", e)

print("s1.stemp:", s1.stemp)    # Temperature
print("s2.shumid:", s2.shumid)  # Humidity
print("Sensor.registry:", Sensor.registry)
\end{lstlisting}

\item Написать программу на Python, которая создает класс `Vehicle` с использованием метода `\_\_new\_\_` для контроля именования. Имена должны начинаться с "veh\_". Метод `\_\_init\_\_` должен быть пустым.

Инструкции:
\begin{enumerate}
    \item Создайте класс `Vehicle`.
    \item Добавьте атрибут класса `fleet` и инициализируйте его пустым словарем.
    \item Переопределите метод `\_\_new\_\_`, принимающий `cls`, `name`, `model`.
    \item В `\_\_new\_\_`: если `name` не начинается с "veh\_", выбросьте `ValueError("Транспортное средство должно иметь префикс 'veh\_'")`.
    \item Извлеките ID транспорта: `veh\_id = name[4:]`.
    \item Создайте экземпляр: `instance = super().\_\_new\_\_(cls)`.
    \item Добавьте ID в словарь: `cls.fleet[veh\_id] = instance`.
    \item Установите атрибут экземпляра: `setattr(instance, f"v{veh\_id}", model)`.
    \item Верните `instance`.
    \item Переопределите метод `\_\_init\_\_` как пустой.
    \item Создайте объект `v1` с именем "veh\_car1" и моделью "Sedan".
    \item Создайте объект `v2` с именем "veh\_truck1" и моделью "Pickup".
    \item Попытайтесь создать объект с именем "bike\_01" — поймайте исключение.
    \item Выведите `v1.vcar1` и `v2.vtruck1`.
    \item Выведите `Vehicle.fleet`.
\end{enumerate}

Пример использования:
\begin{lstlisting}[language=Python]
v1 = Vehicle("veh_car1", "Sedan")
v2 = Vehicle("veh_truck1", "Pickup")

try:
    v3 = Vehicle("bike_01", "Mountain")
except ValueError as e:
    print("Ошибка:", e)

print("v1.vcar1:", v1.vcar1)    # Sedan
print("v2.vtruck1:", v2.vtruck1) # Pickup
print("Vehicle.fleet:", Vehicle.fleet)
\end{lstlisting}

\item Написать программу на Python, которая создает класс `Animal` с использованием метода `\_\_new\_\_` для контроля именования. Имена должны начинаться с "animal\_". Метод `\_\_init\_\_` должен быть пустым.

Инструкции:
\begin{enumerate}
    \item Создайте класс `Animal`.
    \item Добавьте атрибут класса `zoo` и инициализируйте его пустым словарем.
    \item Переопределите метод `\_\_new\_\_`, принимающий `cls`, `name`, `species`.
    \item В `\_\_new\_\_`: если `name` не начинается с "animal\_", выбросьте `ValueError("Животное должно иметь префикс 'animal\_'")`.
    \item Извлеките ID животного: `animal\_id = name[7:]`.
    \item Создайте экземпляр: `instance = super().\_\_new\_\_(cls)`.
    \item Добавьте ID в словарь: `cls.zoo[animal\_id] = instance`.
    \item Установите атрибут экземпляра: `setattr(instance, f"a{animal\_id}", species)`.
    \item Верните `instance`.
    \item Переопределите метод `\_\_init\_\_` как пустой.
    \item Создайте объект `a1` с именем "animal\_lion" и видом "Panthera leo".
    \item Создайте объект `a2` с именем "animal\_elephant" и видом "Loxodonta".
    \item Попытайтесь создать объект с именем "creature\_tiger" — поймайте исключение.
    \item Выведите `a1.alion` и `a2.aelephant`.
    \item Выведите `Animal.zoo`.
\end{enumerate}

Пример использования:
\begin{lstlisting}[language=Python]
a1 = Animal("animal_lion", "Panthera leo")
a2 = Animal("animal_elephant", "Loxodonta")

try:
    a3 = Animal("creature_tiger", "Panthera tigris")
except ValueError as e:
    print("Ошибка:", e)

print("a1.alion:", a1.alion)        # Panthera leo
print("a2.aelephant:", a2.aelephant) # Loxodonta
print("Animal.zoo:", Animal.zoo)
\end{lstlisting}

\item Написать программу на Python, которая создает класс `Plant` с использованием метода `\_\_new\_\_` для контроля именования. Имена должны начинаться с "plant\_". Метод `\_\_init\_\_` должен быть пустым.

Инструкции:
\begin{enumerate}
    \item Создайте класс `Plant`.
    \item Добавьте атрибут класса `greenhouse` и инициализируйте его пустым словарем.
    \item Переопределите метод `\_\_new\_\_`, принимающий `cls`, `name`, `family`.
    \item В `\_\_new\_\_`: если `name` не начинается с "plant\_", выбросьте `ValueError("Растение должно иметь префикс 'plant\_'")`.
    \item Извлеките ID растения: `plant\_id = name[6:]`.
    \item Создайте экземпляр: `instance = super().\_\_new\_\_(cls)`.
    \item Добавьте ID в словарь: `cls.greenhouse[plant\_id] = instance`.
    \item Установите атрибут экземпляра: `setattr(instance, f"pl{plant\_id}", family)`.
    \item Верните `instance`.
    \item Переопределите метод `\_\_init\_\_` как пустой.
    \item Создайте объект `pl1` с именем "plant\_rose" и семейством "Rosaceae".
    \item Создайте объект `pl2` с именем "plant\_oak" и семейством "Fagaceae".
    \item Попытайтесь создать объект с именем "tree\_pine" — поймайте исключение.
    \item Выведите `pl1.plrose` и `pl2.ploak`.
    \item Выведите `Plant.greenhouse`.
\end{enumerate}

Пример использования:
\begin{lstlisting}[language=Python]
pl1 = Plant("plant_rose", "Rosaceae")
pl2 = Plant("plant_oak", "Fagaceae")

try:
    pl3 = Plant("tree_pine", "Pinaceae")
except ValueError as e:
    print("Ошибка:", e)

print("pl1.plrose:", pl1.plrose)  # Rosaceae
print("pl2.ploak:", pl2.ploak)    # Fagaceae
print("Plant.greenhouse:", Plant.greenhouse)
\end{lstlisting}

\item Написать программу на Python, которая создает класс `Planet` с использованием метода `\_\_new\_\_` для контроля именования. Имена должны начинаться с "planet\_". Метод `\_\_init\_\_` должен быть пустым.

Инструкции:
\begin{enumerate}
    \item Создайте класс `Planet`.
    \item Добавьте атрибут класса `solar\_system` и инициализируйте его пустым словарем.
    \item Переопределите метод `\_\_new\_\_`, принимающий `cls`, `name`, `type`.
    \item В `\_\_new\_\_`: если `name` не начинается с "planet\_", выбросьте `ValueError("Планета должна иметь префикс 'planet\_'")`.
    \item Извлеките ID планеты: `planet\_id = name[7:]`.
    \item Создайте экземпляр: `instance = super().\_\_new\_\_(cls)`.
    \item Добавьте ID в словарь: `cls.solar\_system[planet\_id] = instance`.
    \item Установите атрибут экземпляра: `setattr(instance, f"pn{planet\_id}", type)`.
    \item Верните `instance`.
    \item Переопределите метод `\_\_init\_\_` как пустой.
    \item Создайте объект `pn1` с именем "planet\_earth" и типом "Terrestrial".
    \item Создайте объект `pn2` с именем "planet\_jupiter" и типом "Gas Giant".
    \item Попытайтесь создать объект с именем "star\_sun" — поймайте исключение.
    \item Выведите `pn1.pnearth` и `pn2.pnjupiter`.
    \item Выведите `Planet.solar\_system`.
\end{enumerate}

Пример использования:
\begin{lstlisting}[language=Python]
pn1 = Planet("planet_earth", "Terrestrial")
pn2 = Planet("planet_jupiter", "Gas Giant")

try:
    pn3 = Planet("star_sun", "Star")
except ValueError as e:
    print("Ошибка:", e)

print("pn1.pnearth:", pn1.pnearth)    # Terrestrial
print("pn2.pnjupiter:", pn2.pnjupiter) # Gas Giant
print("Planet.solar_system:", Planet.solar_system)
\end{lstlisting}

\item Написать программу на Python, которая создает класс `Star` с использованием метода `\_\_new\_\_` для контроля именования. Имена должны начинаться с "star\_". Метод `\_\_init\_\_` должен быть пустым.

Инструкции:
\begin{enumerate}
    \item Создайте класс `Star`.
    \item Добавьте атрибут класса `galaxy` и инициализируйте его пустым словарем.
    \item Переопределите метод `\_\_new\_\_`, принимающий `cls`, `name`, `class\_type`.
    \item В `\_\_new\_\_`: если `name` не начинается с "star\_", выбросьте `ValueError("Звезда должна иметь префикс 'star\_'")`.
    \item Извлеките ID звезды: `star\_id = name[5:]`.
    \item Создайте экземпляр: `instance = super().\_\_new\_\_(cls)`.
    \item Добавьте ID в словарь: `cls.galaxy[star\_id] = instance`.
    \item Установите атрибут экземпляра: `setattr(instance, f"st{star\_id}", class\_type)`.
    \item Верните `instance`.
    \item Переопределите метод `\_\_init\_\_` как пустой.
    \item Создайте объект `st1` с именем "star\_sol" и классом "G2V".
    \item Создайте объект `st2` с именем "star\_proxima" и классом "M5.5V".
    \item Попытайтесь создать объект с именем "nova\_1" — поймайте исключение.
    \item Выведите `st1.stsol` и `st2.stproxima`.
    \item Выведите `Star.galaxy`.
\end{enumerate}

Пример использования:
\begin{lstlisting}[language=Python]
st1 = Star("star_sol", "G2V")
st2 = Star("star_proxima", "M5.5V")

try:
    st3 = Star("nova_1", "Variable")
except ValueError as e:
    print("Ошибка:", e)

print("st1.stsol:", st1.stsol)      # G2V
print("st2.stproxima:", st2.stproxima) # M5.5V
print("Star.galaxy:", Star.galaxy)
\end{lstlisting}

\item Написать программу на Python, которая создает класс `Galaxy` с использованием метода `\_\_new\_\_` для контроля именования. Имена должны начинаться с "galaxy\_". Метод `\_\_init\_\_` должен быть пустым.

Инструкции:
\begin{enumerate}
    \item Создайте класс `Galaxy`.
    \item Добавьте атрибут класса `universe` и инициализируйте его пустым словарем.
    \item Переопределите метод `\_\_new\_\_`, принимающий `cls`, `name`, `type`.
    \item В `\_\_new\_\_`: если `name` не начинается с "galaxy\_", выбросьте `ValueError("Галактика должна иметь префикс 'galaxy\_'")`.
    \item Извлеките ID галактики: `galaxy\_id = name[7:]`.
    \item Создайте экземпляр: `instance = super().\_\_new\_\_(cls)`.
    \item Добавьте ID в словарь: `cls.universe[galaxy\_id] = instance`.
    \item Установите атрибут экземпляра: `setattr(instance, f"g{galaxy\_id}", type)`.
    \item Верните `instance`.
    \item Переопределите метод `\_\_init\_\_` как пустой.
    \item Создайте объект `g1` с именем "galaxy\_milkyway" и типом "Spiral".
    \item Создайте объект `g2` с именем "galaxy\_andromeda" и типом "Spiral".
    \item Попытайтесь создать объект с именем "cluster\_virgo" — поймайте исключение.
    \item Выведите `g1.gmilkyway` и `g2.gandromeda`.
    \item Выведите `Galaxy.universe`.
\end{enumerate}

Пример использования:
\begin{lstlisting}[language=Python]
g1 = Galaxy("galaxy_milkyway", "Spiral")
g2 = Galaxy("galaxy_andromeda", "Spiral")

try:
    g3 = Galaxy("cluster_virgo", "Cluster")
except ValueError as e:
    print("Ошибка:", e)

print("g1.gmilkyway:", g1.gmilkyway)    # Spiral
print("g2.gandromeda:", g2.gandromeda)  # Spiral
print("Galaxy.universe:", Galaxy.universe)
\end{lstlisting}

\item Написать программу на Python, которая создает класс `Constellation` с использованием метода `\_\_new\_\_` для контроля именования. Имена должны начинаться с "const\_". Метод `\_\_init\_\_` должен быть пустым.

Инструкции:
\begin{enumerate}
    \item Создайте класс `Constellation`.
    \item Добавьте атрибут класса `sky\_map` и инициализируйте его пустым словарем.
    \item Переопределите метод `\_\_new\_\_`, принимающий `cls`, `name`, `stars`.
    \item В `\_\_new\_\_`: если `name` не начинается с "const\_", выбросьте `ValueError("Созвездие должно иметь префикс 'const\_'")`.
    \item Извлеките ID созвездия: `const\_id = name[6:]`.
    \item Создайте экземпляр: `instance = super().\_\_new\_\_(cls)`.
    \item Добавьте ID в словарь: `cls.sky\_map[const\_id] = instance`.
    \item Установите атрибут экземпляра: `setattr(instance, f"c{const\_id}", stars)`.
    \item Верните `instance`.
    \item Переопределите метод `\_\_init\_\_` как пустой.
    \item Создайте объект `c1` с именем "const\_orion" и количеством звезд 81.
    \item Создайте объект `c2` с именем "const\_ursa" и количеством звезд 20.
    \item Попытайтесь создать объект с именем "asterism\_bigdipper" — поймайте исключение.
    \item Выведите `c1.corion` и `c2.cursa`.
    \item Выведите `Constellation.sky\_map`.
\end{enumerate}

Пример использования:
\begin{lstlisting}[language=Python]
c1 = Constellation("const_orion", 81)
c2 = Constellation("const_ursa", 20)

try:
    c3 = Constellation("asterism_bigdipper", 7)
except ValueError as e:
    print("Ошибка:", e)

print("c1.corion:", c1.corion)  # 81
print("c2.cursa:", c2.cursa)    # 20
print("Constellation.sky_map:", Constellation.sky_map)
\end{lstlisting}

\item Написать программу на Python, которая создает класс `Asteroid` с использованием метода `\_\_new\_\_` для контроля именования. Имена должны начинаться с "ast\_". Метод `\_\_init\_\_` должен быть пустым.

Инструкции:
\begin{enumerate}
    \item Создайте класс `Asteroid`.
    \item Добавьте атрибут класса `belt` и инициализируйте его пустым словарем.
    \item Переопределите метод `\_\_new\_\_`, принимающий `cls`, `name`, `diameter`.
    \item В `\_\_new\_\_`: если `name` не начинается с "ast\_", выбросьте `ValueError("Астероид должен иметь префикс 'ast\_'")`.
    \item Извлеките ID астероида: `ast\_id = name[4:]`.
    \item Создайте экземпляр: `instance = super().\_\_new\_\_(cls)`.
    \item Добавьте ID в словарь: `cls.belt[ast\_id] = instance`.
    \item Установите атрибут экземпляра: `setattr(instance, f"a{ast\_id}", diameter)`.
    \item Верните `instance`.
    \item Переопределите метод `\_\_init\_\_` как пустой.
    \item Создайте объект `a1` с именем "ast\_ceres" и диаметром 939.
    \item Создайте объект `a2` с именем "ast\_vesta" и диаметром 525.
    \item Попытайтесь создать объект с именем "meteor\_id8" — поймайте исключение.
    \item Выведите `a1.aceres` и `a2.avesta`.
    \item Выведите `Asteroid.belt`.
\end{enumerate}

Пример использования:
\begin{lstlisting}[language=Python]
a1 = Asteroid("ast_ceres", 939)
a2 = Asteroid("ast_vesta", 525)

try:
    a3 = Asteroid("meteor_id8", 10)
except ValueError as e:
    print("Ошибка:", e)

print("a1.aceres:", a1.aceres)  # 939
print("a2.avesta:", a2.avesta)  # 525
print("Asteroid.belt:", Asteroid.belt)
\end{lstlisting}

\item Написать программу на Python, которая создает класс `Comet` с использованием метода `\_\_new\_\_` для контроля именования. Имена должны начинаться с "comet\_". Метод `\_\_init\_\_` должен быть пустым.

Инструкции:
\begin{enumerate}
    \item Создайте класс `Comet`.
    \item Добавьте атрибут класса `orbits` и инициализируйте его пустым словарем.
    \item Переопределите метод `\_\_new\_\_`, принимающий `cls`, `name`, `period`.
    \item В `\_\_new\_\_`: если `name` не начинается с "comet\_", выбросьте `ValueError("Комета должна иметь префикс 'comet\_'")`.
    \item Извлеките ID кометы: `comet\_id = name[6:]`.
    \item Создайте экземпляр: `instance = super().\_\_new\_\_(cls)`.
    \item Добавьте ID в словарь: `cls.orbits[comet\_id] = instance`.
    \item Установите атрибут экземпляра: `setattr(instance, f"cm{comet\_id}", period)`.
    \item Верните `instance`.
    \item Переопределите метод `\_\_init\_\_` как пустой.
    \item Создайте объект `cm1` с именем "comet\_halley" и периодом 76.
    \item Создайте объект `cm2` с именем "comet\_encke" и периодом 3.3.
    \item Попытайтесь создать объект с именем "meteor\_shower" — поймайте исключение.
    \item Выведите `cm1.cmhalley` и `cm2.cmencke`.
    \item Выведите `Comet.orbits`.
\end{enumerate}

Пример использования:
\begin{lstlisting}[language=Python]
cm1 = Comet("comet_halley", 76)
cm2 = Comet("comet_encke", 3.3)

try:
    cm3 = Comet("meteor_shower", 0.1)
except ValueError as e:
    print("Ошибка:", e)

print("cm1.cmhalley:", cm1.cmhalley)  # 76
print("cm2.cmencke:", cm2.cmencke)    # 3.3
print("Comet.orbits:", Comet.orbits)
\end{lstlisting}

\item Написать программу на Python, которая создает класс `Satellite` с использованием метода `\_\_new\_\_` для контроля именования. Имена должны начинаться с "sat\_". Метод `\_\_init\_\_` должен быть пустым.

Инструкции:
\begin{enumerate}
    \item Создайте класс `Satellite`.
    \item Добавьте атрибут класса `orbiters` и инициализируйте его пустым словарем.
    \item Переопределите метод `\_\_new\_\_`, принимающий `cls`, `name`, `planet`.
    \item В `\_\_new\_\_`: если `name` не начинается с "sat\_", выбросьте `ValueError("Спутник должен иметь префикс 'sat\_'")`.
    \item Извлеките ID спутника: `sat\_id = name[4:]`.
    \item Создайте экземпляр: `instance = super().\_\_new\_\_(cls)`.
    \item Добавьте ID в словарь: `cls.orbiters[sat\_id] = instance`.
    \item Установите атрибут экземпляра: `setattr(instance, f"s{sat\_id}", planet)`.
    \item Верните `instance`.
    \item Переопределите метод `\_\_init\_\_` как пустой.
    \item Создайте объект `s1` с именем "sat\_moon" и планетой "Earth".
    \item Создайте объект `s2` с именем "sat\_phobos" и планетой "Mars".
    \item Попытайтесь создать объект с именем "rover\_curiosity" — поймайте исключение.
    \item Выведите `s1.smoon` и `s2.sphobos`.
    \item Выведите `Satellite.orbiters`.
\end{enumerate}

Пример использования:
\begin{lstlisting}[language=Python]
s1 = Satellite("sat_moon", "Earth")
s2 = Satellite("sat_phobos", "Mars")

try:
    s3 = Satellite("rover_curiosity", "Mars")
except ValueError as e:
    print("Ошибка:", e)

print("s1.smoon:", s1.smoon)    # Earth
print("s2.sphobos:", s2.sphobos) # Mars
print("Satellite.orbiters:", Satellite.orbiters)
\end{lstlisting}

\item Написать программу на Python, которая создает класс `Rocket` с использованием метода `\_\_new\_\_` для контроля именования. Имена должны начинаться с "rocket\_". Метод `\_\_init\_\_` должен быть пустым.

Инструкции:
\begin{enumerate}
    \item Создайте класс `Rocket`.
    \item Добавьте атрибут класса `launchpad` и инициализируйте его пустым словарем.
    \item Переопределите метод `\_\_new\_\_`, принимающий `cls`, `name`, `payload`.
    \item В `\_\_new\_\_`: если `name` не начинается с "rocket\_", выбросьте `ValueError("Ракета должна иметь префикс 'rocket\_'")`.
    \item Извлеките ID ракеты: `rocket\_id = name[7:]`.
    \item Создайте экземпляр: `instance = super().\_\_new\_\_(cls)`.
    \item Добавьте ID в словарь: `cls.launchpad[rocket\_id] = instance`.
    \item Установите атрибут экземпляра: `setattr(instance, f"r{rocket\_id}", payload)`.
    \item Верните `instance`.
    \item Переопределите метод `\_\_init\_\_` как пустой.
    \item Создайте объект `r1` с именем "rocket\_falcon9" и полезной нагрузкой "Starlink".
    \item Создайте объект `r2` с именем "rocket\_atlas" и полезной нагрузкой "GPS".
    \item Попытайтесь создать объект с именем "drone\_delivery" — поймайте исключение.
    \item Выведите `r1.rfalcon9` и `r2.ratlas`.
    \item Выведите `Rocket.launchpad`.
\end{enumerate}

Пример использования:
\begin{lstlisting}[language=Python]
r1 = Rocket("rocket_falcon9", "Starlink")
r2 = Rocket("rocket_atlas", "GPS")

try:
    r3 = Rocket("drone_delivery", "Package")
except ValueError as e:
    print("Ошибка:", e)

print("r1.rfalcon9:", r1.rfalcon9)  # Starlink
print("r2.ratlas:", r2.ratlas)      # GPS
print("Rocket.launchpad:", Rocket.launchpad)
\end{lstlisting}

\item Написать программу на Python, которая создает класс `Drone` с использованием метода `\_\_new\_\_` для контроля именования. Имена должны начинаться с "drone\_". Метод `\_\_init\_\_` должен быть пустым.

Инструкции:
\begin{enumerate}
    \item Создайте класс `Drone`.
    \item Добавьте атрибут класса `fleet` и инициализируйте его пустым словарем.
    \item Переопределите метод `\_\_new\_\_`, принимающий `cls`, `name`, `range`.
    \item В `\_\_new\_\_`: если `name` не начинается с "drone\_", выбросьте `ValueError("Дрон должен иметь префикс 'drone\_'")`.
    \item Извлеките ID дрона: `drone\_id = name[6:]`.
    \item Создайте экземпляр: `instance = super().\_\_new\_\_(cls)`.
    \item Добавьте ID в словарь: `cls.fleet[drone\_id] = instance`.
    \item Установите атрибут экземпляра: `setattr(instance, f"d{drone\_id}", range)`.
    \item Верните `instance`.
    \item Переопределите метод `\_\_init\_\_` как пустой.
    \item Создайте объект `d1` с именем "drone\_x1" и дальностью 5.
    \item Создайте объект `d2` с именем "drone\_x2" и дальностью 10.
    \item Попытайтесь создать объект с именем "robot\_r1" — поймайте исключение.
    \item Выведите `d1.dx1` и `d2.dx2`.
    \item Выведите `Drone.fleet`.
\end{enumerate}

Пример использования:
\begin{lstlisting}[language=Python]
d1 = Drone("drone_x1", 5)
d2 = Drone("drone_x2", 10)

try:
    d3 = Drone("robot_r1", 2)
except ValueError as e:
    print("Ошибка:", e)

print("d1.dx1:", d1.dx1)  # 5
print("d2.dx2:", d2.dx2)  # 10
print("Drone.fleet:", Drone.fleet)
\end{lstlisting}

\item Написать программу на Python, которая создает класс `Robot` с использованием метода `\_\_new\_\_` для контроля именования. Имена должны начинаться с "robot\_". Метод `\_\_init\_\_` должен быть пустым.

Инструкции:
\begin{enumerate}
    \item Создайте класс `Robot`.
    \item Добавьте атрибут класса `factory` и инициализируйте его пустым словарем.
    \item Переопределите метод `\_\_new\_\_`, принимающий `cls`, `name`, `function`.
    \item В `\_\_new\_\_`: если `name` не начинается с "robot\_", выбросьте `ValueError("Робот должен иметь префикс 'robot\_'")`.
    \item Извлеките ID робота: `robot\_id = name[6:]`.
    \item Создайте экземпляр: `instance = super().\_\_new\_\_(cls)`.
    \item Добавьте ID в словарь: `cls.factory[robot\_id] = instance`.
    \item Установите атрибут экземпляра: `setattr(instance, f"rb{robot\_id}", function)`.
    \item Верните `instance`.
    \item Переопределите метод `\_\_init\_\_` как пустой.
    \item Создайте объект `rb1` с именем "robot\_arm" и функцией "Assembly".
    \item Создайте объект `rb2` с именем "robot\_cleaner" и функцией "Cleaning".
    \item Попытайтесь создать объект с именем "android\_unit" — поймайте исключение.
    \item Выведите `rb1.rbarm` и `rb2.rbcleaner`.
    \item Выведите `Robot.factory`.
\end{enumerate}

Пример использования:
\begin{lstlisting}[language=Python]
rb1 = Robot("robot_arm", "Assembly")
rb2 = Robot("robot_cleaner", "Cleaning")

try:
    rb3 = Robot("android_unit", "General")
except ValueError as e:
    print("Ошибка:", e)

print("rb1.rbarm:", rb1.rbarm)      # Assembly
print("rb2.rbcleaner:", rb2.rbcleaner) # Cleaning
print("Robot.factory:", Robot.factory)
\end{lstlisting}

\item Написать программу на Python, которая создает класс `AI` с использованием метода `\_\_new\_\_` для контроля именования. Имена должны начинаться с "ai\_". Метод `\_\_init\_\_` должен быть пустым.

Инструкции:
\begin{enumerate}
    \item Создайте класс `AI`.
    \item Добавьте атрибут класса `network` и инициализируйте его пустым словарем.
    \item Переопределите метод `\_\_new\_\_`, принимающий `cls`, `name`, `capability`.
    \item В `\_\_new\_\_`: если `name` не начинается с "ai\_", выбросьте `ValueError("ИИ должен иметь префикс 'ai\_'")`.
    \item Извлеките ID ИИ: `ai\_id = name[3:]`.
    \item Создайте экземпляр: `instance = super().\_\_new\_\_(cls)`.
    \item Добавьте ID в словарь: `cls.network[ai\_id] = instance`.
    \item Установите атрибут экземпляра: `setattr(instance, f"a{ai\_id}", capability)`.
    \item Верните `instance`.
    \item Переопределите метод `\_\_init\_\_` как пустой.
    \item Создайте объект `a1` с именем "ai\_alpha" и возможностью "NLP".
    \item Создайте объект `a2` с именем "ai\_beta" и возможностью "CV".
    \item Попытайтесь создать объект с именем "ml\_model" — поймайте исключение.
    \item Выведите `a1.aalpha` и `a2.abeta`.
    \item Выведите `AI.network`.
\end{enumerate}

Пример использования:
\begin{lstlisting}[language=Python]
a1 = AI("ai_alpha", "NLP")
a2 = AI("ai_beta", "CV")

try:
    a3 = AI("ml_model", "Regression")
except ValueError as e:
    print("Ошибка:", e)

print("a1.aalpha:", a1.aalpha)  # NLP
print("a2.abeta:", a2.abeta)    # CV
print("AI.network:", AI.network)
\end{lstlisting}

\item Написать программу на Python, которая создает класс `MLModel` с использованием метода `\_\_new\_\_` для контроля именования. Имена должны начинаться с "model\_". Метод `\_\_init\_\_` должен быть пустым.

Инструкции:
\begin{enumerate}
    \item Создайте класс `MLModel`.
    \item Добавьте атрибут класса `repository` и инициализируйте его пустым словарем.
    \item Переопределите метод `\_\_new\_\_`, принимающий `cls`, `name`, `algorithm`.
    \item В `\_\_new\_\_`: если `name` не начинается с "model\_", выбросьте `ValueError("Модель должна иметь префикс 'model\_'")`.
    \item Извлеките ID модели: `model\_id = name[6:]`.
    \item Создайте экземпляр: `instance = super().\_\_new\_\_(cls)`.
    \item Добавьте ID в словарь: `cls.repository[model\_id] = instance`.
    \item Установите атрибут экземпляра: `setattr(instance, f"m{model\_id}", algorithm)`.
    \item Верните `instance`.
    \item Переопределите метод `\_\_init\_\_` как пустой.
    \item Создайте объект `m1` с именем "model\_logreg" и алгоритмом "Logistic Regression".
    \item Создайте объект `m2` с именем "model\_svm" и алгоритмом "Support Vector Machine".
    \item Попытайтесь создать объект с именем "algo\_randomforest" — поймайте исключение.
    \item Выведите `m1.mlogreg` и `m2.msvm`.
    \item Выведите `MLModel.repository`.
\end{enumerate}

Пример использования:
\begin{lstlisting}[language=Python]
m1 = MLModel("model_logreg", "Logistic Regression")
m2 = MLModel("model_svm", "Support Vector Machine")

try:
    m3 = MLModel("algo_randomforest", "Random Forest")
except ValueError as e:
    print("Ошибка:", e)

print("m1.mlogreg:", m1.mlogreg)  # Logistic Regression
print("m2.msvm:", m2.msvm)        # Support Vector Machine
print("MLModel.repository:", MLModel.repository)
\end{lstlisting}

\item Написать программу на Python, которая создает класс `Dataset` с использованием метода `\_\_new\_\_` для контроля именования. Имена должны начинаться с "dataset\_". Метод `\_\_init\_\_` должен быть пустым.

Инструкции:
\begin{enumerate}
    \item Создайте класс `Dataset`.
    \item Добавьте атрибут класса `catalog` и инициализируйте его пустым словарем.
    \item Переопределите метод `\_\_new\_\_`, принимающий `cls`, `name`, `size`.
    \item В `\_\_new\_\_`: если `name` не начинается с "dataset\_", выбросьте `ValueError("Набор данных должен иметь префикс 'dataset\_'")`.
    \item Извлеките ID набора данных: `dataset\_id = name[8:]`.
    \item Создайте экземпляр: `instance = super().\_\_new\_\_(cls)`.
    \item Добавьте ID в словарь: `cls.catalog[dataset\_id] = instance`.
    \item Установите атрибут экземпляра: `setattr(instance, f"ds{dataset\_id}", size)`.
    \item Верните `instance`.
    \item Переопределите метод `\_\_init\_\_` как пустой.
    \item Создайте объект `ds1` с именем "dataset\_train" и размером 10000.
    \item Создайте объект `ds2` с именем "dataset\_test" и размером 2000.
    \item Попытайтесь создать объект с именем "data\_validation" — поймайте исключение.
    \item Выведите `ds1.dstrain` и `ds2.dstest`.
    \item Выведите `Dataset.catalog`.
\end{enumerate}

Пример использования:
\begin{lstlisting}[language=Python]
ds1 = Dataset("dataset_train", 10000)
ds2 = Dataset("dataset_test", 2000)

try:
    ds3 = Dataset("data_validation", 2000)
except ValueError as e:
    print("Ошибка:", e)

print("ds1.dstrain:", ds1.dstrain)  # 10000
print("ds2.dstest:", ds2.dstest)    # 2000
print("Dataset.catalog:", Dataset.catalog)
\end{lstlisting}

\item Написать программу на Python, которая создает класс `Feature` с использованием метода `\_\_new\_\_` для контроля именования. Имена должны начинаться с "feat\_". Метод `\_\_init\_\_` должен быть пустым.

Инструкции:
\begin{enumerate}
    \item Создайте класс `Feature`.
    \item Добавьте атрибут класса `registry` и инициализируйте его пустым словарем.
    \item Переопределите метод `\_\_new\_\_`, принимающий `cls`, `name`, `type`.
    \item В `\_\_new\_\_`: если `name` не начинается с "feat\_", выбросьте `ValueError("Признак должен иметь префикс 'feat\_'")`.
    \item Извлеките ID признака: `feat\_id = name[5:]`.
    \item Создайте экземпляр: `instance = super().\_\_new\_\_(cls)`.
    \item Добавьте ID в словарь: `cls.registry[feat\_id] = instance`.
    \item Установите атрибут экземпляра: `setattr(instance, f"f{feat\_id}", type)`.
    \item Верните `instance`.
    \item Переопределите метод `\_\_init\_\_` как пустой.
    \item Создайте объект `f1` с именем "feat\_age" и типом "Numeric".
    \item Создайте объект `f2` с именем "feat\_gender" и типом "Categorical".
    \item Попытайтесь создать объект с именем "attr\_income" — поймайте исключение.
    \item Выведите `f1.fage` и `f2.fgender`.
    \item Выведите `Feature.registry`.
\end{enumerate}

Пример использования:
\begin{lstlisting}[language=Python]
f1 = Feature("feat_age", "Numeric")
f2 = Feature("feat_gender", "Categorical")

try:
    f3 = Feature("attr_income", "Numeric")
except ValueError as e:
    print("Ошибка:", e)

print("f1.fage:", f1.fage)      # Numeric
print("f2.fgender:", f2.fgender) # Categorical
print("Feature.registry:", Feature.registry)
\end{lstlisting}

\item Написать программу на Python, которая создает класс `Label` с использованием метода `\_\_new\_\_` для контроля именования. Имена должны начинаться с "label\_". Метод `\_\_init\_\_` должен быть пустым.

Инструкции:
\begin{enumerate}
    \item Создайте класс `Label`.
    \item Добавьте атрибут класса `index` и инициализируйте его пустым словарем.
    \item Переопределите метод `\_\_new\_\_`, принимающий `cls`, `name`, `class\_name`.
    \item В `\_\_new\_\_`: если `name` не начинается с "label\_", выбросьте `ValueError("Метка должна иметь префикс 'label\_'")`.
    \item Извлеките ID метки: `label\_id = name[6:]`.
    \item Создайте экземпляр: `instance = super().\_\_new\_\_(cls)`.
    \item Добавьте ID в словарь: `cls.index[label\_id] = instance`.
    \item Установите атрибут экземпляра: `setattr(instance, f"l{label\_id}", class\_name)`.
    \item Верните `instance`.
    \item Переопределите метод `\_\_init\_\_` как пустой.
    \item Создайте объект `l1` с именем "label\_cat" и классом "Animal".
    \item Создайте объект `l2` с именем "label\_car" и классом "Vehicle".
    \item Попытайтесь создать объект с именем "tag\_dog" — поймайте исключение.
    \item Выведите `l1.lcat` и `l2.lcar`.
    \item Выведите `Label.index`.
\end{enumerate}

Пример использования:
\begin{lstlisting}[language=Python]
l1 = Label("label_cat", "Animal")
l2 = Label("label_car", "Vehicle")

try:
    l3 = Label("tag_dog", "Animal")
except ValueError as e:
    print("Ошибка:", e)

print("l1.lcat:", l1.lcat)  # Animal
print("l2.lcar:", l2.lcar)  # Vehicle
print("Label.index:", Label.index)
\end{lstlisting}

\item Написать программу на Python, которая создает класс `Layer` с использованием метода `\_\_new\_\_` для контроля именования. Имена должны начинаться с "layer\_". Метод `\_\_init\_\_` должен быть пустым.

Инструкции:
\begin{enumerate}
    \item Создайте класс `Layer`.
    \item Добавьте атрибут класса `stack` и инициализируйте его пустым словарем.
    \item Переопределите метод `\_\_new\_\_`, принимающий `cls`, `name`, `neurons`.
    \item В `\_\_new\_\_`: если `name` не начинается с "layer\_", выбросьте `ValueError("Слой должен иметь префикс 'layer\_'")`.
    \item Извлеките ID слоя: `layer\_id = name[6:]`.
    \item Создайте экземпляр: `instance = super().\_\_new\_\_(cls)`.
    \item Добавьте ID в словарь: `cls.stack[layer\_id] = instance`.
    \item Установите атрибут экземпляра: `setattr(instance, f"ly{layer\_id}", neurons)`.
    \item Верните `instance`.
    \item Переопределите метод `\_\_init\_\_` как пустой.
    \item Создайте объект `ly1` с именем "layer\_input" и нейронами 784.
    \item Создайте объект `ly2` с именем "layer\_hidden" и нейронами 128.
    \item Попытайтесь создать объект с именем "unit\_output" — поймайте исключение.
    \item Выведите `ly1.lyinput` и `ly2.lyhidden`.
    \item Выведите `Layer.stack`.
\end{enumerate}

Пример использования:
\begin{lstlisting}[language=Python]
ly1 = Layer("layer_input", 784)
ly2 = Layer("layer_hidden", 128)

try:
    ly3 = Layer("unit_output", 10)
except ValueError as e:
    print("Ошибка:", e)

print("ly1.lyinput:", ly1.lyinput)    # 784
print("ly2.lyhidden:", ly2.lyhidden)  # 128
print("Layer.stack:", Layer.stack)
\end{lstlisting}

\item Написать программу на Python, которая создает класс `Neuron` с использованием метода `\_\_new\_\_` для контроля именования. Имена должны начинаться с "neuron\_". Метод `\_\_init\_\_` должен быть пустым.

Инструкции:
\begin{enumerate}
    \item Создайте класс `Neuron`.
    \item Добавьте атрибут класса `brain` и инициализируйте его пустым словарем.
    \item Переопределите метод `\_\_new\_\_`, принимающий `cls`, `name`, `activation`.
    \item В `\_\_new\_\_`: если `name` не начинается с "neuron\_", выбросьте `ValueError("Нейрон должен иметь префикс 'neuron\_'")`.
    \item Извлеките ID нейрона: `neuron\_id = name[7:]`.
    \item Создайте экземпляр: `instance = super().\_\_new\_\_(cls)`.
    \item Добавьте ID в словарь: `cls.brain[neuron\_id] = instance`.
    \item Установите атрибут экземпляра: `setattr(instance, f"n{neuron\_id}", activation)`.
    \item Верните `instance`.
    \item Переопределите метод `\_\_init\_\_` как пустой.
    \item Создайте объект `n1` с именем "neuron\_1" и активацией "ReLU".
    \item Создайте объект `n2` с именем "neuron\_2" и активацией "Sigmoid".
    \item Попытайтесь создать объект с именем "cell\_3" — поймайте исключение.
    \item Выведите `n1.n1` и `n2.n2`.
    \item Выведите `Neuron.brain`.
\end{enumerate}

Пример использования:
\begin{lstlisting}[language=Python]
n1 = Neuron("neuron_1", "ReLU")
n2 = Neuron("neuron_2", "Sigmoid")

try:
    n3 = Neuron("cell_3", "Tanh")
except ValueError as e:
    print("Ошибка:", e)

print("n1.n1:", n1.n1)  # ReLU
print("n2.n2:", n2.n2)  # Sigmoid
print("Neuron.brain:", Neuron.brain)
\end{lstlisting}

\item Написать программу на Python, которая создает класс `Synapse` с использованием метода `\_\_new\_\_` для контроля именования. Имена должны начинаться с "syn\_". Метод `\_\_init\_\_` должен быть пустым.

Инструкции:
\begin{enumerate}
    \item Создайте класс `Synapse`.
    \item Добавьте атрибут класса `connections` и инициализируйте его пустым словарем.
    \item Переопределите метод `\_\_new\_\_`, принимающий `cls`, `name`, `weight`.
    \item В `\_\_new\_\_`: если `name` не начинается с "syn\_", выбросьте `ValueError("Синапс должен иметь префикс 'syn\_'")`.
    \item Извлеките ID синапса: `syn\_id = name[4:]`.
    \item Создайте экземпляр: `instance = super().\_\_new\_\_(cls)`.
    \item Добавьте ID в словарь: `cls.connections[syn\_id] = instance`.
    \item Установите атрибут экземпляра: `setattr(instance, f"s{syn\_id}", weight)`.
    \item Верните `instance`.
    \item Переопределите метод `\_\_init\_\_` как пустой.
    \item Создайте объект `s1` с именем "syn\_a1b1" и весом 0.5.
    \item Создайте объект `s2` с именем "syn\_a2b2" и весом -0.3.
    \item Попытайтесь создать объект с именем "link\_x1y1" — поймайте исключение.
    \item Выведите `s1.sa1b1` и `s2.sa2b2`.
    \item Выведите `Synapse.connections`.
\end{enumerate}

Пример использования:
\begin{lstlisting}[language=Python]
s1 = Synapse("syn_a1b1", 0.5)
s2 = Synapse("syn_a2b2", -0.3)

try:
    s3 = Synapse("link_x1y1", 0.8)
except ValueError as e:
    print("Ошибка:", e)

print("s1.sa1b1:", s1.sa1b1)  # 0.5
print("s2.sa2b2:", s2.sa2b2)  # -0.3
print("Synapse.connections:", Synapse.connections)
\end{lstlisting}

\item Написать программу на Python, которая создает класс `Container` с использованием метода `\_\_new\_\_` для контроля именования. Имена должны начинаться с "container\_". Метод `\_\_init\_\_` должен быть пустым.
Инструкции:
\begin{enumerate}
    \item Создайте класс `Container`.
    \item Добавьте атрибут класса `depot` и инициализируйте его пустым словарем.
    \item Переопределите метод `\_\_new\_\_`, принимающий `cls`, `name`, `volume`.
    \item В `\_\_new\_\_`: если `name` не начинается с "container\_", выбросьте `ValueError("Контейнер должен иметь префикс 'container\_'")`.
    \item Извлеките ID контейнера: `container\_id = name[9:]`.
    \item Создайте экземпляр: `instance = super().\_\_new\_\_(cls)`.
    \item Добавьте ID в словарь: `cls.depot[container\_id] = instance`.
    \item Установите атрибут экземпляра: `setattr(instance, f"cn{container\_id}", volume)`.
    \item Верните `instance`.
    \item Переопределите метод `\_\_init\_\_` как пустой.
    \item Создайте объект `cn1` с именем "container\_20ft" и объемом 33.
    \item Создайте объект `cn2` с именем "container\_40ft" и объемом 67.
    \item Попытайтесь создать объект с именем "box\_small" — поймайте исключение.
    \item Выведите `cn1.cn20ft` и `cn2.cn40ft`.
    \item Выведите `Container.depot`.
\end{enumerate}
Пример использования:
\begin{lstlisting}[language=Python]
cn1 = Container("container_20ft", 33)
cn2 = Container("container_40ft", 67)
try:
    cn3 = Container("box_small", 1)
except ValueError as e:
    print("Ошибка:", e)
print("cn1.cn20ft:", cn1.cn20ft)  # 33
print("cn2.cn40ft:", cn2.cn40ft)  # 67
print("Container.depot:", Container.depot)
\end{lstlisting}


\item Написать программу на Python, которая создает класс `Module` с использованием метода `\_\_new\_\_` для контроля именования. Имена должны начинаться с "mod\_". Метод `\_\_init\_\_` должен быть пустым.
Инструкции:
\begin{enumerate}
    \item Создайте класс `Module`.
    \item Добавьте атрибут класса `system` и инициализируйте его пустым словарем.
    \item Переопределите метод `\_\_new\_\_`, принимающий `cls`, `name`, `version`.
    \item В `\_\_new\_\_`: если `name` не начинается с "mod\_", выбросьте `ValueError("Модуль должен иметь префикс 'mod\_'")`.
    \item Извлеките ID модуля: `mod\_id = name[4:]`.
    \item Создайте экземпляр: `instance = super().\_\_new\_\_(cls)`.
    \item Добавьте ID в словарь: `cls.system[mod\_id] = instance`.
    \item Установите атрибут экземпляра: `setattr(instance, f"md{mod\_id}", version)`.
    \item Верните `instance`.
    \item Переопределите метод `\_\_init\_\_` как пустой.
    \item Создайте объект `md1` с именем "mod\_auth" и версией "1.2.0".
    \item Создайте объект `md2` с именем "mod\_payment" и версией "3.0.1".
    \item Попытайтесь создать объект с именем "lib\_utils" — поймайте исключение.
    \item Выведите `md1.mdauth` и `md2.mdpayment`.
    \item Выведите `Module.system`.
\end{enumerate}
Пример использования:
\begin{lstlisting}[language=Python]
md1 = Module("mod_auth", "1.2.0")
md2 = Module("mod_payment", "3.0.1")
try:
    md3 = Module("lib_utils", "0.9.5")
except ValueError as e:
    print("Ошибка:", e)
print("md1.mdauth:", md1.mdauth)    # 1.2.0
print("md2.mdpayment:", md2.mdpayment) # 3.0.1
print("Module.system:", Module.system)
\end{lstlisting}
\end{enumerate}

\subsubsection{Задача 4}

\begin{enumerate}
\item[1] Написать программу на Python, которая создает класс \texttt{ShoppingCart} для представления корзины покупок. Класс должен содержать методы для добавления и удаления товаров, а также вычисления общего количества. Программа также должна создавать экземпляр класса \texttt{ShoppingCart}, добавлять товары в корзину, удалять товары из корзины и выводить информацию о корзине на экран.

\begin{itemize}
    \item Создайте класс \texttt{ShoppingCart} с методом \texttt{\_\_init\_\_}, который создает пустой список товаров.
    \item Создайте метод \texttt{add\_item}, который принимает название товара и количество в качестве аргументов и добавляет их в список товаров.
    \item Создайте метод \texttt{remove\_item}, который удаляет товар из списка товаров по его названию.
    \item Создайте метод \texttt{calculate\_total}, который вычисляет и возвращает общее количество всех товаров в корзине.
    \item Создайте экземпляр класса \texttt{ShoppingCart} и добавьте товары в корзину.
    \item Выведите информацию о текущих товарах в корзине на экран.
    \item Выведите общее количество всех товаров в корзине на экран.
    \item Удалите товар из корзины и выведите обновленную информацию о товарах в корзине на экран.
    \item Выведите общее количество всех товаров в корзине после удаления товара на экран.
\end{itemize}

\textbf{Пример использования:}

\begin{verbatim}
cart = ShoppingCart()
cart.add_item("Картофель", 100)
cart.add_item("Капуста", 200)
cart.add_item("Апельсин", 150)
print("Число товаров в корзине:")
for item in cart.items:
    print(item[0], "-", item[1])
total_qty = cart.calculate_total()
print("Общее количество:", total_qty)
cart.remove_item("Апельсин")
print("Обновление числа покупок в корзине после удаления апельсина:")
for item in cart.items:
    print(item[0], "-", item[1])
total_qty = cart.calculate_total()
print("Общее количество:", total_qty)
\end{verbatim}

\textbf{Вывод:}
\begin{verbatim}
Число товаров в корзине:
Картофель - 100
Капуста - 200
Апельсин - 150
Общее количество: 450
Обновление числа покупок в корзине после удаления апельсина:
Картофель - 100
Капуста - 200
Общее количество: 300
\end{verbatim}

\item[2] Написать программу на Python, которая создает класс \texttt{BookCollection} для представления коллекции книг. Класс должен содержать методы для добавления и удаления книг, а также подсчета общего количества страниц. Программа также должна создавать экземпляр класса \texttt{BookCollection}, добавлять книги в коллекцию, удалять книги из коллекции и выводить информацию о коллекции на экран.

\begin{itemize}
    \item Создайте класс \texttt{BookCollection} с методом \texttt{\_\_init\_\_}, который создает пустой список книг.
    \item Создайте метод \texttt{add\_book}, который принимает название книги и количество страниц в качестве аргументов и добавляет их в список книг.
    \item Создайте метод \texttt{remove\_book}, который удаляет книгу из списка по её названию.
    \item Создайте метод \texttt{total\_pages}, который вычисляет и возвращает общее количество страниц всех книг в коллекции.
    \item Создайте экземпляр класса \texttt{BookCollection} и добавьте книги в коллекцию.
    \item Выведите информацию о текущих книгах в коллекции на экран.
    \item Выведите общее количество страниц всех книг на экран.
    \item Удалите книгу из коллекции и выведите обновленную информацию о книгах на экран.
    \item Выведите общее количество страниц после удаления книги на экран.
\end{itemize}

\textbf{Пример использования:}

\begin{verbatim}
collection = BookCollection()
collection.add_book("Война и мир", 1225)
collection.add_book("Преступление и наказание", 671)
collection.add_book("Мастер и Маргарита", 480)
print("Книги в коллекции:")
for book in collection.books:
    print(book[0], "-", book[1], "стр.")
total = collection.total_pages()
print("Общее количество страниц:", total)
collection.remove_book("Преступление и наказание")
print("Книги после удаления 'Преступления и наказания':")
for book in collection.books:
    print(book[0], "-", book[1], "стр.")
total = collection.total_pages()
print("Общее количество страниц:", total)
\end{verbatim}

\textbf{Вывод:}
\begin{verbatim}
Книги в коллекции:
Война и мир - 1225 стр.
Преступление и наказание - 671 стр.
Мастер и Маргарита - 480 стр.
Общее количество страниц: 2376
Книги после удаления 'Преступления и наказания':
Война и мир - 1225 стр.
Мастер и Маргарита - 480 стр.
Общее количество страниц: 1705
\end{verbatim}

\item[3] Написать программу на Python, которая создает класс \texttt{Inventory} для представления складского запаса. Класс должен содержать методы для добавления и удаления предметов, а также вычисления общего количества единиц товара. Программа также должна создавать экземпляр класса \texttt{Inventory}, добавлять предметы на склад, удалять предметы со склада и выводить информацию о запасах на экран.

\begin{itemize}
    \item Создайте класс \texttt{Inventory} с методом \texttt{\_\_init\_\_}, который создает пустой список предметов.
    \item Создайте метод \texttt{add\_item}, который принимает название предмета и количество единиц в качестве аргументов и добавляет их в список.
    \item Создайте метод \texttt{remove\_item}, который удаляет предмет из списка по его названию.
    \item Создайте метод \texttt{total\_count}, который вычисляет и возвращает общее количество всех единиц товара на складе.
    \item Создайте экземпляр класса \texttt{Inventory} и добавьте предметы на склад.
    \item Выведите информацию о текущих предметах на складе на экран.
    \item Выведите общее количество единиц товара на экран.
    \item Удалите предмет со склада и выведите обновленную информацию о предметах на экран.
    \item Выведите общее количество единиц товара после удаления предмета на экран.
\end{itemize}

\textbf{Пример использования:}

\begin{verbatim}
inv = Inventory()
inv.add_item("Молотки", 50)
inv.add_item("Отвертки", 120)
inv.add_item("Гвозди", 1000)
print("Предметы на складе:")
for item in inv.items:
    print(item[0], "-", item[1])
total = inv.total_count()
print("Общее количество:", total)
inv.remove_item("Отвертки")
print("Предметы после удаления отверток:")
for item in inv.items:
    print(item[0], "-", item[1])
total = inv.total_count()
print("Общее количество:", total)
\end{verbatim}

\textbf{Вывод:}
\begin{verbatim}
Предметы на складе:
Молотки - 50
Отвертки - 120
Гвозди - 1000
Общее количество: 1170
Предметы после удаления отверток:
Молотки - 50
Гвозди - 1000
Общее количество: 1050
\end{verbatim}

\item[4] Написать программу на Python, которая создает класс \texttt{Playlist} для представления музыкального плейлиста. Класс должен содержать методы для добавления и удаления треков, а также подсчета общего времени воспроизведения. Программа также должна создавать экземпляр класса \texttt{Playlist}, добавлять треки в плейлист, удалять треки из плейлиста и выводить информацию о плейлисте на экран.

\begin{itemize}
    \item Создайте класс \texttt{Playlist} с методом \texttt{\_\_init\_\_}, который создает пустой список треков.
    \item Создайте метод \texttt{add\_track}, который принимает название трека и его длительность (в секундах) в качестве аргументов и добавляет их в список.
    \item Создайте метод \texttt{remove\_track}, который удаляет трек из списка по его названию.
    \item Создайте метод \texttt{total\_duration}, который вычисляет и возвращает общую длительность всех треков в плейлисте (в секундах).
    \item Создайте экземпляр класса \texttt{Playlist} и добавьте треки в плейлист.
    \item Выведите информацию о текущих треках в плейлисте на экран.
    \item Выведите общую длительность всех треков на экран.
    \item Удалите трек из плейлиста и выведите обновленную информацию о треках на экран.
    \item Выведите общую длительность после удаления трека на экран.
\end{itemize}

\textbf{Пример использования:}

\begin{verbatim}
pl = Playlist()
pl.add_track("Bohemian Rhapsody", 354)
pl.add_track("Imagine", 183)
pl.add_track("Smells Like Teen Spirit", 301)
print("Треки в плейлисте:")
for track in pl.tracks:
    print(track[0], "-", track[1], "сек.")
total = pl.total_duration()
print("Общая длительность:", total, "сек.")
pl.remove_track("Imagine")
print("Треки после удаления 'Imagine':")
for track in pl.tracks:
    print(track[0], "-", track[1], "сек.")
total = pl.total_duration()
print("Общая длительность:", total, "сек.")
\end{verbatim}

\textbf{Вывод:}
\begin{verbatim}
Треки в плейлисте:
Bohemian Rhapsody - 354 сек.
Imagine - 183 сек.
Smells Like Teen Spirit - 301 сек.
Общая длительность: 838 сек.
Треки после удаления 'Imagine':
Bohemian Rhapsody - 354 сек.
Smells Like Teen Spirit - 301 сек.
Общая длительность: 655 сек.
\end{verbatim}

\item[5] Написать программу на Python, которая создает класс \texttt{StudentGrades} для представления оценок студента. Класс должен содержать методы для добавления и удаления оценок, а также вычисления среднего балла. Программа также должна создавать экземпляр класса \texttt{StudentGrades}, добавлять оценки, удалять оценки и выводить информацию об успеваемости на экран.

\begin{itemize}
    \item Создайте класс \texttt{StudentGrades} с методом \texttt{\_\_init\_\_}, который создает пустой список оценок.
    \item Создайте метод \texttt{add\_grade}, который принимает название предмета и оценку в качестве аргументов и добавляет их в список.
    \item Создайте метод \texttt{remove\_grade}, который удаляет оценку по названию предмета.
    \item Создайте метод \texttt{average\_grade}, который вычисляет и возвращает средний балл по всем предметам.
    \item Создайте экземпляр класса \texttt{StudentGrades} и добавьте оценки по разным предметам.
    \item Выведите информацию о текущих оценках на экран.
    \item Выведите средний балл на экран.
    \item Удалите оценку по одному из предметов и выведите обновленную информацию.
    \item Выведите средний балл после удаления оценки на экран.
\end{itemize}

\textbf{Пример использования:}

\begin{verbatim}
grades = StudentGrades()
grades.add_grade("Математика", 5)
grades.add_grade("Физика", 4)
grades.add_grade("Информатика", 5)
print("Оценки студента:")
for subject, grade in grades.grades:
    print(subject, "-", grade)
avg = grades.average_grade()
print("Средний балл:", round(avg, 2))
grades.remove_grade("Физика")
print("Оценки после удаления Физики:")
for subject, grade in grades.grades:
    print(subject, "-", grade)
avg = grades.average_grade()
print("Средний балл:", round(avg, 2))
\end{verbatim}

\textbf{Вывод:}
\begin{verbatim}
Оценки студента:
Математика - 5
Физика - 4
Информатика - 5
Средний балл: 4.67
Оценки после удаления Физики:
Математика - 5
Информатика - 5
Средний балл: 5.0
\end{verbatim}

\item[6] Написать программу на Python, которая создает класс \texttt{TaskList} для представления списка задач. Класс должен содержать методы для добавления и удаления задач, а также подсчета общего количества задач. Программа также должна создавать экземпляр класса \texttt{TaskList}, добавлять задачи, удалять задачи и выводить информацию о списке задач на экран.

\begin{itemize}
    \item Создайте класс \texttt{TaskList} с методом \texttt{\_\_init\_\_}, который создает пустой список задач.
    \item Создайте метод \texttt{add\_task}, который принимает описание задачи и приоритет (целое число) в качестве аргументов и добавляет их в список.
    \item Создайте метод \texttt{remove\_task}, который удаляет задачу из списка по её описанию.
    \item Создайте метод \texttt{task\_count}, который возвращает общее количество задач в списке.
    \item Создайте экземпляр класса \texttt{TaskList} и добавьте несколько задач.
    \item Выведите информацию о текущих задачах на экран.
    \item Выведите общее количество задач на экран.
    \item Удалите одну из задач и выведите обновленный список задач.
    \item Выведите общее количество задач после удаления на экран.
\end{itemize}

\textbf{Пример использования:}

\begin{verbatim}
tasks = TaskList()
tasks.add_task("Написать отчет", 1)
tasks.add_task("Проверить почту", 3)
tasks.add_task("Подготовить презентацию", 2)
print("Список задач:")
for desc, priority in tasks.tasks:
    print(desc, "(приоритет", priority, ")")
count = tasks.task_count()
print("Всего задач:", count)
tasks.remove_task("Проверить почту")
print("Список задач после удаления 'Проверить почту':")
for desc, priority in tasks.tasks:
    print(desc, "(приоритет", priority, ")")
count = tasks.task_count()
print("Всего задач:", count)
\end{verbatim}

\textbf{Вывод:}
\begin{verbatim}
Список задач:
Написать отчет (приоритет 1 )
Проверить почту (приоритет 3 )
Подготовить презентацию (приоритет 2 )
Всего задач: 3
Список задач после удаления 'Проверить почту':
Написать отчет (приоритет 1 )
Подготовить презентацию (приоритет 2 )
Всего задач: 2
\end{verbatim}

\item[7] Написать программу на Python, которая создает класс \texttt{BankAccount} для представления банковского счета. Класс должен содержать методы для добавления и снятия средств, а также получения текущего баланса. Программа также должна создавать экземпляр класса \texttt{BankAccount}, выполнять операции пополнения и снятия, и выводить информацию о балансе на экран.

\begin{itemize}
    \item Создайте класс \texttt{BankAccount} с методом \texttt{\_\_init\_\_}, который инициализирует баланс нулём.
    \item Создайте метод \texttt{deposit}, который принимает сумму и увеличивает баланс на неё.
    \item Создайте метод \texttt{withdraw}, который принимает сумму и уменьшает баланс на неё (если достаточно средств).
    \item Создайте метод \texttt{get\_balance}, который возвращает текущий баланс.
    \item Создайте экземпляр класса \texttt{BankAccount}.
    \item Выполните несколько операций пополнения счета.
    \item Выведите текущий баланс на экран.
    \item Выполните операцию снятия средств и выведите обновленный баланс.
    \item Выведите окончательный баланс на экран.
\end{itemize}

\textbf{Пример использования:}

\begin{verbatim}
account = BankAccount()
account.deposit(1000)
account.deposit(500)
print("Баланс после пополнений:", account.get_balance())
account.withdraw(300)
print("Баланс после снятия 300:", account.get_balance())
account.withdraw(200)
print("Окончательный баланс:", account.get_balance())
\end{verbatim}

\textbf{Вывод:}
\begin{verbatim}
Баланс после пополнений: 1500
Баланс после снятия 300: 1200
Окончательный баланс: 1000
\end{verbatim}

\item[8] Написать программу на Python, которая создает класс \texttt{Library} для представления библиотеки. Класс должен содержать методы для добавления и удаления книг, а также подсчета общего количества книг. Программа также должна создавать экземпляр класса \texttt{Library}, добавлять книги, удалять книги и выводить информацию о фонде на экран.

\begin{itemize}
    \item Создайте класс \texttt{Library} с методом \texttt{\_\_init\_\_}, который создает пустой список книг.
    \item Создайте метод \texttt{add\_book}, который принимает название книги и автора в качестве аргументов и добавляет их в список.
    \item Создайте метод \texttt{remove\_book}, который удаляет книгу из списка по её названию.
    \item Создайте метод \texttt{book\_count}, который возвращает общее количество книг в библиотеке.
    \item Создайте экземпляр класса \texttt{Library} и добавьте несколько книг.
    \item Выведите информацию о текущих книгах на экран.
    \item Выведите общее количество книг на экран.
    \item Удалите одну из книг и выведите обновленный список.
    \item Выведите общее количество книг после удаления на экран.
\end{itemize}

\textbf{Пример использования:}

\begin{verbatim}
lib = Library()
lib.add_book("1984", "Джордж Оруэлл")
lib.add_book("Гарри Поттер", "Дж.К. Роулинг")
lib.add_book("Гордость и предубеждение", "Джейн Остин")
print("Книги в библиотеке:")
for title, author in lib.books:
    print(title, "—", author)
count = lib.book_count()
print("Всего книг:", count)
lib.remove_book("Гарри Поттер")
print("Книги после удаления 'Гарри Поттера':")
for title, author in lib.books:
    print(title, "—", author)
count = lib.book_count()
print("Всего книг:", count)
\end{verbatim}

\textbf{Вывод:}
\begin{verbatim}
Книги в библиотеке:
1984 — Джордж Оруэлл
Гарри Поттер — Дж.К. Роулинг
Гордость и предубеждение — Джейн Остин
Всего книг: 3
Книги после удаления 'Гарри Поттера':
1984 — Джордж Оруэлл
Гордость и предубеждение — Джейн Остин
Всего книг: 2
\end{verbatim}

\item[9] Написать программу на Python, которая создает класс \texttt{GroceryList} для представления списка покупок. Класс должен содержать методы для добавления и удаления продуктов, а также подсчета общего количества позиций. Программа также должна создавать экземпляр класса \texttt{GroceryList}, добавлять продукты, удалять продукты и выводить информацию о списке на экран.

\begin{itemize}
    \item Создайте класс \texttt{GroceryList} с методом \texttt{\_\_init\_\_}, который создает пустой список продуктов.
    \item Создайте метод \texttt{add\_product}, который принимает название продукта и количество в качестве аргументов и добавляет их в список.
    \item Создайте метод \texttt{remove\_product}, который удаляет продукт из списка по его названию.
    \item Создайте метод \texttt{total\_items}, который возвращает общее количество различных продуктов в списке.
    \item Создайте экземпляр класса \texttt{GroceryList} и добавьте несколько продуктов.
    \item Выведите информацию о текущих продуктах на экран.
    \item Выведите общее количество позиций на экран.
    \item Удалите один из продуктов и выведите обновленный список.
    \item Выведите общее количество позиций после удаления на экран.
\end{itemize}

\textbf{Пример использования:}

\begin{verbatim}
grocery = GroceryList()
grocery.add_product("Молоко", 2)
grocery.add_product("Хлеб", 1)
grocery.add_product("Яйца", 12)
print("Список покупок:")
for name, qty in grocery.products:
    print(name, "-", qty)
count = grocery.total_items()
print("Всего позиций:", count)
grocery.remove_product("Хлеб")
print("Список после удаления хлеба:")
for name, qty in grocery.products:
    print(name, "-", qty)
count = grocery.total_items()
print("Всего позиций:", count)
\end{verbatim}

\textbf{Вывод:}
\begin{verbatim}
Список покупок:
Молоко - 2
Хлеб - 1
Яйца - 12
Всего позиций: 3
Список после удаления хлеба:
Молоко - 2
Яйца - 12
Всего позиций: 2
\end{verbatim}

\item[10] Написать программу на Python, которая создает класс \texttt{ContactList} для представления списка контактов. Класс должен содержать методы для добавления и удаления контактов, а также подсчета общего количества контактов. Программа также должна создавать экземпляр класса \texttt{ContactList}, добавлять контакты, удалять контакты и выводить информацию о списке на экран.

\begin{itemize}
    \item Создайте класс \texttt{ContactList} с методом \texttt{\_\_init\_\_}, который создает пустой список контактов.
    \item Создайте метод \texttt{add\_contact}, который принимает имя и номер телефона в качестве аргументов и добавляет их в список.
    \item Создайте метод \texttt{remove\_contact}, который удаляет контакт из списка по имени.
    \item Создайте метод \texttt{contact\_count}, который возвращает общее количество контактов в списке.
    \item Создайте экземпляр класса \texttt{ContactList} и добавьте несколько контактов.
    \item Выведите информацию о текущих контактах на экран.
    \item Выведите общее количество контактов на экран.
    \item Удалите один из контактов и выведите обновленный список.
    \item Выведите общее количество контактов после удаления на экран.
\end{itemize}

\textbf{Пример использования:}

\begin{verbatim}
contacts = ContactList()
contacts.add_contact("Анна", "+79001234567")
contacts.add_contact("Борис", "+79007654321")
contacts.add_contact("Виктория", "+79001112233")
print("Контакты:")
for name, phone in contacts.contacts:
    print(name, "-", phone)
count = contacts.contact_count()
print("Всего контактов:", count)
contacts.remove_contact("Борис")
print("Контакты после удаления Бориса:")
for name, phone in contacts.contacts:
    print(name, "-", phone)
count = contacts.contact_count()
print("Всего контактов:", count)
\end{verbatim}

\textbf{Вывод:}
\begin{verbatim}
Контакты:
Анна - +79001234567
Борис - +79007654321
Виктория - +79001112233
Всего контактов: 3
Контакты после удаления Бориса:
Анна - +79001234567
Виктория - +79001112233
Всего контактов: 2
\end{verbatim}

\item[11] Написать программу на Python, которая создает класс \texttt{MovieCollection} для представления коллекции фильмов. Класс должен содержать методы для добавления и удаления фильмов, а также подсчета общего количества фильмов. Программа также должна создавать экземпляр класса \texttt{MovieCollection}, добавлять фильмы, удалять фильмы и выводить информацию о коллекции на экран.

\begin{itemize}
    \item Создайте класс \texttt{MovieCollection} с методом \texttt{\_\_init\_\_}, который создает пустой список фильмов.
    \item Создайте метод \texttt{add\_movie}, который принимает название фильма и год выпуска в качестве аргументов и добавляет их в список.
    \item Создайте метод \texttt{remove\_movie}, который удаляет фильм из списка по его названию.
    \item Создайте метод \texttt{movie\_count}, который возвращает общее количество фильмов в коллекции.
    \item Создайте экземпляр класса \texttt{MovieCollection} и добавьте несколько фильмов.
    \item Выведите информацию о текущих фильмах на экран.
    \item Выведите общее количество фильмов на экран.
    \item Удалите один из фильмов и выведите обновленный список.
    \item Выведите общее количество фильмов после удаления на экран.
\end{itemize}

\textbf{Пример использования:}

\begin{verbatim}
movies = MovieCollection()
movies.add_movie("Крёстный отец", 1972)
movies.add_movie("Побег из Шоушенка", 1994)
movies.add_movie("Тёмный рыцарь", 2008)
print("Фильмы в коллекции:")
for title, year in movies.movies:
    print(title, "(", year, ")")
count = movies.movie_count()
print("Всего фильмов:", count)
movies.remove_movie("Побег из Шоушенка")
print("Фильмы после удаления 'Побега из Шоушенка':")
for title, year in movies.movies:
    print(title, "(", year, ")")
count = movies.movie_count()
print("Всего фильмов:", count)
\end{verbatim}

\textbf{Вывод:}
\begin{verbatim}
Фильмы в коллекции:
Крёстный отец ( 1972 )
Побег из Шоушенка ( 1994 )
Тёмный рыцарь ( 2008 )
Всего фильмов: 3
Фильмы после удаления 'Побега из Шоушенка':
Крёстный отец ( 1972 )
Тёмный рыцарь ( 2008 )
Всего фильмов: 2
\end{verbatim}

\item[12] Написать программу на Python, которая создает класс \texttt{RecipeBook} для представления кулинарной книги. Класс должен содержать методы для добавления и удаления рецептов, а также подсчета общего количества рецептов. Программа также должна создавать экземпляр класса \texttt{RecipeBook}, добавлять рецепты, удалять рецепты и выводить информацию о книге на экран.

\begin{itemize}
    \item Создайте класс \texttt{RecipeBook} с методом \texttt{\_\_init\_\_}, который создает пустой список рецептов.
    \item Создайте метод \texttt{add\_recipe}, который принимает название блюда и время приготовления (в минутах) в качестве аргументов и добавляет их в список.
    \item Создайте метод \texttt{remove\_recipe}, который удаляет рецепт из списка по названию блюда.
    \item Создайте метод \texttt{recipe\_count}, который возвращает общее количество рецептов в книге.
    \item Создайте экземпляр класса \texttt{RecipeBook} и добавьте несколько рецептов.
    \item Выведите информацию о текущих рецептах на экран.
    \item Выведите общее количество рецептов на экран.
    \item Удалите один из рецептов и выведите обновленный список.
    \item Выведите общее количество рецептов после удаления на экран.
\end{itemize}

\textbf{Пример использования:}

\begin{verbatim}
recipes = RecipeBook()
recipes.add_recipe("Борщ", 60)
recipes.add_recipe("Омлет", 10)
recipes.add_recipe("Паста", 20)
print("Рецепты в книге:")
for dish, time in recipes.recipes:
    print(dish, "-", time, "мин.")
count = recipes.recipe_count()
print("Всего рецептов:", count)
recipes.remove_recipe("Омлет")
print("Рецепты после удаления омлета:")
for dish, time in recipes.recipes:
    print(dish, "-", time, "мин.")
count = recipes.recipe_count()
print("Всего рецептов:", count)
\end{verbatim}

\textbf{Вывод:}
\begin{verbatim}
Рецепты в книге:
Борщ - 60 мин.
Омлет - 10 мин.
Паста - 20 мин.
Всего рецептов: 3
Рецепты после удаления омлета:
Борщ - 60 мин.
Паста - 20 мин.
Всего рецептов: 2
\end{verbatim}

\item[13] Написать программу на Python, которая создает класс \texttt{CarGarage} для представления автосервиса. Класс должен содержать методы для добавления и удаления автомобилей, а также подсчета общего количества машин. Программа также должна создавать экземпляр класса \texttt{CarGarage}, добавлять автомобили, удалять автомобили и выводить информацию о гараже на экран.

\begin{itemize}
    \item Создайте класс \texttt{CarGarage} с методом \texttt{\_\_init\_\_}, который создает пустой список автомобилей.
    \item Создайте метод \texttt{add\_car}, который принимает марку и модель автомобиля в качестве аргументов и добавляет их в список.
    \item Создайте метод \texttt{remove\_car}, который удаляет автомобиль из списка по марке.
    \item Создайте метод \texttt{car\_count}, который возвращает общее количество автомобилей в гараже.
    \item Создайте экземпляр класса \texttt{CarGarage} и добавьте несколько автомобилей.
    \item Выведите информацию о текущих автомобилях на экран.
    \item Выведите общее количество машин на экран.
    \item Удалите один из автомобилей и выведите обновленный список.
    \item Выведите общее количество машин после удаления на экран.
\end{itemize}

\textbf{Пример использования:}

\begin{verbatim}
garage = CarGarage()
garage.add_car("Toyota", "Camry")
garage.add_car("BMW", "X5")
garage.add_car("Ford", "Focus")
print("Автомобили в гараже:")
for brand, model in garage.cars:
    print(brand, model)
count = garage.car_count()
print("Всего автомобилей:", count)
garage.remove_car("BMW")
print("Автомобили после удаления BMW:")
for brand, model in garage.cars:
    print(brand, model)
count = garage.car_count()
print("Всего автомобилей:", count)
\end{verbatim}

\textbf{Вывод:}
\begin{verbatim}
Автомобили в гараже:
Toyota Camry
BMW X5
Ford Focus
Всего автомобилей: 3
Автомобили после удаления BMW:
Toyota Camry
Ford Focus
Всего автомобилей: 2
\end{verbatim}

\item[14] Написать программу на Python, которая создает класс \texttt{PetStore} для представления зоомагазина. Класс должен содержать методы для добавления и удаления животных, а также подсчета общего количества питомцев. Программа также должна создавать экземпляр класса \texttt{PetStore}, добавлять животных, удалять животных и выводить информацию о магазине на экран.

\begin{itemize}
    \item Создайте класс \texttt{PetStore} с методом \texttt{\_\_init\_\_}, который создает пустой список животных.
    \item Создайте метод \texttt{add\_pet}, который принимает вид животного и количество в качестве аргументов и добавляет их в список.
    \item Создайте метод \texttt{remove\_pet}, который удаляет животное из списка по виду.
    \item Создайте метод \texttt{total\_pets}, который возвращает общее количество всех питомцев в магазине.
    \item Создайте экземпляр класса \texttt{PetStore} и добавьте несколько видов животных.
    \item Выведите информацию о текущих животных на экран.
    \item Выведите общее количество питомцев на экран.
    \item Удалите один из видов животных и выведите обновленный список.
    \item Выведите общее количество питомцев после удаления на экран.
\end{itemize}

\textbf{Пример использования:}

\begin{verbatim}
store = PetStore()
store.add_pet("Кошки", 5)
store.add_pet("Собаки", 3)
store.add_pet("Попугаи", 10)
print("Животные в магазине:")
for species, count in store.pets:
    print(species, "-", count)
total = store.total_pets()
print("Всего питомцев:", total)
store.remove_pet("Собаки")
print("Животные после удаления собак:")
for species, count in store.pets:
    print(species, "-", count)
total = store.total_pets()
print("Всего питомцев:", total)
\end{verbatim}

\textbf{Вывод:}
\begin{verbatim}
Животные в магазине:
Кошки - 5
Собаки - 3
Попугаи - 10
Всего питомцев: 18
Животные после удаления собак:
Кошки - 5
Попугаи - 10
Всего питомцев: 15
\end{verbatim}

\item[15] Написать программу на Python, которая создает класс \texttt{CourseRoster} для представления списка студентов на курсе. Класс должен содержать методы для добавления и удаления студентов, а также подсчета общего количества учащихся. Программа также должна создавать экземпляр класса \texttt{CourseRoster}, добавлять студентов, удалять студентов и выводить информацию о курсе на экран.

\begin{itemize}
    \item Создайте класс \texttt{CourseRoster} с методом \texttt{\_\_init\_\_}, который создает пустой список студентов.
    \item Создайте метод \texttt{enroll\_student}, который принимает имя студента и его ID в качестве аргументов и добавляет их в список.
    \item Создайте метод \texttt{drop\_student}, который удаляет студента из списка по имени.
    \item Создайте метод \texttt{student\_count}, который возвращает общее количество студентов на курсе.
    \item Создайте экземпляр класса \texttt{CourseRoster} и добавьте несколько студентов.
    \item Выведите информацию о текущих студентах на экран.
    \item Выведите общее количество студентов на экран.
    \item Удалите одного из студентов и выведите обновленный список.
    \item Выведите общее количество студентов после удаления на экран.
\end{itemize}

\textbf{Пример использования:}

\begin{verbatim}
roster = CourseRoster()
roster.enroll_student("Иван", 101)
roster.enroll_student("Мария", 102)
roster.enroll_student("Алексей", 103)
print("Студенты на курсе:")
for name, sid in roster.students:
    print(name, "(ID:", sid, ")")
count = roster.student_count()
print("Всего студентов:", count)
roster.drop_student("Мария")
print("Студенты после отчисления Марии:")
for name, sid in roster.students:
    print(name, "(ID:", sid, ")")
count = roster.student_count()
print("Всего студентов:", count)
\end{verbatim}

\textbf{Вывод:}
\begin{verbatim}
Студенты на курсе:
Иван (ID: 101 )
Мария (ID: 102 )
Алексей (ID: 103 )
Всего студентов: 3
Студенты после отчисления Марии:
Иван (ID: 101 )
Алексей (ID: 103 )
Всего студентов: 2
\end{verbatim}

\item[16] Написать программу на Python, которая создает класс \texttt{TravelItinerary} для представления туристического маршрута. Класс должен содержать методы для добавления и удаления мест, а также подсчета общего количества пунктов назначения. Программа также должна создавать экземпляр класса \texttt{TravelItinerary}, добавлять места, удалять места и выводить информацию о маршруте на экран.

\begin{itemize}
    \item Создайте класс \texttt{TravelItinerary} с методом \texttt{\_\_init\_\_}, который создает пустой список мест.
    \item Создайте метод \texttt{add\_destination}, который принимает название города и количество дней пребывания в качестве аргументов и добавляет их в список.
    \item Создайте метод \texttt{remove\_destination}, который удаляет место из списка по названию города.
    \item Создайте метод \texttt{destination\_count}, который возвращает общее количество пунктов назначения в маршруте.
    \item Создайте экземпляр класса \texttt{TravelItinerary} и добавьте несколько городов.
    \item Выведите информацию о текущих местах на экран.
    \item Выведите общее количество пунктов назначения на экран.
    \item Удалите один из городов и выведите обновленный маршрут.
    \item Выведите общее количество пунктов назначения после удаления на экран.
\end{itemize}

\textbf{Пример использования:}

\begin{verbatim}
itinerary = TravelItinerary()
itinerary.add_destination("Париж", 4)
itinerary.add_destination("Рим", 3)
itinerary.add_destination("Барселона", 5)
print("Маршрут путешествия:")
for city, days in itinerary.destinations:
    print(city, "-", days, "дней")
count = itinerary.destination_count()
print("Всего пунктов:", count)
itinerary.remove_destination("Рим")
print("Маршрут после удаления Рима:")
for city, days in itinerary.destinations:
    print(city, "-", days, "дней")
count = itinerary.destination_count()
print("Всего пунктов:", count)
\end{verbatim}

\textbf{Вывод:}
\begin{verbatim}
Маршрут путешествия:
Париж - 4 дней
Рим - 3 дней
Барселона - 5 дней
Всего пунктов: 3
Маршрут после удаления Рима:
Париж - 4 дней
Барселона - 5 дней
Всего пунктов: 2
\end{verbatim}

\item[17] Написать программу на Python, которая создает класс \texttt{FitnessTracker} для представления тренировочного плана. Класс должен содержать методы для добавления и удаления упражнений, а также подсчета общего количества подходов. Программа также должна создавать экземпляр класса \texttt{FitnessTracker}, добавлять упражнения, удалять упражнения и выводить информацию о плане на экран.

\begin{itemize}
    \item Создайте класс \texttt{FitnessTracker} с методом \texttt{\_\_init\_\_}, который создает пустой список упражнений.
    \item Создайте метод \texttt{add\_exercise}, который принимает название упражнения и количество подходов в качестве аргументов и добавляет их в список.
    \item Создайте метод \texttt{remove\_exercise}, который удаляет упражнение из списка по его названию.
    \item Создайте метод \texttt{total\_sets}, который возвращает общее количество подходов по всем упражнениям.
    \item Создайте экземпляр класса \texttt{FitnessTracker} и добавьте несколько упражнений.
    \item Выведите информацию о текущих упражнениях на экран.
    \item Выведите общее количество подходов на экран.
    \item Удалите одно из упражнений и выведите обновленный план.
    \item Выведите общее количество подходов после удаления на экран.
\end{itemize}

\textbf{Пример использования:}

\begin{verbatim}
tracker = FitnessTracker()
tracker.add_exercise("Приседания", 4)
tracker.add_exercise("Отжимания", 3)
tracker.add_exercise("Подтягивания", 5)
print("Тренировочный план:")
for ex, sets in tracker.exercises:
    print(ex, "-", sets, "подходов")
total = tracker.total_sets()
print("Всего подходов:", total)
tracker.remove_exercise("Отжимания")
print("План после удаления отжиманий:")
for ex, sets in tracker.exercises:
    print(ex, "-", sets, "подходов")
total = tracker.total_sets()
print("Всего подходов:", total)
\end{verbatim}

\textbf{Вывод:}
\begin{verbatim}
Тренировочный план:
Приседания - 4 подходов
Отжимания - 3 подходов
Подтягивания - 5 подходов
Всего подходов: 12
План после удаления отжиманий:
Приседания - 4 подходов
Подтягивания - 5 подходов
Всего подходов: 9
\end{verbatim}

\item[18] Написать программу на Python, которая создает класс \texttt{ExpenseTracker} для представления расходов. Класс должен содержать методы для добавления и удаления трат, а также подсчета общей суммы расходов. Программа также должна создавать экземпляр класса \texttt{ExpenseTracker}, добавлять расходы, удалять расходы и выводить информацию о тратах на экран.

\begin{itemize}
    \item Создайте класс \texttt{ExpenseTracker} с методом \texttt{\_\_init\_\_}, который создает пустой список расходов.
    \item Создайте метод \texttt{add\_expense}, который принимает категорию и сумму в качестве аргументов и добавляет их в список.
    \item Создайте метод \texttt{remove\_expense}, который удаляет расход из списка по категории.
    \item Создайте метод \texttt{total\_expenses}, который возвращает общую сумму всех расходов.
    \item Создайте экземпляр класса \texttt{ExpenseTracker} и добавьте несколько расходов.
    \item Выведите информацию о текущих тратах на экран.
    \item Выведите общую сумму расходов на экран.
    \item Удалите один из расходов и выведите обновленный список.
    \item Выведите общую сумму расходов после удаления на экран.
\end{itemize}

\textbf{Пример использования:}

\begin{verbatim}
expenses = ExpenseTracker()
expenses.add_expense("Продукты", 2500)
expenses.add_expense("Транспорт", 800)
expenses.add_expense("Развлечения", 1200)
print("Расходы:")
for cat, amount in expenses.expenses:
    print(cat, "-", amount, "руб.")
total = expenses.total_expenses()
print("Общая сумма расходов:", total, "руб.")
expenses.remove_expense("Транспорт")
print("Расходы после удаления транспорта:")
for cat, amount in expenses.expenses:
    print(cat, "-", amount, "руб.")
total = expenses.total_expenses()
print("Общая сумма расходов:", total, "руб.")
\end{verbatim}

\textbf{Вывод:}
\begin{verbatim}
Расходы:
Продукты - 2500 руб.
Транспорт - 800 руб.
Развлечения - 1200 руб.
Общая сумма расходов: 4500 руб.
Расходы после удаления транспорта:
Продукты - 2500 руб.
Развлечения - 1200 руб.
Общая сумма расходов: 3700 руб.
\end{verbatim}

\item[19] Написать программу на Python, которая создает класс \texttt{ProjectTasks} для представления задач проекта. Класс должен содержать методы для добавления и удаления задач, а также подсчета общего количества задач. Программа также должна создавать экземпляр класса \texttt{ProjectTasks}, добавлять задачи, удалять задачи и выводить информацию о проекте на экран.

\begin{itemize}
    \item Создайте класс \texttt{ProjectTasks} с методом \texttt{\_\_init\_\_}, который создает пустой список задач.
    \item Создайте метод \texttt{add\_task}, который принимает описание задачи и срок выполнения (в днях) в качестве аргументов и добавляет их в список.
    \item Создайте метод \texttt{remove\_task}, который удаляет задачу из списка по её описанию.
    \item Создайте метод \texttt{task\_count}, который возвращает общее количество задач в проекте.
    \item Создайте экземпляр класса \texttt{ProjectTasks} и добавьте несколько задач.
    \item Выведите информацию о текущих задачах на экран.
    \item Выведите общее количество задач на экран.
    \item Удалите одну из задач и выведите обновленный список.
    \item Выведите общее количество задач после удаления на экран.
\end{itemize}

\textbf{Пример использования:}

\begin{verbatim}
project = ProjectTasks()
project.add_task("Разработка интерфейса", 5)
project.add_task("Тестирование", 3)
project.add_task("Документация", 2)
print("Задачи проекта:")
for desc, days in project.tasks:
    print(desc, "-", days, "дней")
count = project.task_count()
print("Всего задач:", count)
project.remove_task("Тестирование")
print("Задачи после удаления тестирования:")
for desc, days in project.tasks:
    print(desc, "-", days, "дней")
count = project.task_count()
print("Всего задач:", count)
\end{verbatim}

\textbf{Вывод:}
\begin{verbatim}
Задачи проекта:
Разработка интерфейса - 5 дней
Тестирование - 3 дней
Документация - 2 дней
Всего задач: 3
Задачи после удаления тестирования:
Разработка интерфейса - 5 дней
Документация - 2 дней
Всего задач: 2
\end{verbatim}

\item[20] Написать программу на Python, которая создает класс \texttt{EventSchedule} для представления расписания мероприятий. Класс должен содержать методы для добавления и удаления событий, а также подсчета общего количества мероприятий. Программа также должна создавать экземпляр класса \texttt{EventSchedule}, добавлять события, удалять события и выводить информацию о расписании на экран.

\begin{itemize}
    \item Создайте класс \texttt{EventSchedule} с методом \texttt{\_\_init\_\_}, который создает пустой список мероприятий.
    \item Создайте метод \texttt{add\_event}, который принимает название мероприятия и дату проведения в качестве аргументов и добавляет их в список.
    \item Создайте метод \texttt{remove\_event}, который удаляет мероприятие из списка по его названию.
    \item Создайте метод \texttt{event\_count}, который возвращает общее количество мероприятий в расписании.
    \item Создайте экземпляр класса \texttt{EventSchedule} и добавьте несколько мероприятий.
    \item Выведите информацию о текущих мероприятиях на экран.
    \item Выведите общее количество мероприятий на экран.
    \item Удалите одно из мероприятий и выведите обновленное расписание.
    \item Выведите общее количество мероприятий после удаления на экран.
\end{itemize}

\textbf{Пример использования:}

\begin{verbatim}
schedule = EventSchedule()
schedule.add_event("Конференция", "15.05.2024")
schedule.add_event("Воркшоп", "20.05.2024")
schedule.add_event("Выставка", "25.05.2024")
print("Расписание мероприятий:")
for name, date in schedule.events:
    print(name, "-", date)
count = schedule.event_count()
print("Всего мероприятий:", count)
schedule.remove_event("Воркшоп")
print("Расписание после удаления воркшопа:")
for name, date in schedule.events:
    print(name, "-", date)
count = schedule.event_count()
print("Всего мероприятий:", count)
\end{verbatim}

\textbf{Вывод:}
\begin{verbatim}
Расписание мероприятий:
Конференция - 15.05.2024
Воркшоп - 20.05.2024
Выставка - 25.05.2024
Всего мероприятий: 3
Расписание после удаления воркшопа:
Конференция - 15.05.2024
Выставка - 25.05.2024
Всего мероприятий: 2
\end{verbatim}

\item[21] Написать программу на Python, которая создает класс \texttt{GardenPlanner} для представления садового участка. Класс должен содержать методы для добавления и удаления растений, а также подсчета общего количества видов растений. Программа также должна создавать экземпляр класса \texttt{GardenPlanner}, добавлять растения, удалять растения и выводить информацию о саде на экран.

\begin{itemize}
    \item Создайте класс \texttt{GardenPlanner} с методом \texttt{\_\_init\_\_}, который создает пустой список растений.
    \item Создайте метод \texttt{add\_plant}, который принимает название растения и количество экземпляров в качестве аргументов и добавляет их в список.
    \item Создайте метод \texttt{remove\_plant}, который удаляет растение из списка по его названию.
    \item Создайте метод \texttt{plant\_count}, который возвращает общее количество различных видов растений в саду.
    \item Создайте экземпляр класса \texttt{GardenPlanner} и добавьте несколько растений.
    \item Выведите информацию о текущих растениях на экран.
    \item Выведите общее количество видов растений на экран.
    \item Удалите одно из растений и выведите обновленный список.
    \item Выведите общее количество видов растений после удаления на экран.
\end{itemize}

\textbf{Пример использования:}

\begin{verbatim}
garden = GardenPlanner()
garden.add_plant("Розы", 10)
garden.add_plant("Тюльпаны", 20)
garden.add_plant("Лаванда", 5)
print("Растения в саду:")
for name, qty in garden.plants:
    print(name, "-", qty)
count = garden.plant_count()
print("Всего видов растений:", count)
garden.remove_plant("Тюльпаны")
print("Растения после удаления тюльпанов:")
for name, qty in garden.plants:
    print(name, "-", qty)
count = garden.plant_count()
print("Всего видов растений:", count)
\end{verbatim}

\textbf{Вывод:}
\begin{verbatim}
Растения в саду:
Розы - 10
Тюльпаны - 20
Лаванда - 5
Всего видов растений: 3
Растения после удаления тюльпанов:
Розы - 10
Лаванда - 5
Всего видов растений: 2
\end{verbatim}

\item[22] Написать программу на Python, которая создает класс \texttt{Warehouse} для представления склада товаров. Класс должен содержать методы для добавления и удаления товаров, а также подсчета общего количества типов товаров. Программа также должна создавать экземпляр класса \texttt{Warehouse}, добавлять товары, удалять товары и выводить информацию о складе на экран.

\begin{itemize}
    \item Создайте класс \texttt{Warehouse} с методом \texttt{\_\_init\_\_}, который создает пустой список товаров.
    \item Создайте метод \texttt{add\_product}, который принимает название товара и количество единиц в качестве аргументов и добавляет их в список.
    \item Создайте метод \texttt{remove\_product}, который удаляет товар из списка по его названию.
    \item Создайте метод \texttt{product\_types}, который возвращает общее количество различных типов товаров на складе.
    \item Создайте экземпляр класса \texttt{Warehouse} и добавьте несколько товаров.
    \item Выведите информацию о текущих товарах на экран.
    \item Выведите общее количество типов товаров на экран.
    \item Удалите один из товаров и выведите обновленный список.
    \item Выведите общее количество типов товаров после удаления на экран.
\end{itemize}

\textbf{Пример использования:}

\begin{verbatim}
warehouse = Warehouse()
warehouse.add_product("Стулья", 50)
warehouse.add_product("Столы", 20)
warehouse.add_product("Лампы", 100)
print("Товары на складе:")
for name, qty in warehouse.products:
    print(name, "-", qty)
types = warehouse.product_types()
print("Всего типов товаров:", types)
warehouse.remove_product("Столы")
print("Товары после удаления столов:")
for name, qty in warehouse.products:
    print(name, "-", qty)
types = warehouse.product_types()
print("Всего типов товаров:", types)
\end{verbatim}

\textbf{Вывод:}
\begin{verbatim}
Товары на складе:
Стулья - 50
Столы - 20
Лампы - 100
Всего типов товаров: 3
Товары после удаления столов:
Стулья - 50
Лампы - 100
Всего типов товаров: 2
\end{verbatim}

\item[23] Написать программу на Python, которая создает класс \texttt{GameInventory} для представления инвентаря игрока. Класс должен содержать методы для добавления и удаления предметов, а также подсчета общего количества типов предметов. Программа также должна создавать экземпляр класса \texttt{GameInventory}, добавлять предметы, удалять предметы и выводить информацию об инвентаре на экран.

\begin{itemize}
    \item Создайте класс \texttt{GameInventory} с методом \texttt{\_\_init\_\_}, который создает пустой список предметов.
    \item Создайте метод \texttt{add\_item}, который принимает название предмета и количество в качестве аргументов и добавляет их в список.
    \item Создайте метод \texttt{remove\_item}, который удаляет предмет из списка по его названию.
    \item Создайте метод \texttt{item\_types}, который возвращает общее количество различных типов предметов в инвентаре.
    \item Создайте экземпляр класса \texttt{GameInventory} и добавьте несколько предметов.
    \item Выведите информацию о текущих предметах на экран.
    \item Выведите общее количество типов предметов на экран.
    \item Удалите один из предметов и выведите обновленный инвентарь.
    \item Выведите общее количество типов предметов после удаления на экран.
\end{itemize}

\textbf{Пример использования:}

\begin{verbatim}
inventory = GameInventory()
inventory.add_item("Меч", 1)
inventory.add_item("Зелье", 5)
inventory.add_item("Щит", 1)
print("Инвентарь игрока:")
for name, qty in inventory.items:
    print(name, "-", qty)
types = inventory.item_types()
print("Всего типов предметов:", types)
inventory.remove_item("Зелье")
print("Инвентарь после удаления зелий:")
for name, qty in inventory.items:
    print(name, "-", qty)
types = inventory.item_types()
print("Всего типов предметов:", types)
\end{verbatim}

\textbf{Вывод:}
\begin{verbatim}
Инвентарь игрока:
Меч - 1
Зелье - 5
Щит - 1
Всего типов предметов: 3
Инвентарь после удаления зелий:
Меч - 1
Щит - 1
Всего типов предметов: 2
\end{verbatim}

\item[24] Написать программу на Python, которая создает класс \texttt{MusicAlbum} для представления музыкального альбома. Класс должен содержать методы для добавления и удаления треков, а также подсчета общего количества треков. Программа также должна создавать экземпляр класса \texttt{MusicAlbum}, добавлять треки, удалять треки и выводить информацию об альбоме на экран.

\begin{itemize}
    \item Создайте класс \texttt{MusicAlbum} с методом \texttt{\_\_init\_\_}, который создает пустой список треков.
    \item Создайте метод \texttt{add\_track}, который принимает название трека и его длительность (в секундах) в качестве аргументов и добавляет их в список.
    \item Создайте метод \texttt{remove\_track}, который удаляет трек из списка по его названию.
    \item Создайте метод \texttt{track\_count}, который возвращает общее количество треков в альбоме.
    \item Создайте экземпляр класса \texttt{MusicAlbum} и добавьте несколько треков.
    \item Выведите информацию о текущих треках на экран.
    \item Выведите общее количество треков на экран.
    \item Удалите один из треков и выведите обновленный список.
    \item Выведите общее количество треков после удаления на экран.
\end{itemize}

\textbf{Пример использования:}

\begin{verbatim}
album = MusicAlbum()
album.add_track("Yesterday", 125)
album.add_track("Hey Jude", 431)
album.add_track("Let It Be", 243)
print("Треки в альбоме:")
for name, duration in album.tracks:
    print(name, "-", duration, "сек.")
count = album.track_count()
print("Всего треков:", count)
album.remove_track("Hey Jude")
print("Треки после удаления 'Hey Jude':")
for name, duration in album.tracks:
    print(name, "-", duration, "сек.")
count = album.track_count()
print("Всего треков:", count)
\end{verbatim}

\textbf{Вывод:}
\begin{verbatim}
Треки в альбоме:
Yesterday - 125 сек.
Hey Jude - 431 сек.
Let It Be - 243 сек.
Всего треков: 3
Треки после удаления 'Hey Jude':
Yesterday - 125 сек.
Let It Be - 243 сек.
Всего треков: 2
\end{verbatim}

\item[25] Написать программу на Python, которая создает класс \texttt{EmployeeRoster} для представления списка сотрудников. Класс должен содержать методы для добавления и удаления сотрудников, а также подсчета общего количества работников. Программа также должна создавать экземпляр класса \texttt{EmployeeRoster}, добавлять сотрудников, удалять сотрудников и выводить информацию о персонале на экран.

\begin{itemize}
    \item Создайте класс \texttt{EmployeeRoster} с методом \texttt{\_\_init\_\_}, который создает пустой список сотрудников.
    \item Создайте метод \texttt{hire\_employee}, который принимает имя сотрудника и его должность в качестве аргументов и добавляет их в список.
    \item Создайте метод \texttt{fire\_employee}, который удаляет сотрудника из списка по имени.
    \item Создайте метод \texttt{employee\_count}, который возвращает общее количество сотрудников.
    \item Создайте экземпляр класса \texttt{EmployeeRoster} и добавьте несколько сотрудников.
    \item Выведите информацию о текущих сотрудниках на экран.
    \item Выведите общее количество работников на экран.
    \item Удалите одного из сотрудников и выведите обновленный список.
    \item Выведите общее количество работников после удаления на экран.
\end{itemize}

\textbf{Пример использования:}

\begin{verbatim}
roster = EmployeeRoster()
roster.hire_employee("Елена", "Менеджер")
roster.hire_employee("Дмитрий", "Разработчик")
roster.hire_employee("Ольга", "Дизайнер")
print("Сотрудники компании:")
for name, position in roster.employees:
    print(name, "-", position)
count = roster.employee_count()
print("Всего сотрудников:", count)
roster.fire_employee("Дмитрий")
print("Сотрудники после увольнения Дмитрия:")
for name, position in roster.employees:
    print(name, "-", position)
count = roster.employee_count()
print("Всего сотрудников:", count)
\end{verbatim}

\textbf{Вывод:}
\begin{verbatim}
Сотрудники компании:
Елена - Менеджер
Дмитрий - Разработчик
Ольга - Дизайнер
Всего сотрудников: 3
Сотрудники после увольнения Дмитрия:
Елена - Менеджер
Ольга - Дизайнер
Всего сотрудников: 2
\end{verbatim}

\item[26] Написать программу на Python, которая создает класс \texttt{ShoppingWishlist} для представления списка желаний покупателя. Класс должен содержать методы для добавления и удаления товаров, а также подсчета общего количества позиций. Программа также должна создавать экземпляр класса \texttt{ShoppingWishlist}, добавлять товары, удалять товары и выводить информацию о списке желаний на экран.

\begin{itemize}
    \item Создайте класс \texttt{ShoppingWishlist} с методом \texttt{\_\_init\_\_}, который создает пустой список товаров.
    \item Создайте метод \texttt{add\_item}, который принимает название товара и его приоритет (от 1 до 5) в качестве аргументов и добавляет их в список.
    \item Создайте метод \texttt{remove\_item}, который удаляет товар из списка по его названию.
    \item Создайте метод \texttt{item\_count}, который возвращает общее количество товаров в списке желаний.
    \item Создайте экземпляр класса \texttt{ShoppingWishlist} и добавьте несколько товаров.
    \item Выведите информацию о текущих товарах на экран.
    \item Выведите общее количество позиций на экран.
    \item Удалите один из товаров и выведите обновленный список.
    \item Выведите общее количество позиций после удаления на экран.
\end{itemize}

\textbf{Пример использования:}

\begin{verbatim}
wishlist = ShoppingWishlist()
wishlist.add_item("Наушники", 5)
wishlist.add_item("Книга", 3)
wishlist.add_item("Флешка", 2)
print("Список желаний:")
for name, priority in wishlist.items:
    print(name, "(приоритет", priority, ")")
count = wishlist.item_count()
print("Всего позиций:", count)
wishlist.remove_item("Книга")
print("Список после удаления книги:")
for name, priority in wishlist.items:
    print(name, "(приоритет", priority, ")")
count = wishlist.item_count()
print("Всего позиций:", count)
\end{verbatim}

\textbf{Вывод:}
\begin{verbatim}
Список желаний:
Наушники (приоритет 5 )
Книга (приоритет 3 )
Флешка (приоритет 2 )
Всего позиций: 3
Список после удаления книги:
Наушники (приоритет 5 )
Флешка (приоритет 2 )
Всего позиций: 2
\end{verbatim}

\item[27] Написать программу на Python, которая создает класс \texttt{DietPlan} для представления плана питания. Класс должен содержать методы для добавления и удаления блюд, а также подсчета общего количества приемов пищи. Программа также должна создавать экземпляр класса \texttt{DietPlan}, добавлять блюда, удалять блюда и выводить информацию о плане на экран.

\begin{itemize}
    \item Создайте класс \texttt{DietPlan} с методом \texttt{\_\_init\_\_}, который создает пустой список блюд.
    \item Создайте метод \texttt{add\_meal}, который принимает название блюда и количество калорий в качестве аргументов и добавляет их в список.
    \item Создайте метод \texttt{remove\_meal}, который удаляет блюдо из списка по его названию.
    \item Создайте метод \texttt{meal\_count}, который возвращает общее количество блюд в плане.
    \item Создайте экземпляр класса \texttt{DietPlan} и добавьте несколько блюд.
    \item Выведите информацию о текущих блюдах на экран.
    \item Выведите общее количество приемов пищи на экран.
    \item Удалите одно из блюд и выведите обновленный план.
    \item Выведите общее количество приемов пищи после удаления на экран.
\end{itemize}

\textbf{Пример использования:}

\begin{verbatim}
diet = DietPlan()
diet.add_meal("Овсянка", 300)
diet.add_meal("Салат", 150)
diet.add_meal("Курица", 400)
print("План питания:")
for dish, calories in diet.meals:
    print(dish, "-", calories, "ккал")
count = diet.meal_count()
print("Всего приемов пищи:", count)
diet.remove_meal("Салат")
print("План после удаления салата:")
for dish, calories in diet.meals:
    print(dish, "-", calories, "ккал")
count = diet.meal_count()
print("Всего приемов пищи:", count)
\end{verbatim}

\textbf{Вывод:}
\begin{verbatim}
План питания:
Овсянка - 300 ккал
Салат - 150 ккал
Курица - 400 ккал
Всего приемов пищи: 3
План после удаления салата:
Овсянка - 300 ккал
Курица - 400 ккал
Всего приемов пищи: 2
\end{verbatim}

\item[28] Написать программу на Python, которая создает класс \texttt{PhotoAlbum} для представления фотоальбома. Класс должен содержать методы для добавления и удаления фотографий, а также подсчета общего количества снимков. Программа также должна создавать экземпляр класса \texttt{PhotoAlbum}, добавлять фотографии, удалять фотографии и выводить информацию об альбоме на экран.

\begin{itemize}
    \item Создайте класс \texttt{PhotoAlbum} с методом \texttt{\_\_init\_\_}, который создает пустой список фотографий.
    \item Создайте метод \texttt{add\_photo}, который принимает название фотографии и дату съемки в качестве аргументов и добавляет их в список.
    \item Создайте метод \texttt{remove\_photo}, который удаляет фотографию из списка по её названию.
    \item Создайте метод \texttt{photo\_count}, который возвращает общее количество фотографий в альбоме.
    \item Создайте экземпляр класса \texttt{PhotoAlbum} и добавьте несколько фотографий.
    \item Выведите информацию о текущих фотографиях на экран.
    \item Выведите общее количество снимков на экран.
    \item Удалите одну из фотографий и выведите обновленный альбом.
    \item Выведите общее количество снимков после удаления на экран.
\end{itemize}

\textbf{Пример использования:}

\begin{verbatim}
album = PhotoAlbum()
album.add_photo("Пляж", "2023-07-15")
album.add_photo("Горы", "2023-08-20")
album.add_photo("Семья", "2023-12-25")
print("Фотографии в альбоме:")
for name, date in album.photos:
    print(name, "-", date)
count = album.photo_count()
print("Всего фотографий:", count)
album.remove_photo("Горы")
print("Фотографии после удаления 'Горы':")
for name, date in album.photos:
    print(name, "-", date)
count = album.photo_count()
print("Всего фотографий:", count)
\end{verbatim}

\textbf{Вывод:}
\begin{verbatim}
Фотографии в альбоме:
Пляж - 2023-07-15
Горы - 2023-08-20
Семья - 2023-12-25
Всего фотографий: 3
Фотографии после удаления 'Горы':
Пляж - 2023-07-15
Семья - 2023-12-25
Всего фотографий: 2
\end{verbatim}

\item[29] Написать программу на Python, которая создает класс \texttt{StudyMaterials} для представления учебных материалов. Класс должен содержать методы для добавления и удаления материалов, а также подсчета общего количества ресурсов. Программа также должна создавать экземпляр класса \texttt{StudyMaterials}, добавлять материалы, удалять материалы и выводить информацию о ресурсах на экран.

\begin{itemize}
    \item Создайте класс \texttt{StudyMaterials} с методом \texttt{\_\_init\_\_}, который создает пустой список материалов.
    \item Создайте метод \texttt{add\_material}, который принимает название материала и тип (например, "книга", "видео", "статья") в качестве аргументов и добавляет их в список.
    \item Создайте метод \texttt{remove\_material}, который удаляет материал из списка по его названию.
    \item Создайте метод \texttt{material\_count}, который возвращает общее количество учебных материалов.
    \item Создайте экземпляр класса \texttt{StudyMaterials} и добавьте несколько материалов.
    \item Выведите информацию о текущих материалах на экран.
    \item Выведите общее количество ресурсов на экран.
    \item Удалите один из материалов и выведите обновленный список.
    \item Выведите общее количество ресурсов после удаления на экран.
\end{itemize}

\textbf{Пример использования:}

\begin{verbatim}
materials = StudyMaterials()
materials.add_material("Алгоритмы", "книга")
materials.add_material("Python для начинающих", "видео")
materials.add_material("Структуры данных", "статья")
print("Учебные материалы:")
for name, mtype in materials.materials:
    print(name, "-", mtype)
count = materials.material_count()
print("Всего материалов:", count)
materials.remove_material("Python для начинающих")
print("Материалы после удаления видео:")
for name, mtype in materials.materials:
    print(name, "-", mtype)
count = materials.material_count()
print("Всего материалов:", count)
\end{verbatim}

\textbf{Вывод:}
\begin{verbatim}
Учебные материалы:
Алгоритмы - книга
Python для начинающих - видео
Структуры данных - статья
Всего материалов: 3
Материалы после удаления видео:
Алгоритмы - книга
Структуры данных - статья
Всего материалов: 2
\end{verbatim}

\item[30] Написать программу на Python, которая создает класс \texttt{ArtCollection} для представления коллекции произведений искусства. Класс должен содержать методы для добавления и удаления работ, а также подсчета общего количества экспонатов. Программа также должна создавать экземпляр класса \texttt{ArtCollection}, добавлять работы, удалять работы и выводить информацию о коллекции на экран.

\begin{itemize}
    \item Создайте класс \texttt{ArtCollection} с методом \texttt{\_\_init\_\_}, который создает пустой список произведений.
    \item Создайте метод \texttt{add\_artwork}, который принимает название работы и имя художника в качестве аргументов и добавляет их в список.
    \item Создайте метод \texttt{remove\_artwork}, который удаляет работу из списка по её названию.
    \item Создайте метод \texttt{artwork\_count}, который возвращает общее количество произведений в коллекции.
    \item Создайте экземпляр класса \texttt{ArtCollection} и добавьте несколько работ.
    \item Выведите информацию о текущих произведениях на экран.
    \item Выведите общее количество экспонатов на экран.
    \item Удалите одну из работ и выведите обновленный список.
    \item Выведите общее количество экспонатов после удаления на экран.
\end{itemize}

\textbf{Пример использования:}

\begin{verbatim}
art = ArtCollection()
art.add_artwork("Звёздная ночь", "Ван Гог")
art.add_artwork("Мона Лиза", "Леонардо да Винчи")
art.add_artwork("Крик", "Мунк")
print("Произведения в коллекции:")
for title, artist in art.artworks:
    print(title, "—", artist)
count = art.artwork_count()
print("Всего экспонатов:", count)
art.remove_artwork("Мона Лиза")
print("Произведения после удаления 'Моны Лизы':")
for title, artist in art.artworks:
    print(title, "—", artist)
count = art.artwork_count()
print("Всего экспонатов:", count)
\end{verbatim}

\textbf{Вывод:}
\begin{verbatim}
Произведения в коллекции:
Звёздная ночь — Ван Гог
Мона Лиза — Леонардо да Винчи
Крик — Мунк
Всего экспонатов: 3
Произведения после удаления 'Моны Лизы':
Звёздная ночь — Ван Гог
Крик — Мунк
Всего экспонатов: 2
\end{verbatim}

\item[31] Написать программу на Python, которая создает класс \texttt{FlightSchedule} для представления расписания рейсов. Класс должен содержать методы для добавления и удаления рейсов, а также подсчета общего количества перелетов. Программа также должна создавать экземпляр класса \texttt{FlightSchedule}, добавлять рейсы, удалять рейсы и выводить информацию о расписании на экран.

\begin{itemize}
    \item Создайте класс \texttt{FlightSchedule} с методом \texttt{\_\_init\_\_}, который создает пустой список рейсов.
    \item Создайте метод \texttt{add\_flight}, который принимает номер рейса и пункт назначения в качестве аргументов и добавляет их в список.
    \item Создайте метод \texttt{remove\_flight}, который удаляет рейс из списка по его номеру.
    \item Создайте метод \texttt{flight\_count}, который возвращает общее количество рейсов в расписании.
    \item Создайте экземпляр класса \texttt{FlightSchedule} и добавьте несколько рейсов.
    \item Выведите информацию о текущих рейсах на экран.
    \item Выведите общее количество перелетов на экран.
    \item Удалите один из рейсов и выведите обновленное расписание.
    \item Выведите общее количество перелетов после удаления на экран.
\end{itemize}

\textbf{Пример использования:}

\begin{verbatim}
flights = FlightSchedule()
flights.add_flight("SU123", "Париж")
flights.add_flight("SU456", "Лондон")
flights.add_flight("SU789", "Рим")
print("Рейсы:")
for num, dest in flights.flights:
    print(num, "-", dest)
count = flights.flight_count()
print("Всего рейсов:", count)
flights.remove_flight("SU456")
print("Рейсы после отмены SU456:")
for num, dest in flights.flights:
    print(num, "-", dest)
count = flights.flight_count()
print("Всего рейсов:", count)
\end{verbatim}

\textbf{Вывод:}
\begin{verbatim}
Рейсы:
SU123 - Париж
SU456 - Лондон
SU789 - Рим
Всего рейсов: 3
Рейсы после отмены SU456:
SU123 - Париж
SU789 - Рим
Всего рейсов: 2
\end{verbatim}

\item[32] Написать программу на Python, которая создает класс \texttt{RecipeIngredients} для представления ингредиентов рецепта. Класс должен содержать методы для добавления и удаления ингредиентов, а также подсчета общего количества компонентов. Программа также должна создавать экземпляр класса \texttt{RecipeIngredients}, добавлять ингредиенты, удалять ингредиенты и выводить информацию о рецепте на экран.

\begin{itemize}
    \item Создайте класс \texttt{RecipeIngredients} с методом \texttt{\_\_init\_\_}, который создает пустой список ингредиентов.
    \item Создайте метод \texttt{add\_ingredient}, который принимает название ингредиента и количество (в граммах или штуках) в качестве аргументов и добавляет их в список.
    \item Создайте метод \texttt{remove\_ingredient}, который удаляет ингредиент из списка по его названию.
    \item Создайте метод \texttt{ingredient\_count}, который возвращает общее количество ингредиентов в рецепте.
    \item Создайте экземпляр класса \texttt{RecipeIngredients} и добавьте несколько ингредиентов.
    \item Выведите информацию о текущих ингредиентах на экран.
    \item Выведите общее количество компонентов на экран.
    \item Удалите один из ингредиентов и выведите обновленный список.
    \item Выведите общее количество компонентов после удаления на экран.
\end{itemize}

\textbf{Пример использования:}

\begin{verbatim}
recipe = RecipeIngredients()
recipe.add_ingredient("Мука", 200)
recipe.add_ingredient("Сахар", 100)
recipe.add_ingredient("Яйца", 2)
print("Ингредиенты рецепта:")
for name, qty in recipe.ingredients:
    print(name, "-", qty)
count = recipe.ingredient_count()
print("Всего ингредиентов:", count)
recipe.remove_ingredient("Сахар")
print("Ингредиенты после удаления сахара:")
for name, qty in recipe.ingredients:
    print(name, "-", qty)
count = recipe.ingredient_count()
print("Всего ингредиентов:", count)
\end{verbatim}

\textbf{Вывод:}
\begin{verbatim}
Ингредиенты рецепта:
Мука - 200
Сахар - 100
Яйца - 2
Всего ингредиентов: 3
Ингредиенты после удаления сахара:
Мука - 200
Яйца - 2
Всего ингредиентов: 2
\end{verbatim}

\item[33] Написать программу на Python, которая создает класс \texttt{WorkoutPlan} для представления плана тренировок. Класс должен содержать методы для добавления и удаления упражнений, а также подсчета общего количества упражнений. Программа также должна создавать экземпляр класса \texttt{WorkoutPlan}, добавлять упражнения, удалять упражнения и выводить информацию о плане на экран.

\begin{itemize}
    \item Создайте класс \texttt{WorkoutPlan} с методом \texttt{\_\_init\_\_}, который создает пустой список упражнений.
    \item Создайте метод \texttt{add\_exercise}, который принимает название упражнения и количество повторений в качестве аргументов и добавляет их в список.
    \item Создайте метод \texttt{remove\_exercise}, который удаляет упражнение из списка по его названию.
    \item Создайте метод \texttt{exercise\_count}, который возвращает общее количество упражнений в плане.
    \item Создайте экземпляр класса \texttt{WorkoutPlan} и добавьте несколько упражнений.
    \item Выведите информацию о текущих упражнениях на экран.
    \item Выведите общее количество упражнений на экран.
    \item Удалите одно из упражнений и выведите обновленный план.
    \item Выведите общее количество упражнений после удаления на экран.
\end{itemize}

\textbf{Пример использования:}

\begin{verbatim}
workout = WorkoutPlan()
workout.add_exercise("Бег", 30)
workout.add_exercise("Планка", 3)
workout.add_exercise("Приседания", 20)
print("План тренировки:")
for name, reps in workout.exercises:
    print(name, "-", reps)
count = workout.exercise_count()
print("Всего упражнений:", count)
workout.remove_exercise("Планка")
print("План после удаления планки:")
for name, reps in workout.exercises:
    print(name, "-", reps)
count = workout.exercise_count()
print("Всего упражнений:", count)
\end{verbatim}

\textbf{Вывод:}
\begin{verbatim}
План тренировки:
Бег - 30
Планка - 3
Приседания - 20
Всего упражнений: 3
План после удаления планки:
Бег - 30
Приседания - 20
Всего упражнений: 2
\end{verbatim}

\item[34] Написать программу на Python, которая создает класс \texttt{InventoryManager} для представления управления запасами. Класс должен содержать методы для добавления и удаления товаров, а также подсчета общего количества типов товаров. Программа также должна создавать экземпляр класса \texttt{InventoryManager}, добавлять товары, удалять товары и выводить информацию о запасах на экран.

\begin{itemize}
    \item Создайте класс \texttt{InventoryManager} с методом \texttt{\_\_init\_\_}, который создает пустой список товаров.
    \item Создайте метод \texttt{add\_product}, который принимает название товара и количество единиц в качестве аргументов и добавляет их в список.
    \item Создайте метод \texttt{remove\_product}, который удаляет товар из списка по его названию.
    \item Создайте метод \texttt{product\_types}, который возвращает общее количество различных типов товаров.
    \item Создайте экземпляр класса \texttt{InventoryManager} и добавьте несколько товаров.
    \item Выведите информацию о текущих товарах на экран.
    \item Выведите общее количество типов товаров на экран.
    \item Удалите один из товаров и выведите обновленный список.
    \item Выведите общее количество типов товаров после удаления на экран.
\end{itemize}

\textbf{Пример использования:}

\begin{verbatim}
inv = InventoryManager()
inv.add_product("Мыло", 100)
inv.add_product("Шампунь", 50)
inv.add_product("Зубная паста", 75)
print("Товары на складе:")
for name, qty in inv.products:
    print(name, "-", qty)
types = inv.product_types()
print("Всего типов товаров:", types)
inv.remove_product("Шампунь")
print("Товары после удаления шампуня:")
for name, qty in inv.products:
    print(name, "-", qty)
types = inv.product_types()
print("Всего типов товаров:", types)
\end{verbatim}

\textbf{Вывод:}
\begin{verbatim}
Товары на складе:
Мыло - 100
Шампунь - 50
Зубная паста - 75
Всего типов товаров: 3
Товары после удаления шампуня:
Мыло - 100
Зубная паста - 75
Всего типов товаров: 2
\end{verbatim}

\item[35] Написать программу на Python, которая создает класс \texttt{EventGuestList} для представления списка гостей мероприятия. Класс должен содержать методы для добавления и удаления гостей, а также подсчета общего количества приглашенных. Программа также должна создавать экземпляр класса \texttt{EventGuestList}, добавлять гостей, удалять гостей и выводить информацию о списке на экран.

\begin{itemize}
    \item Создайте класс \texttt{EventGuestList} с методом \texttt{\_\_init\_\_}, который создает пустой список гостей.
    \item Создайте метод \texttt{add\_guest}, который принимает имя гостя и его статус (например, "подтвержден", "ожидает") в качестве аргументов и добавляет их в список.
    \item Создайте метод \texttt{remove\_guest}, который удаляет гостя из списка по имени.
    \item Создайте метод \texttt{guest\_count}, который возвращает общее количество гостей в списке.
    \item Создайте экземпляр класса \texttt{EventGuestList} и добавьте несколько гостей.
    \item Выведите информацию о текущих гостях на экран.
    \item Выведите общее количество приглашенных на экран.
    \item Удалите одного из гостей и выведите обновленный список.
    \item Выведите общее количество приглашенных после удаления на экран.
\end{itemize}

\textbf{Пример использования:}

\begin{verbatim}
guests = EventGuestList()
guests.add_guest("Андрей", "подтвержден")
guests.add_guest("Светлана", "ожидает")
guests.add_guest("Михаил", "подтвержден")
print("Список гостей:")
for name, status in guests.guests:
    print(name, "-", status)
count = guests.guest_count()
print("Всего гостей:", count)
guests.remove_guest("Светлана")
print("Список после отмены Светланы:")
for name, status in guests.guests:
    print(name, "-", status)
count = guests.guest_count()
print("Всего гостей:", count)
\end{verbatim}

\textbf{Вывод:}
\begin{verbatim}
Список гостей:
Андрей - подтвержден
Светлана - ожидает
Михаил - подтвержден
Всего гостей: 3
Список после отмены Светланы:
Андрей - подтвержден
Михаил - подтвержден
Всего гостей: 2
\end{verbatim}
\end{enumerate}
\subsubsection{Задача 5}

\begin{enumerate}
\item[1] Написать программу на Python, которая создает класс \texttt{Bank}, представляющий банк. Класс должен содержать методы для создания учетных записей клиентов, внесения депозитов, снятия средств и проверки баланса. Программа также должна создавать экземпляр класса \texttt{Bank}, создавать учетные записи клиентов, вносить депозиты, снимать средства и проверять баланс.

\begin{itemize}
    \item Создайте класс \texttt{Bank} с методом \texttt{\_\_init\_\_}, который создает пустой словарь клиентов.
    \item Создайте метод \texttt{create\_account}, который принимает номер счета и начальный баланс в качестве аргументов. Метод должен проверять, существует ли уже номер счета в словаре клиентов. Если это так, он должен выводить сообщение об ошибке. В противном случае, он должен добавить номер счета в словарь клиентов с начальным балансом в качестве значения.
    \item Создайте метод \texttt{make\_deposit}, который принимает номер счета и сумму в качестве аргументов. Метод должен проверять, существует ли номер счета в словаре клиентов. Если это так, он должен добавить сумму к текущему балансу счета. Если номер счета не существует, он должен вывести сообщение об ошибке.
    \item Создайте метод \texttt{make\_withdrawal}, который принимает номер счета и сумму в качестве аргументов. Метод должен проверять, существует ли номер счета в словаре клиентов. Если это так, он должен проверить, достаточно ли средств на счете для снятия. Если это так, он должен вычесть сумму из текущего баланса счета. В противном случае, он должен вывести сообщение об ошибке, указывающее на недостаточность средств. Если номер счета не существует, он должен вывести сообщение об ошибке.
    \item Создайте метод \texttt{check\_balance}, который принимает номер счета в качестве аргумента. Метод должен проверять, существует ли номер счета в словаре клиентов. Если это так, он должен извлечь и вывести текущий баланс счета. Если номер счета не существует, он должен вывести сообщение об ошибке.
    \item Создайте экземпляр класса \texttt{Bank} и создайте учетные записи клиентов.
    \item Вносите депозиты на счета клиентов.
    \item Снимайте средства со счетов клиентов.
    \item Проверяйте баланс счетов клиентов.
\end{itemize}

\textbf{Пример использования:}

\begin{verbatim}
bank = Bank()
acno1 = "SB-123"
damt1 = 1000
print("Новый номер счета: ", acno1, " Внесенная сумма: ", damt1)
bank.create_account(acno1, damt1)
acno2 = "SB-124"
damt2 = 1500
print("Новый номер счета: ", acno2, " Внесенная сумма: ", damt2)
bank.create_account(acno2, damt2)
wamt1 = 600
print("\nДепозит средств: ", wamt1, " на счет № ", acno1)
bank.make_deposit(acno1, wamt1)
wamt2 = 350
print("Вывод средств: ", wamt2, " со счета № ", acno2)
bank.make_withdrawal(acno2, wamt2)
print("Номер расчетного счета: ", acno1)
bank.check_balance(acno1)
print("Номер расчетного счета: ", acno2)
bank.check_balance(acno2)
wamt3 = 1200
print("Вывод средств: ", wamt3, " со счета № ", acno2)
bank.make_withdrawal(acno2, wamt3)
acno3 = "SB-134"
print("Проверка баланса счета № ", acno3)
bank.check_balance(acno3)  # Non-existent account number
\end{verbatim}

\item[2] Написать программу на Python, которая создает класс \texttt{CreditUnion}, представляющий кредитный союз. Класс должен содержать методы для открытия счетов участников, пополнения баланса, снятия денег и запроса текущего состояния счета. Программа также должна создавать экземпляр класса \texttt{CreditUnion}, открывать счета, выполнять операции и проверять балансы.

\begin{itemize}
    \item Создайте класс \texttt{CreditUnion} с методом \texttt{\_\_init\_\_}, инициализирующим пустой словарь счетов.
    \item Создайте метод \texttt{open\_account}, принимающий идентификатор счета и стартовый остаток. Если счет уже существует, выведите ошибку; иначе — добавьте запись.
    \item Создайте метод \texttt{deposit}, принимающий идентификатор счета и сумму. Если счет существует, увеличьте баланс; иначе — сообщите об ошибке.
    \item Создайте метод \texttt{withdraw}, принимающий идентификатор счета и сумму. Если счет существует и средств достаточно, уменьшите баланс; иначе — выведите соответствующую ошибку.
    \item Создайте метод \texttt{get\_balance}, принимающий идентификатор счета. Если счет существует, выведите его баланс; иначе — сообщите об ошибке.
    \item Создайте экземпляр \texttt{CreditUnion}.
    \item Откройте несколько счетов.
    \item Выполните пополнения.
    \item Выполните снятия.
    \item Проверьте балансы.
\end{itemize}

\textbf{Пример использования:}

\begin{verbatim}
cu = CreditUnion()
cu.open_account("CU-001", 2000)
cu.open_account("CU-002", 500)
cu.deposit("CU-001", 300)
cu.withdraw("CU-002", 200)
cu.get_balance("CU-001")
cu.get_balance("CU-002")
cu.withdraw("CU-002", 400)  # недостаточно средств
cu.get_balance("CU-999")     # несуществующий счет
\end{verbatim}

\item[3] Написать программу на Python, которая создает класс \texttt{SavingsBank}, моделирующий сберегательный банк. Класс должен поддерживать создание счетов, внесение вкладов, снятие средств и проверку баланса. Программа должна демонстрировать работу всех методов на примере нескольких счетов.

\begin{itemize}
    \item Создайте класс \texttt{SavingsBank} с методом \texttt{\_\_init\_\_}, инициализирующим пустой словарь \texttt{accounts}.
    \item Метод \texttt{add\_account} принимает номер счета и начальный депозит. Если счет уже есть — ошибка; иначе — добавление.
    \item Метод \texttt{credit} принимает номер счета и сумму. При наличии счета — пополнение; иначе — ошибка.
    \item Метод \texttt{debit} принимает номер счета и сумму. При наличии счета и достаточном балансе — снятие; иначе — ошибка.
    \item Метод \texttt{show\_balance} принимает номер счета и выводит баланс или сообщение об ошибке.
    \item Создайте экземпляр \texttt{SavingsBank}.
    \item Добавьте два счета.
    \item Пополните один из них.
    \item Снимите средства с другого.
    \item Проверьте балансы обоих и несуществующего счета.
\end{itemize}

\textbf{Пример использования:}

\begin{verbatim}
sb = SavingsBank()
sb.add_account("SAV-101", 1000)
sb.add_account("SAV-102", 800)
sb.credit("SAV-101", 200)
sb.debit("SAV-102", 300)
sb.show_balance("SAV-101")
sb.show_balance("SAV-102")
sb.debit("SAV-102", 600)  # недостаточно
sb.show_balance("SAV-999") # несуществует
\end{verbatim}

\item[4] Написать программу на Python, которая создает класс \texttt{DigitalWallet}, представляющий цифровой кошелек. Класс должен поддерживать регистрацию кошельков, пополнение, списание и проверку баланса.

\begin{itemize}
    \item Создайте класс \texttt{DigitalWallet} с методом \texttt{\_\_init\_\_}, создающим пустой словарь \texttt{wallets}.
    \item Метод \texttt{register\_wallet} принимает ID кошелька и начальный баланс. Если ID занят — ошибка; иначе — регистрация.
    \item Метод \texttt{top\_up} принимает ID и сумму. При существовании кошелька — пополнение; иначе — ошибка.
    \item Метод \texttt{spend} принимает ID и сумму. При наличии кошелька и достаточном балансе — списание; иначе — ошибка.
    \item Метод \texttt{get\_wallet\_balance} принимает ID и выводит баланс или сообщение об ошибке.
    \item Создайте экземпляр \texttt{DigitalWallet}.
    \item Зарегистрируйте два кошелька.
    \item Пополните один.
    \item Потратьте с другого.
    \item Проверьте балансы и попытайтесь проверить несуществующий.
\end{itemize}

\textbf{Пример использования:}

\begin{verbatim}
dw = DigitalWallet()
dw.register_wallet("WAL-01", 500)
dw.register_wallet("WAL-02", 300)
dw.top_up("WAL-01", 100)
dw.spend("WAL-02", 150)
dw.get_wallet_balance("WAL-01")
dw.get_wallet_balance("WAL-02")
dw.spend("WAL-02", 200)  # недостаточно
dw.get_wallet_balance("WAL-99")  # несуществует
\end{verbatim}

\item[5] Написать программу на Python, которая создает класс \texttt{PaymentSystem}, моделирующий систему платежей. Класс должен поддерживать создание счетов, зачисление средств, списание и проверку баланса.

\begin{itemize}
    \item Создайте класс \texttt{PaymentSystem} с методом \texttt{\_\_init\_\_}, инициализирующим пустой словарь \texttt{accounts}.
    \item Метод \texttt{create\_user\_account} принимает идентификатор и начальный баланс. Если уже есть — ошибка; иначе — создание.
    \item Метод \texttt{credit\_account} принимает ID и сумму. При наличии счета — зачисление; иначе — ошибка.
    \item Метод \texttt{debit\_account} принимает ID и сумму. При наличии счета и достаточном балансе — списание; иначе — ошибка.
    \item Метод \texttt{check\_account\_balance} принимает ID и выводит баланс или ошибку.
    \item Создайте экземпляр \texttt{PaymentSystem}.
    \item Создайте два счета.
    \item Зачислите средства на один.
    \item Спишите с другого.
    \item Проверьте балансы и несуществующий счет.
\end{itemize}

\textbf{Пример использования:}

\begin{verbatim}
ps = PaymentSystem()
ps.create_user_account("USR-1", 1200)
ps.create_user_account("USR-2", 700)
ps.credit_account("USR-1", 300)
ps.debit_account("USR-2", 200)
ps.check_account_balance("USR-1")
ps.check_account_balance("USR-2")
ps.debit_account("USR-2", 600)  # недостаточно
ps.check_account_balance("USR-999")  # несуществует
\end{verbatim}

\item[6] Написать программу на Python, которая создает класс \texttt{MicroFinance}, представляющий микрофинансовую организацию. Класс должен поддерживать открытие счетов, пополнение, снятие и проверку баланса.

\begin{itemize}
    \item Создайте класс \texttt{MicroFinance} с методом \texttt{\_\_init\_\_}, создающим пустой словарь \texttt{clients}.
    \item Метод \texttt{open\_client\_account} принимает номер счета и стартовый баланс. Если счет существует — ошибка; иначе — открытие.
    \item Метод \texttt{fund\_account} принимает номер счета и сумму. При наличии счета — пополнение; иначе — ошибка.
    \item Метод \texttt{withdraw\_funds} принимает номер счета и сумму. При наличии счета и достаточном балансе — снятие; иначе — ошибка.
    \item Метод \texttt{view\_balance} принимает номер счета и выводит баланс или сообщение об ошибке.
    \item Создайте экземпляр \texttt{MicroFinance}.
    \item Откройте два счета.
    \item Пополните один.
    \item Снимите с другого.
    \item Проверьте балансы и несуществующий счет.
\end{itemize}

\textbf{Пример использования:}

\begin{verbatim}
mf = MicroFinance()
mf.open_client_account("MF-201", 900)
mf.open_client_account("MF-202", 400)
mf.fund_account("MF-201", 100)
mf.withdraw_funds("MF-202", 150)
mf.view_balance("MF-201")
mf.view_balance("MF-202")
mf.withdraw_funds("MF-202", 300)  # недостаточно
mf.view_balance("MF-999")  # несуществует
\end{verbatim}

\item[7] Написать программу на Python, которая создает класс \texttt{OnlineBank}, моделирующий онлайн-банк. Класс должен поддерживать регистрацию счетов, депозиты, выводы и проверку баланса.

\begin{itemize}
    \item Создайте класс \texttt{OnlineBank} с методом \texttt{\_\_init\_\_}, инициализирующим пустой словарь \texttt{accounts}.
    \item Метод \texttt{register\_account} принимает ID и начальный баланс. Если ID занят — ошибка; иначе — регистрация.
    \item Метод \texttt{deposit\_funds} принимает ID и сумму. При наличии счета — пополнение; иначе — ошибка.
    \item Метод \texttt{withdraw\_funds} принимает ID и сумму. При наличии счета и достаточном балансе — снятие; иначе — ошибка.
    \item Метод \texttt{check\_current\_balance} принимает ID и выводит баланс или ошибку.
    \item Создайте экземпляр \texttt{OnlineBank}.
    \item Зарегистрируйте два счета.
    \item Пополните один.
    \item Снимите с другого.
    \item Проверьте балансы и несуществующий счет.
\end{itemize}

\textbf{Пример использования:}

\begin{verbatim}
ob = OnlineBank()
ob.register_account("ONB-501", 1500)
ob.register_account("ONB-502", 600)
ob.deposit_funds("ONB-501", 200)
ob.withdraw_funds("ONB-502", 250)
ob.check_current_balance("ONB-501")
ob.check_current_balance("ONB-502")
ob.withdraw_funds("ONB-502", 400)  # недостаточно
ob.check_current_balance("ONB-999")  # несуществует
\end{verbatim}

\item[8] Написать программу на Python, которая создает класс \texttt{FinTechApp}, представляющий финтех-приложение. Класс должен поддерживать создание аккаунтов, пополнение, снятие и проверку баланса.

\begin{itemize}
    \item Создайте класс \texttt{FinTechApp} с методом \texttt{\_\_init\_\_}, создающим пустой словарь \texttt{users}.
    \item Метод \texttt{create\_user} принимает логин и начальный баланс. Если логин занят — ошибка; иначе — создание.
    \item Метод \texttt{add\_money} принимает логин и сумму. При наличии аккаунта — пополнение; иначе — ошибка.
    \item Метод \texttt{remove\_money} принимает логин и сумму. При наличии аккаунта и достаточном балансе — снятие; иначе — ошибка.
    \item Метод \texttt{get\_user\_balance} принимает логин и выводит баланс или ошибку.
    \item Создайте экземпляр \texttt{FinTechApp}.
    \item Создайте двух пользователей.
    \item Пополните одного.
    \item Снимите у другого.
    \item Проверьте балансы и несуществующего пользователя.
\end{itemize}

\textbf{Пример использования:}

\begin{verbatim}
ft = FinTechApp()
ft.create_user("alice", 2000)
ft.create_user("bob", 800)
ft.add_money("alice", 300)
ft.remove_money("bob", 200)
ft.get_user_balance("alice")
ft.get_user_balance("bob")
ft.remove_money("bob", 700)  # недостаточно
ft.get_user_balance("charlie")  # несуществует
\end{verbatim}

\item[9] Написать программу на Python, которая создает класс \texttt{CryptoWallet}, моделирующий криптовалютный кошелек. Класс должен поддерживать создание кошельков, пополнение, перевод и проверку баланса.

\begin{itemize}
    \item Создайте класс \texttt{CryptoWallet} с методом \texttt{\_\_init\_\_}, инициализирующим пустой словарь \texttt{wallets}.
    \item Метод \texttt{generate\_wallet} принимает адрес и начальный баланс. Если адрес уже есть — ошибка; иначе — создание.
    \item Метод \texttt{receive\_coins} принимает адрес и сумму. При наличии кошелька — пополнение; иначе — ошибка.
    \item Метод \texttt{send\_coins} принимает адрес и сумму. При наличии кошелька и достаточном балансе — списание; иначе — ошибка.
    \item Метод \texttt{check\_wallet\_balance} принимает адрес и выводит баланс или ошибку.
    \item Создайте экземпляр \texttt{CryptoWallet}.
    \item Создайте два кошелька.
    \item Пополните один.
    \item Отправьте с другого.
    \item Проверьте балансы и несуществующий адрес.
\end{itemize}

\textbf{Пример использования:}

\begin{verbatim}
cw = CryptoWallet()
cw.generate_wallet("0x1a2b", 10.5)
cw.generate_wallet("0x3c4d", 5.0)
cw.receive_coins("0x1a2b", 2.0)
cw.send_coins("0x3c4d", 1.5)
cw.check_wallet_balance("0x1a2b")
cw.check_wallet_balance("0x3c4d")
cw.send_coins("0x3c4d", 4.0)  # недостаточно
cw.check_wallet_balance("0x9999")  # несуществует
\end{verbatim}

\item[10] Написать программу на Python, которая создает класс \texttt{StudentFund}, представляющий студенческий фонд. Класс должен поддерживать создание счетов студентов, внесение средств, снятие и проверку баланса.

\begin{itemize}
    \item Создайте класс \texttt{StudentFund} с методом \texttt{\_\_init\_\_}, создающим пустой словарь \texttt{students}.
    \item Метод \texttt{enroll\_student} принимает ID студента и начальный грант. Если ID уже есть — ошибка; иначе — зачисление.
    \item Метод \texttt{add\_grant} принимает ID и сумму. При наличии студента — пополнение; иначе — ошибка.
    \item Метод \texttt{use\_funds} принимает ID и сумму. При наличии студента и достаточном балансе — списание; иначе — ошибка.
    \item Метод \texttt{view\_student\_balance} принимает ID и выводит баланс или ошибку.
    \item Создайте экземпляр \texttt{StudentFund}.
    \item Зачислите двух студентов.
    \item Пополните одного.
    \item Снимите у другого.
    \item Проверьте балансы и несуществующего студента.
\end{itemize}

\textbf{Пример использования:}

\begin{verbatim}
sf = StudentFund()
sf.enroll_student("STU-01", 5000)
sf.enroll_student("STU-02", 3000)
sf.add_grant("STU-01", 1000)
sf.use_funds("STU-02", 800)
sf.view_student_balance("STU-01")
sf.view_student_balance("STU-02")
sf.use_funds("STU-02", 2500)  # недостаточно
sf.view_student_balance("STU-99")  # несуществует
\end{verbatim}

\item[11] Написать программу на Python, которая создает класс \texttt{GameCurrency}, моделирующий внутриигровую валюту. Класс должен поддерживать создание аккаунтов игроков, начисление монет, трату и проверку баланса.

\begin{itemize}
    \item Создайте класс \texttt{GameCurrency} с методом \texttt{\_\_init\_\_}, инициализирующим пустой словарь \texttt{players}.
    \item Метод \texttt{create\_player} принимает ник и начальный баланс. Если ник занят — ошибка; иначе — создание.
    \item Метод \texttt{award\_coins} принимает ник и сумму. При наличии игрока — начисление; иначе — ошибка.
    \item Метод \texttt{spend\_coins} принимает ник и сумму. При наличии игрока и достаточном балансе — списание; иначе — ошибка.
    \item Метод \texttt{get\_player\_balance} принимает ник и выводит баланс или ошибку.
    \item Создайте экземпляр \texttt{GameCurrency}.
    \item Создайте двух игроков.
    \item Начислите одному.
    \item Потратьте у другого.
    \item Проверьте балансы и несуществующего игрока.
\end{itemize}

\textbf{Пример использования:}

\begin{verbatim}
gc = GameCurrency()
gc.create_player("hero1", 100)
gc.create_player("hero2", 75)
gc.award_coins("hero1", 25)
gc.spend_coins("hero2", 30)
gc.get_player_balance("hero1")
gc.get_player_balance("hero2")
gc.spend_coins("hero2", 50)  # недостаточно
gc.get_player_balance("hero99")  # несуществует
\end{verbatim}

\item[12] Написать программу на Python, которая создает класс \texttt{CharityFund}, представляющий благотворительный фонд. Класс должен поддерживать создание счетов доноров, получение пожертвований, выдачу средств и проверку баланса.

\begin{itemize}
    \item Создайте класс \texttt{CharityFund} с методом \texttt{\_\_init\_\_}, создающим пустой словарь \texttt{donors}.
    \item Метод \texttt{register\_donor} принимает ID и начальный взнос. Если ID есть — ошибка; иначе — регистрация.
    \item Метод \texttt{accept\_donation} принимает ID и сумму. При наличии донора — пополнение; иначе — ошибка.
    \item Метод \texttt{distribute\_funds} принимает ID и сумму. При наличии донора и достаточном балансе — списание; иначе — ошибка.
    \item Метод \texttt{check\_donor\_balance} принимает ID и выводит баланс или ошибку.
    \item Создайте экземпляр \texttt{CharityFund}.
    \item Зарегистрируйте двух доноров.
    \item Примите пожертвование от одного.
    \item Распределите средства от другого.
    \item Проверьте балансы и несуществующего донора.
\end{itemize}

\textbf{Пример использования:}

\begin{verbatim}
cf = CharityFund()
cf.register_donor("DON-1", 2000)
cf.register_donor("DON-2", 1500)
cf.accept_donation("DON-1", 500)
cf.distribute_funds("DON-2", 600)
cf.check_donor_balance("DON-1")
cf.check_donor_balance("DON-2")
cf.distribute_funds("DON-2", 1000)  # недостаточно
cf.check_donor_balance("DON-99")  # несуществует
\end{verbatim}

\item[13] Написать программу на Python, которая создает класс \texttt{TravelWallet}, моделирующий кошелек для путешествий. Класс должен поддерживать создание профилей, пополнение, оплату и проверку баланса.

\begin{itemize}
    \item Создайте класс \texttt{TravelWallet} с методом \texttt{\_\_init\_\_}, инициализирующим пустой словарь \texttt{profiles}.
    \item Метод \texttt{create\_profile} принимает имя профиля и начальный бюджет. Если профиль существует — ошибка; иначе — создание.
    \item Метод \texttt{load\_funds} принимает имя профиля и сумму. При наличии профиля — пополнение; иначе — ошибка.
    \item Метод \texttt{pay\_expense} принимает имя профиля и сумму. При наличии профиля и достаточном балансе — списание; иначе — ошибка.
    \item Метод \texttt{check\_budget} принимает имя профиля и выводит баланс или ошибку.
    \item Создайте экземпляр \texttt{TravelWallet}.
    \item Создайте два профиля.
    \item Пополните один.
    \item Оплатите по другому.
    \item Проверьте балансы и несуществующий профиль.
\end{itemize}

\textbf{Пример использования:}

\begin{verbatim}
tw = TravelWallet()
tw.create_profile("ParisTrip", 3000)
tw.create_profile("TokyoTrip", 2500)
tw.load_funds("ParisTrip", 500)
tw.pay_expense("TokyoTrip", 700)
tw.check_budget("ParisTrip")
tw.check_budget("TokyoTrip")
tw.pay_expense("TokyoTrip", 2000)  # недостаточно
tw.check_budget("LondonTrip")  # несуществует
\end{verbatim}

\item[14] Написать программу на Python, которая создает класс \texttt{SchoolFund}, представляющий школьный фонд. Класс должен поддерживать создание счетов классов, внесение средств, расход и проверку баланса.

\begin{itemize}
    \item Создайте класс \texttt{SchoolFund} с методом \texttt{\_\_init\_\_}, создающим пустой словарь \texttt{classes}.
    \item Метод \texttt{add\_class} принимает номер класса и начальный бюджет. Если класс уже есть — ошибка; иначе — добавление.
    \item Метод \texttt{collect\_money} принимает номер класса и сумму. При наличии класса — пополнение; иначе — ошибка.
    \item Метод \texttt{spend\_money} принимает номер класса и сумму. При наличии класса и достаточном бюджете — списание; иначе — ошибка.
    \item Метод \texttt{get\_class\_balance} принимает номер класса и выводит баланс или ошибку.
    \item Создайте экземпляр \texttt{SchoolFund}.
    \item Добавьте два класса.
    \item Соберите средства у одного.
    \item Потратьте у другого.
    \item Проверьте балансы и несуществующий класс.
\end{itemize}

\textbf{Пример использования:}

\begin{verbatim}
sf = SchoolFund()
sf.add_class("10A", 1200)
sf.add_class("11B", 900)
sf.collect_money("10A", 300)
sf.spend_money("11B", 400)
sf.get_class_balance("10A")
sf.get_class_balance("11B")
sf.spend_money("11B", 600)  # недостаточно
sf.get_class_balance("12C")  # несуществует
\end{verbatim}

\item[15] Написать программу на Python, которая создает класс \texttt{ClubAccount}, моделирующий счет клуба. Класс должен поддерживать создание счетов участников, пополнение взносами, снятие на мероприятия и проверку баланса.

\begin{itemize}
    \item Создайте класс \texttt{ClubAccount} с методом \texttt{\_\_init\_\_}, инициализирующим пустой словарь \texttt{members}.
    \item Метод \texttt{join\_club} принимает ID участника и вступительный взнос. Если ID есть — ошибка; иначе — добавление.
    \item Метод \texttt{pay\_dues} принимает ID и сумму. При наличии участника — пополнение; иначе — ошибка.
    \item Метод \texttt{request\_funds} принимает ID и сумму. При наличии участника и достаточном балансе — списание; иначе — ошибка.
    \item Метод \texttt{check\_member\_balance} принимает ID и выводит баланс или ошибку.
    \item Создайте экземпляр \texttt{ClubAccount}.
    \item Зарегистрируйте двух участников.
    \item Внесите взносы за одного.
    \item Запросите средства у другого.
    \item Проверьте балансы и несуществующего участника.
\end{itemize}

\textbf{Пример использования:}

\begin{verbatim}
ca = ClubAccount()
ca.join_club("MEM-01", 500)
ca.join_club("MEM-02", 400)
ca.pay_dues("MEM-01", 100)
ca.request_funds("MEM-02", 150)
ca.check_member_balance("MEM-01")
ca.check_member_balance("MEM-02")
ca.request_funds("MEM-02", 300)  # недостаточно
ca.check_member_balance("MEM-99")  # несуществует
\end{verbatim}

\item[16] Написать программу на Python, которая создает класс \texttt{ProjectBudget}, представляющий бюджет проекта. Класс должен поддерживать создание проектов, выделение средств, расход и проверку остатка.

\begin{itemize}
    \item Создайте класс \texttt{ProjectBudget} с методом \texttt{\_\_init\_\_}, создающим пустой словарь \texttt{projects}.
    \item Метод \texttt{initiate\_project} принимает код проекта и начальный бюджет. Если проект существует — ошибка; иначе — инициализация.
    \item Метод \texttt{allocate\_funds} принимает код проекта и сумму. При наличии проекта — пополнение; иначе — ошибка.
    \item Метод \texttt{expend\_funds} принимает код проекта и сумму. При наличии проекта и достаточном бюджете — списание; иначе — ошибка.
    \item Метод \texttt{check\_project\_balance} принимает код проекта и выводит баланс или ошибку.
    \item Создайте экземпляр \texttt{ProjectBudget}.
    \item Инициируйте два проекта.
    \item Выделите средства одному.
    \item Потратьте у другого.
    \item Проверьте балансы и несуществующий проект.
\end{itemize}

\textbf{Пример использования:}

\begin{verbatim}
pb = ProjectBudget()
pb.initiate_project("PRJ-Alpha", 10000)
pb.initiate_project("PRJ-Beta", 8000)
pb.allocate_funds("PRJ-Alpha", 2000)
pb.expend_funds("PRJ-Beta", 3000)
pb.check_project_balance("PRJ-Alpha")
pb.check_project_balance("PRJ-Beta")
pb.expend_funds("PRJ-Beta", 6000)  # недостаточно
pb.check_project_balance("PRJ-Gamma")  # несуществует
\end{verbatim}

\item[17] Написать программу на Python, которая создает класс \texttt{EventFund}, моделирующий фонд мероприятия. Класс должен поддерживать создание событий, сбор средств, оплату расходов и проверку баланса.

\begin{itemize}
    \item Создайте класс \texttt{EventFund} с методом \texttt{\_\_init\_\_}, инициализирующим пустой словарь \texttt{events}.
    \item Метод \texttt{create\_event} принимает название события и стартовый бюджет. Если событие есть — ошибка; иначе — создание.
    \item Метод \texttt{collect\_sponsorship} принимает название и сумму. При наличии события — пополнение; иначе — ошибка.
    \item Метод \texttt{pay\_vendor} принимает название и сумму. При наличии события и достаточном бюджете — списание; иначе — ошибка.
    \item Метод \texttt{view\_event\_balance} принимает название и выводит баланс или ошибку.
    \item Создайте экземпляр \texttt{EventFund}.
    \item Создайте два события.
    \item Соберите спонсорские средства для одного.
    \item Оплатите поставщика для другого.
    \item Проверьте балансы и несуществующее событие.
\end{itemize}

\textbf{Пример использования:}

\begin{verbatim}
ef = EventFund()
ef.create_event("Conference", 5000)
ef.create_event("Workshop", 3000)
ef.collect_sponsorship("Conference", 1500)
ef.pay_vendor("Workshop", 1000)
ef.view_event_balance("Conference")
ef.view_event_balance("Workshop")
ef.pay_vendor("Workshop", 2500)  # недостаточно
ef.view_event_balance("Seminar")  # несуществует
\end{verbatim}

\item[18] Написать программу на Python, которая создает класс \texttt{PersonalFinance}, представляющий личные финансы. Класс должен поддерживать создание категорий, пополнение доходами, списание расходами и проверку баланса.

\begin{itemize}
    \item Создайте класс \texttt{PersonalFinance} с методом \texttt{\_\_init\_\_}, создающим пустой словарь \texttt{categories}.
    \item Метод \texttt{add\_category} принимает название категории и начальный баланс. Если категория есть — ошибка; иначе — добавление.
    \item Метод \texttt{record\_income} принимает название и сумму. При наличии категории — пополнение; иначе — ошибка.
    \item Метод \texttt{record\_expense} принимает название и сумму. При наличии категории и достаточном балансе — списание; иначе — ошибка.
    \item Метод \texttt{check\_category\_balance} принимает название и выводит баланс или ошибку.
    \item Создайте экземпляр \texttt{PersonalFinance}.
    \item Добавьте две категории.
    \item Запишите доход в одну.
    \item Запишите расход в другую.
    \item Проверьте балансы и несуществующую категорию.
\end{itemize}

\textbf{Пример использования:}

\begin{verbatim}
pf = PersonalFinance()
pf.add_category("Salary", 25000)
pf.add_category("Entertainment", 2000)
pf.record_income("Salary", 5000)
pf.record_expense("Entertainment", 800)
pf.check_category_balance("Salary")
pf.check_category_balance("Entertainment")
pf.record_expense("Entertainment", 1500)  # недостаточно
pf.check_category_balance("Travel")  # несуществует
\end{verbatim}

\item[19] Написать программу на Python, которая создает класс \texttt{InvestmentAccount}, моделирующий инвестиционный счет. Класс должен поддерживать создание счетов, внесение капитала, снятие прибыли и проверку баланса.

\begin{itemize}
    \item Создайте класс \texttt{InvestmentAccount} с методом \texttt{\_\_init\_\_}, инициализирующим пустой словарь \texttt{accounts}.
    \item Метод \texttt{open\_investment} принимает ID счета и начальный капитал. Если счет есть — ошибка; иначе — открытие.
    \item Метод \texttt{invest\_more} принимает ID и сумму. При наличии счета — пополнение; иначе — ошибка.
    \item Метод \texttt{withdraw\_profit} принимает ID и сумму. При наличии счета и достаточном балансе — списание; иначе — ошибка.
    \item Метод \texttt{check\_investment\_balance} принимает ID и выводит баланс или ошибку.
    \item Создайте экземпляр \texttt{InvestmentAccount}.
    \item Откройте два счета.
    \item Инвестируйте дополнительно в один.
    \item Снимите прибыль с другого.
    \item Проверьте балансы и несуществующий счет.
\end{itemize}

\textbf{Пример использования:}

\begin{verbatim}
ia = InvestmentAccount()
ia.open_investment("INV-01", 10000)
ia.open_investment("INV-02", 7000)
ia.invest_more("INV-01", 2000)
ia.withdraw_profit("INV-02", 1500)
ia.check_investment_balance("INV-01")
ia.check_investment_balance("INV-02")
ia.withdraw_profit("INV-02", 6000)  # недостаточно
ia.check_investment_balance("INV-99")  # несуществует
\end{verbatim}

\item[20] Написать программу на Python, которая создает класс \texttt{FamilyBudget}, представляющий семейный бюджет. Класс должен поддерживать создание членов семьи, пополнение общими доходами, списание личными расходами и проверку баланса.

\begin{itemize}
    \item Создайте класс \texttt{FamilyBudget} с методом \texttt{\_\_init\_\_}, создающим пустой словарь \texttt{members}.
    \item Метод \texttt{add\_family\_member} принимает имя и начальный вклад. Если имя есть — ошибка; иначе — добавление.
    \item Метод \texttt{add\_income} принимает имя и сумму. При наличии члена — пополнение; иначе — ошибка.
    \item Метод \texttt{deduct\_expense} принимает имя и сумму. При наличии члена и достаточном балансе — списание; иначе — ошибка.
    \item Метод \texttt{check\_member\_balance} принимает имя и выводит баланс или ошибку.
    \item Создайте экземпляр \texttt{FamilyBudget}.
    \item Добавьте двух членов семьи.
    \item Добавьте доход одному.
    \item Спишите расход у другого.
    \item Проверьте балансы и несуществующего члена.
\end{itemize}

\textbf{Пример использования:}

\begin{verbatim}
fb = FamilyBudget()
fb.add_family_member("Mother", 20000)
fb.add_family_member("Father", 25000)
fb.add_income("Mother", 5000)
fb.deduct_expense("Father", 3000)
fb.check_member_balance("Mother")
fb.check_member_balance("Father")
fb.deduct_expense("Father", 23000)  # недостаточно
fb.check_member_balance("Child")  # несуществует
\end{verbatim}

\item[21] Написать программу на Python, которая создает класс \texttt{StartupFund}, моделирующий фонд стартапа. Класс должен поддерживать создание стартапов, привлечение инвестиций, оплату расходов и проверку баланса.

\begin{itemize}
    \item Создайте класс \texttt{StartupFund} с методом \texttt{\_\_init\_\_}, инициализирующим пустой словарь \texttt{startups}.
    \item Метод \texttt{launch\_startup} принимает название и начальный капитал. Если стартап есть — ошибка; иначе — запуск.
    \item Метод \texttt{attract\_investment} принимает название и сумму. При наличии стартапа — пополнение; иначе — ошибка.
    \item Метод \texttt{cover\_costs} принимает название и сумму. При наличии стартапа и достаточном балансе — списание; иначе — ошибка.
    \item Метод \texttt{check\_startup\_balance} принимает название и выводит баланс или ошибку.
    \item Создайте экземпляр \texttt{StartupFund}.
    \item Запустите два стартапа.
    \item Привлеките инвестиции в один.
    \item Покройте расходы другого.
    \item Проверьте балансы и несуществующий стартап.
\end{itemize}

\textbf{Пример использования:}

\begin{verbatim}
sf = StartupFund()
sf.launch_startup("TechApp", 50000)
sf.launch_startup("EcoShop", 30000)
sf.attract_investment("TechApp", 20000)
sf.cover_costs("EcoShop", 10000)
sf.check_startup_balance("TechApp")
sf.check_startup_balance("EcoShop")
sf.cover_costs("EcoShop", 25000)  # недостаточно
sf.check_startup_balance("FoodDelivery")  # несуществует
\end{verbatim}

\item[22] Написать программу на Python, которая создает класс \texttt{NonProfitAccount}, представляющий счет некоммерческой организации. Класс должен поддерживать создание проектов, получение грантов, оплату деятельности и проверку баланса.

\begin{itemize}
    \item Создайте класс \texttt{NonProfitAccount} с методом \texttt{\_\_init\_\_}, создающим пустой словарь \texttt{projects}.
    \item Метод \texttt{initiate\_nonprofit\_project} принимает ID и начальный грант. Если проект есть — ошибка; иначе — инициализация.
    \item Метод \texttt{receive\_grant} принимает ID и сумму. При наличии проекта — пополнение; иначе — ошибка.
    \item Метод \texttt{pay\_operational\_costs} принимает ID и сумму. При наличии проекта и достаточном балансе — списание; иначе — ошибка.
    \item Метод \texttt{check\_project\_funds} принимает ID и выводит баланс или ошибку.
    \item Создайте экземпляр \texttt{NonProfitAccount}.
    \item Инициируйте два проекта.
    \item Получите грант на один.
    \item Оплатите расходы другого.
    \item Проверьте балансы и несуществующий проект.
\end{itemize}

\textbf{Пример использования:}

\begin{verbatim}
np = NonProfitAccount()
np.initiate_nonprofit_project("EDU-01", 15000)
np.initiate_nonprofit_project("HEALTH-02", 12000)
np.receive_grant("EDU-01", 5000)
np.pay_operational_costs("HEALTH-02", 4000)
np.check_project_funds("EDU-01")
np.check_project_funds("HEALTH-02")
np.pay_operational_costs("HEALTH-02", 9000)  # недостаточно
np.check_project_funds("ENV-99")  # несуществует
\end{verbatim}

\item[23] Написать программу на Python, которая создает класс \texttt{FreelancerWallet}, моделирующий кошелек фрилансера. Класс должен поддерживать создание профилей, получение оплаты, оплату налогов и проверку баланса.

\begin{itemize}
    \item Создайте класс \texttt{FreelancerWallet} с методом \texttt{\_\_init\_\_}, инициализирующим пустой словарь \texttt{freelancers}.
    \item Метод \texttt{register\_freelancer} принимает ник и начальный баланс. Если ник есть — ошибка; иначе — регистрация.
    \item Метод \texttt{receive\_payment} принимает ник и сумму. При наличии фрилансера — пополнение; иначе — ошибка.
    \item Метод \texttt{pay\_taxes} принимает ник и сумму. При наличии фрилансера и достаточном балансе — списание; иначе — ошибка.
    \item Метод \texttt{check\_freelancer\_balance} принимает ник и выводит баланс или ошибку.
    \item Создайте экземпляр \texttt{FreelancerWallet}.
    \item Зарегистрируйте двух фрилансеров.
    \item Получите оплату для одного.
    \item Оплатите налоги для другого.
    \item Проверьте балансы и несуществующего фрилансера.
\end{itemize}

\textbf{Пример использования:}

\begin{verbatim}
fw = FreelancerWallet()
fw.register_freelancer("dev_alex", 0)
fw.register_freelancer("design_maria", 0)
fw.receive_payment("dev_alex", 10000)
fw.receive_payment("design_maria", 2500)
fw.pay_taxes("design_maria", 2000)
fw.check_freelancer_balance("dev_alex")
fw.check_freelancer_balance("design_maria")
fw.pay_taxes("design_maria", 1000)  # недостаточно
fw.check_freelancer_balance("writer_john")  # несуществует
\end{verbatim}

\item[24] Написать программу на Python, которая создает класс \texttt{RentalIncome}, представляющий доход от аренды. Класс должен поддерживать создание объектов недвижимости, получение арендной платы, оплату расходов и проверку баланса.

\begin{itemize}
    \item Создайте класс \texttt{RentalIncome} с методом \texttt{\_\_init\_\_}, создающим пустой словарь \texttt{properties}.
    \item Метод \texttt{add\_property} принимает адрес и начальный баланс. Если адрес есть — ошибка; иначе — добавление.
    \item Метод \texttt{collect\_rent} принимает адрес и сумму. При наличии объекта — пополнение; иначе — ошибка.
    \item Метод \texttt{pay\_maintenance} принимает адрес и сумму. При наличии объекта и достаточном балансе — списание; иначе — ошибка.
    \item Метод \texttt{check\_property\_balance} принимает адрес и выводит баланс или ошибку.
    \item Создайте экземпляр \texttt{RentalIncome}.
    \item Добавьте два объекта.
    \item Соберите арендную плату с одного.
    \item Оплатите обслуживание другого.
    \item Проверьте балансы и несуществующий адрес.
\end{itemize}

\textbf{Пример использования:}

\begin{verbatim}
ri = RentalIncome()
ri.add_property("123 Main St", 0)
ri.add_property("456 Oak Ave", 0)
ri.collect_rent("123 Main St", 2000)
ri.collect_rent("456 Oak Ave", 700)
ri.pay_maintenance("456 Oak Ave", 300)
ri.check_property_balance("123 Main St")
ri.check_property_balance("456 Oak Ave")
ri.pay_maintenance("456 Oak Ave", 500)  # недостаточно
ri.check_property_balance("789 Pine Rd")  # несуществует
\end{verbatim}

\item[25] Написать программу на Python, которая создает класс \texttt{ScholarshipFund}, моделирующий стипендиальный фонд. Класс должен поддерживать создание получателей, выдачу стипендий, возврат средств и проверку баланса.

\begin{itemize}
    \item Создайте класс \texttt{ScholarshipFund} с методом \texttt{\_\_init\_\_}, инициализирующим пустой словарь \texttt{recipients}.
    \item Метод \texttt{enroll\_recipient} принимает ID и начальную стипендию. Если ID есть — ошибка; иначе — зачисление.
    \item Метод \texttt{award\_scholarship} принимает ID и сумму. При наличии получателя — пополнение; иначе — ошибка.
    \item Метод \texttt{return\_funds} принимает ID и сумму. При наличии получателя и достаточном балансе — списание; иначе — ошибка.
    \item Метод \texttt{check\_recipient\_balance} принимает ID и выводит баланс или ошибку.
    \item Создайте экземпляр \texttt{ScholarshipFund}.
    \item Зачислите двух получателей.
    \item Выдайте стипендию одному.
    \item Примите возврат от другого.
    \item Проверьте балансы и несуществующего получателя.
\end{itemize}

\textbf{Пример использования:}

\begin{verbatim}
sf = ScholarshipFund()
sf.enroll_recipient("SCH-01", 5000)
sf.enroll_recipient("SCH-02", 4000)
sf.award_scholarship("SCH-01", 1000)
sf.return_funds("SCH-02", 500)
sf.check_recipient_balance("SCH-01")
sf.check_recipient_balance("SCH-02")
sf.return_funds("SCH-02", 4000)  # недостаточно
sf.check_recipient_balance("SCH-99")  # несуществует
\end{verbatim}

\item[26] Написать программу на Python, которая создает класс \texttt{Crowdfunding}, представляющий краудфандинговую платформу. Класс должен поддерживать создание кампаний, сбор средств, возврат пожертвований и проверку баланса.

\begin{itemize}
    \item Создайте класс \texttt{Crowdfunding} с методом \texttt{\_\_init\_\_}, создающим пустой словарь \texttt{campaigns}.
    \item Метод \texttt{start\_campaign} принимает название и начальный баланс. Если кампания есть — ошибка; иначе — создание.
    \item Метод \texttt{donate} принимает название и сумму. При наличии кампании — пополнение; иначе — ошибка.
    \item Метод \texttt{refund} принимает название и сумму. При наличии кампании и достаточном балансе — списание; иначе — ошибка.
    \item Метод \texttt{check\_campaign\_balance} принимает название и выводит баланс или ошибку.
    \item Создайте экземпляр \texttt{Crowdfunding}.
    \item Запустите две кампании.
    \item Пожертвуйте в одну.
    \item Верните средства из другой.
    \item Проверьте балансы и несуществующую кампанию.
\end{itemize}

\textbf{Пример использования:}

\begin{verbatim}
cf = Crowdfunding()
cf.start_campaign("BookPublish", 10000)
cf.start_campaign("ArtExhibit", 8000)
cf.donate("BookPublish", 3000)
cf.refund("ArtExhibit", 500)
cf.check_campaign_balance("BookPublish")
cf.check_campaign_balance("ArtExhibit")
cf.refund("ArtExhibit", 8000)  # недостаточно
cf.check_campaign_balance("FilmProject")  # несуществует
\end{verbatim}

\item[27] Написать программу на Python, которая создает класс \texttt{PiggyBank}, моделирующий копилку. Класс должен поддерживать создание копилок, добавление монет, извлечение средств и проверку баланса.

\begin{itemize}
    \item Создайте класс \texttt{PiggyBank} с методом \texttt{\_\_init\_\_}, инициализирующим пустой словарь \texttt{banks}.
    \item Метод \texttt{create\_piggy} принимает имя и начальную сумму. Если имя есть — ошибка; иначе — создание.
    \item Метод \texttt{add\_coins} принимает имя и сумму. При наличии копилки — пополнение; иначе — ошибка.
    \item Метод \texttt{break\_piggy} принимает имя и сумму. При наличии копилки и достаточном балансе — списание; иначе — ошибка.
    \item Метод \texttt{check\_piggy\_balance} принимает имя и выводит баланс или ошибку.
    \item Создайте экземпляр \texttt{PiggyBank}.
    \item Создайте две копилки.
    \item Добавьте монеты в одну.
    \item Разбейте другую частично.
    \item Проверьте балансы и несуществующую копилку.
\end{itemize}

\textbf{Пример использования:}

\begin{verbatim}
pb = PiggyBank()
pb.create_piggy("Vacation", 200)
pb.create_piggy("Gadget", 150)
pb.add_coins("Vacation", 100)
pb.break_piggy("Gadget", 50)
pb.check_piggy_balance("Vacation")
pb.check_piggy_balance("Gadget")
pb.break_piggy("Gadget", 120)  # недостаточно
pb.check_piggy_balance("Car")  # несуществует
\end{verbatim}

\item[28] Написать программу на Python, которая создает класс \texttt{BusinessAccount}, представляющий бизнес-счет. Класс должен поддерживать создание компаний, зачисление выручки, оплату счетов и проверку баланса.

\begin{itemize}
    \item Создайте класс \texttt{BusinessAccount} с методом \texttt{\_\_init\_\_}, создающим пустой словарь \texttt{companies}.
    \item Метод \texttt{register\_business} принимает название и начальный капитал. Если компания есть — ошибка; иначе — регистрация.
    \item Метод \texttt{record\_revenue} принимает название и сумму. При наличии компании — пополнение; иначе — ошибка.
    \item Метод \texttt{pay\_bills} принимает название и сумму. При наличии компании и достаточном балансе — списание; иначе — ошибка.
    \item Метод \texttt{check\_business\_balance} принимает название и выводит баланс или ошибку.
    \item Создайте экземпляр \texttt{BusinessAccount}.
    \item Зарегистрируйте две компании.
    \item Запишите выручку одной.
    \item Оплатите счета другой.
    \item Проверьте балансы и несуществующую компанию.
\end{itemize}

\textbf{Пример использования:}

\begin{verbatim}
ba = BusinessAccount()
ba.register_business("TechCorp", 50000)
ba.register_business("CafeLtd", 20000)
ba.record_revenue("TechCorp", 15000)
ba.pay_bills("CafeLtd", 3000)
ba.check_business_balance("TechCorp")
ba.check_business_balance("CafeLtd")
ba.pay_bills("CafeLtd", 18000)  # недостаточно
ba.check_business_balance("ShopInc")  # несуществует
\end{verbatim}

\item[29] Написать программу на Python, которая создает класс \texttt{GrantManager}, моделирующий управление грантами. Класс должен поддерживать создание грантов, выделение средств, отчетность и проверку баланса.

\begin{itemize}
    \item Создайте класс \texttt{GrantManager} с методом \texttt{\_\_init\_\_}, инициализирующим пустой словарь \texttt{grants}.
    \item Метод \texttt{issue\_grant} принимает код и сумму. Если код есть — ошибка; иначе — создание.
    \item Метод \texttt{disburse\_funds} принимает код и сумму. При наличии гранта и достаточном балансе — списание (выдача средств); иначе — ошибка.
    \item Метод \texttt{submit\_report} принимает код и сумму. При наличии гранта и достаточном балансе — списание; иначе — ошибка.
    \item Метод \texttt{check\_grant\_status} принимает код и выводит баланс или ошибку.
    \item Создайте экземпляр \texttt{GrantManager}.
    \item Выдайте два гранта.
    \item Распределите средства по одному.
    \item Подайте отчет по другому.
    \item Проверьте статусы и несуществующий грант.
\end{itemize}

\textbf{Пример использования:}

\begin{verbatim}
gm = GrantManager()
gm.issue_grant("GR-2024-01", 10000)
gm.issue_grant("GR-2024-02", 8000)
gm.disburse_funds("GR-2024-01", 4000)
gm.submit_report("GR-2024-02", 2000)
gm.check_grant_status("GR-2024-01")
gm.check_grant_status("GR-2024-02")
gm.submit_report("GR-2024-02", 7000)  # недостаточно
gm.check_grant_status("GR-2024-99")  # несуществует
\end{verbatim}

\item[30] Написать программу на Python, которая создает класс \texttt{SubscriptionService}, представляющий сервис подписок. Класс должен поддерживать создание пользователей, оплату подписок, возврат средств и проверку баланса.

\begin{itemize}
    \item Создайте класс \texttt{SubscriptionService} с методом \texttt{\_\_init\_\_}, создающим пустой словарь \texttt{subscribers}.
    \item Метод \texttt{subscribe\_user} принимает email и начальный баланс. Если email есть — ошибка; иначе — подписка.
    \item Метод \texttt{charge\_payment} принимает email и сумму. При наличии пользователя — пополнение; иначе — ошибка.
    \item Метод \texttt{refund\_payment} принимает email и сумму. При наличии пользователя и достаточном балансе — списание; иначе — ошибка.
    \item Метод \texttt{check\_subscription\_balance} принимает email и выводит баланс или ошибку.
    \item Создайте экземпляр \texttt{SubscriptionService}.
    \item Подпишите двух пользователей.
    \item Спишите оплату с одного.
    \item Верните средства другому.
    \item Проверьте балансы и несуществующий email.
\end{itemize}

\textbf{Пример использования:}

\begin{verbatim}
ss = SubscriptionService()
ss.subscribe_user("user1@example.com", 100)
ss.subscribe_user("user2@example.com", 80)
ss.charge_payment("user1@example.com", 20)
ss.refund_payment("user2@example.com", 10)
ss.check_subscription_balance("user1@example.com")
ss.check_subscription_balance("user2@example.com")
ss.refund_payment("user2@example.com", 80)  # недостаточно
ss.check_subscription_balance("user3@example.com")  # несуществует
\end{verbatim}

\item[31] Написать программу на Python, которая создает класс \texttt{LoyaltyProgram}, моделирующий программу лояльности. Класс должен поддерживать создание участников, начисление баллов, списание за вознаграждения и проверку баланса.

\begin{itemize}
    \item Создайте класс \texttt{LoyaltyProgram} с методом \texttt{\_\_init\_\_}, инициализирующим пустой словарь \texttt{members}.
    \item Метод \texttt{enroll\_member} принимает ID и начальные баллы. Если ID есть — ошибка; иначе — зачисление.
    \item Метод \texttt{earn\_points} принимает ID и количество. При наличии участника — пополнение; иначе — ошибка.
    \item Метод \texttt{redeem\_points} принимает ID и количество. При наличии участника и достаточном балансе — списание; иначе — ошибка.
    \item Метод \texttt{check\_points\_balance} принимает ID и выводит баланс или ошибку.
    \item Создайте экземпляр \texttt{LoyaltyProgram}.
    \item Зачислите двух участников.
    \item Начислите баллы одному.
    \item Спишите у другого.
    \item Проверьте балансы и несуществующего участника.
\end{itemize}

\textbf{Пример использования:}

\begin{verbatim}
lp = LoyaltyProgram()
lp.enroll_member("MEM-101", 500)
lp.enroll_member("MEM-102", 300)
lp.earn_points("MEM-101", 200)
lp.redeem_points("MEM-102", 100)
lp.check_points_balance("MEM-101")
lp.check_points_balance("MEM-102")
lp.redeem_points("MEM-102", 250)  # недостаточно
lp.check_points_balance("MEM-999")  # несуществует
\end{verbatim}

\item[32] Написать программу на Python, которая создает класс \texttt{UtilityBill}, представляющий оплату коммунальных услуг. Класс должен поддерживать создание лицевых счетов, внесение платежей, списание задолженностей и проверку баланса.

\begin{itemize}
    \item Создайте класс \texttt{UtilityBill} с методом \texttt{\_\_init\_\_}, создающим пустой словарь \texttt{accounts}.
    \item Метод \texttt{create\_utility\_account} принимает номер и начальный долг. Если номер есть — ошибка; иначе — создание.
    \item Метод \texttt{make\_payment} принимает номер и сумму. При наличии счета — пополнение; иначе — ошибка.
    \item Метод \texttt{apply\_charges} принимает номер и сумму. При наличии счета и достаточном балансе — списание; иначе — ошибка.
    \item Метод \texttt{check\_account\_status} принимает номер и выводит баланс или ошибку.
    \item Создайте экземпляр \texttt{UtilityBill}.
    \item Создайте два счета.
    \item Внесите платеж по одному.
    \item Начислите плату по другому.
    \item Проверьте статусы и несуществующий счет.
\end{itemize}

\textbf{Пример использования:}

\begin{verbatim}
ub = UtilityBill()
ub.create_utility_account("UTIL-01", 0)
ub.create_utility_account("UTIL-02", 0)
ub.make_payment("UTIL-01", 1500)
ub.make_payment("UTIL-02", 1200)
ub.apply_charges("UTIL-02", 800)
ub.check_account_status("UTIL-01")
ub.check_account_status("UTIL-02")
ub.apply_charges("UTIL-02", 1000)  # недостаточно
ub.check_account_status("UTIL-99")  # несуществует
\end{verbatim}

\item[33] Написать программу на Python, которая создает класс \texttt{InsuranceFund}, моделирующий страховой фонд. Класс должен поддерживать создание полисов, уплату премий, выплату возмещений и проверку баланса.

\begin{itemize}
    \item Создайте класс \texttt{InsuranceFund} с методом \texttt{\_\_init\_\_}, инициализирующим пустой словарь \texttt{policies}.
    \item Метод \texttt{issue\_policy} принимает номер полиса и начальный взнос. Если полис есть — ошибка; иначе — выдача.
    \item Метод \texttt{pay\_premium} принимает номер и сумму. При наличии полиса — пополнение; иначе — ошибка.
    \item Метод \texttt{process\_claim} принимает номер и сумму. При наличии полиса и достаточном балансе — списание; иначе — ошибка.
    \item Метод \texttt{check\_policy\_balance} принимает номер и выводит баланс или ошибку.
    \item Создайте экземпляр \texttt{InsuranceFund}.
    \item Выдайте два полиса.
    \item Уплатите премию по одному.
    \item Обработайте заявку по другому.
    \item Проверьте балансы и несуществующий полис.
\end{itemize}

\textbf{Пример использования:}

\begin{verbatim}
ifund = InsuranceFund()
ifund.issue_policy("POL-501", 10000)
ifund.issue_policy("POL-502", 8000)
ifund.pay_premium("POL-501", 2000)
ifund.process_claim("POL-502", 3000)
ifund.check_policy_balance("POL-501")
ifund.check_policy_balance("POL-502")
ifund.process_claim("POL-502", 6000)  # недостаточно
ifund.check_policy_balance("POL-999")  # несуществует
\end{verbatim}

\item[34] Написать программу на Python, которая создает класс \texttt{DonationBox}, представляющий ящик для пожертвований. Класс должен поддерживать создание ящиков, сбор средств, выдачу помощи и проверку баланса.

\begin{itemize}
    \item Создайте класс \texttt{DonationBox} с методом \texttt{\_\_init\_\_}, создающим пустой словарь \texttt{boxes}.
    \item Метод \texttt{install\_box} принимает локацию и начальный сбор. Если локация есть — ошибка; иначе — установка.
    \item Метод \texttt{collect\_donations} принимает локацию и сумму. При наличии ящика — пополнение; иначе — ошибка.
    \item Метод \texttt{distribute\_aid} принимает локацию и сумму. При наличии ящика и достаточном балансе — списание; иначе — ошибка.
    \item Метод \texttt{check\_box\_balance} принимает локацию и выводит баланс или ошибку.
    \item Создайте экземпляр \texttt{DonationBox}.
    \item Установите два ящика.
    \item Соберите пожертвования в один.
    \item Распределите помощь из другого.
    \item Проверьте балансы и несуществующую локацию.
\end{itemize}

\textbf{Пример использования:}

\begin{verbatim}
db = DonationBox()
db.install_box("Hospital", 0)
db.install_box("School", 0)
db.collect_donations("Hospital", 5000)
db.collect_donations("School", 1500)
db.distribute_aid("School", 1000)
db.check_box_balance("Hospital")
db.check_box_balance("School")
db.distribute_aid("School", 2000)  # недостаточно
db.check_box_balance("Park")  # несуществует
\end{verbatim}

\item[35] Написать программу на Python, которая создает класс \texttt{RewardWallet}, моделирующий кошелек вознаграждений. Класс должен поддерживать создание кошельков, начисление бонусов, списание за покупки и проверку баланса.

\begin{itemize}
    \item Создайте класс \texttt{RewardWallet} с методом \texttt{\_\_init\_\_}, инициализирующим пустой словарь \texttt{wallets}.
    \item Метод \texttt{activate\_wallet} принимает ID и начальные бонусы. Если ID есть — ошибка; иначе — активация.
    \item Метод \texttt{award\_bonus} принимает ID и сумму. При наличии кошелька — пополнение; иначе — ошибка.
    \item Метод \texttt{redeem\_reward} принимает ID и сумму. При наличии кошелька и достаточном балансе — списание; иначе — ошибка.
    \item Метод \texttt{check\_reward\_balance} принимает ID и выводит баланс или ошибку.
    \item Создайте экземпляр \texttt{RewardWallet}.
    \item Активируйте два кошелька.
    \item Начислите бонусы одному.
    \item Потратьте у другого.
    \item Проверьте балансы и несуществующий ID.
\end{itemize}

\textbf{Пример использования:}

\begin{verbatim}
rw = RewardWallet()
rw.activate_wallet("RW-001", 1000)
rw.activate_wallet("RW-002", 800)
rw.award_bonus("RW-001", 200)
rw.redeem_reward("RW-002", 300)
rw.check_reward_balance("RW-001")
rw.check_reward_balance("RW-002")
rw.redeem_reward("RW-002", 600)  # недостаточно
rw.check_reward_balance("RW-999")  # несуществует
\end{verbatim}
\end{enumerate}



\end{document}

