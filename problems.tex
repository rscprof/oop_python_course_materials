\documentclass{article}
\usepackage[utf8]{inputenc}
\usepackage[russian]{babel}
\usepackage[T2A]{fontenc}
\usepackage{hyperref}
\usepackage{amssymb}
\usepackage[a4paper]{geometry}
\usepackage{listingsutf8}
\usepackage{tikz}
\lstset{
    inputencoding=utf8,
    language=Python,
    basicstyle=\ttfamily\footnotesize,
    breaklines=true,
    keepspaces=true,
    showstringspaces=false,
	literate={а}{{\selectfont\char224}}1
             {б}{{\selectfont\char225}}1
             {в}{{\selectfont\char226}}1
             {г}{{\selectfont\char227}}1
             {д}{{\selectfont\char228}}1
             {е}{{\selectfont\char229}}1
             {ё}{{\"e}}1
             {ж}{{\selectfont\char230}}1
             {з}{{\selectfont\char231}}1
             {и}{{\selectfont\char232}}1
             {й}{{\selectfont\char233}}1
             {к}{{\selectfont\char234}}1
             {л}{{\selectfont\char235}}1
             {м}{{\selectfont\char236}}1
             {н}{{\selectfont\char237}}1
             {о}{{\selectfont\char238}}1
             {п}{{\selectfont\char239}}1
             {р}{{\selectfont\char240}}1
             {с}{{\selectfont\char241}}1
             {т}{{\selectfont\char242}}1
             {у}{{\selectfont\char243}}1
             {ф}{{\selectfont\char244}}1
             {х}{{\selectfont\char245}}1
             {ц}{{\selectfont\char246}}1
             {ч}{{\selectfont\char247}}1
             {ш}{{\selectfont\char248}}1
             {щ}{{\selectfont\char249}}1
             {ъ}{{\selectfont\char250}}1
             {ы}{{\selectfont\char251}}1
             {ь}{{\selectfont\char252}}1
             {э}{{\selectfont\char253}}1
             {ю}{{\selectfont\char254}}1
             {я}{{\selectfont\char255}}1
             {А}{{\selectfont\char192}}1
             {Б}{{\selectfont\char193}}1
             {В}{{\selectfont\char194}}1
             {Г}{{\selectfont\char195}}1
             {Д}{{\selectfont\char196}}1
             {Е}{{\selectfont\char197}}1
             {Ё}{{\"E}}1
             {Ж}{{\selectfont\char198}}1
             {З}{{\selectfont\char199}}1
             {И}{{\selectfont\char200}}1
             {Й}{{\selectfont\char201}}1
             {К}{{\selectfont\char202}}1
             {Л}{{\selectfont\char203}}1
             {М}{{\selectfont\char204}}1
             {Н}{{\selectfont\char205}}1
             {О}{{\selectfont\char206}}1
             {П}{{\selectfont\char207}}1
             {Р}{{\selectfont\char208}}1
             {С}{{\selectfont\char209}}1
             {Т}{{\selectfont\char210}}1
             {У}{{\selectfont\char211}}1
             {Ф}{{\selectfont\char212}}1
             {Х}{{\selectfont\char213}}1
             {Ц}{{\selectfont\char214}}1
             {Ч}{{\selectfont\char215}}1
             {Ш}{{\selectfont\char216}}1
             {Щ}{{\selectfont\char217}}1
             {Ъ}{{\selectfont\char218}}1
             {Ы}{{\selectfont\char219}}1
             {Ь}{{\selectfont\char220}}1
             {Э}{{\selectfont\char221}}1
             {Ю}{{\selectfont\char222}}1
             {Я}{{\selectfont\char223}}1
}

\begin{document}
{
	\thispagestyle{empty}
	\vspace*{\fill}
	\centering 

	\Large Сборник заданий для семинарских занятий \\
	по курсу \\
	<<Объектно-ориентированное программирование на Python>>\\
	\large
	\vspace{40pt}
	\vspace*{\fill}
	\newpage
}

\tableofcontents
\newpage
\section{Общие сведения}

Сборник содержит задания для семинарских занятий 
по курсу <<Объектно-ориентированное программирование на Python>> 
(32 часа).

Задачник находится в процессе наполнения и новые задания 
появляются перед проведением нового семинара.

Возможна сдача другого кода (например, выполненного в ходе
проектной деятельности), еслои они полностью покрывают материал семинара.

\section{Задания}

\subsection{Семинар <<Правила формирования класса для программирования в IDE PyCharm. Отработка навыков создания простых классов и объектов класса>> 
(2 очных часа)}

В ходе работы создайте 5 классов с соответствующими методами, описанными в индивидуальном задании. 
Предполагается, что пользователь класса не имеет права обращаться к свойствам напрямую 
(соблюдая принцип инкапсуляции), а должен использовать методы. Важно: в задании не всегда указаны 
все необходимые методы и свойства, при необходимости вам надо самостоятельно их добавить.
Продемонстрируйте работоспособность всех методов (из задания) посредством создания запускаемых файлов, где осуществляется вызов методов для разных ситуаций (без ручного ввода, но с выводом результатов в консоль). 
Каждый класс должен сохраняться в отдельном исходном файле. Необходимо соблюдать все стандартные требования к качеству кода (отступы, именования переменных, классов, методов, проверка корректности входных данных).
Для каждого класса создайте отдельный запускаемый файл для проверки всех его методов 
(допускается использование других классов в этих тестах).

Все предлагаемые классы в заданиях упрощенные; для использования в production-окружении они требуют серьезной доработки. Суть задания — в отработке базовых навыков, а не в идеальном моделировании предложенных ситуаций.

Для сдачи работы будьте готовы пояснить или аналогично заданию модифицировать любую часть кода, а также ответить на вопросы:
\begin{enumerate}
    \item Кратко опишите парадигму объектно-ориентированного программирования (ООП).
    \item Что такое класс в парадигме ООП?
    \item Что такое объект (экземпляр) в парадигме ООП?
    \item Что обозначает свойство инкапсуляции в парадигме ООП?
    \item Синтаксис классов в Python (в рамках выполненной работы), создание и работа с объектами в Python.
\end{enumerate}

При выполнении задания предполагается самое простое базовое описание классов, соответствующее следующему 
примеру (вы можете использовать то, что вы ЗНАЕТЕ дополнительно, но это остается на ваше усмотрение):

Если вы нашли в задачнике ошибки, опечатки и другие недостатки, то вы можете сделать pull-request. 

\begin{lstlisting}
class Worker:
    def set_last_name(self, last_name):
        self.last_name = last_name

    def print_last_name(self):
        print (f"Фамилия: {self.last_name}")

    def get_last_name(self):
        return last_name

worker = Worker()
worker.set_last_name(self,"Иванов")
worker.print_last_name()
print(worker.get_last_name())
\end{lstlisting}

\textbf{Срок сдачи работы (начала сдачи):} следующее занятие после его выдачи. В последующие сроки оценка будет снижаться (при отсутствии оправдывающих документов).

\begin{enumerate}

\item
\textbf{Описание ситуации:}
Рассмотрим работу грузовой железнодорожной станции. На станции есть несколько путей, по которым поезда могут прибывать и отправляться. Каждый путь имеет свой номер и может вместить несколько поездов. Поезда формируются из вагонов, каждый из которых может перевозить разные грузы. Работники станции отвечают за диспетчерское управление маневровыми локомотивами, осмотр вагонов, выполнение погрузочно-разгрузочных работ, прием груза к перевозке, ремонт путей, обеспечение безопасности и т.п. Они используют радиостанции для связи друг с другом и для отслеживания положения поездов и передвижения вагонов.

\textbf{Создаваемые классы:} `Путь`, `Поезд`, `Вагон`, `Станция`, `РаботникСтанции`.

Для классов реализовать следующие простые методы (ниже приведен не исчерпывающий список методов; для демонстрации работы классов вам потребуются дополнительные методы, позволяющие отследить состояние объектов), используя для хранения данных списки (`[]`) Python:
\begin{enumerate}
    \item \textbf{Путь:} добавить поезд на путь, убрать поезд с пути, получить список поездов на конкретном пути.
    \item \textbf{Поезд:} прицепить вагон к поезду, отцепить вагон от поезда, получить (распечатать) список вагонов в поезде, вывести информацию о грузе в поезде.
    \item \textbf{Вагон:} добавить номер поезда, в который включался конкретный вагон, удалить номер поезда из истории, отобразить историю поездов для конкретного вагона.
    \item \textbf{РаботникСтанции:} класс, представляющий отдельного работника на станции, имеющий идентификатор, информацию о персональной радиостанции, список закрепленных за ним поездов для осмотра, ФИО, должность.
    \item \textbf{Станция:} добавить станционный путь, добавить поезд на станцию, нанять работника станции, вывести информацию о всех путях, поездах, работниках, удалить путь, удалить поезд, уволить работника.
\end{enumerate}

\item
\textbf{Описание ситуации:}
Рассмотрим работу крупного логистического терминала для обработки грузовых автомобилей. На терминале есть несколько доков (рамп), куда фуры прибывают для проведения погрузочно-разгрузочных работ. Каждый док имеет свой номер и может одновременно обслуживать одну машину. Грузовики перевозят паллеты, каждая из которых содержит определенный товар. Сотрудники терминала отвечают за прием грузовиков, управление погрузочной техникой, проверку сопроводительных документов, приемку и отгрузку товара, а также техническое обслуживание доков. Они используют портативные рации для координации действий и отслеживания статуса обработки автомобилей.

\textbf{Создаваемые классы:} `Док`, `Грузовик`, `Паллета`, `Терминал`, `Сотрудник`.

Для классов реализовать следующие простые методы, используя для хранения данных списки (`[]`) Python:
\begin{enumerate}
    \item \textbf{Док:} занять док конкретным грузовиком, освободить док, получить информацию о грузовике, который сейчас находится на доке.
    \item \textbf{Грузовик:} добавить паллету в грузовик, выгрузить паллету из грузовика, получить (распечатать) список паллет в грузовике, вывести информацию о товарах в грузовике.
    \item \textbf{Паллета:} добавить номер грузовика, в который загружалась конкретная паллета, удалить номер грузовика из истории, отобразить историю перевозок (номера грузовиков) для конкретной паллеты.
    \item \textbf{Сотрудник:} класс, представляющий отдельного сотрудника терминала, имеющий идентификатор, номер рации, список доков, за которые он отвечает, ФИО, должность.
    \item \textbf{Терминал:} добавить новый док на терминале, зарегистрировать прибытие грузовика, нанять нового сотрудника, вывести список всех доков, грузовиков на территории, сотрудников, удалить док, удалить грузовик, уволить сотрудника.
\end{enumerate}

\item
\textbf{Описание ситуации:}
Рассмотрим работу аэропорта. В аэропорту есть несколько взлетно-посадочных полос (ВПП), которые принимают и отправляют рейсы. Каждая ВПП имеет свой номер, длину и статус доступности. Самолеты перевозят пассажиров и их ручную кладь, размещенную в салоне. Авиадиспетчеры управляют движением самолетов, назначают полосы для взлета и посадки, следят за воздушной обстановкой и координируют действия с помощью радиосвязи.

\textbf{Создаваемые классы:} `ВПП`, `Самолет`, `Пассажир`, `Аэропорт`, `Авиадиспетчер`.

Для классов реализовать следующие простые методы, используя для хранения данных списки (`[]`) Python:
\begin{enumerate}
    \item \textbf{ВПП:} занять полосу для взлета/посадки, освободить полосу, получить список рейсов, использовавших полосу.
    \item \textbf{Самолет:} добавить пассажира на борт (включая вес его ручной клади), высадить пассажира, получить (распечатать) список пассажиров на борту, рассчитать общий вес ручной клади.
    \item \textbf{Пассажир:} добавить рейс в историю перелетов пассажира, удалить рейс из истории (ошибка бронирования), отобразить всю историю перелетов.
    \item \textbf{Авиадиспетчер:} класс, представляющий диспетчера, имеющий идентификатор, рабочую частоту, график работы (список интервалов времени в сутках), ФИО.
    \item \textbf{Аэропорт:} добавить новую ВПП, зарегистрировать прибытие самолета, нанять диспетчера, вывести список всех ВПП, самолетов в аэропорту, диспетчеров, удалить ВПП (на ремонт), списать самолет, уволить диспетчера.
\end{enumerate}

\item
\textbf{Описание ситуации:}
Рассмотрим работу речного порта. В порту есть несколько причалов для швартовки грузовых барж и буксиров. Каждый причал имеет уникальный номер и максимальную глубину, определяющую осадку судов, которые могут к нему подойти. Баржи перевозят контейнеры с различными грузами. 
Их характеризуют вес судна, максимальная грузоподъемность и осадка (как без груза, так и с максимальным грузом). Портовые рабочие отвечают за швартовку судов, управление портовыми кранами для погрузки/разгрузки контейнеров, оформление документов и поддержание порядка на территории.

\textbf{Создаваемые классы:} `Причал`, `Баржа`, `Контейнер`, `Порт`, `ПортовыйРабочий`.

Для классов реализовать следующие простые методы, используя для хранения данных списки (`[]`) Python:
\begin{enumerate}
    \item \textbf{Причал:} пришвартовать баржу к причалу, отшвартовать баржу, получить список барж, находящихся у причала.
    \item \textbf{Баржа:} загрузить контейнер на баржу (с указанием веса контейнера), разгрузить контейнер с баржи,
     получить (распечатать) список контейнеров на барже, рассчитать текущую осадку судна 
     (предполагается линейная зависимость осадки от суммарного веса груза и баржи).
    \item \textbf{Контейнер:} добавить номер баржи, на которую погрузили контейнер, удалить номер баржи, отобразить историю перемещений контейнера между баржами.
    \item \textbf{ПортовыйРабочий:} класс, представляющий рабочего, имеющий идентификатор, допуск к работе с краном, список закрепленных причалов, ФИО, должность.
    \item \textbf{Порт:} ввести новый причал в эксплуатацию, принять баржу в акваторию порта, принять на работу рабочего, вывести список причалов, барж в акватории, рабочих, списать причал, отправить баржу, уволить рабочего.
\end{enumerate}

\item
\textbf{Описание ситуации:}
Рассмотрим работу автобусного парка. В парке есть несколько маршрутов, которые обслуживаются автобусами. Каждый маршрут имеет номер и список остановок. Автобусы имеют государственный номер, количество мест и текущий пробег. Водители закреплены за автобусами и маршрутами. Диспетчеры автопарка составляют расписание, следят за выходами автобусов на линию, учетом пробега и техническим состоянием.

\textbf{Создаваемые классы:} `Маршрут`, `Автобус`, `Остановка`, `Автопарк`, `Водитель`.

Для классов реализовать следующие простые методы, используя для хранения данных списки (`[]`) Python:
\begin{enumerate}
    \item \textbf{Маршрут:} добавить остановку в маршрут, удалить остановку из маршрута, получить список всех остановок на маршруте.
    \item \textbf{Автобус:} назначить автобус на маршрут, снять с маршрута, увеличить пробег на заданное значение, получить текущий пробег.
    \item \textbf{Остановка:} добавить маршрут, проходящий через остановку, удалить маршрут, отобразить список всех маршрутов, проходящих через данную остановку.
    \item \textbf{Водитель:} класс, представляющий водителя, имеющий идентификатор, права категории, закрепленный автобус, ФИО, график работы.
    \item \textbf{Автопарк:} добавить новый маршрут, приобрести новый автобус, принять на работу водителя, вывести список маршрутов, автобусов (с указанием их состояния), водителей, списать автобус, уволить водителя.
\end{enumerate}

\item
\textbf{Описание ситуации:}
Рассмотрим работу метрополитена. В метро есть линии, состоящие из станций и тоннелей между ними. Составы из вагонов перемещаются по линиям. Каждая станция имеет название и может быть точкой пересадки на другие линии. Машинисты управляют поездами. Дежурные по станции следят за порядком на платформах и работой оборудования. Управление метрополитеном координирует движение составов.

\textbf{Создаваемые классы:} `ЛинияМетро`, `ПоездМетро`, `Станция`, `УправлениеМетрополитеном`, `Машинист`.

Для классов реализовать следующие простые методы, используя для хранения данных списки (`[]`) Python:
\begin{enumerate}
    \item \textbf{ЛинияМетро:} добавить станцию на линию, получить список станций на линии, получить список поездов на линии.
    \item \textbf{ПоездМетро:} добавить вагон в состав, отцепить вагон, назначить машиниста на поезд.
    \item \textbf{Станция:} добавить линию, проходящую через станцию (для моделирования пересадочных узлов), получить список линий на станции.
    \item \textbf{Машинист:} класс, представляющий машиниста, имеющий идентификатор, допуск к управлению, закрепленный поезд, ФИО, стаж.
    \item \textbf{УправлениеМетрополитеном:} открыть новую линию, ввести новый поезд в эксплуатацию, принять на работу машиниста, вывести список линий, поездов (в депо и на линиях), машинистов, закрыть линию на техобслуживание, списать поезд, вывести полную схему метро (в текстовом виде).
\end{enumerate}

\item
\textbf{Описание ситуации:}
Рассмотрим работу службы доставки пиццы. В службе есть несколько филиалов. Каждый филиал обслуживает определенный район и имеет курьеров. Заказы формируются из позиций меню. Курьеры используют скутеры для доставки. Менеджеры филиалов принимают заказы, назначают курьеров и следят за выполнением заказов.

\textbf{Создаваемые классы:} `Филиал`, `Заказ`, `Курьер`, `Скутер`, `Менеджер`.

Для классов реализовать следующие простые методы, используя для хранения данных списки (`[]`) Python:
\begin{enumerate}
    \item \textbf{Филиал:} добавить курьера в филиал, уволить курьера, получить список активных заказов филиала.
    \item \textbf{Заказ:} добавить позицию в заказ (название + цена), удалить позицию, рассчитать стоимость заказа, изменить статус заказа (принят, готовится, в пути, доставлен).
    \item \textbf{Курьер:} назначить заказ курьеру, завершить доставку заказа, получить список доставленных заказов за смену, закрепить скутер за курьером.
    \item \textbf{Менеджер:} класс, представляющий менеджера, имеющий идентификатор, закрепленный филиал, ФИО, смену.
    \item \textbf{Скутер:} отправить на зарядку, вернуть в строй, увеличить пробег, получить текущий пробег.
\end{enumerate}

\textbf{Описание ситуации:}
Рассмотрим работу трамвайного депо. В депо есть несколько маршрутов, обслуживаемых трамвайными вагонами. Каждый трамвайный вагон имеет бортовой номер, вместимость и текущий пробег. Маршруты состоят из остановок и имеют определенный график движения. Водители трамваев закреплены за конкретными вагонами и маршрутами. Диспетчеры управляют выпуском трамваев на линию и ведут учет технического состояния.

\textbf{Создаваемые классы:} Маршрут, Трамвай, Остановка, Депо, Водитель.

Для классов реализовать следующие простые методы, используя для хранения данных списки ([]) Python:
\begin{enumerate}
\item \textbf{Маршрут:} добавить остановку в маршрут, удалить остановку из маршрута, получить список всех остановок на маршруте.
\item \textbf{Трамвай:} назначить трамвай на маршрут, снять с маршрута, увеличить пробег на заданное значение, получить текущий пробег.
\item \textbf{Остановка:} добавить маршрут, проходящий через остановку, удалить маршрут, отобразить список всех маршрутов, проходящих через данную остановку.
\item \textbf{Водитель:} класс, представляющий водителя, имеющий идентификатор, права категории, закрепленный трамвай, ФИО, график работы.
\item \textbf{Депо:} добавить новый маршрут, принять новый трамвай в депо, принять на работу водителя, выполнить вывод списка маршрутов, трамваев (с указанием их состояния), водителей, списать трамвай, уволить водителя.
\end{enumerate}

\item
\textbf{Описание ситуации:}
Рассмотрим работу морского порта для приёма пассажирских паромов. 
В порту есть несколько причалов, каждый из которых обслуживает один паром за раз. 
Паромы перевозят пассажиров и автомобили. 
Пассажиры покупают билеты, автомобили записываются в список грузовой палубы. 
Сотрудники порта координируют погрузку, проверку билетов и безопасность.
\textbf{Создаваемые классы:} Причал, Паром, Пассажир, Автомобиль, СотрудникПорта.
\begin{enumerate}
    \item \textbf{Причал:} пришвартовать паром, освободить причал, получить информацию о пароме у причала.
    \item \textbf{Паром:} добавить пассажира, добавить автомобиль, высадить пассажира, выгрузить автомобиль.
    \item \textbf{Пассажир:} добавить рейс в историю поездок, удалить рейс из истории, 
    вывести историю поездок.
    \item \textbf{Автомобиль:} зарегистрировать номер парома, удалить номер парома, вывести историю перевозок.
    \item \textbf{СотрудникПорта:} идентификатор, должность, ФИО, список закреплённых причалов.
\end{enumerate}

\item
\textbf{Описание ситуации:}
Рассмотрим работу пригородной электрички. В системе есть станции, 
между которыми курсируют электрички. У каждой электрички есть номер, 
список вагонов и машинист. Пассажиры покупают билеты и 
занимают места в вагонах. Диспетчеры контролируют движение электричек.
\textbf{Создаваемые классы:} Станция, Электричка, Вагон, Пассажир, Диспетчер.
\begin{enumerate}
    \item \textbf{Станция:} принять электричку, отправить электричку, вывести список электричек на станции.
    \item \textbf{Электричка:} добавить вагон, отцепить вагон, получить список вагонов.
    \item \textbf{Вагон:} посадить пассажира, высадить пассажира, вывести список пассажиров.
    \item \textbf{Пассажир:} добавить поездку в историю, удалить поездку, показать историю поездок.
    \item \textbf{Диспетчер:} идентификатор, ФИО, рабочая смена, список контролируемых электричек.
\end{enumerate}

\item
\textbf{Описание ситуации:}
Рассмотрим работу таксопарка. В таксопарке есть автомобили, 
водители и диспетчеры. Автомобиль закрепляется за водителем. 
Диспетчеры принимают заказы и назначают их водителям. Пассажиры совершают поездки.
\textbf{Создаваемые классы:} Таксопарк, Автомобиль, Водитель, Заказ, Диспетчер.
\begin{enumerate}
    \item \textbf{Таксопарк:} добавить автомобиль, принять водителя, вывести список машин и водителей, 
    уволить водителя.
    \item \textbf{Автомобиль:} назначить водителя, снять водителя, увеличить пробег, получить пробег.
    \item \textbf{Водитель:} назначить заказ, завершить заказ, 
    вывести список выполненных заказов.
    \item \textbf{Заказ:} назначить пассажира, завершить поездку, вывести информацию о заказе.
    \item \textbf{Диспетчер:} идентификатор, ФИО, список назначенных заказов.
\end{enumerate}

\item
\textbf{Описание ситуации:}
Рассмотрим работу грузового аэропорта. Самолёты перевозят контейнеры. 
В аэропорту есть ангары для хранения самолётов и площадки для погрузки. 
Работники аэропорта координируют загрузку и выгрузку контейнеров.
\textbf{Создаваемые классы:} Самолёт, Контейнер, Ангар, РаботникАэропорта, Аэропорт.
\begin{enumerate}
    \item \textbf{Самолёт:} загрузить контейнер, выгрузить контейнер, вывести список контейнеров.
    \item \textbf{Контейнер:} добавить номер самолёта, удалить номер самолёта, вывести историю перевозок.
    \item \textbf{Ангар:} принять самолёт, вывести список самолётов, освободить ангар.
    \item \textbf{РаботникАэропорта:} идентификатор, ФИО, должность, список самолётов в обслуживании.
    \item \textbf{Аэропорт:} принять самолёт, убрать самолёт, принять раотника, уволить работника, 
    вывести список самолётов и работников.
\end{enumerate}

\item
\textbf{Описание ситуации:}
Рассмотрим работу велопроката. В прокате есть велосипеды, станции для их хранения, 
клиенты и сотрудники. Клиенты арендуют велосипеды и возвращают их на станцию.
\textbf{Создаваемые классы:} Велосипед, СтанцияПроката, Клиент, Сотрудник, Прокат.
\begin{enumerate}
    \item \textbf{Велосипед:} выдать в аренду, вернуть на станцию, получить пробег.
    \item \textbf{СтанцияПроката:} добавить велосипед, убрать велосипед, вывести список велосипедов.
    \item \textbf{Клиент:} арендовать велосипед, вернуть велосипед, вывести историю аренд.
    \item \textbf{Сотрудник:} идентификатор, ФИО, должность, список закреплённых станций.
    \item \textbf{Прокат:} добавить станцию, демонтировать станцию, 
    вывести список станций и велосипедов, уволить сотрудника, нанять сотрудника, вывести список сотрудников.
\end{enumerate}

\item
\textbf{Описание ситуации:}
Рассмотрим работу речных теплоходов. У каждого теплохода есть рейсы и список пассажиров. 
Пассажиры покупают билеты. Работники пристани обслуживают теплоходы.
\textbf{Создаваемые классы:} Теплоход, Рейс, Пассажир, Пристань, РаботникПристани.
\begin{enumerate}
    \item \textbf{Теплоход:} добавить рейс, убрать рейс, вывести список рейсов.
    \item \textbf{Рейс:} добавить пассажира, удалить пассажира, вывести список пассажиров.
    \item \textbf{Пассажир:} добавить рейс в историю, удалить рейс, вывести историю.
    \item \textbf{Пристань:} принять теплоход, отправить теплоход, вывести список теплоходов.
    \item \textbf{РаботникПристани:} идентификатор, ФИО, должность, закреплённые рейсы.
\end{enumerate}

\item
\textbf{Описание ситуации:}
Рассмотрим работу каршеринга. В системе есть автомобили, клиенты и диспетчеры. 
Автомобили бронируются клиентами и возвращаются после поездки. Диспетчеры контролируют состояние машин.
\textbf{Создаваемые классы:} Автомобиль, Клиент, Диспетчер, Заказ, Каршеринг.
\begin{enumerate}
    \item \textbf{Автомобиль:} выдать клиенту, вернуть, увеличить пробег, вывести пробег.
    \item \textbf{Клиент:} арендовать автомобиль, завершить аренду, вывести историю аренд.
    \item \textbf{Диспетчер:} идентификатор, ФИО, список автомобилей под контролем.
    \item \textbf{Заказ:} назначить автомобиль, завершить поездку, вывести данные заказа.
    \item \textbf{Каршеринг:} добавить автомобиль, списать автомобиль, добавить клиента, удалить клиента, 
    добавить диспетчера, удалить диспетчера,
    вывести список клиентов, диспетчеров и машин.
\end{enumerate}

\item
\textbf{Описание ситуации:}
Рассмотрим работу железнодорожного музея. 
В музее есть экспонаты (локомотивы и вагоны), 
экскурсии и экскурсоводы. Посетители записываются на экскурсии.
\textbf{Создаваемые классы:} Экспонат, Экскурсия, Экскурсовод, Посетитель, Музей.
\begin{enumerate}
    \item \textbf{Экспонат:} добавить к экскурсии, убрать, вывести список экскурсий.
    \item \textbf{Экскурсия:} записать посетителя, удалить, вывести список посетителей.
    \item \textbf{Экскурсовод:} идентификатор, ФИО, список экскурсий.
    \item \textbf{Посетитель:} записаться на экскурсию, отменить запись, вывести историю.
    \item \textbf{Музей:} добавить экспонат, списать экспонат, добавить экскурсовода, уволить экскурсовода, 
    провести экскурсию, вывести список всех экскурсий и экскурсоводов.
\end{enumerate}

\item
\textbf{Описание ситуации:}
Рассмотрим работу автозаправочной станции. На станции есть топливо, 
колонки и операторы. Автомобили приезжают заправляться.
\textbf{Создаваемые классы:} Колонка, Автомобиль, Оператор, Топливо, АЗС.
\begin{enumerate}
    \item \textbf{Колонка:} заправить автомобиль, освободить колонку, вывести статус.
    \item \textbf{Автомобиль:} получить заправку, вывести историю заправок.
    \item \textbf{Оператор:} идентификатор, ФИО, список закреплённых колонок.
    \item \textbf{Топливо:} уменьшить количество, увеличить количество, вывести остаток.
    \item \textbf{АЗС:} добавить колонку, нанять оператора, уволить оператора, демонтировать колонку, 
    вывести список машин, операторов и колонок.
\end{enumerate}


\item \textbf{Описание ситуации:} Рассмотрим работу сортировочного центра курьерской службы. 
В центре есть зоны обработки посылок, конвейерные линии и сотрудники. 
Каждая посылка имеет трек-номер и проходит через несколько этапов обработки. 
Сотрудники сканируют посылки, сортируют их по направлениям и загружают в 
транспортировочные контейнеры. Менеджеры контролируют процесс сортировки и работу оборудования.

\textbf{Создаваемые классы:} `ЗонаОбработки`, `Посылка`, `Конвейер`, `СотрудникЦентра`, `СортировочныйЦентр`.

Для классов реализовать следующие простые методы, используя для хранения данных списки (`[]`) Python:
\begin{enumerate}
    \item \textbf{ЗонаОбработки:} добавить посылку в зону, удалить посылку из зоны, 
    получить список посылок в зоне.
    \item \textbf{Посылка:} добавить статус обработки (принята, сортируется, отправлена), 
    удалить ошибочный статус, отобразить историю статусов обработки.
    \item \textbf{Конвейер:} запустить конвейерную ленту, остановить конвейер, 
    добавить посылку на конвейер, снять посылку с конвейера.
    \item \textbf{СотрудникЦентра:} класс, представляющий сотрудника, имеющий идентификатор, 
    смену, список закрепленных зон обработки, ФИО, должность.
    \item \textbf{СортировочныйЦентр:} добавить новую зону обработки, 
    ввести в эксплуатацию конвейер, нанять сотрудника, вывести список всех зон, конвейеров, 
    сотрудников, удалить зону, вывести из эксплуатации конвейер, уволить сотрудника.
\end{enumerate}

\item \textbf{Описание ситуации:} Рассмотрим работу диспетчерской службы 
городского пассажирского транспорта. 
Диспетчеры отслеживают движение автобусов, троллейбусов и трамваев 
на маршрутах, регулируют интервалы движения, фиксируют отклонения от 
графика. Транспортные средства оснащены GPS-трекерами для передачи местоположения.

\textbf{Создаваемые классы:} `Маршрут`, `ТранспортноеСредство`, `Диспетчер`, `Остановка`, `ДиспетчерскаяСлужба`.

Для классов реализовать следующие простые методы, используя для хранения данных списки (`[]`) Python:
\begin{enumerate}
    \item \textbf{Маршрут:} добавить транспортное средство на маршрут, 
    снять с маршрута, получить список транспорта на маршруте.
    \item \textbf{ТранспортноеСредство:} обновить местоположение (координаты), 
    получить текущее местоположение, добавить информацию о задержке/опережении графика.
    \item \textbf{Диспетчер:} класс, представляющий диспетчера, 
    имеющий идентификатор, смену, список контролируемых маршрутов, ФИО.
    \item \textbf{Остановка:} добавить маршрут, проходящий через остановку, 
    удалить маршрут, получить список маршрутов на остановке.
    \item \textbf{ДиспетчерскаяСлужба:} добавить новый маршрут, зарегистрировать транспортное средство, 
    нанять диспетчера, вывести информацию о всех маршрутах, транспорте, 
    диспетчерах, удалить маршрут, списать транспорт, уволить диспетчера.
\end{enumerate}

\item \textbf{Описание ситуации:} Рассмотрим работу центра технического обслуживания 
городского транспорта. В центре есть ремонтные зоны для разных видов 
транспорта, запасы запчастей и бригады механиков. Транспортные средства 
проходят плановое ТО и внеплановый ремонт.

\textbf{Создаваемые классы:} `РемонтнаяЗона`, `ТранспортноеСредство`, `Запчасть`, `Механик`, `ЦентрТехОбслуживания`.

Для классов реализовать следующие простые методы, используя для хранения данных списки (`[]`) Python:
\begin{enumerate}
    \item \textbf{РемонтнаяЗона:} поставить транспорт на ремонт, 
    завершить ремонт, получить список транспорта в ремонте.
    \item \textbf{ТранспортноеСредство:} добавить запись о ремонте (дата, вид работ), 
    удалить ошибочную запись, отобразить историю ремонтов.
    \item \textbf{Запчасть:} уменьшить количество на складе, увеличить количество, 
    получить текущий остаток.
    \item \textbf{Механик:} класс, представляющий механика, имеющий идентификатор, 
    квалификацию, список закрепленных ремонтных зон, ФИО.
    \item \textbf{ЦентрТехОбслуживания:} добавить ремонтную зону, закупить запчасти, 
    нанять механика, вывести информацию о зонах, запчастях, механиках, удалить зону, уволить механика.
\end{enumerate}

\item \textbf{Описание ситуации:}
Рассмотрим работу логистического центра междугородных автобусных перевозок. 
Автобусы совершают рейсы между городами по определенным маршрутам, 
перевозя пассажиров и их багаж. Диспетчеры формируют расписание, 
продают билеты и контролируют отправление автобусов.

\textbf{Создаваемые классы:} `Автобус`, `Маршрут`, `Пассажир`, `Диспетчер`, `ЛогистическийЦентр`.

Для классов реализовать следующие простые методы, используя для хранения данных списки (`[]`) Python:
\begin{enumerate}
    \item \textbf{Автобус:} назначить на маршрут, снять с маршрута, 
    добавить пассажира, высадить пассажира, получить список пассажиров.
    \item \textbf{Маршрут:} добавить город в маршрут, удалить город, 
    получить список всех городов на маршруте.
    \item \textbf{Пассажир:} купить билет (добавить маршрут в историю), 
    сдать билет (удалить маршрут), показать историю поездок.
    \item \textbf{Диспетчер:} класс, представляющий диспетчера, 
    имеющий идентификатор, список закрепленных маршрутов, ФИО, график работы.
    \item \textbf{ЛогистическийЦентр:} добавить автобус в парк, 
    добавить маршрут, нанять диспетчера, вывести список автобусов, 
    маршрутов и диспетчеров, списать автобус, уволить диспетчера.
\end{enumerate}

\item \textbf{Описание ситуации:} Рассмотрим работу центра управления интеллектуальной 
транспортной системой города. Система включает в себя управление светофорами, 
камеры видеонаблюдения, датчики транспортного потока. Операторы следят 
за дорожной ситуацией и оперативно реагируют на инциденты.

\textbf{Создаваемые классы:} `Перекресток`, `Светофор`, `КамераНаблюдения`, `ОператорИТС`, `ЦентрУправления`.

Для классов реализовать следующие простые методы, используя для хранения данных списки (`[]`) Python:
\begin{enumerate}
    \item \textbf{Перекресток:} добавить светофор к перекрестку, удалить светофор, 
    получить список светофоров на перекрестке.
    \item \textbf{Светофор:} изменить режим работы (красный/желтый/зеленый), 
    получить текущий режим, добавить информацию о неисправности, вывести список неисправностей.
    \item \textbf{КамераНаблюдения:} включить запись, выключить запись, 
    получить статус работы, зафиксировать нарушение ПДД, вывести список нарушений.
    \item \textbf{ОператорИТС:} класс, представляющий оператора, 
    имеющий идентификатор, смену, список контролируемых перекрестков, ФИО.
    \item \textbf{ЦентрУправления:} добавить новый перекресток в систему, 
    установить светофор, установить камеру, нанять оператора, вывести информацию о перекрестках, 
    светофорах, камерах, операторах, удалить перекресток, уволить оператора, снять камеру, снять светофор.
\end{enumerate}

\item \textbf{Описание ситуации:} Рассмотрим работу службы эвакуации аварийных транспортных средств. 
Эвакуаторы дежурят на специальных парковках и выезжают по вызову на места ДТП 
или поломок. Диспетчеры принимают вызовы и направляют ближайший свободный эвакуатор.

\textbf{Создаваемые классы:} `Эвакуатор`, `Вызов`, `ПарковкаЭвакуаторов`, `ДиспетчерЭвакуации`, `СлужбаЭвакуации`.

Для классов реализовать следующие простые методы, используя для хранения данных списки (`[]`) Python:
\begin{enumerate}
    \item \textbf{Эвакуатор:} принять вызов, завершить вызов, 
    получить текущий статус (свободен/занят), обновить местоположение.
    \item \textbf{Вызов:} зафиксировать время принятия, время выполнения, 
    получить статус выполнения.
    \item \textbf{ПарковкаЭвакуаторов:} принять эвакуатор на парковку, 
    выпустить эвакуатор с парковки, 
    получить список эвакуаторов на парковке.
    \item \textbf{ДиспетчерЭвакуации:} класс, представляющий диспетчера, 
    имеющий идентификатор, смену, список обработанных вызовов, ФИО.
    \item \textbf{СлужбаЭвакуации:} добавить эвакуатор в парк, 
    списать эвакуатор, 
    нанять диспетчера, вывести информацию о эвакуаторах, вызовах, диспетчерах, уволить диспетчера.
\end{enumerate}

\item \textbf{Описание ситуации:} Рассмотрим работу центра контроля коммерческих грузоперевозок. 
Система отслеживает движение грузовых автомобилей, контролирует соблюдение маршрутов, 
норм труда водителей и расход топлива. Менеджеры по логистике планируют маршруты 
и анализируют отчеты.

\textbf{Создаваемые классы:} `ГрузовойАвтомобиль`, `МаршрутПеревозки`, `Водитель`, `Рейс`, `МенеджерЛогистики`.

Для классов реализовать следующие простые методы, используя для хранения данных списки (`[]`) Python:
\begin{enumerate}
    \item \textbf{ГрузовойАвтомобиль:} начать рейс, завершить рейс, получить текущий статус, 
    зафиксировать расход топлива.
    \item \textbf{МаршрутПеревозки:} добавить точку маршрута (город, склад), 
    удалить точку, получить полный маршрут.
    \item \textbf{Водитель:} класс, представляющий водителя, 
    имеющий идентификатор, права, график работы, ФИО, стаж.
    \item \textbf{Рейс:} закрепить автомобиль за рейсом, закрепить водителя за рейсом, 
    открепить автомобиль, снять водителя, получить информацию о рейсе.
    \item \textbf{МенеджерЛогистики:} класс, представляющий менеджера, 
    имеющий идентификатор, список контролируемых маршрутов, ФИО.
\end{enumerate}

\item \textbf{Описание ситуации:} Рассмотрим работу службы парковки аэропорта. 
На территории аэропорта есть несколько парковочных зон для разных типов 
транспорта (краткосрочная, долгосрочная, VIP). 
Операторы контролируют занятость мест, прием оплаты и работу шлагбаумов.

\textbf{Создаваемые классы:} `ПарковочнаяЗона`, `ПарковочноеМесто`, `Автомобиль`, `ОператорПарковки`, `СлужбаПарковки`.

Для классов реализовать следующие простые методы, используя для хранения данных списки (`[]`) Python:
\begin{enumerate}
    \item \textbf{ПарковочнаяЗона:} добавить парковочное место, 
    удалить место, получить список мест в зоне, получить список всех автомобилей. Так же парковочной зоне 
    соответсвует стоимость часа стоянки.
    \item \textbf{ПарковочноеМесто:} занять место автомобилем, 
    освободить место, получить текущий статус (свободно/занято).
    \item \textbf{Автомобиль:} зафиксировать время въезда, время выезда + 
    рассчитать стоимость парковки (с учетом стоимости часа), получить историю.
    \item \textbf{ОператорПарковки:} класс, представляющий оператора, 
    имеющий идентификатор, смену, список контролируемых зон, ФИО.
    \item \textbf{СлужбаПарковки:} добавить новую парковочную зону, 
    нанять оператора, вывести информацию о зонах, местах, операторах, удалить зону, уволить оператора.
\end{enumerate}

\item \textbf{Описание ситуации:} Рассмотрим работу центра управления речным судоходством. 
Диспетчеры следят за движением судов по фарватеру, 
распределяют шлюзы, контролируют соблюдение графика движения 
и обеспечивают безопасность судоходства.

\textbf{Создаваемые классы:} `Судно`, `Шлюз`, `Фарватер`, `ДиспетчерСудоходства`, `ЦентрУправления`.

Для классов реализовать следующие простые методы, используя для хранения данных списки (`[]`) Python:
\begin{enumerate}
    \item \textbf{Судно:} начать движение по фарватеру, завершить движение, 
    получить текущее местоположение, зафиксировать прохождение шлюза.
    \item \textbf{Шлюз:} принять судно для шлюзования, 
    завершить шлюзование, получить текущий статус (свободен/занят).
    \item \textbf{Фарватер:} добавить участок фарватера, 
    удалить участок, получить список судов на фарватере.
    \item \textbf{ДиспетчерСудоходства:} класс, представляющий диспетчера, 
    имеющий идентификатор, смену, список контролируемых шлюзов, ФИО.
    \item \textbf{ЦентрУправления:} добавить шлюз в систему, 
    зарегистрировать судно, нанять диспетчера, 
    вывести информацию о шлюзах, фарватерах, судах, диспетчерах, удалить шлюз, уволить диспетчера.
\end{enumerate}

\item \textbf{Описание ситуации:} Рассмотрим работу службы технического контроля метрополитена. 
Инспекторы проверяют состояние путей, тоннелей, подвижного состава и оборудования станций. 
Дефекты фиксируются в системе для оперативного устранения ремонтными бригадами.

\textbf{Создаваемые классы:} `УчастокПути`, `ПодвижнойСостав`, `Инспектор`, `Дефект`, `СлужбаКонтроля`.

Для классов реализовать следующие простые методы, использующие для хранения данных списки (`[]`) Python:
\begin{enumerate}
    \item \textbf{УчастокПути:} добавить информацию о дефекте, 
    получить список неустраненных дефектов на участке.
    \item \textbf{ПодвижнойСостав:} добавить запись о техническом осмотре, 
    удалить ошибочную запись, отобразить историю осмотров.
    \item \textbf{Инспектор:} класс, представляющий инспектора, 
    имеющий идентификатор, квалификацию, список закрепленных участков, ФИО.
    \item \textbf{Дефект:} зафиксировать время обнаружения, 
    время устранения, получить статус устранения.
    \item \textbf{СлужбаКонтроля:} добавить участок пути в систему, 
    зарегистрировать подвижной состав, нанять инспектора, 
    вывести информацию об участках, составе, инспекторах, дефектах, удалить участок, уволить инспектора, 
    снять с эксплуатации подвижной состав.
\end{enumerate}

\item
\textbf{Описание ситуации:}
Рассмотрим работу центра управления умными светофорами на перекрестках. 
Умные светофоры адаптивно меняют режим работы в зависимости от транспортного потока, 
приоритизируя общественный транспорт и спецтранспорт. 
Система анализирует данные с датчиков и камер, оптимизируя пропускную способность перекрестков.

\textbf{Создаваемые классы:} УмныйСветофор, Перекресток, ДатчикТранспортногоПотока, ИнженерАТС, ЦентрУправленияСветофорами.

Для классов реализовать следующие простые методы, используя для хранения данных списки ([]) Python:
\begin{enumerate}
\item \textbf{УмныйСветофор:} изменить длительность фаз (красный/зеленый), 
перейти в аварийный режим, получить текущий режим работы.
\item \textbf{Перекресток:} добавить светофор к перекрестку, удалить светофор, 
получить список всех светофоров перекрестка.
\item \textbf{ДатчикТранспортногоПотока:} установить текущие данные о интенсивности движения, 
получить текущие показания, получить историю показаний.
\item \textbf{ИнженерАТС:} класс, представляющий инженера автоматизированной транспортной системы, 
имеющий идентификатор, квалификацию, список закрепленных перекрестков, ФИО.
\item \textbf{ЦентрУправленияСветофорами:} добавить новый перекресток в систему, 
установить умный светофор, нанять инженера, вывести информацию о перекрестках, светофорах, инженерах, 
удалить перекресток, уволить инженера, снять умный светофор.
\end{enumerate}

\item
\textbf{Описание ситуации:}
Рассмотрим работу монорельсовой транспортной системы. 
Монорельс движется по эстакаде, состоящей из станций и перегонов. 
Составы имеют фиксированное количество вагонов. 
Операторы управляют движением составов, следят за соблюдением графика и безопасностью пассажиров.

\textbf{Создаваемые классы:} СтанцияМонорельса, СоставМонорельса, ВагонМонорельса, ОператорСистемы, УправлениеМонорельсом.

Для классов реализовать следующие простые методы, используя для хранения данных списки ([]) Python:
\begin{enumerate}
\item \textbf{СтанцияМонорельса:} принять состав, отправить состав, 
получить список составов на станции.
\item \textbf{СоставМонорельса:} добавить вагон в состав (при техническом обслуживании), 
удалить вагон, получить список вагонов.
\item \textbf{ВагонМонорельса:} зафиксировать текущий пробег, 
провести техническое обслуживание, получить историю обслуживаний.
\item \textbf{ОператорСистемы:} класс, представляющий оператора, 
имеющий идентификатор, смену, список закрепленных станций, ФИО.
\item \textbf{УправлениеМонорельсом:} добавить новую станцию, ввести состав в эксплуатацию, 
нанять оператора, вывести информацию о станциях, составах, операторах, закрыть станцию на ремонт, 
списать состав, уволить оператора.
\end{enumerate}

\item
\textbf{Описание ситуации:}
Рассмотрим работу канатной дороги. Канатная дорога состоит из линий с опорами и кабинок, 
перемещающихся между станциями. 
Кабинки имеют ограниченную вместимость. Техники обслуживают механизмы и следят за безопасностью.

\textbf{Создаваемые классы:} ЛинияКанатнойДороги, Кабинка, СтанцияКанатнойДороги, Техник, УправлениеКанатнойДорогой.

Для классов реализовать следующие простые методы, используя для хранения данных списки ([]) Python:
\begin{enumerate}
\item \textbf{ЛинияКанатнойДороги:} добавить кабинку на линию, снять кабинку, 
получить список кабинок на линии.
\item \textbf{Кабинка:} запустить в движение, остановить для посадки/высадки, 
получить текущий статус (движется/стоит).
\item \textbf{СтанцияКанатнойДороги:} принять кабинку, отправить кабинку, 
получить список кабинок на станции.
\item \textbf{Техник:} класс, представляющий техника, имеющий идентификатор, 
квалификацию, список закрепленных линий, ФИО.
\item \textbf{УправлениеКанатнойДорогой:} добавить новую линию, 
ввести кабинку в эксплуатацию, нанять техника, вывести информацию о линиях, кабинках, техниках, 
закрыть линию на обслуживание, списать кабинку, уволить техника.
\end{enumerate}

\item
\textbf{Описание ситуации:}
Рассмотрим работу службы доставки с использованием дронов. 
Дроны осуществляют доставку небольших грузов между пунктами выдачи. 
Каждый дрон имеет грузоподъемность и дальность полета. 
Операторы управляют полетами дронов и обслуживают пункты выдачи.

\textbf{Создаваемые классы:} ПунктВыдачи, Дрон, Груз, ОператорДронов, СлужбаДоставки.

Для классов реализовать следующие простые методы, используя для хранения данных списки ([]) Python:
\begin{enumerate}
\item \textbf{ПунктВыдачи:} принять дрон с грузом, отправить дрон, получить список дронов в пункте.
\item \textbf{Дрон:} загрузить груз, выгрузить груз, 
начать полет, завершить полет, получить текущий статус (в полете/на земле).
\item \textbf{Груз:} зарегистрировать отправку, зарегистрировать доставку, 
получить историю перемещений.
\item \textbf{ОператорДронов:} класс, представляющий оператора, 
имеющий идентификатор, смену, список закрепленных пунктов выдачи, ФИО.
\item \textbf{СлужбаДоставки:} добавить новый пункт выдачи, 
ввести дрон в эксплуатацию, нанять оператора, вывести информацию о пунктах, 
дронах, операторах, закрыть пункт, списать дрон, уволить оператора.
\end{enumerate}

\end{enumerate}
\subsection{Семинар <<Конструкторы, наследование и полиморфизм. 1 часть>>  
(2 часа)}


В ходе работы решите 4 задачи. 
Предполагается, что пользователь класса не имеет права обращаться к свойствам напрямую 
(соблюдая принцип инкапсуляции), а должен использовать методы. 

Продемонстрируйте работоспособность всех методов (из задания) 
посредством создания запускаемых файлов, где осуществляется 
вызов методов для разных ситуаций 
(без ручного ввода, но с выводом результатов в консоль). 

Каждый класс должен сохраняться в отдельном исходном файле. 
Необходимо соблюдать все стандартные требования к качеству кода 
(отступы, именования переменных, классов, методов, 
проверка корректности входных данных).
Для каждого класса создайте отдельный запускаемый файл для проверки всех его методов 
(допускается использование других классов в этих тестах).

Все предлагаемые классы в заданиях упрощенные; для использования в production-окружении они требуют серьезной доработки. Суть задания — в отработке базовых навыков, а не в идеальном моделировании предложенных ситуаций.

Для сдачи работы будьте готовы пояснить или аналогично заданию модифицировать любую часть кода, а также ответить на вопросы:
\begin{enumerate}
    \item Что обозначает свойство наследования в парадигме ООП?
    \item Что обозначает свойство полиморфизма в парадигме ООП?
    \item Опишите реализацию наследования в Python
    \item Как создать конструктор в Python
    \item Как реализовать абстрактный класс в Python (и что это значит)
    \item Как реализовать абстрактные методы в Python (и что это значит)
\end{enumerate}

Если вы нашли в задачнике ошибки, опечатки и другие недостатки, то вы можете сделать pull-request. 

\textbf{Срок сдачи работы (начала сдачи):} через одно занятие после его выдачи. В последующие сроки оценка будет снижаться (при отсутствии оправдывающих документов).

\textbf{Задача 1}

\begin{enumerate}
    \item 

Напишите программу, которая создаёт класс
\texttt{Circle} с методами для вычисления площади
и длины окружности (периметра). Программа должна запрашивать у пользователя радиус
и выводить вычисленные площадь и длину окружности.

\subsection*{Инструкции:}
\begin{enumerate}
\item Создайте класс \texttt{Circle} с методом
\texttt{\_\_init\_\_}, который принимает радиус окружности в
качестве аргумента и сохраняет его в атрибуте \texttt{self.\_\_radius}.

\item Создайте метод \texttt{calculate\_circle\_area},
без аргументов, который вычисляет площадь круга по формуле:
\[
\pi \cdot \texttt{\_\_radius}^2
\]

\item Создайте метод \texttt{calculate\_circle\_perimeter} без аргументов,
который вычисляет длину окружности по формуле:
\[
2 \cdot \pi \cdot \texttt{\_\_radius}
\]

\item Напишите цикл, который повторяется 10 раз. В каждой итерации программа должна:
\begin{enumerate}
\item запрашивать у пользователя радиус окружности,
\item создавать экземпляр класса \texttt{Circle} с этим радиусом,
\item вычислять площадь и длину окружности с помощью соответствующих методов,
\item выводить результаты на экран.
\end{enumerate}
\end{enumerate}

\subsection*{Пример использования:}
\begin{verbatim}
radius = 3
circle = Circle(radius)
area = circle.calculate_circle_area()
perimeter = circle.calculate_circle_perimeter()
print(f"Площадь окружности равна: {area}")
print(f"Периметр окружности равен: {perimeter}")
\end{verbatim}

\textbf{Вывод:}
\begin{verbatim}
Площадь окружности равна: 28.274333882308138
Периметр окружности равен: 18.84955592153876
\end{verbatim}

\item 
Напишите программу, которая создаёт класс \texttt{Square} с методами для вычисления площади
и периметра. Программа должна запрашивать у пользователя длину стороны
и выводить вычисленные площадь и периметр.

\subsection*{Инструкции:}
\begin{enumerate}
\item Создайте класс \texttt{Square} с методом
\texttt{\_\_init\_\_}, который принимает длину стороны квадрата в
качестве аргумента и сохраняет её в атрибуте \texttt{self.\_\_side}.

\item Создайте метод \texttt{calculate\_area},
без аргументов, который вычисляет площадь квадрата по формуле:
\[
\texttt{\_\_side}^2
\]

\item Создайте метод \texttt{calculate\_perimeter} без аргументов,
который вычисляет периметр квадрата по формуле:
\[
4 \cdot \texttt{\_\_side}
\]

\item Напишите цикл, который повторяется 10 раз. В каждой итерации программа должна:
\begin{enumerate}
\item запрашивать у пользователя длину стороны квадрата,
\item создавать экземпляр класса \texttt{Square} с этой длиной,
\item вычислять площадь и периметр с помощью соответствующих методов,
\item выводить результаты на экран.
\end{enumerate}
\end{enumerate}

\subsection*{Пример использования:}
\begin{verbatim}
side = 5
square = Square(side)
area = square.calculate_area()
perimeter = square.calculate_perimeter()
print(f"Площадь квадрата равна: {area}")
print(f"Периметр квадрата равен: {perimeter}")
\end{verbatim}

\textbf{Вывод:}
\begin{verbatim}
Площадь квадрата равна: 25
Периметр квадрата равен: 20
\end{verbatim}

\item
Напишите программу, которая создаёт класс \texttt{Rectangle} с методами для вычисления площади
и периметра. Программа должна запрашивать у пользователя ширину прямоугольника
(при соотношении сторон 1:2) и выводить вычисленные площадь и периметр.

\subsection*{Инструкции:}
\begin{enumerate}
\item Создайте класс \texttt{Rectangle} с методом
\texttt{\_\_init\_\_}, который принимает ширину прямоугольника в
качестве аргумента и сохраняет её в атрибуте \texttt{self.\_\_width}.
Высота прямоугольника должна быть в два раза больше ширины.

\item Создайте метод \texttt{calculate\_area},
без аргументов, который вычисляет площадь прямоугольника по формуле:
\[
\texttt{\_\_width} \cdot (2 \cdot \texttt{\_\_width})
\]

\item Создайте метод \texttt{calculate\_perimeter} без аргументов,
который вычисляет периметр прямоугольника по формуле:
\[
2 \cdot (\texttt{\_\_width} + 2 \cdot \texttt{\_\_width})
\]

\item Напишите цикл, который повторяется 10 раз. В каждой итерации программа должна:
\begin{enumerate}
\item запрашивать у пользователя ширину прямоугольника,
\item создавать экземпляр класса \texttt{Rectangle} с этой шириной,
\item вычислять площадь и периметр с помощью соответствующих методов,
\item выводить результаты на экран.
\end{enumerate}
\end{enumerate}

\subsection*{Пример использования:}
\begin{verbatim}
width = 3
rectangle = Rectangle(width)
area = rectangle.calculate_area()
perimeter = rectangle.calculate_perimeter()
print(f"Площадь прямоугольника равна: {area}")
print(f"Периметр прямоугольника равен: {perimeter}")
\end{verbatim}

\textbf{Вывод:}
\begin{verbatim}
Площадь прямоугольника равна: 18
Периметр прямоугольника равен: 18
\end{verbatim}

\item
Напишите программу, которая создаёт класс \texttt{Triangle} с методами для вычисления площади
и периметра. Программа должна запрашивать у пользователя длину стороны
равностороннего треугольника и выводить вычисленные площадь и периметр.

\subsection*{Инструкции:}
\begin{enumerate}
\item Создайте класс \texttt{Triangle} с методом
\texttt{\_\_init\_\_}, который принимает длину стороны треугольника в
качестве аргумента и сохраняет её в атрибуте \texttt{self.\_\_side}.

\item Создайте метод \texttt{calculate\_area},
без аргументов, который вычисляет площадь равностороннего треугольника по формуле:
\[
\frac{\sqrt{3}}{4} \cdot \texttt{\_\_side}^2
\]

\item Создайте метод \texttt{calculate\_perimeter} без аргументов,
который вычисляет периметр треугольника по формуле:
\[
3 \cdot \texttt{\_\_side}
\]

\item Напишите цикл, который повторяется 10 раз. В каждой итерации программа должна:
\begin{enumerate}
\item запрашивать у пользователя длину стороны треугольника,
\item создавать экземпляр класса \texttt{Triangle} с этой длиной,
\item вычислять площадь и периметр с помощью соответствующих методов,
\item выводить результаты на экран.
\end{enumerate}
\end{enumerate}

\subsection*{Пример использования:}
\begin{verbatim}
side = 4
triangle = Triangle(side)
area = triangle.calculate_area()
perimeter = triangle.calculate_perimeter()
print(f"Площадь треугольника равна: {area}")
print(f"Периметр треугольника равен: {perimeter}")
\end{verbatim}

\textbf{Вывод:}
\begin{verbatim}
Площадь треугольника равна: 6.928203230275509
Периметр треугольника равен: 12
\end{verbatim}

\item
Напишите программу, которая создаёт класс \texttt{Sphere} с методами для вычисления площади поверхности
и объёма. Программа должна запрашивать у пользователя радиус сферы
и выводить вычисленные площадь поверхности и объём.

\subsection*{Инструкции:}
\begin{enumerate}
\item Создайте класс \texttt{Sphere} с методом
\texttt{\_\_init\_\_}, который принимает радиус сферы в
качестве аргумента и сохраняет его в атрибуте \texttt{self.\_\_radius}.

\item Создайте метод \texttt{calculate\_surface\_area},
без аргументов, который вычисляет площадь поверхности сферы по формуле:
\[
4 \cdot \pi \cdot \texttt{\_\_radius}^2
\]

\item Создайте метод \texttt{calculate\_volume} без аргументов,
который вычисляет объём сферы по формуле:
\[
\frac{4}{3} \cdot \pi \cdot \texttt{\_\_radius}^3
\]

\item Напишите цикл, который повторяется 10 раз. В каждой итерации программа должна:
\begin{enumerate}
\item запрашивать у пользователя радиус сферы,
\item создавать экземпляр класса \texttt{Sphere} с этим радиусом,
\item вычислять площадь поверхности и объём с помощью соответствующих методов,
\item выводить результаты на экран.
\end{enumerate}
\end{enumerate}

\subsection*{Пример использования:}
\begin{verbatim}
radius = 2
sphere = Sphere(radius)
surface_area = sphere.calculate_surface_area()
volume = sphere.calculate_volume()
print(f"Площадь поверхности сферы равна: {surface_area}")
print(f"Объём сферы равен: {volume}")
\end{verbatim}

\textbf{Вывод:}
\begin{verbatim}
Площадь поверхности сферы равна: 50.26548245743669
Объём сферы равен: 33.510321638291124
\end{verbatim}

\item
Напишите программу, которая создаёт класс \texttt{Cylinder} с методами для вычисления объёма
и площади боковой поверхности. Программа должна запрашивать у пользователя радиус основания
и выводить вычисленные объём и площадь боковой поверхности (высота цилиндра фиксирована и равна 5).

\subsection*{Инструкции:}
\begin{enumerate}
\item Создайте класс \texttt{Cylinder} с методом
\texttt{\_\_init\_\_}, который принимает радиус основания цилиндра в
качестве аргумента и сохраняет его в атрибуте \texttt{self.\_\_radius}.
Высота цилиндра фиксирована и равна 5.

\item Создайте метод \texttt{calculate\_volume},
без аргументов, который вычисляет объём цилиндра по формуле:
\[
\pi \cdot \texttt{\_\_radius}^2 \cdot 5
\]

\item Создайте метод \texttt{calculate\_lateral\_area} без аргументов,
который вычисляет площадь боковой поверхности цилиндра по формуле:
\[
2 \cdot \pi \cdot \texttt{\_\_radius} \cdot 5
\]

\item Напишите цикл, который повторяется 10 раз. В каждой итерации программа должна:
\begin{enumerate}
\item запрашивать у пользователя радиус основания цилиндра,
\item создавать экземпляр класса \texttt{Cylinder} с этим радиусом,
\item вычислять объём и площадь боковой поверхности с помощью соответствующих методов,
\item выводить результаты на экран.
\end{enumerate}
\end{enumerate}

\subsection*{Пример использования:}
\begin{verbatim}
radius = 3
cylinder = Cylinder(radius)
volume = cylinder.calculate_volume()
lateral_area = cylinder.calculate_lateral_area()
print(f"Объём цилиндра равен: {volume}")
print(f"Площадь боковой поверхности равна: {lateral_area}")
\end{verbatim}

\textbf{Вывод:}
\begin{verbatim}
Объём цилиндра равен: 141.3716694115407
Площадь боковой поверхности равна: 94.24777960769379
\end{verbatim}

\item
Напишите программу, которая создаёт класс \texttt{Cone} с методами для вычисления объёма
и площади боковой поверхности. Программа должна запрашивать у пользователя радиус основания
и выводить вычисленные объём и площадь боковой поверхности (высота конуса фиксирована и равна 10).

\subsection*{Инструкции:}
\begin{enumerate}
\item Создайте класс \texttt{Cone} с методом
\texttt{\_\_init\_\_}, который принимает радиус основания конуса в
качестве аргумента и сохраняет его в атрибуте \texttt{self.\_\_radius}.
Высота конуса фиксирована и равна 10.

\item Создайте метод \texttt{calculate\_volume},
без аргументов, который вычисляет объём конуса по формуле:
\[
\frac{1}{3} \cdot \pi \cdot \texttt{\_\_radius}^2 \cdot 10
\]

\item Создайте метод \texttt{calculate\_lateral\_area} без аргументов,
который вычисляет площадь боковой поверхности конуса по формуле:
\[
\pi \cdot \texttt{\_\_radius} \cdot \sqrt{\texttt{\_\_radius}^2 + 10^2}
\]

\item Напишите цикл, который повторяется 10 раз. В каждой итерации программа должна:
\begin{enumerate}
\item запрашивать у пользователя радиус основания конуса,
\item создавать экземпляр класса \texttt{Cone} с этим радиусом,
\item вычислять объём и площадь боковой поверхности с помощью соответствующих методов,
\item выводить результаты на экран.
\end{enumerate}
\end{enumerate}

\subsection*{Пример использования:}
\begin{verbatim}
radius = 3
cone = Cone(radius)
volume = cone.calculate_volume()
lateral_area = cone.calculate_lateral_area()
print(f"Объём конуса равен: {volume}")
print(f"Площадь боковой поверхности равна: {lateral_area}")
\end{verbatim}

\textbf{Вывод:}
\begin{verbatim}
Объём конуса равен: 94.24777960769379
Площадь боковой поверхности равна: 94.86832980505137
\end{verbatim}

\item
Напишите программу, которая создаёт класс \texttt{Cube} с методами для вычисления объёма
и площади полной поверхности. Программа должна запрашивать у пользователя длину ребра куба
и выводить вычисленные объём и площадь.

\subsection*{Инструкции:}
\begin{enumerate}
\item Создайте класс \texttt{Cube} с методом
\texttt{\_\_init\_\_}, который принимает длину ребра куба в
качестве аргумента и сохраняет её в атрибуте \texttt{self.\_\_side}.

\item Создайте метод \texttt{calculate\_volume},
без аргументов, который вычисляет объём куба по формуле:
\[
\texttt{\_\_side}^3
\]

\item Создайте метод \texttt{calculate\_surface\_area} без аргументов,
который вычисляет площадь полной поверхности куба по формуле:
\[
6 \cdot \texttt{\_\_side}^2
\]

\item Напишите цикл, который повторяется 10 раз. В каждой итерации программа должна:
\begin{enumerate}
\item запрашивать у пользователя длину ребра куба,
\item создавать экземпляр класса \texttt{Cube} с этой длиной,
\item вычислять объём и площадь полной поверхности с помощью соответствующих методов,
\item выводить результаты на экран.
\end{enumerate}
\end{enumerate}

\subsection*{Пример использования:}
\begin{verbatim}
side = 4
cube = Cube(side)
volume = cube.calculate_volume()
surface_area = cube.calculate_surface_area()
print(f"Объём куба равен: {volume}")
print(f"Площадь полной поверхности равна: {surface_area}")
\end{verbatim}

\textbf{Вывод:}
\begin{verbatim}
Объём куба равен: 64
Площадь полной поверхности равна: 96
\end{verbatim}

\item
Напишите программу, которая создаёт класс \texttt{Parallelogram} с методами для вычисления площади
и периметра. Программа должна запрашивать у пользователя длину основания параллелограмма
и выводить вычисленные площадь и периметр (высота параллелограмма фиксирована и равна 8, 
а боковая сторона равна 6).

\subsection*{Инструкции:}
\begin{enumerate}
\item Создайте класс \texttt{Parallelogram} с методом
\texttt{\_\_init\_\_}, который принимает длину основания параллелограмма в
качестве аргумента и сохраняет её в атрибуте \texttt{self.\_\_base}.
Высота параллелограмма фиксирована и равна 8, а боковая сторона равна 6.

\item Создайте метод \texttt{calculate\_area},
без аргументов, который вычисляет площадь параллелограмма по формуле:
\[
\texttt{\_\_base} \cdot 8
\]

\item Создайте метод \texttt{calculate\_perimeter} без аргументов,
который вычисляет периметр параллелограмма по формуле:
\[
2 \cdot (\texttt{\_\_base} + 6)
\]

\item Напишите цикл, который повторяется 10 раз. В каждой итерации программа должна:
\begin{enumerate}
\item запрашивать у пользователя длину основания параллелограмма,
\item создавать экземпляр класса \texttt{Parallelogram} с этой длиной,
\item вычислять площадь и периметр с помощью соответствующих методов,
\item выводить результаты на экран.
\end{enumerate}
\end{enumerate}

\subsection*{Пример использования:}
\begin{verbatim}
base = 5
parallelogram = Parallelogram(base)
area = parallelogram.calculate_area()
perimeter = parallelogram.calculate_perimeter()
print(f"Площадь параллелограмма равна: {area}")
print(f"Периметр параллелограмма равен: {perimeter}")
\end{verbatim}

\textbf{Вывод:}
\begin{verbatim}
Площадь параллелограмма равна: 40
Периметр параллелограмма равен: 22
\end{verbatim}


\item
Напишите программу, которая создаёт класс \texttt{Ellipse} с методами для вычисления площади
и приближённого значения периметра. Программа должна запрашивать у пользователя длину большой полуоси
и выводить вычисленные площадь и периметр (длина малой полуоси фиксирована и равна 3).

\subsection*{Инструкции:}
\begin{enumerate}
\item Создайте класс \texttt{Ellipse} с методом
\texttt{\_\_init\_\_}, который принимает длину большой полуоси эллипса в
качестве аргумента и сохраняет её в атрибуте \texttt{self.\_\_major\_axis}.
Длина малой полуоси фиксирована и равна 3.

\item Создайте метод \texttt{calculate\_area},
без аргументов, который вычисляет площадь эллипса по формуле:
\[
\pi \cdot \texttt{\_\_major\_axis} \cdot 3
\]

\item Создайте метод \texttt{calculate\_perimeter} без аргументов,
который вычисляет приближённое значение периметра эллипса по формуле Рамануджана:
\[
\pi \cdot \left(3(\texttt{\_\_major\_axis} + 3) - \sqrt{(3\texttt{\_\_major\_axis} + 3)(\texttt{\_\_major\_axis} + 9)}\right)
\]

\item Напишите цикл, который повторяется 10 раз. В каждой итерации программа должна:
\begin{enumerate}
\item запрашивать у пользователя длину большой полуоси эллипса,
\item создавать экземпляр класса \texttt{Ellipse} с этой длиной,
\item вычислять площадь и периметр с помощью соответствующих методов,
\item выводить результаты на экран.
\end{enumerate}
\end{enumerate}

\subsection*{Пример использования:}
\begin{verbatim}
major_axis = 5
ellipse = Ellipse(major_axis)
area = ellipse.calculate_area()
perimeter = ellipse.calculate_perimeter()
print(f"Площадь эллипса равна: {area}")
print(f"Периметр эллипса равен: {perimeter}")
\end{verbatim}

\textbf{Вывод:}
\begin{verbatim}
Площадь эллипса равна: 47.12388980384689
Периметр эллипса равен: 25.74488980384689
\end{verbatim}

\item
Напишите программу, которая создаёт класс \texttt{BankAccount} с методами для вычисления начисленных процентов
и суммы налога на доход. Программа должна запрашивать у пользователя начальный баланс счёта
и выводить вычисленные проценты и налог (процентная ставка фиксирована и равна 5\%, 
налоговая ставка на доход фиксирована и равна 13\%).

\subsection*{Инструкции:}
\begin{enumerate}
\item Создайте класс \texttt{BankAccount} с методом
\texttt{\_\_init\_\_}, который принимает начальный баланс счёта в
качестве аргумента и сохраняет его в атрибуте \texttt{self.\_\_balance}.

\item Создайте метод \texttt{calculate\_interest},
без аргументов, который вычисляет начисленные проценты по формуле:
\[
\texttt{\_\_balance} \cdot 0.05
\]

\item Создайте метод \texttt{calculate\_tax} без аргументов,
который вычисляет сумму налога на полученный доход (проценты) по формуле:
\[
(\texttt{\_\_balance} \cdot 0.05) \cdot 0.13
\]

\item Напишите цикл, который повторяется 10 раз. В каждой итерации программа должна:
\begin{enumerate}
\item запрашивать у пользователя начальный баланс счёта,
\item создавать экземпляр класса \texttt{BankAccount} с этим балансом,
\item вычислять начисленные проценты и сумму налога с помощью соответствующих методов,
\item выводить результаты на экран.
\end{enumerate}
\end{enumerate}

\subsection*{Пример использования:}
\begin{verbatim}
balance = 1000
account = BankAccount(balance)
interest = account.calculate_interest()
tax = account.calculate_tax()
print(f"Начисленные проценты: {interest}")
print(f"Сумма налога на доход: {tax}")
\end{verbatim}

\textbf{Вывод:}
\begin{verbatim}
Начисленные проценты: 50.0
Сумма налога на доход: 6.5
\end{verbatim}

\item
Напишите программу, которая создаёт класс \texttt{TemperatureConverter} с методами для преобразования температуры
из градусов Цельсия в Фаренгейты и Кельвины. Программа должна запрашивать у пользователя температуру в Цельсиях
и выводить преобразованные значения.

\subsection*{Инструкции:}
\begin{enumerate}
\item Создайте класс \texttt{TemperatureConverter} с методом
\texttt{\_\_init\_\_}, который принимает температуру в градусах Цельсия в
качестве аргумента и сохраняет её в атрибуте \texttt{self.\_\_celsius}.

\item Создайте метод \texttt{to\_fahrenheit},
без аргументов, который преобразует температуру в Фаренгейты по формуле:
\[
(\texttt{\_\_celsius} \times \frac{9}{5}) + 32
\]

\item Создайте метод \texttt{to\_kelvin} без аргументов,
который преобразует температуру в Кельвины по формуле:
\[
\texttt{\_\_celsius} + 273.15
\]

\item Напишите цикл, который повторяется 10 раз. В каждой итерации программа должна:
\begin{enumerate}
\item запрашивать у пользователя температуру в градусах Цельсия,
\item создавать экземпляр класса \texttt{TemperatureConverter} с этим значением,
\item вычислять температуру в Фаренгейтах и Кельвинах с помощью соответствующих методов,
\item выводить результаты на экран.
\end{enumerate}
\end{enumerate}

\subsection*{Пример использования:}
\begin{verbatim}
celsius = 25
converter = TemperatureConverter(celsius)
fahrenheit = converter.to_fahrenheit()
kelvin = converter.to_kelvin()
print(f"Температура в Фаренгейтах: {fahrenheit}")
print(f"Температура в Кельвинах: {kelvin}")
\end{verbatim}

\textbf{Вывод:}
\begin{verbatim}
Температура в Фаренгейтах: 77.0
Температура в Кельвинах: 298.15
\end{verbatim}

\item
Напишите программу, которая создаёт класс \texttt{DistanceConverter} с методами для преобразования расстояния
из метров в километры и мили. Программа должна запрашивать у пользователя расстояние в метрах
и выводить преобразованные значения.

\subsection*{Инструкции:}
\begin{enumerate}
\item Создайте класс \texttt{DistanceConverter} с методом
\texttt{\_\_init\_\_}, который принимает расстояние в метрах в
качестве аргумента и сохраняет его в атрибуте \texttt{self.\_\_meters}.

\item Создайте метод \texttt{to\_kilometers},
без аргументов, который преобразует расстояние в километры по формуле:
\[
\texttt{\_\_meters} \div 1000
\]

\item Создайте метод \texttt{to\_miles} без аргументов,
который преобразует расстояние в мили по формуле:
\[
\texttt{\_\_meters} \div 1609.344
\]

\item Напишите цикл, который повторяется 10 раз. В каждой итерации программа должна:
\begin{enumerate}
\item запрашивать у пользователя расстояние в метрах,
\item создавать экземпляр класса \texttt{DistanceConverter} с этим значением,
\item вычислять расстояние в километрах и милях с помощью соответствующих методов,
\item выводить результаты на экран.
\end{enumerate}
\end{enumerate}

\subsection*{Пример использования:}
\begin{verbatim}
meters = 1609.344
converter = DistanceConverter(meters)
kilometers = converter.to_kilometers()
miles = converter.to_miles()
print(f"Расстояние в километрах: {kilometers}")
print(f"Расстояние в милях: {miles}")
\end{verbatim}

\textbf{Вывод:}
\begin{verbatim}
Расстояние в километрах: 1.609344
Расстояние в милях: 1.0
\end{verbatim}

\item
Напишите программу, которая создаёт класс \texttt{WeightConverter} с методами для преобразования массы
из килограммов в граммы и фунты. Программа должна запрашивать у пользователя массу в килограммах
и выводить преобразованные значения.

\subsection*{Инструкции:}
\begin{enumerate}
\item Создайте класс \texttt{WeightConverter} с методом
\texttt{\_\_init\_\_}, который принимает массу в килограммах в
качестве аргумента и сохраняет её в атрибуте \texttt{self.\_\_kg}.

\item Создайте метод \texttt{to\_grams},
без аргументов, который преобразует массу в граммы по формуле:
\[
\texttt{\_\_kg} \times 1000
\]

\item Создайте метод \texttt{to\_pounds} без аргументов,
который преобразует массу в фунты по формуле:
\[
\texttt{\_\_kg} \times 2.20462
\]

\item Напишите цикл, который повторяется 10 раз. В каждой итерации программа должна:
\begin{enumerate}
\item запрашивать у пользователя массу в килограммах,
\item создавать экземпляр класса \texttt{WeightConverter} с этим значением,
\item вычислять массу в граммах и фунтах с помощью соответствующих методов,
\item выводить результаты на экран.
\end{enumerate}
\end{enumerate}

\subsection*{Пример использования:}
\begin{verbatim}
kg = 2.5
converter = WeightConverter(kg)
grams = converter.to_grams()
pounds = converter.to_pounds()
print(f"Масса в граммах: {grams}")
print(f"Масса в фунтах: {pounds}")
\end{verbatim}

\textbf{Вывод:}
\begin{verbatim}
Масса в граммах: 2500.0
Масса в фунтах: 5.51155
\end{verbatim}

\item 


Напишите программу, которая создаёт класс \texttt{TimeConverter} с методами для преобразования времени
из секунд в минуты и часы. Программа должна запрашивать у пользователя время в секундах
и выводить преобразованные значения.

\subsection*{Инструкции:}
\begin{enumerate}
\item Создайте класс \texttt{TimeConverter} с методом
\texttt{\_\_init\_\_}, который принимает время в секундах в
качестве аргумента и сохраняет его в атрибуте \texttt{self.\_\_seconds}.

\item Создайте метод \texttt{to\_minutes},
без аргументов, который преобразует время в минуты по формуле:
\[
\texttt{\_\_seconds} \div 60
\]

\item Создайте метод \texttt{to\_hours} без аргументов,
который преобразует время в часы по формуле:
\[
\texttt{\_\_seconds} \div 3600
\]

\item Напишите цикл, который повторяется 10 раз. В каждой итерации программа должна:
\begin{enumerate}
\item запрашивать у пользователя время в секундах,
\item создавать экземпляр класса \texttt{TimeConverter} с этим значением,
\item вычислять время в минутах и часах с помощью соответствующих методов,
\item выводить результаты на экран.
\end{enumerate}
\end{enumerate}

\subsection*{Пример использования:}
\begin{verbatim}
seconds = 7200
converter = TimeConverter(seconds)
minutes = converter.to_minutes()
hours = converter.to_hours()
print(f"Время в минутах: {minutes}")
print(f"Время в часах: {hours}")
\end{verbatim}

\textbf{Вывод:}
\begin{verbatim}
Время в минутах: 120.0
Время в часах: 2.0
\end{verbatim}

\item


Напишите программу, которая создаёт класс \texttt{SpeedConverter} с методами для преобразования скорости
из километров в час в метры в секунду и мили в час. Программа должна запрашивать у пользователя скорость в км/ч
и выводить преобразованные значения.

\subsection*{Инструкции:}
\begin{enumerate}
\item Создайте класс \texttt{SpeedConverter} с методом
\texttt{\_\_init\_\_}, который принимает скорость в км/ч в
качестве аргумента и сохраняет её в атрибуте \texttt{self.\_\_kmh}.

\item Создайте метод \texttt{to\_ms},
без аргументов, который преобразует скорость в м/с по формуле:
\[
\texttt{\_\_kmh} \times \frac{1000}{3600}
\]

\item Создайте метод \texttt{to\_mph} без аргументов,
который преобразует скорость в мили/ч по формуле:
\[
\texttt{\_\_kmh} \div 1.60934
\]

\item Напишите цикл, который повторяется 10 раз. В каждой итерации программа должна:
\begin{enumerate}
\item запрашивать у пользователя скорость в км/ч,
\item создавать экземпляр класса \texttt{SpeedConverter} с этим значением,
\item вычислять скорость в м/с и милях/ч с помощью соответствующих методов,
\item выводить результаты на экран.
\end{enumerate}
\end{enumerate}

\subsection*{Пример использования:}
\begin{verbatim}
kmh = 100
converter = SpeedConverter(kmh)
ms = converter.to_ms()
mph = converter.to_mph()
print(f"Скорость в м/с: {ms}")
print(f"Скорость в милях/ч: {mph}")
\end{verbatim}

\textbf{Вывод:}
\begin{verbatim}
Скорость в м/с: 27.77777777777778
Скорость в милях/ч: 62.13727366498068
\end{verbatim}

\item 

Напишите программу, которая создаёт класс \texttt{AreaConverter} с методами для преобразования площади
из квадратных метров в гектары и акры. Программа должна запрашивать у пользователя площадь в м²
и выводить преобразованные значения.

\subsection*{Инструкции:}
\begin{enumerate}
\item Создайте класс \texttt{AreaConverter} с методом
\texttt{\_\_init\_\_}, который принимает площадь в м² в
качестве аргумента и сохраняет её в атрибуте \texttt{self.\_\_sq\_meters}.

\item Создайте метод \texttt{to\_hectares},
без аргументов, который преобразует площадь в гектары по формуле:
\[
\texttt{\_\_sq\_meters} \div 10000
\]

\item Создайте метод \texttt{to\_acres} без аргументов,
который преобразует площадь в акры по формуле:
\[
\texttt{\_\_sq\_meters} \div 4046.86
\]

\item Напишите цикл, который повторяется 10 раз. В каждой итерации программа должна:
\begin{enumerate}
\item запрашивать у пользователя площадь в м²,
\item создавать экземпляр класса \texttt{AreaConverter} с этим значением,
\item вычислять площадь в гектарах и акрах с помощью соответствующих методов,
\item выводить результаты на экран.
\end{enumerate}
\end{enumerate}

\subsection*{Пример использования:}
\begin{verbatim}
sq_meters = 10000
converter = AreaConverter(sq_meters)
hectares = converter.to_hectares()
acres = converter.to_acres()
print(f"Площадь в гектарах: {hectares}")
print(f"Площадь в акрах: {acres}")
\end{verbatim}

\textbf{Вывод:}
\begin{verbatim}
Площадь в гектары: 1.0
Площадь в акрах: 2.4710514233241505
\end{verbatim}

\item 

Напишите программу, которая создаёт класс \texttt{VolumeConverter} с методами для преобразования объёма
из литров в галлоны и кубические метры. Программа должна запрашивать у пользователя объём в литрах
и выводить преобразованные значения.

\subsection*{Инструкции:}
\begin{enumerate}
\item Создайте класс \texttt{VolumeConverter} с методом
\texttt{\_\_init\_\_}, который принимает объём в литрах в
качестве аргумента и сохраняет его в атрибуте \texttt{self.\_\_liters}.

\item Создайте метод \texttt{to\_gallons},
без аргументов, который преобразует объём в галлоны по формуле:
\[
\texttt{\_\_liters} \div 3.78541
\]

\item Создайте метод \texttt{to\_cubic\_meters} без аргументов,
который преобразует объём в кубические метры по формуле:
\[
\texttt{\_\_liters} \div 1000
\]

\item Напишите цикл, который повторяется 10 раз. В каждой итерации программа должна:
\begin{enumerate}
\item запрашивать у пользователя объём в литрах,
\item создавать экземпляр класса \texttt{VolumeConverter} с этим значением,
\item вычислять объём в галлонах и кубических метрах с помощью соответствующих методов,
\item выводить результаты на экран.
\end{enumerate}
\end{enumerate}

\subsection*{Пример использования:}
\begin{verbatim}
liters = 10
converter = VolumeConverter(liters)
gallons = converter.to_gallons()
cubic_meters = converter.to_cubic_meters()
print(f"Объём в галлонах: {gallons}")
print(f"Объём в кубических метрах: {cubic_meters}")
\end{verbatim}

\textbf{Вывод:}
\begin{verbatim}
Объём в галлонах: 2.641720523581484
Объём в кубических метрах: 0.01
\end{verbatim}

\item

Напишите программу, которая создаёт класс \texttt{EnergyConverter} с методами для преобразования энергии
из джоулей в калории и киловатт-часы. Программа должна запрашивать у пользователя энергию в джоулях
и выводить преобразованные значения.

\subsection*{Инструкции:}
\begin{enumerate}
\item Создайте класс \texttt{EnergyConverter} с методом
\texttt{\_\_init\_\_}, который принимает энергию в джоулях в
качестве аргумента и сохраняет её в атрибуте \texttt{self.\_\_joules}.

\item Создайте метод \texttt{to\_calories},
без аргументов, который преобразует энергию в калории по формуле:
\[
\texttt{\_\_joules} \div 4.184
\]

\item Создайте метод \texttt{to\_kwh} без аргументов,
который преобразует энергию в киловатт-часы по формуле:
\[
\texttt{\_\_joules} \div 3.6 \times 10^6
\]

\item Напишите цикл, который повторяется 10 раз. В каждой итерации программа должна:
\begin{enumerate}
\item запрашивать у пользователя энергию в джоулях,
\item создавать экземпляр класса \texttt{EnergyConverter} с этим значением,
\item вычислять энергию в калориях и киловатт-часах с помощью соответствующих методов,
\item выводить результаты на экран.
\end{enumerate}
\end{enumerate}

\subsection*{Пример использования:}
\begin{verbatim}
joules = 10000
converter = EnergyConverter(joules)
calories = converter.to_calories()
kwh = converter.to_kwh()
print(f"Энергия в калориях: {calories}")
print(f"Энергия в киловатт-часах: {kwh}")
\end{verbatim}

\textbf{Вывод:}
\begin{verbatim}
Энергия в калориях: 2390.057361376673
Энергия в киловатт-часах: 0.002777777777777778
\end{verbatim}

\item 

Напишите программу, которая создаёт класс \texttt{PowerConverter} с методами для преобразования мощности
из ватт в лошадиные силы и киловатты. Программа должна запрашивать у пользователя мощность в ваттах
и выводить преобразованные значения.

\subsection*{Инструкции:}
\begin{enumerate}
\item Создайте класс \texttt{PowerConverter} с методом
\texttt{\_\_init\_\_}, который принимает мощность в ваттах в
качестве аргумента и сохраняет её в атрибуте \texttt{self.\_\_watts}.

\item Создайте метод \texttt{to\_horsepower},
без аргументов, который преобразует мощность в лошадиные силы по формуле:
\[
\texttt{\_\_watts} \div 745.7
\]

\item Создайте метод \texttt{to\_kilowatts} без аргументов,
который преобразует мощность в киловатты по формуле:
\[
\texttt{\_\_watts} \div 1000
\]

\item Напишите цикл, который повторяется 10 раз. В каждой итерации программа должна:
\begin{enumerate}
\item запрашивать у пользователя мощность в ваттах,
\item создавать экземпляр класса \texttt{PowerConverter} с этим значением,
\item вычислять мощность в л.с. и киловаттах с помощью соответствующих методов,
\item выводить результаты на экран.
\end{enumerate}
\end{enumerate}

\subsection*{Пример использования:}
\begin{verbatim}
watts = 1000
converter = PowerConverter(watts)
horsepower = converter.to_horsepower()
kilowatts = converter.to_kilowatts()
print(f"Мощность в л.с.: {horsepower}")
print(f"Мощность в киловаттах: {kilowatts}")
\end{verbatim}

\textbf{Вывод:}
\begin{verbatim}
Мощность в л.с.: 1.3410220903956017
Мощность в киловаттах: 1.0
\end{verbatim}


\item

Напишите программу, которая создаёт класс \texttt{PressureConverter} с методами для преобразования давления
из паскалей в атмосферы и бары. Программа должна запрашивать у пользователя давление в паскалях
и выводить преобразованные значения.

\subsection*{Инструкции:}
\begin{enumerate}
\item Создайте класс \texttt{PressureConverter} с методом
\texttt{\_\_init\_\_}, который принимает давление в паскалях в
качестве аргумента и сохраняет его в атрибуте \texttt{self.\_\_pascals}.

\item Создайте метод \texttt{to\_atm},
без аргументов, который преобразует давление в атмосферы по формуле:
\[
\texttt{\_\_pascals} \div 101325
\]

\item Создайте метод \texttt{to\_bar} без аргументов,
который преобразует давление в бары по формуле:
\[
\texttt{\_\_pascals} \div 100000
\]

\item Напишите цикл, который повторяется 10 раз. В каждой итерации программа должна:
\begin{enumerate}
\item запрашивать у пользователя давление в паскалях,
\item создавать экземпляр класса \texttt{PressureConverter} с этим значением,
\item вычислять давление в атмосферах и барах с помощью соответствующих методов,
\item выводить результаты на экран.
\end{enumerate}
\end{enumerate}

\subsection*{Пример использования:}
\begin{verbatim}
pascals = 101325
converter = PressureConverter(pascals)
atm = converter.to_atm()
bar = converter.to_bar()
print(f"Давление в атмосферах: {atm}")
print(f"Давление в барах: {bar}")
\end{verbatim}

\textbf{Вывод:}
\begin{verbatim}
Давление в атмосферах: 1.0
Давление в барах: 1.01325
\end{verbatim}

\item 

Напишите программу, которая создаёт класс \texttt{ForceConverter} с методами для преобразования силы
из ньютонов в дины и фунты-силы. Программа должна запрашивать у пользователя силу в ньютонах
и выводить преобразованные значения.

\subsection*{Инструкции:}
\begin{enumerate}
\item Создайте класс \texttt{ForceConverter} с методом
\texttt{\_\_init\_\_}, который принимает силу в ньютонах в
качестве аргумента и сохраняет её в атрибуте \texttt{self.\_\_newtons}.

\item Создайте метод \texttt{to\_dyne},
без аргументов, который преобразует силу в дины по формуле:
\[
\texttt{\_\_newtons} \times 100000
\]

\item Создайте метод \texttt{to\_pound\_force} без аргументов,
который преобразует силу в фунты-силы по формуле:
\[
\texttt{\_\_newtons} \div 4.44822
\]

\item Напишите цикл, который повторяется 10 раз. В каждой итерации программа должна:
\begin{enumerate}
\item запрашивать у пользователя силу в ньютонах,
\item создавать экземпляр класса \texttt{ForceConverter} с этим значением,
\item вычислять силу в динах и фунтах-силы с помощью соответствующих методов,
\item выводить результаты на экран.
\end{enumerate}
\end{enumerate}

\subsection*{Пример использования:}
\begin{verbatim}
newtons = 10
converter = ForceConverter(newtons)
dyne = converter.to_dyne()
pound_force = converter.to_pound_force()
print(f"Сила в динах: {dyne}")
print(f"Сила в фунтах-силы: {pound_force}")
\end{verbatim}

\textbf{Вывод:}
\begin{verbatim}
Сила в динах: 1000000.0
Сила в фунтах-силы: 2.248089430997145
\end{verbatim}

\item 

\subsection*{Задание: Конвертер силы}
Напишите программу, которая создаёт класс \texttt{ForceConverter} с методами для преобразования силы
из ньютонов в дины и фунты-силы. Программа должна запрашивать у пользователя силу в ньютонах
и выводить преобразованные значения.

\subsection*{Инструкции:}
\begin{enumerate}
\item Создайте класс \texttt{ForceConverter} с методом
\texttt{\_\_init\_\_}, который принимает силу в ньютонах в
качестве аргумента и сохраняет её в атрибуте \texttt{self.\_\_newtons}.

\item Создайте метод \texttt{to\_dyne},
без аргументов, который преобразует силу в дины по формуле:
\[
\texttt{\_\_newtons} \times 100000
\]

\item Создайте метод \texttt{to\_pound\_force} без аргументов,
который преобразует силу в фунты-силы по формуле:
\[
\texttt{\_\_newtons} \div 4.44822
\]

\item Напишите цикл, который повторяется 10 раз. В каждой итерации программа должна:
\begin{enumerate}
\item запрашивать у пользователя силу в ньютонах,
\item создавать экземпляр класса \texttt{ForceConverter} с этим значением,
\item вычислять силу в динах и фунтах-силы с помощью соответствующих методов,
\item выводить результаты на экран.
\end{enumerate}
\end{enumerate}

\subsection*{Пример использования:}
\begin{verbatim}
newtons = 10
converter = ForceConverter(newtons)
dyne = converter.to_dyne()
pound_force = converter.to_pound_force()
print(f"Сила в динах: {dyne}")
print(f"Сила в фунтах-силы: {pound_force}")
\end{verbatim}

\textbf{Вывод:}
\begin{verbatim}
Сила в динах: 1000000.0
Сила в фунтах-силы: 2.248089430997145
\end{verbatim}

\item

Напишите программу, которая создаёт класс \texttt{ResistanceConverter} с методами для преобразования электрического сопротивления
из омов в килоомы и мегаомы. Программа должна запрашивать у пользователя сопротивление в омах
и выводить преобразованные значения.

\subsection*{Инструкции:}
\begin{enumerate}
\item Создайте класс \texttt{ResistanceConverter} с методом
\texttt{\_\_init\_\_}, который принимает сопротивление в омах в
качестве аргумента и сохраняет его в атрибуте \texttt{self.\_\_ohms}.

\item Создайте метод \texttt{to\_kiloohms},
без аргументов, который преобразует сопротивление в килоомы по формуле:
\[
\texttt{\_\_ohms} \div 1000
\]

\item Создайте метод \texttt{to\_megaohms} без аргументов,
который преобразует сопротивление в мегаомы по формуле:
\[
\texttt{\_\_ohms} \div 1000000
\]

\item Напишите цикл, который повторяется 10 раз. В каждой итерации программа должна:
\begin{enumerate}
\item запрашивать у пользователя сопротивление в омах,
\item создавать экземпляр класса \texttt{ResistanceConverter} с этим значением,
\item вычислять сопротивление в килоомах и мегаомах с помощью соответствующих методов,
\item выводить результаты на экран.
\end{enumerate}
\end{enumerate}

\subsection*{Пример использования:}
\begin{verbatim}
ohms = 10000
converter = ResistanceConverter(ohms)
kiloohms = converter.to_kiloohms()
megaohms = converter.to_megaohms()
print(f"Сопротивление в килоомах: {kiloohms}")
print(f"Сопротивление в мегаомах: {megaohms}")
\end{verbatim}

\textbf{Вывод:}
\begin{verbatim}
Сопротивление в килоомах: 10.0
Сопротивление в мегаомах: 0.01
\end{verbatim}

\item 


\section*{Дополнительные задания}

\item
Напишите программу, которая создаёт класс \texttt{Pentagon} с методами для вычисления площади
и периметра правильного пятиугольника. Программа должна запрашивать у пользователя длину сторону
и выводить вычисленные площадь и периметр.

\subsection*{Инструкции:}
\begin{enumerate}
\item Создайте класс \texttt{Pentagon} с методом
\texttt{\_\_init\_\_}, который принимает длину стороны пятиугольника в
качестве аргумента и сохраняет её в атрибуте \texttt{self.\_\_side}.

\item Создайте метод \texttt{calculate\_area},
без аргументов, который вычисляет площадь правильного пятиугольника по формуле:
\[
\frac{1}{4} \sqrt{5(5 + 2\sqrt{5})} \cdot \texttt{\_\_side}^2
\]

\item Создайте метод \texttt{calculate\_perimeter} без аргументов,
который вычисляет периметр пятиугольника по формуле:
\[
5 \cdot \texttt{\_\_side}
\]

\item Напишите цикл, который повторяется 10 раз. В каждой итерации программа должна:
\begin{enumerate}
\item запрашивать у пользователя длину стороны пятиугольника,
\item создавать экземпляр класса \texttt{Pentagon} с этой длиной,
\item вычислять площадь и периметр с помощью соответствующих методов,
\item выводить результаты на экран.
\end{enumerate}
\end{enumerate}

\subsection*{Пример использования:}
\begin{verbatim}
side = 5
pentagon = Pentagon(side)
area = pentagon.calculate_area()
perimeter = pentagon.calculate_perimeter()
print(f"Площадь пятиугольника: {area}")
print(f"Периметр пятиугольника: {perimeter}")
\end{verbatim}

\textbf{Вывод:}
\begin{verbatim}
Площадь пятиугольника: 43.01193501472417
Периметр пятиугольника: 25
\end{verbatim}

\item
Напишите программу, которая создаёт класс \texttt{Hexagon} с методами для вычисления площади
и периметра правильного шестиугольника. Программа должна запрашивать у пользователя длину стороны
и выводить вычисленные площадь и периметр.

\subsection*{Инструкции:}
\begin{enumerate}
\item Создайте класс \texttt{Hexagon} с методом
\texttt{\_\_init\_\_}, который принимает длину стороны шестиугольника в
качестве аргумента и сохраняет её в атрибуте \texttt{self.\_\_side}.

\item Создайте метод \texttt{calculate\_area},
без аргументов, который вычисляет площадь правильного шестиугольника по формуле:
\[
\frac{3\sqrt{3}}{2} \cdot \texttt{\_\_side}^2
\]

\item Создайте метод \texttt{calculate\_perimeter} без аргументов,
который вычисляет периметр шестиугольника по формуле:
\[
6 \cdot \texttt{\_\_side}
\]

\item Напишите цикл, который повторяется 10 раз. В каждой итерации программа должна:
\begin{enumerate}
\item запрашивать у пользователя длину стороны шестиугольника,
\item создавать экземпляр класса \texttt{Hexagon} с этой длиной,
\item вычислять площадь и периметр с помощью соответствующих методов,
\item выводить результаты на экран.
\end{enumerate}
\end{enumerate}

\subsection*{Пример использования:}
\begin{verbatim}
side = 4
hexagon = Hexagon(side)
area = hexagon.calculate_area()
perimeter = hexagon.calculate_perimeter()
print(f"Площадь шестиугольника: {area}")
print(f"Периметр шестиугольника: {perimeter}")
\end{verbatim}

\textbf{Вывод:}
\begin{verbatim}
Площадь шестиугольника: 41.569219381653056
Периметр шестиугольника: 24
\end{verbatim}

\item
Напишите программу, которая создаёт класс \texttt{AngleConverter} с методами для преобразования углов
из градусов в радианы и грады. Программа должна запрашивать у пользователя угол в градусах
и выводить преобразованные значения.

\subsection*{Инструкции:}
\begin{enumerate}
\item Создайте класс \texttt{AngleConverter} с методом
\texttt{\_\_init\_\_}, который принимает угол в градусах в
качестве аргумента и сохраняет его в атрибуте \texttt{self.\_\_degrees}.

\item Создайте метод \texttt{to\_radians},
без аргументов, который преобразует угол в радианы по формуле:
\[
\texttt{\_\_degrees} \times \frac{\pi}{180}
\]

\item Создайте метод \texttt{to\_gradians} без аргументов,
который преобразует угол в грады по формуле:
\[
\texttt{\_\_degrees} \times \frac{10}{9}
\]

\item Напишите цикл, который повторяется 10 раз. В каждой итерации программа должна:
\begin{enumerate}
\item запрашивать у пользователя угол в градусах,
\item создавать экземпляр класса \texttt{AngleConverter} с этим значением,
\item вычислять угол в радианах и градах с помощью соответствующих методов,
\item выводить результаты на экран.
\end{enumerate}
\end{enumerate}

\subsection*{Пример использования:}
\begin{verbatim}
degrees = 90
converter = AngleConverter(degrees)
radians = converter.to_radians()
gradians = converter.to_gradians()
print(f"Угол в радианах: {radians}")
print(f"Угол в градах: {gradians}")
\end{verbatim}

\textbf{Вывод:}
\begin{verbatim}
Угол в радианах: 1.5707963267948966
Угол в градах: 100.0
\end{verbatim}

\item
Напишите программу, которая создаёт класс \texttt{Tetrahedron} с методами для вычисления объёма
и площади поверхности правильного тетраэдра. Программа должна запрашивать у пользователя длину ребра
и выводить вычисленные объём и площадь поверхности.

\subsection*{Инструкции:}
\begin{enumerate}
\item Создайте класс \texttt{Tetrahedron} с методом
\texttt{\_\_init\_\_}, который принимает длину ребра тетраэдра в
качестве аргумента и сохраняет её в атрибуте \texttt{self.\_\_edge}.

\item Создайте метод \texttt{calculate\_volume},
без аргументов, который вычисляет объём тетраэдра по формуле:
\[
\frac{\texttt{\_\_edge}^3}{6\sqrt{2}}
\]

\item Создайте метод \texttt{calculate\_surface\_area} без аргументов,
который вычисляет площадь поверхности тетраэдра по формуле:
\[
\sqrt{3} \cdot \texttt{\_\_edge}^2
\]

\item Напишите цикл, который повторяется 10 раз. В каждой итерации программа должна:
\begin{enumerate}
\item запрашивать у пользователя длину ребра тетраэдра,
\item создавать экземпляр класса \texttt{Tetrahedron} с этой длиной,
\item вычислять объём и площадь поверхности с помощью соответствующих методов,
\item выводить результаты на экран.
\end{enumerate}
\end{enumerate}

\subsection*{Пример использования:}
\begin{verbatim}
edge = 3
tetrahedron = Tetrahedron(edge)
volume = tetrahedron.calculate_volume()
surface_area = tetrahedron.calculate_surface_area()
print(f"Объём тетраэдра: {volume}")
print(f"Площадь поверхности: {surface_area}")
\end{verbatim}

\textbf{Вывод:}
\begin{verbatim}
Объём тетраэдра: 3.181980515339464
Площадь поверхности: 15.588457268119896
\end{verbatim}

\item
Напишите программу, которая создаёт класс \texttt{CubicMeterConverter} с методами для преобразования объёма
из кубических метров в литры и кубические футы. Программа должна запрашивать у пользователя объём в кубометрах
и выводить преобразованные значения.

\subsection*{Инструкции:}
\begin{enumerate}
\item Создайте класс \texttt{CubicMeterConverter} с методом
\texttt{\_\_init\_\_}, который принимает объём в кубических метрах в
качестве аргумента и сохраняет его в атрибуте \texttt{self.\_\_cubic\_meters}.

\item Создайте метод \texttt{to\_liters},
без аргументов, который преобразует объём в литры по формуле:
\[
\texttt{\_\_cubic\_meters} \times 1000
\]

\item Создайте метод \texttt{\_\_cubic\_feet} без аргументов,
который преобразует объём в кубические футы по формуле:
\[
\texttt{\_\_cubic\_meters} \times 35.3147
\]

\item Напишите цикл, который повторяется 10 раз. В каждой итерации программа должна:
\begin{enumerate}
\item запрашивать у пользователя объём в кубических метрах,
\item создавать экземпляр класса \texttt{CubicMeterConverter} с этим значением,
\item вычислять объём в литрах и кубических футах с помощью соответствующих методов,
\item выводить результаты на экран.
\end{enumerate}
\end{enumerate}

\subsection*{Пример использования:}
\begin{verbatim}
cubic_meters = 2
converter = CubicMeterConverter(cubic_meters)
liters = converter.to_liters()
cubic_feet = converter.to_cubic_feet()
print(f"Объём в литрах: {liters}")
print(f"Объём в кубических футах: {cubic_feet}")
\end{verbatim}

\textbf{Вывод:}
\begin{verbatim}
Объём в литрах: 2000.0
Объём в кубических футах: 70.6294
\end{verbatim}

\item
Напишите программу, которая создаёт класс \texttt{RightTriangle} с методами для вычисления гипотенузы
и площади прямоугольного треугольника. Программа должна запрашивать у пользователя длину одного катета
(второй катет фиксирован и равен 4) и выводить вычисленные гипотенузу и площадь.

\subsection*{Инструкции:}
\begin{enumerate}
\item Создайте класс \texttt{RightTriangle} с методом
\texttt{\_\_init\_\_}, который принимает длину первого катета в
качестве аргумента и сохраняет его в атрибуте \texttt{self.\_\_cathetus}.
Второй катет фиксирован и равен 4.

\item Создайте метод \texttt{calculate\_hypotenuse},
без аргументов, который вычисляет гипотенузу по формуле:
\[
\sqrt{\texttt{\_\_cathetus}^2 + 4^2}
\]

\item Создайте метод \texttt{calculate\_area} без аргументов,
который вычисляет площадь треугольника по формуле:
\[
\frac{\texttt{\_\_cathetus} \times 4}{2}
\]

\item Напишите цикл, который повторяется 10 раз. В каждой итерации программа должна:
\begin{enumerate}
\item запрашивать у пользователя длину катета,
\item создавать экземпляр класса \texttt{RightTriangle} с этой длиной,
\item вычислять гипотенузу и площадь с помощью соответствующих методов,
\item выводить результаты на экран.
\end{enumerate}
\end{enumerate}

\subsection*{Пример использования:}
\begin{verbatim}
cathetus = 3
triangle = RightTriangle(cathetus)
hypotenuse = triangle.calculate_hypotenuse()
area = triangle.calculate_area()
print(f"Гипотенуза: {hypotenuse}")
print(f"Площадь: {area}")
\end{verbatim}

\textbf{Вывод:}
\begin{verbatim}
Гипотенуза: 5.0
Площадь: 6.0
\end{verbatim}

\end{enumerate}

\textbf{Задача 2}

\begin{enumerate}
    \item

Написать программу, которая создаёт класс \texttt{LeapYearChecker} 
для определения високосного года. В классе должен быть статический метод
\texttt{is\_leap\_year} и возвращать \texttt{True}, если год високосный, 
и \texttt{False} в противном случае. 
Программа также должна использовать цикл для проверки каждого года от 
2000 до 2099 и вывода результата на экран.

\subsection*{Инструкции:}
\begin{enumerate}
    \item Создайте класс \texttt{LeapYearChecker}.
    \item Создайте \textbf{статический} метод \texttt{is\_leap\_year}, который принимает год в качестве аргумента и проверяет, является ли год високосным. Если год делится на 4 без остатка и не делится на 100 без остатка, или делится на 400 без остатка, то возвращает \texttt{True}. В противном случае возвращает \texttt{False}.
    \item Используйте цикл для проверки каждого года от 2000 до 2099 (включительно), вызывая статический метод \texttt{is\_leap\_year} и выводя результат на экран.
\end{enumerate}

\subsection*{Пример использования:}
\begin{lstlisting}[language=Python]
    v = LeapYearChecker.is_leap_year(1999)
\end{lstlisting}
Вывод (первые и последние строки):
\begin{verbatim}
2000 True
2001 False
...
2098 False
2099 False
\end{verbatim}

\item
Написать программу, которая создаёт класс \texttt{PrimeChecker} 
для определения простого числа. В классе должен быть статический метод
\texttt{is\_prime} и возвращать \texttt{True}, если число простое, 
и \texttt{False} в противном случае. 
Программа также должна использовать цикл для проверки каждого числа от 
1 до 100 и вывода результата на экран.

\subsection*{Инструкции:}
\begin{enumerate}
    \item Создайте класс \texttt{PrimeChecker}.
    \item Создайте \textbf{статический} метод \texttt{is\_prime}, который принимает число в качестве аргумента и проверяет, является ли число простым. Простое число делится только на 1 и на само себя.
    \item Используйте цикл для проверки каждого числа от 1 до 100 (включительно), вызывая статический метод \texttt{is\_prime} и выводя результат на экран.
\end{enumerate}

\subsection*{Пример использования:}
\begin{lstlisting}[language=Python]
    v = PrimeChecker.is_prime(17)
\end{lstlisting}
Вывод (первые и последние строки):
\begin{verbatim}
1 False
2 True
3 True
...
98 False
99 False
100 False
\end{verbatim}

\item
Написать программу, которая создаёт класс \texttt{EvenChecker} 
для определения чётности числа. В классе должен быть статический метод
\texttt{is\_even} и возвращать \texttt{True}, если число чётное, 
и \texttt{False} в противном случае. 
Программа также должна использовать цикл для проверки каждого числа от 
1 до 50 и вывода результата на экран.

\subsection*{Инструкции:}
\begin{enumerate}
    \item Создайте класс \texttt{EvenChecker}.
    \item Создайте \textbf{статический} метод \texttt{is\_even}, который принимает число в качестве аргумента и проверяет, является ли число чётным.
    \item Используйте цикл для проверки каждого числа от 1 до 50 (включительно), вызывая статический метод \texttt{is\_even} и выводя результат на экран.
\end{enumerate}

\subsection*{Пример использования:}
\begin{lstlisting}[language=Python]
    v = EvenChecker.is_even(25)
\end{lstlisting}
Вывод (первые и последние строки):
\begin{verbatim}
1 False
2 True
3 False
...
48 True
49 False
50 True
\end{verbatim}

\item
Написать программу, которая создаёт класс \texttt{SquareChecker} 
для определения квадратного числа. В классе должен быть статический метод
\texttt{is\_square} и возвращать \texttt{True}, если число является квадратом целого числа, 
и \texttt{False} в противном случае. 
Программа также должна использовать цикл для проверки каждого числа от 
1 до 100 и вывода результата на экран.

\subsection*{Инструкции:}
\begin{enumerate}
    \item Создайте класс \texttt{SquareChecker}.
    \item Создайте \textbf{статический} метод \texttt{is\_square}, который принимает число в качестве аргумента и проверяет, является ли число квадратом целого числа.
    \item Используйте цикл для проверки каждого числа от 1 до 100 (включительно), вызывая статический метод \texttt{is\_square} и выводя результат на экран.
\end{enumerate}

\subsection*{Пример использования:}
\begin{lstlisting}[language=Python]
    v = SquareChecker.is_square(36)
\end{lstlisting}
Вывод (первые и последние строки):
\begin{verbatim}
1 True
2 False
3 False
...
99 False
100 True
\end{verbatim}

\item
Написать программу, которая создаёт класс \texttt{FactorialCalculator} 
для вычисления факториала числа. В классе должен быть статический метод
\texttt{factorial} и возвращать факториал числа. 
Программа также должна использовать цикл для вычисления факториала каждого числа от 
1 до 10 и вывода результата на экран.

\subsection*{Инструкции:}
\begin{enumerate}
    \item Создайте класс \texttt{FactorialCalculator}.
    \item Создайте \textbf{статический} метод \texttt{factorial}, который принимает число в качестве аргумента и возвращает его факториал.
    \item Используйте цикл для вычисления факториала каждого числа от 1 до 10 (включительно), вызывая статический метод \texttt{factorial} и выводя результат на экран.
\end{enumerate}

\subsection*{Пример использования:}
\begin{lstlisting}[language=Python]
    v = FactorialCalculator.factorial(5)
\end{lstlisting}
Вывод (первые и последние строки):
\begin{verbatim}
1 1
2 2
3 6
...
9 362880
10 3628800
\end{verbatim}

\item
Написать программу, которая создаёт класс \texttt{PalindromeChecker} 
для определения палиндрома числа. В классе должен быть статический метод
\texttt{is\_palindrome} и возвращать \texttt{True}, если число является палиндромом, 
и \texttt{False} в противном случае. 
Программа также должна использовать цикл для проверки каждого числа от 
100 до 200 и вывода результата на экран.

\subsection*{Инструкции:}
\begin{enumerate}
    \item Создайте класс \texttt{PalindromeChecker}.
    \item Создайте \textbf{статический} метод \texttt{is\_palindrome}, который принимает число в качестве аргумента и проверяет, является ли число палиндромом (читается одинаково слева направо и справа налево).
    \item Используйте цикл для проверки каждого числа от 100 до 200 (включительно), вызывая статический метод \texttt{is\_palindrome} и выводя результат на экран.
\end{enumerate}

\subsection*{Пример использования:}
\begin{lstlisting}[language=Python]
    v = PalindromeChecker.is_palindrome(121)
\end{lstlisting}
Вывод (первые и последние строки):
\begin{verbatim}
100 False
101 True
102 False
...
199 False
200 False
\end{verbatim}

\item
Написать программу, которая создаёт класс \texttt{ArmstrongChecker} 
для определения числа Армстронга. В классе должен быть статический метод
\texttt{is\_armstrong} и возвращать \texttt{True}, если число является числом Армстронга, 
и \texttt{False} в противном случае. 
Программа также должна использовать цикл для проверки каждого числа от 
100 до 500 и вывода результата на экран.

\subsection*{Инструкции:}
\begin{enumerate}
    \item Создайте класс \texttt{ArmstrongChecker}.
    \item Создайте \textbf{статический} метод \texttt{is\_armstrong}, который принимает число в качестве аргумента и проверяет, является ли число числом Армстронга (сумма цифр в степени, равной количеству цифр, равна самому числу).
    \item Используйте цикл для проверки каждого числа от 100 до 500 (включительно), вызывая статический метод \texttt{is\_armstrong} и выводя результат на экран.
\end{enumerate}

\subsection*{Пример использования:}
\begin{lstlisting}[language=Python]
    v = ArmstrongChecker.is_armstrong(153)
\end{lstlisting}
Вывод (первые и последние строки):
\begin{verbatim}
100 False
101 False
102 False
...
499 False
500 False
\end{verbatim}

\item
Написать программу, которая создаёт класс \texttt{PerfectNumberChecker} 
для определения совершенного числа. В классе должен быть статический метод
\texttt{is\_perfect} и возвращать \texttt{True}, если число является совершенным, 
и \texttt{False} в противном случае. 
Программа также должна использовать цикл для проверки каждого числа от 
1 до 1000 и вывода результата на экран.

\subsection*{Инструкции:}
\begin{enumerate}
    \item Создайте класс \texttt{PerfectNumberChecker}.
    \item Создайте \textbf{статический} метод \texttt{is\_perfect}, который принимает число в качестве аргумента и проверяет, является ли число совершенным (сумма делителей равна числу).
    \item Используйте цикл для проверки каждого числа от 1 до 1000 (включительно), вызывая статический метод \texttt{is\_perfect} и выводя результат на экран.
\end{enumerate}

\subsection*{Пример использования:}
\begin{lstlisting}[language=Python]
    v = PerfectNumberChecker.is_perfect(28)
\end{lstlisting}
Вывод (первые и последние строки):
\begin{verbatim}
1 False
2 False
3 False
...
998 False
999 False
1000 False
\end{verbatim}

\item
Написать программу, которая создаёт класс \texttt{FibonacciChecker} 
для проверки числа Фибоначчи. В классе должен быть статический метод
\texttt{is\_fibonacci} и возвращать \texttt{True}, если число является числом Фибоначчи, 
и \texttt{False} в противном случае. 
Программа также должна использовать цикл для проверки каждого числа от 
1 до 100 и вывода результата на экран.

\subsection*{Инструкции:}
\begin{enumerate}
    \item Создайте класс \texttt{FibonacciChecker}.
    \item Создайте \textbf{статический} метод \texttt{is\_fibonacci}, который принимает число в качестве аргумента и проверяет, является ли число числом Фибоначчи.
    \item Используйте цикл для проверки каждого числа от 1 до 100 (включительно), вызывая статический метод \texttt{is\_fibonacci} и выводя результат на экран.
\end{enumerate}

\subsection*{Пример использования:}
\begin{lstlisting}[language=Python]
    v = FibonacciChecker.is_fibonacci(21)
\end{lstlisting}
Вывод (первые и последние строки):
\begin{verbatim}
1 True
2 True
3 True
...
98 False
99 False
100 False
\end{verbatim}

\item
Написать программу, которая создаёт класс \texttt{PowerOfTwoChecker} 
для проверки степени двойки. В классе должен быть статический метод
\texttt{is\_power\_of\_two} и возвращать \texttt{True}, если число является степенью двойки, 
и \texttt{False} в противном случае. 
Программа также должна использовать цикл для проверки каждого числа от 
1 до 128 и вывода результата на экран.

\subsection*{Инструкции:}
\begin{enumerate}
    \item Создайте класс \texttt{PowerOfTwoChecker}.
    \item Создайте \textbf{статический} метод \texttt{is\_power\_of\_two}, который принимает число в качестве аргумента и проверяет, является ли число степенью двойки.
    \item Используйте цикл для проверки каждого числа от 1 до 128 (включительно), вызывая статический метод \texttt{is\_power\_of\_two} и выводя результат на экран.
\end{enumerate}

\subsection*{Пример использования:}
\begin{lstlisting}[language=Python]
    v = PowerOfTwoChecker.is_power_of_two(64)
\end{lstlisting}
Вывод (первые и последние строки):
\begin{verbatim}
1 True
2 True
3 False
...
127 False
128 True
\end{verbatim}

\item
Написать программу, которая создаёт класс \texttt{SumOfDigitsCalculator} 
для вычисления суммы цифр числа. В классе должен быть статический метод
\texttt{sum\_of\_digits} и возвращать сумму цифр. 
Программа также должна использовать цикл для вычисления суммы цифр каждого числа от 
1 до 50 и вывода результата на экран.

\subsection*{Инструкции:}
\begin{enumerate}
    \item Создайте класс \texttt{SumOfDigitsCalculator}.
    \item Создайте \textbf{статический} метод \texttt{sum\_of\_digits}, который принимает число в качестве аргумента и возвращает сумму его цифр.
    \item Используйте цикл для вычисления суммы цифр каждого числа от 1 до 50 (включительно), вызывая статический метод \texttt{sum\_of\_digits} и выводя результат на экран.
\end{enumerate}

\subsection*{Пример использования:}
\begin{lstlisting}[language=Python]
    v = SumOfDigitsCalculator.sum_of_digits(123)
\end{lstlisting}
Вывод (первые и последние строки):
\begin{verbatim}
1 1
2 2
3 3
...
49 13
50 5
\end{verbatim}

\item
Написать программу, которая создаёт класс \texttt{PrimeSumCalculator} 
для вычисления суммы простых чисел в диапазоне. В классе должен быть статический метод
\texttt{sum\_of\_primes} и возвращать сумму простых чисел в заданном диапазоне. 
Программа также должна использовать цикл для вычисления суммы простых чисел от 
1 до 100 и вывода результата на экран.

\subsection*{Инструкции:}
\begin{enumerate}
    \item Создайте класс \texttt{PrimeSumCalculator}.
    \item Создайте \textbf{статический} метод \texttt{sum\_of\_primes}, который принимает два аргумента (начало и конец диапазона) и возвращает сумму простых чисел в этом диапазоне.
    \item Используйте метод для вычисления суммы простых чисел от 1 до 100 и выведите результат.
\end{enumerate}

\subsection*{Пример использования:}
\begin{lstlisting}[language=Python]
    v = PrimeSumCalculator.sum_of_primes(1, 10)
\end{lstlisting}
Вывод:
\begin{verbatim}
Сумма простых чисел от 1 до 100: 1060
\end{verbatim}

\item
Написать программу, которая создаёт класс \texttt{DigitCountCalculator} 
для подсчёта количества цифр в числе. В классе должен быть статический метод
\texttt{digit\_count} и возвращать количество цифр. 
Программа также должна использовать цикл для подсчёта цифр каждого числа от 
1 до 100 и вывода результата на экран.

\subsection*{Инструкции:}
\begin{enumerate}
    \item Создайте класс \texttt{DigitCountCalculator}.
    \item Создайте \textbf{статический} метод \texttt{digit\_count}, который принимает число в качестве аргумента и возвращает количество его цифр.
    \item Используйте цикл для подсчёта цифр каждого числа от 1 до 100 (включительно), вызывая статический метод \texttt{digit\_count} и выводя результат на экран.
\end{enumerate}

\subsection*{Пример использования:}
\begin{lstlisting}[language=Python]
    v = DigitCountCalculator.digit_count(12345)
\end{lstlisting}
Вывод (первые и последние строки):
\begin{verbatim}
1 1
2 1
3 1
...
99 2
100 3
\end{verbatim}

\item
Написать программу, которая создаёт класс \texttt{BinaryConverter} 
для преобразования числа в двоичное представление. В классе должен быть статический метод
\texttt{to\_binary} и возвращать строку с двоичным представлением числа. 
Программа также должна использовать цикл для преобразования каждого числа от 
1 до 16 и вывода результата на экран.

\subsection*{Инструкции:}
\begin{enumerate}
    \item Создайте класс \texttt{BinaryConverter}.
    \item Создайте \textbf{статический} метод \texttt{to\_binary}, который принимает число в качестве аргумента и возвращает его двоичное представление в виде строки.
    \item Используйте цикл для преобразования каждого числа от 1 до 16 (включительно), вызывая статический метод \texttt{to\_binary} и выводя результат на экран.
\end{enumerate}

\subsection*{Пример использования:}
\begin{lstlisting}[language=Python]
    v = BinaryConverter.to_binary(10)
\end{lstlisting}
Вывод (первые и последние строки):
\begin{verbatim}
1 1
2 10
3 11
...
15 1111
16 10000
\end{verbatim}

\item
Написать программу, которая создаёт класс \texttt{HexConverter} 
для преобразования числа в шестнадцатеричное представление. В классе должен быть статический метод
\texttt{to\_hex} и возвращать строку с шестнадцатеричным представлением числа. 
Программа также должна использовать цикл для преобразования каждого числа от 
1 до 20 и вывода результата на экран.

\subsection*{Инструкции:}
\begin{enumerate}
    \item Создайте класс \texttt{HexConverter}.
    \item Создайте \textbf{статический} метод \texttt{to\_hex}, который принимает число в качестве аргумента и возвращает его шестнадцатеричное представление в виде строки.
    \item Используйте цикл для преобразования каждого числа от 1 до 20 (включительно), вызывая статический метод \texttt{to\_hex} и выводя результат на экран.
\end{enumerate}

\subsection*{Пример использования:}
\begin{lstlisting}[language=Python]
    v = HexConverter.to_hex(255)
\end{lstlisting}
Вывод (первые и последние строки):
\begin{verbatim}
1 1
2 2
3 3
...
19 13
20 14
\end{verbatim}

\item
Написать программу, которая создаёт класс \texttt{DivisorChecker} 
для проверки делителей числа. В классе должен быть статический метод
\texttt{get\_divisors} и возвращать список делителей числа. 
Программа также должна использовать цикл для вывода делителей каждого числа от 
1 до 20 и вывода результата на экран.

\subsection*{Инструкции:}
\begin{enumerate}
    \item Создайте класс \texttt{DivisorChecker}.
    \item Создайте \textbf{статический} метод \texttt{get\_divisors}, который принимает число в качестве аргумента и возвращает список его делителей.
    \item Используйте цикл для вывода делителей каждого числа от 1 до 20 (включительно), вызывая статический метод \texttt{get\_divisors} и выводя результат на экран.
\end{enumerate}

\subsection*{Пример использования:}
\begin{lstlisting}[language=Python]
    v = DivisorChecker.get_divisors(12)
\end{lstlisting}
Вывод (первые и последние строки):
\begin{verbatim}
1 [1]
2 [1, 2]
3 [1, 3]
...
19 [1, 19]
20 [1, 2, 4, 5, 10, 20]
\end{verbatim}

\item
Написать программу, которая создаёт класс \texttt{Multiplier} 
для создания таблицы умножения. В классе должен быть статический метод
\texttt{multiply\_table} и выводить таблицу умножения для заданного числа. 
Программа также должна использовать цикл для вывода таблицы умножения для чисел от 
1 до 10 и вывода результата на экран.

\subsection*{Инструкции:}
\begin{enumerate}
    \item Создайте класс \texttt{Multiplier}.
    \item Создайте \textbf{статический} метод \texttt{multiply\_table}, который принимает число в качестве аргумента и выводит таблицу умножения для этого числа от 1 до 10.
    \item Используйте цикл для вывода таблицы умножения для чисел от 1 до 10 (включительно), вызывая статический метод \texttt{multiply\_table} и выводя результат на экран.
\end{enumerate}

\subsection*{Пример использования:}
\begin{lstlisting}[language=Python]
    Multiplier.multiply_table(5)
\end{lstlisting}
Вывод (для числа 5):
\begin{verbatim}
5 * 1 = 5
5 * 2 = 10
...
5 * 10 = 50
\end{verbatim}

\item
Написать программу, которая создаёт класс \texttt{GCDCalculator} 
для вычисления НОД двух чисел. В классе должен быть статический метод
\texttt{gcd} и возвращать наибольший общий делитель. 
Программа также должна использовать цикл для вычисления НОД чисел 
(1,1), (2,4), (3,9), ..., (10,100) и вывода результата на экран.

\subsection*{Инструкции:}
\begin{enumerate}
    \item Создайте класс \texttt{GCDCalculator}.
    \item Создайте \textbf{статический} метод \texttt{gcd}, который принимает два числа в качестве аргументов и возвращает их наибольший общий делитель.
    \item Используйте цикл для вычисления НОД пар чисел (1,1), (2,4), (3,9), ..., (10,100), вызывая статический метод \texttt{gcd} и выводя результат на экран.
\end{enumerate}

\subsection*{Пример использования:}
\begin{lstlisting}[language=Python]
    v = GCDCalculator.gcd(48, 18)
\end{lstlisting}
Вывод:
\begin{verbatim}
НОД(1, 1) = 1
НОД(2, 4) = 2
НОД(3, 9) = 3
...
НОД(10, 100) = 10
\end{verbatim}

\item
Написать программу, которая создаёт класс \texttt{LCMCalculator} 
для вычисления НОК двух чисел. В классе должен быть статический метод
\texttt{lcm} и возвращать наименьшее общее кратное. 
Программа также должна использовать цикл для вычисления НОК чисел 
(1,1), (2,3), (3,5), ..., (10,11) и вывода результата на экран.

\subsection*{Инструкции:}
\begin{enumerate}
    \item Создайте класс \texttt{LCMCalculator}.
    \item Создайте \textbf{статический} метод \texttt{lcm}, который принимает два числа в качестве аргументов и возвращает их наименьшее общее кратное.
    \item Используйте цикл для вычисления НОК пар чисел (1,1), (2,3), (3,5), ..., (10,11), вызывая статический метод \texttt{lcm} и выводя результат на экран.
\end{enumerate}

\subsection*{Пример использования:}
\begin{lstlisting}[language=Python]
    v = LCMCalculator.lcm(4, 6)
\end{lstlisting}
Вывод:
\begin{verbatim}
НОК(1, 1) = 1
НОК(2, 3) = 6
НОК(3, 5) = 15
...
НОК(10, 11) = 110
\end{verbatim}

\item
Написать программу, которая создаёт класс \texttt{DigitReverse} 
для разворота цифр числа. В классе должен быть статический метод
\texttt{reverse\_digits} и возвращать число с обратным порядком цифр. 
Программа также должна использовать цикл для разворота каждого числа от 
10 до 20 и вывода результата на экран.

\subsection*{Инструкции:}
\begin{enumerate}
    \item Создайте класс \texttt{DigitReverse}.
    \item Создайте \textbf{статический} метод \texttt{reverse\_digits}, который принимает число в качестве аргумента и возвращает число с обратным порядком цифр.
    \item Используйте цикл для разворота каждого числа от 10 до 20 (включительно), вызывая статический метод \texttt{reverse\_digits} и выводя результат на экран.
\end{enumerate}

\subsection*{Пример использования:}
\begin{lstlisting}[language=Python]
    v = DigitReverse.reverse_digits(123)
\end{lstlisting}
Вывод:
\begin{verbatim}
10 1
11 11
12 21
13 31
...
19 91
20 2
\end{verbatim}

\item
Написать программу, которая создаёт класс \texttt{NumberTypeChecker} 
для определения типа числа (положительное/отрицательное/ноль). В классе должен быть статический метод
\texttt{check\_number\_type} и возвращать строку с типом числа. 
Программа также должна использовать цикл для проверки чисел 
[-5, -4, -3, -2, -1, 0, 1, 2, 3, 4, 5] и вывода результата на экран.

\subsection*{Инструкции:}
\begin{enumerate}
    \item Создайте класс \texttt{NumberTypeChecker}.
    \item Создайте \textbf{статический} метод \texttt{check\_number\_type}, который принимает число в качестве аргумента и возвращает строку "positive", "negative" или "zero".
    \item Используйте цикл для проверки чисел [-5, -4, -3, -2, -1, 0, 1, 2, 3, 4, 5], вызывая статический метод \texttt{check\_number\_type} и выводя результат на экран.
\end{enumerate}

\subsection*{Пример использования:}
\begin{lstlisting}[language=Python]
    v = NumberTypeChecker.check_number_type(-7)
\end{lstlisting}
Вывод:
\begin{verbatim}
-5 negative
-4 negative
-3 negative
-2 negative
-1 negative
0 zero
1 positive
2 positive
3 positive
4 positive
5 positive
\end{verbatim}

\item
Написать программу, которая создаёт класс \texttt{FactorialChecker} 
для проверки факториала числа. В классе должен быть статический метод
\texttt{is\_factorial} и возвращать \texttt{True}, если число является факториалом какого-либо числа, 
и \texttt{False} в противном случае. 
Программа также должна использовать цикл для проверки каждого числа от 
1 до 120 и вывода результата на экран.

\subsection*{Инструкции:}
\begin{enumerate}
    \item Создайте класс \texttt{FactorialChecker}.
    \item Создайте \textbf{статический} метод \texttt{is\_factorial}, который принимает число в качестве аргумента и проверяет, является ли число факториалом какого-либо числа.
    \item Используйте цикл для проверки каждого числа от 1 до 120 (включительно), вызывая статический метод \texttt{is\_factorial} и выводя результат на экран.
\end{enumerate}

\subsection*{Пример использования:}
\begin{lstlisting}[language=Python]
    v = FactorialChecker.is_factorial(24)
\end{lstlisting}
Вывод (первые и последние строки):
\begin{verbatim}
1 True
2 True
3 False
...
119 False
120 True
\end{verbatim}

\item
Написать программу, которая создаёт класс \texttt{PowerChecker} 
для проверки степени числа. В классе должен быть статический метод
\texttt{is\_power} и возвращать \texttt{True}, если число является степенью заданного основания, 
и \texttt{False} в противном случае. 
Программа также должна использовать цикл для проверки каждого числа от 
1 до 100 относительно основания 3 и вывода результата на экран.

\subsection*{Инструкции:}
\begin{enumerate}
    \item Создайте класс \texttt{PowerChecker}.
    \item Создайте \textbf{статический} метод \texttt{is\_power}, который принимает число и основание в качестве аргументов и проверяет, является ли число степенью основания.
    \item Используйте цикл для проверки каждого числа от 1 до 100 (включительно) относительно основания 3, вызывая статический метод \texttt{is\_power} и выводя результат на экран.
\end{enumerate}

\subsection*{Пример использования:}
\begin{lstlisting}[language=Python]
    v = PowerChecker.is_power(81, 3)
\end{lstlisting}
Вывод (первые и последние строки):
\begin{verbatim}
1 True
2 False
3 True
...
99 False
100 False
\end{verbatim}

\item
Написать программу, которая создаёт класс \texttt{DigitProductCalculator} 
для вычисления произведения цифр числа. В классе должен быть статический метод
\texttt{digit\_product} и возвращать произведение цифр. 
Программа также должна использовать цикл для вычисления произведения цифр каждого числа от 
1 до 50 и вывода результата на экран.

\subsection*{Инструкции:}
\begin{enumerate}
    \item Создайте класс \texttt{DigitProductCalculator}.
    \item Создайте \textbf{статический} метод \texttt{digit\_product}, который принимает число в качестве аргумента и возвращает произведение его цифр.
    \item Используйте цикл для вычисления произведения цифр каждого числа от 1 до 50 (включительно), вызывая статический метод \texttt{digit\_product} и выводя результат на экран.
\end{enumerate}

\subsection*{Пример использования:}
\begin{lstlisting}[language=Python]
    v = DigitProductCalculator.digit_product(123)
\end{lstlisting}
Вывод (первые и последние строки):
\begin{verbatim}
1 1
2 2
3 3
...
49 36
50 0
\end{verbatim}

\item
Написать программу, которая создаёт класс \texttt{NumberLengthChecker} 
для проверки длины числа. В классе должен быть статический метод
\texttt{get\_length} и возвращать количество цифр в числе. 
Программа также должна использовать цикл для проверки длины каждого числа от 
1 до 1000 с шагом 100 и вывода результата на экран.

\subsection*{Инструкции:}
\begin{enumerate}
    \item Создайте класс \texttt{NumberLengthChecker}.
    \item Создайте \textbf{статический} метод \texttt{get\_length}, который принимает число в качестве аргумента и возвращает количество его цифр.
    \item Используйте цикл для проверки длины чисел 1, 100, 200, 300, 400, 500, 600, 700, 800, 900, 1000, вызывая статический метод \texttt{get\_length} и выводя результат на экран.
\end{enumerate}

\subsection*{Пример использования:}
\begin{lstlisting}[language=Python]
    v = NumberLengthChecker.get_length(12345)
\end{lstlisting}
Вывод:
\begin{verbatim}
1 1
100 3
200 3
300 3
400 3
500 3
600 3
700 3
800 3
900 3
1000 4
\end{verbatim}

\item
Написать программу, которая создаёт класс \texttt{NumberSquareSumCalculator} 
для вычисления суммы квадратов чисел. В классе должен быть статический метод
\texttt{square\_sum} и возвращать сумму квадратов чисел в диапазоне. 
Программа также должна использовать метод для вычисления суммы квадратов чисел от 
1 до 10 и вывода результата на экран.

\subsection*{Инструкции:}
\begin{enumerate}
    \item Создайте класс \texttt{NumberSquareSumCalculator}.
    \item Создайте \textbf{статический} метод \texttt{square\_sum}, который принимает два аргумента (начало и конец диапазона) и возвращает сумму квадратов чисел в этом диапазоне.
    \item Используйте метод для вычисления суммы квадратов чисел от 1 до 10 и выведите результат.
\end{enumerate}

\subsection*{Пример использования:}
\begin{lstlisting}[language=Python]
    v = NumberSquareSumCalculator.square_sum(1, 3)
\end{lstlisting}
Вывод:
\begin{verbatim}
Сумма квадратов чисел от 1 до 10: 385
\end{verbatim}

\item
Написать программу, которая создаёт класс \texttt{NumberCubeSumCalculator} 
для вычисления суммы кубов чисел. В классе должен быть статический метод
\texttt{cube\_sum} и возвращать сумму кубов чисел в диапазоне. 
Программа также должна использовать метод для вычисления суммы кубов чисел от 
1 до 10 и вывода результата на экран.

\subsection*{Инструкции:}
\begin{enumerate}
    \item Создайте класс \texttt{NumberCubeSumCalculator}.
    \item Создайте \textbf{статический} метод \texttt{cube\_sum}, который принимает два аргумента (начало и конец диапазона) и возвращает сумму кубов чисел в этом диапазоне.
    \item Используйте метод для вычисления суммы кубов чисел от 1 до 10 и выведите результат.
\end{enumerate}

\subsection*{Пример использования:}
\begin{lstlisting}[language=Python]
    v = NumberCubeSumCalculator.cube_sum(1, 3)
\end{lstlisting}
Вывод:
\begin{verbatim}
Сумма кубов чисел от 1 до 10: 3025
\end{verbatim}

\item
Написать программу, которая создаёт класс \texttt{NumberRangeChecker} 
для проверки числа на принадлежность диапазону. В классе должен быть статический метод
\texttt{in\_range} и возвращать \texttt{True}, если число находится в заданном диапазоне, 
и \texttt{False} в противном случае. 
Программа также должна использовать цикл для проверки чисел от 
-5 до 5 на принадлежность диапазону [0, 10] и вывода результата на экран.

\subsection*{Инструкции:}
\begin{enumerate}
    \item Создайте класс \texttt{NumberRangeChecker}.
    \item Создайте \textbf{статический} метод \texttt{in\_range}, который принимает число, начало и конец диапазона и проверяет, находится ли число в этом диапазоне.
    \item Используйте цикл для проверки чисел от -5 до 5 (включительно) на принадлежность диапазону [0, 10], вызывая статический метод \texttt{in\_range} и выводя результат на экран.
\end{enumerate}

\subsection*{Пример использования:}
\begin{lstlisting}[language=Python]
    v = NumberRangeChecker.in_range(5, 0, 10)
\end{lstlisting}
Вывод:
\begin{verbatim}
-5 False
-4 False
-3 False
-2 False
-1 False
0 True
1 True
2 True
3 True
4 True
5 True
\end{verbatim}

\item
Написать программу, которая создаёт класс \texttt{NumberSignChecker} 
для проверки знака числа. В классе должен быть статический метод
\texttt{get\_sign} и возвращать строку с знаком числа (+, - или 0). 
Программа также должна использовать цикл для проверки чисел 
[-5, -4, -3, -2, -1, 0, 1, 2, 3, 4, 5] и вывода результата на экран.

\subsection*{Инструкции:}
\begin{enumerate}
    \item Создайте класс \texttt{NumberSignChecker}.
    \item Создайте \textbf{статический} метод \texttt{get\_sign}, который принимает число в качестве аргумента и возвращает строку с его знаком (+, - или 0).
    \item Используйте цикл для проверки чисел [-5, -4, -3, -2, -1, 0, 1, 2, 3, 4, 5], вызывая статический метод \texttt{get\_sign} и выводя результат на экран.
\end{enumerate}

\subsection*{Пример использования:}
\begin{lstlisting}[language=Python]
    v = NumberSignChecker.get_sign(-7)
\end{lstlisting}
Вывод:
\begin{verbatim}
-5 -
-4 -
-3 -
-2 -
-1 -
0 0
1 +
2 +
3 +
4 +
5 +
\end{verbatim}

\item
Написать программу, которая создаёт класс \texttt{NumberPalindromeChecker} 
для проверки палиндрома числа. В классе должен быть статический метод
\texttt{is\_palindrome} и возвращать \texttt{True}, если число является палиндромом, 
и \texttt{False} в противном случае. 
Программа также должна использовать цикл для проверки каждого числа от 
100 до 150 и вывода результата на экран.

\subsection*{Инструкции:}
\begin{enumerate}
    \item Создайте класс \texttt{NumberPalindromeChecker}.
    \item Создайте \textbf{статический} метод \texttt{is\_palindrome}, который принимает число в качестве аргумента и проверяет, является ли число палиндромом.
    \item Используйте цикл для проверки каждого числа от 100 до 150 (включительно), вызывая статический метод \texttt{is\_palindrome} и выводя результат на экран.
\end{enumerate}

\subsection*{Пример использования:}
\begin{lstlisting}[language=Python]
    v = NumberPalindromeChecker.is_palindrome(121)
\end{lstlisting}
Вывод (первые и последние строки):
\begin{verbatim}
100 False
101 True
102 False
...
149 False
150 False
\end{verbatim}

\item
Написать программу, которая создаёт класс \texttt{NumberAscendingChecker} 
для проверки, что цифры числа идут в порядке возрастания. В классе должен быть статический метод
\texttt{is\_ascending} и возвращать \texttt{True}, если цифры числа идут в порядке возрастания, 
и \texttt{False} в противном случае. 
Программа также должна использовать цикл для проверки каждого числа от 
10 до 100 и вывода результата на экран.

\subsection*{Инструкции:}
\begin{enumerate}
    \item Создайте класс \texttt{NumberAscendingChecker}.
    \item Создайте \textbf{статический} метод \texttt{is\_ascending}, который принимает число в качестве аргумента и проверяет, идут ли его цифры в порядке возрастания.
    \item Используйте цикл для проверки каждого числа от 10 до 100 (включительно), вызывая статический метод \texttt{is\_ascending} и выводя результат на экран.
\end{enumerate}

\subsection*{Пример использования:}
\begin{lstlisting}[language=Python]
    v = NumberAscendingChecker.is_ascending(123)
\end{lstlisting}
Вывод (первые и последние строки):
\begin{verbatim}
10 False
11 False
12 True
13 True
...
98 False
99 False
100 False
\end{verbatim}

\item
Написать программу, которая создаёт класс \texttt{NumberDescendingChecker} 
для проверки, что цифры числа идут в порядке убывания. В классе должен быть статический метод
\texttt{is\_descending} и возвращать \texttt{True}, если цифры числа идут в порядке убывания, 
и \texttt{False} в противном случае. 
Программа также должна использовать цикл для проверки каждого числа от 
10 до 100 и вывода результата на экран.

\subsection*{Инструкции:}
\begin{enumerate}
    \item Создайте класс \texttt{NumberDescendingChecker}.
    \item Создайте \textbf{статический} метод \texttt{is\_descending}, который принимает число в качестве аргумента и проверяет, идут ли его цифры в порядке убывания.
    \item Используйте цикл для проверки каждого числа от 10 до 100 (включительно), вызывая статический метод \texttt{is\_descending} и выводя результат на экран.
\end{enumerate}

\subsection*{Пример использования:}
\begin{lstlisting}[language=Python]
    v = NumberDescendingChecker.is_descending(321)
\end{lstlisting}
Вывод (первые и последние строки):
\begin{verbatim}
10 False
11 False
12 False
13 False
...
98 True
99 True
100 False
\end{verbatim}

\item
Написать программу, которая создаёт класс \texttt{NumberPrimeDigitChecker} 
для проверки, что все цифры числа простые. В классе должен быть статический метод
\texttt{all\_digits\_prime} и возвращать \texttt{True}, если все цифры числа простые, 
и \texttt{False} в противном случае. 
Программа также должна использовать цикл для проверки каждого числа от 
10 до 100 и вывода результата на экран.

\subsection*{Инструкции:}
\begin{enumerate}
    \item Создайте класс \texttt{NumberPrimeDigitChecker}.
    \item Создайте \textbf{статический} метод \texttt{all\_digits\_prime}, который принимает число в качестве аргумента и проверяет, являются ли все его цифры простыми числами.
    \item Используйте цикл для проверки каждого числа от 10 до 100 (включительно), вызывая статический метод \texttt{all\_digits\_prime} и выводя результат на экран.
\end{enumerate}

\subsection*{Пример использования:}
\begin{lstlisting}[language=Python]
    v = NumberPrimeDigitChecker.all_digits_prime(23)
\end{lstlisting}
Вывод (первые и последние строки):
\begin{verbatim}
10 False
11 False
12 False
13 False
...
98 False
99 False
100 False
\end{verbatim}

\item
Написать программу, которая создаёт класс \texttt{NumberEvenDigitChecker} 
для проверки, что все цифры числа чётные. В классе должен быть статический метод
\texttt{all\_digits\_even} и возвравать \texttt{True}, если все цифры числа чётные, 
и \texttt{False} в противном случае. 
Программа также должна использовать цикл для проверки каждого числа от 
10 до 100 и вывода результата на экран.

\subsection*{Инструкции:}
\begin{enumerate}
    \item Создайте класс \texttt{NumberEvenDigitChecker}.
    \item Создайте \textbf{статический} метод \texttt{all\_digits\_even}, который принимает число в качестве аргумента и проверяет, являются ли все его цифры чётными.
    \item Используйте цикл для проверки каждого числа от 10 до 100 (включительно), вызывая статический метод \texttt{all\_digits\_even} и выводя результат на экран.
\end{enumerate}

\subsection*{Пример использования:}
\begin{lstlisting}[language=Python]
    v = NumberEvenDigitChecker.all_digits_even(24)
\end{lstlisting}
Вывод (первые и последние строки):
\begin{verbatim}
10 False
11 False
12 False
13 False
...
98 False
99 False
100 False
\end{verbatim}

\item
Написать программу, которая создаёт класс \texttt{NumberOddDigitChecker} 
для проверки, что все цифры числа нечётные. В классе должен быть статический метод
\texttt{all\_digits\_odd} и возвращать \texttt{True}, если все цифры числа нечётные, 
и \texttt{False} в противном случае. 
Программа также должна использовать цикл для проверки каждого числа от 
10 до 100 и вывода результата на экран.

\subsection*{Инструкции:}
\begin{enumerate}
    \item Создайте класс \texttt{NumberOddDigitChecker}.
    \item Создайте \textbf{статический} метод \texttt{all\_digits\_odd}, который принимает число в качестве аргумента и проверяет, являются ли все его цифры нечётными.
    \item Используйте цикл для проверки каждого числа от 10 до 100 (включительно), вызывая статический метод \texttt{all\_digits\_odd} и выводя результат на экран.
\end{enumerate}

\subsection*{Пример использования:}
\begin{lstlisting}[language=Python]
    v = NumberOddDigitChecker.all_digits_odd(135)
\end{lstlisting}
Вывод (первые и последние строки):
\begin{verbatim}
10 False
11 True
12 False
13 True
...
98 False
99 True
100 False
\end{verbatim}


\end{enumerate}

\textbf{Задача 3}

\begin{enumerate}

\item
Написать программу на Python, которая создает класс \texttt{Person} для представления сотрудника персонала. Класс должен содержать закрытые атрибуты \texttt{\_\_name}, \texttt{\_\_country}, \texttt{\_\_date\_of\_birth} и метод \texttt{calculate\_age}. Доступ к атрибутам только через методы-геттеры. Создать экземпляры и вывести информацию о каждом человеке.

\subsection*{Инструкции:}
\begin{enumerate}
    \item Создайте класс \texttt{Person} с методом \texttt{\_\_init\_\_}, который принимает имя, страну и дату рождения.
    \item Создайте методы-геттеры: \texttt{get\_name()}, \texttt{get\_country()}, \texttt{get\_date\_of\_birth()}.
    \item Создайте метод \texttt{calculate\_age()} для вычисления возраста.
    \item Создайте несколько экземпляров класса \texttt{Person}.
    \item Выведите данные каждого человека через методы класса.
\end{enumerate}

\subsection*{Пример использования:}
\begin{lstlisting}[caption=Пример кода]
from datetime import date

person1 = Person("Иванов Иван Иванович", "Россия", date(1946, 8, 15))
person2 = Person("Петров Сергей Александрович", "Белоруссия", date(1982, 10, 22))

print("Персона 1:")
print("Имя: ", person1.get_name())
print("Страна: ", person1.get_country())
print("Дата рождения: ", person1.get_date_of_birth())
print("Возраст: ", person1.calculate_age())

print("Персона 2:")
print("Имя: ", person2.get_name())
print("Страна: ", person2.get_country())
print("Дата рождения: ", person2.get_date_of_birth())
print("Возраст: ", person2.calculate_age())
\end{lstlisting}

\subsection*{Вывод:}
\begin{lstlisting}[caption=Ожидаемый вывод]
Персона 1:
Имя:  Иванов Иван Иванович
Страна:  Россия
Дата рождения:  1946-08-15
Возраст:  77
Персона 2:
Имя:  Петров Сергей Александрович
Страна:  Белоруссия
Дата рождения:  1982-10-22
Возраст:  41
\end{lstlisting}

% ================= Вариант 2 =================
\item
Создайте класс \texttt{Student} с закрытыми атрибутами \texttt{\_\_full\_name}, \texttt{\_\_enrollment\_date}, \texttt{\_\_major}. Реализуйте методы-геттеры и метод \texttt{study\_duration()} для вычисления количества лет с момента зачисления.

\subsection*{Инструкции:}
\begin{enumerate}
    \item Создайте класс \texttt{Student} с методом \texttt{\_\_init\_\_}.
    \item Методы-геттеры: \texttt{get\_full\_name()}, \texttt{get\_enrollment\_date()}, \texttt{get\_major()}.
    \item Метод \texttt{study\_duration()} вычисляет количество лет с зачисления.
    \item Создайте несколько экземпляров класса.
    \item Выведите данные каждого студента.
\end{enumerate}

\subsection*{Пример использования:}
\begin{lstlisting}[caption=Пример кода]
from datetime import date

student1 = Student("Сидоров Алексей", date(2018, 9, 1), "Математика")
student2 = Student("Иванова Мария", date(2021, 9, 1), "Физика")

print("Студент 1:")
print("Имя: ", student1.get_full_name())
print("Направление: ", student1.get_major())
print("Дата зачисления: ", student1.get_enrollment_date())
print("Стаж учёбы: ", student1.study_duration())

print("Студент 2:")
print("Имя: ", student2.get_full_name())
print("Направление: ", student2.get_major())
print("Дата зачисления: ", student2.get_enrollment_date())
print("Стаж учёбы: ", student2.study_duration())
\end{lstlisting}

\subsection*{Вывод:}
\begin{lstlisting}[caption=Ожидаемый вывод]
Студент 1:
Имя:  Сидоров Алексей
Направление:  Математика
Дата зачисления:  2018-09-01
Стаж учёбы:  5
Студент 2:
Имя:  Иванова Мария
Направление:  Физика
Дата зачисления:  2021-09-01
Стаж учёбы:  2
\end{lstlisting}

\item
Создайте класс \texttt{Employee} с закрытыми атрибутами \texttt{\_\_name}, \texttt{\_\_position}, \texttt{\_\_hire\_date}. Реализуйте методы-геттеры и метод \texttt{work\_experience()} для вычисления количества лет работы.

\subsection*{Инструкции:}
\begin{enumerate}
    \item Создайте класс \texttt{Employee} с методом \texttt{\_\_init\_\_}.
    \item Методы-геттеры: \texttt{get\_name()}, \texttt{get\_position()}, \texttt{get\_hire\_date()}.
    \item Метод \texttt{work\_experience()} вычисляет стаж в годах.
    \item Создайте несколько экземпляров класса.
    \item Выведите данные каждого сотрудника.
\end{enumerate}

\subsection*{Пример использования:}
\begin{lstlisting}[caption=Пример кода]
from datetime import date

emp1 = Employee("Кузнецов Дмитрий", "Инженер", date(2010, 5, 10))
emp2 = Employee("Смирнова Ольга", "Менеджер", date(2015, 8, 1))

print("Сотрудник 1:")
print("Имя: ", emp1.get_name())
print("Должность: ", emp1.get_position())
print("Дата приёма: ", emp1.get_hire_date())
print("Стаж: ", emp1.work_experience())

print("Сотрудник 2:")
print("Имя: ", emp2.get_name())
print("Должность: ", emp2.get_position())
print("Дата приёма: ", emp2.get_hire_date())
print("Стаж: ", emp2.work_experience())
\end{lstlisting}

\subsection*{Вывод:}
\begin{lstlisting}[caption=Ожидаемый вывод]
Сотрудник 1:
Имя:  Кузнецов Дмитрий
Должность:  Инженер
Дата приёма:  2010-05-10
Стаж:  17
Сотрудник 2:
Имя:  Смирнова Ольга
Должность:  Менеджер
Дата приёма:  2015-08-01
Стаж:  8
\end{lstlisting}

% ================= Вариант 4 =================
\item
Создайте класс \texttt{Book} с закрытыми атрибутами \texttt{\_\_title}, \texttt{\_\_author}, \texttt{\_\_publish\_date}. Реализуйте геттеры и метод \texttt{book\_age()} для вычисления возраста книги.

\subsection*{Инструкции:}
\begin{enumerate}
    \item Создайте класс \texttt{Book}.
    \item Методы-геттеры: \texttt{get\_title()}, \texttt{get\_author()}, \texttt{get\_publish\_date()}.
    \item Метод \texttt{book\_age()} вычисляет возраст книги.
    \item Создайте экземпляры класса.
    \item Выведите данные каждой книги.
\end{enumerate}

\subsection*{Пример использования:}
\begin{lstlisting}[caption=Пример кода]
from datetime import date

book1 = Book("Программирование на Python", "Иванов И.И.", date(2015, 3, 10))
book2 = Book("Алгебра", "Петров П.П.", date(2000, 9, 1))

print("Книга 1:")
print("Название: ", book1.get_title())
print("Автор: ", book1.get_author())
print("Дата публикации: ", book1.get_publish_date())
print("Возраст книги: ", book1.book_age())

print("Книга 2:")
print("Название: ", book2.get_title())
print("Автор: ", book2.get_author())
print("Дата публикации: ", book2.get_publish_date())
print("Возраст книги: ", book2.book_age())
\end{lstlisting}

\subsection*{Вывод:}
\begin{lstlisting}[caption=Ожидаемый вывод]
Книга 1:
Название:  Программирование на Python
Автор:  Иванов И.И.
Дата публикации:  2015-03-10
Возраст книги:  8
Книга 2:
Название:  Алгебра
Автор:  Петров П.П.
Дата публикации:  2000-09-01
Возраст книги:  23
\end{lstlisting}

% ================= Вариант 5 =================
\item
Создайте класс \texttt{Car} с закрытыми атрибутами \texttt{\_\_model}, \texttt{\_\_manufacturer}, \texttt{\_\_production\_date}. Геттеры и метод \texttt{car\_age()} для вычисления возраста автомобиля.

\subsection*{Инструкции:}
\begin{enumerate}
    \item Создайте класс \texttt{Car}.
    \item Методы-геттеры: \texttt{get\_model()}, \texttt{get\_manufacturer()}, \texttt{get\_production\_date()}.
    \item Метод \texttt{car\_age()} вычисляет возраст автомобиля.
    \item Создайте экземпляры класса.
    \item Выведите данные каждого автомобиля.
\end{enumerate}

\subsection*{Пример использования:}
\begin{lstlisting}[caption=Пример кода]
from datetime import date

car1 = Car("Camry", "Toyota", date(2012, 6, 15))
car2 = Car("Focus", "Ford", date(2018, 4, 20))

print("Автомобиль 1:")
print("Модель: ", car1.get_model())
print("Производитель: ", car1.get_manufacturer())
print("Дата выпуска: ", car1.get_production_date())
print("Возраст авто: ", car1.car_age())

print("Автомобиль 2:")
print("Модель: ", car2.get_model())
print("Производитель: ", car2.get_manufacturer())
print("Дата выпуска: ", car2.get_production_date())
print("Возраст авто: ", car2.car_age())
\end{lstlisting}

\subsection*{Вывод:}
\begin{lstlisting}[caption=Ожидаемый вывод]
Автомобиль 1:
Модель:  Camry
Производитель:  Toyota
Дата выпуска:  2012-06-15
Возраст авто:  11
Автомобиль 2:
Модель:  Focus
Производитель:  Ford
Дата выпуска:  2018-04-20
Возраст авто:  5
\end{lstlisting}

% ================= Вариант 6 =================
\item
Создайте класс \texttt{Pet} с закрытыми атрибутами \texttt{\_\_name}, \texttt{\_\_species}, \texttt{\_\_birth\_date}. Реализуйте методы-геттеры и метод \texttt{pet\_age()} для вычисления возраста питомца. Создайте несколько экземпляров и выведите их данные.

\subsection*{Инструкции:}
\begin{enumerate}
    \item Создайте класс \texttt{Pet} с методом \texttt{\_\_init\_\_}.
    \item Методы-геттеры: \texttt{get\_name()}, \texttt{get\_species()}, \texttt{get\_birth\_date()}.
    \item Метод \texttt{pet\_age()} вычисляет возраст питомца в годах.
    \item Создайте несколько экземпляров класса.
    \item Выведите данные каждого питомца через методы класса.
\end{enumerate}

\subsection*{Пример использования:}
\begin{lstlisting}[caption=Пример кода]
from datetime import date

pet1 = Pet("Барсик", "Кошка", date(2018, 5, 12))
pet2 = Pet("Рекс", "Собака", date(2015, 8, 1))

print("Питомец 1:")
print("Имя: ", pet1.get_name())
print("Вид: ", pet1.get_species())
print("Дата рождения: ", pet1.get_birth_date())
print("Возраст: ", pet1.pet_age())

print("Питомец 2:")
print("Имя: ", pet2.get_name())
print("Вид: ", pet2.get_species())
print("Дата рождения: ", pet2.get_birth_date())
print("Возраст: ", pet2.pet_age())
\end{lstlisting}

\subsection*{Вывод:}
\begin{lstlisting}[caption=Ожидаемый вывод]
Питомец 1:
Имя:  Барсик
Вид:  Кошка
Дата рождения:  2018-05-12
Возраст:  7
Питомец 2:
Имя:  Рекс
Вид:  Собака
Дата рождения:  2015-08-01
Возраст:  10
\end{lstlisting}

% ================= Вариант 7 =================
\item
Создайте класс \texttt{Membership} с закрытыми атрибутами \texttt{\_\_member\_name}, \texttt{\_\_membership\_type}, \texttt{\_\_join\_date}. Реализуйте методы-геттеры и метод \texttt{membership\_duration()} для вычисления длительности членства в годах.

\subsection*{Инструкции:}
\begin{enumerate}
    \item Создайте класс \texttt{Membership}.
    \item Методы-геттеры: \texttt{get\_member\_name()}, \texttt{get\_membership\_type()}, \texttt{get\_join\_date()}.
    \item Метод \texttt{membership\_duration()} вычисляет длительность членства в годах.
    \item Создайте несколько экземпляров.
    \item Выведите данные каждого участника.
\end{enumerate}

\subsection*{Пример использования:}
\begin{lstlisting}[caption=Пример кода]
from datetime import date

member1 = Membership("Иванов Иван", "Золотой", date(2018, 3, 15))
member2 = Membership("Петров Петр", "Серебряный", date(2020, 6, 1))

print("Член 1:")
print("Имя: ", member1.get_member_name())
print("Тип членства: ", member1.get_membership_type())
print("Дата вступления: ", member1.get_join_date())
print("Длительность членства: ", member1.membership_duration())

print("Член 2:")
print("Имя: ", member2.get_member_name())
print("Тип членства: ", member2.get_membership_type())
print("Дата вступления: ", member2.get_join_date())
print("Длительность членства: ", member2.membership_duration())
\end{lstlisting}

\subsection*{Вывод:}
\begin{lstlisting}[caption=Ожидаемый вывод]
Член 1:
Имя:  Иванов Иван
Тип членства:  Золотой
Дата вступления:  2018-03-15
Длительность членства:  5
Член 2:
Имя:  Петров Петр
Тип членства:  Серебряный
Дата вступления:  2020-06-01
Длительность членства:  3
\end{lstlisting}

% ================= Вариант 8 =================
\item
Создайте класс \texttt{Event} с закрытыми атрибутами \texttt{\_\_event\_name}, \texttt{\_\_location}, \texttt{\_\_event\_date}. Реализуйте методы-геттеры и метод \texttt{days\_until\_event()} для вычисления количества дней до события.

\subsection*{Инструкции:}
\begin{enumerate}
    \item Создайте класс \texttt{Event}.
    \item Методы-геттеры: \texttt{get\_event\_name()}, \texttt{get\_location()}, \texttt{get\_event\_date()}.
    \item Метод \texttt{days\_until\_event()} вычисляет дни до события.
    \item Создайте несколько экземпляров.
    \item Выведите данные каждого события.
\end{enumerate}

\subsection*{Пример использования:}
\begin{lstlisting}[caption=Пример кода]
from datetime import date

event1 = Event("Концерт", "Стадион", date(2025, 12, 1))
event2 = Event("Выставка", "Музей", date(2025, 11, 20))

print("Событие 1:")
print("Название: ", event1.get_event_name())
print("Место: ", event1.get_location())
print("Дата: ", event1.get_event_date())
print("Дней до события: ", event1.days_until_event())

print("Событие 2:")
print("Название: ", event2.get_event_name())
print("Место: ", event2.get_location())
print("Дата: ", event2.get_event_date())
print("Дней до события: ", event2.days_until_event())
\end{lstlisting}

\subsection*{Вывод:}
\begin{lstlisting}[caption=Ожидаемый вывод]
Событие 1:
Название:  Концерт
Место:  Стадион
Дата:  2025-12-01
Дней до события:  112
Событие 2:
Название:  Выставка
Место:  Музей
Дата:  2025-11-20
Дней до события:  101
\end{lstlisting}

% ================= Вариант 9 =================
\item
Создайте класс \texttt{Course} с закрытыми атрибутами \texttt{\_\_course\_name}, \texttt{\_\_start\_date}, \texttt{\_\_duration\_weeks}. Реализуйте методы-геттеры и метод \texttt{weeks\_elapsed()} для вычисления прошедших недель с начала курса.

\subsection*{Инструкции:}
\begin{enumerate}
    \item Создайте класс \texttt{Course}.
    \item Методы-геттеры: \texttt{get\_course\_name()}, \texttt{get\_start\_date()}, \texttt{get\_duration\_weeks()}.
    \item Метод \texttt{weeks\_elapsed()} вычисляет количество прошедших недель.
    \item Создайте несколько экземпляров.
    \item Выведите данные каждого курса.
\end{enumerate}

\subsection*{Пример использования:}
\begin{lstlisting}[caption=Пример кода]
from datetime import date

course1 = Course("Python", date(2025, 1, 1), 12)
course2 = Course("Алгебра", date(2025, 2, 1), 10)

print("Курс 1:")
print("Название: ", course1.get_course_name())
print("Дата начала: ", course1.get_start_date())
print("Продолжительность (недель): ", course1.get_duration_weeks())
print("Прошло недель: ", course1.weeks_elapsed())

print("Курс 2:")
print("Название: ", course2.get_course_name())
print("Дата начала: ", course2.get_start_date())
print("Продолжительность (недель): ", course2.get_duration_weeks())
print("Прошло недель: ", course2.weeks_elapsed())
\end{lstlisting}

\subsection*{Вывод:}
\begin{lstlisting}[caption=Ожидаемый вывод]
Курс 1:
Название:  Python
Дата начала:  2025-01-01
Продолжительность (недель):  12
Прошло недель:  36
Курс 2:
Название:  Алгебра
Дата начала:  2025-02-01
Продолжительность (недель):  10
Прошло недель:  31
\end{lstlisting}

% ================= Вариант 10 =================
\item
Создайте класс \texttt{Subscription} с закрытыми атрибутами \texttt{\_\_user}, \texttt{\_\_plan}, \texttt{\_\_start\_date}. Реализуйте методы-геттеры и метод \texttt{subscription\_age()} для вычисления возраста подписки в годах.

\subsection*{Инструкции:}
\begin{enumerate}
    \item Создайте класс \texttt{Subscription}.
    \item Методы-геттеры: \texttt{get\_user()}, \texttt{get\_plan()}, \texttt{get\_start\_date()}.
    \item Метод \texttt{subscription\_age()} вычисляет возраст подписки.
    \item Создайте несколько экземпляров.
    \item Выведите данные каждой подписки.
\end{enumerate}

\subsection*{Пример использования:}
\begin{lstlisting}[caption=Пример кода]
from datetime import date

sub1 = Subscription("Иванов И.", "Premium", date(2021, 3, 1))
sub2 = Subscription("Петров П.", "Basic", date(2022, 7, 15))

print("Подписка 1:")
print("Пользователь: ", sub1.get_user())
print("План: ", sub1.get_plan())
print("Дата начала: ", sub1.get_start_date())
print("Возраст подписки: ", sub1.subscription_age())

print("Подписка 2:")
print("Пользователь: ", sub2.get_user())
print("План: ", sub2.get_plan())
print("Дата начала: ", sub2.get_start_date())
print("Возраст подписки: ", sub2.subscription_age())
\end{lstlisting}

\subsection*{Вывод:}
\begin{lstlisting}[caption=Ожидаемый вывод]
Подписка 1:
Пользователь:  Иванов И.
План:  Premium
Дата начала:  2021-03-01
Возраст подписки:  4
Подписка 2:
Пользователь:  Петров П.
План:  Basic
Дата начала:  2022-07-15
Возраст подписки:  3
\end{lstlisting}

% ================= Вариант 11 =================
\item
Создайте класс \texttt{Flight} с закрытыми атрибутами \texttt{\_\_flight\_number}, \texttt{\_\_departure\_date}, \texttt{\_\_destination}. Реализуйте методы-геттеры и метод \texttt{days\_until\_departure()} для вычисления количества дней до вылета.

\subsection*{Инструкции:}
\begin{enumerate}
    \item Создайте класс \texttt{Flight}.
    \item Методы-геттеры: \texttt{get\_flight\_number()}, \texttt{get\_departure\_date()}, \texttt{get\_destination()}.
    \item Метод \texttt{days\_until\_departure()} вычисляет количество дней до вылета.
    \item Создайте несколько экземпляров.
    \item Выведите данные каждого рейса.
\end{enumerate}

\subsection*{Пример использования:}
\begin{lstlisting}[caption=Пример кода]
from datetime import date

flight1 = Flight("SU123", date(2025, 10, 15), "Москва")
flight2 = Flight("AF456", date(2025, 11, 1), "Париж")

print("Рейс 1:")
print("Номер: ", flight1.get_flight_number())
print("Дата вылета: ", flight1.get_departure_date())
print("Пункт назначения: ", flight1.get_destination())
print("Дней до вылета: ", flight1.days_until_departure())

print("Рейс 2:")
print("Номер: ", flight2.get_flight_number())
print("Дата вылета: ", flight2.get_departure_date())
print("Пункт назначения: ", flight2.get_destination())
print("Дней до вылета: ", flight2.days_until_departure())
\end{lstlisting}

\subsection*{Вывод:}
\begin{lstlisting}[caption=Ожидаемый вывод]
Рейс 1:
Номер:  SU123
Дата вылета:  2025-10-15
Пункт назначения:  Москва
Дней до вылета:  54
Рейс 2:
Номер:  AF456
Дата вылета:  2025-11-01
Пункт назначения:  Париж
Дней до вылета:  71
\end{lstlisting}

% ================= Вариант 12 =================
\item
Создайте класс \texttt{Project} с закрытыми атрибутами \texttt{\_\_project\_name}, \texttt{\_\_start\_date}, \texttt{\_\_deadline}. Реализуйте методы-геттеры и метод \texttt{days\_remaining()} для вычисления количества дней до завершения проекта.

\subsection*{Инструкции:}
\begin{enumerate}
    \item Создайте класс \texttt{Project}.
    \item Методы-геттеры: \texttt{get\_project\_name()}, \texttt{get\_start\_date()}, \texttt{get\_deadline()}.
    \item Метод \texttt{days\_remaining()} вычисляет дни до дедлайна.
    \item Создайте несколько экземпляров.
    \item Выведите данные каждого проекта.
\end{enumerate}

\subsection*{Пример использования:}
\begin{lstlisting}[caption=Пример кода]
from datetime import date

project1 = Project("Разработка сайта", date(2025, 9, 1), date(2025, 12, 1))
project2 = Project("Анализ данных", date(2025, 10, 1), date(2025, 11, 30))

print("Проект 1:")
print("Название: ", project1.get_project_name())
print("Дата начала: ", project1.get_start_date())
print("Дедлайн: ", project1.get_deadline())
print("Дней до завершения: ", project1.days_remaining())

print("Проект 2:")
print("Название: ", project2.get_project_name())
print("Дата начала: ", project2.get_start_date())
print("Дедлайн: ", project2.get_deadline())
print("Дней до завершения: ", project2.days_remaining())
\end{lstlisting}

\subsection*{Вывод:}
\begin{lstlisting}[caption=Ожидаемый вывод]
Проект 1:
Название:  Разработка сайта
Дата начала:  2025-09-01
Дедлайн:  2025-12-01
Дней до завершения:  101
Проект 2:
Название:  Анализ данных
Дата начала:  2025-10-01
Дедлайн:  2025-11-30
Дней до завершения:  91
\end{lstlisting}

% ================= Вариант 13 =================
\item
Создайте класс \texttt{Doctor} с закрытыми атрибутами \texttt{\_\_full\_name}, \texttt{\_\_specialty}, \texttt{\_\_birth\_date}. Реализуйте методы-геттеры и метод \texttt{calculate\_age()} для вычисления возраста врача.

\subsection*{Инструкции:}
\begin{enumerate}
    \item Создайте класс \texttt{Doctor}.
    \item Методы-геттеры: \texttt{get\_full\_name()}, \texttt{get\_specialty()}, \texttt{get\_birth\_date()}.
    \item Метод \texttt{calculate\_age()} вычисляет возраст.
    \item Создайте несколько экземпляров.
    \item Выведите данные каждого врача.
\end{enumerate}

\subsection*{Пример использования:}
\begin{lstlisting}[caption=Пример кода]
from datetime import date

doc1 = Doctor("Иванов И.И.", "Терапевт", date(1980, 5, 12))
doc2 = Doctor("Петров П.П.", "Хирург", date(1975, 8, 1))

print("Врач 1:")
print("Имя: ", doc1.get_full_name())
print("Специальность: ", doc1.get_specialty())
print("Дата рождения: ", doc1.get_birth_date())
print("Возраст: ", doc1.calculate_age())

print("Врач 2:")
print("Имя: ", doc2.get_full_name())
print("Специальность: ", doc2.get_specialty())
print("Дата рождения: ", doc2.get_birth_date())
print("Возраст: ", doc2.calculate_age())
\end{lstlisting}

\subsection*{Вывод:}
\begin{lstlisting}[caption=Ожидаемый вывод]
Врач 1:
Имя:  Иванов И.И.
Специальность:  Терапевт
Дата рождения:  1980-05-12
Возраст:  45
Врач 2:
Имя:  Петров П.П.
Специальность:  Хирург
Дата рождения:  1975-08-01
Возраст:  50
\end{lstlisting}

% ================= Вариант 14 =================
\item
Создайте класс \texttt{Patient} с закрытыми атрибутами \texttt{\_\_full\_name}, \texttt{\_\_admission\_date}, \texttt{\_\_diagnosis}. Реализуйте методы-геттеры и метод \texttt{hospital\_stay()} для вычисления количества дней пребывания в больнице.

\subsection*{Инструкции:}
\begin{enumerate}
    \item Создайте класс \texttt{Patient}.
    \item Методы-геттеры: \texttt{get\_full\_name()}, \texttt{get\_admission\_date()}, \texttt{get\_diagnosis()}.
    \item Метод \texttt{hospital\_stay()} вычисляет дни пребывания.
    \item Создайте несколько экземпляров.
    \item Выведите данные каждого пациента.
\end{enumerate}

\subsection*{Пример использования:}
\begin{lstlisting}[caption=Пример кода]
from datetime import date

patient1 = Patient("Сидоров С.С.", date(2025, 9, 1), "ОРВИ")
patient2 = Patient("Кузнецов К.К.", date(2025, 8, 28), "Грипп")

print("Пациент 1:")
print("Имя: ", patient1.get_full_name())
print("Дата госпитализации: ", patient1.get_admission_date())
print("Диагноз: ", patient1.get_diagnosis())
print("Дней в больнице: ", patient1.hospital_stay())

print("Пациент 2:")
print("Имя: ", patient2.get_full_name())
print("Дата госпитализации: ", patient2.get_admission_date())
print("Диагноз: ", patient2.get_diagnosis())
print("Дней в больнице: ", patient2.hospital_stay())
\end{lstlisting}

\subsection*{Вывод:}
\begin{lstlisting}[caption=Ожидаемый вывод]
Пациент 1:
Имя:  Сидоров С.С.
Дата госпитализации:  2025-09-01
Диагноз:  ОРВИ
Дней в больнице:  15
Пациент 2:
Имя:  Кузнецов К.К.
Дата госпитализации:  2025-08-28
Диагноз:  Грипп
Дней в больнице:  19
\end{lstlisting}

% ================= Вариант 15 =================
\item
Создайте класс \texttt{Concert} с закрытыми атрибутами \texttt{\_\_artist}, \texttt{\_\_venue}, \texttt{\_\_concert\_date}. Реализуйте методы-геттеры и метод \texttt{days\_until\_concert()}.

\subsection*{Инструкции:}
\begin{enumerate}
    \item Создайте класс \texttt{Concert}.
    \item Методы-геттеры: \texttt{get\_artist()}, \texttt{get\_venue()}, \texttt{get\_concert\_date()}.
    \item Метод \texttt{days\_until\_concert()} вычисляет дни до концерта.
    \item Создайте несколько экземпляров.
    \item Выведите данные каждого концерта.
\end{enumerate}

\subsection*{Пример использования:}
\begin{lstlisting}[caption=Пример кода]
from datetime import date

concert1 = Concert("Imagine Dragons", "Лужники", date(2025, 10, 10))
concert2 = Concert("Coldplay", "O2 Arena", date(2025, 11, 5))

print("Концерт 1:")
print("Исполнитель: ", concert1.get_artist())
print("Место: ", concert1.get_venue())
print("Дата: ", concert1.get_concert_date())
print("Дней до концерта: ", concert1.days_until_concert())

print("Концерт 2:")
print("Исполнитель: ", concert2.get_artist())
print("Место: ", concert2.get_venue())
print("Дата: ", concert2.get_concert_date())
print("Дней до концерта: ", concert2.days_until_concert())
\end{lstlisting}

\subsection*{Вывод:}
\begin{lstlisting}[caption=Ожидаемый вывод]
Концерт 1:
Исполнитель:  Imagine Dragons
Место:  Лужники
Дата:  2025-10-10
Дней до концерта:  49
Концерт 2:
Исполнитель:  Coldplay
Место:  O2 Arena
Дата:  2025-11-05
Дней до концерта:  75
\end{lstlisting}

% ================= Вариант 16 =================
\item
Создайте класс \texttt{Holiday} с закрытыми атрибутами \texttt{\_\_name}, \texttt{\_\_country}, \texttt{\_\_holiday\_date}. Реализуйте методы-геттеры и метод \texttt{days\_until\_holiday()}.

\subsection*{Инструкции:}
\begin{enumerate}
    \item Создайте класс \texttt{Holiday}.
    \item Методы-геттеры: \texttt{get\_name()}, \texttt{get\_country()}, \texttt{get\_holiday\_date()}.
    \item Метод \texttt{days\_until\_holiday()} вычисляет дни до праздника.
    \item Создайте несколько экземпляров.
    \item Выведите данные каждого праздника.
\end{enumerate}

\subsection*{Пример использования:}
\begin{lstlisting}[caption=Пример кода]
from datetime import date

holiday1 = Holiday("Новый Год", "Россия", date(2026, 1, 1))
holiday2 = Holiday("Рождество", "Германия", date(2025, 12, 25))

print("Праздник 1:")
print("Название: ", holiday1.get_name())
print("Страна: ", holiday1.get_country())
print("Дата: ", holiday1.get_holiday_date())
print("Дней до праздника: ", holiday1.days_until_holiday())

print("Праздник 2:")
print("Название: ", holiday2.get_name())
print("Страна: ", holiday2.get_country())
print("Дата: ", holiday2.get_holiday_date())
print("Дней до праздника: ", holiday2.days_until_holiday())
\end{lstlisting}

\subsection*{Вывод:}
\begin{lstlisting}[caption=Ожидаемый вывод]
Праздник 1:
Название:  Новый Год
Страна:  Россия
Дата:  2026-01-01
Дней до праздника:  83
Праздник 2:
Название:  Рождество
Страна:  Германия
Дата:  2025-12-25
Дней до праздника:  67
\end{lstlisting}
% ================= Вариант 17 =================
\item
Создайте класс \texttt{Employee} с закрытыми атрибутами \texttt{\_\_full\_name}, \texttt{\_\_position}, \texttt{\_\_hire\_date}. Реализуйте методы-геттеры и метод \texttt{years\_worked()} для вычисления стажа работы в годах.

\subsection*{Инструкции:}
\begin{enumerate}
    \item Создайте класс \texttt{Employee}.
    \item Методы-геттеры: \texttt{get\_full\_name()}, \texttt{get\_position()}, \texttt{get\_hire\_date()}.
    \item Метод \texttt{years\_worked()} вычисляет стаж в годах.
    \item Создайте несколько экземпляров.
    \item Выведите данные каждого сотрудника.
\end{enumerate}

\subsection*{Пример использования:}
\begin{lstlisting}[caption=Пример кода]
from datetime import date

emp1 = Employee("Иванов И.И.", "Менеджер", date(2015, 4, 1))
emp2 = Employee("Петров П.П.", "Разработчик", date(2018, 7, 15))

print("Сотрудник 1:")
print("Имя: ", emp1.get_full_name())
print("Должность: ", emp1.get_position())
print("Дата приема: ", emp1.get_hire_date())
print("Стаж: ", emp1.years_worked())

print("Сотрудник 2:")
print("Имя: ", emp2.get_full_name())
print("Должность: ", emp2.get_position())
print("Дата приема: ", emp2.get_hire_date())
print("Стаж: ", emp2.years_worked())
\end{lstlisting}

\subsection*{Вывод:}
\begin{lstlisting}[caption=Ожидаемый вывод]
Сотрудник 1:
Имя:  Иванов И.И.
Должность:  Менеджер
Дата приема:  2015-04-01
Стаж:  10
Сотрудник 2:
Имя:  Петров П.П.
Должность:  Разработчик
Дата приема:  2018-07-15
Стаж:  7
\end{lstlisting}

% ================= Вариант 18 =================
\item
Создайте класс \texttt{LibraryBook} с закрытыми атрибутами \texttt{\_\_title}, \texttt{\_\_author}, \texttt{\_\_publication\_date}. Реализуйте методы-геттеры и метод \texttt{book\_age()} для вычисления возраста книги.

\subsection*{Инструкции:}
\begin{enumerate}
    \item Создайте класс \texttt{LibraryBook}.
    \item Методы-геттеры: \texttt{get\_title()}, \texttt{get\_author()}, \texttt{get\_publication\_date()}.
    \item Метод \texttt{book\_age()} вычисляет возраст книги в годах.
    \item Создайте несколько экземпляров.
    \item Выведите данные каждой книги.
\end{enumerate}

\subsection*{Пример использования:}
\begin{lstlisting}[caption=Пример кода]
from datetime import date

book1 = LibraryBook("Война и мир", "Толстой", date(1869, 1, 1))
book2 = LibraryBook("Мастер и Маргарита", "Булгаков", date(1967, 5, 1))

print("Книга 1:")
print("Название: ", book1.get_title())
print("Автор: ", book1.get_author())
print("Дата публикации: ", book1.get_publication_date())
print("Возраст книги: ", book1.book_age())

print("Книга 2:")
print("Название: ", book2.get_title())
print("Автор: ", book2.get_author())
print("Дата публикации: ", book2.get_publication_date())
print("Возраст книги: ", book2.book_age())
\end{lstlisting}

\subsection*{Вывод:}
\begin{lstlisting}[caption=Ожидаемый вывод]
Книга 1:
Название:  Война и мир
Автор:  Толстой
Дата публикации:  1869-01-01
Возраст книги:  156
Книга 2:
Название:  Мастер и Маргарита
Автор:  Булгаков
Дата публикации:  1967-05-01
Возраст книги:  59
\end{lstlisting}

% ================= Вариант 19 =================
\item
Создайте класс \texttt{Vehicle} с закрытыми атрибутами \texttt{\_\_brand}, \texttt{\_\_model}, \texttt{\_\_manufacture\_date}. Реализуйте методы-геттеры и метод \texttt{vehicle\_age()}.

\subsection*{Инструкции:}
\begin{enumerate}
    \item Создайте класс \texttt{Vehicle}.
    \item Методы-геттеры: \texttt{get\_brand()}, \texttt{get\_model()}, \texttt{get\_manufacture\_date()}.
    \item Метод \texttt{vehicle\_age()} вычисляет возраст транспортного средства.
    \item Создайте несколько экземпляров.
    \item Выведите данные каждого транспортного средства.
\end{enumerate}

\subsection*{Пример использования:}
\begin{lstlisting}[caption=Пример кода]
from datetime import date

vehicle1 = Vehicle("Toyota", "Camry", date(2015, 5, 1))
vehicle2 = Vehicle("BMW", "X5", date(2018, 3, 10))

print("Транспорт 1:")
print("Марка: ", vehicle1.get_brand())
print("Модель: ", vehicle1.get_model())
print("Дата производства: ", vehicle1.get_manufacture_date())
print("Возраст: ", vehicle1.vehicle_age())

print("Транспорт 2:")
print("Марка: ", vehicle2.get_brand())
print("Модель: ", vehicle2.get_model())
print("Дата производства: ", vehicle2.get_manufacture_date())
print("Возраст: ", vehicle2.vehicle_age())
\end{lstlisting}

\subsection*{Вывод:}
\begin{lstlisting}[caption=Ожидаемый вывод]
Транспорт 1:
Марка:  Toyota
Модель:  Camry
Дата производства:  2015-05-01
Возраст:  10
Транспорт 2:
Марка:  BMW
Модель:  X5
Дата производства:  2018-03-10
Возраст:  7
\end{lstlisting}

% ================= Вариант 20 =================
\item
Создайте класс \texttt{Student} с закрытыми атрибутами \texttt{\_\_full\_name}, \texttt{\_\_enrollment\_date}, \texttt{\_\_major}. Реализуйте методы-геттеры и метод \texttt{study\_years()}.

\subsection*{Инструкции:}
\begin{enumerate}
    \item Создайте класс \texttt{Student}.
    \item Методы-геттеры: \texttt{get\_full\_name()}, \texttt{get\_enrollment\_date()}, \texttt{get\_major()}.
    \item Метод \texttt{study\_years()} вычисляет количество лет учебы.
    \item Создайте несколько экземпляров.
    \item Выведите данные каждого студента.
\end{enumerate}

\subsection*{Пример использования:}
\begin{lstlisting}[caption=Пример кода]
from datetime import date

student1 = Student("Иванов И.И.", date(2020, 9, 1), "Математика")
student2 = Student("Петров П.П.", date(2021, 9, 1), "Физика")

print("Студент 1:")
print("Имя: ", student1.get_full_name())
print("Дата зачисления: ", student1.get_enrollment_date())
print("Специальность: ", student1.get_major())
print("Лет учебы: ", student1.study_years())

print("Студент 2:")
print("Имя: ", student2.get_full_name())
print("Дата зачисления: ", student2.get_enrollment_date())
print("Специальность: ", student2.get_major())
print("Лет учебы: ", student2.study_years())
\end{lstlisting}

\subsection*{Вывод:}
\begin{lstlisting}[caption=Ожидаемый вывод]
Студент 1:
Имя:  Иванов И.И.
Дата зачисления:  2020-09-01
Специальность:  Математика
Лет учебы:  5
Студент 2:
Имя:  Петров П.П.
Дата зачисления:  2021-09-01
Специальность:  Физика
Лет учебы:  4
\end{lstlisting}

% ================= Вариант 21 =================
\item
Создайте класс \texttt{Ticket} с закрытыми атрибутами \texttt{\_\_ticket\_number}, \texttt{\_\_issue\_date}, \texttt{\_\_valid\_until}. Реализуйте методы-геттеры и метод \texttt{days\_valid()}.

\subsection*{Инструкции:}
\begin{enumerate}
    \item Создайте класс \texttt{Ticket}.
    \item Методы-геттеры: \texttt{get\_ticket\_number()}, \texttt{get\_issue\_date()}, \texttt{get\_valid\_until()}.
    \item Метод \texttt{days\_valid()} вычисляет дни до окончания действия билета.
    \item Создайте несколько экземпляров.
    \item Выведите данные каждого билета.
\end{enumerate}

\subsection*{Пример использования:}
\begin{lstlisting}[caption=Пример кода]
from datetime import date

ticket1 = Ticket("A123", date(2025, 9, 1), date(2025, 12, 1))
ticket2 = Ticket("B456", date(2025, 10, 1), date(2026, 1, 1))

print("Билет 1:")
print("Номер: ", ticket1.get_ticket_number())
print("Дата выдачи: ", ticket1.get_issue_date())
print("Действителен до: ", ticket1.get_valid_until())
print("Дней до окончания: ", ticket1.days_valid())

print("Билет 2:")
print("Номер: ", ticket2.get_ticket_number())
print("Дата выдачи: ", ticket2.get_issue_date())
print("Действителен до: ", ticket2.get_valid_until())
print("Дней до окончания: ", ticket2.days_valid())
\end{lstlisting}

\subsection*{Вывод:}
\begin{lstlisting}[caption=Ожидаемый вывод]
Билет 1:
Номер:  A123
Дата выдачи:  2025-09-01
Действителен до:  2025-12-01
Дней до окончания:  91
Билет 2:
Номер:  B456
Дата выдачи:  2025-10-01
Действителен до:  2026-01-01
Дней до окончания:  92
\end{lstlisting}

% ================= Вариант 22 =================
\item
Создайте класс \texttt{Appointment} с закрытыми атрибутами \texttt{\_\_client}, \texttt{\_\_service}, \texttt{\_\_appointment\_date}. Реализуйте методы-геттеры и метод \texttt{days\_until\_appointment()}.
\subsection*{Инструкции:}
\begin{enumerate}
    \item Создайте класс \texttt{Appointment}.
    \item Методы-геттеры: \texttt{get\_client()}, \texttt{get\_service()}, \texttt{get\_appointment\_date()}.
    \item Метод \texttt{days\_until\_appointment()} вычисляет дни до приёма.
    \item Создайте несколько экземпляров.
    \item Выведите данные каждого приёма.
\end{enumerate}

\subsection*{Пример использования:}
\begin{lstlisting}[caption=Пример кода]
from datetime import date

app1 = Appointment("Иванов И.", "Массаж", date(2025, 10, 5))
app2 = Appointment("Петров П.", "Стрижка", date(2025, 10, 15))

print("Приём 1:")
print("Клиент: ", app1.get_client())
print("Услуга: ", app1.get_service())
print("Дата: ", app1.get_appointment_date())
print("Дней до приёма: ", app1.days_until_appointment())

print("Приём 2:")
print("Клиент: ", app2.get_client())
print("Услуга: ", app2.get_service())
print("Дата: ", app2.get_appointment_date())
print("Дней до приёма: ", app2.days_until_appointment())
\end{lstlisting}

\subsection*{Вывод:}
\begin{lstlisting}[caption=Ожидаемый вывод]
Приём 1:
Клиент:  Иванов И.
Услуга:  Массаж
Дата:  2025-10-05
Дней до приёма:  44
Приём 2:
Клиент:  Петров П.
Услуга:  Стрижка
Дата:  2025-10-15
Дней до приёма:  54
\end{lstlisting}
% ================= Вариант 23 =================
\item
Создайте класс \texttt{Subscription} с закрытыми атрибутами \texttt{\_\_subscriber}, \texttt{\_\_start\_date}, \texttt{\_\_end\_date}. Реализуйте методы-геттеры и метод \texttt{days\_remaining()}.

\subsection*{Инструкции:}
\begin{enumerate}
    \item Создайте класс \texttt{Subscription}.
    \item Методы-геттеры: \texttt{get\_subscriber()}, \texttt{get\_start\_date()}, \texttt{get\_end\_date()}.
    \item Метод \texttt{days\_remaining()} вычисляет дни до окончания подписки.
    \item Создайте несколько экземпляров.
    \item Выведите данные каждой подписки.
\end{enumerate}

\subsection*{Пример использования:}
\begin{lstlisting}[caption=Пример кода]
from datetime import date

sub1 = Subscription("Иванов И.", date(2025, 1, 1), date(2025, 12, 31))
sub2 = Subscription("Петров П.", date(2025, 6, 1), date(2026, 5, 31))

print("Подписка 1:")
print("Абонент: ", sub1.get_subscriber())
print("Дата начала: ", sub1.get_start_date())
print("Дата окончания: ", sub1.get_end_date())
print("Дней до окончания: ", sub1.days_remaining())

print("Подписка 2:")
print("Абонент: ", sub2.get_subscriber())
print("Дата начала: ", sub2.get_start_date())
print("Дата окончания: ", sub2.get_end_date())
print("Дней до окончания: ", sub2.days_remaining())
\end{lstlisting}

\subsection*{Вывод:}
\begin{lstlisting}[caption=Ожидаемый вывод]
Подписка 1:
Абонент:  Иванов И.
Дата начала:  2025-01-01
Дата окончания:  2025-12-31
Дней до окончания:  113
Подписка 2:
Абонент:  Петров П.
Дата начала:  2025-06-01
Дата окончания:  2026-05-31
Дней до окончания:  245
\end{lstlisting}

% ================= Вариант 24 =================
\item
Создайте класс \texttt{MembershipCard} с закрытыми атрибутами \texttt{\_\_owner}, \texttt{\_\_issue\_date}, \texttt{\_\_expiry\_date}. Реализуйте методы-геттеры и метод \texttt{days\_until\_expiry()}.

\subsection*{Инструкции:}
\begin{enumerate}
    \item Создайте класс \texttt{MembershipCard}.
    \item Методы-геттеры: \texttt{get\_owner()}, \texttt{get\_issue\_date()}, \texttt{get\_expiry\_date()}.
    \item Метод \texttt{days\_until\_expiry()} вычисляет дни до истечения действия карты.
    \item Создайте несколько экземпляров.
    \item Выведите данные каждой карты.
\end{enumerate}

\subsection*{Пример использования:}
\begin{lstlisting}[caption=Пример кода]
from datetime import date

card1 = MembershipCard("Иванов И.", date(2025, 1, 1), date(2026, 1, 1))
card2 = MembershipCard("Петров П.", date(2025, 5, 1), date(2026, 5, 1))

print("Карта 1:")
print("Владелец: ", card1.get_owner())
print("Дата выдачи: ", card1.get_issue_date())
print("Срок действия: ", card1.get_expiry_date())
print("Дней до окончания: ", card1.days_until_expiry())

print("Карта 2:")
print("Владелец: ", card2.get_owner())
print("Дата выдачи: ", card2.get_issue_date())
print("Срок действия: ", card2.get_expiry_date())
print("Дней до окончания: ", card2.days_until_expiry())
\end{lstlisting}

\subsection*{Вывод:}
\begin{lstlisting}[caption=Ожидаемый вывод]
Карта 1:
Владелец:  Иванов И.
Дата выдачи:  2025-01-01
Срок действия:  2026-01-01
Дней до окончания:  113
Карта 2:
Владелец:  Петров П.
Дата выдачи:  2025-05-01
Срок действия:  2026-05-01
Дней до окончания:  204
\end{lstlisting}

% ================= Вариант 25 =================
\item
Создайте класс \texttt{Event} с закрытыми атрибутами \texttt{\_\_title}, \texttt{\_\_location}, \texttt{\_\_event\_date}. Реализуйте методы-геттеры и метод \texttt{days\_until\_event()}.

\subsection*{Инструкции:}
\begin{enumerate}
    \item Создайте класс \texttt{Event}.
    \item Методы-геттеры: \texttt{get\_title()}, \texttt{get\_location()}, \texttt{get\_event\_date()}.
    \item Метод \texttt{days\_until\_event()} вычисляет дни до события.
    \item Создайте несколько экземпляров.
    \item Выведите данные каждого события.
\end{enumerate}

\subsection*{Пример использования:}
\begin{lstlisting}[caption=Пример кода]
from datetime import date

event1 = Event("Фестиваль науки", "Москва", date(2025, 10, 20))
event2 = Event("Конференция IT", "Санкт-Петербург", date(2025, 11, 10))

print("Событие 1:")
print("Название: ", event1.get_title())
print("Место: ", event1.get_location())
print("Дата: ", event1.get_event_date())
print("Дней до события: ", event1.days_until_event())

print("Событие 2:")
print("Название: ", event2.get_title())
print("Место: ", event2.get_location())
print("Дата: ", event2.get_event_date())
print("Дней до события: ", event2.days_until_event())
\end{lstlisting}

\subsection*{Вывод:}
\begin{lstlisting}[caption=Ожидаемый вывод]
Событие 1:
Название:  Фестиваль науки
Место:  Москва
Дата:  2025-10-20
Дней до события:  59
Событие 2:
Название:  Конференция IT
Место:  Санкт-Петербург
Дата:  2025-11-10
Дней до события:  80
\end{lstlisting}

% ================= Вариант 26 =================
\item
Создайте класс \texttt{CarRental} с закрытыми атрибутами \texttt{\_\_client}, \texttt{\_\_rental\_date}, \texttt{\_\_return\_date}. Реализуйте методы-геттеры и метод \texttt{rental\_duration()}.

\subsection*{Инструкции:}
\begin{enumerate}
    \item Создайте класс \texttt{CarRental}.
    \item Методы-геттеры: \texttt{get\_client()}, \texttt{get\_rental\_date()}, \texttt{get\_return\_date()}.
    \item Метод \texttt{rental\_duration()} вычисляет длительность аренды в днях.
    \item Создайте несколько экземпляров.
    \item Выведите данные каждой аренды.
\end{enumerate}

\subsection*{Пример использования:}
\begin{lstlisting}[caption=Пример кода]
from datetime import date

rental1 = CarRental("Иванов И.", date(2025, 10, 1), date(2025, 10, 10))
rental2 = CarRental("Петров П.", date(2025, 11, 1), date(2025, 11, 5))

print("Аренда 1:")
print("Клиент: ", rental1.get_client())
print("Дата аренды: ", rental1.get_rental_date())
print("Дата возврата: ", rental1.get_return_date())
print("Длительность аренды: ", rental1.rental_duration())

print("Аренда 2:")
print("Клиент: ", rental2.get_client())
print("Дата аренды: ", rental2.get_rental_date())
print("Дата возврата: ", rental2.get_return_date())
print("Длительность аренды: ", rental2.rental_duration())
\end{lstlisting}

\subsection*{Вывод:}
\begin{lstlisting}[caption=Ожидаемый вывод]
Аренда 1:
Клиент:  Иванов И.
Дата аренды:  2025-10-01
Дата возврата:  2025-10-10
Длительность аренды:  9
Аренда 2:
Клиент:  Петров П.
Дата аренды:  2025-11-01
Дата возврата:  2025-11-05
Длительность аренды:  4
\end{lstlisting}

% ================= Вариант 27 =================
\item
Создайте класс \texttt{Visa} с закрытыми атрибутами \texttt{\_\_holder}, \texttt{\_\_issue\_date}, \texttt{\_\_expiry\_date}. Реализуйте методы-геттеры и метод \texttt{days\_until\_expiry()}.

\subsection*{Инструкции:}
\begin{enumerate}
    \item Создайте класс \texttt{Visa}.
    \item Методы-геттеры: \texttt{get\_holder()}, \texttt{get\_issue\_date()}, \texttt{get\_expiry\_date()}.
    \item Метод \texttt{days\_until\_expiry()} вычисляет дни до окончания визы.
    \item Создайте несколько экземпляров.
    \item Выведите данные каждой визы.
\end{enumerate}

\subsection*{Пример использования:}
\begin{lstlisting}[caption=Пример кода]
from datetime import date

visa1 = Visa("Иванов И.", date(2025, 1, 1), date(2026, 1, 1))
visa2 = Visa("Петров П.", date(2025, 6, 1), date(2026, 6, 1))

print("Виза 1:")
print("Держатель: ", visa1.get_holder())
print("Дата выдачи: ", visa1.get_issue_date())
print("Дата окончания: ", visa1.get_expiry_date())
print("Дней до окончания: ", visa1.days_until_expiry())

print("Виза 2:")
print("Держатель: ", visa2.get_holder())
print("Дата выдачи: ", visa2.get_issue_date())
print("Дата окончания: ", visa2.get_expiry_date())
print("Дней до окончания: ", visa2.days_until_expiry())
\end{lstlisting}

\subsection*{Вывод:}
\begin{lstlisting}[caption=Ожидаемый вывод]
Виза 1:
Держатель:  Иванов И.
Дата выдачи:  2025-01-01
Дата окончания:  2026-01-01
Дней до окончания:  113
Виза 2:
Держатель:  Петров П.
Дата выдачи:  2025-06-01
Дата окончания:  2026-06-01
Дней до окончания:  204
\end{lstlisting}

% ================= Вариант 28 =================
\item
Создайте класс \texttt{Reservation} с закрытыми атрибутами \texttt{\_\_guest}, \texttt{\_\_checkin\_date}, \texttt{\_\_checkout\_date}. Реализуйте методы-геттеры и метод \texttt{stay\_duration()}.

\subsection*{Инструкции:}
\begin{enumerate}
    \item Создайте класс \texttt{Reservation}.
    \item Методы-геттеры: \texttt{get\_guest()}, \texttt{get\_checkin\_date()}, \texttt{get\_checkout\_date()}.
    \item Метод \texttt{stay\_duration()} вычисляет продолжительность пребывания в днях.
    \item Создайте несколько экземпляров.
    \item Выведите данные каждой брони.
\end{enumerate}

\subsection*{Пример использования:}
\begin{lstlisting}[caption=Пример кода]
from datetime import date

res1 = Reservation("Иванов И.", date(2025, 10, 1), date(2025, 10, 7))
res2 = Reservation("Петров П.", date(2025, 11, 5), date(2025, 11, 12))

print("Бронь 1:")
print("Гость: ", res1.get_guest())
print("Дата заезда: ", res1.get_checkin_date())
print("Дата выезда: ", res1.get_checkout_date())
print("Продолжительность пребывания: ", res1.stay_duration())

print("Бронь 2:")
print("Гость: ", res2.get_guest())
print("Дата заезда: ", res2.get_checkin_date())
print("Дата выезда: ", res2.get_checkout_date())
print("Продолжительность пребывания: ", res2.stay_duration())
\end{lstlisting}

\subsection*{Вывод:}
\begin{lstlisting}[caption=Ожидаемый вывод]
Бронь 1:
Гость:  Иванов И.
Дата заезда:  2025-10-01
Дата выезда:  2025-10-07
Продолжительность пребывания:  6
Бронь 2:
Гость:  Петров П.
Дата заезда:  2025-11-05
Дата выезда:  2025-11-12
Продолжительность пребывания:  7
\end{lstlisting}

% ================= Вариант 29 =================
\item
Создайте класс \texttt{Conference} с закрытыми атрибутами \texttt{\_\_name}, \texttt{\_\_city}, \texttt{\_\_start\_date}. Реализуйте методы-геттеры и метод \texttt{days\_until\_start()}.

\subsection*{Инструкции:}
\begin{enumerate}
    \item Создайте класс \texttt{Conference}.
    \item Методы-геттеры: \texttt{get\_name()}, \texttt{get\_city()}, \texttt{get\_start\_date()}.
    \item Метод \texttt{days\_until\_start()} вычисляет дни до начала конференции.
    \item Создайте несколько экземпляров.
    \item Выведите данные каждой конференции.
\end{enumerate}

\subsection*{Пример использования:}
\begin{lstlisting}[caption=Пример кода]
from datetime import date

conf1 = Conference("PythonConf", "Москва", date(2025, 10, 20))
conf2 = Conference("DataScience Summit", "Санкт-Петербург", date(2025, 11, 15))

print("Конференция 1:")
print("Название: ", conf1.get_name())
print("Город: ", conf1.get_city())
print("Дата начала: ", conf1.get_start_date())
print("Дней до начала: ", conf1.days_until_start())

print("Конференция 2:")
print("Название: ", conf2.get_name())
print("Город: ", conf2.get_city())
print("Дата начала: ", conf2.get_start_date())
print("Дней до начала: ", conf2.days_until_start())
\end{lstlisting}

\subsection*{Вывод:}
\begin{lstlisting}[caption=Ожидаемый вывод]
Конференция 1:
Название:  PythonConf
Город:  Москва
Дата начала:  2025-10-20
Дней до начала:  59
Конференция 2:
Название:  DataScience Summit
Город:  Санкт-Петербург
Дата начала:  2025-11-15
Дней до начала:  85
\end{lstlisting}

% ================= Вариант 30 =================
\item
Создайте класс \texttt{Medication} с закрытыми атрибутами \texttt{\_\_name}, \texttt{\_\_manufacturer}, \texttt{\_\_expiry\_date}. Реализуйте методы-геттеры и метод \texttt{days\_until\_expiry()}.

\subsection*{Инструкции:}
\begin{enumerate}
    \item Создайте класс \texttt{Medication}.
    \item Методы-геттеры: \texttt{get\_name()}, \texttt{get\_manufacturer()}, \texttt{get\_expiry\_date()}.
    \item Метод \texttt{days\_until\_expiry()} вычисляет дни до окончания срока годности.
    \item Создайте несколько экземпляров.
    \item Выведите данные каждого лекарства.
\end{enumerate}

\subsection*{Пример использования:}
\begin{lstlisting}[caption=Пример кода]
from datetime import date

med1 = Medication("Парацетамол", "Фармком", date(2026, 1, 1))
med2 = Medication("Ибупрофен", "БиоФарм", date(2025, 12, 1))

print("Лекарство 1:")
print("Название: ", med1.get_name())
print("Производитель: ", med1.get_manufacturer())
print("Срок годности: ", med1.get_expiry_date())
print("Дней до окончания: ", med1.days_until_expiry())

print("Лекарство 2:")
print("Название: ", med2.get_name())
print("Производитель: ", med2.get_manufacturer())
print("Срок годности: ", med2.get_expiry_date())
print("Дней до окончания: ", med2.days_until_expiry())
\end{lstlisting}

\subsection*{Вывод:}
\begin{lstlisting}[caption=Ожидаемый вывод]
Лекарство 1:
Название:  Парацетамол
Производитель:  Фармком
Срок годности:  2026-01-01
Дней до окончания:  113
Лекарство 2:
Название:  Ибупрофен
Производитель:  БиоФарм
Срок годности:  2025-12-01
Дней до окончания:  92
\end{lstlisting}

% ================= Вариант 31 =================
\item
Создайте класс \texttt{Project} с закрытыми атрибутами \texttt{\_\_title}, \texttt{\_\_start\_date}, \texttt{\_\_deadline}. Реализуйте методы-геттеры и метод \texttt{days\_until\_deadline()}.

\subsection*{Инструкции:}
\begin{enumerate}
    \item Создайте класс \texttt{Project}.
    \item Методы-геттеры: \texttt{get\_title()}, \texttt{get\_start\_date()}, \texttt{get\_deadline()}.
    \item Метод \texttt{days\_until\_deadline()} вычисляет дни до дедлайна.
    \item Создайте несколько экземпляров.
    \item Выведите данные каждого проекта.
\end{enumerate}

\subsection*{Пример использования:}
\begin{lstlisting}[caption=Пример кода]
from datetime import date

proj1 = Project("Разработка сайта", date(2025, 9, 1), date(2025, 12, 1))
proj2 = Project("Мобильное приложение", date(2025, 10, 1), date(2026, 1, 15))

print("Проект 1:")
print("Название: ", proj1.get_title())
print("Дата начала: ", proj1.get_start_date())
print("Дедлайн: ", proj1.get_deadline())
print("Дней до дедлайна: ", proj1.days_until_deadline())

print("Проект 2:")
print("Название: ", proj2.get_title())
print("Дата начала: ", proj2.get_start_date())
print("Дедлайн: ", proj2.get_deadline())
print("Дней до дедлайна: ", proj2.days_until_deadline())
\end{lstlisting}

\subsection*{Вывод:}
\begin{lstlisting}[caption=Ожидаемый вывод]
Проект 1:
Название:  Разработка сайта
Дата начала:  2025-09-01
Дедлайн:  2025-12-01
Дней до дедлайна:  91
Проект 2:
Название:  Мобильное приложение
Дата начала:  2025-10-01
Дедлайн:  2026-01-15
Дней до дедлайна:  106
\end{lstlisting}


\end{enumerate}


\textbf{Задача 4}

\input{problem_inheritence_task4}
\subsection{Семинар <<Структуры данных в ООП-реализации>>  
(2 часа)}


В ходе работы решите 4 задачи. 
Предполагается, что пользователь класса не имеет права обращаться к свойствам напрямую 
(соблюдая принцип инкапсуляции), а должен использовать методы. 

Продемонстрируйте работоспособность всех методов (из задания) 
посредством создания запускаемых файлов, где осуществляется 
вызов методов для разных ситуаций 
(без ручного ввода, но с выводом результатов в консоль). 

Каждый класс должен сохраняться в отдельном исходном файле. 
Необходимо соблюдать все стандартные требования к качеству кода 
(отступы, именования переменных, классов, методов, 
проверка корректности входных данных).

Задания этого семинара предназначены для освоения не только ООП, но и структур данных, поэтому
требуется структуры формировать вручную без использования библиотечных вариантов.

Для сдачи работы будьте готовы пояснить или аналогично заданию модифицировать любую часть кода, а также ответить на вопросы:
\begin{enumerate}
    \item Что обозначает свойство наследования в парадигме ООП?
    \item Что обозначает свойство полиморфизма в парадигме ООП?
    \item Опишите реализацию наследования в Python
    \item Как создать конструктор в Python
    \item Как реализовать абстрактный класс в Python (и что это значит)
    \item Как реализовать абстрактные методы в Python (и что это значит)
    \item Опишите бинарное дерево
    \item Как вставить элемент в бинарное дерево
    \item Как найи элемент в бинарном дереве
    \item Опишите, что такое стек
    \item Опишите, что такое очередь
    \item Опишите двусвязный список
    \item Сравните стек, очередь и двусвязный список
\end{enumerate}

Если вы нашли в задачнике ошибки, опечатки и другие недостатки, то вы можете сделать pull-request. 

\textbf{Срок сдачи работы (начала сдачи):} через одно занятие после его выдачи. В последующие сроки оценка будет снижаться (при отсутствии оправдывающих документов).


\subsubsection{Задача 1 (дерево)} 


\begin{enumerate}
\item Написать программу на Python, которая реализует бинарное дерево поиска с инкапсуляцией внутренней структуры. Программа должна создавать экземпляры класса TreeNode, которые представляют узлы дерева, и класса SearchTree, который представляет дерево поиска. Класс SearchTree должен содержать методы для добавления, поиска и удаления элементов из дерева, при этом все вспомогательные методы должны быть приватными. Программа также должна создавать дерево поиска, вставлять в него случайные числа и выполнять поиск элементов в дереве.

Инструкции:
\begin{enumerate}
    \item Создайте класс TreeNode с методом \_\_init\_\_, который принимает значение в качестве аргумента и сохраняет его в атрибуте self.data. Атрибуты left и right должны быть инициализированы как None.
    \item Создайте класс SearchTree с методом \_\_init\_\_, который инициализирует корневой узел дерева как None.
    \item Создайте публичный метод add в классе SearchTree, который добавляет значение в дерево. Если корневой узел отсутствует, создайте новый узел с добавляемым значением. В противном случае, вызовите приватный метод \_add\_helper, передав ему корень и значение.
    \item Создайте приватный метод \_add\_helper в классе SearchTree, который рекурсивно добавляет значение в дерево. Если значение меньше или равно значению текущего узла, добавьте его в левое поддерево. Если значение строго больше значения текущего узла, добавьте его в правое поддерево.
    \item Создайте публичный метод locate в классе SearchTree, который ищет значение в дереве. Если дерево пустое, верните None. В противном случае, вызовите приватный метод \_locate\_helper, передав ему корень и искомое значение.
    \item Создайте приватный метод \_locate\_helper в классе SearchTree, который рекурсивно ищет значение в дереве. Если текущий узел равен None или значение текущего узла равно искомому значению, верните текущий узел. В противном случае, рекурсивно вызывайте метод \_locate\_helper для поиска значения в левом поддереве (если искомое значение меньше или равно значению текущего узла) или в правом поддереве (если искомое значение больше).
    \item Создайте экземпляр класса SearchTree и вставьте в него 15 случайных чисел от 1 до 30.
    \item Выполните поиск элементов в дереве и выведите результаты на экран.
\end{enumerate}

Пример использования:
\begin{lstlisting}[language=Python]
tree = SearchTree()
for i in range(15):
    tree.add(random.randint(1, 30))

print("Поиск элементов:")
print(tree.locate(7))   # Обнаружено, возвращен узел (7)
print(tree.locate(25))  # Не обнаружено, возвращено None
print(tree.locate(15))  # Обнаружено, возвращен узел (15)
\end{lstlisting}

\begin{figure}[h]
\centering
\begin{tikzpicture}[level distance=1.5cm,
  level 1/.style={sibling distance=3cm},
  level 2/.style={sibling distance=1.5cm}]
  \node {10}
    child {node {5}
      child {node {2}}
      child {node {8}}
    }
    child {node {15}
      child {node {12}}
      child {node {20}}
    };
\end{tikzpicture}
\caption{Пример бинарного дерева поиска}
\end{figure}

\item Написать программу на Python, которая реализует бинарное дерево поиска с соблюдением принципов инкапсуляции. Программа должна создавать экземпляры класса Vertex, которые представляют узлы дерева, и класса BinaryTree, который представляет дерево поиска. Класс BinaryTree должен содержать методы для вставки, поиска и удаления элементов, при этом все рекурсивные вспомогательные методы должны быть скрыты от внешнего доступа. Программа также должна создавать дерево поиска, вставлять в него случайные числа и выполнять поиск элементов в дереве.

Инструкции:
\begin{enumerate}
    \item Создайте класс Vertex с методом \_\_init\_\_, который принимает значение value и сохраняет его в атрибуте self.key. Атрибуты self.left\_child и self.right\_child должны быть инициализированы как None.
    \item Создайте класс BinaryTree с методом \_\_init\_\_, который инициализирует атрибут self.top как None.
    \item Создайте публичный метод put в классе BinaryTree, который вставляет значение в дерево. Если self.top отсутствует, создайте новый узел с вставляемым значением. В противном случае, вызовите приватный метод \_put\_recursively, передав ему self.top и значение.
    \item Создайте приватный метод \_put\_recursively в классе BinaryTree, который рекурсивно вставляет значение в дерево. Если значение строго меньше значения текущего узла, вставьте его в левое поддерево. Если значение больше или равно значению текущего узла, вставьте его в правое поддерево.
    \item Создайте публичный метод find в классе BinaryTree, который ищет значение в дереве. Если дерево пустое, верните None. В противном случае, вызовите приватный метод \_find\_recursively, передав ему self.top и искомое значение.
    \item Создайте приватный метод \_find\_recursively в классе BinaryTree, который рекурсивно ищет значение в дереве. Если текущий узел равен None или значение текущего узла равно искомому значению, верните текущий узел. В противном случае, рекурсивно вызывайте метод \_find\_recursively для поиска значения в левом поддереве (если искомое значение меньше текущего) или в правом поддереве (если искомое значение больше или равно текущему).
    \item Создайте экземпляр класса BinaryTree и вставьте в него 18 случайных чисел от 5 до 35.
    \item Выполните поиск элементов в дереве и выведите результаты на экран.
\end{enumerate}

Пример использования:
\begin{lstlisting}[language=Python]
bt = BinaryTree()
for i in range(18):
    bt.put(random.randint(5, 35))

print("Поиск элементов:")
print(bt.find(10))  # Обнаружено, возвращен узел (10)
print(bt.find(40))  # Не обнаружено, возвращено None
print(bt.find(22))  # Обнаружено, возвращен узел (22)
\end{lstlisting}

\begin{figure}[h]
\centering
\begin{tikzpicture}[level distance=1.5cm,
  level 1/.style={sibling distance=3cm},
  level 2/.style={sibling distance=1.5cm}]
  \node {18}
    child {node {9}
      child {node {4}}
      child {node {14}}
    }
    child {node {25}
      child {node {21}}
      child {node {30}}
    };
\end{tikzpicture}
\caption{Пример бинарного дерева поиска}
\end{figure}

\item Написать программу на Python, которая реализует бинарное дерево поиска с инкапсуляцией внутренней логики. Программа должна создавать экземпляры класса BNode, которые представляют узлы дерева, и класса BSTree, который представляет дерево поиска. Класс BSTree должен содержать методы для вставки, поиска и удаления элементов, при этом все рекурсивные функции должны быть приватными. Программа также должна создавать дерево поиска, вставлять в него случайные числа и выполнять поиск элементов в дереве.

Инструкции:
\begin{enumerate}
    \item Создайте класс BNode с методом \_\_init\_\_, который принимает параметр item и сохраняет его в атрибуте self.element. Атрибуты self.left\_branch и self.right\_branch должны быть инициализированы как None.
    \item Создайте класс BSTree с методом \_\_init\_\_, который инициализирует атрибут self.root\_node как None.
    \item Создайте публичный метод insert\_value в классе BSTree, который вставляет значение в дерево. Если self.root\_node отсутствует, создайте новый узел с вставляемым значением. В противном случае, вызовите приватный метод \_recursive\_insert, передав ему self.root\_node и значение.
    \item Создайте приватный метод \_recursive\_insert в классе BSTree, который рекурсивно вставляет значение в дерево. Если значение меньше или равно значению текущего узла, вставьте его в левое поддерево. Если значение строго больше значения текущего узла, вставьте его в правое поддерево.
    \item Создайте публичный метод retrieve в классе BSTree, который ищет значение в дереве. Если дерево пустое, верните None. В противном случае, вызовите приватный метод \_recursive\_retrieve, передав ему self.root\_node и искомое значение.
    \item Создайте приватный метод \_recursive\_retrieve в классе BSTree, который рекурсивно ищет значение в дереве. Если текущий узел равен None или значение текущего узла равно искомому значению, верните текущий узел. В противном случае, рекурсивно вызывайте метод \_recursive\_retrieve для поиска значения в левом поддереве (если искомое значение меньше или равно текущему) или в правом поддереве (если искомое значение больше).
    \item Создайте экземпляр класса BSTree и вставьте в него 20 случайных чисел от 1 до 40.
    \item Выполните поиск элементов в дереве и выведите результаты на экран.
\end{enumerate}

Пример использования:
\begin{lstlisting}[language=Python]
bst = BSTree()
for i in range(20):
    bst.insert_value(random.randint(1, 40))

print("Поиск элементов:")
print(bst.retrieve(12))  # Обнаружено, возвращен узел (12)
print(bst.retrieve(50))  # Не обнаружено, возвращено None
print(bst.retrieve(33))  # Обнаружено, возвращен узел (33)
\end{lstlisting}

\begin{figure}[h]
\centering
\begin{tikzpicture}[level distance=1.5cm,
  level 1/.style={sibling distance=3cm},
  level 2/.style={sibling distance=1.5cm}]
  \node {20}
    child {node {10}
      child {node {5}}
      child {node {15}}
    }
    child {node {30}
      child {node {25}}
      child {node {35}}
    };
\end{tikzpicture}
\caption{Пример бинарного дерева поиска}
\end{figure}

\item Написать программу на Python, которая реализует бинарное дерево поиска с инкапсуляцией. Программа должна создавать экземпляры класса ElementNode, которые представляют узлы дерева, и класса OrderedTree, который представляет дерево поиска. Класс OrderedTree должен содержать методы для вставки, поиска и удаления элементов, при этом все вспомогательные методы должны быть приватными. Программа также должна создавать дерево поиска, вставлять в него случайные числа и выполнять поиск элементов в дереве.

Инструкции:
\begin{enumerate}
    \item Создайте класс ElementNode с методом \_\_init\_\_, который принимает параметр content и сохраняет его в атрибуте self.payload. Атрибуты self.left и self.right должны быть инициализированы как None.
    \item Создайте класс OrderedTree с методом \_\_init\_\_, который инициализирует атрибут self.head как None.
    \item Создайте публичный метод store в классе OrderedTree, который вставляет значение в дерево. Если self.head отсутствует, создайте новый узел с вставляемым значением. В противном случае, вызовите приватный метод \_store\_recursive, передав ему self.head и значение.
    \item Создайте приватный метод \_store\_recursive в классе OrderedTree, который рекурсивно вставляет значение в дерево. Если значение строго меньше значения текущего узла, вставьте его в левое поддерево. Если значение больше или равно значению текущего узла, вставьте его в правое поддерево.
    \item Создайте публичный метод query в классе OrderedTree, который ищет значение в дереве. Если дерево пустое, верните None. В противном случае, вызовите приватный метод \_query\_recursive, передав ему self.head и искомое значение.
    \item Создайте приватный метод \_query\_recursive в классе OrderedTree, который рекурсивно ищет значение в дереве. Если текущий узел равен None или значение текущего узла равно искомому значению, верните текущий узел. В противном случае, рекурсивно вызывайте метод \_query\_recursive для поиска значения в левом поддереве (если искомое значение меньше текущего) или в правом поддереве (если искомое значение больше или равно текущему).
    \item Создайте экземпляр класса OrderedTree и вставьте в него 17 случайных чисел от 3 до 33.
    \item Выполните поиск элементов в дереве и выведите результаты на экран.
\end{enumerate}

Пример использования:
\begin{lstlisting}[language=Python]
ot = OrderedTree()
for i in range(17):
    ot.store(random.randint(3, 33))

print("Поиск элементов:")
print(ot.query(8))   # Обнаружено, возвращен узел (8)
print(ot.query(45))  # Не обнаружено, возвращено None
print(ot.query(27))  # Обнаружено, возвращен узел (27)
\end{lstlisting}

\begin{figure}[h]
\centering
\begin{tikzpicture}[level distance=1.5cm,
  level 1/.style={sibling distance=3cm},
  level 2/.style={sibling distance=1.5cm}]
  \node {17}
    child {node {8}
      child {node {3}}
      child {node {12}}
    }
    child {node {27}
      child {node {22}}
      child {node {32}}
    };
\end{tikzpicture}
\caption{Пример бинарного дерева поиска}
\end{figure}

\item Написать программу на Python, которая реализует бинарное дерево поиска с инкапсуляцией внутренней структуры. Программа должна создавать экземпляры класса DataNode, которые представляют узлы дерева, и класса SortedTree, который представляет дерево поиска. Класс SortedTree должен содержать методы для вставки, поиска и удаления элементов, при этом все рекурсивные методы должны быть скрыты. Программа также должна создавать дерево поиска, вставлять в него случайные числа и выполнять поиск элементов в дереве.

Инструкции:
\begin{enumerate}
    \item Создайте класс DataNode с методом \_\_init\_\_, который принимает параметр val и сохраняет его в атрибуте self.entry. Атрибуты self.left и self.right должны быть инициализированы как None.
    \item Создайте класс SortedTree с методом \_\_init\_\_, который инициализирует атрибут self.first\_node как None.
    \item Создайте публичный метод enqueue в классе SortedTree, который вставляет значение в дерево. Если self.first\_node отсутствует, создайте новый узел с вставляемым значением. В противном случае, вызовите приватный метод \_enqueue\_helper, передав ему self.first\_node и значение.
    \item Создайте приватный метод \_enqueue\_helper в классе SortedTree, который рекурсивно вставляет значение в дерево. Если значение меньше или равно значению текущего узла, вставьте его в левое поддерево. Если значение строго больше значения текущего узла, вставьте его в правое поддерево.
    \item Создайте публичный метод lookup в классе SortedTree, который ищет значение в дереве. Если дерево пустое, верните None. В противном случае, вызовите приватный метод \_lookup\_helper, передав ему self.first\_node и искомое значение.
    \item Создайте приватный метод \_lookup\_helper в классе SortedTree, который рекурсивно ищет значение в дереве. Если текущий узел равен None или значение текущего узла равно искомому значению, верните текущий узел. В противном случае, рекурсивно вызывайте метод \_lookup\_helper для поиска значения в левом поддереве (если искомое значение меньше или равно текущему) или в правом поддереве (если искомое значение больше).
    \item Создайте экземпляр класса SortedTree и вставьте в него 16 случайных чисел от 2 до 28.
    \item Выполните поиск элементов в дереве и выведите результаты на экран.
\end{enumerate}

Пример использования:
\begin{lstlisting}[language=Python]
st = SortedTree()
for i in range(16):
    st.enqueue(random.randint(2, 28))

print("Поиск элементов:")
print(st.lookup(6))   # Обнаружено, возвращен узел (6)
print(st.lookup(35))  # Не обнаружено, возвращено None
print(st.lookup(19))  # Обнаружено, возвращен узел (19)
\end{lstlisting}

\begin{figure}[h]
\centering
\begin{tikzpicture}[level distance=1.5cm,
  level 1/.style={sibling distance=3cm},
  level 2/.style={sibling distance=1.5cm}]
  \node {14}
    child {node {6}
      child {node {2}}
      child {node {10}}
    }
    child {node {22}
      child {node {18}}
      child {node {26}}
    };
\end{tikzpicture}
\caption{Пример бинарного дерева поиска}
\end{figure}

\item Написать программу на Python, которая реализует бинарное дерево поиска с инкапсуляцией. Программа должна создавать экземпляры класса BinNode, которые представляют узлы дерева, и класса LookupTree, который представляет дерево поиска. Класс LookupTree должен содержать методы для вставки, поиска и удаления элементов, при этом все вспомогательные методы должны быть приватными. Программа также должна создавать дерево поиска, вставлять в него случайные числа и выполнять поиск элементов в дереве.

Инструкции:
\begin{enumerate}
    \item Создайте класс BinNode с методом \_\_init\_\_, который принимает параметр num и сохраняет его в атрибуте self.number. Атрибуты self.left и self.right должны быть инициализированы как None.
    \item Создайте класс LookupTree с методом \_\_init\_\_, который инициализирует атрибут self.initial\_node как None.
    \item Создайте публичный метод add\_entry в классе LookupTree, который вставляет значение в дерево. Если self.initial\_node отсутствует, создайте новый узел с вставляемым значением. В противном случае, вызовите приватный метод \_add\_entry\_rec, передав ему self.initial\_node и значение.
    \item Создайте приватный метод \_add\_entry\_rec в классе LookupTree, который рекурсивно вставляет значение в дерево. Если значение строго меньше значения текущего узла, вставьте его в левое поддерево. Если значение больше или равно значению текущего узла, вставьте его в правое поддерево.
    \item Создайте публичный метод fetch в классе LookupTree, который ищет значение в дереве. Если дерево пустое, верните None. В противном случае, вызовите приватный метод \_fetch\_rec, передав ему self.initial\_node и искомое значение.
    \item Создайте приватный метод \_fetch\_rec в классе LookupTree, который рекурсивно ищет значение в дереве. Если текущий узел равен None или значение текущего узла равно искомому значению, верните текущий узел. В противном случае, рекурсивно вызывайте метод \_fetch\_rec для поиска значения в левом поддереве (если искомое значение меньше текущего) или в правом поддереве (если искомое значение больше или равно текущему).
    \item Создайте экземпляр класса LookupTree и вставьте в него 19 случайных чисел от 4 до 34.
    \item Выполните поиск элементов в дереве и выведите результаты на экран.
\end{enumerate}

Пример использования:
\begin{lstlisting}[language=Python]
lt = LookupTree()
for i in range(19):
    lt.add_entry(random.randint(4, 34))

print("Поиск элементов:")
print(lt.fetch(9))   # Обнаружено, возвращен узел (9)
print(lt.fetch(40))  # Не обнаружено, возвращено None
print(lt.fetch(24))  # Обнаружено, возвращен узел (24)
\end{lstlisting}

\begin{figure}[h]
\centering
\begin{tikzpicture}[level distance=1.5cm,
  level 1/.style={sibling distance=3cm},
  level 2/.style={sibling distance=1.5cm}]
  \node {19}
    child {node {9}
      child {node {4}}
      child {node {14}}
    }
    child {node {29}
      child {node {24}}
      child {node {34}}
    };
\end{tikzpicture}
\caption{Пример бинарного дерева поиска}
\end{figure}

\item Написать программу на Python, которая реализует бинарное дерево поиска с инкапсуляцией. Программа должна создавать экземпляры класса NodeItem, которые представляют узлы дерева, и класса BinaryTreeSearch, который представляет дерево поиска. Класс BinaryTreeSearch должен содержать методы для вставки, поиска и удаления элементов, при этом все рекурсивные методы должны быть приватными. Программа также должна создавать дерево поиска, вставлять в него случайные числа и выполнять поиск элементов в дереве.

Инструкции:
\begin{enumerate}
    \item Создайте класс NodeItem с методом \_\_init\_\_, который принимает параметр item\_value и сохраняет его в атрибуте self.val. Атрибуты self.left и self.right должны быть инициализированы как None.
    \item Создайте класс BinaryTreeSearch с методом \_\_init\_\_, который инициализирует атрибут self.start\_node как None.
    \item Создайте публичный метод insert\_item в классе BinaryTreeSearch, который вставляет значение в дерево. Если self.start\_node отсутствует, создайте новый узел с вставляемым значением. В противном случае, вызовите приватный метод \_insert\_item\_helper, передав ему self.start\_node и значение.
    \item Создайте приватный метод \_insert\_item\_helper в классе BinaryTreeSearch, который рекурсивно вставляет значение в дерево. Если значение меньше или равно значению текущего узла, вставьте его в левое поддерево. Если значение строго больше значения текущего узла, вставьте его в правое поддерево.
    \item Создайте публичный метод find\_item в классе BinaryTreeSearch, который ищет значение в дереве. Если дерево пустое, верните None. В противном случае, вызовите приватный метод \_find\_item\_helper, передав ему self.start\_node и искомое значение.
    \item Создайте приватный метод \_find\_item\_helper в классе BinaryTreeSearch, который рекурсивно ищет значение в дереве. Если текущий узел равен None или значение текущего узла равно искомому значению, верните текущий узел. В противном случае, рекурсивно вызывайте метод \_find\_item\_helper для поиска значения в левом поддереве (если искомое значение меньше или равно текущему) или в правом поддереве (если искомое значение больше).
    \item Создайте экземпляр класса BinaryTreeSearch и вставьте в него 21 случайное число от 1 до 38.
    \item Выполните поиск элементов в дереве и выведите результаты на экран.
\end{enumerate}

Пример использования:
\begin{lstlisting}[language=Python]
bts = BinaryTreeSearch()
for i in range(21):
    bts.insert_item(random.randint(1, 38))

print("Поиск элементов:")
print(bts.find_item(11))  # Обнаружено, возвращен узел (11)
print(bts.find_item(50))  # Не обнаружено, возвращено None
print(bts.find_item(29))  # Обнаружено, возвращен узел (29)
\end{lstlisting}

\begin{figure}[h]
\centering
\begin{tikzpicture}[level distance=1.5cm,
  level 1/.style={sibling distance=3cm},
  level 2/.style={sibling distance=1.5cm}]
  \node {21}
    child {node {11}
      child {node {5}}
      child {node {16}}
    }
    child {node {31}
      child {node {26}}
      child {node {36}}
    };
\end{tikzpicture}
\caption{Пример бинарного дерева поиска}
\end{figure}

\item Написать программу на Python, которая реализует бинарное дерево поиска с инкапсуляцией. Программа должна создавать экземпляры класса TreeVertex, которые представляют узлы дерева, и класса SearchBinTree, который представляет дерево поиска. Класс SearchBinTree должен содержать методы для вставки, поиска и удаления элементов, при этом все вспомогательные методы должны быть приватными. Программа также должна создавать дерево поиска, вставлять в него случайные числа и выполнять поиск элементов в дереве.

Инструкции:
\begin{enumerate}
    \item Создайте класс TreeVertex с методом \_\_init\_\_, который принимает параметр vertex\_data и сохраняет его в атрибуте self.info. Атрибуты self.left и self.right должны быть инициализированы как None.
    \item Создайте класс SearchBinTree с методом \_\_init\_\_, который инициализирует атрибут self.root\_vertex как None.
    \item Создайте публичный метод insert\_data в классе SearchBinTree, который вставляет значение в дерево. Если self.root\_vertex отсутствует, создайте новый узел с вставляемым значением. В противном случае, вызовите приватный метод \_insert\_data\_rec, передав ему self.root\_vertex и значение.
    \item Создайте приватный метод \_insert\_data\_rec в классе SearchBinTree, который рекурсивно вставляет значение в дерево. Если значение строго меньше значения текущего узла, вставьте его в левое поддерево. Если значение больше или равно значению текущего узла, вставьте его в правое поддерево.
    \item Создайте публичный метод search\_data в классе SearchBinTree, который ищет значение в дереве. Если дерево пустое, верните None. В противном случае, вызовите приватный метод \_search\_data\_rec, передав ему self.root\_vertex и искомое значение.
    \item Создайте приватный метод \_search\_data\_rec в классе SearchBinTree, который рекурсивно ищет значение в дереве. Если текущий узел равен None или значение текущего узла равно искомому значению, верните текущий узел. В противном случае, рекурсивно вызывайте метод \_search\_data\_rec для поиска значения в левом поддереве (если искомое значение меньше текущего) или в правом поддереве (если искомое значение больше или равно текущему).
    \item Создайте экземпляр класса SearchBinTree и вставьте в него 14 случайных чисел от 6 до 36.
    \item Выполните поиск элементов в дереве и выведите результаты на экран.
\end{enumerate}

Пример использования:
\begin{lstlisting}[language=Python]
sbt = SearchBinTree()
for i in range(14):
    sbt.insert_data(random.randint(6, 36))

print("Поиск элементов:")
print(sbt.search_data(13))  # Обнаружено, возвращен узел (13)
print(sbt.search_data(42))  # Не обнаружено, возвращено None
print(sbt.search_data(28))  # Обнаружено, возвращен узел (28)
\end{lstlisting}

\begin{figure}[h]
\centering
\begin{tikzpicture}[level distance=1.5cm,
  level 1/.style={sibling distance=3cm},
  level 2/.style={sibling distance=1.5cm}]
  \node {23}
    child {node {13}
      child {node {8}}
      child {node {18}}
    }
    child {node {33}
      child {node {28}}
      child {node {38}}
    };
\end{tikzpicture}
\caption{Пример бинарного дерева поиска}
\end{figure}

\item Написать программу на Python, которая реализует бинарное дерево поиска с инкапсуляцией. Программа должна создавать экземпляры класса BranchNode, которые представляют узлы дерева, и класса BinaryTreeLookup, который представляет дерево поиска. Класс BinaryTreeLookup должен содержать методы для вставки, поиска и удаления элементов, при этом все рекурсивные методы должны быть приватными. Программа также должна создавать дерево поиска, вставлять в него случайные числа и выполнять поиск элементов в дереве.

Инструкции:
\begin{enumerate}
    \item Создайте класс BranchNode с методом \_\_init\_\_, который принимает параметр node\_val и сохраняет его в атрибуте self.data\_point. Атрибуты self.left\_link и self.right\_link должны быть инициализированы как None.
    \item Создайте класс BinaryTreeLookup с методом \_\_init\_\_, который инициализирует атрибут self.base\_node как None.
    \item Создайте публичный метод add\_point в классе BinaryTreeLookup, который вставляет значение в дерево. Если self.base\_node отсутствует, создайте новый узел с вставляемым значением. В противном случае, вызовите приватный метод \_add\_point\_recursive, передав ему self.base\_node и значение.
    \item Создайте приватный метод \_add\_point\_recursive в классе BinaryTreeLookup, который рекурсивно вставляет значение в дерево. Если значение меньше или равно значению текущего узла, вставьте его в левое поддерево. Если значение строго больше значения текущего узла, вставьте его в правое поддерево.
    \item Создайте публичный метод locate\_point в классе BinaryTreeLookup, который ищет значение в дереве. Если дерево пустое, верните None. В противном случае, вызовите приватный метод \_locate\_point\_recursive, передав ему self.base\_node и искомое значение.
    \item Создайте приватный метод \_locate\_point\_recursive в классе BinaryTreeLookup, который рекурсивно ищет значение в дереве. Если текущий узел равен None или значение текущего узла равно искомому значению, верните текущий узел. В противном случае, рекурсивно вызывайте метод \_locate\_point\_recursive для поиска значения в левом поддереве (если искомое значение меньше или равно текущему) или в правом поддереве (если искомое значение больше).
    \item Создайте экземпляр класса BinaryTreeLookup и вставьте в него 13 случайных чисел от 7 до 37.
    \item Выполните поиск элементов в дереве и выведите результаты на экран.
\end{enumerate}

Пример использования:
\begin{lstlisting}[language=Python]
btl = BinaryTreeLookup()
for i in range(13):
    btl.add_point(random.randint(7, 37))

print("Поиск элементов:")
print(btl.locate_point(14))  # Обнаружено, возвращен узел (14)
print(btl.locate_point(45))  # Не обнаружено, возвращено None
print(btl.locate_point(27))  # Обнаружено, возвращен узел (27)
\end{lstlisting}

\begin{figure}[h]
\centering
\begin{tikzpicture}[level distance=1.5cm,
  level 1/.style={sibling distance=3cm},
  level 2/.style={sibling distance=1.5cm}]
  \node {24}
    child {node {14}
      child {node {9}}
      child {node {19}}
    }
    child {node {34}
      child {node {29}}
      child {node {39}}
    };
\end{tikzpicture}
\caption{Пример бинарного дерева поиска}
\end{figure}

\item Написать программу на Python, которая реализует бинарное дерево поиска с инкапсуляцией. Программа должна создавать экземпляры класса TreeNodeStruct, которые представляют узлы дерева, и класса BinSearchStructure, который представляет дерево поиска. Класс BinSearchStructure должен содержать методы для вставки, поиска и удаления элементов, при этом все вспомогательные методы должны быть приватными. Программа также должна создавать дерево поиска, вставлять в него случайные числа и выполнять поиск элементов в дереве.

Инструкции:
\begin{enumerate}
    \item Создайте класс TreeNodeStruct с методом \_\_init\_\_, который принимает параметр struct\_value и сохраняет его в атрибуте self.node\_value. Атрибуты self.left\_sub и self.right\_sub должны быть инициализированы как None.
    \item Создайте класс BinSearchStructure с методом \_\_init\_\_, который инициализирует атрибут self.top\_element как None.
    \item Создайте публичный метод insert\_struct в классе BinSearchStructure, который вставляет значение в дерево. Если self.top\_element отсутствует, создайте новый узел с вставляемым значением. В противном случае, вызовите приватный метод \_insert\_struct\_helper, передав ему self.top\_element и значение.
    \item Создайте приватный метод \_insert\_struct\_helper в классе BinSearchStructure, который рекурсивно вставляет значение в дерево. Если значение строго меньше значения текущего узла, вставьте его в левое поддерево. Если значение больше или равно значению текущего узла, вставьте его в правое поддерево.
    \item Создайте публичный метод find\_struct в классе BinSearchStructure, который ищет значение в дереве. Если дерево пустое, верните None. В противном случае, вызовите приватный метод \_find\_struct\_helper, передав ему self.top\_element и искомое значение.
    \item Создайте приватный метод \_find\_struct\_helper в классе BinSearchStructure, который рекурсивно ищет значение в дереве. Если текущий узел равен None или значение текущего узла равно искомому значению, верните текущий узел. В противном случае, рекурсивно вызывайте метод \_find\_struct\_helper для поиска значения в левом поддереве (если искомое значение меньше текущего) или в правом поддереве (если искомое значение больше или равно текущему).
    \item Создайте экземпляр класса BinSearchStructure и вставьте в него 22 случайных числа от 2 до 42.
    \item Выполните поиск элементов в дереве и выведите результаты на экран.
\end{enumerate}

Пример использования:
\begin{lstlisting}[language=Python]
bss = BinSearchStructure()
for i in range(22):
    bss.insert_struct(random.randint(2, 42))

print("Поиск элементов:")
print(bss.find_struct(15))  # Обнаружено, возвращен узел (15)
print(bss.find_struct(55))  # Не обнаружено, возвращено None
print(bss.find_struct(35))  # Обнаружено, возвращен узел (35)
\end{lstlisting}

\begin{figure}[h]
\centering
\begin{tikzpicture}[level distance=1.5cm,
  level 1/.style={sibling distance=3cm},
  level 2/.style={sibling distance=1.5cm}]
  \node {25}
    child {node {15}
      child {node {10}}
      child {node {20}}
    }
    child {node {35}
      child {node {30}}
      child {node {40}}
    };
\end{tikzpicture}
\caption{Пример бинарного дерева поиска}
\end{figure}

\item Написать программу на Python, которая реализует бинарное дерево поиска с инкапсуляцией. Программа должна создавать экземпляры класса NodeElement, которые представляют узлы дерева, и класса TreeIndex, который представляет дерево поиска. Класс TreeIndex должен содержать методы для вставки, поиска и удаления элементов, при этом все рекурсивные методы должны быть приватными. Программа также должна создавать дерево поиска, вставлять в него случайные числа и выполнять поиск элементов в дереве.

Инструкции:
\begin{enumerate}
    \item Создайте класс NodeElement с методом \_\_init\_\_, который принимает параметр elem\_value и сохраняет его в атрибуте self.index\_key. Атрибуты self.left и self.right должны быть инициализированы как None.
    \item Создайте класс TreeIndex с методом \_\_init\_\_, который инициализирует атрибут self.root\_elem как None.
    \item Создайте публичный метод add\_key в классе TreeIndex, который вставляет значение в дерево. Если self.root\_elem отсутствует, создайте новый узел с вставляемым значением. В противном случае, вызовите приватный метод \_add\_key\_rec, передав ему self.root\_elem и значение.
    \item Создайте приватный метод \_add\_key\_rec в классе TreeIndex, который рекурсивно вставляет значение в дерево. Если значение меньше или равно значению текущего узла, вставьте его в левое поддерево. Если значение строго больше значения текущего узла, вставьте его в правое поддерево.
    \item Создайте публичный метод get\_key в классе TreeIndex, который ищет значение в дереве. Если дерево пустое, верните None. В противном случае, вызовите приватный метод \_get\_key\_rec, передав ему self.root\_elem и искомое значение.
    \item Создайте приватный метод \_get\_key\_rec в классе TreeIndex, который рекурсивно ищет значение в дереве. Если текущий узел равен None или значение текущего узла равно искомому значению, верните текущий узел. В противном случае, рекурсивно вызывайте метод \_get\_key\_rec для поиска значения в левом поддереве (если искомое значение меньше или равно текущему) или в правом поддереве (если искомое значение больше).
    \item Создайте экземпляр класса TreeIndex и вставьте в него 23 случайных числа от 3 до 43.
    \item Выполните поиск элементов в дереве и выведите результаты на экран.
\end{enumerate}

Пример использования:
\begin{lstlisting}[language=Python]
ti = TreeIndex()
for i in range(23):
    ti.add_key(random.randint(3, 43))

print("Поиск элементов:")
print(ti.get_key(16))  # Обнаружено, возвращен узел (16)
print(ti.get_key(56))  # Не обнаружено, возвращено None
print(ti.get_key(36))  # Обнаружено, возвращен узел (36)
\end{lstlisting}

\begin{figure}[h]
\centering
\begin{tikzpicture}[level distance=1.5cm,
  level 1/.style={sibling distance=3cm},
  level 2/.style={sibling distance=1.5cm}]
  \node {26}
    child {node {16}
      child {node {11}}
      child {node {21}}
    }
    child {node {36}
      child {node {31}}
      child {node {41}}
    };
\end{tikzpicture}
\caption{Пример бинарного дерева поиска}
\end{figure}

\item Написать программу на Python, которая реализует бинарное дерево поиска с инкапсуляцией. Программа должна создавать экземпляры класса BinElement, которые представляют узлы дерева, и класса IndexTree, который представляет дерево поиска. Класс IndexTree должен содержать методы для вставки, поиска и удаления элементов, при этом все вспомогательные методы должны быть приватными. Программа также должна создавать дерево поиска, вставлять в него случайные числа и выполнять поиск элементов в дереве.

Инструкции:
\begin{enumerate}
    \item Создайте класс BinElement с методом \_\_init\_\_, который принимает параметр bin\_val и сохраняет его в атрибуте self.key\_value. Атрибуты self.left\_node и self.right\_node должны быть инициализированы как None.
    \item Создайте класс IndexTree с методом \_\_init\_\_, который инициализирует атрибут self.first\_element как None.
    \item Создайте публичный метод insert\_key в классе IndexTree, который вставляет значение в дерево. Если self.first\_element отсутствует, создайте новый узел с вставляемым значением. В противном случае, вызовите приватный метод \_insert\_key\_helper, передав ему self.first\_element и значение.
    \item Создайте приватный метод \_insert\_key\_helper в классе IndexTree, который рекурсивно вставляет значение в дерево. Если значение строго меньше значения текущего узла, вставьте его в левое поддерево. Если значение больше или равно значению текущего узла, вставьте его в правое поддерево.
    \item Создайте публичный метод search\_key в классе IndexTree, который ищет значение в дереве. Если дерево пустое, верните None. В противном случае, вызовите приватный метод \_search\_key\_helper, передав ему self.first\_element и искомое значение.
    \item Создайте приватный метод \_search\_key\_helper в классе IndexTree, который рекурсивно ищет значение в дереве. Если текущий узел равен None или значение текущего узла равно искомому значению, верните текущий узел. В противном случае, рекурсивно вызывайте метод \_search\_key\_helper для поиска значения в левом поддереве (если искомое значение меньше текущего) или в правом поддереве (если искомое значение больше или равно текущему).
    \item Создайте экземпляр класса IndexTree и вставьте в него 24 случайных числа от 4 до 44.
    \item Выполните поиск элементов в дереве и выведите результаты на экран.
\end{enumerate}

Пример использования:
\begin{lstlisting}[language=Python]
it = IndexTree()
for i in range(24):
    it.insert_key(random.randint(4, 44))

print("Поиск элементов:")
print(it.search_key(17))  # Обнаружено, возвращен узел (17)
print(it.search_key(57))  # Не обнаружено, возвращено None
print(it.search_key(37))  # Обнаружено, возвращен узел (37)
\end{lstlisting}

\begin{figure}[h]
\centering
\begin{tikzpicture}[level distance=1.5cm,
  level 1/.style={sibling distance=3cm},
  level 2/.style={sibling distance=1.5cm}]
  \node {27}
    child {node {17}
      child {node {12}}
      child {node {22}}
    }
    child {node {37}
      child {node {32}}
      child {node {42}}
    };
\end{tikzpicture}
\caption{Пример бинарного дерева поиска}
\end{figure}

\item Написать программу на Python, которая реализует бинарное дерево поиска с инкапсуляцией. Программа должна создавать экземпляры класса SearchNode, которые представляют узлы дерева, и класса BinaryTreeIndex, который представляет дерево поиска. Класс BinaryTreeIndex должен содержать методы для вставки, поиска и удаления элементов, при этом все рекурсивные методы должны быть приватными. Программа также должна создавать дерево поиска, вставлять в него случайные числа и выполнять поиск элементов в дереве.

Инструкции:
\begin{enumerate}
    \item Создайте класс SearchNode с методом \_\_init\_\_, который принимает параметр search\_val и сохраняет его в атрибуте self.node\_key. Атрибуты self.left\_child и self.right\_child должны быть инициализированы как None.
    \item Создайте класс BinaryTreeIndex с методом \_\_init\_\_, который инициализирует атрибут self.initial\_element как None.
    \item Создайте публичный метод add\_node в классе BinaryTreeIndex, который вставляет значение в дерево. Если self.initial\_element отсутствует, создайте новый узел с вставляемым значением. В противном случае, вызовите приватный метод \_add\_node\_recursive, передав ему self.initial\_element и значение.
    \item Создайте приватный метод \_add\_node\_recursive в классе BinaryTreeIndex, который рекурсивно вставляет значение в дерево. Если значение меньше или равно значению текущего узла, вставьте его в левое поддерево. Если значение строго больше значения текущего узла, вставьте его в правое поддерево.
    \item Создайте публичный метод find\_node в классе BinaryTreeIndex, который ищет значение в дереве. Если дерево пустое, верните None. В противном случае, вызовите приватный метод \_find\_node\_recursive, передав ему self.initial\_element и искомое значение.
    \item Создайте приватный метод \_find\_node\_recursive в классе BinaryTreeIndex, который рекурсивно ищет значение в дереве. Если текущий узел равен None или значение текущего узла равно искомому значению, верните текущий узел. В противном случае, рекурсивно вызывайте метод \_find\_node\_recursive для поиска значения в левом поддереве (если искомое значение меньше или равно текущему) или в правом поддереве (если искомое значение больше).
    \item Создайте экземпляр класса BinaryTreeIndex и вставьте в него 25 случайных чисел от 5 до 45.
    \item Выполните поиск элементов в дереве и выведите результаты на экран.
\end{enumerate}

Пример использования:
\begin{lstlisting}[language=Python]
bti = BinaryTreeIndex()
for i in range(25):
    bti.add_node(random.randint(5, 45))

print("Поиск элементов:")
print(bti.find_node(18))  # Обнаружено, возвращен узел (18)
print(bti.find_node(58))  # Не обнаружено, возвращено None
print(bti.find_node(38))  # Обнаружено, возвращен узел (38)
\end{lstlisting}

\begin{figure}[h]
\centering
\begin{tikzpicture}[level distance=1.5cm,
  level 1/.style={sibling distance=3cm},
  level 2/.style={sibling distance=1.5cm}]
  \node {28}
    child {node {18}
      child {node {13}}
      child {node {23}}
    }
    child {node {38}
      child {node {33}}
      child {node {43}}
    };
\end{tikzpicture}
\caption{Пример бинарного дерева поиска}
\end{figure}

\item Написать программу на Python, которая реализует бинарное дерево поиска с инкапсуляцией. Программа должна создавать экземпляры класса IndexNode, которые представляют узлы дерева, и класса SearchStructure, который представляет дерево поиска. Класс SearchStructure должен содержать методы для вставки, поиска и удаления элементов, при этом все вспомогательные методы должны быть приватными. Программа также должна создавать дерево поиска, вставлять в него случайные числа и выполнять поиск элементов в дереве.

Инструкции:
\begin{enumerate}
    \item Создайте класс IndexNode с методом \_\_init\_\_, который принимает параметр idx\_value и сохраняет его в атрибуте self.element\_key. Атрибуты self.left\_elem и self.right\_elem должны быть инициализированы как None.
    \item Создайте класс SearchStructure с методом \_\_init\_\_, который инициализирует атрибут self.start\_element как None.
    \item Создайте публичный метод insert\_elem в классе SearchStructure, который вставляет значение в дерево. Если self.start\_element отсутствует, создайте новый узел с вставляемым значением. В противном случае, вызовите приватный метод \_insert\_elem\_rec, передав ему self.start\_element и значение.
    \item Создайте приватный метод \_insert\_elem\_rec в классе SearchStructure, который рекурсивно вставляет значение в дерево. Если значение строго меньше значения текущего узла, вставьте его в левое поддерево. Если значение больше или равно значению текущего узла, вставьте его в правое поддерево.
    \item Создайте публичный метод locate\_elem в классе SearchStructure, который ищет значение в дереве. Если дерево пустое, верните None. В противном случае, вызовите приватный метод \_locate\_elem\_rec, передав ему self.start\_element и искомое значение.
    \item Создайте приватный метод \_locate\_elem\_rec в классе SearchStructure, который рекурсивно ищет значение в дереве. Если текущий узел равен None или значение текущего узла равно искомому значению, верните текущий узел. В противном случае, рекурсивно вызывайте метод \_locate\_elem\_rec для поиска значения в левом поддереве (если искомое значение меньше текущего) или в правом поддереве (если искомое значение больше или равно текущему).
    \item Создайте экземпляр класса SearchStructure и вставьте в него 26 случайных чисел от 6 до 46.
    \item Выполните поиск элементов в дереве и выведите результаты на экран.
\end{enumerate}

Пример использования:
\begin{lstlisting}[language=Python]
ss = SearchStructure()
for i in range(26):
    ss.insert_elem(random.randint(6, 46))

print("Поиск элементов:")
print(ss.locate_elem(19))  # Обнаружено, возвращен узел (19)
print(ss.locate_elem(59))  # Не обнаружено, возвращено None
print(ss.locate_elem(39))  # Обнаружено, возвращен узел (39)
\end{lstlisting}

\begin{figure}[h]
\centering
\begin{tikzpicture}[level distance=1.5cm,
  level 1/.style={sibling distance=3cm},
  level 2/.style={sibling distance=1.5cm}]
  \node {29}
    child {node {19}
      child {node {14}}
      child {node {24}}
    }
    child {node {39}
      child {node {34}}
      child {node {44}}
    };
\end{tikzpicture}
\caption{Пример бинарного дерева поиска}
\end{figure}

\item Написать программу на Python, которая реализует бинарное дерево поиска с инкапсуляцией. Программа должна создавать экземпляры класса KeyValueNode, которые представляют узлы дерева, и класса BinaryTreeMap, который представляет дерево поиска. Класс BinaryTreeMap должен содержать методы для вставки, поиска и удаления элементов, при этом все рекурсивные методы должны быть приватными. Программа также должна создавать дерево поиска, вставлять в него случайные числа и выполнять поиск элементов в дереве.

Инструкции:
\begin{enumerate}
    \item Создайте класс KeyValueNode с методом \_\_init\_\_, который принимает параметр key\_val и сохраняет его в атрибуте self.map\_key. Атрибуты self.left\_branch и self.right\_branch должны быть инициализированы как None.
    \item Создайте класс BinaryTreeMap с методом \_\_init\_\_, который инициализирует атрибут self.root\_key как None.
    \item Создайте публичный метод put\_key в классе BinaryTreeMap, который вставляет значение в дерево. Если self.root\_key отсутствует, создайте новый узел с вставляемым значением. В противном случае, вызовите приватный метод \_put\_key\_helper, передав ему self.root\_key и значение.
    \item Создайте приватный метод \_put\_key\_helper в классе BinaryTreeMap, который рекурсивно вставляет значение в дерево. Если значение меньше или равно значению текущего узла, вставьте его в левое поддерево. Если значение строго больше значения текущего узла, вставьте его в правое поддерево.
    \item Создайте публичный метод get\_key в классе BinaryTreeMap, который ищет значение в дереве. Если дерево пустое, верните None. В противном случае, вызовите приватный метод \_get\_key\_helper, передав ему self.root\_key и искомое значение.
    \item Создайте приватный метод \_get\_key\_helper в классе BinaryTreeMap, который рекурсивно ищет значение в дереве. Если текущий узел равен None или значение текущего узла равно искомому значению, верните текущий узел. В противном случае, рекурсивно вызывайте метод \_get\_key\_helper для поиска значения в левом поддереве (если искомое значение меньше или равно текущему) или в правом поддереве (если искомое значение больше).
    \item Создайте экземпляр класса BinaryTreeMap и вставьте в него 27 случайных чисел от 7 до 47.
    \item Выполните поиск элементов в дереве и выведите результаты на экран.
\end{enumerate}

Пример использования:
\begin{lstlisting}[language=Python]
btm = BinaryTreeMap()
for i in range(27):
    btm.put_key(random.randint(7, 47))

print("Поиск элементов:")
print(btm.get_key(20))  # Обнаружено, возвращен узел (20)
print(btm.get_key(60))  # Не обнаружено, возвращено None
print(btm.get_key(40))  # Обнаружено, возвращен узел (40)
\end{lstlisting}

\begin{figure}[h]
\centering
\begin{tikzpicture}[level distance=1.5cm,
  level 1/.style={sibling distance=3cm},
  level 2/.style={sibling distance=1.5cm}]
  \node {30}
    child {node {20}
      child {node {15}}
      child {node {25}}
    }
    child {node {40}
      child {node {35}}
      child {node {45}}
    };
\end{tikzpicture}
\caption{Пример бинарного дерева поиска}
\end{figure}

\item Написать программу на Python, которая реализует бинарное дерево поиска с инкапсуляцией. Программа должна создавать экземпляры класса MapNode, которые представляют узлы дерева, и класса KeyTree, который представляет дерево поиска. Класс KeyTree должен содержать методы для вставки, поиска и удаления элементов, при этом все вспомогательные методы должны быть приватными. Программа также должна создавать дерево поиска, вставлять в него случайные числа и выполнять поиск элементов в дереве.

Инструкции:
\begin{enumerate}
    \item Создайте класс MapNode с методом \_\_init\_\_, который принимает параметр map\_value и сохраняет его в атрибуте self.tree\_key. Атрибуты self.left\_part и self.right\_part должны быть инициализированы как None.
    \item Создайте класс KeyTree с методом \_\_init\_\_, который инициализирует атрибут self.base\_key как None.
    \item Создайте публичный метод insert\_map в классе KeyTree, который вставляет значение в дерево. Если self.base\_key отсутствует, создайте новый узел с вставляемым значением. В противном случае, вызовите приватный метод \_insert\_map\_rec, передав ему self.base\_key и значение.
    \item Создайте приватный метод \_insert\_map\_rec в классе KeyTree, который рекурсивно вставляет значение в дерево. Если значение строго меньше значения текущего узла, вставьте его в левое поддерево. Если значение больше или равно значению текущего узла, вставьте его в правое поддерево.
    \item Создайте публичный метод search\_map в классе KeyTree, который ищет значение в дереве. Если дерево пустое, верните None. В противном случае, вызовите приватный метод \_search\_map\_rec, передав ему self.base\_key и искомое значение.
    \item Создайте приватный метод \_search\_map\_rec в классе KeyTree, который рекурсивно ищет значение в дереве. Если текущий узел равен None или значение текущего узла равно искомому значению, верните текущий узел. В противном случае, рекурсивно вызывайте метод \_search\_map\_rec для поиска значения в левом поддереве (если искомое значение меньше текущего) или в правом поддереве (если искомое значение больше или равно текущему).
    \item Создайте экземпляр класса KeyTree и вставьте в него 28 случайных чисел от 8 до 48.
    \item Выполните поиск элементов в дереве и выведите результаты на экран.
\end{enumerate}

Пример использования:
\begin{lstlisting}[language=Python]
kt = KeyTree()
for i in range(28):
    kt.insert_map(random.randint(8, 48))

print("Поиск элементов:")
print(kt.search_map(21))  # Обнаружено, возвращен узел (21)
print(kt.search_map(61))  # Не обнаружено, возвращено None
print(kt.search_map(41))  # Обнаружено, возвращен узел (41)
\end{lstlisting}

\begin{figure}[h]
\centering
\begin{tikzpicture}[level distance=1.5cm,
  level 1/.style={sibling distance=3cm},
  level 2/.style={sibling distance=1.5cm}]
  \node {31}
    child {node {21}
      child {node {16}}
      child {node {26}}
    }
    child {node {41}
      child {node {36}}
      child {node {46}}
    };
\end{tikzpicture}
\caption{Пример бинарного дерева поиска}
\end{figure}

\item Написать программу на Python, которая реализует бинарное дерево поиска с инкапсуляцией. Программа должна создавать экземпляры класса TreeKeyNode, которые представляют узлы дерева, и класса ValueTree, который представляет дерево поиска. Класс ValueTree должен содержать методы для вставки, поиска и удаления элементов, при этом все рекурсивные методы должны быть приватными. Программа также должна создавать дерево поиска, вставлять в него случайные числа и выполнять поиск элементов в дереве.

Инструкции:
\begin{enumerate}
    \item Создайте класс TreeKeyNode с методом \_\_init\_\_, который принимает параметр tree\_key\_val и сохраняет его в атрибуте self.value\_key. Атрибуты self.left и self.right должны быть инициализированы как None.
    \item Создайте класс ValueTree с методом \_\_init\_\_, который инициализирует атрибут self.first\_key как None.
    \item Создайте публичный метод add\_value в классе ValueTree, который вставляет значение в дерево. Если self.first\_key отсутствует, создайте новый узел с вставляемым значением. В противном случае, вызовите приватный метод \_add\_value\_helper, передав ему self.first\_key и значение.
    \item Создайте приватный метод \_add\_value\_helper в классе ValueTree, который рекурсивно вставляет значение в дерево. Если значение меньше или равно значению текущего узла, вставьте его в левое поддерево. Если значение строго больше значения текущего узла, вставьте его в правое поддерево.
    \item Создайте публичный метод retrieve\_value в классе ValueTree, который ищет значение в дереве. Если дерево пустое, верните None. В противном случае, вызовите приватный метод \_retrieve\_value\_helper, передав ему self.first\_key и искомое значение.
    \item Создайте приватный метод \_retrieve\_value\_helper в классе ValueTree, который рекурсивно ищет значение в дереве. Если текущий узел равен None или значение текущего узла равно искомому значению, верните текущий узел. В противном случае, рекурсивно вызывайте метод \_retrieve\_value\_helper для поиска значения в левом поддереве (если искомое значение меньше или равно текущему) или в правом поддереве (если искомое значение больше).
    \item Создайте экземпляр класса ValueTree и вставьте в него 29 случайных чисел от 9 до 49.
    \item Выполните поиск элементов в дереве и выведите результаты на экран.
\end{enumerate}

Пример использования:
\begin{lstlisting}[language=Python]
vt = ValueTree()
for i in range(29):
    vt.add_value(random.randint(9, 49))

print("Поиск элементов:")
print(vt.retrieve_value(22))  # Обнаружено, возвращен узел (22)
print(vt.retrieve_value(62))  # Не обнаружено, возвращено None
print(vt.retrieve_value(42))  # Обнаружено, возвращен узел (42)
\end{lstlisting}

\begin{figure}[h]
\centering
\begin{tikzpicture}[level distance=1.5cm,
  level 1/.style={sibling distance=3cm},
  level 2/.style={sibling distance=1.5cm}]
  \node {32}
    child {node {22}
      child {node {17}}
      child {node {27}}
    }
    child {node {42}
      child {node {37}}
      child {node {47}}
    };
\end{tikzpicture}
\caption{Пример бинарного дерева поиска}
\end{figure}

\item Написать программу на Python, которая реализует бинарное дерево поиска с инкапсуляцией. Программа должна создавать экземпляры класса ValueNode, которые представляют узлы дерева, и класса KeyedTree, который представляет дерево поиска. Класс KeyedTree должен содержать методы для вставки, поиска и удаления элементов, при этом все вспомогательные методы должны быть приватными. Программа также должна создавать дерево поиска, вставлять в него случайные числа и выполнять поиск элементов в дереве.

Инструкции:
\begin{enumerate}
    \item Создайте класс ValueNode с методом \_\_init\_\_, который принимает параметр node\_value и сохраняет его в атрибуте self.keyed\_value. Атрибуты self.left\_side и self.right\_side должны быть инициализированы как None.
    \item Создайте класс KeyedTree с методом \_\_init\_\_, который инициализирует атрибут self.start\_key как None.
    \item Создайте публичный метод store\_value в классе KeyedTree, который вставляет значение в дерево. Если self.start\_key отсутствует, создайте новый узел с вставляемым значением. В противном случае, вызовите приватный метод \_store\_value\_rec, передав ему self.start\_key и значение.
    \item Создайте приватный метод \_store\_value\_rec в классе KeyedTree, который рекурсивно вставляет значение в дерево. Если значение строго меньше значения текущего узла, вставьте его в левое поддерево. Если значение больше или равно значению текущего узла, вставьте его в правое поддерево.
    \item Создайте публичный метод fetch\_value в классе KeyedTree, который ищет значение в дереве. Если дерево пустое, верните None. В противном случае, вызовите приватный метод \_fetch\_value\_rec, передав ему self.start\_key и искомое значение.
    \item Создайте приватный метод \_fetch\_value\_rec в классе KeyedTree, который рекурсивно ищет значение в дереве. Если текущий узел равен None или значение текущего узла равно искомому значению, верните текущий узел. В противном случае, рекурсивно вызывайте метод \_fetch\_value\_rec для поиска значения в левом поддереве (если искомое значение меньше текущего) или в правом поддереве (если искомое значение больше или равно текущему).
    \item Создайте экземпляр класса KeyedTree и вставьте в него 30 случайных чисел от 10 до 50.
    \item Выполните поиск элементов в дереве и выведите результаты на экран.
\end{enumerate}

Пример использования:
\begin{lstlisting}[language=Python]
kt = KeyedTree()
for i in range(30):
    kt.store_value(random.randint(10, 50))

print("Поиск элементов:")
print(kt.fetch_value(23))  # Обнаружено, возвращен узел (23)
print(kt.fetch_value(63))  # Не обнаружено, возвращено None
print(kt.fetch_value(43))  # Обнаружено, возвращен узел (43)
\end{lstlisting}

\begin{figure}[h]
\centering
\begin{tikzpicture}[level distance=1.5cm,
  level 1/.style={sibling distance=3cm},
  level 2/.style={sibling distance=1.5cm}]
  \node {33}
    child {node {23}
      child {node {18}}
      child {node {28}}
    }
    child {node {43}
      child {node {38}}
      child {node {48}}
    };
\end{tikzpicture}
\caption{Пример бинарного дерева поиска}
\end{figure}

\item Написать программу на Python, которая реализует бинарное дерево поиска с инкапсуляцией. Программа должна создавать экземпляры класса KeyedNode, которые представляют узлы дерева, и класса ValuedTree, который представляет дерево поиска. Класс ValuedTree должен содержать методы для вставки, поиска и удаления элементов, при этом все рекурсивные методы должны быть приватными. Программа также должна создавать дерево поиска, вставлять в него случайные числа и выполнять поиск элементов в дереве.

Инструкции:
\begin{enumerate}
    \item Создайте класс KeyedNode с методом \_\_init\_\_, который принимает параметр keyed\_val и сохраняет его в атрибуте self.node\_content. Атрибуты self.left\_path и self.right\_path должны быть инициализированы как None.
    \item Создайте класс ValuedTree с методом \_\_init\_\_, который инициализирует атрибут self.root\_content как None.
    \item Создайте публичный метод insert\_content в классе ValuedTree, который вставляет значение в дерево. Если self.root\_content отсутствует, создайте новый узел с вставляемым значением. В противном случае, вызовите приватный метод \_insert\_content\_helper, передав ему self.root\_content и значение.
    \item Создайте приватный метод \_insert\_content\_helper в классе ValuedTree, который рекурсивно вставляет значение в дерево. Если значение меньше или равно значению текущего узла, вставьте его в левое поддерево. Если значение строго больше значения текущего узла, вставьте его в правое поддерево.
    \item Создайте публичный метод search\_content в классе ValuedTree, который ищет значение в дереве. Если дерево пустое, верните None. В противном случае, вызовите приватный метод \_search\_content\_helper, передав ему self.root\_content и искомое значение.
    \item Создайте приватный метод \_search\_content\_helper в классе ValuedTree, который рекурсивно ищет значение в дереве. Если текущий узел равен None или значение текущего узла равно искомому значению, верните текущий узел. В противном случае, рекурсивно вызывайте метод \_search\_content\_helper для поиска значения в левом поддереве (если искомое значение меньше или равно текущему) или в правом поддереве (если искомое значение больше).
    \item Создайте экземпляр класса ValuedTree и вставьте в него 31 случайное число от 11 до 51.
    \item Выполните поиск элементов в дереве и выведите результаты на экран.
\end{enumerate}

Пример использования:
\begin{lstlisting}[language=Python]
vt = ValuedTree()
for i in range(31):
    vt.insert_content(random.randint(11, 51))

print("Поиск элементов:")
print(vt.search_content(24))  # Обнаружено, возвращен узел (24)
print(vt.search_content(64))  # Не обнаружено, возвращено None
print(vt.search_content(44))  # Обнаружено, возвращен узел (44)
\end{lstlisting}

\begin{figure}[h]
\centering
\begin{tikzpicture}[level distance=1.5cm,
  level 1/.style={sibling distance=3cm},
  level 2/.style={sibling distance=1.5cm}]
  \node {34}
    child {node {24}
      child {node {19}}
      child {node {29}}
    }
    child {node {44}
      child {node {39}}
      child {node {49}}
    };
\end{tikzpicture}
\caption{Пример бинарного дерева поиска}
\end{figure}

\item Написать программу на Python, которая реализует бинарное дерево поиска с инкапсуляцией. Программа должна создавать экземпляры класса ContentNode, которые представляют узлы дерева, и класса KeyTreeStructure, который представляет дерево поиска. Класс KeyTreeStructure должен содержать методы для вставки, поиска и удаления элементов, при этом все вспомогательные методы должны быть приватными. Программа также должна создавать дерево поиска, вставлять в него случайные числа и выполнять поиск элементов в дереве.

Инструкции:
\begin{enumerate}
    \item Создайте класс ContentNode с методом \_\_init\_\_, который принимает параметр content\_val и сохраняет его в атрибуте self.node\_data. Атрибуты self.left\_item и self.right\_item должны быть инициализированы как None.
    \item Создайте класс KeyTreeStructure с методом \_\_init\_\_, который инициализирует атрибут self.top\_data как None.
    \item Создайте публичный метод add\_data в классе KeyTreeStructure, который вставляет значение в дерево. Если self.top\_data отсутствует, создайте новый узел с вставляемым значением. В противном случае, вызовите приватный метод \_add\_data\_rec, передав ему self.top\_data и значение.
    \item Создайте приватный метод \_add\_data\_rec в классе KeyTreeStructure, который рекурсивно вставляет значение в дерево. Если значение строго меньше значения текущего узла, вставьте его в левое поддерево. Если значение больше или равно значению текущего узла, вставьте его в правое поддерево.
    \item Создайте публичный метод find\_data в классе KeyTreeStructure, который ищет значение в дереве. Если дерево пустое, верните None. В противном случае, вызовите приватный метод \_find\_data\_rec, передав ему self.top\_data и искомое значение.
    \item Создайте приватный метод \_find\_data\_rec в классе KeyTreeStructure, который рекурсивно ищет значение в дереве. Если текущий узел равен None или значение текущего узла равно искомому значению, верните текущий узел. В противном случае, рекурсивно вызывайте метод \_find\_data\_rec для поиска значения в левом поддереве (если искомое значение меньше текущего) или в правом поддереве (если искомое значение больше или равно текущему).
    \item Создайте экземпляр класса KeyTreeStructure и вставьте в него 32 случайных числа от 12 до 52.
    \item Выполните поиск элементов в дереве и выведите результаты на экран.
\end{enumerate}

Пример использования:
\begin{lstlisting}[language=Python]
kts = KeyTreeStructure()
for i in range(32):
    kts.add_data(random.randint(12, 52))

print("Поиск элементов:")
print(kts.find_data(25))  # Обнаружено, возвращен узел (25)
print(kts.find_data(65))  # Не обнаружено, возвращено None
print(kts.find_data(45))  # Обнаружено, возвращен узел (45)
\end{lstlisting}

\begin{figure}[h]
\centering
\begin{tikzpicture}[level distance=1.5cm,
  level 1/.style={sibling distance=3cm},
  level 2/.style={sibling distance=1.5cm}]
  \node {35}
    child {node {25}
      child {node {20}}
      child {node {30}}
    }
    child {node {45}
      child {node {40}}
      child {node {50}}
    };
\end{tikzpicture}
\caption{Пример бинарного дерева поиска}
\end{figure}

\item Написать программу на Python, которая реализует бинарное дерево поиска с инкапсуляцией. Программа должна создавать экземпляры класса DataNode, которые представляют узлы дерева, и класса ContentTree, который представляет дерево поиска. Класс ContentTree должен содержать методы для вставки, поиска и удаления элементов, при этом все рекурсивные методы должны быть приватными. Программа также должна создавать дерево поиска, вставлять в него случайные числа и выполнять поиск элементов в дереве.

Инструкции:
\begin{enumerate}
    \item Создайте класс DataNode с методом \_\_init\_\_, который принимает параметр data\_value и сохраняет его в атрибуте self.tree\_content. Атрибуты self.left\_entry и self.right\_entry должны быть инициализированы как None.
    \item Создайте класс ContentTree с методом \_\_init\_\_, который инициализирует атрибут self.root\_content как None.
    \item Создайте публичный метод insert\_entry в классе ContentTree, который вставляет значение в дерево. Если self.root\_content отсутствует, создайте новый узел с вставляемым значением. В противном случае, вызовите приватный метод \_insert\_entry\_helper, передав ему self.root\_content и значение.
    \item Создайте приватный метод \_insert\_entry\_helper в классе ContentTree, который рекурсивно вставляет значение в дерево. Если значение меньше или равно значению текущего узла, вставьте его в левое поддерево. Если значение строго больше значения текущего узла, вставьте его в правое поддерево.
    \item Создайте публичный метод search\_entry в классе ContentTree, который ищет значение в дереве. Если дерево пустое, верните None. В противном случае, вызовите приватный метод \_search\_entry\_helper, передав ему self.root\_content и искомое значение.
    \item Создайте приватный метод \_search\_entry\_helper в классе ContentTree, который рекурсивно ищет значение в дереве. Если текущий узел равен None или значение текущего узла равно искомому значению, верните текущий узел. В противном случае, рекурсивно вызывайте метод \_search\_entry\_helper для поиска значения в левом поддереве (если искомое значение меньше или равно текущему) или в правом поддереве (если искомое значение больше).
    \item Создайте экземпляр класса ContentTree и вставьте в него 33 случайных числа от 13 до 53.
    \item Выполните поиск элементов в дереве и выведите результаты на экран.
\end{enumerate}

Пример использования:
\begin{lstlisting}[language=Python]
ct = ContentTree()
for i in range(33):
    ct.insert_entry(random.randint(13, 53))

print("Поиск элементов:")
print(ct.search_entry(26))  # Обнаружено, возвращен узел (26)
print(ct.search_entry(66))  # Не обнаружено, возвращено None
print(ct.search_entry(46))  # Обнаружено, возвращен узел (46)
\end{lstlisting}

\begin{figure}[h]
\centering
\begin{tikzpicture}[level distance=1.5cm,
  level 1/.style={sibling distance=3cm},
  level 2/.style={sibling distance=1.5cm}]
  \node {36}
    child {node {26}
      child {node {21}}
      child {node {31}}
    }
    child {node {46}
      child {node {41}}
      child {node {51}}
    };
\end{tikzpicture}
\caption{Пример бинарного дерева поиска}
\end{figure}

\item Написать программу на Python, которая реализует бинарное дерево поиска с инкапсуляцией. Программа должна создавать экземпляры класса EntryNode, которые представляют узлы дерева, и класса DataStructureTree, который представляет дерево поиска. Класс DataStructureTree должен содержать методы для вставки, поиска и удаления элементов, при этом все вспомогательные методы должны быть приватными. Программа также должна создавать дерево поиска, вставлять в него случайные числа и выполнять поиск элементов в дереве.

Инструкции:
\begin{enumerate}
    \item Создайте класс EntryNode с методом \_\_init\_\_, который принимает параметр entry\_val и сохраняет его в атрибуте self.content\_item. Атрибуты self.left\_data и self.right\_data должны быть инициализированы как None.
    \item Создайте класс DataStructureTree с методом \_\_init\_\_, который инициализирует атрибут self.first\_item как None.
    \item Создайте публичный метод add\_item в классе DataStructureTree, который вставляет значение в дерево. Если self.first\_item отсутствует, создайте новый узел с вставляемым значением. В противном случае, вызовите приватный метод \_add\_item\_rec, передав ему self.first\_item и значение.
    \item Создайте приватный метод \_add\_item\_rec в классе DataStructureTree, который рекурсивно вставляет значение в дерево. Если значение строго меньше значения текущего узла, вставьте его в левое поддерево. Если значение больше или равно значению текущего узла, вставьте его в правое поддерево.
    \item Создайте публичный метод locate\_item в классе DataStructureTree, который ищет значение в дереве. Если дерево пустое, верните None. В противном случае, вызовите приватный метод \_locate\_item\_rec, передав ему self.first\_item и искомое значение.
    \item Создайте приватный метод \_locate\_item\_rec в классе DataStructureTree, который рекурсивно ищет значение в дереве. Если текущий узел равен None или значение текущего узла равно искомому значению, верните текущий узел. В противном случае, рекурсивно вызывайте метод \_locate\_item\_rec для поиска значения в левом поддереве (если искомое значение меньше текущего) или в правом поддереве (если искомое значение больше или равно текущему).
    \item Создайте экземпляр класса DataStructureTree и вставьте в него 34 случайных числа от 14 до 54.
    \item Выполните поиск элементов в дереве и выведите результаты на экран.
\end{enumerate}

Пример использования:
\begin{lstlisting}[language=Python]
dst = DataStructureTree()
for i in range(34):
    dst.add_item(random.randint(14, 54))

print("Поиск элементов:")
print(dst.locate_item(27))  # Обнаружено, возвращен узел (27)
print(dst.locate_item(67))  # Не обнаружено, возвращено None
print(dst.locate_item(47))  # Обнаружено, возвращен узел (47)
\end{lstlisting}

\begin{figure}[h]
\centering
\begin{tikzpicture}[level distance=1.5cm,
  level 1/.style={sibling distance=3cm},
  level 2/.style={sibling distance=1.5cm}]
  \node {37}
    child {node {27}
      child {node {22}}
      child {node {32}}
    }
    child {node {47}
      child {node {42}}
      child {node {52}}
    };
\end{tikzpicture}
\caption{Пример бинарного дерева поиска}
\end{figure}

\item Написать программу на Python, которая реализует бинарное дерево поиска с инкапсуляцией. Программа должна создавать экземпляры класса ItemNode, которые представляют узлы дерева, и класса EntryTree, который представляет дерево поиска. Класс EntryTree должен содержать методы для вставки, поиска и удаления элементов, при этом все рекурсивные методы должны быть приватными. Программа также должна создавать дерево поиска, вставлять в него случайные числа и выполнять поиск элементов в дереве.

Инструкции:
\begin{enumerate}
    \item Создайте класс ItemNode с методом \_\_init\_\_, который принимает параметр item\_value и сохраняет его в атрибуте self.data\_entry. Атрибуты self.left\_position и self.right\_position должны быть инициализированы как None.
    \item Создайте класс EntryTree с методом \_\_init\_\_, который инициализирует атрибут self.root\_entry как None.
    \item Создайте публичный метод insert\_position в классе EntryTree, который вставляет значение в дерево. Если self.root\_entry отсутствует, создайте новый узел с вставляемым значением. В противном случае, вызовите приватный метод \_insert\_position\_helper, передав ему self.root\_entry и значение.
    \item Создайте приватный метод \_insert\_position\_helper в классе EntryTree, который рекурсивно вставляет значение в дерево. Если значение меньше или равно значению текущего узла, вставьте его в левое поддерево. Если значение строго больше значения текущего узла, вставьте его в правое поддерево.
    \item Создайте публичный метод find\_position в классе EntryTree, который ищет значение в дереве. Если дерево пустое, верните None. В противном случае, вызовите приватный метод \_find\_position\_helper, передав ему self.root\_entry и искомое значение.
    \item Создайте приватный метод \_find\_position\_helper в классе EntryTree, который рекурсивно ищет значение в дереве. Если текущий узел равен None или значение текущего узла равно искомому значению, верните текущий узел. В противном случае, рекурсивно вызывайте метод \_find\_position\_helper для поиска значения в левом поддереве (если искомое значение меньше или равно текущему) или в правом поддереве (если искомое значение больше).
    \item Создайте экземпляр класса EntryTree и вставьте в него 35 случайных чисел от 15 до 55.
    \item Выполните поиск элементов в дереве и выведите результаты на экран.
\end{enumerate}

Пример использования:
\begin{lstlisting}[language=Python]
et = EntryTree()
for i in range(35):
    et.insert_position(random.randint(15, 55))

print("Поиск элементов:")
print(et.find_position(28))  # Обнаружено, возвращен узел (28)
print(et.find_position(68))  # Не обнаружено, возвращено None
print(et.find_position(48))  # Обнаружено, возвращен узел (48)
\end{lstlisting}

\begin{figure}[h]
\centering
\begin{tikzpicture}[level distance=1.5cm,
  level 1/.style={sibling distance=3cm},
  level 2/.style={sibling distance=1.5cm}]
  \node {38}
    child {node {28}
      child {node {23}}
      child {node {33}}
    }
    child {node {48}
      child {node {43}}
      child {node {53}}
    };
\end{tikzpicture}
\caption{Пример бинарного дерева поиска}
\end{figure}

\item Написать программу на Python, которая реализует бинарное дерево поиска с инкапсуляцией. Программа должна создавать экземпляры класса PositionNode, которые представляют узлы дерева, и класса ItemStructure, который представляет дерево поиска. Класс ItemStructure должен содержать методы для вставки, поиска и удаления элементов, при этом все вспомогательные методы должны быть приватными. Программа также должна создавать дерево поиска, вставлять в него случайные числа и выполнять поиск элементов в дереве.

Инструкции:
\begin{enumerate}
    \item Создайте класс PositionNode с методом \_\_init\_\_, который принимает параметр position\_val и сохраняет его в атрибуте self.entry\_data. Атрибуты self.left\_slot и self.right\_slot должны быть инициализированы как None.
    \item Создайте класс ItemStructure с методом \_\_init\_\_, который инициализирует атрибут self.top\_entry как None.
    \item Создайте публичный метод add\_slot в классе ItemStructure, который вставляет значение в дерево. Если self.top\_entry отсутствует, создайте новый узел с вставляемым значением. В противном случае, вызовите приватный метод \_add\_slot\_rec, передав ему self.top\_entry и значение.
    \item Создайте приватный метод \_add\_slot\_rec в классе ItemStructure, который рекурсивно вставляет значение в дерево. Если значение строго меньше значения текущего узла, вставьте его в левое поддерево. Если значение больше или равно значению текущего узла, вставьте его в правое поддерево.
    \item Создайте публичный метод search\_slot в классе ItemStructure, который ищет значение в дереве. Если дерево пустое, верните None. В противном случае, вызовите приватный метод \_search\_slot\_rec, передав ему self.top\_entry и искомое значение.
    \item Создайте приватный метод \_search\_slot\_rec в классе ItemStructure, который рекурсивно ищет значение в дереве. Если текущий узел равен None или значение текущего узла равно искомому значению, верните текущий узел. В противном случае, рекурсивно вызывайте метод \_search\_slot\_rec для поиска значения в левом поддереве (если искомое значение меньше текущего) или в правом поддереве (если искомое значение больше или равно текущему).
    \item Создайте экземпляр класса ItemStructure и вставьте в него 36 случайных чисел от 16 до 56.
    \item Выполните поиск элементов в дереве и выведите результаты на экран.
\end{enumerate}

Пример использования:
\begin{lstlisting}[language=Python]
is_ = ItemStructure()
for i in range(36):
    is_.add_slot(random.randint(16, 56))

print("Поиск элементов:")
print(is_.search_slot(29))  # Обнаружено, возвращен узел (29)
print(is_.search_slot(69))  # Не обнаружено, возвращено None
print(is_.search_slot(49))  # Обнаружено, возвращен узел (49)
\end{lstlisting}

\begin{figure}[h]
\centering
\begin{tikzpicture}[level distance=1.5cm,
  level 1/.style={sibling distance=3cm},
  level 2/.style={sibling distance=1.5cm}]
  \node {39}
    child {node {29}
      child {node {24}}
      child {node {34}}
    }
    child {node {49}
      child {node {44}}
      child {node {54}}
    };
\end{tikzpicture}
\caption{Пример бинарного дерева поиска}
\end{figure}

\item Написать программу на Python, которая реализует бинарное дерево поиска с инкапсуляцией. Программа должна создавать экземпляры класса SlotNode, которые представляют узлы дерева, и класса PositionTree, который представляет дерево поиска. Класс PositionTree должен содержать методы для вставки, поиска и удаления элементов, при этом все рекурсивные методы должны быть приватными. Программа также должна создавать дерево поиска, вставлять в него случайные числа и выполнять поиск элементов в дереве.

Инструкции:
\begin{enumerate}
    \item Создайте класс SlotNode с методом \_\_init\_\_, который принимает параметр slot\_value и сохраняет его в атрибуте self.item\_position. Атрибуты self.left\_place и self.right\_place должны быть инициализированы как None.
    \item Создайте класс PositionTree с методом \_\_init\_\_, который инициализирует атрибут self.first\_position как None.
    \item Создайте публичный метод insert\_place в классе PositionTree, который вставляет значение в дерево. Если self.first\_position отсутствует, создайте новый узел с вставляемым значением. В противном случае, вызовите приватный метод \_insert\_place\_helper, передав ему self.first\_position и значение.
    \item Создайте приватный метод \_insert\_place\_helper в классе PositionTree, который рекурсивно вставляет значение в дерево. Если значение меньше или равно значению текущего узла, вставьте его в левое поддерево. Если значение строго больше значения текущего узла, вставьте его в правое поддерево.
    \item Создайте публичный метод locate\_place в классе PositionTree, который ищет значение в дереве. Если дерево пустое, верните None. В противном случае, вызовите приватный метод \_locate\_place\_helper, передав ему self.first\_position и искомое значение.
    \item Создайте приватный метод \_locate\_place\_helper в классе PositionTree, который рекурсивно ищет значение в дереве. Если текущий узел равен None или значение текущего узла равно искомому значению, верните текущий узел. В противном случае, рекурсивно вызывайте метод \_locate\_place\_helper для поиска значения в левом поддереве (если искомое значение меньше или равно текущему) или в правом поддереве (если искомое значение больше).
    \item Создайте экземпляр класса PositionTree и вставьте в него 37 случайных чисел от 17 до 57.
    \item Выполните поиск элементов в дереве и выведите результаты на экран.
\end{enumerate}

Пример использования:
\begin{lstlisting}[language=Python]
pt = PositionTree()
for i in range(37):
    pt.insert_place(random.randint(17, 57))

print("Поиск элементов:")
print(pt.locate_place(30))  # Обнаружено, возвращен узел (30)
print(pt.locate_place(70))  # Не обнаружено, возвращено None
print(pt.locate_place(50))  # Обнаружено, возвращен узел (50)
\end{lstlisting}

\begin{figure}[h]
\centering
\begin{tikzpicture}[level distance=1.5cm,
  level 1/.style={sibling distance=3cm},
  level 2/.style={sibling distance=1.5cm}]
  \node {40}
    child {node {30}
      child {node {25}}
      child {node {35}}
    }
    child {node {50}
      child {node {45}}
      child {node {55}}
    };
\end{tikzpicture}
\caption{Пример бинарного дерева поиска}
\end{figure}

\item Написать программу на Python, которая реализует бинарное дерево поиска с инкапсуляцией. Программа должна создавать экземпляры класса PlaceNode, которые представляют узлы дерева, и класса SlotTree, который представляет дерево поиска. Класс SlotTree должен содержать методы для вставки, поиска и удаления элементов, при этом все вспомогательные методы должны быть приватными. Программа также должна создавать дерево поиска, вставлять в него случайные числа и выполнять поиск элементов в дереве.

Инструкции:
\begin{enumerate}
    \item Создайте класс PlaceNode с методом \_\_init\_\_, который принимает параметр place\_val и сохраняет его в атрибуте self.position\_item. Атрибуты self.left\_spot и self.right\_spot должны быть инициализированы как None.
    \item Создайте класс SlotTree с методом \_\_init\_\_, который инициализирует атрибут self.root\_position как None.
    \item Создайте публичный метод add\_spot в классе SlotTree, который вставляет значение в дерево. Если self.root\_position отсутствует, создайте новый узел с вставляемым значением. В противном случае, вызовите приватный метод \_add\_spot\_rec, передав ему self.root\_position и значение.
    \item Создайте приватный метод \_add\_spot\_rec в классе SlotTree, который рекурсивно вставляет значение в дерево. Если значение строго меньше значения текущего узла, вставьте его в левое поддерево. Если значение больше или равно значению текущего узла, вставьте его в правое поддерево.
    \item Создайте публичный метод find\_spot в классе SlotTree, который ищет значение в дереве. Если дерево пустое, верните None. В противном случае, вызовите приватный метод \_find\_spot\_rec, передав ему self.root\_position и искомое значение.
    \item Создайте приватный метод \_find\_spot\_rec в классе SlotTree, который рекурсивно ищет значение в дереве. Если текущий узел равен None или значение текущего узла равно искомому значению, верните текущий узел. В противном случае, рекурсивно вызывайте метод \_find\_spot\_rec для поиска значения в левом поддереве (если искомое значение меньше текущего) или в правом поддереве (если искомое значение больше или равно текущему).
    \item Создайте экземпляр класса SlotTree и вставьте в него 38 случайных чисел от 18 до 58.
    \item Выполните поиск элементов в дереве и выведите результаты на экран.
\end{enumerate}

Пример использования:
\begin{lstlisting}[language=Python]
st = SlotTree()
for i in range(38):
    st.add_spot(random.randint(18, 58))

print("Поиск элементов:")
print(st.find_spot(31))  # Обнаружено, возвращен узел (31)
print(st.find_spot(71))  # Не обнаружено, возвращено None
print(st.find_spot(51))  # Обнаружено, возвращен узел (51)
\end{lstlisting}

\begin{figure}[h]
\centering
\begin{tikzpicture}[level distance=1.5cm,
  level 1/.style={sibling distance=3cm},
  level 2/.style={sibling distance=1.5cm}]
  \node {41}
    child {node {31}
      child {node {26}}
      child {node {36}}
    }
    child {node {51}
      child {node {46}}
      child {node {56}}
    };
\end{tikzpicture}
\caption{Пример бинарного дерева поиска}
\end{figure}

\item Написать программу на Python, которая реализует бинарное дерево поиска с инкапсуляцией. Программа должна создавать экземпляры класса SpotNode, которые представляют узлы дерева, и класса PlaceIndex, который представляет дерево поиска. Класс PlaceIndex должен содержать методы для вставки, поиска и удаления элементов, при этом все рекурсивные методы должны быть приватными. Программа также должна создавать дерево поиска, вставлять в него случайные числа и выполнять поиск элементов в дереве.

Инструкции:
\begin{enumerate}
    \item Создайте класс SpotNode с методом \_\_init\_\_, который принимает параметр spot\_value и сохраняет его в атрибуте self.index\_position. Атрибуты self.left\_location и self.right\_location должны быть инициализированы как None.
    \item Создайте класс PlaceIndex с методом \_\_init\_\_, который инициализирует атрибут self.start\_position как None.
    \item Создайте публичный метод insert\_location в классе PlaceIndex, который вставляет значение в дерево. Если self.start\_position отсутствует, создайте новый узел с вставляемым значением. В противном случае, вызовите приватный метод \_insert\_location\_helper, передав ему self.start\_position и значение.
    \item Создайте приватный метод \_insert\_location\_helper в классе PlaceIndex, который рекурсивно вставляет значение в дерево. Если значение меньше или равно значению текущего узла, вставьте его в левое поддерево. Если значение строго больше значения текущего узла, вставьте его в правое поддерево.
    \item Создайте публичный метод search\_location в классе PlaceIndex, который ищет значение в дереве. Если дерево пустое, верните None. В противном случае, вызовите приватный метод \_search\_location\_helper, передав ему self.start\_position и искомое значение.
    \item Создайте приватный метод \_search\_location\_helper в классе PlaceIndex, который рекурсивно ищет значение в дереве. Если текущий узел равен None или значение текущего узла равно искомому значению, верните текущий узел. В противном случае, рекурсивно вызывайте метод \_search\_location\_helper для поиска значения в левом поддереве (если искомое значение меньше или равно текущему) или в правом поддереве (если искомое значение больше).
    \item Создайте экземпляр класса PlaceIndex и вставьте в него 39 случайных чисел от 19 до 59.
    \item Выполните поиск элементов в дереве и выведите результаты на экран.
\end{enumerate}

Пример использования:
\begin{lstlisting}[language=Python]
pi = PlaceIndex()
for i in range(39):
    pi.insert_location(random.randint(19, 59))

print("Поиск элементов:")
print(pi.search_location(32))  # Обнаружено, возвращен узел (32)
print(pi.search_location(72))  # Не обнаружено, возвращено None
print(pi.search_location(52))  # Обнаружено, возвращен узел (52)
\end{lstlisting}

\begin{figure}[h]
\centering
\begin{tikzpicture}[level distance=1.5cm,
  level 1/.style={sibling distance=3cm},
  level 2/.style={sibling distance=1.5cm}]
  \node {42}
    child {node {32}
      child {node {27}}
      child {node {37}}
    }
    child {node {52}
      child {node {47}}
      child {node {57}}
    };
\end{tikzpicture}
\caption{Пример бинарного дерева поиска}
\end{figure}

\item Написать программу на Python, которая реализует бинарное дерево поиска с инкапсуляцией. Программа должна создавать экземпляры класса LocationNode, которые представляют узлы дерева, и класса SpotTree, который представляет дерево поиска. Класс SpotTree должен содержать методы для вставки, поиска и удаления элементов, при этом все вспомогательные методы должны быть приватными. Программа также должна создавать дерево поиска, вставлять в него случайные числа и выполнять поиск элементов в дереве.

Инструкции:
\begin{enumerate}
    \item Создайте класс LocationNode с методом \_\_init\_\_, который принимает параметр location\_val и сохраняет его в атрибуте self.tree\_spot. Атрибуты self.left\_site и self.right\_site должны быть инициализированы как None.
    \item Создайте класс SpotTree с методом \_\_init\_\_, который инициализирует атрибут self.base\_spot как None.
    \item Создайте публичный метод add\_site в классе SpotTree, который вставляет значение в дерево. Если self.base\_spot отсутствует, создайте новый узел с вставляемым значением. В противном случае, вызовите приватный метод \_add\_site\_rec, передав ему self.base\_spot и значение.
    \item Создайте приватный метод \_add\_site\_rec в классе SpotTree, который рекурсивно вставляет значение в дерево. Если значение строго меньше значения текущего узла, вставьте его в левое поддерево. Если значение больше или равно значению текущего узла, вставьте его в правое поддерево.
    \item Создайте публичный метод locate\_site в классе SpotTree, который ищет значение в дереве. Если дерево пустое, верните None. В противном случае, вызовите приватный метод \_locate\_site\_rec, передав ему self.base\_spot и искомое значение.
    \item Создайте приватный метод \_locate\_site\_rec в классе SpotTree, который рекурсивно ищет значение в дереве. Если текущий узел равен None или значение текущего узла равно искомому значению, верните текущий узел. В противном случае, рекурсивно вызывайте метод \_locate\_site\_rec для поиска значения в левом поддереве (если искомое значение меньше текущего) или в правом поддереве (если искомое значение больше или равно текущему).
    \item Создайте экземпляр класса SpotTree и вставьте в него 40 случайных чисел от 20 до 60.
    \item Выполните поиск элементов в дереве и выведите результаты на экран.
\end{enumerate}

Пример использования:
\begin{lstlisting}[language=Python]
spot_tree = SpotTree()
for i in range(40):
    spot_tree.add_site(random.randint(20, 60))

print("Поиск элементов:")
print(spot_tree.locate_site(33))  # Обнаружено, возвращен узел (33)
print(spot_tree.locate_site(73))  # Не обнаружено, возвращено None
print(spot_tree.locate_site(53))  # Обнаружено, возвращен узел (53)
\end{lstlisting}

\begin{figure}[h]
\centering
\begin{tikzpicture}[level distance=1.5cm,
  level 1/.style={sibling distance=3cm},
  level 2/.style={sibling distance=1.5cm}]
  \node {43}
    child {node {33}
      child {node {28}}
      child {node {38}}
    }
    child {node {53}
      child {node {48}}
      child {node {58}}
    };
\end{tikzpicture}
\caption{Пример бинарного дерева поиска}
\end{figure}

\item Написать программу на Python, которая реализует бинарное дерево поиска с инкапсуляцией. Программа должна создавать экземпляры класса SiteNode, которые представляют узлы дерева, и класса LocationIndex, который представляет дерево поиска. Класс LocationIndex должен содержать методы для вставки, поиска и удаления элементов, при этом все рекурсивные методы должны быть приватными. Программа также должна создавать дерево поиска, вставлять в него случайные числа и выполнять поиск элементов в дереве.

Инструкции:
\begin{enumerate}
    \item Создайте класс SiteNode с методом \_\_init\_\_, который принимает параметр site\_value и сохраняет его в атрибуте self.index\_location. Атрибуты self.left\_zone и self.right\_zone должны быть инициализированы как None.
    \item Создайте класс LocationIndex с методом \_\_init\_\_, который инициализирует атрибут self.root\_location как None.
    \item Создайте публичный метод insert\_zone в классе LocationIndex, который вставляет значение в дерево. Если self.root\_location отсутствует, создайте новый узел с вставляемым значением. В противном случае, вызовите приватный метод \_insert\_zone\_helper, передав ему self.root\_location и значение.
    \item Создайте приватный метод \_insert\_zone\_helper в классе LocationIndex, который рекурсивно вставляет значение в дерево. Если значение меньше или равно значению текущего узла, вставьте его в левое поддерево. Если значение строго больше значения текущего узла, вставьте его в правое поддерево.
    \item Создайте публичный метод find\_zone в классе LocationIndex, который ищет значение в дереве. Если дерево пустое, верните None. В противном случае, вызовите приватный метод \_find\_zone\_helper, передав ему self.root\_location и искомое значение.
    \item Создайте приватный метод \_find\_zone\_helper в классе LocationIndex, который рекурсивно ищет значение в дереве. Если текущий узел равен None или значение текущего узла равно искомому значению, верните текущий узел. В противном случае, рекурсивно вызывайте метод \_find\_zone\_helper для поиска значения в левом поддереве (если искомое значение меньше или равно текущему) или в правом поддереве (если искомое значение больше).
    \item Создайте экземпляр класса LocationIndex и вставьте в него 41 случайное число от 21 до 61.
    \item Выполните поиск элементов в дереве и выведите результаты на экран.
\end{enumerate}

Пример использования:
\begin{lstlisting}[language=Python]
li = LocationIndex()
for i in range(41):
    li.insert_zone(random.randint(21, 61))

print("Поиск элементов:")
print(li.find_zone(34))  # Обнаружено, возвращен узел (34)
print(li.find_zone(74))  # Не обнаружено, возвращено None
print(li.find_zone(54))  # Обнаружено, возвращен узел (54)
\end{lstlisting}

\begin{figure}[h]
\centering
\begin{tikzpicture}[level distance=1.5cm,
  level 1/.style={sibling distance=3cm},
  level 2/.style={sibling distance=1.5cm}]
  \node {44}
    child {node {34}
      child {node {29}}
      child {node {39}}
    }
    child {node {54}
      child {node {49}}
      child {node {59}}
    };
\end{tikzpicture}
\caption{Пример бинарного дерева поиска}
\end{figure}

\item Написать программу на Python, которая реализует бинарное дерево поиска с инкапсуляцией. Программа должна создавать экземпляры класса ZoneNode, которые представляют узлы дерева, и класса SiteStructure, который представляет дерево поиска. Класс SiteStructure должен содержать методы для вставки, поиска и удаления элементов, при этом все вспомогательные методы должны быть приватными. Программа также должна создавать дерево поиска, вставлять в него случайные числа и выполнять поиск элементов в дереве.

Инструкции:
\begin{enumerate}
    \item Создайте класс ZoneNode с методом \_\_init\_\_, который принимает параметр zone\_val и сохраняет его в атрибуте self.structure\_site. Атрибуты self.left\_region и self.right\_region должны быть инициализированы как None.
    \item Создайте класс SiteStructure с методом \_\_init\_\_, который инициализирует атрибут self.top\_site как None.
    \item Создайте публичный метод add\_region в классе SiteStructure, который вставляет значение в дерево. Если self.top\_site отсутствует, создайте новый узел с вставляемым значением. В противном случае, вызовите приватный метод \_add\_region\_rec, передав ему self.top\_site и значение.
    \item Создайте приватный метод \_add\_region\_rec в классе SiteStructure, который рекурсивно вставляет значение в дерево. Если значение строго меньше значения текущего узла, вставьте его в левое поддерево. Если значение больше или равно значению текущего узла, вставьте его в правое поддерево.
    \item Создайте публичный метод search\_region в классе SiteStructure, который ищет значение в дереве. Если дерево пустое, верните None. В противном случае, вызовите приватный метод \_search\_region\_rec, передав ему self.top\_site и искомое значение.
    \item Создайте приватный метод \_search\_region\_rec в классе SiteStructure, который рекурсивно ищет значение в дереве. Если текущий узел равен None или значение текущего узла равно искомому значению, верните текущий узел. В противном случае, рекурсивно вызывайте метод \_search\_region\_rec для поиска значения в левом поддереве (если искомое значение меньше текущего) или в правом поддереве (если искомое значение больше или равно текущему).
    \item Создайте экземпляр класса SiteStructure и вставьте в него 42 случайных числа от 22 до 62.
    \item Выполните поиск элементов в дереве и выведите результаты на экран.
\end{enumerate}

Пример использования:
\begin{lstlisting}[language=Python]
ss = SiteStructure()
for i in range(42):
    ss.add_region(random.randint(22, 62))

print("Поиск элементов:")
print(ss.search_region(35))  # Обнаружено, возвращен узел (35)
print(ss.search_region(75))  # Не обнаружено, возвращено None
print(ss.search_region(55))  # Обнаружено, возвращен узел (55)
\end{lstlisting}

\begin{figure}[h]
\centering
\begin{tikzpicture}[level distance=1.5cm,
  level 1/.style={sibling distance=3cm},
  level 2/.style={sibling distance=1.5cm}]
  \node {45}
    child {node {35}
      child {node {30}}
      child {node {40}}
    }
    child {node {55}
      child {node {50}}
      child {node {60}}
    };
\end{tikzpicture}
\caption{Пример бинарного дерева поиска}
\end{figure}

\item Написать программу на Python, которая реализует бинарное дерево поиска с инкапсуляцией. Программа должна создавать экземпляры класса RegionNode, которые представляют узлы дерева, и класса ZoneTree, который представляет дерево поиска. Класс ZoneTree должен содержать методы для вставки, поиска и удаления элементов, при этом все рекурсивные методы должны быть приватными. Программа также должна создавать дерево поиска, вставлять в него случайные числа и выполнять поиск элементов в дереве.

Инструкции:
\begin{enumerate}
    \item Создайте класс RegionNode с методом \_\_init\_\_, который принимает параметр region\_value и сохраняет его в атрибуте self.tree\_zone. Атрибуты self.left\_area и self.right\_area должны быть инициализированы как None.
    \item Создайте класс ZoneTree с методом \_\_init\_\_, который инициализирует атрибут self.first\_zone как None.
    \item Создайте публичный метод insert\_area в классе ZoneTree, который вставляет значение в дерево. Если self.first\_zone отсутствует, создайте новый узел с вставляемым значением. В противном случае, вызовите приватный метод \_insert\_area\_helper, передав ему self.first\_zone и значение.
    \item Создайте приватный метод \_insert\_area\_helper в классе ZoneTree, который рекурсивно вставляет значение в дерево. Если значение меньше или равно значению текущего узла, вставьте его в левое поддерево. Если значение строго больше значения текущего узла, вставьте его в правое поддерево.
    \item Создайте публичный метод locate\_area в классе ZoneTree, который ищет значение в дереве. Если дерево пустое, верните None. В противном случае, вызовите приватный метод \_locate\_area\_rec, передав ему self.first\_zone и искомое значение.
    \item Создайте приватный метод \_locate\_area\_rec в классе ZoneTree, который рекурсивно ищет значение в дереве. Если текущий узел равен None или значение текущего узла равно искомому значению, верните текущий узел. В противном случае, рекурсивно вызывайте метод \_locate\_area\_rec для поиска значения в левом поддереве (если искомое значение меньше или равно текущему) или в правом поддереве (если искомое значение больше).
    \item Создайте экземпляр класса ZoneTree и вставьте в него 43 случайных числа от 23 до 63.
    \item Выполните поиск элементов в дереве и выведите результаты на экран.
\end{enumerate}

Пример использования:
\begin{lstlisting}[language=Python]
zt = ZoneTree()
for i in range(43):
    zt.insert_area(random.randint(23, 63))

print("Поиск элементов:")
print(zt.locate_area(36))  # Обнаружено, возвращен узел (36)
print(zt.locate_area(76))  # Не обнаружено, возвращено None
print(zt.locate_area(56))  # Обнаружено, возвращен узел (56)
\end{lstlisting}

\begin{figure}[h]
\centering
\begin{tikzpicture}[level distance=1.5cm,
  level 1/.style={sibling distance=3cm},
  level 2/.style={sibling distance=1.5cm}]
  \node {46}
    child {node {36}
      child {node {31}}
      child {node {41}}
    }
    child {node {56}
      child {node {51}}
      child {node {61}}
    };
\end{tikzpicture}
\caption{Пример бинарного дерева поиска}
\end{figure}

\item Написать программу на Python, которая реализует бинарное дерево поиска с инкапсуляцией. Программа должна создавать экземпляры класса AreaNode, которые представляют узлы дерева, и класса RegionIndex, который представляет дерево поиска. Класс RegionIndex должен содержать методы для вставки, поиска и удаления элементов, при этом все вспомогательные методы должны быть приватными. Программа также должна создавать дерево поиска, вставлять в него случайные числа и выполнять поиск элементов в дереве.

Инструкции:
\begin{enumerate}
    \item Создайте класс AreaNode с методом \_\_init\_\_, который принимает параметр area\_val и сохраняет его в атрибуте self.index\_region. Атрибуты self.left\_district и self.right\_district должны быть инициализированы как None.
    \item Создайте класс RegionIndex с методом \_\_init\_\_, который инициализирует атрибут self.root\_region как None.
    \item Создайте публичный метод add\_district в классе RegionIndex, который вставляет значение в дерево. Если self.root\_region отсутствует, создайте новый узел с вставляемым значением. В противном случае, вызовите приватный метод \_add\_district\_rec, передав ему self.root\_region и значение.
    \item Создайте приватный метод \_add\_district\_rec в классе RegionIndex, который рекурсивно вставляет значение в дерево. Если значение строго меньше значения текущего узла, вставьте его в левое поддерево. Если значение больше или равно значению текущего узла, вставьте его в правое поддерево.
    \item Создайте публичный метод find\_district в классе RegionIndex, который ищет значение в дереве. Если дерево пустое, верните None. В противном случае, вызовите приватный метод \_find\_district\_helper, передав ему self.root\_region и искомое значение.
    \item Создайте приватный метод \_find\_district\_helper в классе RegionIndex, который рекурсивно ищет значение в дереве. Если текущий узел равен None или значение текущего узла равно искомому значению, верните текущий узел. В противном случае, рекурсивно вызывайте метод \_find\_district\_helper для поиска значения в левом поддереве (если искомое значение меньше текущего) или в правом поддереве (если искомое значение больше или равно текущему).
    \item Создайте экземпляр класса RegionIndex и вставьте в него 44 случайных числа от 24 до 64.
    \item Выполните поиск элементов в дереве и выведите результаты на экран.
\end{enumerate}

Пример использования:
\begin{lstlisting}[language=Python]
ri = RegionIndex()
for i in range(44):
    ri.add_district(random.randint(24, 64))

print("Поиск элементов:")
print(ri.find_district(37))  # Обнаружено, возвращен узел (37)
print(ri.find_district(77))  # Не обнаружено, возвращено None
print(ri.find_district(57))  # Обнаружено, возвращен узел (57)
\end{lstlisting}

\begin{figure}[h]
\centering
\begin{tikzpicture}[level distance=1.5cm,
  level 1/.style={sibling distance=3cm},
  level 2/.style={sibling distance=1.5cm}]
  \node {47}
    child {node {37}
      child {node {32}}
      child {node {42}}
    }
    child {node {57}
      child {node {52}}
      child {node {62}}
    };
\end{tikzpicture}
\caption{Пример бинарного дерева поиска}
\end{figure}

\item Написать программу на Python, которая реализует бинарное дерево поиска с инкапсуляцией. Программа должна создавать экземпляры класса DistrictNode, которые представляют узлы дерева, и класса AreaTree, который представляет дерево поиска. Класс AreaTree должен содержать методы для вставки, поиска и удаления элементов, при этом все рекурсивные методы должны быть приватными. Программа также должна создавать дерево поиска, вставлять в него случайные числа и выполнять поиск элементов в дереве.

Инструкции:
\begin{enumerate}
    \item Создайте класс DistrictNode с методом \_\_init\_\_, который принимает параметр district\_value и сохраняет его в атрибуте self.tree\_area. Атрибуты self.left\_sector и self.right\_sector должны быть инициализированы как None.
    \item Создайте класс AreaTree с методом \_\_init\_\_, который инициализирует атрибут self.start\_area как None.
    \item Создайте публичный метод insert\_sector в классе AreaTree, который вставляет значение в дерево. Если self.start\_area отсутствует, создайте новый узел с вставляемым значением. В противном случае, вызовите приватный метод \_insert\_sector\_helper, передав ему self.start\_area и значение.
    \item Создайте приватный метод \_insert\_sector\_helper в классе AreaTree, который рекурсивно вставляет значение в дерево. Если значение меньше или равно значению текущего узла, вставьте его в левое поддерево. Если значение строго больше значения текущего узла, вставьте его в правое поддерево.
    \item Создайте публичный метод search\_sector в классе AreaTree, который ищет значение в дереве. Если дерево пустое, верните None. В противном случае, вызовите приватный метод \_search\_sector\_rec, передав ему self.start\_area и искомое значение.
    \item Создайте приватный метод \_search\_sector\_rec в классе AreaTree, который рекурсивно ищет значение в дереве. Если текущий узел равен None или значение текущего узла равно искомому значению, верните текущий узел. В противном случае, рекурсивно вызывайте метод \_search\_sector\_rec для поиска значения в левом поддереве (если искомое значение меньше или равно текущему) или в правом поддереве (если искомое значение больше).
    \item Создайте экземпляр класса AreaTree и вставьте в него 45 случайных чисел от 25 до 65.
    \item Выполните поиск элементов в дереве и выведите результаты на экран.
\end{enumerate}

Пример использования:
\begin{lstlisting}[language=Python]
at = AreaTree()
for i in range(45):
    at.insert_sector(random.randint(25, 65))

print("Поиск элементов:")
print(at.search_sector(38))  # Обнаружено, возвращен узел (38)
print(at.search_sector(78))  # Не обнаружено, возвращено None
print(at.search_sector(58))  # Обнаружено, возвращен узел (58)
\end{lstlisting}

\begin{figure}[h]
\centering
\begin{tikzpicture}[level distance=1.5cm,
  level 1/.style={sibling distance=3cm},
  level 2/.style={sibling distance=1.5cm}]
  \node {48}
    child {node {38}
      child {node {33}}
      child {node {43}}
    }
    child {node {58}
      child {node {53}}
      child {node {63}}
    };
\end{tikzpicture}
\caption{Пример бинарного дерева поиска}
\end{figure}

\item Написать программу на Python, которая реализует бинарное дерево поиска с инкапсуляцией. Программа должна создавать экземпляры класса SectorNode, которые представляют узлы дерева, и класса DistrictStructure, который представляет дерево поиска. Класс DistrictStructure должен содержать методы для вставки, поиска и удаления элементов, при этом все вспомогательные методы должны быть приватными. Программа также должна создавать дерево поиска, вставлять в него случайные числа и выполнять поиск элементов в дереве.

Инструкции:
\begin{enumerate}
    \item Создайте класс SectorNode с методом \_\_init\_\_, который принимает параметр sector\_val и сохраняет его в атрибуте self.structure\_district. Атрибуты self.left\_block и self.right\_block должны быть инициализированы как None.
    \item Создайте класс DistrictStructure с методом \_\_init\_\_, который инициализирует атрибут self.top\_district как None.
    \item Создайте публичный метод add\_block в классе DistrictStructure, который вставляет значение в дерево. Если self.top\_district отсутствует, создайте новый узел с вставляемым значением. В противном случае, вызовите приватный метод \_add\_block\_rec, передав ему self.top\_district и значение.
    \item Создайте приватный метод \_add\_block\_rec в классе DistrictStructure, который рекурсивно вставляет значение в дерево. Если значение строго меньше значения текущего узла, вставьте его в левое поддерево. Если значение больше или равно значению текущего узла, вставьте его в правое поддерево.
    \item Создайте публичный метод locate\_block в классе DistrictStructure, который ищет значение в дереве. Если дерево пустое, верните None. В противном случае, вызовите приватный метод \_locate\_block\_helper, передав ему self.top\_district и искомое значение.
    \item Создайте приватный метод \_locate\_block\_helper в классе DistrictStructure, который рекурсивно ищет значение в дереве. Если текущий узел равен None или значение текущего узла равно искомому значению, верните текущий узел. В противном случае, рекурсивно вызывайте метод \_locate\_block\_helper для поиска значения в левом поддереве (если искомое значение меньше текущего) или в правом поддереве (если искомое значение больше или равно текущему).
    \item Создайте экземпляр класса DistrictStructure и вставьте в него 46 случайных чисел от 26 до 66.
    \item Выполните поиск элементов в дереве и выведите результаты на экран.
\end{enumerate}

Пример использования:
\begin{lstlisting}[language=Python]
ds = DistrictStructure()
for i in range(46):
    ds.add_block(random.randint(26, 66))

print("Поиск элементов:")
print(ds.locate_block(39))  # Обнаружено, возвращен узел (39)
print(ds.locate_block(79))  # Не обнаружено, возвращено None
print(ds.locate_block(59))  # Обнаружено, возвращен узел (59)
\end{lstlisting}

\begin{figure}[h]
\centering
\begin{tikzpicture}[level distance=1.5cm,
  level 1/.style={sibling distance=3cm},
  level 2/.style={sibling distance=1.5cm}]
  \node {49}
    child {node {39}
      child {node {34}}
      child {node {44}}
    }
    child {node {59}
      child {node {54}}
      child {node {64}}
    };
\end{tikzpicture}
\caption{Пример бинарного дерева поиска}
\end{figure}

\item Написать программу на Python, которая реализует бинарное дерево поиска с инкапсуляцией. Программа должна создавать экземпляры класса BlockNode, которые представляют узлы дерева, и класса SectorIndex, который представляет дерево поиска. Класс SectorIndex должен содержать методы для вставки, поиска и удаления элементов, при этом все рекурсивные методы должны быть приватными. Программа также должна создавать дерево поиска, вставлять в него случайные числа и выполнять поиск элементов в дереве.

Инструкции:
\begin{enumerate}
    \item Создайте класс BlockNode с методом \_\_init\_\_, который принимает параметр block\_value и сохраняет его в атрибуте self.index\_sector. Атрибуты self.left\_unit и self.right\_unit должны быть инициализированы как None.
    \item Создайте класс SectorIndex с методом \_\_init\_\_, который инициализирует атрибут self.root\_sector как None.
    \item Создайте публичный метод insert\_unit в классе SectorIndex, который вставляет значение в дерево. Если self.root\_sector отсутствует, создайте новый узел с вставляемым значением. В противном случае, вызовите приватный метод \_insert\_unit\_helper, передав ему self.root\_sector и значение.
    \item Создайте приватный метод \_insert\_unit\_helper в классе SectorIndex, который рекурсивно вставляет значение в дерево. Если значение меньше или равно значению текущего узла, вставьте его в левое поддерево. Если значение строго больше значения текущего узла, вставьте его в правое поддерево.
    \item Создайте публичный метод find\_unit в классе SectorIndex, который ищет значение в дереве. Если дерево пустое, верните None. В противном случае, вызовите приватный метод \_find\_unit\_rec, передав ему self.root\_sector и искомое значение.
    \item Создайте приватный метод \_find\_unit\_rec в классе SectorIndex, который рекурсивно ищет значение в дереве. Если текущий узел равен None или значение текущего узла равно искомому значению, верните текущий узел. В противном случае, рекурсивно вызывайте метод \_find\_unit\_rec для поиска значения в левом поддереве (если искомое значение меньше или равно текущему) или в правом поддереве (если искомое значение больше).
    \item Создайте экземпляр класса SectorIndex и вставьте в него 47 случайных чисел от 27 до 67.
    \item Выполните поиск элементов в дереве и выведите результаты на экран.
\end{enumerate}

Пример использования:
\begin{lstlisting}[language=Python]
si = SectorIndex()
for i in range(47):
    si.insert_unit(random.randint(27, 67))

print("Поиск элементов:")
print(si.find_unit(40))  # Обнаружено, возвращен узел (40)
print(si.find_unit(80))  # Не обнаружено, возвращено None
print(si.find_unit(60))  # Обнаружено, возвращен узел (60)
\end{lstlisting}

\begin{figure}[h]
\centering
\begin{tikzpicture}[level distance=1.5cm,
  level 1/.style={sibling distance=3cm},
  level 2/.style={sibling distance=1.5cm}]
  \node {50}
    child {node {40}
      child {node {35}}
      child {node {45}}
    }
    child {node {60}
      child {node {55}}
      child {node {65}}
    };
\end{tikzpicture}
\caption{Пример бинарного дерева поиска}
\end{figure}

\end{enumerate}

\subsubsection{Задача 2 (стек)}

\begin{enumerate}
    \item Написать программу на Python, которая создает класс Stack для представления стека с инкапсуляцией внутреннего состояния. Класс должен содержать методы push, pop, is\_empty, size и peek, которые реализуют операции вталкивания, выталкивания, проверки пустоты, получения размера и просмотра вершины стека соответственно. Программа также должна создавать экземпляр класса Stack, вталкивать в него элементы, выталкивать элементы и выводить информацию о стеке на экран.

Инструкции:
\begin{enumerate}
    \item Создайте класс Stack с методом \_\_init\_\_, который принимает необязательный аргумент initial\_element. Если он передан, стек инициализируется с этим элементом (в виде списка из одного элемента), иначе — пустым списком.
    \item Создайте метод push, который принимает элемент в качестве аргумента и вталкивает его в стек только в том случае, если он не равен текущему верхнему элементу (если стек не пуст). Если стек пуст, элемент добавляется без проверки.
    \item Создайте метод pop, который выталкивает верхний элемент из стека и возвращает его. Если стек пуст, метод должен вернуть None и вывести сообщение "Стек пуст — извлечение невозможно" в стандартный поток ошибок (sys.stderr).
    \item Создайте метод is\_empty, который возвращает True, если стек пуст, и False в противном случае.
    \item Создайте метод size, который возвращает текущее количество элементов в стеке.
    \item Создайте метод peek, который возвращает верхний элемент стека, если стек не пуст. Если стек пуст, возвращает None и выводит сообщение "Стек пуст — просмотр невозможен" в sys.stderr.
    \item Создайте экземпляр класса Stack, передав в конструктор начальный элемент 10.
    \item Последовательно вызовите метод push с аргументами: 10, 20, 20, 30, 40 (обратите внимание, что повторяющийся элемент 20 не должен быть добавлен дважды подряд).
    \item Выведите размер стека и верхний элемент.
    \item Вызовите метод pop дважды, каждый раз выводя вытолкнутый элемент.
    \item После каждого pop выводите текущий размер стека и результат вызова peek.
\end{enumerate}

Пример использования:
\begin{lstlisting}[language=Python]
import sys

stack = Stack(10)
stack.push(10)   # не добавится, т.к. равен верхнему
stack.push(20)   # добавится
stack.push(20)   # не добавится, т.к. равен верхнему
stack.push(30)
stack.push(40)

print("Размер стека:", stack.size())
print("Верхний элемент:", stack.peek())

popped = stack.pop()
print("Вытолкнут:", popped)
print("Размер после pop:", stack.size())
print("Верхний элемент:", stack.peek())

popped = stack.pop()
print("Вытолкнут:", popped)
print("Размер после pop:", stack.size())
print("Верхний элемент:", stack.peek())
\end{lstlisting}

\item Написать программу на Python, которая создает класс Stack для представления стека с инкапсуляцией. Класс должен содержать методы push, pop, is\_empty, size и peek, которые реализуют операции вталкивания, выталкивания, проверки пустоты, получения размера и просмотра вершины стека соответственно. Программа также должна создавать экземпляр класса Stack, вталкивать в него элементы, выталкивать элементы и выводить информацию о стеке на экран.

Инструкции:
\begin{enumerate}
    \item Создайте класс Stack с методом \_\_init\_\_, который инициализирует пустой стек. Дополнительно принимает необязательный параметр max\_size, ограничивающий максимальное количество элементов в стеке (по умолчанию — None, то есть без ограничений).
    \item Создайте метод push, который принимает два аргумента: element и force=False. Элемент добавляется в стек, только если не превышает max\_size. Если force=True, то элемент добавляется даже при превышении лимита (с заменой самого нижнего элемента, если стек полон).
    \item Создайте метод pop, который выталкивает верхний элемент из стека и возвращает его. Если стек пуст, возвращает строку "Стек пуст".
    \item Создайте метод is\_empty, который возвращает True, если стек пуст, и False в противном случае.
    \item Создайте метод size, который возвращает текущее количество элементов в стеке.
    \item Создайте метод peek, который возвращает верхний элемент стека, если стек не пуст. Если стек пуст, возвращает строку "Нет элементов для просмотра".
    \item Создайте экземпляр класса Stack с max\_size=3.
    \item Последовательно вызовите push с элементами 5, 15, 25 (все добавятся).
    \item Попытайтесь добавить 35 без force — не должно добавиться.
    \item Добавьте 35 с force=True — должен замениться нижний элемент (5), стек станет [15, 25, 35].
    \item Выведите размер стека и верхний элемент.
    \item Вызовите pop и выведите результат.
    \item Повторите вывод размера и верхнего элемента.
\end{enumerate}

Пример использования:
\begin{lstlisting}[language=Python]
stack = Stack(max_size=3)
stack.push(5)
stack.push(15)
stack.push(25)
stack.push(35)          # не добавится
stack.push(35, force=True)  # добавится с заменой нижнего

print("Размер стека:", stack.size())
print("Верхний элемент:", stack.peek())

popped = stack.pop()
print("Вытолкнут:", popped)
print("Размер после pop:", stack.size())
print("Верхний элемент:", stack.peek())
\end{lstlisting}

\item Написать программу на Python, которая создает класс Stack для представления стека с инкапсуляцией. Класс должен содержать методы push, pop, is\_empty, size и peek, которые реализуют операции вталкивания, выталкивания, проверки пустоты, получения размера и просмотра вершины стека соответственно. Программа также должна создавать экземпляр класса Stack, вталкивать в него элементы, выталкивать элементы и выводить информацию о стеке на экран.

Инструкции:
\begin{enumerate}
    \item Создайте класс Stack с методом \_\_init\_\_, который инициализирует пустой стек. Может принимать список элементов в качестве аргумента items, который будет использован для первоначального заполнения стека (в порядке, как в списке: первый элемент — внизу стека).
    \item Создайте метод push, который принимает один элемент и добавляет его в стек. Если добавляемый элемент отрицательный, он не добавляется, а в sys.stderr выводится предупреждение "Отрицательные значения не допускаются".
    \item Создайте метод pop, который выталкивает верхний элемент из стека и возвращает его. Если стек пуст, выбрасывает исключение IndexError с сообщением "pop from empty stack".
    \item Создайте метод is\_empty, который возвращает True, если стек пуст, и False в противном случае.
    \item Создайте метод size, который возвращает текущее количество элементов в стеке.
    \item Создайте метод peek, который возвращает верхний элемент стека, если стек не пуст. Если стек пуст, выбрасывает исключение IndexError с сообщением "peek from empty stack".
    \item Создайте экземпляр класса Stack, передав в конструктор список [1, 2, 3].
    \item Добавьте элементы 4, -5 (не добавится), 6.
    \item Выведите размер стека и результат peek.
    \item Вызовите pop трижды, каждый раз выводя результат.
    \item После каждого pop проверяйте is\_empty и выводите результат.
\end{enumerate}

Пример использования:
\begin{lstlisting}[language=Python]
import sys

stack = Stack([1, 2, 3])
stack.push(4)
stack.push(-5)  # не добавится, выведет предупреждение
stack.push(6)

print("Размер стека:", stack.size())
print("Верхний элемент:", stack.peek())

for _ in range(3):
    popped = stack.pop()
    print("Вытолкнут:", popped)
    print("Стек пуст?", stack.is_empty())
\end{lstlisting}

\item Написать программу на Python, которая создает класс Stack для представления стека с инкапсуляцией. Класс должен содержать методы push, pop, is\_empty, size и peek, которые реализуют операции вталкивания, выталкивания, проверки пустоты, получения размера и просмотра вершины стека соответственно. Программа также должна создавать экземпляр класса Stack, вталкивать в него элементы, выталкивать элементы и выводить информацию о стеке на экран.

Инструкции:
\begin{enumerate}
    \item Создайте класс Stack с методом \_\_init\_\_, который инициализирует пустой стек. Принимает необязательный аргумент allow\_duplicates (по умолчанию True). Если False, то дубликаты (элементы, уже присутствующие в стеке) не добавляются.
    \item Создайте метод push, который принимает элемент и добавляет его в стек, только если allow\_duplicates=True или если такого элемента еще нет в стеке. Возвращает True, если элемент добавлен, и False — если не добавлен.
    \item Создайте метод pop, который выталкивает верхний элемент из стека и возвращает его. Если стек пуст, возвращает None.
    \item Создайте метод is\_empty, который возвращает True, если стек пуст, и False в противном случае.
    \item Создайте метод size, который возвращает текущее количество элементов в стеке.
    \item Создайте метод peek, который возвращает верхний элемент стека, если стек не пуст. Если стек пуст, возвращает None.
    \item Создайте экземпляр класса Stack с allow\_duplicates=False.
    \item Добавьте элементы 10, 20, 10 (второй 10 не добавится), 30.
    \item Выведите размер стека и верхний элемент.
    \item Вызовите pop, выведите результат.
    \item Повторите вывод размера и верхнего элемента.
\end{enumerate}

Пример использования:
\begin{lstlisting}[language=Python]
stack = Stack(allow_duplicates=False)
print(stack.push(10))  # True
print(stack.push(20))  # True
print(stack.push(10))  # False (дубликат)
print(stack.push(30))  # True

print("Размер стека:", stack.size())
print("Верхний элемент:", stack.peek())

popped = stack.pop()
print("Вытолкнут:", popped)
print("Размер после pop:", stack.size())
print("Верхний элемент:", stack.peek())
\end{lstlisting}

\item Написать программу на Python, которая создает класс Stack для представления стека с инкапсуляцией. Класс должен содержать методы push, pop, is\_empty, size и peek, которые реализуют операции вталкивания, выталкивания, проверки пустоты, получения размера и просмотра вершины стека соответственно. Программа также должна создавать экземпляр класса Stack, вталкивать в него элементы, выталкивать элементы и выводить информацию о стеке на экран.

Инструкции:
\begin{enumerate}
    \item Создайте класс Stack с методом \_\_init\_\_, который инициализирует пустой стек. Может принимать параметр name (строка) для именования стека (используется только для отладки, не влияет на логику).
    \item Создайте метод push, который принимает элемент и добавляет его в стек. Если элемент не является числом (int или float), он не добавляется, а в sys.stderr выводится сообщение "Только числовые значения разрешены".
    \item Создайте метод pop, который выталкивает верхний элемент из стека и возвращает его. Если стек пуст, возвращает None.
    \item Создайте метод is\_empty, который возвращает True, если стек пуст, и False в противном случае.
    \item Создайте метод size, который возвращает текущее количество элементов в стеке.
    \item Создайте метод peek, который возвращает верхний элемент стека, если стек не пуст. Если стек пуст, возвращает None.
    \item Создайте экземпляр класса Stack с именем "NumericStack".
    \item Добавьте элементы: 3.14, 42, "hello" (не добавится), 100, [1,2] (не добавится).
    \item Выведите размер стека и верхний элемент.
    \item Вызовите pop дважды, выводя каждый раз результат.
    \item После каждого pop выводите размер стека.
\end{enumerate}

Пример использования:
\begin{lstlisting}[language=Python]
import sys

stack = Stack(name="NumericStack")
stack.push(3.14)
stack.push(42)
stack.push("hello")   # не добавится
stack.push(100)
stack.push([1,2])     # не добавится

print("Размер стека:", stack.size())
print("Верхний элемент:", stack.peek())

popped = stack.pop()
print("Вытолкнут:", popped)
print("Размер после pop:", stack.size())

popped = stack.pop()
print("Вытолкнут:", popped)
print("Размер после pop:", stack.size())
\end{lstlisting}

\item Написать программу на Python, которая создает класс Stack для представления стека с инкапсуляцией. Класс должен содержать методы push, pop, is\_empty, size и peek, которые реализуют операции вталкивания, выталкивания, проверки пустоты, получения размера и просмотра вершины стека соответственно. Программа также должна создавать экземпляр класса Stack, вталкивать в него элементы, выталкивать элементы и выводить информацию о стеке на экран.

Инструкции:
\begin{enumerate}
    \item Создайте класс Stack с методом \_\_init\_\_, который инициализирует пустой стек. Принимает необязательный параметр auto\_reverse=False. Если True, то при добавлении элемента он вставляется не наверх, а вниз стека (реализуя поведение, обратное обычному стеку).
    \item Создайте метод push, который принимает элемент и добавляет его: если auto\_reverse=False — наверх (как обычно), если True — вниз (в начало внутреннего списка).
    \item Создайте метод pop, который выталкивает верхний элемент из стека (последний добавленный, если auto\_reverse=False, или первый добавленный, если auto\_reverse=True) и возвращает его. Если стек пуст, возвращает "EMPTY".
    \item Создайте метод is\_empty, который возвращает True, если стек пуст, и False в противном случае.
    \item Создайте метод size, который возвращает текущее количество элементов в стеке.
    \item Создайте метод peek, который возвращает верхний элемент стека (последний в списке, если auto\_reverse=False, или первый, если auto\_reverse=True), если стек не пуст. Если стек пуст, возвращает "NO ELEMENT".
    \item Создайте экземпляр класса Stack с auto\_reverse=True.
    \item Добавьте элементы: 1, 2, 3 (в стеке будет [3, 2, 1], где 3 — верх).
    \item Выведите размер стека и результат peek (должен быть 3).
    \item Вызовите pop, выведите результат (должен быть 3).
    \item Повторите вывод размера и peek (теперь верх — 2).
\end{enumerate}

Пример использования:
\begin{lstlisting}[language=Python]
stack = Stack(auto_reverse=True)
stack.push(1)
stack.push(2)
stack.push(3)  # стек: [3,2,1], верх - 3

print("Размер стека:", stack.size())
print("Верхний элемент:", stack.peek())

popped = stack.pop()
print("Вытолкнут:", popped)  # 3
print("Размер после pop:", stack.size())
print("Верхний элемент:", stack.peek())  # 2
\end{lstlisting}

\item Написать программу на Python, которая создает класс Stack для представления стека с инкапсуляцией. Класс должен содержать методы push, pop, is\_empty, size и peek, которые реализуют операции вталкивания, выталкивания, проверки пустоты, получения размера и просмотра вершины стека соответственно. Программа также должна создавать экземпляр класса Stack, вталкивать в него элементы, выталкивать элементы и выводить информацию о стеке на экран.

Инструкции:
\begin{enumerate}
    \item Создайте класс Stack с методом \_\_init\_\_, который инициализирует пустой стек. Принимает параметр case\_sensitive=True. Используется только если элементы — строки.
    \item Создайте метод push, который принимает элемент. Если элемент — строка и case\_sensitive=False, то перед добавлением преобразует её в нижний регистр. Добавляет элемент в стек.
    \item Создайте метод pop, который выталкивает верхний элемент из стека и возвращает его. Если стек пуст, возвращает пустую строку "".
    \item Создайте метод is\_empty, который возвращает True, если стек пуст, и False в противном случае.
    \item Создайте метод size, который возвращает текущее количество элементов в стеке.
    \item Создайте метод peek, который возвращает верхний элемент стека, если стек не пуст. Если стек пуст, возвращает пустую строку "".
    \item Создайте экземпляр класса Stack с case\_sensitive=False.
    \item Добавьте строки: "Hello", "WORLD", "Python".
    \item Выведите размер стека и верхний элемент (должен быть "python").
    \item Вызовите pop, выведите результат.
    \item Повторите вывод размера и верхнего элемента.
\end{enumerate}

Пример использования:
\begin{lstlisting}[language=Python]
stack = Stack(case_sensitive=False)
stack.push("Hello")
stack.push("WORLD")
stack.push("Python")

print("Размер стека:", stack.size())
print("Верхний элемент:", stack.peek())  # "python"

popped = stack.pop()
print("Вытолкнут:", popped)  # "python"
print("Размер после pop:", stack.size())
print("Верхний элемент:", stack.peek())  # "world"
\end{lstlisting}

\item Написать программу на Python, которая создает класс Stack для представления стека с инкапсуляцией. Класс должен содержать методы push, pop, is\_empty, size и peek, которые реализуют операции вталкивания, выталкивания, проверки пустоты, получения размера и просмотра вершины стека соответственно. Программа также должна создавать экземпляр класса Stack, вталкивать в него элементы, выталкивать элементы и выводить информацию о стеке на экран.

Инструкции:
\begin{enumerate}
    \item Создайте класс Stack с методом \_\_init\_\_, который инициализирует пустой стек. Принимает параметр min\_value=None. Если задан, то при добавлении элемента проверяется, что он >= min\_value.
    \item Создайте метод push, который принимает элемент. Если min\_value задан и элемент < min\_value, элемент не добавляется, а метод возвращает False. Иначе — добавляет и возвращает True.
    \item Создайте метод pop, который выталкивает верхний элемент из стека и возвращает его. Если стек пуст, возвращает None.
    \item Создайте метод is\_empty, который возвращает True, если стек пуст, и False в противном случае.
    \item Создайте метод size, который возвращает текущее количество элементов в стеке.
    \item Создайте метод peek, который возвращает верхний элемент стека, если стек не пуст. Если стек пуст, возвращает None.
    \item Создайте экземпляр класса Stack с min\_value=10.
    \item Добавьте элементы: 5 (не добавится), 15, 20, 8 (не добавится), 25.
    \item Выведите размер стека и верхний элемент.
    \item Вызовите pop, выведите результат.
    \item Повторите вывод размера и верхнего элемента.
\end{enumerate}

Пример использования:
\begin{lstlisting}[language=Python]
stack = Stack(min_value=10)
print(stack.push(5))   # False
print(stack.push(15))  # True
print(stack.push(20))  # True
print(stack.push(8))   # False
print(stack.push(25))  # True

print("Размер стека:", stack.size())
print("Верхний элемент:", stack.peek())

popped = stack.pop()
print("Вытолкнут:", popped)  # 25
print("Размер после pop:", stack.size())
print("Верхний элемент:", stack.peek())  # 20
\end{lstlisting}

\item Написать программу на Python, которая создает класс Stack для представления стека с инкапсуляцией. Класс должен содержать методы push, pop, is\_empty, size и peek, которые реализуют операции вталкивания, выталкивания, проверки пустоты, получения размера и просмотра вершины стека соответственно. Программа также должна создавать экземпляр класса Stack, вталкивать в него элементы, выталкивать элементы и выводить информацию о стеке на экран.

Инструкции:
\begin{enumerate}
    \item Создайте класс Stack с методом \_\_init\_\_, который инициализирует пустой стек. Принимает параметр max\_increments=0 — максимальное количество добавлений. Если 0 — без ограничений.
    \item Создайте метод push, который принимает элемент. Если max\_increments > 0 и количество вызовов push превысило max\_increments, элемент не добавляется, метод возвращает False. Иначе — добавляет и возвращает True.
    \item Создайте метод pop, который выталкивает верхний элемент из стека и возвращает его. Если стек пуст, возвращает строку "---".
    \item Создайте метод is\_empty, который возвращает True, если стек пуст, и False в противном случае.
    \item Создайте метод size, который возвращает текущее количество элементов в стеке.
    \item Создайте метод peek, который возвращает верхний элемент стека, если стек не пуст. Если стек пуст, возвращает строку "---".
    \item Создайте экземпляр класса Stack с max\_increments=3.
    \item Добавьте элементы: 100, 200, 300, 400 (последний не добавится).
    \item Выведите размер стека и верхний элемент.
    \item Вызовите pop, выведите результат.
    \item Повторите вывод размера и верхнего элемента.
\end{enumerate}

Пример использования:
\begin{lstlisting}[language=Python]
stack = Stack(max_increments=3)
print(stack.push(100))  # True
print(stack.push(200))  # True
print(stack.push(300))  # True
print(stack.push(400))  # False

print("Размер стека:", stack.size())
print("Верхний элемент:", stack.peek())

popped = stack.pop()
print("Вытолкнут:", popped)  # 300
print("Размер после pop:", stack.size())
print("Верхний элемент:", stack.peek())  # 200
\end{lstlisting}

\item Написать программу на Python, которая создает класс Stack для представления стека с инкапсуляцией. Класс должен содержать методы push, pop, is\_empty, size и peek, которые реализуют операции вталкивания, выталкивания, проверки пустоты, получения размера и просмотра вершины стека соответственно. Программа также должна создавать экземпляр класса Stack, вталкивать в него элементы, выталкивать элементы и выводить информацию о стеке на экран.

Инструкции:
\begin{enumerate}
    \item Создайте класс Stack с методом \_\_init\_\_, который инициализирует пустой стек. Принимает параметр validate\_type=None. Если задан (например, int), то при добавлении проверяется, что элемент является экземпляром этого типа.
    \item Создайте метод push, который принимает элемент. Если validate\_type задан и элемент не является его экземпляром, элемент не добавляется, метод возвращает False. Иначе — добавляет и возвращает True.
    \item Создайте метод pop, который выталкивает верхний элемент из стека и возвращает его. Если стек пуст, возвращает None.
    \item Создайте метод is\_empty, который возвращает True, если стек пуст, и False в противном случае.
    \item Создайте метод size, который возвращает текущее количество элементов в стеке.
    \item Создайте метод peek, который возвращает верхний элемент стека, если стек не пуст. Если стек пуст, возвращает None.
    \item Создайте экземпляр класса Stack с validate\_type=int.
    \item Добавьте элементы: 10, "20" (не добавится), 30, 40.5 (не добавится), 50.
    \item Выведите размер стека и верхний элемент.
    \item Вызовите pop, выведите результат.
    \item Повторите вывод размера и верхнего элемента.
\end{enumerate}

Пример использования:
\begin{lstlisting}[language=Python]
stack = Stack(validate_type=int)
print(stack.push(10))    # True
print(stack.push("20"))  # False
print(stack.push(30))    # True
print(stack.push(40.5))  # False
print(stack.push(50))    # True

print("Размер стека:", stack.size())
print("Верхний элемент:", stack.peek())

popped = stack.pop()
print("Вытолкнут:", popped)  # 50
print("Размер после pop:", stack.size())
print("Верхний элемент:", stack.peek())  # 30
\end{lstlisting}

\item Написать программу на Python, которая создает класс Stack для представления стека с инкапсуляцией. Класс должен содержать методы push, pop, is\_empty, size и peek, которые реализуют операции вталкивания, выталкивания, проверки пустоты, получения размера и просмотра вершины стека соответственно. Программа также должна создавать экземпляр класса Stack, вталкивать в него элементы, выталкивать элементы и выводить информацию о стеке на экран.

Инструкции:
\begin{enumerate}
    \item Создайте класс Stack с методом \_\_init\_\_, который инициализирует пустой стек. Принимает параметр unique\_per\_session=False. Если True, то не позволяет добавлять один и тот же элемент дважды за всё время жизни стека (даже если он был удален).
    \item Создайте метод push, который принимает элемент. Если unique\_per\_session=True и элемент уже когда-либо был добавлен (даже если потом удален), он не добавляется, метод возвращает False. Иначе — добавляет и возвращает True.
    \item Создайте метод pop, который выталкивает верхний элемент из стека и возвращает его. Если стек пуст, возвращает None.
    \item Создайте метод is\_empty, который возвращает True, если стек пуст, и False в противном случае.
    \item Создайте метод size, который возвращает текущее количество элементов в стеке.
    \item Создайте метод peek, который возвращает верхний элемент стека, если стек не пуст. Если стек пуст, возвращает None.
    \item Создайте экземпляр класса Stack с unique\_per\_session=True.
    \item Добавьте элементы: 7, 14, 7 (не добавится), 21, 14 (не добавится).
    \item Выведите размер стека и верхний элемент.
    \item Вызовите pop, выведите результат.
    \item Попробуйте добавить 21 снова (не должно добавиться).
    \item Выведите размер стека.
\end{enumerate}

Пример использования:
\begin{lstlisting}[language=Python]
stack = Stack(unique_per_session=True)
print(stack.push(7))   # True
print(stack.push(14))  # True
print(stack.push(7))   # False
print(stack.push(21))  # True
print(stack.push(14))  # False

print("Размер стека:", stack.size())
print("Верхний элемент:", stack.peek())

popped = stack.pop()
print("Вытолкнут:", popped)  # 21

print(stack.push(21))  # False (уже был)
print("Размер стека:", stack.size())  # по-прежнему 2
\end{lstlisting}

\item Написать программу на Python, которая создает класс Stack для представления стека с инкапсуляцией. Класс должен содержать методы push, pop, is\_empty, size и peek, которые реализуют операции вталкивания, выталкивания, проверки пустоты, получения размера и просмотра вершины стека соответственно. Программа также должна создавать экземпляр класса Stack, вталкивать в него элементы, выталкивать элементы и выводить информацию о стеке на экран.

Инструкции:
\begin{enumerate}
    \item Создайте класс Stack с методом \_\_init\_\_, который инициализирует пустой стек. Принимает параметр push\_limit\_per\_call=1 (по умолчанию). Если >1, то метод push может принимать несколько элементов (через *args) и добавлять их все за один вызов (но не более push\_limit\_per\_call элементов за вызов).
    \item Создайте метод push, который принимает один или несколько элементов (если push\_limit\_per\_call > 1). Если передано больше элементов, чем push\_limit\_per\_call, добавляются только первые push\_limit\_per\_call элементов, остальные игнорируются. Возвращает количество реально добавленных элементов.
    \item Создайте метод pop, который выталкивает верхний элемент из стека и возвращает его. Если стек пуст, возвращает None.
    \item Создайте метод is\_empty, который возвращает True, если стек пуст, и False в противном случае.
    \item Создайте метод size, который возвращает текущее количество элементов в стеке.
    \item Создайте метод peek, который возвращает верхний элемент стека, если стек не пуст. Если стек пуст, возвращает None.
    \item Создайте экземпляр класса Stack с push\_limit\_per\_call=3.
    \item Вызовите push с элементами 1, 2, 3, 4, 5 — добавятся только 1,2,3.
    \item Вызовите push с элементами 6, 7 — добавятся оба.
    \item Выведите размер стека и верхний элемент.
    \item Вызовите pop, выведите результат.
    \item Повторите вывод размера и верхнего элемента.
\end{enumerate}

Пример использования:
\begin{lstlisting}[language=Python]
stack = Stack(push_limit_per_call=3)
added = stack.push(1, 2, 3, 4, 5)  # добавит 1,2,3; вернет 3
print("Добавлено:", added)

added = stack.push(6, 7)  # добавит 6,7; вернет 2
print("Добавлено:", added)

print("Размер стека:", stack.size())
print("Верхний элемент:", stack.peek())

popped = stack.pop()
print("Вытолкнут:", popped)  # 7
print("Размер после pop:", stack.size())
print("Верхний элемент:", stack.peek())  # 6
\end{lstlisting}

\item Написать программу на Python, которая создает класс Stack для представления стека с инкапсуляцией. Класс должен содержать методы push, pop, is\_empty, size и peek, которые реализуют операции вталкивания, выталкивания, проверки пустоты, получения размера и просмотра вершины стека соответственно. Программа также должна создавать экземпляр класса Stack, вталкивать в него элементы, выталкивать элементы и выводить информацию о стеке на экран.

Инструкции:
\begin{enumerate}
    \item Создайте класс Stack с методом \_\_init\_\_, который инициализирует пустой стек. Принимает параметр pop\_multiple=False. Если True, то метод pop может принимать необязательный аргумент count (по умолчанию 1) и возвращать список из count верхних элементов.
    \item Создайте метод push, который принимает один элемент и добавляет его в стек. Возвращает None.
    \item Создайте метод pop, который, если pop\_multiple=False, выталкивает один верхний элемент и возвращает его. Если pop\_multiple=True, принимает count (по умолчанию 1) и возвращает список из count верхних элементов (если запрошено больше, чем есть, возвращает все). Если стек пуст, возвращает пустой список [] (в режиме pop\_multiple) или None (в обычном режиме).
    \item Создайте метод is\_empty, который возвращает True, если стек пуст, и False в противном случае.
    \item Создайте метод size, который возвращает текущее количество элементов в стеке.
    \item Создайте метод peek, который возвращает верхний элемент стека, если стек не пуст. Если стек пуст, возвращает None. Не поддерживает множественный просмотр.
    \item Создайте экземпляр класса Stack с pop\_multiple=True.
    \item Добавьте элементы: 10, 20, 30, 40, 50.
    \item Выведите размер стека и верхний элемент.
    \item Вызовите pop с count=3, выведите результат (должен быть [50,40,30]).
    \item Выведите размер стека и верхний элемент (теперь 20).
\end{enumerate}

Пример использования:
\begin{lstlisting}[language=Python]
stack = Stack(pop_multiple=True)
stack.push(10)
stack.push(20)
stack.push(30)
stack.push(40)
stack.push(50)

print("Размер стека:", stack.size())
print("Верхний элемент:", stack.peek())

popped = stack.pop(count=3)
print("Вытолкнуты:", popped)  # [50, 40, 30]

print("Размер после pop:", stack.size())
print("Верхний элемент:", stack.peek())  # 20
\end{lstlisting}

\item Написать программу на Python, которая создает класс Stack для представления стека с инкапсуляцией. Класс должен содержать методы push, pop, is\_empty, size и peek, которые реализуют операции вталкивания, выталкивания, проверки пустоты, получения размера и просмотра вершины стека соответственно. Программа также должна создавать экземпляр класса Stack, вталкивать в него элементы, выталкивать элементы и выводить информацию о стеке на экран.

Инструкции:
\begin{enumerate}
    \item Создайте класс Stack с методом \_\_init\_\_, который инициализирует пустой стек. Принимает параметр on\_push\_callback=None — функция, которая будет вызываться после каждого успешного добавления элемента (с аргументом — добавленным элементом).
    \item Создайте метод push, который принимает элемент и добавляет его в стек. Если on\_push\_callback не None, вызывает её с добавленным элементом. Возвращает добавленный элемент.
    \item Создайте метод pop, который выталкивает верхний элемент из стека и возвращает его. Если стек пуст, возвращает None.
    \item Создайте метод is\_empty, который возвращает True, если стек пуст, и False в противном случае.
    \item Создайте метод size, который возвращает текущее количество элементов в стеке.
    \item Создайте метод peek, который возвращает верхний элемент стека, если стек не пуст. Если стек пуст, возвращает None.
    \item Создайте функцию logger(x): print(f"[LOG] Добавлен: {x}")
    \item Создайте экземпляр класса Stack, передав logger в on\_push\_callback.
    \item Добавьте элементы: 101, 202, 303 (при каждом добавлении должно выводиться сообщение).
    \item Выведите размер стека и верхний элемент.
    \item Вызовите pop, выведите результат.
    \item Повторите вывод размера и верхнего элемента.
\end{enumerate}

Пример использования:
\begin{lstlisting}[language=Python]
def logger(x):
    print(f"[LOG] Добавлен: {x}")

stack = Stack(on_push_callback=logger)
stack.push(101)  # выведет [LOG] Добавлен: 101
stack.push(202)  # выведет [LOG] Добавлен: 202
stack.push(303)  # выведет [LOG] Добавлен: 303

print("Размер стека:", stack.size())
print("Верхний элемент:", stack.peek())

popped = stack.pop()
print("Вытолкнут:", popped)  # 303
print("Размер после pop:", stack.size())
print("Верхний элемент:", stack.peek())  # 202
\end{lstlisting}

\item Написать программу на Python, которая создает класс Stack для представления стека с инкапсуляцией. Класс должен содержать методы push, pop, is\_empty, size и peek, которые реализуют операции вталкивания, выталкивания, проверки пустоты, получения размера и просмотра вершины стека соответственно. Программа также должна создавать экземпляр класса Stack, вталкивать в него элементы, выталкивать элементы и выводить информацию о стеке на экран.

Инструкции:
\begin{enumerate}
    \item Создайте класс Stack с методом \_\_init\_\_, который инициализирует пустой стек. Принимает параметр compress\_on\_push=False. Если True, то при добавлении элемента, равного текущему верхнему, вместо добавления нового элемента увеличивается счетчик дубликатов у верхнего элемента (стек хранит пары (элемент, счетчик)).
    \item Создайте метод push, который принимает элемент. Если compress\_on\_push=True и элемент равен текущему верхнему, увеличивает счетчик верхнего элемента. Иначе — добавляет новый элемент (со счетчиком 1, если режим сжатия включен).
    \item Создайте метод pop, который выталкивает верхний элемент. Если режим сжатия включен и счетчик >1, уменьшает счетчик и возвращает элемент. Если счетчик=1, удаляет элемент. Если стек пуст, возвращает None.
    \item Создайте метод is\_empty, который возвращает True, если стек пуст, и False в противном случае.
    \item Создайте метод size, который возвращает общее количество элементов (с учетом счетчиков, если режим сжатия включен).
    \item Создайте метод peek, который возвращает верхний элемент (не счетчик, а само значение), если стек не пуст. Если стек пуст, возвращает None.
    \item Создайте экземпляр класса Stack с compress\_on\_push=True.
    \item Добавьте элементы: 5, 5, 5, 10, 10, 15.
    \item Выведите размер стека (должен быть 6) и верхний элемент (15).
    \item Вызовите pop, выведите результат (15).
    \item Вызовите pop, выведите результат (10) — счетчик у 10 должен уменьшиться с 2 до 1.
    \item Выведите размер стека (должен быть 4).
\end{enumerate}

Пример использования:
\begin{lstlisting}[language=Python]
stack = Stack(compress_on_push=True)
stack.push(5)
stack.push(5)
stack.push(5)
stack.push(10)
stack.push(10)
stack.push(15)

print("Размер стека:", stack.size())     # 6
print("Верхний элемент:", stack.peek())   # 15

popped = stack.pop()
print("Вытолкнут:", popped)  # 15

popped = stack.pop()
print("Вытолкнут:", popped)  # 10

print("Размер после двух pop:", stack.size())  # 4
\end{lstlisting}

\item Написать программу на Python, которая создает класс Stack для представления стека с инкапсуляцией. Класс должен содержать методы push, pop, is\_empty, size и peek, которые реализуют операции вталкивания, выталкивания, проверки пустоты, получения размера и просмотра вершины стека соответственно. Программа также должна создавать экземпляр класса Stack, вталкивать в него элементы, выталкивать элементы и выводить информацию о стеке на экран.

Инструкции:
\begin{enumerate}
    \item Создайте класс Stack с методом \_\_init\_\_, который инициализирует пустой стек. Принимает параметр immutable\_pop=False. Если True, то метод pop не удаляет элемент из стека, а только возвращает его (поведение как peek, но называется pop).
    \item Создайте метод push, который принимает элемент и добавляет его в стек.
    \item Создайте метод pop, который, если immutable\_pop=False, выталкивает верхний элемент и возвращает его. Если immutable\_pop=True, возвращает верхний элемент, не удаляя его. Если стек пуст, возвращает None.
    \item Создайте метод is\_empty, который возвращает True, если стек пуст, и False в противном случае.
    \item Создайте метод size, который возвращает текущее количество элементов в стеке.
    \item Создайте метод peek, который возвращает верхний элемент стека, если стек не пуст. Если стек пуст, возвращает None. (Поведение не зависит от immutable\_pop.)
    \item Создайте экземпляр класса Stack с immutable\_pop=True.
    \item Добавьте элементы: 1, 3, 5, 7.
    \item Выведите размер стека и результат pop (должен быть 7, но стек не изменится).
    \item Снова вызовите pop, снова выведите результат (опять 7).
    \item Выведите размер стека (по-прежнему 4).
\end{enumerate}

Пример использования:
\begin{lstlisting}[language=Python]
stack = Stack(immutable_pop=True)
stack.push(1)
stack.push(3)
stack.push(5)
stack.push(7)

print("Размер стека:", stack.size())
print("Первый pop:", stack.pop())  # 7
print("Второй pop:", stack.pop())  # 7 (стек не изменился)
print("Размер стека:", stack.size())  # 4
\end{lstlisting}

\item Написать программу на Python, которая создает класс Stack для представления стека с инкапсуляцией. Класс должен содержать методы push, pop, is\_empty, size и peek, которые реализуют операции вталкивания, выталкивания, проверки пустоты, получения размера и просмотра вершины стека соответственно. Программа также должна создавать экземпляр класса Stack, вталкивать в него элементы, выталкивать элементы и выводить информацию о стеке на экран.

Инструкции:
\begin{enumerate}
    \item Создайте класс Stack с методом \_\_init\_\_, который инициализирует пустой стек. Принимает параметр track\_history=False. Если True, то сохраняет историю всех когда-либо находившихся в стеке элементов (даже удаленных) в отдельном списке.
    \item Создайте метод push, который принимает элемент, добавляет его в стек, и если track\_history=True, добавляет его и в историю.
    \item Создайте метод pop, который выталкивает верхний элемент из стека и возвращает его. Если стек пуст, возвращает None.
    \item Создайте метод is\_empty, который возвращает True, если стек пуст, и False в противном случае.
    \item Создайте метод size, который возвращает текущее количество элементов в стеке.
    \item Создайте метод peek, который возвращает верхний элемент стека, если стек не пуст. Если стек пуст, возвращает None.
    \item Создайте метод get\_history (только если track\_history=True), который возвращает копию списка истории.
    \item Создайте экземпляр класса Stack с track\_history=True.
    \item Добавьте элементы: 2, 4, 6.
    \item Вызовите pop (вернет 6).
    \item Добавьте 8.
    \item Выведите текущий стек (через peek и size) и историю (должна быть [2,4,6,8]).
\end{enumerate}

Пример использования:
\begin{lstlisting}[language=Python]
stack = Stack(track_history=True)
stack.push(2)
stack.push(4)
stack.push(6)
stack.pop()  # 6
stack.push(8)

print("Текущий размер:", stack.size())      # 2
print("Верхний элемент:", stack.peek())     # 8
print("История:", stack.get_history())      # [2,4,6,8]
\end{lstlisting}

\item Написать программу на Python, которая создает класс Stack для представления стека с инкапсуляцией. Класс должен содержать методы push, pop, is\_empty, size и peek, которые реализуют операции вталкивания, выталкивания, проверки пустоты, получения размера и просмотра вершины стека соответственно. Программа также должна создавать экземпляр класса Stack, вталкивать в него элементы, выталкивать элементы и выводить информацию о стеке на экран.

Инструкции:
\begin{enumerate}
    \item Создайте класс Stack с методом \_\_init\_\_, который инициализирует пустой стек. Принимает параметр push\_only\_even=False. Если True, то добавляются только четные числа (остальные игнорируются).
    \item Создайте метод push, который принимает элемент. Если push\_only\_even=True и элемент не является четным целым числом, он не добавляется. Иначе — добавляется.
    \item Создайте метод pop, который выталкивает верхний элемент из стека и возвращает его. Если стек пуст, возвращает None.
    \item Создайте метод is\_empty, который возвращает True, если стек пуст, и False в противном случае.
    \item Создайте метод size, который возвращает текущее количество элементов в стеке.
    \item Создайте метод peek, который возвращает верхний элемент стека, если стек не пуст. Если стек пуст, возвращает None.
    \item Создайте экземпляр класса Stack с push\_only\_even=True.
    \item Добавьте элементы: 1 (игнорируется), 2, 3 (игнорируется), 4, 5 (игнорируется), 6.
    \item Выведите размер стека (должен быть 3) и верхний элемент (6).
    \item Вызовите pop, выведите результат (6).
    \item Повторите вывод размера и верхнего элемента (теперь 4).
\end{enumerate}

Пример использования:
\begin{lstlisting}[language=Python]
stack = Stack(push_only_even=True)
stack.push(1)  # игнорируется
stack.push(2)
stack.push(3)  # игнорируется
stack.push(4)
stack.push(5)  # игнорируется
stack.push(6)

print("Размер стека:", stack.size())     # 3
print("Верхний элемент:", stack.peek())   # 6

popped = stack.pop()
print("Вытолкнут:", popped)  # 6

print("Размер после pop:", stack.size())    # 2
print("Верхний элемент:", stack.peek())     # 4
\end{lstlisting}

\item Написать программу на Python, которая создает класс Stack для представления стека с инкапсуляцией. Класс должен содержать методы push, pop, is\_empty, size и peek, которые реализуют операции вталкивания, выталкивания, проверки пустоты, получения размера и просмотра вершины стека соответственно. Программа также должна создавать экземпляр класса Stack, вталкивать в него элементы, выталкивать элементы и выводить информацию о стеке на экран.

Инструкции:
\begin{enumerate}
    \item Создайте класс Stack с методом \_\_init\_\_, который инициализирует пустой стек. Принимает параметр reverse\_pop=False. Если True, то метод pop возвращает не верхний, а нижний элемент стека (и удаляет его).
    \item Создайте метод push, который принимает элемент и добавляет его в стек (наверх).
    \item Создайте метод pop, который, если reverse\_pop=False, выталкивает верхний элемент и возвращает его. Если reverse\_pop=True, выталкивает нижний элемент и возвращает его. Если стек пуст, возвращает None.
    \item Создайте метод is\_empty, который возвращает True, если стек пуст, и False в противном случае.
    \item Создайте метод size, который возвращает текущее количество элементов в стеке.
    \item Создайте метод peek, который возвращает верхний элемент стека, если стек не пуст. Если стек пуст, возвращает None. (Не зависит от reverse\_pop.)
    \item Создайте экземпляр класса Stack с reverse\_pop=True.
    \item Добавьте элементы: 10, 20, 30 (в стеке: [10,20,30], верх — 30).
    \item Выведите результат peek (должен быть 30).
    \item Вызовите pop — должен вернуться 10 (нижний), стек станет [20,30].
    \item Выведите размер и снова peek (должен быть 30).
\end{enumerate}

Пример использования:
\begin{lstlisting}[language=Python]
stack = Stack(reverse_pop=True)
stack.push(10)
stack.push(20)
stack.push(30)

print("Верхний элемент (peek):", stack.peek())  # 30
popped = stack.pop()
print("Вытолкнут (нижний):", popped)            # 10
print("Размер после pop:", stack.size())        # 2
print("Верхний элемент (peek):", stack.peek())  # 30
\end{lstlisting}

\item Написать программу на Python, которая создает класс Stack для представления стека с инкапсуляцией. Класс должен содержать методы push, pop, is\_empty, size и peek, которые реализуют операции вталкивания, выталкивания, проверки пустоты, получения размера и просмотра вершины стека соответственно. Программа также должна создавать экземпляр класса Stack, вталкивать в него элементы, выталкивать элементы и выводить информацию о стеке на экран.

Инструкции:
\begin{enumerate}
    \item Создайте класс Stack с методом \_\_init\_\_, который инициализирует пустой стек. Принимает параметр push\_with\_timestamp=False. Если True, то при добавлении элемент сохраняется вместе с текущим временем (в формате Unix timestamp).
    \item Создайте метод push, который принимает элемент. Если push\_with\_timestamp=True, сохраняет пару (элемент, time.time()). Иначе — только элемент.
    \item Создайте метод pop, который выталкивает верхний элемент. Если режим с временем включен, возвращает пару (элемент, timestamp). Иначе — только элемент. Если стек пуст, возвращает None.
    \item Создайте метод is\_empty, который возвращает True, если стек пуст, и False в противном случае.
    \item Создайте метод size, который возвращает текущее количество элементов в стеке.
    \item Создайте метод peek, который возвращает верхний элемент (или пару, если включен режим времени), если стек не пуст. Если стек пуст, возвращает None.
    \item Создайте экземпляр класса Stack с push\_with\_timestamp=True.
    \item Добавьте элементы: "first", "second", "third".
    \item Выведите размер стека и результат peek (должна быть пара ("third", timestamp)).
    \item Вызовите pop, выведите результат (тоже пара).
    \item Повторите вывод размера и peek.
\end{enumerate}

Пример использования:
\begin{lstlisting}[language=Python]
import time

stack = Stack(push_with_timestamp=True)
stack.push("first")
stack.push("second")
stack.push("third")

print("Размер стека:", stack.size())
peek_result = stack.peek()
print("Верхний элемент и время:", peek_result)  # ('third', 1712345678.123456)

popped = stack.pop()
print("Вытолкнут:", popped)  # ('third', 1712345678.123456)

print("Размер после pop:", stack.size())
print("Верхний элемент:", stack.peek())  # ('second', ...)
\end{lstlisting}

\item Написать программу на Python, которая создает класс Stack для представления стека с инкапсуляцией. Класс должен содержать методы push, pop, is\_empty, size и peek, которые реализуют операции вталкивания, выталкивания, проверки пустоты, получения размера и просмотра вершины стека соответственно. Программа также должна создавать экземпляр класса Stack, вталкивать в него элементы, выталкивать элементы и выводить информацию о стеке на экран.

Инструкции:
\begin{enumerate}
    \item Создайте класс Stack с методом \_\_init\_\_, который инициализирует пустой стек. Принимает параметр push\_pairs=False. Если True, то метод push ожидает два аргумента (key, value) и сохраняет их как кортеж. Если False — один аргумент.
    \item Создайте метод push, который, если push\_pairs=False, принимает один элемент. Если push\_pairs=True, принимает два аргумента (key, value) и сохраняет (key, value). Возвращает сохраненный элемент (или кортеж).
    \item Создайте метод pop, который выталкивает верхний элемент (или кортеж) и возвращает его. Если стек пуст, возвращает None.
    \item Создайте метод is\_empty, который возвращает True, если стек пуст, и False в противном случае.
    \item Создайте метод size, который возвращает текущее количество элементов в стеке.
    \item Создайте метод peek, который возвращает верхний элемент (или кортеж), если стек не пуст. Если стек пуст, возвращает None.
    \item Создайте экземпляр класса Stack с push\_pairs=True.
    \item Добавьте пары: ("a", 1), ("b", 2), ("c", 3).
    \item Выведите размер стека и результат peek (должен быть ("c",3)).
    \item Вызовите pop, выведите результат.
    \item Повторите вывод размера и peek.
\end{enumerate}

Пример использования:
\begin{lstlisting}[language=Python]
stack = Stack(push_pairs=True)
stack.push("a", 1)
stack.push("b", 2)
stack.push("c", 3)

print("Размер стека:", stack.size())
print("Верхний элемент:", stack.peek())  # ('c', 3)

popped = stack.pop()
print("Вытолкнут:", popped)  # ('c', 3)

print("Размер после pop:", stack.size())
print("Верхний элемент:", stack.peek())  # ('b', 2)
\end{lstlisting}

\item Написать программу на Python, которая создает класс Stack для представления стека с инкапсуляцией. Класс должен содержать методы push, pop, is\_empty, size и peek, которые реализуют операции вталкивания, выталкивания, проверки пустоты, получения размера и просмотра вершины стека соответственно. Программа также должна создавать экземпляр класса Stack, вталкивать в него элементы, выталкивать элементы и выводить информацию о стеке на экран.

Инструкции:
\begin{enumerate}
    \item Создайте класс Stack с методом \_\_init\_\_, который инициализирует пустой стек. Принимает параметр auto\_dedup=False. Если True, то при добавлении элемента, который уже есть в стеке (не обязательно на вершине), сначала удаляет все его предыдущие вхождения.
    \item Создайте метод push, который принимает элемент. Если auto\_dedup=True и такой элемент уже есть в стеке, удаляет все его вхождения, затем добавляет новый элемент. Иначе — просто добавляет.
    \item Создайте метод pop, который выталкивает верхний элемент из стека и возвращает его. Если стек пуст, возвращает None.
    \item Создайте метод is\_empty, который возвращает True, если стек пуст, и False в противном случае.
    \item Создайте метод size, который возвращает текущее количество элементов в стеке.
    \item Создайте метод peek, который возвращает верхний элемент стека, если стек не пуст. Если стек пуст, возвращает None.
    \item Создайте экземпляр класса Stack с auto\_dedup=True.
    \item Добавьте элементы: 1, 2, 1, 3, 2, 4.
    \item После каждого добавления выводите содержимое стека (реализуйте вспомогательный метод \_debug\_list, возвращающий список элементов снизу вверх — только для отладки, не включайте в задание студентам; в решении можно использовать stack.\_items, если инкапсуляция не строгая).
    \item Выведите итоговый размер и верхний элемент.
\end{enumerate}

Пример использования (с отладочным выводом для ясности):
\begin{lstlisting}[language=Python]
# (В решении студент не обязан реализовывать _debug_list, но для проверки можно временно добавить)
stack = Stack(auto_dedup=True)
stack.push(1)  # стек: [1]
stack.push(2)  # стек: [1,2]
stack.push(1)  # удаляет старую 1, добавляет новую -> [2,1]
stack.push(3)  # [2,1,3]
stack.push(2)  # удаляет 2, добавляет новую -> [1,3,2]
stack.push(4)  # [1,3,2,4]

print("Размер стека:", stack.size())     # 4
print("Верхний элемент:", stack.peek())   # 4
\end{lstlisting}

\item Написать программу на Python, которая создает класс Stack для представления стека с инкапсуляцией. Класс должен содержать методы push, pop, is\_empty, size и peek, которые реализуют операции вталкивания, выталкивания, проверки пустоты, получения размера и просмотра вершины стека соответственно. Программа также должна создавать экземпляр класса Stack, вталкивать в него элементы, выталкивать элементы и выводить информацию о стеке на экран.

Инструкции:
\begin{enumerate}
    \item Создайте класс Stack с методом \_\_init\_\_, который инициализирует пустой стек. Принимает параметр push\_if\_max=False. Если True, то элемент добавляется только если он больше всех текущих элементов в стеке.
    \item Создайте метод push, который принимает элемент. Если push\_if\_max=True и элемент не является строго больше всех элементов в стеке, он не добавляется. Иначе — добавляется.
    \item Создайте метод pop, который выталкивает верхний элемент из стека и возвращает его. Если стек пуст, возвращает None.
    \item Создайте метод is\_empty, который возвращает True, если стек пуст, и False в противном случае.
    \item Создайте метод size, который возвращает текущее количество элементов в стеке.
    \item Создайте метод peek, который возвращает верхний элемент стека, если стек не пуст. Если стек пуст, возвращает None.
    \item Создайте экземпляр класса Stack с push\_if\_max=True.
    \item Добавьте элементы: 5, 3 (не добавится, т.к. 3<5), 10, 7 (не добавится, т.к. 7<10), 15.
    \item Выведите размер стека (должен быть 3) и верхний элемент (15).
    \item Вызовите pop, выведите результат (15).
    \item Повторите вывод размера и верхнего элемента (теперь 10).
\end{enumerate}

Пример использования:
\begin{lstlisting}[language=Python]
stack = Stack(push_if_max=True)
stack.push(5)
stack.push(3)   # не добавится
stack.push(10)
stack.push(7)   # не добавится
stack.push(15)

print("Размер стека:", stack.size())     # 3
print("Верхний элемент:", stack.peek())   # 15

popped = stack.pop()
print("Вытолкнут:", popped)  # 15

print("Размер после pop:", stack.size())    # 2
print("Верхний элемент:", stack.peek())     # 10
\end{lstlisting}

\item Написать программу на Python, которая создает класс Stack для представления стека с инкапсуляцией. Класс должен содержать методы push, pop, is\_empty, size и peek, которые реализуют операции вталкивания, выталкивания, проверки пустоты, получения размера и просмотра вершины стека соответственно. Программа также должна создавать экземпляр класса Stack, вталкивать в него элементы, выталкивать элементы и выводить информацию о стеке на экран.

Инструкции:
\begin{enumerate}
    \item Создайте класс Stack с методом \_\_init\_\_, который инициализирует пустой стек. Принимает параметр cumulative=False. Если True, то при добавлении элемента он суммируется с предыдущим верхним элементом (первый элемент добавляется как есть).
    \item Создайте метод push, который принимает элемент. Если cumulative=True и стек не пуст, то добавляемый элемент становится element + текущий\_верх. Затем этот результат добавляется в стек. Если стек пуст, добавляется element как есть.
    \item Создайте метод pop, который выталкивает верхний элемент из стека и возвращает его. Если стек пуст, возвращает None.
    \item Создайте метод is\_empty, который возвращает True, если стек пуст, и False в противном случае.
    \item Создайте метод size, который возвращает текущее количество элементов в стеке.
    \item Создайте метод peek, который возвращает верхний элемент стека, если стек не пуст. Если стек пуст, возвращает None.
    \item Создайте экземпляр класса Stack с cumulative=True.
    \item Добавьте элементы: 1, 2, 3, 4.
    \item Выведите содержимое стека после каждого добавления (для проверки: после 1 → [1]; после 2 → [1,3]; после 3 → [1,3,6]; после 4 → [1,3,6,10]).
    \item Выведите итоговый размер и верхний элемент (10).
    \item Вызовите pop, выведите результат (10).
    \item Повторите вывод размера и верхнего элемента (теперь 6).
\end{enumerate}

Пример использования:
\begin{lstlisting}[language=Python]
stack = Stack(cumulative=True)
stack.push(1)  # [1]
stack.push(2)  # [1, 1+2=3]
stack.push(3)  # [1,3, 3+3=6]
stack.push(4)  # [1,3,6, 6+4=10]

print("Размер стека:", stack.size())     # 4
print("Верхний элемент:", stack.peek())   # 10

popped = stack.pop()
print("Вытолкнут:", popped)  # 10

print("Размер после pop:", stack.size())    # 3
print("Верхний элемент:", stack.peek())     # 6
\end{lstlisting}

\item Написать программу на Python, которая создает класс Stack для представления стека с инкапсуляцией. Класс должен содержать методы push, pop, is\_empty, size и peek, которые реализуют операции вталкивания, выталкивания, проверки пустоты, получения размера и просмотра вершины стека соответственно. Программа также должна создавать экземпляр класса Stack, вталкивать в него элементы, выталкивать элементы и выводить информацию о стеке на экран.

Инструкции:
\begin{enumerate}
    \item Создайте класс Stack с методом \_\_init\_\_, который инициализирует пустой стек. Принимает параметр push\_squared=False. Если True, то при добавлении элемент возводится в квадрат перед добавлением.
    \item Создайте метод push, который принимает элемент. Если push\_squared=True, добавляет element**2. Иначе — element.
    \item Создайте метод pop, который выталкивает верхний элемент из стека и возвращает его. Если стек пуст, возвращает None.
    \item Создайте метод is\_empty, который возвращает True, если стек пуст, и False в противном случае.
    \item Создайте метод size, который возвращает текущее количество элементов в стеке.
    \item Создайте метод peek, который возвращает верхний элемент стека, если стек не пуст. Если стек пуст, возвращает None.
    \item Создайте экземпляр класса Stack с push\_squared=True.
    \item Добавьте элементы: 2, 3, 4, 5.
    \item Выведите размер стека и верхний элемент (должен быть 25).
    \item Вызовите pop, выведите результат (25).
    \item Повторите вывод размера и верхнего элемента (теперь 16).
\end{enumerate}

Пример использования:
\begin{lstlisting}[language=Python]
stack = Stack(push_squared=True)
stack.push(2)  # добавит 4
stack.push(3)  # добавит 9
stack.push(4)  # добавит 16
stack.push(5)  # добавит 25

print("Размер стека:", stack.size())     # 4
print("Верхний элемент:", stack.peek())   # 25

popped = stack.pop()
print("Вытолкнут:", popped)  # 25

print("Размер после pop:", stack.size())    # 3
print("Верхний элемент:", stack.peek())     # 16
\end{lstlisting}

\item Написать программу на Python, которая создает класс Stack для представления стека с инкапсуляцией. Класс должен содержать методы push, pop, is\_empty, size и peek, которые реализуют операции вталкивания, выталкивания, проверки пустоты, получения размера и просмотра вершины стека соответственно. Программа также должна создавать экземпляр класса Stack, вталкивать в него элементы, выталкивать элементы и выводить информацию о стеке на экран.

Инструкции:
\begin{enumerate}
    \item Создайте класс Stack с методом \_\_init\_\_, который инициализирует пустой стек. Принимает параметр push\_absolute=False. Если True, то при добавлении сохраняется абсолютное значение элемента (abs(element)).
    \item Создайте метод push, который принимает элемент. Если push\_absolute=True, добавляет abs(element). Иначе — element.
    \item Создайте метод pop, который выталкивает верхний элемент из стека и возвращает его. Если стек пуст, возвращает None.
    \item Создайте метод is\_empty, который возвращает True, если стек пуст, и False в противном случае.
    \item Создайте метод size, который возвращает текущее количество элементов в стеке.
    \item Создайте метод peek, который возвращает верхний элемент стека, если стек не пуст. Если стек пуст, возвращает None.
    \item Создайте экземпляр класса Stack с push\_absolute=True.
    \item Добавьте элементы: -5, 3, -8, 2.
    \item Выведите размер стека и верхний элемент (должен быть 2).
    \item Вызовите pop, выведите результат (2).
    \item Повторите вывод размера и верхнего элемента (теперь 8).
\end{enumerate}

Пример использования:
\begin{lstlisting}[language=Python]
stack = Stack(push_absolute=True)
stack.push(-5)  # добавит 5
stack.push(3)   # добавит 3
stack.push(-8)  # добавит 8
stack.push(2)   # добавит 2

print("Размер стека:", stack.size())     # 4
print("Верхний элемент:", stack.peek())   # 2

popped = stack.pop()
print("Вытолкнут:", popped)  # 2

print("Размер после pop:", stack.size())    # 3
print("Верхний элемент:", stack.peek())     # 8
\end{lstlisting}

\item Написать программу на Python, которая создает класс Stack для представления стека с инкапсуляцией. Класс должен содержать методы push, pop, is\_empty, size и peek, которые реализуют операции вталкивания, выталкивания, проверки пустоты, получения размера и просмотра вершины стека соответственно. Программа также должна создавать экземпляр класса Stack, вталкивать в него элементы, выталкивать элементы и выводить информацию о стеке на экран.

Инструкции:
\begin{enumerate}
    \item Создайте класс Stack с методом \_\_init\_\_, который инициализирует пустой стек. Принимает параметр push\_rounded=False. Если True, то при добавлении элемент округляется до целого числа (round(element)).
    \item Создайте метод push, который принимает элемент. Если push\_rounded=True, добавляет round(element). Иначе — element.
    \item Создайте метод pop, который выталкивает верхний элемент из стека и возвращает его. Если стек пуст, возвращает None.
    \item Создайте метод is\_empty, который возвращает True, если стек пуст, и False в противном случае.
    \item Создайте метод size, который возвращает текущее количество элементов в стеке.
    \item Создайте метод peek, который возвращает верхний элемент стека, если стек не пуст. Если стек пуст, возвращает None.
    \item Создайте экземпляр класса Stack с push\_rounded=True.
    \item Добавьте элементы: 3.2, 4.7, 5.1, 6.9.
    \item Выведите размер стека и верхний элемент (должен быть 7).
    \item Вызовите pop, выведите результат (7).
    \item Повторите вывод размера и верхнего элемента (теперь 5).
\end{enumerate}

Пример использования:
\begin{lstlisting}[language=Python]
stack = Stack(push_rounded=True)
stack.push(3.2)  # 3
stack.push(4.7)  # 5
stack.push(5.1)  # 5
stack.push(6.9)  # 7

print("Размер стека:", stack.size())     # 4
print("Верхний элемент:", stack.peek())   # 7

popped = stack.pop()
print("Вытолкнут:", popped)  # 7

print("Размер после pop:", stack.size())    # 3
print("Верхний элемент:", stack.peek())     # 5
\end{lstlisting}

\item Написать программу на Python, которая создает класс Stack для представления стека с инкапсуляцией. Класс должен содержать методы push, pop, is\_empty, size и peek, которые реализуют операции вталкивания, выталкивания, проверки пустоты, получения размера и просмотра вершины стека соответственно. Программа также должна создавать экземпляр класса Stack, вталкивать в него элементы, выталкивать элементы и выводить информацию о стеке на экран.

Инструкции:
\begin{enumerate}
    \item Создайте класс Stack с методом \_\_init\_\_, который инициализирует пустой стек. Принимает параметр push\_negated=False. Если True, то при добавлении элемент сохраняется с обратным знаком (-element).
    \item Создайте метод push, который принимает элемент. Если push\_negated=True, добавляет -element. Иначе — element.
    \item Создайте метод pop, который выталкивает верхний элемент из стека и возвращает его. Если стек пуст, возвращает None.
    \item Создайте метод is\_empty, который возвращает True, если стек пуст, и False в противном случае.
    \item Создайте метод size, который возвращает текущее количество элементов в стеке.
    \item Создайте метод peek, который возвращает верхний элемент стека, если стек не пуст. Если стек пуст, возвращает None.
    \item Создайте экземпляр класса Stack с push\_negated=True.
    \item Добавьте элементы: 10, 20, 30, 40.
    \item Выведите размер стека и верхний элемент (должен быть -40).
    \item Вызовите pop, выведите результат (-40).
    \item Повторите вывод размера и верхнего элемента (теперь -30).
\end{enumerate}

Пример использования:
\begin{lstlisting}[language=Python]
stack = Stack(push_negated=True)
stack.push(10)  # -10
stack.push(20)  # -20
stack.push(30)  # -30
stack.push(40)  # -40

print("Размер стека:", stack.size())     # 4
print("Верхний элемент:", stack.peek())   # -40

popped = stack.pop()
print("Вытолкнут:", popped)  # -40

print("Размер после pop:", stack.size())    # 3
print("Верхний элемент:", stack.peek())     # -30
\end{lstlisting}

\item Написать программу на Python, которая создает класс Stack для представления стека с инкапсуляцией. Класс должен содержать методы push, pop, is\_empty, size и peek, которые реализуют операции вталкивания, выталкивания, проверки пустоты, получения размера и просмотра вершины стека соответственно. Программа также должна создавать экземпляр класса Stack, вталкивать в него элементы, выталкивать элементы и выводить информацию о стеке на экран.

Инструкции:
\begin{enumerate}
    \item Создайте класс Stack с методом \_\_init\_\_, который инициализирует пустой стек. Принимает параметр push\_doubled=False. Если True, то при добавлении элемент умножается на 2.
    \item Создайте метод push, который принимает элемент. Если push\_doubled=True, добавляет element * 2. Иначе — element.
    \item Создайте метод pop, который выталкивает верхний элемент из стека и возвращает его. Если стек пуст, возвращает None.
    \item Создайте метод is\_empty, который возвращает True, если стек пуст, и False в противном случае.
    \item Создайте метод size, который возвращает текущее количество элементов в стеке.
    \item Создайте метод peek, который возвращает верхний элемент стека, если стек не пуст. Если стек пуст, возвращает None.
    \item Создайте экземпляр класса Stack с push\_doubled=True.
    \item Добавьте элементы: 1, 2, 3, 4.
    \item Выведите размер стека и верхний элемент (должен быть 8).
    \item Вызовите pop, выведите результат (8).
    \item Повторите вывод размера и верхнего элемента (теперь 6).
\end{enumerate}

Пример использования:
\begin{lstlisting}[language=Python]
stack = Stack(push_doubled=True)
stack.push(1)  # 2
stack.push(2)  # 4
stack.push(3)  # 6
stack.push(4)  # 8

print("Размер стека:", stack.size())     # 4
print("Верхний элемент:", stack.peek())   # 8

popped = stack.pop()
print("Вытолкнут:", popped)  # 8

print("Размер после pop:", stack.size())    # 3
print("Верхний элемент:", stack.peek())     # 6
\end{lstlisting}

\item Написать программу на Python, которая создает класс Stack для представления стека с инкапсуляцией. Класс должен содержать методы push, pop, is\_empty, size и peek, которые реализуют операции вталкивания, выталкивания, проверки пустоты, получения размера и просмотра вершины стека соответственно. Программа также должна создавать экземпляр класса Stack, вталкивать в него элементы, выталкивать элементы и выводить информацию о стеке на экран.

Инструкции:
\begin{enumerate}
    \item Создайте класс Stack с методом \_\_init\_\_, который инициализирует пустой стек. Принимает параметр push\_halved=False. Если True, то при добавлении элемент делится на 2.0.
    \item Создайте метод push, который принимает элемент. Если push\_halved=True, добавляет element / 2.0. Иначе — element.
    \item Создайте метод pop, который выталкивает верхний элемент из стека и возвращает его. Если стек пуст, возвращает None.
    \item Создайте метод is\_empty, который возвращает True, если стек пуст, и False в противном случае.
    \item Создайте метод size, который возвращает текущее количество элементов в стеке.
    \item Создайте метод peek, который возвращает верхний элемент стека, если стек не пуст. Если стек пуст, возвращает None.
    \item Создайте экземпляр класса Stack с push\_halved=True.
    \item Добавьте элементы: 4, 8, 12, 16.
    \item Выведите размер стека и верхний элемент (должен быть 8.0).
    \item Вызовите pop, выведите результат (8.0).
    \item Повторите вывод размера и верхнего элемента (теперь 6.0).
\end{enumerate}

Пример использования:
\begin{lstlisting}[language=Python]
stack = Stack(push_halved=True)
stack.push(4)   # 2.0
stack.push(8)   # 4.0
stack.push(12)  # 6.0
stack.push(16)  # 8.0

print("Размер стека:", stack.size())     # 4
print("Верхний элемент:", stack.peek())   # 8.0

popped = stack.pop()
print("Вытолкнут:", popped)  # 8.0

print("Размер после pop:", stack.size())    # 3
print("Верхний элемент:", stack.peek())     # 6.0
\end{lstlisting}

\item Написать программу на Python, которая создает класс Stack для представления стека с инкапсуляцией. Класс должен содержать методы push, pop, is\_empty, size и peek, которые реализуют операции вталкивания, выталкивания, проверки пустоты, получения размера и просмотра вершины стека соответственно. Программа также должна создавать экземпляр класса Stack, вталкивать в него элементы, выталкивать элементы и выводить информацию о стеке на экран.

Инструкции:
\begin{enumerate}
    \item Создайте класс Stack с методом \_\_init\_\_, который инициализирует пустой стек. Принимает параметр push\_as\_string=False. Если True, то при добавлении элемент преобразуется в строку str(element).
    \item Создайте метод push, который принимает элемент. Если push\_as\_string=True, добавляет str(element). Иначе — element.
    \item Создайте метод pop, который выталкивает верхний элемент из стека и возвращает его. Если стек пуст, возвращает None.
    \item Создайте метод is\_empty, который возвращает True, если стек пуст, и False в противном случае.
    \item Создайте метод size, который возвращает текущее количество элементов в стеке.
    \item Создайте метод peek, который возвращает верхний элемент стека, если стек не пуст. Если стек пуст, возвращает None.
    \item Создайте экземпляр класса Stack с push\_as\_string=True.
    \item Добавьте элементы: 100, 200, 300, 400.
    \item Выведите размер стека и верхний элемент (должен быть "400").
    \item Вызовите pop, выведите результат ("400").
    \item Повторите вывод размера и верхнего элемента (теперь "300").
\end{enumerate}

Пример использования:
\begin{lstlisting}[language=Python]
stack = Stack(push_as_string=True)
stack.push(100)  # "100"
stack.push(200)  # "200"
stack.push(300)  # "300"
stack.push(400)  # "400"

print("Размер стека:", stack.size())     # 4
print("Верхний элемент:", stack.peek())   # "400"

popped = stack.pop()
print("Вытолкнут:", popped)  # "400"

print("Размер после pop:", stack.size())    # 3
print("Верхний элемент:", stack.peek())     # "300"
\end{lstlisting}

\item Написать программу на Python, которая создает класс Stack для представления стека с инкапсуляцией. Класс должен содержать методы push, pop, is\_empty, size и peek, которые реализуют операции вталкивания, выталкивания, проверки пустоты, получения размера и просмотра вершины стека соответственно. Программа также должна создавать экземпляр класса Stack, вталкивать в него элементы, выталкивать элементы и выводить информацию о стеке на экран.

Инструкции:
\begin{enumerate}
    \item Создайте класс Stack с методом \_\_init\_\_, который инициализирует пустой стек. Принимает параметр push\_with\_index=False. Если True, то при добавлении сохраняется кортеж (элемент, порядковый\_номер\_добавления).
    \item Создайте метод push, который принимает элемент. Если push\_with\_index=True, добавляет (element, self.\_counter), где \_counter — внутренний счетчик, увеличивающийся при каждом добавлении. Иначе — element.
    \item Создайте метод pop, который выталкивает верхний элемент (или кортеж) и возвращает его. Если стек пуст, возвращает None.
    \item Создайте метод is\_empty, который возвращает True, если стек пуст, и False в противном случае.
    \item Создайте метод size, который возвращает текущее количество элементов в стеке.
    \item Создайте метод peek, который возвращает верхний элемент (или кортеж), если стек не пуст. Если стек пуст, возвращает None.
    \item Создайте экземпляр класса Stack с push\_with\_index=True.
    \item Добавьте элементы: "alpha", "beta", "gamma".
    \item Выведите размер стека и результат peek (должен быть ("gamma", 2) — если считать с 0).
    \item Вызовите pop, выведите результат.
    \item Повторите вывод размера и peek.
\end{enumerate}

Пример использования:
\begin{lstlisting}[language=Python]
stack = Stack(push_with_index=True)
stack.push("alpha")  # ("alpha", 0)
stack.push("beta")   # ("beta", 1)
stack.push("gamma")  # ("gamma", 2)

print("Размер стека:", stack.size())
print("Верхний элемент:", stack.peek())  # ('gamma', 2)

popped = stack.pop()
print("Вытолкнут:", popped)  # ('gamma', 2)

print("Размер после pop:", stack.size())
print("Верхний элемент:", stack.peek())  # ('beta', 1)
\end{lstlisting}

\item Написать программу на Python, которая создает класс Stack для представления стека с инкапсуляцией. Класс должен содержать методы push, pop, is\_empty, size и peek, которые реализуют операции вталкивания, выталкивания, проверки пустоты, получения размера и просмотра вершины стека соответственно. Программа также должна создавать экземпляр класса Stack, вталкивать в него элементы, выталкивать элементы и выводить информацию о стеке на экран.

Инструкции:
\begin{enumerate}
    \item Создайте класс Stack с методом \_\_init\_\_, который инициализирует пустой стек. Принимает параметр push\_unique\_top=False. Если True, то при добавлении, если элемент равен текущему верхнему, он не добавляется.
    \item Создайте метод push, который принимает элемент. Если push\_unique\_top=True и стек не пуст и element == текущий\_верх, то элемент не добавляется. Иначе — добавляется.
    \item Создайте метод pop, который выталкивает верхний элемент из стека и возвращает его. Если стек пуст, возвращает None.
    \item Создайте метод is\_empty, который возвращает True, если стек пуст, и False в противном случае.
    \item Создайте метод size, который возвращает текущее количество элементов в стеке.
    \item Создайте метод peek, который возвращает верхний элемент стека, если стек не пуст. Если стек пуст, возвращает None.
    \item Создайте экземпляр класса Stack с push\_unique\_top=True.
    \item Добавьте элементы: 1, 2, 2, 3, 3, 3, 4.
    \item Выведите размер стека (должен быть 4) и верхний элемент (4).
    \item Вызовите pop, выведите результат (4).
    \item Повторите вывод размера и верхнего элемента (теперь 3).
\end{enumerate}

Пример использования:
\begin{lstlisting}[language=Python]
stack = Stack(push_unique_top=True)
stack.push(1)
stack.push(2)
stack.push(2)  # не добавится
stack.push(3)
stack.push(3)  # не добавится
stack.push(3)  # не добавится
stack.push(4)

print("Размер стека:", stack.size())     # 4
print("Верхний элемент:", stack.peek())   # 4

popped = stack.pop()
print("Вытолкнут:", popped)  # 4

print("Размер после pop:", stack.size())    # 3
print("Верхний элемент:", stack.peek())     # 3
\end{lstlisting}

\item Написать программу на Python, которая создает класс Stack для представления стека с инкапсуляцией. Класс должен содержать методы push, pop, is\_empty, size и peek, которые реализуют операции вталкивания, выталкивания, проверки пустоты, получения размера и просмотра вершины стека соответственно. Программа также должна создавать экземпляр класса Stack, вталкивать в него элементы, выталкивать элементы и выводить информацию о стеке на экран.

Инструкции:
\begin{enumerate}
    \item Создайте класс Stack с методом \_\_init\_\_, который инициализирует пустой стек. Принимает параметр push\_even\_only=False. Если True, то добавляются только четные числа.
    \item Создайте метод push, который принимает элемент. Если push\_even\_only=True и element \% 2 != 0, элемент не добавляется. Иначе — добавляется.
    \item Создайте метод pop, который выталкивает верхний элемент из стека и возвращает его. Если стек пуст, возвращает None.
    \item Создайте метод is\_empty, который возвращает True, если стек пуст, и False в противном случае.
    \item Создайте метод size, который возвращает текущее количество элементов в стеке.
    \item Создайте метод peek, который возвращает верхний элемент стека, если стек не пуст. Если стек пуст, возвращает None.
    \item Создайте экземпляр класса Stack с push\_even\_only=True.
    \item Добавьте элементы: 1 (не добавится), 2, 3 (не добавится), 4, 5 (не добавится), 6.
    \item Выведите размер стека (должен быть 3) и верхний элемент (6).
    \item Вызовите pop, выведите результат (6).
    \item Повторите вывод размера и верхнего элемента (теперь 4).
\end{enumerate}

Пример использования:
\begin{lstlisting}[language=Python]
stack = Stack(push_even_only=True)
stack.push(1)  # нет
stack.push(2)
stack.push(3)  # нет
stack.push(4)
stack.push(5)  # нет
stack.push(6)

print("Размер стека:", stack.size())     # 3
print("Верхний элемент:", stack.peek())   # 6

popped = stack.pop()
print("Вытолкнут:", popped)  # 6

print("Размер после pop:", stack.size())    # 2
print("Верхний элемент:", stack.peek())     # 4
\end{lstlisting}

\item Написать программу на Python, которая создает класс Stack для представления стека с инкапсуляцией. Класс должен содержать методы push, pop, is\_empty, size и peek, которые реализуют операции вталкивания, выталкивания, проверки пустоты, получения размера и просмотра вершины стека соответственно. Программа также должна создавать экземпляр класса Stack, вталкивать в него элементы, выталкивать элементы и выводить информацию о стеке на экран.

Инструкции:
\begin{enumerate}
    \item Создайте класс Stack с методом \_\_init\_\_, который инициализирует пустой стек. Принимает параметр push\_odd\_only=False. Если True, то добавляются только нечетные числа.
    \item Создайте метод push, который принимает элемент. Если push\_odd\_only=True и element \% 2 == 0, элемент не добавляется. Иначе — добавляется.
    \item Создайте метод pop, который выталкивает верхний элемент из стека и возвращает его. Если стек пуст, возвращает None.
    \item Создайте метод is\_empty, который возвращает True, если стек пуст, и False в противном случае.
    \item Создайте метод size, который возвращает текущее количество элементов в стеке.
    \item Создайте метод peek, который возвращает верхний элемент стека, если стек не пуст. Если стек пуст, возвращает None.
    \item Создайте экземпляр класса Stack с push\_odd\_only=True.
    \item Добавьте элементы: 2 (не добавится), 1, 4 (не добавится), 3, 6 (не добавится), 5.
    \item Выведите размер стека (должен быть 3) и верхний элемент (5).
    \item Вызовите pop, выведите результат (5).
    \item Повторите вывод размера и верхнего элемента (теперь 3).
\end{enumerate}

Пример использования:
\begin{lstlisting}[language=Python]
stack = Stack(push_odd_only=True)
stack.push(2)  # нет
stack.push(1)
stack.push(4)  # нет
stack.push(3)
stack.push(6)  # нет
stack.push(5)

print("Размер стека:", stack.size())     # 3
print("Верхний элемент:", stack.peek())   # 5

popped = stack.pop()
print("Вытолкнут:", popped)  # 5

print("Размер после pop:", stack.size())    # 2
print("Верхний элемент:", stack.peek())     # 3
\end{lstlisting}

\end{enumerate}

\subsubsection{Задача 3 (двусвязный список)}

\begin{enumerate}
\item Написать программу на Python, которая создает класс DoublyLinkedList, представляющий \textbf{двусвязный список} с инкапсуляцией внутренней структуры. Класс должен содержать методы для отображения данных, вставки и удаления узлов. Программа также должна создавать экземпляр класса, вставлять узлы и удалять узлы.

Инструкции:
\begin{enumerate}
    \item Создайте класс Node с методом \_\_init\_\_, который принимает данные data и сохраняет их в атрибуте self.\_data. Также инициализирует self.\_next и self.\_prev как None.
    \item Создайте класс DoublyLinkedList с методом \_\_init\_\_, который инициализирует self.\_head и self.\_tail как None.
    \item Создайте метод display в классе DoublyLinkedList, который выводит все элементы списка через пробел, двигаясь от головы к хвосту. Если список пуст — выводит "Список пуст".
    \item Создайте метод insert в классе DoublyLinkedList, который принимает значение и вставляет новый узел \textbf{в конец списка}. Обновляет self.\_tail и ссылки prev/next.
    \item Создайте метод delete в классе DoublyLinkedList, который принимает значение и удаляет \textbf{первое} вхождение узла с этим значением. Корректно обновляет соседние ссылки и self.\_head/self.\_tail при необходимости.
    \item Создайте экземпляр класса DoublyLinkedList.
    \item Вставьте узлы со значениями 10, 20, 30, 40.
    \item Вызовите display и выведите результат.
    \item Вставьте узел со значением 50.
    \item Снова вызовите display.
    \item Удалите узел со значением 20.
    \item Снова вызовите display.
\end{enumerate}

Пример использования:
\begin{lstlisting}[language=Python]
dll = DoublyLinkedList()
dll.insert(10)
dll.insert(20)
dll.insert(30)
dll.insert(40)

print("Initial Doubly Linked List:")
dll.display()

dll.insert(50)
print("After inserting 50:")
dll.display()

dll.delete(20)
print("After deleting 20:")
dll.display()
\end{lstlisting}

\item Написать программу на Python, которая создает класс DoublyLinkedList, представляющий \textbf{двусвязный список} с инкапсуляцией. Класс должен содержать методы для отображения данных, вставки и удаления узлов. Программа также должна создавать экземпляр класса, вставлять узлы и удалять узлы.

Инструкции:
\begin{enumerate}
    \item Создайте класс Node с методом \_\_init\_\_, который принимает item и сохраняет его в self.\_value. Инициализирует self.\_next и self.\_previous как None.
    \item Создайте класс DoublyLinkedList с методом \_\_init\_\_, который инициализирует self.\_first и self.\_last как None.
    \item Создайте метод display в классе DoublyLinkedList, который выводит элементы списка от первого к последнему, разделенные запятыми. Если список пуст — выводит "Нет элементов".
    \item Создайте метод insert в классе DoublyLinkedList, который принимает элемент и вставляет его \textbf{в начало списка}. Обновляет self.\_first и ссылки.
    \item Создайте метод delete в классе DoublyLinkedList, который принимает значение и удаляет \textbf{последнее} вхождение узла с этим значением. Корректно обновляет связи и границы списка.
    \item Создайте экземпляр класса DoublyLinkedList.
    \item Вставьте узлы: 5, 15, 25, 15.
    \item Вызовите display.
    \item Вставьте узел 35 в начало.
    \item Снова вызовите display.
    \item Удалите последнее вхождение 15.
    \item Снова вызовите display.
\end{enumerate}

Пример использования:
\begin{lstlisting}[language=Python]
dll = DoublyLinkedList()
dll.insert(5)
dll.insert(15)
dll.insert(25)
dll.insert(15)

print("Initial Doubly Linked List:")
dll.display()

dll.insert(35)
print("After inserting 35 at start:")
dll.display()

dll.delete(15)
print("After deleting last occurrence of 15:")
dll.display()
\end{lstlisting}

\item Написать программу на Python, которая создает класс DoublyLinkedList, представляющий \textbf{двусвязный список} с инкапсуляцией. Класс должен содержать методы для отображения данных, вставки и удаления узлов. Программа также должна создавать экземпляр класса, вставлять узлы и удалять узлы.

Инструкции:
\begin{enumerate}
    \item Создайте класс Node с методом \_\_init\_\_, который принимает content и сохраняет его в self.\_payload. Инициализирует self.\_forward и self.\_backward как None.
    \item Создайте класс DoublyLinkedList с методом \_\_init\_\_, который инициализирует self.\_root и self.\_end как None.
    \item Создайте метод display в классе DoublyLinkedList, который выводит элементы в формате "[элемент1] <-> [элемент2] <-> ...". Если пуст — "Пусто".
    \item Создайте метод insert в классе DoublyLinkedList, который принимает значение и вставляет его \textbf{после первого узла} (если список не пуст; если пуст — вставляет как первый).
    \item Создайте метод delete в классе DoublyLinkedList, который принимает значение и удаляет \textbf{все вхождения} этого значения. Обновляет ссылки и границы.
    \item Создайте экземпляр класса DoublyLinkedList.
    \item Вставьте узлы: 100, 200, 300.
    \item Вызовите display.
    \item Вставьте 150 после первого узла.
    \item Снова вызовите display.
    \item Удалите все вхождения 150.
    \item Снова вызовите display.
\end{enumerate}

Пример использования:
\begin{lstlisting}[language=Python]
dll = DoublyLinkedList()
dll.insert(100)
dll.insert(200)
dll.insert(300)

print("Initial Doubly Linked List:")
dll.display()

dll.insert(150)
print("After inserting 150 after first:")
dll.display()

dll.delete(150)
print("After deleting all 150s:")
dll.display()
\end{lstlisting}

\item Написать программу на Python, которая создает класс DoublyLinkedList, представляющий \textbf{двусвязный список} с инкапсуляцией. Класс должен содержать методы для отображения данных, вставки и удаления узлов. Программа также должна создавать экземпляр класса, вставлять узлы и удалять узлы.

Инструкции:
\begin{enumerate}
    \item Создайте класс Node с методом \_\_init\_\_, который принимает entry и сохраняет его в self.\_item. Инициализирует self.\_succ и self.\_pred как None.
    \item Создайте класс DoublyLinkedList с методом \_\_init\_\_, который инициализирует self.\_top и self.\_bottom как None.
    \item Создайте метод display в классе DoublyLinkedList, который выводит элементы в обратном порядке (от хвоста к голове), разделенные " | ". Если пуст — "Обратный просмотр: пусто".
    \item Создайте метод insert в классе DoublyLinkedList, который принимает значение и вставляет его \textbf{перед последним узлом} (если узлов >1; если 0 или 1 — вставляет в конец).
    \item Создайте метод delete в классе DoublyLinkedList, который принимает значение и удаляет первый найденный узел. Если узел — единственный, обнуляет self.\_top и self.\_bottom.
    \item Создайте экземпляр класса DoublyLinkedList.
    \item Вставьте узлы: 7, 14, 21.
    \item Вызовите display.
    \item Вставьте 18 перед последним узлом.
    \item Снова вызовите display.
    \item Удалите узел со значением 14.
    \item Снова вызовите display.
\end{enumerate}

Пример использования:
\begin{lstlisting}[language=Python]
dll = DoublyLinkedList()
dll.insert(7)
dll.insert(14)
dll.insert(21)

print("Initial Doubly Linked List (reversed):")
dll.display()

dll.insert(18)
print("After inserting 18 before last:")
dll.display()

dll.delete(14)
print("After deleting 14:")
dll.display()
\end{lstlisting}

\item Написать программу на Python, которая создает класс DoublyLinkedList, представляющий \textbf{двусвязный список} с инкапсуляцией. Класс должен содержать методы для отображения данных, вставки и удаления узлов. Программа также должна создавать экземпляр класса, вставлять узлы и удалять узлы.

Инструкции:
\begin{enumerate}
    \item Создайте класс Node с методом \_\_init\_\_, который принимает value и сохраняет его в self.\_key. Инициализирует self.\_link\_next и self.\_link\_prev как None.
    \item Создайте класс DoublyLinkedList с методом \_\_init\_\_, который инициализирует self.\_header и self.\_trailer как None.
    \item Создайте метод display в классе DoublyLinkedList, который выводит элементы в квадратных скобках через запятую: [a, b, c]. Если пуст — [].
    \item Создайте метод insert в классе DoublyLinkedList, который принимает значение и вставляет его \textbf{только если такого значения еще нет в списке}. Вставляет в конец.
    \item Создайте метод delete в классе DoublyLinkedList, который принимает значение и удаляет узел, если он существует. Если не существует — ничего не делает.
    \item Создайте экземпляр класса DoublyLinkedList.
    \item Вставьте узлы: 3, 6, 9, 6 (второй 6 не вставится).
    \item Вызовите display.
    \item Вставьте 12.
    \item Снова вызовите display.
    \item Удалите 6.
    \item Снова вызовите display.
\end{enumerate}

Пример использования:
\begin{lstlisting}[language=Python]
dll = DoublyLinkedList()
dll.insert(3)
dll.insert(6)
dll.insert(9)
dll.insert(6)  # игнорируется

print("Initial Doubly Linked List:")
dll.display()

dll.insert(12)
print("After inserting 12:")
dll.display()

dll.delete(6)
print("After deleting 6:")
dll.display()
\end{lstlisting}

\item Написать программу на Python, которая создает класс DoublyLinkedList, представляющий \textbf{двусвязный список} с инкапсуляцией. Класс должен содержать методы для отображения данных, вставки и удаления узлов. Программа также должна создавать экземпляр класса, вставлять узлы и удалять узлы.

Инструкции:
\begin{enumerate}
    \item Создайте класс Node с методом \_\_init\_\_, который принимает data\_point и сохраняет его в self.\_datum. Инициализирует self.\_next\_node и self.\_prev\_node как None.
    \item Создайте класс DoublyLinkedList с методом \_\_init\_\_, который инициализирует self.\_start и self.\_finish как None.
    \item Создайте метод display в классе DoublyLinkedList, который выводит элементы в формате "Элементы: val1 -> val2 -> val3", двигаясь от начала к концу. Если пуст — "Элементы: (нет)".
    \item Создайте метод insert в классе DoublyLinkedList, который принимает значение и вставляет его \textbf{только если оно больше последнего элемента} (если список не пуст). Если пуст — вставляет. Иначе — игнорирует.
    \item Создайте метод delete в классе DoublyLinkedList, который принимает значение и удаляет \textbf{первый узел}, если он равен значению. Не ищет дальше.
    \item Создайте экземпляр класса DoublyLinkedList.
    \item Вставьте узлы: 1, 5, 3 (игнорируется), 10.
    \item Вызовите display.
    \item Вставьте 7 (игнорируется, т.к. 7 < 10).
    \item Снова вызовите display.
    \item Удалите 5.
    \item Снова вызовите display.
\end{enumerate}

Пример использования:
\begin{lstlisting}[language=Python]
dll = DoublyLinkedList()
dll.insert(1)
dll.insert(5)
dll.insert(3)  # игнорируется
dll.insert(10)

print("Initial Doubly Linked List:")
dll.display()

dll.insert(7)  # игнорируется
print("After attempting to insert 7:")
dll.display()

dll.delete(5)
print("After deleting 5:")
dll.display()
\end{lstlisting}

\item Написать программу на Python, которая создает класс DoublyLinkedList, представляющий \textbf{двусвязный список} с инкапсуляцией. Класс должен содержать методы для отображения данных, вставки и удаления узлов. Программа также должна создавать экземпляр класса, вставлять узлы и удалять узлы.

Инструкции:
\begin{enumerate}
    \item Создайте класс Node с методом \_\_init\_\_, который принимает item\_value и сохраняет его в self.\_content. Инициализирует self.\_ptr\_next и self.\_ptr\_prev как None.
    \item Создайте класс DoublyLinkedList с методом \_\_init\_\_, который инициализирует self.\_head\_node и self.\_tail\_node как None.
    \item Создайте метод display в классе DoublyLinkedList, который выводит элементы в виде строки, разделенной точками: "a.b.c". Если пуст — "пусто".
    \item Создайте метод insert в классе DoublyLinkedList, который принимает значение и вставляет его \textbf{в середину списка} (если четное количество — после левой средней позиции; если нечетное — в центр). Если список пуст — вставляет как первый.
    \item Создайте метод delete в классе DoublyLinkedList, который принимает значение и удаляет \textbf{все узлы с этим значением}.
    \item Создайте экземпляр класса DoublyLinkedList.
    \item Вставьте узлы: 10, 20, 30.
    \item Вызовите display.
    \item Вставьте 25 в середину (между 20 и 30).
    \item Снова вызовите display.
    \item Удалите все вхождения 25.
    \item Снова вызовите display.
\end{enumerate}

Пример использования:
\begin{lstlisting}[language=Python]
dll = DoublyLinkedList()
dll.insert(10)
dll.insert(20)
dll.insert(30)

print("Initial Doubly Linked List:")
dll.display()

dll.insert(25)
print("After inserting 25 in middle:")
dll.display()

dll.delete(25)
print("After deleting 25:")
dll.display()
\end{lstlisting}

\item Написать программу на Python, которая создает класс DoublyLinkedList, представляющий \textbf{двусвязный список} с инкапсуляцией. Класс должен содержать методы для отображения данных, вставки и удаления узлов. Программа также должна создавать экземпляр класса, вставлять узлы и удалять узлы.

Инструкции:
\begin{enumerate}
    \item Создайте класс Node с методом \_\_init\_\_, который принимает node\_data и сохраняет его в self.\_info. Инициализирует self.\_nxt и self.\_prv как None.
    \item Создайте класс DoublyLinkedList с методом \_\_init\_\_, который инициализирует self.\_front и self.\_rear как None.
    \item Создайте метод display в классе DoublyLinkedList, который выводит элементы в формате "Front->Back: [значения]" и "Back->Front: [значения в обратном порядке]". Если пуст — "Список пуст в обоих направлениях".
    \item Создайте метод insert в классе DoublyLinkedList, который принимает значение и вставляет его \textbf{в начало, только если значение четное}. Если нечетное — вставляет в конец.
    \item Создайте метод delete в классе DoublyLinkedList, который принимает значение и удаляет \textbf{первое вхождение}.
    \item Создайте экземпляр класса DoublyLinkedList.
    \item Вставьте узлы: 4 (в начало), 7 (в конец), 6 (в начало), 9 (в конец).
    \item Вызовите display.
    \item Вставьте 8 (в начало).
    \item Снова вызовите display.
    \item Удалите 7.
    \item Снова вызовите display.
\end{enumerate}

Пример использования:
\begin{lstlisting}[language=Python]
dll = DoublyLinkedList()
dll.insert(4)
dll.insert(7)
dll.insert(6)
dll.insert(9)

print("Initial Doubly Linked List:")
dll.display()

dll.insert(8)
print("After inserting 8:")
dll.display()

dll.delete(7)
print("After deleting 7:")
dll.display()
\end{lstlisting}

\item Написать программу на Python, которая создает класс DoublyLinkedList, представляющий \textbf{двусвязный список} с инкапсуляцией. Класс должен содержать методы для отображения данных, вставки и удаления узлов. Программа также должна создавать экземпляр класса, вставлять узлы и удалять узлы.

Инструкции:
\begin{enumerate}
    \item Создайте класс Node с методом \_\_init\_\_, который принимает value и сохраняет его в self.\_element. Инициализирует self.\_next\_elem и self.\_prev\_elem как None.
    \item Создайте класс DoublyLinkedList с методом \_\_init\_\_, который инициализирует self.\_head\_elem и self.\_tail\_elem как None.
    \item Создайте метод display в классе DoublyLinkedList, который выводит элементы в виде: "HEAD <-> val1 <-> val2 <-> ... <-> TAIL". Если пуст — "HEAD <-> TAIL (пусто)".
    \item Создайте метод insert в классе DoublyLinkedList, который принимает значение и вставляет его \textbf{после узла с наименьшим значением} (если несколько — после первого). Если список пуст — вставляет как единственный.
    \item Создайте метод delete в классе DoublyLinkedList, который принимает значение и удаляет \textbf{последнее вхождение}.
    \item Создайте экземпляр класса DoublyLinkedList.
    \item Вставьте узлы: 50, 30, 40.
    \item Вызовите display.
    \item Вставьте 35 (после 30 — минимального).
    \item Снова вызовите display.
    \item Удалите последнее вхождение 40.
    \item Снова вызовите display.
\end{enumerate}

Пример использования:
\begin{lstlisting}[language=Python]
dll = DoublyLinkedList()
dll.insert(50)
dll.insert(30)
dll.insert(40)

print("Initial Doubly Linked List:")
dll.display()

dll.insert(35)
print("After inserting 35 after min:")
dll.display()

dll.delete(40)
print("After deleting last occurrence of 40:")
dll.display()
\end{lstlisting}

\item Написать программу на Python, которая создает класс DoublyLinkedList, представляющий \textbf{двусвязный список} с инкапсуляцией. Класс должен содержать методы для отображения данных, вставки и удаления узлов. Программа также должна создавать экземпляр класса, вставлять узлы и удалять узлы.

Инструкции:
\begin{enumerate}
    \item Создайте класс Node с методом \_\_init\_\_, который принимает data и сохраняет его в self.\_val. Инициализирует self.\_link\_f и self.\_link\_b как None.
    \item Создайте класс DoublyLinkedList с методом \_\_init\_\_, который инициализирует self.\_first\_item и self.\_last\_item как None.
    \item Создайте метод display в классе DoublyLinkedList, который выводит элементы в виде: "Элементы (прямой порядок): ...", а затем "Элементы (обратный порядок): ...". Если пуст — "Нет данных".
    \item Создайте метод insert в классе DoublyLinkedList, который принимает значение и вставляет его \textbf{перед узлом с наибольшим значением} (если несколько — перед первым). Если список пуст — вставляет как единственный.
    \item Создайте метод delete в классе DoublyLinkedList, который принимает значение и удаляет \textbf{все вхождения}.
    \item Создайте экземпляр класса DoublyLinkedList.
    \item Вставьте узлы: 5, 15, 10.
    \item Вызовите display.
    \item Вставьте 12 (перед 15 — максимальным).
    \item Снова вызовите display.
    \item Удалите все вхождения 10.
    \item Снова вызовите display.
\end{enumerate}

Пример использования:
\begin{lstlisting}[language=Python]
dll = DoublyLinkedList()
dll.insert(5)
dll.insert(15)
dll.insert(10)

print("Initial Doubly Linked List:")
dll.display()

dll.insert(12)
print("After inserting 12 before max:")
dll.display()

dll.delete(10)
print("After deleting all 10s:")
dll.display()
\end{lstlisting}

\item Написать программу на Python, которая создает класс DoublyLinkedList, представляющий \textbf{двусвязный список} с инкапсуляцией. Класс должен содержать методы для отображения данных, вставки и удаления узлов. Программа также должна создавать экземпляр класса, вставлять узлы и удалять узлы.

Инструкции:
\begin{enumerate}
    \item Создайте класс Node с методом \_\_init\_\_, который принимает item и сохраняет его в self.\_data\_field. Инициализирует self.\_next\_ref и self.\_prev\_ref как None.
    \item Создайте класс DoublyLinkedList с методом \_\_init\_\_, который инициализирует self.\_entry\_point и self.\_exit\_point как None.
    \item Создайте метод display в классе DoublyLinkedList, который выводит элементы в одну строку, разделенные " => ", и в конце добавляет " => None". Если пуст — "None".
    \item Создайте метод insert в классе DoublyLinkedList, который принимает значение и вставляет его \textbf{в позицию, равную значению по модулю длины списка} (если список не пуст; если пуст — вставляет как первый). Например, при длине 3 и значении 7: 7 \% 3 = 1 → вставка на позицию 1 (второй элемент).
    \item Создайте метод delete в классе DoublyLinkedList, который принимает значение и удаляет \textbf{первое вхождение}.
    \item Создайте экземпляр класса DoublyLinkedList.
    \item Вставьте узлы: 2, 4, 6.
    \item Вызовите display.
    \item Вставьте 5 (5 \% 3 = 2 → вставка на позицию 2, т.е. после 4, перед 6).
    \item Снова вызовите display.
    \item Удалите 4.
    \item Снова вызовите display.
\end{enumerate}

Пример использования:
\begin{lstlisting}[language=Python]
dll = DoublyLinkedList()
dll.insert(2)
dll.insert(4)
dll.insert(6)

print("Initial Doubly Linked List:")
dll.display()

dll.insert(5)
print("After inserting 5 at position 5 % 3 = 2:")
dll.display()

dll.delete(4)
print("After deleting 4:")
dll.display()
\end{lstlisting}

\item Написать программу на Python, которая создает класс DoublyLinkedList, представляющий \textbf{двусвязный список} с инкапсуляцией. Класс должен содержать методы для отображения данных, вставки и удаления узлов. Программа также должна создавать экземпляр класса, вставлять узлы и удалять узлы.

Инструкции:
\begin{enumerate}
    \item Создайте класс Node с методом \_\_init\_\_, который принимает content и сохраняет его в self.\_stored\_value. Инициализирует self.\_connection\_next и self.\_connection\_prev как None.
    \item Создайте класс DoublyLinkedList с методом \_\_init\_\_, который инициализирует self.\_input и self.\_output как None.
    \item Создайте метод display в классе DoublyLinkedList, который выводит элементы в формате: "List: [значения через пробел] (размер: N)". Если пуст — "List: [] (размер: 0)".
    \item Создайте метод insert в классе DoublyLinkedList, который принимает значение и вставляет его \textbf{только если оно не отрицательное}. Вставляет в конец.
    \item Создайте метод delete в классе DoublyLinkedList, который принимает значение и удаляет \textbf{первое вхождение, только если значение положительное}. Если значение <=0 — ничего не делает.
    \item Создайте экземпляр класса DoublyLinkedList.
    \item Вставьте узлы: -1 (игнорируется), 8, 0, 12, -5 (игнорируется).
    \item Вызовите display.
    \item Вставьте 10.
    \item Снова вызовите display.
    \item Удалите 0 (не удаляется, т.к. не положительное).
    \item Снова вызовите display.
\end{enumerate}

Пример использования:
\begin{lstlisting}[language=Python]
dll = DoublyLinkedList()
dll.insert(-1)  # игнорируется
dll.insert(8)
dll.insert(0)
dll.insert(12)
dll.insert(-5)  # игнорируется

print("Initial Doubly Linked List:")
dll.display()

dll.insert(10)
print("After inserting 10:")
dll.display()

dll.delete(0)  # не удаляется
print("After attempting to delete 0:")
dll.display()
\end{lstlisting}

\item Написать программу на Python, которая создает класс DoublyLinkedList, представляющий \textbf{двусвязный список} с инкапсуляцией. Класс должен содержать методы для отображения данных, вставки и удаления узлов. Программа также должна создавать экземпляр класса, вставлять узлы и удалять узлы.

Инструкции:
\begin{enumerate}
    \item Создайте класс Node с методом \_\_init\_\_, который принимает data и сохраняет его в self.\_record. Инициализирует self.\_next\_entry и self.\_prev\_entry как None.
    \item Создайте класс DoublyLinkedList с методом \_\_init\_\_, который инициализирует self.\_head\_record и self.\_tail\_record как None.
    \item Создайте метод display в классе DoublyLinkedList, который выводит элементы в виде: "Записи: val1, val2, ..., valN". Если пуст — "Записей нет".
    \item Создайте метод insert в классе DoublyLinkedList, который принимает значение и вставляет его \textbf{в начало, если значение нечетное, и в конец, если четное}.
    \item Создайте метод delete в классе DoublyLinkedList, который принимает значение и удаляет \textbf{все узлы с этим значением}.
    \item Создайте экземпляр класса DoublyLinkedList.
    \item Вставьте узлы: 3 (в начало), 4 (в конец), 5 (в начало), 6 (в конец).
    \item Вызовите display.
    \item Вставьте 7 (в начало).
    \item Снова вызовите display.
    \item Удалите все вхождения 4.
    \item Снова вызовите display.
\end{enumerate}

Пример использования:
\begin{lstlisting}[language=Python]
dll = DoublyLinkedList()
dll.insert(3)
dll.insert(4)
dll.insert(5)
dll.insert(6)

print("Initial Doubly Linked List:")
dll.display()

dll.insert(7)
print("After inserting 7:")
dll.display()

dll.delete(4)
print("After deleting all 4s:")
dll.display()
\end{lstlisting}

\item Написать программу на Python, которая создает класс DoublyLinkedList, представляющий \textbf{двусвязный список} с инкапсуляцией. Класс должен содержать методы для отображения данных, вставки и удаления узлов. Программа также должна создавать экземпляр класса, вставлять узлы и удалять узлы.

Инструкции:
\begin{enumerate}
    \item Создайте класс Node с методом \_\_init\_\_, который принимает value и сохраняет его в self.\_cell. Инициализирует self.\_cell\_next и self.\_cell\_prev как None.
    \item Создайте класс DoublyLinkedList с методом \_\_init\_\_, который инициализирует self.\_first\_cell и self.\_last\_cell как None.
    \item Создайте метод display в классе DoublyLinkedList, который выводит элементы в виде: "Ячейки: [значения]" и отдельно "Количество: N". Если пуст — "Список ячеек пуст".
    \item Создайте метод insert в классе DoublyLinkedList, который принимает значение и вставляет его \textbf{после каждого узла, значение которого кратно 3} (если таких нет — вставляет в конец).
    \item Создайте метод delete в классе DoublyLinkedList, который принимает значение и удаляет \textbf{первое вхождение}.
    \item Создайте экземпляр класса DoublyLinkedList.
    \item Вставьте узлы: 6, 9, 4.
    \item Вызовите display.
    \item Вставьте 12 (вставится после 6 и после 9 — но по условию вставляется только один узел; вставим после первого кратного 3, т.е. после 6).
    \item Снова вызовите display.
    \item Удалите 9.
    \item Снова вызовите display.
\end{enumerate}

Пример использования:
\begin{lstlisting}[language=Python]
dll = DoublyLinkedList()
dll.insert(6)
dll.insert(9)
dll.insert(4)

print("Initial Doubly Linked List:")
dll.display()

dll.insert(12)
print("After inserting 12 after first multiple of 3:")
dll.display()

dll.delete(9)
print("After deleting 9:")
dll.display()
\end{lstlisting}

\item Написать программу на Python, которая создает класс DoublyLinkedList, представляющий \textbf{двусвязный список} с инкапсуляцией. Класс должен содержать методы для отображения данных, вставки и удаления узлов. Программа также должна создавать экземпляр класса, вставлять узлы и удалять узлы.

Инструкции:
\begin{enumerate}
    \item Создайте класс Node с методом \_\_init\_\_, который принимает item и сохраняет его в self.\_slot. Инициализирует self.\_slot\_next и self.\_slot\_prev как None.
    \item Создайте класс DoublyLinkedList с методом \_\_init\_\_, который инициализирует self.\_start\_slot и self.\_end\_slot как None.
    \item Создайте метод display в классе DoublyLinkedList, который выводит элементы в виде: "Слоты: val1 | val2 | val3". Если пуст — "Слоты отсутствуют".
    \item Создайте метод insert в классе DoublyLinkedList, который принимает значение и вставляет его \textbf{перед каждым узлом, значение которого кратно 5} (если таких нет — вставляет в начало).
    \item Создайте метод delete в классе DoublyLinkedList, который принимает значение и удаляет \textbf{последнее вхождение}.
    \item Создайте экземпляр класса DoublyLinkedList.
    \item Вставьте узлы: 10, 15, 7.
    \item Вызовите display.
    \item Вставьте 5 (вставится перед 10 и перед 15 — но по условию вставляется только один узел; вставим перед первым кратным 5, т.е. перед 10).
    \item Снова вызовите display.
    \item Удалите последнее вхождение 15.
    \item Снова вызовите display.
\end{enumerate}

Пример использования:
\begin{lstlisting}[language=Python]
dll = DoublyLinkedList()
dll.insert(10)
dll.insert(15)
dll.insert(7)

print("Initial Doubly Linked List:")
dll.display()

dll.insert(5)
print("After inserting 5 before first multiple of 5:")
dll.display()

dll.delete(15)
print("After deleting last occurrence of 15:")
dll.display()
\end{lstlisting}

\item Написать программу на Python, которая создает класс DoublyLinkedList, представляющий \textbf{двусвязный список} с инкапсуляцией. Класс должен содержать методы для отображения данных, вставки и удаления узлов. Программа также должна создавать экземпляр класса, вставлять узлы и удалять узлы.

Инструкции:
\begin{enumerate}
    \item Создайте класс Node с методом \_\_init\_\_, который принимает data и сохраняет его в self.\_block. Инициализирует self.\_block\_next и self.\_block\_prev как None.
    \item Создайте класс DoublyLinkedList с методом \_\_init\_\_, который инициализирует self.\_head\_block и self.\_tail\_block как None.
    \item Создайте метод display в классе DoublyLinkedList, который выводит элементы в виде: "Блоки: [значения]" и "Обратный порядок: [значения в обратном порядке]". Если пуст — "Нет блоков".
    \item Создайте метод insert в классе DoublyLinkedList, который принимает значение и вставляет его \textbf{только если сумма цифр значения четная}. Вставляет в конец.
    \item Создайте метод delete в классе DoublyLinkedList, который принимает значение и удаляет \textbf{все вхождения}.
    \item Создайте экземпляр класса DoublyLinkedList.
    \item Вставьте узлы: 23 (2+3=5 — нечет, не вставляется), 24 (2+4=6 — чет, вставляется), 35 (3+5=8 — чет, вставляется), 13 (1+3=4 — чет, вставляется).
    \item Вызовите display.
    \item Вставьте 46 (4+6=10 — чет, вставляется).
    \item Снова вызовите display.
    \item Удалите все вхождения 24.
    \item Снова вызовите display.
\end{enumerate}

Пример использования:
\begin{lstlisting}[language=Python]
dll = DoublyLinkedList()
dll.insert(23)  # не вставляется
dll.insert(24)
dll.insert(35)
dll.insert(13)

print("Initial Doubly Linked List:")
dll.display()

dll.insert(46)
print("After inserting 46:")
dll.display()

dll.delete(24)
print("After deleting all 24s:")
dll.display()
\end{lstlisting}

\item Написать программу на Python, которая создает класс DoublyLinkedList, представляющий \textbf{двусвязный список} с инкапсуляцией. Класс должен содержать методы для отображения данных, вставки и удаления узлов. Программа также должна создавать экземпляр класса, вставлять узлы и удалять узлы.

Инструкции:
\begin{enumerate}
    \item Создайте класс Node с методом \_\_init\_\_, который принимает value и сохраняет его в self.\_unit. Инициализирует self.\_unit\_next и self.\_unit\_prev как None.
    \item Создайте класс DoublyLinkedList с методом \_\_init\_\_, который инициализирует self.\_first\_unit и self.\_last\_unit как None.
    \item Создайте метод display в классе DoublyLinkedList, который выводит элементы в виде: "Единицы: val1 → val2 → val3 → null". Если пуст — "null".
    \item Создайте метод insert в классе DoublyLinkedList, который принимает значение и вставляет его \textbf{только если оно простое число} (используйте вспомогательную функцию is\_prime). Вставляет в начало.
    \item Создайте метод delete в классе DoublyLinkedList, который принимает значение и удаляет \textbf{первое вхождение}.
    \item Создайте вспомогательную функцию is\_prime(n).
    \item Создайте экземпляр класса DoublyLinkedList.
    \item Вставьте узлы: 4 (не простое), 5 (простое), 6 (не простое), 7 (простое), 8 (не простое), 11 (простое).
    \item Вызовите display.
    \item Вставьте 13 (простое).
    \item Снова вызовите display.
    \item Удалите 7.
    \item Снова вызовите display.
\end{enumerate}

Пример использования:
\begin{lstlisting}[language=Python]
def is_prime(n):
    if n < 2:
        return False
    for i in range(2, int(n**0.5)+1):
        if n % i == 0:
            return False
    return True

dll = DoublyLinkedList()
dll.insert(4)   # нет
dll.insert(5)   # да
dll.insert(6)   # нет
dll.insert(7)   # да
dll.insert(8)   # нет
dll.insert(11)  # да

print("Initial Doubly Linked List:")
dll.display()

dll.insert(13)
print("After inserting 13:")
dll.display()

dll.delete(7)
print("After deleting 7:")
dll.display()
\end{lstlisting}

\item Написать программу на Python, которая создает класс DoublyLinkedList, представляющий \textbf{двусвязный список} с инкапсуляцией. Класс должен содержать методы для отображения данных, вставки и удаления узлов. Программа также должна создавать экземпляр класса, вставлять узлы и удалять узлы.

Инструкции:
\begin{enumerate}
    \item Создайте класс Node с методом \_\_init\_\_, который принимает item и сохраняет его в self.\_segment. Инициализирует self.\_seg\_next и self.\_seg\_prev как None.
    \item Создайте класс DoublyLinkedList с методом \_\_init\_\_, который инициализирует self.\_head\_seg и self.\_tail\_seg как None.
    \item Создайте метод display в классе DoublyLinkedList, который выводит элементы в виде: "Сегменты (вперед): ...", "Сегменты (назад): ...". Если пуст — "Список сегментов пуст".
    \item Создайте метод insert в классе DoublyLinkedList, который принимает значение и вставляет его \textbf{только если оно палиндром} (например, 121, 33). Вставляет в конец.
    \item Создайте метод delete в классе DoublyLinkedList, который принимает значение и удаляет \textbf{последнее вхождение}.
    \item Создайте экземпляр класса DoublyLinkedList.
    \item Вставьте узлы: 12 (не палиндром), 22 (палиндром), 34 (не палиндром), 55 (палиндром), 121 (палиндром).
    \item Вызовите display.
    \item Вставьте 33 (палиндром).
    \item Снова вызовите display.
    \item Удалите последнее вхождение 55.
    \item Снова вызовите display.
\end{enumerate}

Пример использования:
\begin{lstlisting}[language=Python]
dll = DoublyLinkedList()
dll.insert(12)  # нет
dll.insert(22)  # да
dll.insert(34)  # нет
dll.insert(55)  # да
dll.insert(121) # да

print("Initial Doubly Linked List:")
dll.display()

dll.insert(33)
print("After inserting 33:")
dll.display()

dll.delete(55)
print("After deleting last occurrence of 55:")
dll.display()
\end{lstlisting}

\item Написать программу на Python, которая создает класс DoublyLinkedList, представляющий \textbf{двусвязный список} с инкапсуляцией. Класс должен содержать методы для отображения данных, вставки и удаления узлов. Программа также должна создавать экземпляр класса, вставлять узлы и удалять узлы.

Инструкции:
\begin{enumerate}
    \item Создайте класс Node с методом \_\_init\_\_, который принимает data и сохраняет его в self.\_piece. Инициализирует self.\_piece\_next и self.\_piece\_prev как None.
    \item Создайте класс DoublyLinkedList с методом \_\_init\_\_, который инициализирует self.\_first\_piece и self.\_last\_piece как None.
    \item Создайте метод display в классе DoublyLinkedList, который выводит элементы в виде: "Части: val1 - val2 - val3". Если пуст — "Нет частей".
    \item Создайте метод insert в классе DoublyLinkedList, который принимает значение и вставляет его \textbf{только если оно степень двойки} (1,2,4,8,16...). Вставляет в начало.
    \item Создайте метод delete в классе DoublyLinkedList, который принимает значение и удаляет \textbf{все вхождения}.
    \item Создайте экземпляр класса DoublyLinkedList.
    \item Вставьте узлы: 3 (нет), 4 (да), 5 (нет), 8 (да), 9 (нет), 16 (да).
    \item Вызовите display.
    \item Вставьте 32 (да).
    \item Снова вызовите display.
    \item Удалите все вхождения 8.
    \item Снова вызовите display.
\end{enumerate}

Пример использования:
\begin{lstlisting}[language=Python]
dll = DoublyLinkedList()
dll.insert(3)   # нет
dll.insert(4)   # да
dll.insert(5)   # нет
dll.insert(8)   # да
dll.insert(9)   # нет
dll.insert(16)  # да

print("Initial Doubly Linked List:")
dll.display()

dll.insert(32)
print("After inserting 32:")
dll.display()

dll.delete(8)
print("After deleting all 8s:")
dll.display()
\end{lstlisting}

\item Написать программу на Python, которая создает класс DoublyLinkedList, представляющий \textbf{двусвязный список} с инкапсуляцией. Класс должен содержать методы для отображения данных, вставки и удаления узлов. Программа также должна создавать экземпляр класса, вставлять узлы и удалять узлы.

Инструкции:
\begin{enumerate}
    \item Создайте класс Node с методом \_\_init\_\_, который принимает value и сохраняет его в self.\_fragment. Инициализирует self.\_frag\_next и self.\_frag\_prev как None.
    \item Создайте класс DoublyLinkedList с методом \_\_init\_\_, который инициализирует self.\_start\_frag и self.\_end\_frag как None.
    \item Создайте метод display в классе DoublyLinkedList, который выводит элементы в виде: "Фрагменты → val1 → val2 → val3 → конец". Если пуст — "Фрагменты: конец".
    \item Создайте метод insert в классе DoublyLinkedList, который принимает значение и вставляет его \textbf{только если оно делится на 3 без остатка}. Вставляет в конец.
    \item Создайте метод delete в классе DoublyLinkedList, который принимает значение и удаляет \textbf{первое вхождение}.
    \item Создайте экземпляр класса DoublyLinkedList.
    \item Вставьте узлы: 1 (нет), 3 (да), 4 (нет), 6 (да), 7 (нет), 9 (да).
    \item Вызовите display.
    \item Вставьте 12 (да).
    \item Снова вызовите display.
    \item Удалите 6.
    \item Снова вызовите display.
\end{enumerate}

Пример использования:
\begin{lstlisting}[language=Python]
dll = DoublyLinkedList()
dll.insert(1)  # нет
dll.insert(3)  # да
dll.insert(4)  # нет
dll.insert(6)  # да
dll.insert(7)  # нет
dll.insert(9)  # да

print("Initial Doubly Linked List:")
dll.display()

dll.insert(12)
print("After inserting 12:")
dll.display()

dll.delete(6)
print("After deleting 6:")
dll.display()
\end{lstlisting}

\item Написать программу на Python, которая создает класс DoublyLinkedList, представляющий \textbf{двусвязный список} с инкапсуляцией. Класс должен содержать методы для отображения данных, вставки и удаления узлов. Программа также должна создавать экземпляр класса, вставлять узлы и удалять узлы.

Инструкции:
\begin{enumerate}
    \item Создайте класс Node с методом \_\_init\_\_, который принимает item и сохраняет его в self.\_chunk. Инициализирует self.\_chunk\_next и self.\_chunk\_prev как None.
    \item Создайте класс DoublyLinkedList с методом \_\_init\_\_, который инициализирует self.\_head\_chunk и self.\_tail\_chunk как None.
    \item Создайте метод display в классе DoublyLinkedList, который выводит элементы в виде: "Чанки: [значения]" и "Размер: N". Если пуст — "Чанков нет".
    \item Создайте метод insert в классе DoublyLinkedList, который принимает значение и вставляет его \textbf{только если оно не делится на 5}. Вставляет в начало.
    \item Создайте метод delete в классе DoublyLinkedList, который принимает значение и удаляет \textbf{последнее вхождение}.
    \item Создайте экземпляр класса DoublyLinkedList.
    \item Вставьте узлы: 10 (делится на 5 — не вставляется), 11 (не делится — вставляется), 15 (делится — не вставляется), 16 (не делится — вставляется), 20 (делится — не вставляется), 21 (не делится — вставляется).
    \item Вызовите display.
    \item Вставьте 26 (не делится — вставляется).
    \item Снова вызовите display.
    \item Удалите последнее вхождение 16.
    \item Снова вызовите display.
\end{enumerate}

Пример использования:
\begin{lstlisting}[language=Python]
dll = DoublyLinkedList()
dll.insert(10)  # нет
dll.insert(11)  # да
dll.insert(15)  # нет
dll.insert(16)  # да
dll.insert(20)  # нет
dll.insert(21)  # да

print("Initial Doubly Linked List:")
dll.display()

dll.insert(26)
print("After inserting 26:")
dll.display()

dll.delete(16)
print("After deleting last occurrence of 16:")
dll.display()
\end{lstlisting}

\item Написать программу на Python, которая создает класс DoublyLinkedList, представляющий \textbf{двусвязный список} с инкапсуляцией. Класс должен содержать методы для отображения данных, вставки и удаления узлов. Программа также должна создавать экземпляр класса, вставлять узлы и удалять узлы.

Инструкции:
\begin{enumerate}
    \item Создайте класс Node с методом \_\_init\_\_, который принимает data и сохраняет его в self.\_item\_data. Инициализирует self.\_next\_item и self.\_prev\_item как None.
    \item Создайте класс DoublyLinkedList с методом \_\_init\_\_, который инициализирует self.\_first\_data и self.\_last\_data как None.
    \item Создайте метод display в классе DoublyLinkedList, который выводит элементы в виде: "Данные (→): val1, val2, val3" и "Данные (←): val3, val2, val1". Если пуст — "Данные отсутствуют".
    \item Создайте метод insert в классе DoublyLinkedList, который принимает значение и вставляет его \textbf{только если оно больше 10}. Вставляет в конец.
    \item Создайте метод delete в классе DoublyLinkedList, который принимает значение и удаляет \textbf{все вхождения}.
    \item Создайте экземпляр класса DoublyLinkedList.
    \item Вставьте узлы: 5 (нет), 15 (да), 8 (нет), 20 (да), 12 (да).
    \item Вызовите display.
    \item Вставьте 25 (да).
    \item Снова вызовите display.
    \item Удалите все вхождения 20.
    \item Снова вызовите display.
\end{enumerate}

Пример использования:
\begin{lstlisting}[language=Python]
dll = DoublyLinkedList()
dll.insert(5)   # нет
dll.insert(15)  # да
dll.insert(8)   # нет
dll.insert(20)  # да
dll.insert(12)  # да

print("Initial Doubly Linked List:")
dll.display()

dll.insert(25)
print("After inserting 25:")
dll.display()

dll.delete(20)
print("After deleting all 20s:")
dll.display()
\end{lstlisting}

\item Написать программу на Python, которая создает класс DoublyLinkedList, представляющий \textbf{двусвязный список} с инкапсуляцией. Класс должен содержать методы для отображения данных, вставки и удаления узлов. Программа также должна создавать экземпляр класса, вставлять узлы и удалять узлы.

Инструкции:
\begin{enumerate}
    \item Создайте класс Node с методом \_\_init\_\_, который принимает value и сохраняет его в self.\_node\_value. Инициализирует self.\_node\_next и self.\_node\_prev как None.
    \item Создайте класс DoublyLinkedList с методом \_\_init\_\_, который инициализирует self.\_start\_node и self.\_end\_node как None.
    \item Создайте метод display в классе DoublyLinkedList, который выводит элементы в виде: "Узлы: val1 <-> val2 <-> val3". Если пуст — "Нет узлов".
    \item Создайте метод insert в классе DoublyLinkedList, который принимает значение и вставляет его \textbf{только если оно меньше 50}. Вставляет в начало.
    \item Создайте метод delete в классе DoublyLinkedList, который принимает значение и удаляет \textbf{первое вхождение}.
    \item Создайте экземпляр класса DoublyLinkedList.
    \item Вставьте узлы: 60 (нет), 30 (да), 70 (нет), 40 (да), 45 (да).
    \item Вызовите display.
    \item Вставьте 25 (да).
    \item Снова вызовите display.
    \item Удалите 40.
    \item Снова вызовите display.
\end{enumerate}

Пример использования:
\begin{lstlisting}[language=Python]
dll = DoublyLinkedList()
dll.insert(60)  # нет
dll.insert(30)  # да
dll.insert(70)  # нет
dll.insert(40)  # да
dll.insert(45)  # да

print("Initial Doubly Linked List:")
dll.display()

dll.insert(25)
print("After inserting 25:")
dll.display()

dll.delete(40)
print("After deleting 40:")
dll.display()
\end{lstlisting}

\item Написать программу на Python, которая создает класс DoublyLinkedList, представляющий \textbf{двусвязный список} с инкапсуляцией. Класс должен содержать методы для отображения данных, вставки и удаления узлов. Программа также должна создавать экземпляр класса, вставлять узлы и удалять узлы.

Инструкции:
\begin{enumerate}
    \item Создайте класс Node с методом \_\_init\_\_, который принимает item и сохраняет его в self.\_data\_item. Инициализирует self.\_item\_next и self.\_item\_prev как None.
    \item Создайте класс DoublyLinkedList с методом \_\_init\_\_, который инициализирует self.\_head\_item и self.\_tail\_item как None.
    \item Создайте метод display в классе DoublyLinkedList, который выводит элементы в виде: "Элементы списка: val1 val2 val3 (всего N)". Если пуст — "Список пуст".
    \item Создайте метод insert в классе DoublyLinkedList, который принимает значение и вставляет его \textbf{только если оно не равно 0}. Вставляет в конец.
    \item Создайте метод delete в классе DoublyLinkedList, который принимает значение и удаляет \textbf{последнее вхождение}.
    \item Создайте экземпляр класса DoublyLinkedList.
    \item Вставьте узлы: 0 (нет), 10 (да), 0 (нет), 20 (да), 30 (да).
    \item Вызовите display.
    \item Вставьте 40 (да).
    \item Снова вызовите display.
    \item Удалите последнее вхождение 20.
    \item Снова вызовите display.
\end{enumerate}

Пример использования:
\begin{lstlisting}[language=Python]
dll = DoublyLinkedList()
dll.insert(0)   # нет
dll.insert(10)  # да
dll.insert(0)   # нет
dll.insert(20)  # да
dll.insert(30)  # да

print("Initial Doubly Linked List:")
dll.display()

dll.insert(40)
print("After inserting 40:")
dll.display()

dll.delete(20)
print("After deleting last occurrence of 20:")
dll.display()
\end{lstlisting}

\item Написать программу на Python, которая создает класс DoublyLinkedList, представляющий \textbf{двусвязный список} с инкапсуляцией. Класс должен содержать методы для отображения данных, вставки и удаления узлов. Программа также должна создавать экземпляр класса, вставлять узлы и удалять узлы.

Инструкции:
\begin{enumerate}
    \item Создайте класс Node с методом \_\_init\_\_, который принимает data и сохраняет его в self.\_list\_data. Инициализирует self.\_data\_next и self.\_data\_prev как None.
    \item Создайте класс DoublyLinkedList с методом \_\_init\_\_, который инициализирует self.\_first\_list и self.\_last\_list как None.
    \item Создайте метод display в классе DoublyLinkedList, который выводит элементы в виде: "Список: val1 | val2 | val3 | ...". Если пуст — "Пустой список".
    \item Создайте метод insert в классе DoublyLinkedList, который принимает значение и вставляет его \textbf{только если оно положительное}. Вставляет в начало.
    \item Создайте метод delete в классе DoublyLinkedList, который принимает значение и удаляет \textbf{все вхождения}.
    \item Создайте экземпляр класса DoublyLinkedList.
    \item Вставьте узлы: -5 (нет), 15 (да), -3 (нет), 25 (да), 0 (нет, если считать 0 не положительным).
    \item Вызовите display.
    \item Вставьте 35 (да).
    \item Снова вызовите display.
    \item Удалите все вхождения 25.
    \item Снова вызовите display.
\end{enumerate}

Пример использования:
\begin{lstlisting}[language=Python]
dll = DoublyLinkedList()
dll.insert(-5)  # нет
dll.insert(15)  # да
dll.insert(-3)  # нет
dll.insert(25)  # да
dll.insert(0)   # нет

print("Initial Doubly Linked List:")
dll.display()

dll.insert(35)
print("After inserting 35:")
dll.display()

dll.delete(25)
print("After deleting all 25s:")
dll.display()
\end{lstlisting}

\item Написать программу на Python, которая создает класс DoublyLinkedList, представляющий \textbf{двусвязный список} с инкапсуляцией. Класс должен содержать методы для отображения данных, вставки и удаления узлов. Программа также должна создавать экземпляр класса, вставлять узлы и удалять узлы.

Инструкции:
\begin{enumerate}
    \item Создайте класс Node с методом \_\_init\_\_, который принимает value и сохраняет его в self.\_entry\_value. Инициализирует self.\_value\_next и self.\_value\_prev как None.
    \item Создайте класс DoublyLinkedList с методом \_\_init\_\_, который инициализирует self.\_head\_value и self.\_tail\_value как None.
    \item Создайте метод display в классе DoublyLinkedList, который выводит элементы в виде: "Значения → val1 → val2 → val3 → конец". Если пуст — "→ конец".
    \item Создайте метод insert в классе DoublyLinkedList, который принимает значение и вставляет его \textbf{только если оно нечетное}. Вставляет в конец.
    \item Создайте метод delete в классе DoublyLinkedList, который принимает значение и удаляет \textbf{первое вхождение}.
    \item Создайте экземпляр класса DoublyLinkedList.
    \item Вставьте узлы: 2 (нет), 3 (да), 4 (нет), 5 (да), 6 (нет), 7 (да).
    \item Вызовите display.
    \item Вставьте 9 (да).
    \item Снова вызовите display.
    \item Удалите 5.
    \item Снова вызовите display.
\end{enumerate}

Пример использования:
\begin{lstlisting}[language=Python]
dll = DoublyLinkedList()
dll.insert(2)  # нет
dll.insert(3)  # да
dll.insert(4)  # нет
dll.insert(5)  # да
dll.insert(6)  # нет
dll.insert(7)  # да

print("Initial Doubly Linked List:")
dll.display()

dll.insert(9)
print("After inserting 9:")
dll.display()

dll.delete(5)
print("After deleting 5:")
dll.display()
\end{lstlisting}

\item Написать программу на Python, которая создает класс DoublyLinkedList, представляющий \textbf{двусвязный список} с инкапсуляцией. Класс должен содержать методы для отображения данных, вставки и удаления узлов. Программа также должна создавать экземпляр класса, вставлять узлы и удалять узлы.

Инструкции:
\begin{enumerate}
    \item Создайте класс Node с методом \_\_init\_\_, который принимает item и сохраняет его в self.\_data\_point. Инициализирует self.\_point\_next и self.\_point\_prev как None.
    \item Создайте класс DoublyLinkedList с методом \_\_init\_\_, который инициализирует self.\_start\_point и self.\_end\_point как None.
    \item Создайте метод display в классе DoublyLinkedList, который выводит элементы в виде: "Точки: val1, val2, val3 (обратно: val3, val2, val1)". Если пуст — "Точек нет".
    \item Создайте метод insert в классе DoublyLinkedList, который принимает значение и вставляет его \textbf{только если оно четное}. Вставляет в начало.
    \item Создайте метод delete в классе DoublyLinkedList, который принимает значение и удаляет \textbf{последнее вхождение}.
    \item Создайте экземпляр класса DoublyLinkedList.
    \item Вставьте узлы: 1 (нет), 4 (да), 3 (нет), 6 (да), 5 (нет), 8 (да).
    \item Вызовите display.
    \item Вставьте 10 (да).
    \item Снова вызовите display.
    \item Удалите последнее вхождение 6.
    \item Снова вызовите display.
\end{enumerate}

Пример использования:
\begin{lstlisting}[language=Python]
dll = DoublyLinkedList()
dll.insert(1)  # нет
dll.insert(4)  # да
dll.insert(3)  # нет
dll.insert(6)  # да
dll.insert(5)  # нет
dll.insert(8)  # да

print("Initial Doubly Linked List:")
dll.display()

dll.insert(10)
print("After inserting 10:")
dll.display()

dll.delete(6)
print("After deleting last occurrence of 6:")
dll.display()
\end{lstlisting}

\item Написать программу на Python, которая создает класс DoublyLinkedList, представляющий \textbf{двусвязный список} с инкапсуляцией. Класс должен содержать методы для отображения данных, вставки и удаления узлов. Программа также должна создавать экземпляр класса, вставлять узлы и удалять узлы.

Инструкции:
\begin{enumerate}
    \item Создайте класс Node с методом \_\_init\_\_, который принимает data и сохраняет его в self.\_node\_data. Инициализирует self.\_data\_link\_next и self.\_data\_link\_prev как None.
    \item Создайте класс DoublyLinkedList с методом \_\_init\_\_, который инициализирует self.\_first\_link и self.\_last\_link как None.
    \item Создайте метод display в классе DoublyLinkedList, который выводит элементы в виде: "Связи: val1 <-> val2 <-> val3". Если пуст — "Связи отсутствуют".
    \item Создайте метод insert в классе DoublyLinkedList, который принимает значение и вставляет его \textbf{только если оно кратно 4}. Вставляет в конец.
    \item Создайте метод delete в классе DoublyLinkedList, который принимает значение и удаляет \textbf{все вхождения}.
    \item Создайте экземпляр класса DoublyLinkedList.
    \item Вставьте узлы: 2 (нет), 4 (да), 6 (нет), 8 (да), 10 (нет), 12 (да).
    \item Вызовите display.
    \item Вставьте 16 (да).
    \item Снова вызовите display.
    \item Удалите все вхождения 8.
    \item Снова вызовите display.
\end{enumerate}

Пример использования:
\begin{lstlisting}[language=Python]
dll = DoublyLinkedList()
dll.insert(2)   # нет
dll.insert(4)   # да
dll.insert(6)   # нет
dll.insert(8)   # да
dll.insert(10)  # нет
dll.insert(12)  # да

print("Initial Doubly Linked List:")
dll.display()

dll.insert(16)
print("After inserting 16:")
dll.display()

dll.delete(8)
print("After deleting all 8s:")
dll.display()
\end{lstlisting}

\item Написать программу на Python, которая создает класс DoublyLinkedList, представляющий \textbf{двусвязный список} с инкапсуляцией. Класс должен содержать методы для отображения данных, вставки и удаления узлов. Программа также должна создавать экземпляр класса, вставлять узлы и удалять узлы.

Инструкции:
\begin{enumerate}
    \item Создайте класс Node с методом \_\_init\_\_, который принимает value и сохраняет его в self.\_item\_val. Инициализирует self.\_val\_next и self.\_val\_prev как None.
    \item Создайте класс DoublyLinkedList с методом \_\_init\_\_, который инициализирует self.\_head\_val и self.\_tail\_val как None.
    \item Создайте метод display в классе DoublyLinkedList, который выводит элементы в виде: "Значения: val1 - val2 - val3 (размер N)". Если пуст — "Нет значений".
    \item Создайте метод insert в классе DoublyLinkedList, который принимает значение и вставляет его \textbf{только если оно заканчивается на 5}. Вставляет в начало.
    \item Создайте метод delete в классе DoublyLinkedList, который принимает значение и удаляет \textbf{первое вхождение}.
    \item Создайте экземпляр класса DoublyLinkedList.
    \item Вставьте узлы: 10 (нет), 15 (да), 20 (нет), 25 (да), 30 (нет), 35 (да).
    \item Вызовите display.
    \item Вставьте 45 (да).
    \item Снова вызовите display.
    \item Удалите 25.
    \item Снова вызовите display.
\end{enumerate}

Пример использования:
\begin{lstlisting}[language=Python]
dll = DoublyLinkedList()
dll.insert(10)  # нет
dll.insert(15)  # да
dll.insert(20)  # нет
dll.insert(25)  # да
dll.insert(30)  # нет
dll.insert(35)  # да

print("Initial Doubly Linked List:")
dll.display()

dll.insert(45)
print("After inserting 45:")
dll.display()

dll.delete(25)
print("After deleting 25:")
dll.display()
\end{lstlisting}

\item Написать программу на Python, которая создает класс DoublyLinkedList, представляющий \textbf{двусвязный список} с инкапсуляцией. Класс должен содержать методы для отображения данных, вставки и удаления узлов. Программа также должна создавать экземпляр класса, вставлять узлы и удалять узлы.

Инструкции:
\begin{enumerate}
    \item Создайте класс Node с методом \_\_init\_\_, который принимает item и сохраняет его в self.\_data\_field. Инициализирует self.\_field\_next и self.\_field\_prev как None.
    \item Создайте класс DoublyLinkedList с методом \_\_init\_\_, который инициализирует self.\_first\_field и self.\_last\_field как None.
    \item Создайте метод display в классе DoublyLinkedList, который выводит элементы в виде: "Поля: val1 → val2 → val3 → null". Если пуст — "null".
    \item Создайте метод insert в классе DoublyLinkedList, который принимает значение и вставляет его \textbf{только если первая цифра числа — 1}. Вставляет в конец.
    \item Создайте метод delete в классе DoublyLinkedList, который принимает значение и удаляет \textbf{последнее вхождение}.
    \item Создайте экземпляр класса DoublyLinkedList.
    \item Вставьте узлы: 5 (нет), 12 (да), 23 (нет), 18 (да), 31 (нет), 19 (да).
    \item Вызовите display.
    \item Вставьте 11 (да).
    \item Снова вызовите display.
    \item Удалите последнее вхождение 18.
    \item Снова вызовите display.
\end{enumerate}

Пример использования:
\begin{lstlisting}[language=Python]
dll = DoublyLinkedList()
dll.insert(5)   # нет
dll.insert(12)  # да
dll.insert(23)  # нет
dll.insert(18)  # да
dll.insert(31)  # нет
dll.insert(19)  # да

print("Initial Doubly Linked List:")
dll.display()

dll.insert(11)
print("After inserting 11:")
dll.display()

dll.delete(18)
print("After deleting last occurrence of 18:")
dll.display()
\end{lstlisting}

\item Написать программу на Python, которая создает класс DoublyLinkedList, представляющий \textbf{двусвязный список} с инкапсуляцией. Класс должен содержать методы для отображения данных, вставки и удаления узлов. Программа также должна создавать экземпляр класса, вставлять узлы и удалять узлы.

Инструкции:
\begin{enumerate}
    \item Создайте класс Node с методом \_\_init\_\_, который принимает data и сохраняет его в self.\_record\_data. Инициализирует self.\_data\_record\_next и self.\_data\_record\_prev как None.
    \item Создайте класс DoublyLinkedList с методом \_\_init\_\_, который инициализирует self.\_head\_record и self.\_tail\_record как None.
    \item Создайте метод display в классе DoublyLinkedList, который выводит элементы в виде: "Записи: [val1, val2, val3]". Если пуст — "[]".
    \item Создайте метод insert в классе DoublyLinkedList, который принимает значение и вставляет его \textbf{только если оно начинается с цифры 2}. Вставляет в начало.
    \item Создайте метод delete в классе DoublyLinkedList, который принимает значение и удаляет \textbf{все вхождения}.
    \item Создайте экземпляр класса DoublyLinkedList.
    \item Вставьте узлы: 15 (нет), 25 (да), 35 (нет), 28 (да), 45 (нет), 22 (да).
    \item Вызовите display.
    \item Вставьте 20 (да).
    \item Снова вызовите display.
    \item Удалите все вхождения 28.
    \item Снова вызовите display.
\end{enumerate}

Пример использования:
\begin{lstlisting}[language=Python]
dll = DoublyLinkedList()
dll.insert(15)  # нет
dll.insert(25)  # да
dll.insert(35)  # нет
dll.insert(28)  # да
dll.insert(45)  # нет
dll.insert(22)  # да

print("Initial Doubly Linked List:")
dll.display()

dll.insert(20)
print("After inserting 20:")
dll.display()

dll.delete(28)
print("After deleting all 28s:")
dll.display()
\end{lstlisting}

\item Написать программу на Python, которая создает класс DoublyLinkedList, представляющий \textbf{двусвязный список} с инкапсуляцией. Класс должен содержать методы для отображения данных, вставки и удаления узлов. Программа также должна создавать экземпляр класса, вставлять узлы и удалять узлы.

Инструкции:
\begin{enumerate}
    \item Создайте класс Node с методом \_\_init\_\_, который принимает value и сохраняет его в self.\_cell\_value. Инициализирует self.\_value\_cell\_next и self.\_value\_cell\_prev как None.
    \item Создайте класс DoublyLinkedList с методом \_\_init\_\_, который инициализирует self.\_first\_cell и self.\_last\_cell как None.
    \item Создайте метод display в классе DoublyLinkedList, который выводит элементы в виде: "Ячейки: val1 | val2 | val3 (всего N)". Если пуст — "Нет ячеек".
    \item Создайте метод insert в классе DoublyLinkedList, который принимает значение и вставляет его \textbf{только если сумма его цифр нечетная}. Вставляет в конец.
    \item Создайте метод delete в классе DoublyLinkedList, который принимает значение и удаляет \textbf{первое вхождение}.
    \item Создайте экземпляр класса DoublyLinkedList.
    \item Вставьте узлы: 12 (1+2=3 — нечет, да), 14 (1+4=5 — нечет, да), 16 (1+6=7 — нечет, да), 18 (1+8=9 — нечет, да), 20 (2+0=2 — чет, нет).
    \item Вызовите display.
    \item Вставьте 21 (2+1=3 — нечет, да).
    \item Снова вызовите display.
    \item Удалите 16.
    \item Снова вызовите display.
\end{enumerate}

Пример использования:
\begin{lstlisting}[language=Python]
dll = DoublyLinkedList()
dll.insert(12)  # да
dll.insert(14)  # да
dll.insert(16)  # да
dll.insert(18)  # да
dll.insert(20)  # нет

print("Initial Doubly Linked List:")
dll.display()

dll.insert(21)
print("After inserting 21:")
dll.display()

dll.delete(16)
print("After deleting 16:")
dll.display()
\end{lstlisting}

\item Написать программу на Python, которая создает класс DoublyLinkedList, представляющий \textbf{двусвязный список} с инкапсуляцией. Класс должен содержать методы для отображения данных, вставки и удаления узлов. Программа также должна создавать экземпляр класса, вставлять узлы и удалять узлы.

Инструкции:
\begin{enumerate}
    \item Создайте класс Node с методом \_\_init\_\_, который принимает item и сохраняет его в self.\_slot\_data. Инициализирует self.\_data\_slot\_next и self.\_data\_slot\_prev как None.
    \item Создайте класс DoublyLinkedList с методом \_\_init\_\_, который инициализирует self.\_head\_slot и self.\_tail\_slot как None.
    \item Создайте метод display в классе DoublyLinkedList, который выводит элементы в виде: "Слоты → val1 → val2 → val3 → конец". Если пуст — "→ конец".
    \item Создайте метод insert в классе DoublyLinkedList, который принимает значение и вставляет его \textbf{только если оно заканчивается на 0}. Вставляет в начало.
    \item Создайте метод delete в классе DoublyLinkedList, который принимает значение и удаляет \textbf{последнее вхождение}.
    \item Создайте экземпляр класса DoublyLinkedList.
    \item Вставьте узлы: 5 (нет), 10 (да), 15 (нет), 20 (да), 25 (нет), 30 (да).
    \item Вызовите display.
    \item Вставьте 40 (да).
    \item Снова вызовите display.
    \item Удалите последнее вхождение 20.
    \item Снова вызовите display.
\end{enumerate}

Пример использования:
\begin{lstlisting}[language=Python]
dll = DoublyLinkedList()
dll.insert(5)   # нет
dll.insert(10)  # да
dll.insert(15)  # нет
dll.insert(20)  # да
dll.insert(25)  # нет
dll.insert(30)  # да

print("Initial Doubly Linked List:")
dll.display()

dll.insert(40)
print("After inserting 40:")
dll.display()

dll.delete(20)
print("After deleting last occurrence of 20:")
dll.display()
\end{lstlisting}

\item Написать программу на Python, которая создает класс DoublyLinkedList, представляющий \textbf{двусвязный список} с инкапсуляцией. Класс должен содержать методы для отображения данных, вставки и удаления узлов. Программа также должна создавать экземпляр класса, вставлять узлы и удалять узлы.

Инструкции:
\begin{enumerate}
    \item Создайте класс Node с методом \_\_init\_\_, который принимает data и сохраняет его в self.\_block\_data. Инициализирует self.\_data\_block\_next и self.\_data\_block\_prev как None.
    \item Создайте класс DoublyLinkedList с методом \_\_init\_\_, который инициализирует self.\_first\_block и self.\_last\_block как None.
    \item Создайте метод display в классе DoublyLinkedList, который выводит элементы в виде: "Блоки: val1, val2, val3 (обратный: val3, val2, val1)". Если пуст — "Пусто".
    \item Создайте метод insert в классе DoublyLinkedList, который принимает значение и вставляет его \textbf{только если оно простое и больше 10}. Вставляет в конец.
    \item Создайте метод delete в классе DoublyLinkedList, который принимает значение и удаляет \textbf{все вхождения}.
    \item Создайте экземпляр класса DoublyLinkedList.
    \item Вставьте узлы: 7 (простое, но <=10 — нет), 11 (да), 13 (да), 15 (нет), 17 (да), 9 (нет).
    \item Вызовите display.
    \item Вставьте 19 (да).
    \item Снова вызовите display.
    \item Удалите все вхождения 13.
    \item Снова вызовите display.
\end{enumerate}

Пример использования:
\begin{lstlisting}[language=Python]
def is_prime(n):
    if n < 2:
        return False
    for i in range(2, int(n**0.5)+1):
        if n % i == 0:
            return False
    return True

dll = DoublyLinkedList()
dll.insert(7)   # нет
dll.insert(11)  # да
dll.insert(13)  # да
dll.insert(15)  # нет
dll.insert(17)  # да
dll.insert(9)   # нет

print("Initial Doubly Linked List:")
dll.display()

dll.insert(19)
print("After inserting 19:")
dll.display()

dll.delete(13)
print("After deleting all 13s:")
dll.display()
\end{lstlisting}

\item Написать программу на Python, которая создает класс DoublyLinkedList, представляющий \textbf{двусвязный список} с инкапсуляцией. Класс должен содержать методы для отображения данных, вставки и удаления узлов. Программа также должна создавать экземпляр класса, вставлять узлы и удалять узлы.

Инструкции:
\begin{enumerate}
    \item Создайте класс Node с методом \_\_init\_\_, который принимает value и сохраняет его в self.\_unit\_value. Инициализирует self.\_value\_unit\_next и self.\_value\_unit\_prev как None.
    \item Создайте класс DoublyLinkedList с методом \_\_init\_\_, который инициализирует self.\_head\_unit и self.\_tail\_unit как None.
    \item Создайте метод display в классе DoublyLinkedList, который выводит элементы в виде: "Единицы: val1 <-> val2 <-> val3". Если пуст — "Нет данных".
    \item Создайте метод insert в классе DoublyLinkedList, который принимает значение и вставляет его \textbf{только если оно палиндром и двузначное}. Вставляет в начало.
    \item Создайте метод delete в классе DoublyLinkedList, который принимает значение и удаляет \textbf{первое вхождение}.
    \item Создайте экземпляр класса DoublyLinkedList.
    \item Вставьте узлы: 121 (трехзначное — нет), 22 (да), 34 (нет), 55 (да), 5 (однозначное — нет), 66 (да).
    \item Вызовите display.
    \item Вставьте 77 (да).
    \item Снова вызовите display.
    \item Удалите 55.
    \item Снова вызовите display.
\end{enumerate}

Пример использования:
\begin{lstlisting}[language=Python]
dll = DoublyLinkedList()
dll.insert(121)  # нет
dll.insert(22)   # да
dll.insert(34)   # нет
dll.insert(55)   # да
dll.insert(5)    # нет
dll.insert(66)   # да

print("Initial Doubly Linked List:")
dll.display()

dll.insert(77)
print("After inserting 77:")
dll.display()

dll.delete(55)
print("After deleting 55:")
dll.display()
\end{lstlisting}

\end{enumerate}

\subsubsection{Задача 4 (очередь)}

\begin{enumerate}
\item Написать программу на Python, которая создает класс Queue для представления структуры данных очереди с инкапсуляцией. Класс должен содержать методы enqueue, dequeue и is\_empty, которые реализуют операции добавления элементов в очередь, удаления элементов из очереди и проверки пустоты очереди соответственно. Программа также должна создавать экземпляр класса Queue, добавлять элементы в очередь, удалять элементы из очереди и выводить информацию о состоянии очереди на экран.

Инструкции:
\begin{enumerate}
    \item Создайте класс Queue с методом \_\_init\_\_, который инициализирует пустую очередь (внутренний список \_elements). Принимает необязательный параметр max\_size (по умолчанию None — без ограничений).
    \item Создайте метод enqueue, который принимает элемент и добавляет его в конец очереди, только если не превышает max\_size. Если превышает — выбрасывает ValueError("Очередь переполнена").
    \item Создайте метод dequeue, который удаляет и возвращает элемент из начала очереди. Если очередь пуста — выбрасывает IndexError("Очередь пуста").
    \item Создайте метод is\_empty, который возвращает True, если очередь пуста, и False в противном случае.
    \item Создайте приватный метод \_debug\_list (только для отладки, не включайте в задание студентам; в решении можно использовать queue.\_elements) для вывода внутреннего состояния.
    \item Создайте экземпляр класса Queue с max\_size=5.
    \item Добавьте элементы: 100, 200, 300, 400, 500.
    \item Попытайтесь добавить 600 — должно вызвать исключение (перехватите его и выведите сообщение).
    \item Выведите текущее состояние очереди.
    \item Вызовите dequeue дважды, выводя каждый раз удаленный элемент.
    \item Выведите обновленное состояние очереди.
\end{enumerate}

Пример использования:
\begin{lstlisting}[language=Python]
queue = Queue(max_size=5)
queue.enqueue(100)
queue.enqueue(200)
queue.enqueue(300)
queue.enqueue(400)
queue.enqueue(500)

try:
    queue.enqueue(600)
except ValueError as e:
    print("Ошибка:", e)

print("Current Queue:", queue._elements)  # только для проверки

dequeued_item = queue.dequeue()
print("Dequeued item:", dequeued_item)

dequeued_item = queue.dequeue()
print("Dequeued item:", dequeued_item)

print("Updated Queue:", queue._elements)
\end{lstlisting}

\item Написать программу на Python, которая создает класс Queue для представления структуры данных очереди с инкапсуляцией. Класс должен содержать методы enqueue, dequeue и is\_empty, которые реализуют операции добавления элементов в очередь, удаления элементов из очереди и проверки пустоты очереди соответственно. Программа также должна создавать экземпляр класса Queue, добавлять элементы в очередь, удалять элементы из очереди и выводить информацию о состоянии очереди на экран.

Инструкции:
\begin{enumerate}
    \item Создайте класс Queue с методом \_\_init\_\_, который инициализирует пустую очередь (список \_items). Принимает параметр allow\_duplicates=True. Если False, то не добавляет элемент, если он уже есть в очереди.
    \item Создайте метод enqueue, который принимает элемент. Если allow\_duplicates=False и элемент уже есть в очереди — не добавляет и возвращает False. Иначе — добавляет в конец и возвращает True.
    \item Создайте метод dequeue, который удаляет и возвращает первый элемент. Если очередь пуста — возвращает None (не выбрасывает исключение).
    \item Создайте метод is\_empty, который возвращает True, если очередь пуста, и False в противном случае.
    \item Создайте экземпляр класса Queue с allow\_duplicates=False.
    \item Добавьте элементы: 10, 20, 10 (не добавится), 30, 20 (не добавится), 40.
    \item Выведите текущее состояние очереди.
    \item Вызовите dequeue трижды, выводя каждый раз удаленный элемент.
    \item Выведите обновленное состояние очереди.
\end{enumerate}

Пример использования:
\begin{lstlisting}[language=Python]
queue = Queue(allow_duplicates=False)
print(queue.enqueue(10))  # True
print(queue.enqueue(20))  # True
print(queue.enqueue(10))  # False
print(queue.enqueue(30))  # True
print(queue.enqueue(20))  # False
print(queue.enqueue(40))  # True

print("Current Queue:", queue._items)

for _ in range(3):
    dequeued_item = queue.dequeue()
    print("Dequeued item:", dequeued_item)

print("Updated Queue:", queue._items)
\end{lstlisting}

\item Написать программу на Python, которая создает класс Queue для представления структуры данных очереди с инкапсуляцией. Класс должен содержать методы enqueue, dequeue и is\_empty, которые реализуют операции добавления элементов в очередь, удаления элементов из очереди и проверки пустоты очереди соответственно. Программа также должна создавать экземпляр класса Queue, добавлять элементы в очередь, удалять элементы из очереди и выводить информацию о состоянии очереди на экран.

Инструкции:
\begin{enumerate}
    \item Создайте класс Queue с методом \_\_init\_\_, который инициализирует пустую очередь (список \_data). Принимает параметр auto\_reverse=False. Если True, то enqueue добавляет в начало, а dequeue удаляет с конца (поведение стека, но интерфейс очереди).
    \item Создайте метод enqueue, который добавляет элемент: если auto\_reverse=False — в конец, если True — в начало.
    \item Создайте метод dequeue, который удаляет и возвращает элемент: если auto\_reverse=False — из начала, если True — из конца. Если очередь пуста — выбрасывает IndexError("Пусто").
    \item Создайте метод is\_empty, который возвращает True, если очередь пуста, и False в противном случае.
    \item Создайте экземпляр класса Queue с auto\_reverse=True.
    \item Добавьте элементы: 1, 2, 3, 4, 5.
    \item Выведите текущее состояние очереди.
    \item Вызовите dequeue дважды, выводя каждый раз удаленный элемент.
    \item Выведите обновленное состояние очереди.
\end{enumerate}

Пример использования:
\begin{lstlisting}[language=Python]
queue = Queue(auto_reverse=True)
queue.enqueue(1)
queue.enqueue(2)
queue.enqueue(3)
queue.enqueue(4)
queue.enqueue(5)

print("Current Queue:", queue._data)  # [5,4,3,2,1]

dequeued_item = queue.dequeue()  # удаляет 1
print("Dequeued item:", dequeued_item)

dequeued_item = queue.dequeue()  # удаляет 2
print("Dequeued item:", dequeued_item)

print("Updated Queue:", queue._data)  # [5,4,3]
\end{lstlisting}

\item Написать программу на Python, которая создает класс Queue для представления структуры данных очереди с инкапсуляцией. Класс должен содержать методы enqueue, dequeue и is\_empty, которые реализуют операции добавления элементов в очередь, удаления элементов из очереди и проверки пустоты очереди соответственно. Программа также должна создавать экземпляр класса Queue, добавлять элементы в очередь, удалять элементы из очереди и выводить информацию о состоянии очереди на экран.

Инструкции:
\begin{enumerate}
    \item Создайте класс Queue с методом \_\_init\_\_, который инициализирует пустую очередь (список \_buffer). Принимает параметр dequeue\_all\_at\_once=False. Если True, то dequeue возвращает список всех элементов и очищает очередь.
    \item Создайте метод enqueue, который добавляет элемент в конец очереди.
    \item Создайте метод dequeue, который, если dequeue\_all\_at\_once=False, удаляет и возвращает первый элемент. Если True — возвращает список всех элементов и очищает очередь. Если очередь пуста — возвращает пустой список [].
    \item Создайте метод is\_empty, который возвращает True, если очередь пуста, и False в противном случае.
    \item Создайте экземпляр класса Queue с dequeue\_all\_at\_once=True.
    \item Добавьте элементы: 5, 15, 25, 35.
    \item Выведите текущее состояние очереди.
    \item Вызовите dequeue (вернет [5,15,25,35] и очистит очередь).
    \item Выведите результат dequeue и состояние очереди после вызова.
\end{enumerate}

Пример использования:
\begin{lstlisting}[language=Python]
queue = Queue(dequeue_all_at_once=True)
queue.enqueue(5)
queue.enqueue(15)
queue.enqueue(25)
queue.enqueue(35)

print("Current Queue:", queue._buffer)

dequeued_items = queue.dequeue()
print("Dequeued items:", dequeued_items)  # [5,15,25,35]
print("Updated Queue:", queue._buffer)    # []
\end{lstlisting}

\item Написать программу на Python, которая создает класс Queue для представления структуры данных очереди с инкапсуляцией. Класс должен содержать методы enqueue, dequeue и is\_empty, которые реализуют операции добавления элементов в очередь, удаления элементов из очереди и проверки пустоты очереди соответственно. Программа также должна создавать экземпляр класса Queue, добавлять элементы в очередь, удалять элементы из очереди и выводить информацию о состоянии очереди на экран.

Инструкции:
\begin{enumerate}
    \item Создайте класс Queue с методом \_\_init\_\_, который инициализирует пустую очередь (список \_store). Принимает параметр on\_enqueue\_callback=None — функция, вызываемая при каждом добавлении (с аргументом — добавленным элементом).
    \item Создайте метод enqueue, который добавляет элемент в конец и, если on\_enqueue\_callback не None, вызывает её с элементом.
    \item Создайте метод dequeue, который удаляет и возвращает первый элемент. Если очередь пуста — выбрасывает IndexError("Нельзя извлечь из пустой очереди").
    \item Создайте метод is\_empty, который возвращает True, если очередь пуста, и False в противном случае.
    \item Создайте функцию printer(x): print(f"[+] Добавлен: {x}")
    \item Создайте экземпляр класса Queue, передав printer в on\_enqueue\_callback.
    \item Добавьте элементы: 101, 202, 303.
    \item Выведите текущее состояние очереди.
    \item Вызовите dequeue, выведите удаленный элемент.
    \item Выведите обновленное состояние очереди.
\end{enumerate}

Пример использования:
\begin{lstlisting}[language=Python]
def printer(x):
    print(f"[+] Добавлен: {x}")

queue = Queue(on_enqueue_callback=printer)
queue.enqueue(101)  # [+] Добавлен: 101
queue.enqueue(202)  # [+] Добавлен: 202
queue.enqueue(303)  # [+] Добавлен: 303

print("Current Queue:", queue._store)

dequeued_item = queue.dequeue()
print("Dequeued item:", dequeued_item)

print("Updated Queue:", queue._store)
\end{lstlisting}

\item Написать программу на Python, которая создает класс Queue для представления структуры данных очереди с инкапсуляцией. Класс должен содержать методы enqueue, dequeue и is\_empty, которые реализуют операции добавления элементов в очередь, удаления элементов из очереди и проверки пустоты очереди соответственно. Программа также должна создавать экземпляр класса Queue, добавлять элементы в очередь, удалять элементы из очереди и выводить информацию о состоянии очереди на экран.

Инструкции:
\begin{enumerate}
    \item Создайте класс Queue с методом \_\_init\_\_, который инициализирует пустую очередь (список \_pool). Принимает параметр compress\_on\_enqueue=False. Если True, то при добавлении элемента, равного последнему в очереди, вместо добавления увеличивает счетчик дубликатов у последнего элемента (хранит пары (элемент, счетчик)).
    \item Создайте метод enqueue, который, если compress\_on\_enqueue=True и очередь не пуста и элемент == последний\_элемент, увеличивает счетчик последнего элемента. Иначе — добавляет новый элемент (со счетчиком 1, если режим сжатия включен).
    \item Создайте метод dequeue, который удаляет первый элемент. Если режим сжатия включен и счетчик >1, уменьшает счетчик и возвращает элемент. Если счетчик=1, удаляет элемент. Если очередь пуста — выбрасывает IndexError("Очередь пуста").
    \item Создайте метод is\_empty, который возвращает True, если очередь пуста, и False в противном случае.
    \item Создайте экземпляр класса Queue с compress\_on\_enqueue=True.
    \item Добавьте элементы: 7, 7, 7, 14, 14, 21.
    \item Выведите текущее состояние очереди (внутреннее представление).
    \item Вызовите dequeue трижды, выводя каждый раз удаленный элемент.
    \item Выведите обновленное состояние очереди.
\end{enumerate}

Пример использования:
\begin{lstlisting}[language=Python]
queue = Queue(compress_on_enqueue=True)
queue.enqueue(7)
queue.enqueue(7)
queue.enqueue(7)
queue.enqueue(14)
queue.enqueue(14)
queue.enqueue(21)

print("Current Queue:", queue._pool)  # [(7,3), (14,2), (21,1)]

for _ in range(3):
    dequeued_item = queue.dequeue()
    print("Dequeued item:", dequeued_item)  # 7, 7, 7

print("Updated Queue:", queue._pool)  # [(14,2), (21,1)]
\end{lstlisting}

\item Написать программу на Python, которая создает класс Queue для представления структуры данных очереди с инкапсуляцией. Класс должен содержать методы enqueue, dequeue и is\_empty, которые реализуют операции добавления элементов в очередь, удаления элементов из очереди и проверки пустоты очереди соответственно. Программа также должна создавать экземпляр класса Queue, добавлять элементы в очередь, удалять элементы из очереди и выводить информацию о состоянии очереди на экран.

Инструкции:
\begin{enumerate}
    \item Создайте класс Queue с методом \_\_init\_\_, который инициализирует пустую очередь (список \_line). Принимает параметр immutable\_dequeue=False. Если True, то dequeue возвращает первый элемент, но не удаляет его.
    \item Создайте метод enqueue, который добавляет элемент в конец очереди.
    \item Создайте метод dequeue, который, если immutable\_dequeue=False, удаляет и возвращает первый элемент. Если True — возвращает первый элемент, не удаляя его. Если очередь пуста — возвращает None.
    \item Создайте метод is\_empty, который возвращает True, если очередь пуста, и False в противном случае.
    \item Создайте экземпляр класса Queue с immutable\_dequeue=True.
    \item Добавьте элементы: 1, 3, 5.
    \item Выведите текущее состояние очереди.
    \item Вызовите dequeue дважды, выводя каждый раз результат (должен быть 1 оба раза).
    \item Выведите состояние очереди (не должно измениться).
\end{enumerate}

Пример использования:
\begin{lstlisting}[language=Python]
queue = Queue(immutable_dequeue=True)
queue.enqueue(1)
queue.enqueue(3)
queue.enqueue(5)

print("Current Queue:", queue._line)

print("Dequeued item:", queue.dequeue())  # 1
print("Dequeued item:", queue.dequeue())  # 1 (не удалилось)

print("Updated Queue:", queue._line)  # [1,3,5]
\end{lstlisting}

\item Написать программу на Python, которая создает класс Queue для представления структуры данных очереди с инкапсуляцией. Класс должен содержать методы enqueue, dequeue и is\_empty, которые реализуют операции добавления элементов в очередь, удаления элементов из очереди и проверки пустоты очереди соответственно. Программа также должна создавать экземпляр класса Queue, добавлять элементы в очередь, удалять элементы из очереди и выводить информацию о состоянии очереди на экран.

Инструкции:
\begin{enumerate}
    \item Создайте класс Queue с методом \_\_init\_\_, который инициализирует пустую очередь (список \_stream). Принимает параметр track\_history=False. Если True, сохраняет историю всех когда-либо добавленных элементов (даже удаленных) в отдельном списке \_history.
    \item Создайте метод enqueue, который добавляет элемент в конец \_stream и, если track\_history=True, добавляет его в \_history.
    \item Создайте метод dequeue, который удаляет и возвращает первый элемент из \_stream. Если очередь пуста — выбрасывает IndexError("Пусто").
    \item Создайте метод is\_empty, который возвращает True, если \_stream пуст, и False в противном случае.
    \item Создайте метод get\_history (только если track\_history=True), возвращающий копию \_history.
    \item Создайте экземпляр класса Queue с track\_history=True.
    \item Добавьте элементы: 2, 4, 6.
    \item Вызовите dequeue (вернет 2).
    \item Добавьте 8.
    \item Выведите текущую очередь и историю.
\end{enumerate}

Пример использования:
\begin{lstlisting}[language=Python]
queue = Queue(track_history=True)
queue.enqueue(2)
queue.enqueue(4)
queue.enqueue(6)
queue.dequeue()  # 2
queue.enqueue(8)

print("Current Queue:", queue._stream)    # [4,6,8]
print("History:", queue.get_history())    # [2,4,6,8]
\end{lstlisting}

\item Написать программу на Python, которая создает класс Queue для представления структуры данных очереди с инкапсуляцией. Класс должен содержать методы enqueue, dequeue и is\_empty, которые реализуют операции добавления элементов в очередь, удаления элементов из очереди и проверки пустоты очереди соответственно. Программа также должна создавать экземпляр класса Queue, добавлять элементы в очередь, удалять элементы из очереди и выводить информацию о состоянии очереди на экран.

Инструкции:
\begin{enumerate}
    \item Создайте класс Queue с методом \_\_init\_\_, который инициализирует пустую очередь (список \_flow). Принимает параметр enqueue\_only\_even=False. Если True, то добавляются только четные числа.
    \item Создайте метод enqueue, который добавляет элемент в конец, только если enqueue\_only\_even=False или элемент четный.
    \item Создайте метод dequeue, который удаляет и возвращает первый элемент. Если очередь пуста — выбрасывает IndexError("Очередь пуста").
    \item Создайте метод is\_empty, который возвращает True, если очередь пуста, и False в противном случае.
    \item Создайте экземпляр класса Queue с enqueue\_only\_even=True.
    \item Добавьте элементы: 1 (игнорируется), 2, 3 (игнорируется), 4, 5 (игнорируется), 6.
    \item Выведите текущее состояние очереди.
    \item Вызовите dequeue, выведите удаленный элемент.
    \item Выведите обновленное состояние очереди.
\end{enumerate}

Пример использования:
\begin{lstlisting}[language=Python]
queue = Queue(enqueue_only_even=True)
queue.enqueue(1)  # игнорируется
queue.enqueue(2)
queue.enqueue(3)  # игнорируется
queue.enqueue(4)
queue.enqueue(5)  # игнорируется
queue.enqueue(6)

print("Current Queue:", queue._flow)  # [2,4,6]

dequeued_item = queue.dequeue()
print("Dequeued item:", dequeued_item)  # 2

print("Updated Queue:", queue._flow)  # [4,6]
\end{lstlisting}

\item Написать программу на Python, которая создает класс Queue для представления структуры данных очереди с инкапсуляцией. Класс должен содержать методы enqueue, dequeue и is\_empty, которые реализуют операции добавления элементов в очередь, удаления элементов из очереди и проверки пустоты очереди соответственно. Программа также должна создавать экземпляр класса Queue, добавлять элементы в очередь, удалять элементы из очереди и выводить информацию о состоянии очереди на экран.

Инструкции:
\begin{enumerate}
    \item Создайте класс Queue с методом \_\_init\_\_, который инициализирует пустую очередь (список \_pipe). Принимает параметр reverse\_dequeue=False. Если True, то dequeue удаляет и возвращает не первый, а последний элемент.
    \item Создайте метод enqueue, который добавляет элемент в конец очереди.
    \item Создайте метод dequeue, который, если reverse\_dequeue=False, удаляет и возвращает первый элемент. Если True — удаляет и возвращает последний элемент. Если очередь пуста — выбрасывает IndexError("Пусто").
    \item Создайте метод is\_empty, который возвращает True, если очередь пуста, и False в противном случае.
    \item Создайте экземпляр класса Queue с reverse\_dequeue=True.
    \item Добавьте элементы: 10, 20, 30.
    \item Выведите текущее состояние очереди.
    \item Вызовите dequeue — должен вернуться 30 (последний).
    \item Выведите обновленное состояние очереди.
\end{enumerate}

Пример использования:
\begin{lstlisting}[language=Python]
queue = Queue(reverse_dequeue=True)
queue.enqueue(10)
queue.enqueue(20)
queue.enqueue(30)

print("Current Queue:", queue._pipe)  # [10,20,30]

dequeued_item = queue.dequeue()  # 30
print("Dequeued item:", dequeued_item)

print("Updated Queue:", queue._pipe)  # [10,20]
\end{lstlisting}

\item Написать программу на Python, которая создает класс Queue для представления структуры данных очереди с инкапсуляцией. Класс должен содержать методы enqueue, dequeue и is\_empty, которые реализуют операции добавления элементов в очередь, удаления элементов из очереди и проверки пустоты очереди соответственно. Программа также должна создавать экземпляр класса Queue, добавлять элементы в очередь, удалять элементы из очереди и выводить информацию о состоянии очереди на экран.

Инструкции:
\begin{enumerate}
    \item Создайте класс Queue с методом \_\_init\_\_, который инициализирует пустую очередь (список \_channel). Принимает параметр enqueue\_with\_timestamp=False. Если True, то при добавлении сохраняет пару (элемент, time.time()).
    \item Создайте метод enqueue, который, если enqueue\_with\_timestamp=True, добавляет (элемент, timestamp). Иначе — элемент.
    \item Создайте метод dequeue, который удаляет и возвращает первый элемент (или пару). Если очередь пуста — выбрасывает IndexError("Очередь пуста").
    \item Создайте метод is\_empty, который возвращает True, если очередь пуста, и False в противном случае.
    \item Создайте экземпляр класса Queue с enqueue\_with\_timestamp=True.
    \item Добавьте элементы: "first", "second", "third".
    \item Выведите текущее состояние очереди.
    \item Вызовите dequeue, выведите результат (пару).
    \item Выведите обновленное состояние очереди.
\end{enumerate}

Пример использования:
\begin{lstlisting}[language=Python]
import time

queue = Queue(enqueue_with_timestamp=True)
queue.enqueue("first")
queue.enqueue("second")
queue.enqueue("third")

print("Current Queue:", queue._channel)

dequeued_item = queue.dequeue()
print("Dequeued item:", dequeued_item)  # ('first', timestamp)

print("Updated Queue:", queue._channel)
\end{lstlisting}

\item Написать программу на Python, которая создает класс Queue для представления структуры данных очереди с инкапсуляцией. Класс должен содержать методы enqueue, dequeue и is\_empty, которые реализуют операции добавления элементов в очередь, удаления элементов из очереди и проверки пустоты очереди соответственно. Программа также должна создавать экземпляр класса Queue, добавлять элементы в очередь, удалять элементы из очереди и выводить информацию о состоянии очереди на экран.

Инструкции:
\begin{enumerate}
    \item Создайте класс Queue с методом \_\_init\_\_, который инициализирует пустую очередь (список \_tube). Принимает параметр enqueue\_pairs=False. Если True, то enqueue принимает два аргумента (key, value) и сохраняет кортеж (key, value).
    \item Создайте метод enqueue, который, если enqueue\_pairs=False, принимает один элемент. Если True — два аргумента и сохраняет кортеж.
    \item Создайте метод dequeue, который удаляет и возвращает первый элемент (или кортеж). Если очередь пуста — выбрасывает IndexError("Пусто").
    \item Создайте метод is\_empty, который возвращает True, если очередь пуста, и False в противном случае.
    \item Создайте экземпляр класса Queue с enqueue\_pairs=True.
    \item Добавьте пары: ("a", 1), ("b", 2), ("c", 3).
    \item Выведите текущее состояние очереди.
    \item Вызовите dequeue, выведите результат.
    \item Выведите обновленное состояние очереди.
\end{enumerate}

Пример использования:
\begin{lstlisting}[language=Python]
queue = Queue(enqueue_pairs=True)
queue.enqueue("a", 1)
queue.enqueue("b", 2)
queue.enqueue("c", 3)

print("Current Queue:", queue._tube)

dequeued_item = queue.dequeue()
print("Dequeued item:", dequeued_item)  # ('a', 1)

print("Updated Queue:", queue._tube)
\end{lstlisting}

\item Написать программу на Python, которая создает класс Queue для представления структуры данных очереди с инкапсуляцией. Класс должен содержать методы enqueue, dequeue и is\_empty, которые реализуют операции добавления элементов в очередь, удаления элементов из очереди и проверки пустоты очереди соответственно. Программа также должна создавать экземпляр класса Queue, добавлять элементы в очередь, удалять элементы из очереди и выводить информацию о состоянии очереди на экран.

Инструкции:
\begin{enumerate}
    \item Создайте класс Queue с методом \_\_init\_\_, который инициализирует пустую очередь (список \_conduit). Принимает параметр auto\_dedup=False. Если True, то при добавлении, если элемент уже есть в очереди, сначала удаляет все его предыдущие вхождения.
    \item Создайте метод enqueue, который, если auto\_dedup=True и элемент уже есть, удаляет все его вхождения, затем добавляет в конец. Иначе — просто добавляет.
    \item Создайте метод dequeue, который удаляет и возвращает первый элемент. Если очередь пуста — выбрасывает IndexError("Очередь пуста").
    \item Создайте метод is\_empty, который возвращает True, если очередь пуста, и False в противном случае.
    \item Создайте экземпляр класса Queue с auto\_dedup=True.
    \item Добавьте элементы: 1, 2, 1, 3, 2, 4.
    \item Выведите текущее состояние очереди.
    \item Вызовите dequeue, выведите удаленный элемент.
    \item Выведите обновленное состояние очереди.
\end{enumerate}

Пример использования:
\begin{lstlisting}[language=Python]
queue = Queue(auto_dedup=True)
queue.enqueue(1)  # [1]
queue.enqueue(2)  # [1,2]
queue.enqueue(1)  # удаляет старую 1 -> [2,1]
queue.enqueue(3)  # [2,1,3]
queue.enqueue(2)  # удаляет 2 -> [1,3,2]
queue.enqueue(4)  # [1,3,2,4]

print("Current Queue:", queue._conduit)

dequeued_item = queue.dequeue()
print("Dequeued item:", dequeued_item)  # 1

print("Updated Queue:", queue._conduit)  # [3,2,4]
\end{lstlisting}

\item Написать программу на Python, которая создает класс Queue для представления структуры данных очереди с инкапсуляцией. Класс должен содержать методы enqueue, dequeue и is\_empty, которые реализуют операции добавления элементов в очередь, удаления элементов из очереди и проверки пустоты очереди соответственно. Программа также должна создавать экземпляр класса Queue, добавлять элементы в очередь, удалять элементы из очереди и выводить информацию о состоянии очереди на экран.

Инструкции:
\begin{enumerate}
    \item Создайте класс Queue с методом \_\_init\_\_, который инициализирует пустую очередь (список \_duct). Принимает параметр enqueue\_if\_max=False. Если True, то элемент добавляется только если он больше всех текущих элементов в очереди.
    \item Создайте метод enqueue, который добавляет элемент, только если enqueue\_if\_max=False или элемент > всех элементов в очереди.
    \item Создайте метод dequeue, который удаляет и возвращает первый элемент. Если очередь пуста — выбрасывает IndexError("Пусто").
    \item Создайте метод is\_empty, который возвращает True, если очередь пуста, и False в противном случае.
    \item Создайте экземпляр класса Queue с enqueue\_if\_max=True.
    \item Добавьте элементы: 5, 3 (не добавится), 10, 7 (не добавится), 15.
    \item Выведите текущее состояние очереди.
    \item Вызовите dequeue, выведите удаленный элемент.
    \item Выведите обновленное состояние очереди.
\end{enumerate}

Пример использования:
\begin{lstlisting}[language=Python]
queue = Queue(enqueue_if_max=True)
queue.enqueue(5)
queue.enqueue(3)   # не добавится
queue.enqueue(10)
queue.enqueue(7)   # не добавится
queue.enqueue(15)

print("Current Queue:", queue._duct)  # [5,10,15]

dequeued_item = queue.dequeue()
print("Dequeued item:", dequeued_item)  # 5

print("Updated Queue:", queue._duct)  # [10,15]
\end{lstlisting}

\item Написать программу на Python, которая создает класс Queue для представления структуры данных очереди с инкапсуляцией. Класс должен содержать методы enqueue, dequeue и is\_empty, которые реализуют операции добавления элементов в очередь, удаления элементов из очереди и проверки пустоты очереди соответственно. Программа также должна создавать экземпляр класса Queue, добавлять элементы в очередь, удалять элементы из очереди и выводить информацию о состоянии очереди на экран.

Инструкции:
\begin{enumerate}
    \item Создайте класс Queue с методом \_\_init\_\_, который инициализирует пустую очередь (список \_pipe). Принимает параметр cumulative=False. Если True, то при добавлении элемент становится element + последний\_элемент (если очередь не пуста). Первый элемент добавляется как есть.
    \item Создайте метод enqueue, который, если cumulative=True и очередь не пуста, добавляет element + последний\_элемент. Иначе — element.
    \item Создайте метод dequeue, который удаляет и возвращает первый элемент. Если очередь пуста — выбрасывает IndexError("Очередь пуста").
    \item Создайте метод is\_empty, который возвращает True, если очередь пуста, и False в противном случае.
    \item Создайте экземпляр класса Queue с cumulative=True.
    \item Добавьте элементы: 1, 2, 3, 4.
    \item Выведите текущее состояние очереди.
    \item Вызовите dequeue, выведите удаленный элемент.
    \item Выведите обновленное состояние очереди.
\end{enumerate}

Пример использования:
\begin{lstlisting}[language=Python]
queue = Queue(cumulative=True)
queue.enqueue(1)  # [1]
queue.enqueue(2)  # [1, 1+2=3]
queue.enqueue(3)  # [1,3, 3+3=6]
queue.enqueue(4)  # [1,3,6, 6+4=10]

print("Current Queue:", queue._pipe)

dequeued_item = queue.dequeue()
print("Dequeued item:", dequeued_item)  # 1

print("Updated Queue:", queue._pipe)  # [3,6,10]
\end{lstlisting}

\item Написать программу на Python, которая создает класс Queue для представления структуры данных очереди с инкапсуляцией. Класс должен содержать методы enqueue, dequeue и is\_empty, которые реализуют операции добавления элементов в очередь, удаления элементов из очереди и проверки пустоты очереди соответственно. Программа также должна создавать экземпляр класса Queue, добавлять элементы в очередь, удалять элементы из очереди и выводить информацию о состоянии очереди на экран.

Инструкции:
\begin{enumerate}
    \item Создайте класс Queue с методом \_\_init\_\_, который инициализирует пустую очередь (список \_line). Принимает параметр enqueue\_squared=False. Если True, то при добавлении сохраняется element**2.
    \item Создайте метод enqueue, который добавляет element**2, если enqueue\_squared=True, иначе — element.
    \item Создайте метод dequeue, который удаляет и возвращает первый элемент. Если очередь пуста — выбрасывает IndexError("Пусто").
    \item Создайте метод is\_empty, который возвращает True, если очередь пуста, и False в противном случае.
    \item Создайте экземпляр класса Queue с enqueue\_squared=True.
    \item Добавьте элементы: 2, 3, 4, 5.
    \item Выведите текущее состояние очереди.
    \item Вызовите dequeue, выведите удаленный элемент.
    \item Выведите обновленное состояние очереди.
\end{enumerate}

Пример использования:
\begin{lstlisting}[language=Python]
queue = Queue(enqueue_squared=True)
queue.enqueue(2)  # 4
queue.enqueue(3)  # 9
queue.enqueue(4)  # 16
queue.enqueue(5)  # 25

print("Current Queue:", queue._line)

dequeued_item = queue.dequeue()
print("Dequeued item:", dequeued_item)  # 4

print("Updated Queue:", queue._line)  # [9,16,25]
\end{lstlisting}

\item Написать программу на Python, которая создает класс Queue для представления структуры данных очереди с инкапсуляцией. Класс должен содержать методы enqueue, dequeue и is\_empty, которые реализуют операции добавления элементов в очередь, удаления элементов из очереди и проверки пустоты очереди соответственно. Программа также должна создавать экземпляр класса Queue, добавлять элементы в очередь, удалять элементы из очереди и выводить информацию о состоянии очереди на экран.

Инструкции:
\begin{enumerate}
    \item Создайте класс Queue с методом \_\_init\_\_, который инициализирует пустую очередь (список \_stream). Принимает параметр enqueue\_absolute=False. Если True, то при добавлении сохраняется abs(element).
    \item Создайте метод enqueue, который добавляет abs(element), если enqueue\_absolute=True, иначе — element.
    \item Создайте метод dequeue, который удаляет и возвращает первый элемент. Если очередь пуста — выбрасывает IndexError("Очередь пуста").
    \item Создайте метод is\_empty, который возвращает True, если очередь пуста, и False в противном случае.
    \item Создайте экземпляр класса Queue с enqueue\_absolute=True.
    \item Добавьте элементы: -5, 3, -8, 2.
    \item Выведите текущее состояние очереди.
    \item Вызовите dequeue, выведите удаленный элемент.
    \item Выведите обновленное состояние очереди.
\end{enumerate}

Пример использования:
\begin{lstlisting}[language=Python]
queue = Queue(enqueue_absolute=True)
queue.enqueue(-5)  # 5
queue.enqueue(3)   # 3
queue.enqueue(-8)  # 8
queue.enqueue(2)   # 2

print("Current Queue:", queue._stream)

dequeued_item = queue.dequeue()
print("Dequeued item:", dequeued_item)  # 5

print("Updated Queue:", queue._stream)  # [3,8,2]
\end{lstlisting}

\item Написать программу на Python, которая создает класс Queue для представления структуры данных очереди с инкапсуляцией. Класс должен содержать методы enqueue, dequeue и is\_empty, которые реализуют операции добавления элементов в очередь, удаления элементов из очереди и проверки пустоты очереди соответственно. Программа также должна создавать экземпляр класса Queue, добавлять элементы в очередь, удалять элементы из очереди и выводить информацию о состоянии очереди на экран.

Инструкции:
\begin{enumerate}
    \item Создайте класс Queue с методом \_\_init\_\_, который инициализирует пустую очередь (список \_buffer). Принимает параметр enqueue\_rounded=False. Если True, то при добавлении сохраняется round(element).
    \item Создайте метод enqueue, который добавляет round(element), если enqueue\_rounded=True, иначе — element.
    \item Создайте метод dequeue, который удаляет и возвращает первый элемент. Если очередь пуста — выбрасывает IndexError("Пусто").
    \item Создайте метод is\_empty, который возвращает True, если очередь пуста, и False в противном случае.
    \item Создайте экземпляр класса Queue с enqueue\_rounded=True.
    \item Добавьте элементы: 3.2, 4.7, 5.1, 6.9.
    \item Выведите текущее состояние очереди.
    \item Вызовите dequeue, выведите удаленный элемент.
    \item Выведите обновленное состояние очереди.
\end{enumerate}

Пример использования:
\begin{lstlisting}[language=Python]
queue = Queue(enqueue_rounded=True)
queue.enqueue(3.2)  # 3
queue.enqueue(4.7)  # 5
queue.enqueue(5.1)  # 5
queue.enqueue(6.9)  # 7

print("Current Queue:", queue._buffer)

dequeued_item = queue.dequeue()
print("Dequeued item:", dequeued_item)  # 3

print("Updated Queue:", queue._buffer)  # [5,5,7]
\end{lstlisting}

\item Написать программу на Python, которая создает класс Queue для представления структуры данных очереди с инкапсуляцией. Класс должен содержать методы enqueue, dequeue и is\_empty, которые реализуют операции добавления элементов в очередь, удаления элементов из очереди и проверки пустоты очереди соответственно. Программа также должна создавать экземпляр класса Queue, добавлять элементы в очередь, удалять элементы из очереди и выводить информацию о состоянии очереди на экран.

Инструкции:
\begin{enumerate}
    \item Создайте класс Queue с методом \_\_init\_\_, который инициализирует пустую очередь (список \_store). Принимает параметр enqueue\_negated=False. Если True, то при добавлении сохраняется -element.
    \item Создайте метод enqueue, который добавляет -element, если enqueue\_negated=True, иначе — element.
    \item Создайте метод dequeue, который удаляет и возвращает первый элемент. Если очередь пуста — выбрасывает IndexError("Очередь пуста").
    \item Создайте метод is\_empty, который возвращает True, если очередь пуста, и False в противном случае.
    \item Создайте экземпляр класса Queue с enqueue\_negated=True.
    \item Добавьте элементы: 10, 20, 30, 40.
    \item Выведите текущее состояние очереди.
    \item Вызовите dequeue, выведите удаленный элемент.
    \item Выведите обновленное состояние очереди.
\end{enumerate}

Пример использования:
\begin{lstlisting}[language=Python]
queue = Queue(enqueue_negated=True)
queue.enqueue(10)  # -10
queue.enqueue(20)  # -20
queue.enqueue(30)  # -30
queue.enqueue(40)  # -40

print("Current Queue:", queue._store)

dequeued_item = queue.dequeue()
print("Dequeued item:", dequeued_item)  # -10

print("Updated Queue:", queue._store)  # [-20,-30,-40]
\end{lstlisting}

\item Написать программу на Python, которая создает класс Queue для представления структуры данных очереди с инкапсуляцией. Класс должен содержать методы enqueue, dequeue и is\_empty, которые реализуют операции добавления элементов в очередь, удаления элементов из очереди и проверки пустоты очереди соответственно. Программа также должна создавать экземпляр класса Queue, добавлять элементы в очередь, удалять элементы из очереди и выводить информацию о состоянии очереди на экран.

Инструкции:
\begin{enumerate}
    \item Создайте класс Queue с методом \_\_init\_\_, который инициализирует пустую очередь (список \_pool). Принимает параметр enqueue\_doubled=False. Если True, то при добавлении сохраняется element * 2.
    \item Создайте метод enqueue, который добавляет element * 2, если enqueue\_doubled=True, иначе — element.
    \item Создайте метод dequeue, который удаляет и возвращает первый элемент. Если очередь пуста — выбрасывает IndexError("Пусто").
    \item Создайте метод is\_empty, который возвращает True, если очередь пуста, и False в противном случае.
    \item Создайте экземпляр класса Queue с enqueue\_doubled=True.
    \item Добавьте элементы: 1, 2, 3, 4.
    \item Выведите текущее состояние очереди.
    \item Вызовите dequeue, выведите удаленный элемент.
    \item Выведите обновленное состояние очереди.
\end{enumerate}

Пример использования:
\begin{lstlisting}[language=Python]
queue = Queue(enqueue_doubled=True)
queue.enqueue(1)  # 2
queue.enqueue(2)  # 4
queue.enqueue(3)  # 6
queue.enqueue(4)  # 8

print("Current Queue:", queue._pool)

dequeued_item = queue.dequeue()
print("Dequeued item:", dequeued_item)  # 2

print("Updated Queue:", queue._pool)  # [4,6,8]
\end{lstlisting}

\item Написать программу на Python, которая создает класс Queue для представления структуры данных очереди с инкапсуляцией. Класс должен содержать методы enqueue, dequeue и is\_empty, которые реализуют операции добавления элементов в очередь, удаления элементов из очереди и проверки пустоты очереди соответственно. Программа также должна создавать экземпляр класса Queue, добавлять элементы в очередь, удалять элементы из очереди и выводить информацию о состоянии очереди на экран.

Инструкции:
\begin{enumerate}
    \item Создайте класс Queue с методом \_\_init\_\_, который инициализирует пустую очередь (список \_reservoir). Принимает параметр enqueue\_halved=False. Если True, то при добавлении сохраняется element / 2.0.
    \item Создайте метод enqueue, который добавляет element / 2.0, если enqueue\_halved=True, иначе — element.
    \item Создайте метод dequeue, который удаляет и возвращает первый элемент. Если очередь пуста — выбрасывает IndexError("Очередь пуста").
    \item Создайте метод is\_empty, который возвращает True, если очередь пуста, и False в противном случае.
    \item Создайте экземпляр класса Queue с enqueue\_halved=True.
    \item Добавьте элементы: 4, 8, 12, 16.
    \item Выведите текущее состояние очереди.
    \item Вызовите dequeue, выведите удаленный элемент.
    \item Выведите обновленное состояние очереди.
\end{enumerate}

Пример использования:
\begin{lstlisting}[language=Python]
queue = Queue(enqueue_halved=True)
queue.enqueue(4)   # 2.0
queue.enqueue(8)   # 4.0
queue.enqueue(12)  # 6.0
queue.enqueue(16)  # 8.0

print("Current Queue:", queue._reservoir)

dequeued_item = queue.dequeue()
print("Dequeued item:", dequeued_item)  # 2.0

print("Updated Queue:", queue._reservoir)  # [4.0,6.0,8.0]
\end{lstlisting}

\item Написать программу на Python, которая создает класс Queue для представления структуры данных очереди с инкапсуляцией. Класс должен содержать методы enqueue, dequeue и is\_empty, которые реализуют операции добавления элементов в очередь, удаления элементов из очереди и проверки пустоты очереди соответственно. Программа также должна создавать экземпляр класса Queue, добавлять элементы в очередь, удалять элементы из очереди и выводить информацию о состоянии очереди на экран.

Инструкции:
\begin{enumerate}
    \item Создайте класс Queue с методом \_\_init\_\_, который инициализирует пустую очередь (список \_tank). Принимает параметр enqueue\_as\_string=False. Если True, то при добавлении сохраняется str(element).
    \item Создайте метод enqueue, который добавляет str(element), если enqueue\_as\_string=True, иначе — element.
    \item Создайте метод dequeue, который удаляет и возвращает первый элемент. Если очередь пуста — выбрасывает IndexError("Пусто").
    \item Создайте метод is\_empty, который возвращает True, если очередь пуста, и False в противном случае.
    \item Создайте экземпляр класса Queue с enqueue\_as\_string=True.
    \item Добавьте элементы: 100, 200, 300, 400.
    \item Выведите текущее состояние очереди.
    \item Вызовите dequeue, выведите удаленный элемент.
    \item Выведите обновленное состояние очереди.
\end{enumerate}

Пример использования:
\begin{lstlisting}[language=Python]
queue = Queue(enqueue_as_string=True)
queue.enqueue(100)  # "100"
queue.enqueue(200)  # "200"
queue.enqueue(300)  # "300"
queue.enqueue(400)  # "400"

print("Current Queue:", queue._tank)

dequeued_item = queue.dequeue()
print("Dequeued item:", dequeued_item)  # "100"

print("Updated Queue:", queue._tank)  # ["200","300","400"]
\end{lstlisting}

\item Написать программу на Python, которая создает класс Queue для представления структуры данных очереди с инкапсуляцией. Класс должен содержать методы enqueue, dequeue и is\_empty, которые реализуют операции добавления элементов в очередь, удаления элементов из очереди и проверки пустоты очереди соответственно. Программа также должна создавать экземпляр класса Queue, добавлять элементы в очередь, удалять элементы из очереди и выводить информацию о состоянии очереди на экран.

Инструкции:
\begin{enumerate}
    \item Создайте класс Queue с методом \_\_init\_\_, который инициализирует пустую очередь (список \_container). Принимает параметр enqueue\_with\_index=False. Если True, то при добавлении сохраняется кортеж (element, порядковый\_номер\_добавления).
    \item Создайте метод enqueue, который добавляет (element, self.\_counter), где \_counter — внутренний счетчик, увеличивающийся при каждом добавлении. Иначе — element.
    \item Создайте метод dequeue, который удаляет и возвращает первый элемент (или кортеж). Если очередь пуста — выбрасывает IndexError("Очередь пуста").
    \item Создайте метод is\_empty, который возвращает True, если очередь пуста, и False в противном случае.
    \item Создайте экземпляр класса Queue с enqueue\_with\_index=True.
    \item Добавьте элементы: "alpha", "beta", "gamma".
    \item Выведите текущее состояние очереди.
    \item Вызовите dequeue, выведите удаленный элемент.
    \item Выведите обновленное состояние очереди.
\end{enumerate}

Пример использования:
\begin{lstlisting}[language=Python]
queue = Queue(enqueue_with_index=True)
queue.enqueue("alpha")  # ("alpha", 0)
queue.enqueue("beta")   # ("beta", 1)
queue.enqueue("gamma")  # ("gamma", 2)

print("Current Queue:", queue._container)

dequeued_item = queue.dequeue()
print("Dequeued item:", dequeued_item)  # ('alpha', 0)

print("Updated Queue:", queue._container)  # [('beta',1), ('gamma',2)]
\end{lstlisting}

\item Написать программу на Python, которая создает класс Queue для представления структуры данных очереди с инкапсуляцией. Класс должен содержать методы enqueue, dequeue и is\_empty, которые реализуют операции добавления элементов в очередь, удаления элементов из очереди и проверки пустоты очереди соответственно. Программа также должна создавать экземпляр класса Queue, добавлять элементы в очередь, удалять элементы из очереди и выводить информацию о состоянии очереди на экран.

Инструкции:
\begin{enumerate}
    \item Создайте класс Queue с методом \_\_init\_\_, который инициализирует пустую очередь (список \_vessel). Принимает параметр enqueue\_unique\_rear=False. Если True, то при добавлении, если элемент равен текущему последнему, он не добавляется.
    \item Создайте метод enqueue, который добавляет элемент, только если enqueue\_unique\_rear=False или очередь пуста или element != последний\_элемент.
    \item Создайте метод dequeue, который удаляет и возвращает первый элемент. Если очередь пуста — выбрасывает IndexError("Пусто").
    \item Создайте метод is\_empty, который возвращает True, если очередь пуста, и False в противном случае.
    \item Создайте экземпляр класса Queue с enqueue\_unique\_rear=True.
    \item Добавьте элементы: 1, 2, 2, 3, 3, 3, 4.
    \item Выведите текущее состояние очереди.
    \item Вызовите dequeue, выведите удаленный элемент.
    \item Выведите обновленное состояние очереди.
\end{enumerate}

Пример использования:
\begin{lstlisting}[language=Python]
queue = Queue(enqueue_unique_rear=True)
queue.enqueue(1)
queue.enqueue(2)
queue.enqueue(2)  # не добавится
queue.enqueue(3)
queue.enqueue(3)  # не добавится
queue.enqueue(3)  # не добавится
queue.enqueue(4)

print("Current Queue:", queue._vessel)  # [1,2,3,4]

dequeued_item = queue.dequeue()
print("Dequeued item:", dequeued_item)  # 1

print("Updated Queue:", queue._vessel)  # [2,3,4]
\end{lstlisting}

\item Написать программу на Python, которая создает класс Queue для представления структуры данных очереди с инкапсуляцией. Класс должен содержать методы enqueue, dequeue и is\_empty, которые реализуют операции добавления элементов в очередь, удаления элементов из очереди и проверки пустоты очереди соответственно. Программа также должна создавать экземпляр класса Queue, добавлять элементы в очередь, удалять элементы из очереди и выводить информацию о состоянии очереди на экран.

Инструкции:
\begin{enumerate}
    \item Создайте класс Queue с методом \_\_init\_\_, который инициализирует пустую очередь (список \_bin). Принимает параметр enqueue\_even\_only=False. Если True, то добавляются только четные числа.
    \item Создайте метод enqueue, который добавляет элемент, только если enqueue\_even\_only=False или element \% 2 == 0.
    \item Создайте метод dequeue, который удаляет и возвращает первый элемент. Если очередь пуста — выбрасывает IndexError("Очередь пуста").
    \item Создайте метод is\_empty, который возвращает True, если очередь пуста, и False в противном случае.
    \item Создайте экземпляр класса Queue с enqueue\_even\_only=True.
    \item Добавьте элементы: 1 (не добавится), 2, 3 (не добавится), 4, 5 (не добавится), 6.
    \item Выведите текущее состояние очереди.
    \item Вызовите dequeue, выведите удаленный элемент.
    \item Выведите обновленное состояние очереди.
\end{enumerate}

Пример использования:
\begin{lstlisting}[language=Python]
queue = Queue(enqueue_even_only=True)
queue.enqueue(1)  # нет
queue.enqueue(2)
queue.enqueue(3)  # нет
queue.enqueue(4)
queue.enqueue(5)  # нет
queue.enqueue(6)

print("Current Queue:", queue._bin)  # [2,4,6]

dequeued_item = queue.dequeue()
print("Dequeued item:", dequeued_item)  # 2

print("Updated Queue:", queue._bin)  # [4,6]
\end{lstlisting}

\item Написать программу на Python, которая создает класс Queue для представления структуры данных очереди с инкапсуляцией. Класс должен содержать методы enqueue, dequeue и is\_empty, которые реализуют операции добавления элементов в очередь, удаления элементов из очереди и проверки пустоты очереди соответственно. Программа также должна создавать экземпляр класса Queue, добавлять элементы в очередь, удалять элементы из очереди и выводить информацию о состоянии очереди на экран.

Инструкции:
\begin{enumerate}
    \item Создайте класс Queue с методом \_\_init\_\_, который инициализирует пустую очередь (список \_box). Принимает параметр enqueue\_odd\_only=False. Если True, то добавляются только нечетные числа.
    \item Создайте метод enqueue, который добавляет элемент, только если enqueue\_odd\_only=False или element \% 2 != 0.
    \item Создайте метод dequeue, который удаляет и возвращает первый элемент. Если очередь пуста — выбрасывает IndexError("Пусто").
    \item Создайте метод is\_empty, который возвращает True, если очередь пуста, и False в противном случае.
    \item Создайте экземпляр класса Queue с enqueue\_odd\_only=True.
    \item Добавьте элементы: 2 (не добавится), 1, 4 (не добавится), 3, 6 (не добавится), 5.
    \item Выведите текущее состояние очереди.
    \item Вызовите dequeue, выведите удаленный элемент.
    \item Выведите обновленное состояние очереди.
\end{enumerate}

Пример использования:
\begin{lstlisting}[language=Python]
queue = Queue(enqueue_odd_only=True)
queue.enqueue(2)  # нет
queue.enqueue(1)
queue.enqueue(4)  # нет
queue.enqueue(3)
queue.enqueue(6)  # нет
queue.enqueue(5)

print("Current Queue:", queue._box)  # [1,3,5]

dequeued_item = queue.dequeue()
print("Dequeued item:", dequeued_item)  # 1

print("Updated Queue:", queue._box)  # [3,5]
\end{lstlisting}

\item Написать программу на Python, которая создает класс Queue для представления структуры данных очереди с инкапсуляцией. Класс должен содержать методы enqueue, dequeue и is\_empty, которые реализуют операции добавления элементов в очередь, удаления элементов из очереди и проверки пустоты очереди соответственно. Программа также должна создавать экземпляр класса Queue, добавлять элементы в очередь, удалять элементы из очереди и выводить информацию о состоянии очереди на экран.

Инструкции:
\begin{enumerate}
    \item Создайте класс Queue с методом \_\_init\_\_, который инициализирует пустую очередь (список \_crate). Принимает параметр enqueue\_positive\_only=False. Если True, то добавляются только положительные числа (>0).
    \item Создайте метод enqueue, который добавляет элемент, только если enqueue\_positive\_only=False или element > 0.
    \item Создайте метод dequeue, который удаляет и возвращает первый элемент. Если очередь пуста — выбрасывает IndexError("Очередь пуста").
    \item Создайте метод is\_empty, который возвращает True, если очередь пуста, и False в противном случае.
    \item Создайте экземпляр класса Queue с enqueue\_positive\_only=True.
    \item Добавьте элементы: -1 (не добавится), 0 (не добавится), 1, 2, -5 (не добавится), 3.
    \item Выведите текущее состояние очереди.
    \item Вызовите dequeue, выведите удаленный элемент.
    \item Выведите обновленное состояние очереди.
\end{enumerate}

Пример использования:
\begin{lstlisting}[language=Python]
queue = Queue(enqueue_positive_only=True)
queue.enqueue(-1)  # нет
queue.enqueue(0)   # нет
queue.enqueue(1)
queue.enqueue(2)
queue.enqueue(-5)  # нет
queue.enqueue(3)

print("Current Queue:", queue._crate)  # [1,2,3]

dequeued_item = queue.dequeue()
print("Dequeued item:", dequeued_item)  # 1

print("Updated Queue:", queue._crate)  # [2,3]
\end{lstlisting}

\item Написать программу на Python, которая создает класс Queue для представления структуры данных очереди с инкапсуляцией. Класс должен содержать методы enqueue, dequeue и is\_empty, которые реализуют операции добавления элементов в очередь, удаления элементов из очереди и проверки пустоты очереди соответственно. Программа также должна создавать экземпляр класса Queue, добавлять элементы в очередь, удалять элементы из очереди и выводить информацию о состоянии очереди на экран.

Инструкции:
\begin{enumerate}
    \item Создайте класс Queue с методом \_\_init\_\_, который инициализирует пустую очередь (список \_carton). Принимает параметр enqueue\_nonzero\_only=False. Если True, то добавляются только ненулевые числа.
    \item Создайте метод enqueue, который добавляет элемент, только если enqueue\_nonzero\_only=False или element != 0.
    \item Создайте метод dequeue, который удаляет и возвращает первый элемент. Если очередь пуста — выбрасывает IndexError("Пусто").
    \item Создайте метод is\_empty, который возвращает True, если очередь пуста, и False в противном случае.
    \item Создайте экземпляр класса Queue с enqueue\_nonzero\_only=True.
    \item Добавьте элементы: 0 (не добавится), 5, 0 (не добавится), 10, 15.
    \item Выведите текущее состояние очереди.
    \item Вызовите dequeue, выведите удаленный элемент.
    \item Выведите обновленное состояние очереди.
\end{enumerate}

Пример использования:
\begin{lstlisting}[language=Python]
queue = Queue(enqueue_nonzero_only=True)
queue.enqueue(0)   # нет
queue.enqueue(5)
queue.enqueue(0)   # нет
queue.enqueue(10)
queue.enqueue(15)

print("Current Queue:", queue._carton)  # [5,10,15]

dequeued_item = queue.dequeue()
print("Dequeued item:", dequeued_item)  # 5

print("Updated Queue:", queue._carton)  # [10,15]
\end{lstlisting}

\item Написать программу на Python, которая создает класс Queue для представления структуры данных очереди с инкапсуляцией. Класс должен содержать методы enqueue, dequeue и is\_empty, которые реализуют операции добавления элементов в очередь, удаления элементов из очереди и проверки пустоты очереди соответственно. Программа также должна создавать экземпляр класса Queue, добавлять элементы в очередь, удалять элементы из очереди и выводить информацию о состоянии очереди на экран.

Инструкции:
\begin{enumerate}
    \item Создайте класс Queue с методом \_\_init\_\_, который инициализирует пустую очередь (список \_package). Принимает параметр enqueue\_prime\_only=False. Если True, то добавляются только простые числа (реализуйте простую проверку).
    \item Создайте метод enqueue, который добавляет элемент, только если enqueue\_prime\_only=False или element — простое число.
    \item Создайте метод dequeue, который удаляет и возвращает первый элемент. Если очередь пуста — выбрасывает IndexError("Очередь пуста").
    \item Создайте метод is\_empty, который возвращает True, если очередь пуста, и False в противном случае.
    \item Создайте вспомогательную функцию is\_prime(n) (вне класса).
    \item Создайте экземпляр класса Queue с enqueue\_prime\_only=True.
    \item Добавьте элементы: 4 (не простое), 5 (простое), 6 (не простое), 7 (простое), 8 (не простое), 11 (простое).
    \item Выведите текущее состояние очереди.
    \item Вызовите dequeue, выведите удаленный элемент.
    \item Выведите обновленное состояние очереди.
\end{enumerate}

Пример использования:
\begin{lstlisting}[language=Python]
def is_prime(n):
    if n < 2:
        return False
    for i in range(2, int(n**0.5)+1):
        if n % i == 0:
            return False
    return True

queue = Queue(enqueue_prime_only=True)
queue.enqueue(4)   # нет
queue.enqueue(5)   # да
queue.enqueue(6)   # нет
queue.enqueue(7)   # да
queue.enqueue(8)   # нет
queue.enqueue(11)  # да

print("Current Queue:", queue._package)  # [5,7,11]

dequeued_item = queue.dequeue()
print("Dequeued item:", dequeued_item)  # 5

print("Updated Queue:", queue._package)  # [7,11]
\end{lstlisting}

\item Написать программу на Python, которая создает класс Queue для представления структуры данных очереди с инкапсуляцией. Класс должен содержать методы enqueue, dequeue и is\_empty, которые реализуют операции добавления элементов в очередь, удаления элементов из очереди и проверки пустоты очереди соответственно. Программа также должна создавать экземпляр класса Queue, добавлять элементы в очередь, удалять элементы из очереди и выводить информацию о состоянии очереди на экран.

Инструкции:
\begin{enumerate}
    \item Создайте класс Queue с методом \_\_init\_\_, который инициализирует пустую очередь (список \_parcel). Принимает параметр enqueue\_fibonacci\_only=False. Если True, то добавляются только числа Фибоначчи (до 100: 0,1,1,2,3,5,8,13,21,34,55,89).
    \item Создайте метод enqueue, который добавляет элемент, только если enqueue\_fibonacci\_only=False или element входит в FIB\_SET.
    \item Создайте метод dequeue, который удаляет и возвращает первый элемент. Если очередь пуста — выбрасывает IndexError("Пусто").
    \item Создайте метод is\_empty, который возвращает True, если очередь пуста, и False в противном случае.
    \item Создайте экземпляр класса Queue с enqueue\_fibonacci\_only=True.
    \item Добавьте элементы: 4 (не Фибоначчи), 5 (Фибоначчи), 6 (не Фибоначчи), 8 (Фибоначчи), 7 (не Фибоначчи), 13 (Фибоначчи).
    \item Выведите текущее состояние очереди.
    \item Вызовите dequeue, выведите удаленный элемент.
    \item Выведите обновленное состояние очереди.
\end{enumerate}

Пример использования:
\begin{lstlisting}[language=Python]
FIB_SET = {0, 1, 2, 3, 5, 8, 13, 21, 34, 55, 89}

queue = Queue(enqueue_fibonacci_only=True)
queue.enqueue(4)   # нет
queue.enqueue(5)   # да
queue.enqueue(6)   # нет
queue.enqueue(8)   # да
queue.enqueue(7)   # нет
queue.enqueue(13)  # да

print("Current Queue:", queue._parcel)  # [5,8,13]

dequeued_item = queue.dequeue()
print("Dequeued item:", dequeued_item)  # 5

print("Updated Queue:", queue._parcel)  # [8,13]
\end{lstlisting}

\item Написать программу на Python, которая создает класс Queue для представления структуры данных очереди с инкапсуляцией. Класс должен содержать методы enqueue, dequeue и is\_empty, которые реализуют операции добавления элементов в очередь, удаления элементов из очереди и проверки пустоты очереди соответственно. Программа также должна создавать экземпляр класса Queue, добавлять элементы в очередь, удалять элементы из очереди и выводить информацию о состоянии очереди на экран.

Инструкции:
\begin{enumerate}
    \item Создайте класс Queue с методом \_\_init\_\_, который инициализирует пустую очередь (список \_sack). Принимает параметр enqueue\_palindrome\_only=False. Если True, то добавляются только числа-палиндромы.
    \item Создайте метод enqueue, который добавляет элемент, только если enqueue\_palindrome\_only=False или element — палиндром (str(element) == str(element)[::-1]).
    \item Создайте метод dequeue, который удаляет и возвращает первый элемент. Если очередь пуста — выбрасывает IndexError("Очередь пуста").
    \item Создайте метод is\_empty, который возвращает True, если очередь пуста, и False в противном случае.
    \item Создайте экземпляр класса Queue с enqueue\_palindrome\_only=True.
    \item Добавьте элементы: 12 (не палиндром), 22 (палиндром), 34 (не палиндром), 55 (палиндром), 123 (не палиндром), 121 (палиндром).
    \item Выведите текущее состояние очереди.
    \item Вызовите dequeue, выведите удаленный элемент.
    \item Выведите обновленное состояние очереди.
\end{enumerate}

Пример использования:
\begin{lstlisting}[language=Python]
queue = Queue(enqueue_palindrome_only=True)
queue.enqueue(12)   # нет
queue.enqueue(22)   # да
queue.enqueue(34)   # нет
queue.enqueue(55)   # да
queue.enqueue(123)  # нет
queue.enqueue(121)  # да

print("Current Queue:", queue._sack)  # [22,55,121]

dequeued_item = queue.dequeue()
print("Dequeued item:", dequeued_item)  # 22

print("Updated Queue:", queue._sack)  # [55,121]
\end{lstlisting}

\item Написать программу на Python, которая создает класс Queue для представления структуры данных очереди с инкапсуляцией. Класс должен содержать методы enqueue, dequeue и is\_empty, которые реализуют операции добавления элементов в очередь, удаления элементов из очереди и проверки пустоты очереди соответственно. Программа также должна создавать экземпляр класса Queue, добавлять элементы в очередь, удалять элементы из очереди и выводить информацию о состоянии очереди на экран.

Инструкции:
\begin{enumerate}
    \item Создайте класс Queue с методом \_\_init\_\_, который инициализирует пустую очередь (список \_bag). Принимает параметр enqueue\_power\_of\_two=False. Если True, то добавляются только степени двойки.
    \item Создайте метод enqueue, который добавляет элемент, только если enqueue\_power\_of\_two=False или element > 0 и (element \& (element-1)) == 0.
    \item Создайте метод dequeue, который удаляет и возвращает первый элемент. Если очередь пуста — выбрасывает IndexError("Пусто").
    \item Создайте метод is\_empty, который возвращает True, если очередь пуста, и False в противном случае.
    \item Создайте экземпляр класса Queue с enqueue\_power\_of\_two=True.
    \item Добавьте элементы: 3 (не степень), 4 (степень), 5 (не степень), 8 (степень), 9 (не степень), 16 (степень).
    \item Выведите текущее состояние очереди.
    \item Вызовите dequeue, выведите удаленный элемент.
    \item Выведите обновленное состояние очереди.
\end{enumerate}

Пример использования:
\begin{lstlisting}[language=Python]
queue = Queue(enqueue_power_of_two=True)
queue.enqueue(3)   # нет
queue.enqueue(4)   # да
queue.enqueue(5)   # нет
queue.enqueue(8)   # да
queue.enqueue(9)   # нет
queue.enqueue(16)  # да

print("Current Queue:", queue._bag)  # [4,8,16]

dequeued_item = queue.dequeue()
print("Dequeued item:", dequeued_item)  # 4

print("Updated Queue:", queue._bag)  # [8,16]
\end{lstlisting}

\item Написать программу на Python, которая создает класс Queue для представления структуры данных очереди с инкапсуляцией. Класс должен содержать методы enqueue, dequeue и is\_empty, которые реализуют операции добавления элементов в очередь, удаления элементов из очереди и проверки пустоты очереди соответственно. Программа также должна создавать экземпляр класса Queue, добавлять элементы в очередь, удалять элементы из очереди и выводить информацию о состоянии очереди на экран.

Инструкции:
\begin{enumerate}
    \item Создайте класс Queue с методом \_\_init\_\_, который инициализирует пустую очередь (список \_suitcase). Принимает параметр enqueue\_divisible\_by\_three=False. Если True, то добавляются только числа, делящиеся на 3.
    \item Создайте метод enqueue, который добавляет элемент, только если enqueue\_divisible\_by\_three=False или element \% 3 == 0.
    \item Создайте метод dequeue, который удаляет и возвращает первый элемент. Если очередь пуста — выбрасывает IndexError("Очередь пуста").
    \item Создайте метод is\_empty, который возвращает True, если очередь пуста, и False в противном случае.
    \item Создайте экземпляр класса Queue с enqueue\_divisible\_by\_three=True.
    \item Добавьте элементы: 1 (нет), 3 (да), 4 (нет), 6 (да), 7 (нет), 9 (да).
    \item Выведите текущее состояние очереди.
    \item Вызовите dequeue, выведите удаленный элемент.
    \item Выведите обновленное состояние очереди.
\end{enumerate}

Пример использования:
\begin{lstlisting}[language=Python]
queue = Queue(enqueue_divisible_by_three=True)
queue.enqueue(1)  # нет
queue.enqueue(3)  # да
queue.enqueue(4)  # нет
queue.enqueue(6)  # да
queue.enqueue(7)  # нет
queue.enqueue(9)  # да

print("Current Queue:", queue._suitcase)  # [3,6,9]

dequeued_item = queue.dequeue()
print("Dequeued item:", dequeued_item)  # 3

print("Updated Queue:", queue._suitcase)  # [6,9]
\end{lstlisting}

\item Написать программу на Python, которая создает класс Queue для представления структуры данных очереди с инкапсуляцией. Класс должен содержать методы enqueue, dequeue и is\_empty, которые реализуют операции добавления элементов в очередь, удаления элементов из очереди и проверки пустоты очереди соответственно. Программа также должна создавать экземпляр класса Queue, добавлять элементы в очередь, удалять элементы из очереди и выводить информацию о состоянии очереди на экран.

Инструкции:
\begin{enumerate}
    \item Создайте класс Queue с методом \_\_init\_\_, который инициализирует пустую очередь (список \_luggage). Принимает параметр enqueue\_greater\_than\_prev=False. Если True, то элемент добавляется только если он строго больше предыдущего добавленного элемента (первый — всегда).
    \item Создайте метод enqueue, который добавляет элемент, только если enqueue\_greater\_than\_prev=False или очередь пуста или element > последний\_элемент.
    \item Создайте метод dequeue, который удаляет и возвращает первый элемент. Если очередь пуста — выбрасывает IndexError("Пусто").
    \item Создайте метод is\_empty, который возвращает True, если очередь пуста, и False в противном случае.
    \item Создайте экземпляр класса Queue с enqueue\_greater\_than\_prev=True.
    \item Добавьте элементы: 5, 3 (не добавится), 7, 6 (не добавится), 10, 8 (не добавится).
    \item Выведите текущее состояние очереди.
    \item Вызовите dequeue, выведите удаленный элемент.
    \item Выведите обновленное состояние очереди.
\end{enumerate}

Пример использования:
\begin{lstlisting}[language=Python]
queue = Queue(enqueue_greater_than_prev=True)
queue.enqueue(5)
queue.enqueue(3)  # нет
queue.enqueue(7)
queue.enqueue(6)  # нет
queue.enqueue(10)
queue.enqueue(8)  # нет

print("Current Queue:", queue._luggage)  # [5,7,10]

dequeued_item = queue.dequeue()
print("Dequeued item:", dequeued_item)  # 5

print("Updated Queue:", queue._luggage)  # [7,10]
\end{lstlisting}

\item Написать программу на Python, которая создает класс Queue для представления структуры данных очереди с инкапсуляцией. Класс должен содержать методы enqueue, dequeue и is\_empty, которые реализуют операции добавления элементов в очередь, удаления элементов из очереди и проверки пустоты очереди соответственно. Программа также должна создавать экземпляр класса Queue, добавлять элементы в очередь, удалять элементы из очереди и выводить информацию о состоянии очереди на экран.

Инструкции:
\begin{enumerate}
    \item Создайте класс Queue с методом \_\_init\_\_, который инициализирует пустую очередь (список \_trunk). Принимает параметр enqueue\_less\_than\_prev=False. Если True, то элемент добавляется только если он строго меньше предыдущего добавленного элемента (первый — всегда).
    \item Создайте метод enqueue, который добавляет элемент, только если enqueue\_less\_than\_prev=False или очередь пуста или element < последний\_элемент.
    \item Создайте метод dequeue, который удаляет и возвращает первый элемент. Если очередь пуста — выбрасывает IndexError("Очередь пуста").
    \item Создайте метод is\_empty, который возвращает True, если очередь пуста, и False в противном случае.
    \item Создайте экземпляр класса Queue с enqueue\_less\_than\_prev=True.
    \item Добавьте элементы: 10, 15 (не добавится), 8, 9 (не добавится), 5, 7 (не добавится).
    \item Выведите текущее состояние очереди.
    \item Вызовите dequeue, выведите удаленный элемент.
    \item Выведите обновленное состояние очереди.
\end{enumerate}

Пример использования:
\begin{lstlisting}[language=Python]
queue = Queue(enqueue_less_than_prev=True)
queue.enqueue(10)
queue.enqueue(15)  # нет
queue.enqueue(8)
queue.enqueue(9)   # нет
queue.enqueue(5)
queue.enqueue(7)   # нет

print("Current Queue:", queue._trunk)  # [10,8,5]

dequeued_item = queue.dequeue()
print("Dequeued item:", dequeued_item)  # 10

print("Updated Queue:", queue._trunk)  # [8,5]
\end{lstlisting}

\end{enumerate}



\end{document}

