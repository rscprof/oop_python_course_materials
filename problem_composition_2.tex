\begin{enumerate}
\item[1] \textbf{Расчёт площади стен в зависимости от наличия встроенных картин и зеркал}
\begin{enumerate}
    \item Создайте два класса: \texttt{PictureMirror} и \texttt{Room}.  
    Класс \texttt{PictureMirror} представляет отдельный встроенный элемент (картину или зеркало). Его конструктор принимает два аргумента:  
    \texttt{width} — ширина элемента в метрах (положительное дробное число),  
    \texttt{height} — высота элемента в метрах (положительное дробное число).  
    Объект этого класса хранит только эти два значения.  
    Класс \texttt{Room} описывает прямоугольную комнату. Его конструктор принимает три аргумента:  
    \texttt{width} — ширина комнаты в метрах,  
    \texttt{length} — длина комнаты в метрах,  
    \texttt{height} — высота стен в метрах.  
    Все значения должны быть положительными. Объект \texttt{Room} хранит геометрические размеры комнаты и список объектов \texttt{PictureMirror}, изначально пустой.

    \item В классе \texttt{Room} реализуйте следующие методы:  
    \begin{itemize}
        \item \texttt{add\_item(self, item: PictureMirror) -> None} — добавляет переданный объект \texttt{PictureMirror} в внутренний список встроенных элементов комнаты. Метод не проверяет, помещается ли элемент на стене; предполагается, что все элементы корректно размещены.
        \item \texttt{get\_area\_to\_cover(self) -> float} — вычисляет и возвращает площадь стен, подлежащую отделке. Общая площадь стен комнаты рассчитывается по формуле \(2 \cdot \texttt{height} \cdot (\texttt{width} + \texttt{length})\). Из этой площади вычитается суммарная площадь всех встроенных элементов (каждый элемент вносит вклад \(\texttt{width} \cdot \texttt{height}\)). Результат не может быть отрицательным: если суммарная площадь элементов превышает площадь стен, метод возвращает 0.0.
        \item \texttt{get\_panels\_count(self, panel\_width: float, panel\_height: float) -> int} — рассчитывает минимальное количество декоративных панелей, необходимых для отделки вычисленной ранее площади. Площадь одной панели равна \(\texttt{panel\_width} \cdot \texttt{panel\_height}\). Количество панелей определяется как результат деления площади под отделку на площадь одной панели, округлённый вверх до ближайшего целого (поскольку панели продаются только целиком).
    \end{itemize}

    \item Создайте три различных экземпляра класса \texttt{Room} с разными размерами и разным набором встроенных элементов (например, комната без элементов, комната с одной большой картиной, комната с несколькими зеркалами). Для каждого экземпляра вызовите методы \texttt{add\_item} (при необходимости), \texttt{get\_area\_to\_cover} и \texttt{get\_panels\_count}, чтобы продемонстрировать корректность реализации.

    \item Запросите у пользователя данные для одной комнаты: ширину, длину и высоту комнаты (все — дробные числа), а также ширину и высоту одной декоративной панели (дробные числа).

    \item Выведите на экран два значения: площадь стен под отделку (в квадратных метрах, с дробной частью) и минимальное количество необходимых панелей (целое число, округлённое вверх).
\end{enumerate}

\item[2] \textbf{Расчёт площади стен в зависимости от наличия встроенных настенных устройств}
\begin{enumerate}
    \item Создайте два класса: \texttt{WallDevice} и \texttt{Room}.  
    Класс \texttt{WallDevice} представляет отдельное настенное устройство (например, панель управления). Его конструктор принимает два аргумента:  
    \texttt{width} — ширина устройства в метрах (положительное дробное число),  
    \texttt{height} — высота устройства в метрах (положительное дробное число).  
    Объект хранит только эти два значения.  
    Класс \texttt{Room} описывает прямоугольную комнату. Его конструктор принимает три аргумента:  
    \texttt{width} — ширина комнаты в метрах,  
    \texttt{length} — длина комнаты в метрах,  
    \texttt{height} — высота стен в метрах.  
    Все значения должны быть положительными. Объект \texttt{Room} хранит размеры комнаты и список объектов \texttt{WallDevice}, изначально пустой.

    \item В классе \texttt{Room} реализуйте следующие методы:  
    \begin{itemize}
        \item \texttt{add\_device(self, dev: WallDevice) -> None} — добавляет переданный объект \texttt{WallDevice} в список встроенных устройств комнаты.
        \item \texttt{get\_area\_to\_cover(self) -> float} — вычисляет площадь стен под отделку: из общей площади стен \(2 \cdot \texttt{height} \cdot (\texttt{width} + \texttt{length})\) вычитается суммарная площадь всех устройств. Результат не может быть меньше нуля.
        \item \texttt{get\_tiles\_count(self, tile\_width: float, tile\_height: float) -> int} — рассчитывает количество керамических плиток, необходимых для облицовки. Площадь одной плитки равна \(\texttt{tile\_width} \cdot \texttt{tile\_height}\). Количество плиток — это результат деления площади под отделку на площадь плитки, округлённый вверх до целого числа.
    \end{itemize}

    \item Создайте три различных экземпляра класса \texttt{Room} с разными параметрами и разным числом устройств. Для каждого вызовите методы для получения площади и количества плиток.

    \item Запросите у пользователя размеры комнаты (ширина, длина, высота) и размеры одной плитки (ширина и высота), все — дробные числа.

    \item Выведите площадь стен под облицовку (м²) и минимальное количество плиток (целое число, округлённое вверх).
\end{enumerate}

\item[3] \textbf{Расчёт площади стен в зависимости от наличия встроенных светильников и бра}
\begin{enumerate}
    \item Создайте два класса: \texttt{Lamp} и \texttt{Room}.  
    Класс \texttt{Lamp} представляет один настенный светильник или бра. Его конструктор принимает:  
    \texttt{width} — ширина светильника в метрах,  
    \texttt{height} — высота светильника в метрах.  
    Оба значения — положительные дробные числа.  
    Класс \texttt{Room} описывает комнату. Его конструктор принимает:  
    \texttt{width}, \texttt{length}, \texttt{height} — размеры комнаты в метрах (все положительные).  
    Объект \texttt{Room} хранит эти размеры и список объектов \texttt{Lamp}.

    \item В классе \texttt{Room} реализуйте методы:  
    \begin{itemize}
        \item \texttt{add\_lamp(self, lamp: Lamp) -> None} — добавляет светильник в список встроенных элементов.
        \item \texttt{get\_area\_to\_cover(self) -> float} — возвращает площадь стен без учёта площадей всех светильников. Общая площадь стен: \(2 \cdot \texttt{height} \cdot (\texttt{width} + \texttt{length})\). Из неё вычитается сумма площадей всех светильников. Результат $\geqslant$ 0.
        \item \texttt{get\_wallpaper\_rolls(self, roll\_width: float, roll\_length: float) -> int} — вычисляет количество рулонов обоев. Площадь одного рулона: \(\texttt{roll\_width} \cdot \texttt{roll\_length}\). Количество рулонов — частное от деления площади под оклейку на площадь рулона, округлённое вверх до целого.
    \end{itemize}

    \item Создайте три разных экземпляра \texttt{Room} (с разным числом светильников) и протестируйте методы.

    \item Запросите у пользователя размеры комнаты и размеры одного рулона обоев (все — дробные числа).

    \item Выведите площадь под оклейку (м²) и количество рулонов обоев (целое число, округлённое вверх).
\end{enumerate}

\item[4] \textbf{Расчёт площади стен в зависимости от наличия встроенных полок и стеллажей}
\begin{enumerate}
    \item Создайте два класса: \texttt{Shelf} и \texttt{Room}.  
    Класс \texttt{Shelf} описывает одну полку или стеллаж. Его конструктор принимает:  
    \texttt{width} — ширина в метрах,  
    \texttt{height} — высота в метрах (оба — положительные дробные числа).  
    Класс \texttt{Room} описывает комнату с параметрами:  
    \texttt{width}, \texttt{length}, \texttt{height} — размеры комнаты в метрах.  
    Объект \texttt{Room} хранит список объектов \texttt{Shelf}.

    \item В классе \texttt{Room} реализуйте методы:  
    \begin{itemize}
        \item \texttt{add\_shelf(self, shelf: Shelf) -> None} — добавляет полку в комнату.
        \item \texttt{get\_area\_to\_cover(self) -> float} — площадь стен под покраску: общая площадь стен минус суммарная площадь всех полок (результат $\geqslant$ 0).
        \item \texttt{get\_paint\_liters(self, coverage: float) -> float} — вычисляет необходимый объём краски в литрах. Аргумент \texttt{coverage} задаёт, сколько квадратных метров можно покрыть одним литром краски (м²/л). Объём краски = площадь под покраску / \texttt{coverage}. Результат может быть дробным, так как краску можно купить нецелыми банками.
    \end{itemize}

    \item Создайте три различных комнаты с разным числом полок и проверьте работу методов.

    \item Запросите у пользователя размеры комнаты и значение \texttt{coverage} (дробное число).

    \item Выведите площадь под покраску (м²) и необходимое количество литров краски (с дробной частью).
\end{enumerate}

\item[5] \textbf{Расчёт площади стен в зависимости от наличия встроенных розеток и выключателей}
\begin{enumerate}
    \item Создайте два класса: \texttt{SocketSwitch} и \texttt{Room}.  
    Класс \texttt{SocketSwitch} представляет одну розетку или выключатель. Его конструктор принимает:  
    \texttt{width} — ширина в метрах,  
    \texttt{height} — высота в метрах (положительные дробные числа).  
    Класс \texttt{Room} описывает комнату с размерами \texttt{width}, \texttt{length}, \texttt{height} (все — положительные дробные числа) и хранит список объектов \texttt{SocketSwitch}.

    \item В классе \texttt{Room} реализуйте методы:  
    \begin{itemize}
        \item \texttt{add\_electrical(self, el: SocketSwitch) -> None} — добавляет электроарматуру в комнату.
        \item \texttt{get\_area\_to\_cover(self) -> float} — площадь стен под штукатурку: общая площадь стен минус сумма площадей всех розеток и выключателей (результат $\geqslant$ 0).
        \item \texttt{get\_plaster\_bags(self, bag\_coverage: float) -> int} — количество мешков штукатурки. Аргумент \texttt{bag\_coverage} — сколько квадратных метров покрывает один мешок (м²/мешок). Количество мешков = площадь под штукатурку / \texttt{bag\_coverage}, округлённое вверх до целого.
    \end{itemize}

    \item Создайте три разных экземпляра \texttt{Room} с разным числом электроустройств и протестируйте методы.

    \item Запросите у пользователя размеры комнаты и \texttt{bag\_coverage} (дробное число).

    \item Выведите площадь под штукатурку (м²) и количество мешков (целое число, округлённое вверх).
\end{enumerate}

\item[6] \textbf{Расчёт площади стен в зависимости от наличия встроенных вентиляционных решёток}
\begin{enumerate}
    \item Создайте два класса: \texttt{VentGrille} и \texttt{Room}.  
    Класс \texttt{VentGrille} описывает одну вентиляционную решётку. Его конструктор принимает:  
    \texttt{width} — ширина решётки в метрах,  
    \texttt{height} — высота решётки в метрах (положительные дробные числа).  
    Класс \texttt{Room} описывает комнату с размерами \texttt{width}, \texttt{length}, \texttt{height} и хранит список объектов \texttt{VentGrille}.

    \item В классе \texttt{Room} реализуйте методы:  
    \begin{itemize}
        \item \texttt{add\_vent(self, vent: VentGrille) -> None} — добавляет решётку в комнату.
        \item \texttt{get\_area\_to\_cover(self) -> float} — площадь стен под обшивку: общая площадь стен минус сумма площадей всех решёток (результат $\geqslant$ 0).
        \item \texttt{get\_panel\_sheets(self, sheet\_width: float, sheet\_height: float) -> int} — количество листов панелей. Площадь одного листа = \(\texttt{sheet\_width} \cdot \texttt{sheet\_height}\). Количество листов — частное от деления площади под обшивку на площадь листа, округлённое вверх до целого.
    \end{itemize}

    \item Создайте три различных комнаты с разным числом решёток и проверьте методы.

    \item Запросите у пользователя размеры комнаты и размеры одного листа панели (все — дробные числа).

    \item Выведите площадь под обшивку (м²) и количество листов (целое число, округлённое вверх).
\end{enumerate}

\item[7] \textbf{Расчёт площади стен в зависимости от наличия встроенных настенных кондиционеров}
\begin{enumerate}
    \item Создайте два класса: \texttt{WallAC} и \texttt{Room}.  
    Класс \texttt{WallAC} представляет один настенный кондиционер. Его конструктор принимает:  
    \texttt{width} — ширина в метрах,  
    \texttt{height} — высота в метрах (положительные дробные числа).  
    Класс \texttt{Room} описывает комнату с размерами \texttt{width}, \texttt{length}, \texttt{height} и хранит список объектов \texttt{WallAC}.

    \item В классе \texttt{Room} реализуйте методы:  
    \begin{itemize}
        \item \texttt{add\_ac(self, ac: WallAC) -> None} — добавляет кондиционер в комнату.
        \item \texttt{get\_area\_to\_cover(self) -> float} — площадь стен под оклейку обоями: общая площадь стен минус сумма площадей всех кондиционеров (результат $\geqslant$ 0).
        \item \texttt{get\_wallpaper\_rolls(self, roll\_width: float, roll\_length: float) -> int} — количество рулонов обоев. Площадь рулона = \(\texttt{roll\_width} \cdot \texttt{roll\_length}\). Количество рулонов — частное от деления площади под оклейку на площадь рулона, округлённое вверх до целого.
    \end{itemize}

    \item Создайте три разных экземпляра \texttt{Room} с разным числом кондиционеров и протестируйте методы.

    \item Запросите у пользователя размеры комнаты и размеры рулона обоев (все — дробные числа).

    \item Выведите площадь под оклейку (м²) и количество рулонов (целое число, округлённое вверх).
\end{enumerate}

\item[8] \textbf{Расчёт площади стен в зависимости от наличия встроенных настенных экранов}
\begin{enumerate}
    \item Создайте два класса: \texttt{WallScreen} и \texttt{Room}.  
    Класс \texttt{WallScreen} описывает один настенный экран. Его конструктор принимает:  
    \texttt{width} — ширина в метрах,  
    \texttt{height} — высота в метрах (положительные дробные числа).  
    Класс \texttt{Room} описывает комнату с размерами \texttt{width}, \texttt{length}, \texttt{height} и хранит список объектов \texttt{WallScreen}.

    \item В классе \texttt{Room} реализуйте методы:  
    \begin{itemize}
        \item \texttt{add\_screen(self, scr: WallScreen) -> None} — добавляет экран в комнату.
        \item \texttt{get\_area\_to\_cover(self) -> float} — площадь стен под покраску: общая площадь стен минус сумма площадей всех экранов (результат $\geqslant$ 0).
        \item \texttt{get\_paint\_cans(self, can\_coverage: float) -> int} — количество банок краски. Аргумент \texttt{can\_coverage} — сколько квадратных метров покрывает одна банка (м²/банка). Количество банок = площадь под покраску / \texttt{can\_coverage}, округлённое вверх до целого.
    \end{itemize}

    \item Создайте три различных комнаты с разным числом экранов и проверьте работу методов.

    \item Запросите у пользователя размеры комнаты и \texttt{can\_coverage} (дробное число).

    \item Выведите площадь под покраску (м²) и количество банок (целое число, округлённое вверх).
\end{enumerate}

\item[9] \textbf{Расчёт площади стен в зависимости от наличия встроенных настенных сейфов}
\begin{enumerate}
    \item Создайте два класса: \texttt{WallSafe} и \texttt{Room}.  
    Класс \texttt{WallSafe} представляет один настенный сейф. Его конструктор принимает:  
    \texttt{width} — ширина в метрах,  
    \texttt{height} — высота в метрах (положительные дробные числа).  
    Класс \texttt{Room} описывает комнату с размерами \texttt{width}, \texttt{length}, \texttt{height} и хранит список объектов \texttt{WallSafe}.

    \item В классе \texttt{Room} реализуйте методы:  
    \begin{itemize}
        \item \texttt{add\_safe(self, safe: WallSafe) -> None} — добавляет сейф в комнату.
        \item \texttt{get\_area\_to\_cover(self) -> float} — площадь стен под облицовку: общая площадь стен минус сумма площадей всех сейфов (результат $\geqslant$ 0).
        \item \texttt{get\_tiles\_count(self, tile\_width: float, tile\_height: float) -> int} — количество плиток. Площадь одной плитки = \(\texttt{tile\_width} \cdot \texttt{tile\_height}\). Количество плиток — частное от деления площади под облицовку на площадь плитки, округлённое вверх до целого.
    \end{itemize}

    \item Создайте три разных экземпляра \texttt{Room} с разным числом сейфов и протестируйте методы.

    \item Запросите у пользователя размеры комнаты и размеры плитки (все — дробные числа).

    \item Выведите площадь под облицовку (м²) и количество плиток (целое число, округлённое вверх).
\end{enumerate}

\item[10] \textbf{Расчёт площади стен в зависимости от наличия встроенных настенных досок}
\begin{enumerate}
    \item Создайте два класса: \texttt{WallBoard} и \texttt{Room}.  
    Класс \texttt{WallBoard} описывает одну настенную доску. Его конструктор принимает:  
    \texttt{width} — ширина в метрах,  
    \texttt{height} — высота в метрах (положительные дробные числа).  
    Класс \texttt{Room} описывает комнату с размерами \texttt{width}, \texttt{length}, \texttt{height} и хранит список объектов \texttt{WallBoard}.

    \item В классе \texttt{Room} реализуйте методы:  
    \begin{itemize}
        \item \texttt{add\_board(self, board: WallBoard) -> None} — добавляет доску в комнату.
        \item \texttt{get\_area\_to\_cover(self) -> float} — площадь стен под драпировку: общая площадь стен минус сумма площадей всех досок (результат $\geqslant$ 0).
        \item \texttt{get\_fabric\_meters(self, fabric\_width: float) -> float} — необходимая длина ткани в метрах. Аргумент \texttt{fabric\_width} — ширина ткани в метрах. Длина ткани = площадь под драпировку / \texttt{fabric\_width}. Результат может быть дробным, так как ткань продаётся погонными метрами.
    \end{itemize}

    \item Создайте три различных комнаты с разным числом досок и проверьте методы.

    \item Запросите у пользователя размеры комнаты и ширину ткани (дробные числа).

    \item Выведите площадь под драпировку (м²) и количество метров ткани (с дробной частью).
\end{enumerate}

\item[11] \textbf{Расчёт площади стен в зависимости от наличия встроенных настенных календарей}
\begin{enumerate}
    \item Создайте два класса: \texttt{WallCalendar} и \texttt{Room}.  
    Класс \texttt{WallCalendar} представляет один настенный календарь. Его конструктор принимает:  
    \texttt{width} — ширина в метрах,  
    \texttt{height} — высота в метрах (положительные дробные числа).  
    Класс \texttt{Room} описывает комнату с размерами \texttt{width}, \texttt{length}, \texttt{height} и хранит список объектов \texttt{WallCalendar}.

    \item В классе \texttt{Room} реализуйте методы:  
    \begin{itemize}
        \item \texttt{add\_calendar(self, cal: WallCalendar) -> None} — добавляет календарь в комнату.
        \item \texttt{get\_area\_to\_cover(self) -> float} — площадь стен под оклейку: общая площадь стен минус сумма площадей всех календарей (результат $\geqslant$ 0).
        \item \texttt{get\_wallpaper\_rolls(self, roll\_width: float, roll\_length: float) -> int} — количество рулонов обоев. Площадь рулона = \(\texttt{roll\_width} \cdot \texttt{roll\_length}\). Количество рулонов — частное от деления площади под оклейку на площадь рулона, округлённое вверх до целого.
    \end{itemize}

    \item Создайте три разных экземпляра \texttt{Room} с разным числом календарей и протестируйте методы.

    \item Запросите у пользователя размеры комнаты и размеры рулона обоев (все — дробные числа).

    \item Выведите площадь под оклейку (м²) и количество рулонов (целое число, округлённое вверх).
\end{enumerate}

\item[12] \textbf{Расчёт площади стен в зависимости от наличия встроенных настенных карт}
\begin{enumerate}
    \item Создайте два класса: \texttt{WallMap} и \texttt{Room}.  
    Класс \texttt{WallMap} описывает одну настенную карту. Его конструктор принимает:  
    \texttt{width} — ширина в метрах,  
    \texttt{height} — высота в метрах (положительные дробные числа).  
    Класс \texttt{Room} описывает комнату с размерами \texttt{width}, \texttt{length}, \texttt{height} и хранит список объектов \texttt{WallMap}.

    \item В классе \texttt{Room} реализуйте методы:  
    \begin{itemize}
        \item \texttt{add\_map(self, map: WallMap) -> None} — добавляет карту в комнату.
        \item \texttt{get\_area\_to\_cover(self) -> float} — площадь стен под покраску: общая площадь стен минус сумма площадей всех карт (результат $\geqslant$ 0).
        \item \texttt{get\_paint\_liters(self, coverage: float) -> float} — объём краски в литрах. Аргумент \texttt{coverage} — расход краски (м²/л). Объём = площадь под покраску / \texttt{coverage}. Результат может быть дробным.
    \end{itemize}

    \item Создайте три различных комнаты с разным числом карт и проверьте методы.

    \item Запросите у пользователя размеры комнаты и \texttt{coverage} (дробное число).

    \item Выведите площадь под покраску (м²) и количество литров краски (с дробной частью).
\end{enumerate}

\item[13] \textbf{Расчёт площади стен в зависимости от наличия встроенных настенных террариумов}
\begin{enumerate}
    \item Создайте два класса: \texttt{WallTerrarium} и \texttt{Room}.  
    Класс \texttt{WallTerrarium} представляет один настенный террариум. Его конструктор принимает:  
    \texttt{width} — ширина в метрах,  
    \texttt{height} — высота в метрах (положительные дробные числа).  
    Класс \texttt{Room} описывает комнату с размерами \texttt{width}, \texttt{length}, \texttt{height} и хранит список объектов \texttt{WallTerrarium}.

    \item В классе \texttt{Room} реализуйте методы:  
    \begin{itemize}
        \item \texttt{add\_terrarium(self, terr: WallTerrarium) -> None} — добавляет террариум в комнату.
        \item \texttt{get\_area\_to\_cover(self) -> float} — площадь стен под отделку: общая площадь стен минус сумма площадей всех террариумов (результат $\geqslant$ 0).
        \item \texttt{get\_panels\_count(self, panel\_width: float, panel\_height: float) -> int} — количество декоративных панелей. Площадь одной панели = \(\texttt{panel\_width} \cdot \texttt{panel\_height}\). Количество панелей — частное от деления площади под отделку на площадь панели, округлённое вверх до целого.
    \end{itemize}

    \item Создайте три разных экземпляра \texttt{Room} с разным числом террариумов и протестируйте методы.

    \item Запросите у пользователя размеры комнаты и размеры панели (все — дробные числа).

    \item Выведите площадь под отделку (м²) и количество панелей (целое число, округлённое вверх).
\end{enumerate}

\item[14] \textbf{Расчёт площади стен в зависимости от наличия встроенных настенных аквариумов}
\begin{enumerate}
    \item Создайте два класса: \texttt{WallAquarium} и \texttt{Room}.  
    Класс \texttt{WallAquarium} описывает один настенный аквариум. Его конструктор принимает:  
    \texttt{width} — ширина в метрах,  
    \texttt{height} — высота в метрах (положительные дробные числа).  
    Класс \texttt{Room} описывает комнату с размерами \texttt{width}, \texttt{length}, \texttt{height} и хранит список объектов \texttt{WallAquarium}.

    \item В классе \texttt{Room} реализуйте методы:  
    \begin{itemize}
        \item \texttt{add\_aquarium(self, aq: WallAquarium) -> None} — добавляет аквариум в комнату.
        \item \texttt{get\_area\_to\_cover(self) -> float} — площадь стен под облицовку: общая площадь стен минус сумма площадей всех аквариумов (результат $\geqslant$ 0).
        \item \texttt{get\_tiles\_count(self, tile\_width: float, tile\_height: float) -> int} — количество плиток. Площадь одной плитки = \(\texttt{tile\_width} \cdot \texttt{tile\_height}\). Количество плиток — частное от деления площади под облицовку на площадь плитки, округлённое вверх до целого.
    \end{itemize}

    \item Создайте три различных комнаты с разным числом аквариумов и проверьте методы.

    \item Запросите у пользователя размеры комнаты и размеры плитки (все — дробные числа).

    \item Выведите площадь под облицовку (м²) и количество плиток (целое число, округлённое вверх).
\end{enumerate}

\item[15] \textbf{Расчёт площади стен в зависимости от наличия встроенных настенных динамиков}
\begin{enumerate}
    \item Создайте два класса: \texttt{WallSpeaker} и \texttt{Room}.  
    Класс \texttt{WallSpeaker} представляет один настенный динамик. Его конструктор принимает:  
    \texttt{width} — ширина в метрах,  
    \texttt{height} — высота в метрах (положительные дробные числа).  
    Класс \texttt{Room} описывает комнату с размерами \texttt{width}, \texttt{length}, \texttt{height} и хранит список объектов \texttt{WallSpeaker}.

    \item В классе \texttt{Room} реализуйте методы:  
    \begin{itemize}
        \item \texttt{add\_speaker(self, sp: WallSpeaker) -> None} — добавляет динамик в комнату.
        \item \texttt{get\_area\_to\_cover(self) -> float} — площадь стен под оклейку: общая площадь стен минус сумма площадей всех динамиков (результат $\geqslant$ 0).
        \item \texttt{get\_wallpaper\_rolls(self, roll\_width: float, roll\_length: float) -> int} — количество рулонов обоев. Площадь рулона = \(\texttt{roll\_width} \cdot \texttt{roll\_length}\). Количество рулонов — частное от деления площади под оклейку на площадь рулона, округлённое вверх до целого.
    \end{itemize}

    \item Создайте три разных экземпляра \texttt{Room} с разным числом динамиков и протестируйте методы.

    \item Запросите у пользователя размеры комнаты и размеры рулона обоев (все — дробные числа).

    \item Выведите площадь под оклейку (м²) и количество рулонов (целое число, округлённое вверх).
\end{enumerate}

\item[16] \textbf{Расчёт площади стен в зависимости от наличия встроенных настенных датчиков}
\begin{enumerate}
    \item Создайте два класса: \texttt{WallSensor} и \texttt{Room}.  
    Класс \texttt{WallSensor} описывает один настенный датчик. Его конструктор принимает:  
    \texttt{width} — ширина в метрах,  
    \texttt{height} — высота в метрах (положительные дробные числа).  
    Класс \texttt{Room} описывает комнату с размерами \texttt{width}, \texttt{length}, \texttt{height} и хранит список объектов \texttt{WallSensor}.

    \item В классе \texttt{Room} реализуйте методы:  
    \begin{itemize}
        \item \texttt{add\_sensor(self, sens: WallSensor) -> None} — добавляет датчик в комнату.
        \item \texttt{get\_area\_to\_cover(self) -> float} — площадь стен под покраску: общая площадь стен минус сумма площадей всех датчиков (результат $\geqslant$ 0).
        \item \texttt{get\_paint\_cans(self, can\_coverage: float) -> int} — количество банок краски. Аргумент \texttt{can\_coverage} — покрытие одной банки (м²/банка). Количество банок = площадь под покраску / \texttt{can\_coverage}, округлённое вверх до целого.
    \end{itemize}

    \item Создайте три различных комнаты с разным числом датчиков и проверьте методы.

    \item Запросите у пользователя размеры комнаты и \texttt{can\_coverage} (дробное число).

    \item Выведите площадь под покраску (м²) и количество банок (целое число, округлённое вверх).
\end{enumerate}

\item[17] \textbf{Расчёт площади стен в зависимости от наличия встроенных настенных панно}
\begin{enumerate}
    \item Создайте два класса: \texttt{WallPanel} и \texttt{Room}.  
    Класс \texttt{WallPanel} представляет одно настенное панно. Его конструктор принимает:  
    \texttt{width} — ширина в метрах,  
    \texttt{height} — высота в метрах (положительные дробные числа).  
    Класс \texttt{Room} описывает комнату с размерами \texttt{width}, \texttt{length}, \texttt{height} и хранит список объектов \texttt{WallPanel}.

    \item В классе \texttt{Room} реализуйте методы:  
    \begin{itemize}
        \item \texttt{add\_panel(self, p: WallPanel) -> None} — добавляет панно в комнату.
        \item \texttt{get\_area\_to\_cover(self) -> float} — площадь стен под драпировку: общая площадь стен минус сумма площадей всех панно (результат $\geqslant$ 0).
        \item \texttt{get\_fabric\_meters(self, fabric\_width: float) -> float} — длина ткани в метрах. Аргумент \texttt{fabric\_width} — ширина ткани. Длина = площадь под драпировку / \texttt{fabric\_width}. Результат может быть дробным.
    \end{itemize}

    \item Создайте три разных экземпляра \texttt{Room} с разным числом панно и протестируйте методы.

    \item Запросите у пользователя размеры комнаты и ширину ткани (дробные числа).

    \item Выведите площадь под драпировку (м²) и количество метров ткани (с дробной частью).
\end{enumerate}

\item[18] \textbf{Расчёт площади стен в зависимости от наличия встроенных настенных рамок}
\begin{enumerate}
    \item Создайте два класса: \texttt{WallFrame} и \texttt{Room}.  
    Класс \texttt{WallFrame} описывает одну настенную рамку. Его конструктор принимает:  
    \texttt{width} — ширина в метрах,  
    \texttt{height} — высота в метрах (положительные дробные числа).  
    Класс \texttt{Room} описывает комнату с размерами \texttt{width}, \texttt{length}, \texttt{height} и хранит список объектов \texttt{WallFrame}.

    \item В классе \texttt{Room} реализуйте методы:  
    \begin{itemize}
        \item \texttt{add\_frame(self, f: WallFrame) -> None} — добавляет рамку в комнату.
        \item \texttt{get\_area\_to\_cover(self) -> float} — площадь стен под оклейку: общая площадь стен минус сумма площадей всех рамок (результат $\geqslant$ 0).
        \item \texttt{get\_wallpaper\_rolls(self, roll\_width: float, roll\_length: float) -> int} — количество рулонов обоев. Площадь рулона = \(\texttt{roll\_width} \cdot \texttt{roll\_length}\). Количество рулонов — частное от деления площади под оклейку на площадь рулона, округлённое вверх до целого.
    \end{itemize}

    \item Создайте три различных комнаты с разным числом рамок и проверьте методы.

    \item Запросите у пользователя размеры комнаты и размеры рулона обоев (все — дробные числа).

    \item Выведите площадь под оклейку (м²) и количество рулонов (целое число, округлённое вверх).
\end{enumerate}

\item[19] \textbf{Расчёт площади стен в зависимости от наличия встроенных настенных витрин}
\begin{enumerate}
    \item Создайте два класса: \texttt{WallShowcase} и \texttt{Room}.  
    Класс \texttt{WallShowcase} представляет одну настенную витрину. Его конструктор принимает:  
    \texttt{width} — ширина в метрах,  
    \texttt{height} — высота в метрах (положительные дробные числа).  
    Класс \texttt{Room} описывает комнату с размерами \texttt{width}, \texttt{length}, \texttt{height} и хранит список объектов \texttt{WallShowcase}.

    \item В классе \texttt{Room} реализуйте методы:  
    \begin{itemize}
        \item \texttt{add\_showcase(self, sc: WallShowcase) -> None} — добавляет витрину в комнату.
        \item \texttt{get\_area\_to\_cover(self) -> float} — площадь стен под облицовку: общая площадь стен минус сумма площадей всех витрин (результат $\geqslant$ 0).
        \item \texttt{get\_tiles\_count(self, tile\_width: float, tile\_height: float) -> int} — количество плиток. Площадь одной плитки = \(\texttt{tile\_width} \cdot \texttt{tile\_height}\). Количество плиток — частное от деления площади под облицовку на площадь плитки, округлённое вверх до целого.
    \end{itemize}

    \item Создайте три разных экземпляра \texttt{Room} с разным числом витрин и протестируйте методы.

    \item Запросите у пользователя размеры комнаты и размеры плитки (все — дробные числа).

    \item Выведите площадь под облицовку (м²) и количество плиток (целое число, округлённое вверх).
\end{enumerate}

\item[20] \textbf{Расчёт площади стен в зависимости от наличия встроенных настенных кронштейнов}
\begin{enumerate}
    \item Создайте два класса: \texttt{WallBracket} и \texttt{Room}.  
    Класс \texttt{WallBracket} описывает один настенный кронштейн. Его конструктор принимает:  
    \texttt{width} — ширина в метрах,  
    \texttt{height} — высота в метрах (положительные дробные числа).  
    Класс \texttt{Room} описывает комнату с размерами \texttt{width}, \texttt{length}, \texttt{height} и хранит список объектов \texttt{WallBracket}.

    \item В классе \texttt{Room} реализуйте методы:  
    \begin{itemize}
        \item \texttt{add\_bracket(self, br: WallBracket) -> None} — добавляет кронштейн в комнату.
        \item \texttt{get\_area\_to\_cover(self) -> float} — площадь стен под покраску: общая площадь стен минус сумма площадей всех кронштейнов (результат $\geqslant$ 0).
        \item \texttt{get\_paint\_liters(self, coverage: float) -> float} — объём краски в литрах. Аргумент \texttt{coverage} — расход (м²/л). Объём = площадь под покраску / \texttt{coverage}. Результат может быть дробным.
    \end{itemize}

    \item Создайте три различных комнаты с разным числом кронштейнов и проверьте методы.

    \item Запросите у пользователя размеры комнаты и \texttt{coverage} (дробное число).

    \item Выведите площадь под покраску (м²) и количество литров краски (с дробной частью).
\end{enumerate}

\item[21] \textbf{Расчёт площади стен в зависимости от наличия встроенных настенных жалюзи}
\begin{enumerate}
    \item Создайте два класса: \texttt{WallBlind} и \texttt{Room}.  
    Класс \texttt{WallBlind} представляет одни настенные жалюзи. Его конструктор принимает:  
    \texttt{width} — ширина в метрах,  
    \texttt{height} — высота в метрах (положительные дробные числа).  
    Класс \texttt{Room} описывает комнату с размерами \texttt{width}, \texttt{length}, \texttt{height} и хранит список объектов \texttt{WallBlind}.

    \item В классе \texttt{Room} реализуйте методы:  
    \begin{itemize}
        \item \texttt{add\_blind(self, bl: WallBlind) -> None} — добавляет жалюзи в комнату.
        \item \texttt{get\_area\_to\_cover(self) -> float} — площадь стен под драпировку: общая площадь стен минус сумма площадей всех жалюзи (результат $\geqslant$ 0).
        \item \texttt{get\_fabric\_meters(self, fabric\_width: float) -> float} — длина ткани в метрах. Аргумент \texttt{fabric\_width} — ширина ткани. Длина = площадь под драпировку / \texttt{fabric\_width}. Результат может быть дробным.
    \end{itemize}

    \item Создайте три разных экземпляра \texttt{Room} с разным числом жалюзи и протестируйте методы.

    \item Запросите у пользователя размеры комнаты и ширину ткани (дробные числа).

    \item Выведите площадь под драпировку (м²) и количество метров ткани (с дробной частью).
\end{enumerate}

\item[22] \textbf{Расчёт площади стен в зависимости от наличия встроенных настенных флагов}
\begin{enumerate}
    \item Создайте два класса: \texttt{WallFlag} и \texttt{Room}.  
    Класс \texttt{WallFlag} описывает один настенный флаг. Его конструктор принимает:  
    \texttt{width} — ширина в метрах,  
    \texttt{height} — высота в метрах (положительные дробные числа).  
    Класс \texttt{Room} описывает комнату с размерами \texttt{width}, \texttt{length}, \texttt{height} и хранит список объектов \texttt{WallFlag}.

    \item В классе \texttt{Room} реализуйте методы:  
    \begin{itemize}
        \item \texttt{add\_flag(self, fl: WallFlag) -> None} — добавляет флаг в комнату.
        \item \texttt{get\_area\_to\_cover(self) -> float} — площадь стен под оклейку: общая площадь стен минус сумма площадей всех флагов (результат $\geqslant$ 0).
        \item \texttt{get\_wallpaper\_rolls(self, roll\_width: float, roll\_length: float) -> int} — количество рулонов обоев. Площадь рулона = \(\texttt{roll\_width} \cdot \texttt{roll\_length}\). Количество рулонов — частное от деления площади под оклейку на площадь рулона, округлённое вверх до целого.
    \end{itemize}

    \item Создайте три различных комнаты с разным числом флагов и проверьте методы.

    \item Запросите у пользователя размеры комнаты и размеры рулона обоев (все — дробные числа).

    \item Выведите площадь под оклейку (м²) и количество рулонов (целое число, округлённое вверх).
\end{enumerate}

\item[23] \textbf{Расчёт площади стен в зависимости от наличия встроенных грифельных досок}
\begin{enumerate}
    \item Создайте два класса: \texttt{Chalkboard} и \texttt{Room}.  
    Класс \texttt{Chalkboard} представляет одну грифельную доску. Его конструктор принимает:  
    \texttt{width} — ширина в метрах,  
    \texttt{height} — высота в метрах (положительные дробные числа).  
    Класс \texttt{Room} описывает комнату с размерами \texttt{width}, \texttt{length}, \texttt{height} и хранит список объектов \texttt{Chalkboard}.

    \item В классе \texttt{Room} реализуйте методы:  
    \begin{itemize}
        \item \texttt{add\_board(self, cb: Chalkboard) -> None} — добавляет доску в комнату.
        \item \texttt{get\_area\_to\_cover(self) -> float} — площадь стен под покраску: общая площадь стен минус сумма площадей всех досок (результат $\geqslant$ 0).
        \item \texttt{get\_paint\_cans(self, can\_coverage: float) -> int} — количество банок краски. Аргумент \texttt{can\_coverage} — покрытие одной банки (м²/банка). Количество банок = площадь под покраску / \texttt{can\_coverage}, округлённое вверх до целого.
    \end{itemize}

    \item Создайте три разных экземпляра \texttt{Room} с разным числом досок и протестируйте методы.

    \item Запросите у пользователя размеры комнаты и \texttt{can\_coverage} (дробное число).

    \item Выведите площадь под покраску (м²) и количество банок (целое число, округлённое вверх).
\end{enumerate}

\item[24] \textbf{Расчёт площади стен в зависимости от наличия встроенных маркерных досок}
\begin{enumerate}
    \item Создайте два класса: \texttt{Whiteboard} и \texttt{Room}.  
    Класс \texttt{Whiteboard} описывает одну маркерную доску. Его конструктор принимает:  
    \texttt{width} — ширина в метрах,  
    \texttt{height} — высота в метрах (положительные дробные числа).  
    Класс \texttt{Room} описывает комнату с размерами \texttt{width}, \texttt{length}, \texttt{height} и хранит список объектов \texttt{Whiteboard}.

    \item В классе \texttt{Room} реализуйте методы:  
    \begin{itemize}
        \item \texttt{add\_board(self, wb: Whiteboard) -> None} — добавляет доску в комнату.
        \item \texttt{get\_area\_to\_cover(self) -> float} — площадь стен под отделку: общая площадь стен минус сумма площадей всех досок (результат $\geqslant$ 0).
        \item \texttt{get\_panels\_count(self, panel\_width: float, panel\_height: float) -> int} — количество декоративных панелей. Площадь одной панели = \(\texttt{panel\_width} \cdot \texttt{panel\_height}\). Количество панелей — частное от деления площади под отделку на площадь панели, округлённое вверх до целого.
    \end{itemize}

    \item Создайте три различных комнаты с разным числом досок и проверьте методы.

    \item Запросите у пользователя размеры комнаты и размеры панели (все — дробные числа).

    \item Выведите площадь под отделку (м²) и количество панелей (целое число, округлённое вверх).
\end{enumerate}

\item[25] \textbf{Расчёт площади стен в зависимости от наличия встроенных зеркал}
\begin{enumerate}
    \item Создайте два класса: \texttt{Mirror} и \texttt{Room}.  
    Класс \texttt{Mirror} представляет одно зеркало. Его конструктор принимает:  
    \texttt{width} — ширина в метрах,  
    \texttt{height} — высота в метрах (положительные дробные числа).  
    Класс \texttt{Room} описывает комнату с размерами \texttt{width}, \texttt{length}, \texttt{height} и хранит список объектов \texttt{Mirror}.

    \item В классе \texttt{Room} реализуйте методы:  
    \begin{itemize}
        \item \texttt{add\_mirror(self, m: Mirror) -> None} — добавляет зеркало в комнату.
        \item \texttt{get\_area\_to\_cover(self) -> float} — площадь стен под облицовку: общая площадь стен минус сумма площадей всех зеркал (результат $\geqslant$ 0).
        \item \texttt{get\_tiles\_count(self, tile\_width: float, tile\_height: float) -> int} — количество плиток. Площадь одной плитки = \(\texttt{tile\_width} \cdot \texttt{tile\_height}\). Количество плиток — частное от деления площади под облицовку на площадь плитки, округлённое вверх до целого.
    \end{itemize}

    \item Создайте три разных экземпляра \texttt{Room} с разным числом зеркал и протестируйте методы.

    \item Запросите у пользователя размеры комнаты и размеры плитки (все — дробные числа).

    \item Выведите площадь под облицовку (м²) и количество плиток (целое число, округлённое вверх).
\end{enumerate}

\item[26] \textbf{Расчёт площади стен в зависимости от наличия встроенных часов}
\begin{enumerate}
    \item Создайте два класса: \texttt{WallClock} и \texttt{Room}.  
    Класс \texttt{WallClock} описывает одни настенные часы. Его конструктор принимает:  
    \texttt{width} — ширина в метрах,  
    \texttt{height} — высота в метрах (положительные дробные числа).  
    Класс \texttt{Room} описывает комнату с размерами \texttt{width}, \texttt{length}, \texttt{height} и хранит список объектов \texttt{WallClock}.

    \item В классе \texttt{Room} реализуйте методы:  
    \begin{itemize}
        \item \texttt{add\_clock(self, cl: WallClock) -> None} — добавляет часы в комнату.
        \item \texttt{get\_area\_to\_cover(self) -> float} — площадь стен под оклейку: общая площадь стен минус сумма площадей всех часов (результат $\geqslant$ 0).
        \item \texttt{get\_wallpaper\_rolls(self, roll\_width: float, roll\_length: float) -> int} — количество рулонов обоев. Площадь рулона = \(\texttt{roll\_width} \cdot \texttt{roll\_length}\). Количество рулонов — частное от деления площади под оклейку на площадь рулона, округлённое вверх до целого.
    \end{itemize}

    \item Создайте три различных комнаты с разным числом часов и проверьте методы.

    \item Запросите у пользователя размеры комнаты и размеры рулона обоев (все — дробные числа).

    \item Выведите площадь под оклейку (м²) и количество рулонов (целое число, округлённое вверх).
\end{enumerate}

\item[27] \textbf{Расчёт площади стен в зависимости от наличия встроенных термометров}
\begin{enumerate}
    \item Создайте два класса: \texttt{Thermometer} и \texttt{Room}.  
    Класс \texttt{Thermometer} представляет один настенный термометр. Его конструктор принимает:  
    \texttt{width} — ширина в метрах,  
    \texttt{height} — высота в метрах (положительные дробные числа).  
    Класс \texttt{Room} описывает комнату с размерами \texttt{width}, \texttt{length}, \texttt{height} и хранит список объектов \texttt{Thermometer}.

    \item В классе \texttt{Room} реализуйте методы:  
    \begin{itemize}
        \item \texttt{add\_thermometer(self, t: Thermometer) -> None} — добавляет термометр в комнату.
        \item \texttt{get\_area\_to\_cover(self) -> float} — площадь стен под покраску: общая площадь стен минус сумма площадей всех термометров (результат $\geqslant$ 0).
        \item \texttt{get\_paint\_liters(self, coverage: float) -> float} — объём краски в литрах. Аргумент \texttt{coverage} — расход (м²/л). Объём = площадь под покраску / \texttt{coverage}. Результат может быть дробным.
    \end{itemize}

    \item Создайте три разных экземпляра \texttt{Room} с разным числом термометров и протестируйте методы.

    \item Запросите у пользователя размеры комнаты и \texttt{coverage} (дробное число).

    \item Выведите площадь под покраску (м²) и количество литров краски (с дробной частью).
\end{enumerate}

\item[28] \textbf{Расчёт площади стен в зависимости от наличия встроенных барометров}
\begin{enumerate}
    \item Создайте два класса: \texttt{Barometer} и \texttt{Room}.  
    Класс \texttt{Barometer} описывает один настенный барометр. Его конструктор принимает:  
    \texttt{width} — ширина в метрах,  
    \texttt{height} — высота в метрах (положительные дробные числа).  
    Класс \texttt{Room} описывает комнату с размерами \texttt{width}, \texttt{length}, \texttt{height} и хранит список объектов \texttt{Barometer}.

    \item В классе \texttt{Room} реализуйте методы:  
    \begin{itemize}
        \item \texttt{add\_barometer(self, b: Barometer) -> None} — добавляет барометр в комнату.
        \item \texttt{get\_area\_to\_cover(self) -> float} — площадь стен под драпировку: общая площадь стен минус сумма площадей всех барометров (результат $\geqslant$ 0).
        \item \texttt{get\_fabric\_meters(self, fabric\_width: float) -> float} — длина ткани в метрах. Аргумент \texttt{fabric\_width} — ширина ткани. Длина = площадь под драпировку / \texttt{fabric\_width}. Результат может быть дробным.
    \end{itemize}

    \item Создайте три различных комнаты с разным числом барометров и проверьте методы.

    \item Запросите у пользователя размеры комнаты и ширину ткани (дробные числа).

    \item Выведите площадь под драпировку (м²) и количество метров ткани (с дробной частью).
\end{enumerate}

\item[29] \textbf{Расчёт площади стен в зависимости от наличия встроенных гидрометров}
\begin{enumerate}
    \item Создайте два класса: \texttt{Hygrometer} и \texttt{Room}.  
    Класс \texttt{Hygrometer} представляет один настенный гидрометр. Его конструктор принимает:  
    \texttt{width} — ширина в метрах,  
    \texttt{height} — высота в метрах (положительные дробные числа).  
    Класс \texttt{Room} описывает комнату с размерами \texttt{width}, \texttt{length}, \texttt{height} и хранит список объектов \texttt{Hygrometer}.

    \item В классе \texttt{Room} реализуйте методы:  
    \begin{itemize}
        \item \texttt{add\_hygrometer(self, h: Hygrometer) -> None} — добавляет гидрометр в комнату.
        \item \texttt{get\_area\_to\_cover(self) -> float} — площадь стен под оклейку: общая площадь стен минус сумма площадей всех гидрометров (результат $\geqslant$ 0).
        \item \texttt{get\_wallpaper\_rolls(self, roll\_width: float, roll\_length: float) -> int} — количество рулонов обоев. Площадь рулона = \(\texttt{roll\_width} \cdot \texttt{roll\_length}\). Количество рулонов — частное от деления площади под оклейку на площадь рулона, округлённое вверх до целого.
    \end{itemize}

    \item Создайте три разных экземпляра \texttt{Room} с разным числом гидрометров и протестируйте методы.

    \item Запросите у пользователя размеры комнаты и размеры рулона обоев (все — дробные числа).

    \item Выведите площадь под оклейку (м²) и количество рулонов (целое число, округлённое вверх).
\end{enumerate}

\item[30] \textbf{Расчёт площади стен в зависимости от наличия встроенных настенных растений}
\begin{enumerate}
    \item Создайте два класса: \texttt{WallPlant} и \texttt{Room}.  
    Класс \texttt{WallPlant} описывает одно настенное растение (в кашпо или модуле). Его конструктор принимает:  
    \texttt{width} — ширина в метрах,  
    \texttt{height} — высота в метрах (положительные дробные числа).  
    Класс \texttt{Room} описывает комнату с размерами \texttt{width}, \texttt{length}, \texttt{height} и хранит список объектов \texttt{WallPlant}.

    \item В классе \texttt{Room} реализуйте методы:  
    \begin{itemize}
        \item \texttt{add\_plant(self, p: WallPlant) -> None} — добавляет растение в комнату.
        \item \texttt{get\_area\_to\_cover(self) -> float} — площадь стен под покраску: общая площадь стен минус сумма площадей всех растений (результат $\geqslant$ 0).
        \item \texttt{get\_paint\_cans(self, can\_coverage: float) -> int} — количество банок краски. Аргумент \texttt{can\_coverage} — покрытие одной банки (м²/банка). Количество банок = площадь под покраску / \texttt{can\_coverage}, округлённое вверх до целого.
    \end{itemize}

    \item Создайте три различных комнаты с разным числом растений и проверьте методы.

    \item Запросите у пользователя размеры комнаты и \texttt{can\_coverage} (дробное число).

    \item Выведите площадь под покраску (м²) и количество банок (целое число, округлённое вверх).
\end{enumerate}

\item[31] \textbf{Расчёт площади стен в зависимости от наличия встроенных настенных фонарей}
\begin{enumerate}
    \item Создайте два класса: \texttt{WallLantern} и \texttt{Room}.  
    Класс \texttt{WallLantern} представляет один настенный фонарь. Его конструктор принимает:  
    \texttt{width} — ширина в метрах,  
    \texttt{height} — высота в метрах (положительные дробные числа).  
    Класс \texttt{Room} описывает комнату с размерами \texttt{width}, \texttt{length}, \texttt{height} и хранит список объектов \texttt{WallLantern}.

    \item В классе \texttt{Room} реализуйте методы:  
    \begin{itemize}
        \item \texttt{add\_lantern(self, l: WallLantern) -> None} — добавляет фонарь в комнату.
        \item \texttt{get\_area\_to\_cover(self) -> float} — площадь стен под отделку: общая площадь стен минус сумма площадей всех фонарей (результат $\geqslant$ 0).
        \item \texttt{get\_panels\_count(self, panel\_width: float, panel\_height: float) -> int} — количество декоративных панелей. Площадь одной панели = \(\texttt{panel\_width} \cdot \texttt{panel\_height}\). Количество панелей — частное от деления площади под отделку на площадь панели, округлённое вверх до целого.
    \end{itemize}

    \item Создайте три разных экземпляра \texttt{Room} с разным числом фонарей и протестируйте методы.

    \item Запросите у пользователя размеры комнаты и размеры панели (все — дробные числа).

    \item Выведите площадь под отделку (м²) и количество панелей (целое число, округлённое вверх).
\end{enumerate}

\item[32] \textbf{Расчёт площади стен в зависимости от наличия встроенных настенных вентиляторов}
\begin{enumerate}
    \item Создайте два класса: \texttt{WallFan} и \texttt{Room}.  
    Класс \texttt{WallFan} описывает один настенный вентилятор. Его конструктор принимает:  
    \texttt{width} — ширина в метрах,  
    \texttt{height} — высота в метрах (положительные дробные числа).  
    Класс \texttt{Room} описывает комнату с размерами \texttt{width}, \texttt{length}, \texttt{height} и хранит список объектов \texttt{WallFan}.

    \item В классе \texttt{Room} реализуйте методы:  
    \begin{itemize}
        \item \texttt{add\_fan(self, f: WallFan) -> None} — добавляет вентилятор в комнату.
        \item \texttt{get\_area\_to\_cover(self) -> float} — площадь стен под облицовку: общая площадь стен минус сумма площадей всех вентиляторов (результат $\geqslant$ 0).
        \item \texttt{get\_tiles\_count(self, tile\_width: float, tile\_height: float) -> int} — количество плиток. Площадь одной плитки = \(\texttt{tile\_width} \cdot \texttt{tile\_height}\). Количество плиток — частное от деления площади под облицовку на площадь плитки, округлённое вверх до целого.
    \end{itemize}

    \item Создайте три различных комнаты с разным числом вентиляторов и проверьте методы.

    \item Запросите у пользователя размеры комнаты и размеры плитки (все — дробные числа).

    \item Выведите площадь под облицовку (м²) и количество плиток (целое число, округлённое вверх).
\end{enumerate}

\item[33] \textbf{Расчёт площади стен в зависимости от наличия встроенных увлажнителей}
\begin{enumerate}
    \item Создайте два класса: \texttt{WallHumidifier} и \texttt{Room}.  
    Класс \texttt{WallHumidifier} представляет один настенный увлажнитель. Его конструктор принимает:  
    \texttt{width} — ширина в метрах,  
    \texttt{height} — высота в метрах (положительные дробные числа).  
    Класс \texttt{Room} описывает комнату с размерами \texttt{width}, \texttt{length}, \texttt{height} и хранит список объектов \texttt{WallHumidifier}.

    \item В классе \texttt{Room} реализуйте методы:  
    \begin{itemize}
        \item \texttt{add\_humidifier(self, h: WallHumidifier) -> None} — добавляет увлажнитель в комнату.
        \item \texttt{get\_area\_to\_cover(self) -> float} — площадь стен под оклейку: общая площадь стен минус сумма площадей всех увлажнителей (результат $\geqslant$ 0).
        \item \texttt{get\_wallpaper\_rolls(self, roll\_width: float, roll\_length: float) -> int} — количество рулонов обоев. Площадь рулона = \(\texttt{roll\_width} \cdot \texttt{roll\_length}\). Количество рулонов — частное от деления площади под оклейку на площадь рулона, округлённое вверх до целого.
    \end{itemize}

    \item Создайте три разных экземпляра \texttt{Room} с разным числом увлажнителей и протестируйте методы.

    \item Запросите у пользователя размеры комнаты и размеры рулона обоев (все — дробные числа).

    \item Выведите площадь под оклейку (м²) и количество рулонов (целое число, округлённое вверх).
\end{enumerate}

\item[34] \textbf{Расчёт площади стен в зависимости от наличия встроенных обогревателей}
\begin{enumerate}
    \item Создайте два класса: \texttt{WallHeater} и \texttt{Room}.  
    Класс \texttt{WallHeater} описывает один настенный обогреватель. Его конструктор принимает:  
    \texttt{width} — ширина в метрах,  
    \texttt{height} — высота в метрах (положительные дробные числа).  
    Класс \texttt{Room} описывает комнату с размерами \texttt{width}, \texttt{length}, \texttt{height} и хранит список объектов \texttt{WallHeater}.

    \item В классе \texttt{Room} реализуйте методы:  
    \begin{itemize}
        \item \texttt{add\_heater(self, h: WallHeater) -> None} — добавляет обогреватель в комнату.
        \item \texttt{get\_area\_to\_cover(self) -> float} — площадь стен под покраску: общая площадь стен минус сумма площадей всех обогревателей (результат $\geqslant$ 0).
        \item \texttt{get\_paint\_liters(self, coverage: float) -> float} — объём краски в литрах. Аргумент \texttt{coverage} — расход (м²/л). Объём = площадь под покраску / \texttt{coverage}. Результат может быть дробным.
    \end{itemize}

    \item Создайте три различных комнаты с разным числом обогревателей и проверьте методы.

    \item Запросите у пользователя размеры комнаты и \texttt{coverage} (дробное число).

    \item Выведите площадь под покраску (м²) и количество литров краски (с дробной частью).
\end{enumerate}

\item[35] \textbf{Расчёт площади стен в зависимости от наличия окон и дверей}
\begin{enumerate}
    \item Создайте два класса: \texttt{WinDoor} и \texttt{Room}.  
    Класс \texttt{WinDoor} представляет один проём (окно или дверь). Его конструктор принимает:  
    \texttt{width} — ширина проёма в метрах,  
    \texttt{height} — высота проёма в метрах (положительные дробные числа).  
    Класс \texttt{Room} описывает комнату с размерами \texttt{width}, \texttt{length}, \texttt{height} и хранит список объектов \texttt{WinDoor}.

    \item В классе \texttt{Room} реализуйте методы:  
    \begin{itemize}
        \item \texttt{add\_windoor(self, wd: WinDoor) -> None} — добавляет проём в комнату.
        \item \texttt{get\_area\_to\_cover(self) -> float} — площадь стен под оклейку: общая площадь стен минус сумма площадей всех проёмов (результат $\geqslant$ 0).
        \item \texttt{get\_rolls\_count(self, roll\_width: float, roll\_length: float) -> int} — количество рулонов обоев. Площадь одного рулона = \(\texttt{roll\_width} \cdot \texttt{roll\_length}\). Количество рулонов — частное от деления площади под оклейку на площадь рулона, округлённое вверх до целого.
    \end{itemize}

    \item Создайте три разных экземпляра \texttt{Room} с разным числом проёмов и протестируйте методы.

    \item Запросите у пользователя размеры комнаты и размеры рулона обоев (все — дробные числа).

    \item Выведите площадь под оклейку (м²) и количество рулонов обоев (целое число, округлённое вверх).
\end{enumerate}
\end{enumerate}