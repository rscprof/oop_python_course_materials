\subsubsection{Задача 4}

\begin{enumerate}
\item[1] Написать программу на Python, которая создает класс \texttt{ShoppingCart} для представления корзины покупок. Класс должен содержать методы для добавления и удаления товаров, а также вычисления общего количества. Программа также должна создавать экземпляр класса \texttt{ShoppingCart}, добавлять товары в корзину, удалять товары из корзины и выводить информацию о корзине на экран.

\begin{itemize}
    \item Создайте класс \texttt{ShoppingCart} с методом \texttt{\_\_init\_\_}, который создает пустой список товаров.
    \item Создайте метод \texttt{add\_item}, который принимает название товара и количество в качестве аргументов и добавляет их в список товаров.
    \item Создайте метод \texttt{remove\_item}, который удаляет товар из списка товаров по его названию.
    \item Создайте метод \texttt{calculate\_total}, который вычисляет и возвращает общее количество всех товаров в корзине.
    \item Создайте экземпляр класса \texttt{ShoppingCart} и добавьте товары в корзину.
    \item Выведите информацию о текущих товарах в корзине на экран.
    \item Выведите общее количество всех товаров в корзине на экран.
    \item Удалите товар из корзины и выведите обновленную информацию о товарах в корзине на экран.
    \item Выведите общее количество всех товаров в корзине после удаления товара на экран.
\end{itemize}

\textbf{Пример использования:}

\begin{verbatim}
cart = ShoppingCart()
cart.add_item("Картофель", 100)
cart.add_item("Капуста", 200)
cart.add_item("Апельсин", 150)
print("Число товаров в корзине:")
for item in cart.items:
    print(item[0], "-", item[1])
total_qty = cart.calculate_total()
print("Общее количество:", total_qty)
cart.remove_item("Апельсин")
print("Обновление числа покупок в корзине после удаления апельсина:")
for item in cart.items:
    print(item[0], "-", item[1])
total_qty = cart.calculate_total()
print("Общее количество:", total_qty)
\end{verbatim}

\textbf{Вывод:}
\begin{verbatim}
Число товаров в корзине:
Картофель - 100
Капуста - 200
Апельсин - 150
Общее количество: 450
Обновление числа покупок в корзине после удаления апельсина:
Картофель - 100
Капуста - 200
Общее количество: 300
\end{verbatim}

\item[2] Написать программу на Python, которая создает класс \texttt{BookCollection} для представления коллекции книг. Класс должен содержать методы для добавления и удаления книг, а также подсчета общего количества страниц. Программа также должна создавать экземпляр класса \texttt{BookCollection}, добавлять книги в коллекцию, удалять книги из коллекции и выводить информацию о коллекции на экран.

\begin{itemize}
    \item Создайте класс \texttt{BookCollection} с методом \texttt{\_\_init\_\_}, который создает пустой список книг.
    \item Создайте метод \texttt{add\_book}, который принимает название книги и количество страниц в качестве аргументов и добавляет их в список книг.
    \item Создайте метод \texttt{remove\_book}, который удаляет книгу из списка по её названию.
    \item Создайте метод \texttt{total\_pages}, который вычисляет и возвращает общее количество страниц всех книг в коллекции.
    \item Создайте экземпляр класса \texttt{BookCollection} и добавьте книги в коллекцию.
    \item Выведите информацию о текущих книгах в коллекции на экран.
    \item Выведите общее количество страниц всех книг на экран.
    \item Удалите книгу из коллекции и выведите обновленную информацию о книгах на экран.
    \item Выведите общее количество страниц после удаления книги на экран.
\end{itemize}

\textbf{Пример использования:}

\begin{verbatim}
collection = BookCollection()
collection.add_book("Война и мир", 1225)
collection.add_book("Преступление и наказание", 671)
collection.add_book("Мастер и Маргарита", 480)
print("Книги в коллекции:")
for book in collection.books:
    print(book[0], "-", book[1], "стр.")
total = collection.total_pages()
print("Общее количество страниц:", total)
collection.remove_book("Преступление и наказание")
print("Книги после удаления 'Преступления и наказания':")
for book in collection.books:
    print(book[0], "-", book[1], "стр.")
total = collection.total_pages()
print("Общее количество страниц:", total)
\end{verbatim}

\textbf{Вывод:}
\begin{verbatim}
Книги в коллекции:
Война и мир - 1225 стр.
Преступление и наказание - 671 стр.
Мастер и Маргарита - 480 стр.
Общее количество страниц: 2376
Книги после удаления 'Преступления и наказания':
Война и мир - 1225 стр.
Мастер и Маргарита - 480 стр.
Общее количество страниц: 1705
\end{verbatim}

\item[3] Написать программу на Python, которая создает класс \texttt{Inventory} для представления складского запаса. Класс должен содержать методы для добавления и удаления предметов, а также вычисления общего количества единиц товара. Программа также должна создавать экземпляр класса \texttt{Inventory}, добавлять предметы на склад, удалять предметы со склада и выводить информацию о запасах на экран.

\begin{itemize}
    \item Создайте класс \texttt{Inventory} с методом \texttt{\_\_init\_\_}, который создает пустой список предметов.
    \item Создайте метод \texttt{add\_item}, который принимает название предмета и количество единиц в качестве аргументов и добавляет их в список.
    \item Создайте метод \texttt{remove\_item}, который удаляет предмет из списка по его названию.
    \item Создайте метод \texttt{total\_count}, который вычисляет и возвращает общее количество всех единиц товара на складе.
    \item Создайте экземпляр класса \texttt{Inventory} и добавьте предметы на склад.
    \item Выведите информацию о текущих предметах на складе на экран.
    \item Выведите общее количество единиц товара на экран.
    \item Удалите предмет со склада и выведите обновленную информацию о предметах на экран.
    \item Выведите общее количество единиц товара после удаления предмета на экран.
\end{itemize}

\textbf{Пример использования:}

\begin{verbatim}
inv = Inventory()
inv.add_item("Молотки", 50)
inv.add_item("Отвертки", 120)
inv.add_item("Гвозди", 1000)
print("Предметы на складе:")
for item in inv.items:
    print(item[0], "-", item[1])
total = inv.total_count()
print("Общее количество:", total)
inv.remove_item("Отвертки")
print("Предметы после удаления отверток:")
for item in inv.items:
    print(item[0], "-", item[1])
total = inv.total_count()
print("Общее количество:", total)
\end{verbatim}

\textbf{Вывод:}
\begin{verbatim}
Предметы на складе:
Молотки - 50
Отвертки - 120
Гвозди - 1000
Общее количество: 1170
Предметы после удаления отверток:
Молотки - 50
Гвозди - 1000
Общее количество: 1050
\end{verbatim}

\item[4] Написать программу на Python, которая создает класс \texttt{Playlist} для представления музыкального плейлиста. Класс должен содержать методы для добавления и удаления треков, а также подсчета общего времени воспроизведения. Программа также должна создавать экземпляр класса \texttt{Playlist}, добавлять треки в плейлист, удалять треки из плейлиста и выводить информацию о плейлисте на экран.

\begin{itemize}
    \item Создайте класс \texttt{Playlist} с методом \texttt{\_\_init\_\_}, который создает пустой список треков.
    \item Создайте метод \texttt{add\_track}, который принимает название трека и его длительность (в секундах) в качестве аргументов и добавляет их в список.
    \item Создайте метод \texttt{remove\_track}, который удаляет трек из списка по его названию.
    \item Создайте метод \texttt{total\_duration}, который вычисляет и возвращает общую длительность всех треков в плейлисте (в секундах).
    \item Создайте экземпляр класса \texttt{Playlist} и добавьте треки в плейлист.
    \item Выведите информацию о текущих треках в плейлисте на экран.
    \item Выведите общую длительность всех треков на экран.
    \item Удалите трек из плейлиста и выведите обновленную информацию о треках на экран.
    \item Выведите общую длительность после удаления трека на экран.
\end{itemize}

\textbf{Пример использования:}

\begin{verbatim}
pl = Playlist()
pl.add_track("Bohemian Rhapsody", 354)
pl.add_track("Imagine", 183)
pl.add_track("Smells Like Teen Spirit", 301)
print("Треки в плейлисте:")
for track in pl.tracks:
    print(track[0], "-", track[1], "сек.")
total = pl.total_duration()
print("Общая длительность:", total, "сек.")
pl.remove_track("Imagine")
print("Треки после удаления 'Imagine':")
for track in pl.tracks:
    print(track[0], "-", track[1], "сек.")
total = pl.total_duration()
print("Общая длительность:", total, "сек.")
\end{verbatim}

\textbf{Вывод:}
\begin{verbatim}
Треки в плейлисте:
Bohemian Rhapsody - 354 сек.
Imagine - 183 сек.
Smells Like Teen Spirit - 301 сек.
Общая длительность: 838 сек.
Треки после удаления 'Imagine':
Bohemian Rhapsody - 354 сек.
Smells Like Teen Spirit - 301 сек.
Общая длительность: 655 сек.
\end{verbatim}

\item[5] Написать программу на Python, которая создает класс \texttt{StudentGrades} для представления оценок студента. Класс должен содержать методы для добавления и удаления оценок, а также вычисления среднего балла. Программа также должна создавать экземпляр класса \texttt{StudentGrades}, добавлять оценки, удалять оценки и выводить информацию об успеваемости на экран.

\begin{itemize}
    \item Создайте класс \texttt{StudentGrades} с методом \texttt{\_\_init\_\_}, который создает пустой список оценок.
    \item Создайте метод \texttt{add\_grade}, который принимает название предмета и оценку в качестве аргументов и добавляет их в список.
    \item Создайте метод \texttt{remove\_grade}, который удаляет оценку по названию предмета.
    \item Создайте метод \texttt{average\_grade}, который вычисляет и возвращает средний балл по всем предметам.
    \item Создайте экземпляр класса \texttt{StudentGrades} и добавьте оценки по разным предметам.
    \item Выведите информацию о текущих оценках на экран.
    \item Выведите средний балл на экран.
    \item Удалите оценку по одному из предметов и выведите обновленную информацию.
    \item Выведите средний балл после удаления оценки на экран.
\end{itemize}

\textbf{Пример использования:}

\begin{verbatim}
grades = StudentGrades()
grades.add_grade("Математика", 5)
grades.add_grade("Физика", 4)
grades.add_grade("Информатика", 5)
print("Оценки студента:")
for subject, grade in grades.grades:
    print(subject, "-", grade)
avg = grades.average_grade()
print("Средний балл:", round(avg, 2))
grades.remove_grade("Физика")
print("Оценки после удаления Физики:")
for subject, grade in grades.grades:
    print(subject, "-", grade)
avg = grades.average_grade()
print("Средний балл:", round(avg, 2))
\end{verbatim}

\textbf{Вывод:}
\begin{verbatim}
Оценки студента:
Математика - 5
Физика - 4
Информатика - 5
Средний балл: 4.67
Оценки после удаления Физики:
Математика - 5
Информатика - 5
Средний балл: 5.0
\end{verbatim}

\item[6] Написать программу на Python, которая создает класс \texttt{TaskList} для представления списка задач. Класс должен содержать методы для добавления и удаления задач, а также подсчета общего количества задач. Программа также должна создавать экземпляр класса \texttt{TaskList}, добавлять задачи, удалять задачи и выводить информацию о списке задач на экран.

\begin{itemize}
    \item Создайте класс \texttt{TaskList} с методом \texttt{\_\_init\_\_}, который создает пустой список задач.
    \item Создайте метод \texttt{add\_task}, который принимает описание задачи и приоритет (целое число) в качестве аргументов и добавляет их в список.
    \item Создайте метод \texttt{remove\_task}, который удаляет задачу из списка по её описанию.
    \item Создайте метод \texttt{task\_count}, который возвращает общее количество задач в списке.
    \item Создайте экземпляр класса \texttt{TaskList} и добавьте несколько задач.
    \item Выведите информацию о текущих задачах на экран.
    \item Выведите общее количество задач на экран.
    \item Удалите одну из задач и выведите обновленный список задач.
    \item Выведите общее количество задач после удаления на экран.
\end{itemize}

\textbf{Пример использования:}

\begin{verbatim}
tasks = TaskList()
tasks.add_task("Написать отчет", 1)
tasks.add_task("Проверить почту", 3)
tasks.add_task("Подготовить презентацию", 2)
print("Список задач:")
for desc, priority in tasks.tasks:
    print(desc, "(приоритет", priority, ")")
count = tasks.task_count()
print("Всего задач:", count)
tasks.remove_task("Проверить почту")
print("Список задач после удаления 'Проверить почту':")
for desc, priority in tasks.tasks:
    print(desc, "(приоритет", priority, ")")
count = tasks.task_count()
print("Всего задач:", count)
\end{verbatim}

\textbf{Вывод:}
\begin{verbatim}
Список задач:
Написать отчет (приоритет 1 )
Проверить почту (приоритет 3 )
Подготовить презентацию (приоритет 2 )
Всего задач: 3
Список задач после удаления 'Проверить почту':
Написать отчет (приоритет 1 )
Подготовить презентацию (приоритет 2 )
Всего задач: 2
\end{verbatim}

\item[7] Написать программу на Python, которая создает класс \texttt{BankAccount} для представления банковского счета. Класс должен содержать методы для добавления и снятия средств, а также получения текущего баланса. Программа также должна создавать экземпляр класса \texttt{BankAccount}, выполнять операции пополнения и снятия, и выводить информацию о балансе на экран.

\begin{itemize}
    \item Создайте класс \texttt{BankAccount} с методом \texttt{\_\_init\_\_}, который инициализирует баланс нулём.
    \item Создайте метод \texttt{deposit}, который принимает сумму и увеличивает баланс на неё.
    \item Создайте метод \texttt{withdraw}, который принимает сумму и уменьшает баланс на неё (если достаточно средств).
    \item Создайте метод \texttt{get\_balance}, который возвращает текущий баланс.
    \item Создайте экземпляр класса \texttt{BankAccount}.
    \item Выполните несколько операций пополнения счета.
    \item Выведите текущий баланс на экран.
    \item Выполните операцию снятия средств и выведите обновленный баланс.
    \item Выведите окончательный баланс на экран.
\end{itemize}

\textbf{Пример использования:}

\begin{verbatim}
account = BankAccount()
account.deposit(1000)
account.deposit(500)
print("Баланс после пополнений:", account.get_balance())
account.withdraw(300)
print("Баланс после снятия 300:", account.get_balance())
account.withdraw(200)
print("Окончательный баланс:", account.get_balance())
\end{verbatim}

\textbf{Вывод:}
\begin{verbatim}
Баланс после пополнений: 1500
Баланс после снятия 300: 1200
Окончательный баланс: 1000
\end{verbatim}

\item[8] Написать программу на Python, которая создает класс \texttt{Library} для представления библиотеки. Класс должен содержать методы для добавления и удаления книг, а также подсчета общего количества книг. Программа также должна создавать экземпляр класса \texttt{Library}, добавлять книги, удалять книги и выводить информацию о фонде на экран.

\begin{itemize}
    \item Создайте класс \texttt{Library} с методом \texttt{\_\_init\_\_}, который создает пустой список книг.
    \item Создайте метод \texttt{add\_book}, который принимает название книги и автора в качестве аргументов и добавляет их в список.
    \item Создайте метод \texttt{remove\_book}, который удаляет книгу из списка по её названию.
    \item Создайте метод \texttt{book\_count}, который возвращает общее количество книг в библиотеке.
    \item Создайте экземпляр класса \texttt{Library} и добавьте несколько книг.
    \item Выведите информацию о текущих книгах на экран.
    \item Выведите общее количество книг на экран.
    \item Удалите одну из книг и выведите обновленный список.
    \item Выведите общее количество книг после удаления на экран.
\end{itemize}

\textbf{Пример использования:}

\begin{verbatim}
lib = Library()
lib.add_book("1984", "Джордж Оруэлл")
lib.add_book("Гарри Поттер", "Дж.К. Роулинг")
lib.add_book("Гордость и предубеждение", "Джейн Остин")
print("Книги в библиотеке:")
for title, author in lib.books:
    print(title, "—", author)
count = lib.book_count()
print("Всего книг:", count)
lib.remove_book("Гарри Поттер")
print("Книги после удаления 'Гарри Поттера':")
for title, author in lib.books:
    print(title, "—", author)
count = lib.book_count()
print("Всего книг:", count)
\end{verbatim}

\textbf{Вывод:}
\begin{verbatim}
Книги в библиотеке:
1984 — Джордж Оруэлл
Гарри Поттер — Дж.К. Роулинг
Гордость и предубеждение — Джейн Остин
Всего книг: 3
Книги после удаления 'Гарри Поттера':
1984 — Джордж Оруэлл
Гордость и предубеждение — Джейн Остин
Всего книг: 2
\end{verbatim}

\item[9] Написать программу на Python, которая создает класс \texttt{GroceryList} для представления списка покупок. Класс должен содержать методы для добавления и удаления продуктов, а также подсчета общего количества позиций. Программа также должна создавать экземпляр класса \texttt{GroceryList}, добавлять продукты, удалять продукты и выводить информацию о списке на экран.

\begin{itemize}
    \item Создайте класс \texttt{GroceryList} с методом \texttt{\_\_init\_\_}, который создает пустой список продуктов.
    \item Создайте метод \texttt{add\_product}, который принимает название продукта и количество в качестве аргументов и добавляет их в список.
    \item Создайте метод \texttt{remove\_product}, который удаляет продукт из списка по его названию.
    \item Создайте метод \texttt{total\_items}, который возвращает общее количество различных продуктов в списке.
    \item Создайте экземпляр класса \texttt{GroceryList} и добавьте несколько продуктов.
    \item Выведите информацию о текущих продуктах на экран.
    \item Выведите общее количество позиций на экран.
    \item Удалите один из продуктов и выведите обновленный список.
    \item Выведите общее количество позиций после удаления на экран.
\end{itemize}

\textbf{Пример использования:}

\begin{verbatim}
grocery = GroceryList()
grocery.add_product("Молоко", 2)
grocery.add_product("Хлеб", 1)
grocery.add_product("Яйца", 12)
print("Список покупок:")
for name, qty in grocery.products:
    print(name, "-", qty)
count = grocery.total_items()
print("Всего позиций:", count)
grocery.remove_product("Хлеб")
print("Список после удаления хлеба:")
for name, qty in grocery.products:
    print(name, "-", qty)
count = grocery.total_items()
print("Всего позиций:", count)
\end{verbatim}

\textbf{Вывод:}
\begin{verbatim}
Список покупок:
Молоко - 2
Хлеб - 1
Яйца - 12
Всего позиций: 3
Список после удаления хлеба:
Молоко - 2
Яйца - 12
Всего позиций: 2
\end{verbatim}

\item[10] Написать программу на Python, которая создает класс \texttt{ContactList} для представления списка контактов. Класс должен содержать методы для добавления и удаления контактов, а также подсчета общего количества контактов. Программа также должна создавать экземпляр класса \texttt{ContactList}, добавлять контакты, удалять контакты и выводить информацию о списке на экран.

\begin{itemize}
    \item Создайте класс \texttt{ContactList} с методом \texttt{\_\_init\_\_}, который создает пустой список контактов.
    \item Создайте метод \texttt{add\_contact}, который принимает имя и номер телефона в качестве аргументов и добавляет их в список.
    \item Создайте метод \texttt{remove\_contact}, который удаляет контакт из списка по имени.
    \item Создайте метод \texttt{contact\_count}, который возвращает общее количество контактов в списке.
    \item Создайте экземпляр класса \texttt{ContactList} и добавьте несколько контактов.
    \item Выведите информацию о текущих контактах на экран.
    \item Выведите общее количество контактов на экран.
    \item Удалите один из контактов и выведите обновленный список.
    \item Выведите общее количество контактов после удаления на экран.
\end{itemize}

\textbf{Пример использования:}

\begin{verbatim}
contacts = ContactList()
contacts.add_contact("Анна", "+79001234567")
contacts.add_contact("Борис", "+79007654321")
contacts.add_contact("Виктория", "+79001112233")
print("Контакты:")
for name, phone in contacts.contacts:
    print(name, "-", phone)
count = contacts.contact_count()
print("Всего контактов:", count)
contacts.remove_contact("Борис")
print("Контакты после удаления Бориса:")
for name, phone in contacts.contacts:
    print(name, "-", phone)
count = contacts.contact_count()
print("Всего контактов:", count)
\end{verbatim}

\textbf{Вывод:}
\begin{verbatim}
Контакты:
Анна - +79001234567
Борис - +79007654321
Виктория - +79001112233
Всего контактов: 3
Контакты после удаления Бориса:
Анна - +79001234567
Виктория - +79001112233
Всего контактов: 2
\end{verbatim}

\item[11] Написать программу на Python, которая создает класс \texttt{MovieCollection} для представления коллекции фильмов. Класс должен содержать методы для добавления и удаления фильмов, а также подсчета общего количества фильмов. Программа также должна создавать экземпляр класса \texttt{MovieCollection}, добавлять фильмы, удалять фильмы и выводить информацию о коллекции на экран.

\begin{itemize}
    \item Создайте класс \texttt{MovieCollection} с методом \texttt{\_\_init\_\_}, который создает пустой список фильмов.
    \item Создайте метод \texttt{add\_movie}, который принимает название фильма и год выпуска в качестве аргументов и добавляет их в список.
    \item Создайте метод \texttt{remove\_movie}, который удаляет фильм из списка по его названию.
    \item Создайте метод \texttt{movie\_count}, который возвращает общее количество фильмов в коллекции.
    \item Создайте экземпляр класса \texttt{MovieCollection} и добавьте несколько фильмов.
    \item Выведите информацию о текущих фильмах на экран.
    \item Выведите общее количество фильмов на экран.
    \item Удалите один из фильмов и выведите обновленный список.
    \item Выведите общее количество фильмов после удаления на экран.
\end{itemize}

\textbf{Пример использования:}

\begin{verbatim}
movies = MovieCollection()
movies.add_movie("Крёстный отец", 1972)
movies.add_movie("Побег из Шоушенка", 1994)
movies.add_movie("Тёмный рыцарь", 2008)
print("Фильмы в коллекции:")
for title, year in movies.movies:
    print(title, "(", year, ")")
count = movies.movie_count()
print("Всего фильмов:", count)
movies.remove_movie("Побег из Шоушенка")
print("Фильмы после удаления 'Побега из Шоушенка':")
for title, year in movies.movies:
    print(title, "(", year, ")")
count = movies.movie_count()
print("Всего фильмов:", count)
\end{verbatim}

\textbf{Вывод:}
\begin{verbatim}
Фильмы в коллекции:
Крёстный отец ( 1972 )
Побег из Шоушенка ( 1994 )
Тёмный рыцарь ( 2008 )
Всего фильмов: 3
Фильмы после удаления 'Побега из Шоушенка':
Крёстный отец ( 1972 )
Тёмный рыцарь ( 2008 )
Всего фильмов: 2
\end{verbatim}

\item[12] Написать программу на Python, которая создает класс \texttt{RecipeBook} для представления кулинарной книги. Класс должен содержать методы для добавления и удаления рецептов, а также подсчета общего количества рецептов. Программа также должна создавать экземпляр класса \texttt{RecipeBook}, добавлять рецепты, удалять рецепты и выводить информацию о книге на экран.

\begin{itemize}
    \item Создайте класс \texttt{RecipeBook} с методом \texttt{\_\_init\_\_}, который создает пустой список рецептов.
    \item Создайте метод \texttt{add\_recipe}, который принимает название блюда и время приготовления (в минутах) в качестве аргументов и добавляет их в список.
    \item Создайте метод \texttt{remove\_recipe}, который удаляет рецепт из списка по названию блюда.
    \item Создайте метод \texttt{recipe\_count}, который возвращает общее количество рецептов в книге.
    \item Создайте экземпляр класса \texttt{RecipeBook} и добавьте несколько рецептов.
    \item Выведите информацию о текущих рецептах на экран.
    \item Выведите общее количество рецептов на экран.
    \item Удалите один из рецептов и выведите обновленный список.
    \item Выведите общее количество рецептов после удаления на экран.
\end{itemize}

\textbf{Пример использования:}

\begin{verbatim}
recipes = RecipeBook()
recipes.add_recipe("Борщ", 60)
recipes.add_recipe("Омлет", 10)
recipes.add_recipe("Паста", 20)
print("Рецепты в книге:")
for dish, time in recipes.recipes:
    print(dish, "-", time, "мин.")
count = recipes.recipe_count()
print("Всего рецептов:", count)
recipes.remove_recipe("Омлет")
print("Рецепты после удаления омлета:")
for dish, time in recipes.recipes:
    print(dish, "-", time, "мин.")
count = recipes.recipe_count()
print("Всего рецептов:", count)
\end{verbatim}

\textbf{Вывод:}
\begin{verbatim}
Рецепты в книге:
Борщ - 60 мин.
Омлет - 10 мин.
Паста - 20 мин.
Всего рецептов: 3
Рецепты после удаления омлета:
Борщ - 60 мин.
Паста - 20 мин.
Всего рецептов: 2
\end{verbatim}

\item[13] Написать программу на Python, которая создает класс \texttt{CarGarage} для представления автосервиса. Класс должен содержать методы для добавления и удаления автомобилей, а также подсчета общего количества машин. Программа также должна создавать экземпляр класса \texttt{CarGarage}, добавлять автомобили, удалять автомобили и выводить информацию о гараже на экран.

\begin{itemize}
    \item Создайте класс \texttt{CarGarage} с методом \texttt{\_\_init\_\_}, который создает пустой список автомобилей.
    \item Создайте метод \texttt{add\_car}, который принимает марку и модель автомобиля в качестве аргументов и добавляет их в список.
    \item Создайте метод \texttt{remove\_car}, который удаляет автомобиль из списка по марке.
    \item Создайте метод \texttt{car\_count}, который возвращает общее количество автомобилей в гараже.
    \item Создайте экземпляр класса \texttt{CarGarage} и добавьте несколько автомобилей.
    \item Выведите информацию о текущих автомобилях на экран.
    \item Выведите общее количество машин на экран.
    \item Удалите один из автомобилей и выведите обновленный список.
    \item Выведите общее количество машин после удаления на экран.
\end{itemize}

\textbf{Пример использования:}

\begin{verbatim}
garage = CarGarage()
garage.add_car("Toyota", "Camry")
garage.add_car("BMW", "X5")
garage.add_car("Ford", "Focus")
print("Автомобили в гараже:")
for brand, model in garage.cars:
    print(brand, model)
count = garage.car_count()
print("Всего автомобилей:", count)
garage.remove_car("BMW")
print("Автомобили после удаления BMW:")
for brand, model in garage.cars:
    print(brand, model)
count = garage.car_count()
print("Всего автомобилей:", count)
\end{verbatim}

\textbf{Вывод:}
\begin{verbatim}
Автомобили в гараже:
Toyota Camry
BMW X5
Ford Focus
Всего автомобилей: 3
Автомобили после удаления BMW:
Toyota Camry
Ford Focus
Всего автомобилей: 2
\end{verbatim}

\item[14] Написать программу на Python, которая создает класс \texttt{PetStore} для представления зоомагазина. Класс должен содержать методы для добавления и удаления животных, а также подсчета общего количества питомцев. Программа также должна создавать экземпляр класса \texttt{PetStore}, добавлять животных, удалять животных и выводить информацию о магазине на экран.

\begin{itemize}
    \item Создайте класс \texttt{PetStore} с методом \texttt{\_\_init\_\_}, который создает пустой список животных.
    \item Создайте метод \texttt{add\_pet}, который принимает вид животного и количество в качестве аргументов и добавляет их в список.
    \item Создайте метод \texttt{remove\_pet}, который удаляет животное из списка по виду.
    \item Создайте метод \texttt{total\_pets}, который возвращает общее количество всех питомцев в магазине.
    \item Создайте экземпляр класса \texttt{PetStore} и добавьте несколько видов животных.
    \item Выведите информацию о текущих животных на экран.
    \item Выведите общее количество питомцев на экран.
    \item Удалите один из видов животных и выведите обновленный список.
    \item Выведите общее количество питомцев после удаления на экран.
\end{itemize}

\textbf{Пример использования:}

\begin{verbatim}
store = PetStore()
store.add_pet("Кошки", 5)
store.add_pet("Собаки", 3)
store.add_pet("Попугаи", 10)
print("Животные в магазине:")
for species, count in store.pets:
    print(species, "-", count)
total = store.total_pets()
print("Всего питомцев:", total)
store.remove_pet("Собаки")
print("Животные после удаления собак:")
for species, count in store.pets:
    print(species, "-", count)
total = store.total_pets()
print("Всего питомцев:", total)
\end{verbatim}

\textbf{Вывод:}
\begin{verbatim}
Животные в магазине:
Кошки - 5
Собаки - 3
Попугаи - 10
Всего питомцев: 18
Животные после удаления собак:
Кошки - 5
Попугаи - 10
Всего питомцев: 15
\end{verbatim}

\item[15] Написать программу на Python, которая создает класс \texttt{CourseRoster} для представления списка студентов на курсе. Класс должен содержать методы для добавления и удаления студентов, а также подсчета общего количества учащихся. Программа также должна создавать экземпляр класса \texttt{CourseRoster}, добавлять студентов, удалять студентов и выводить информацию о курсе на экран.

\begin{itemize}
    \item Создайте класс \texttt{CourseRoster} с методом \texttt{\_\_init\_\_}, который создает пустой список студентов.
    \item Создайте метод \texttt{enroll\_student}, который принимает имя студента и его ID в качестве аргументов и добавляет их в список.
    \item Создайте метод \texttt{drop\_student}, который удаляет студента из списка по имени.
    \item Создайте метод \texttt{student\_count}, который возвращает общее количество студентов на курсе.
    \item Создайте экземпляр класса \texttt{CourseRoster} и добавьте несколько студентов.
    \item Выведите информацию о текущих студентах на экран.
    \item Выведите общее количество студентов на экран.
    \item Удалите одного из студентов и выведите обновленный список.
    \item Выведите общее количество студентов после удаления на экран.
\end{itemize}

\textbf{Пример использования:}

\begin{verbatim}
roster = CourseRoster()
roster.enroll_student("Иван", 101)
roster.enroll_student("Мария", 102)
roster.enroll_student("Алексей", 103)
print("Студенты на курсе:")
for name, sid in roster.students:
    print(name, "(ID:", sid, ")")
count = roster.student_count()
print("Всего студентов:", count)
roster.drop_student("Мария")
print("Студенты после отчисления Марии:")
for name, sid in roster.students:
    print(name, "(ID:", sid, ")")
count = roster.student_count()
print("Всего студентов:", count)
\end{verbatim}

\textbf{Вывод:}
\begin{verbatim}
Студенты на курсе:
Иван (ID: 101 )
Мария (ID: 102 )
Алексей (ID: 103 )
Всего студентов: 3
Студенты после отчисления Марии:
Иван (ID: 101 )
Алексей (ID: 103 )
Всего студентов: 2
\end{verbatim}

\item[16] Написать программу на Python, которая создает класс \texttt{TravelItinerary} для представления туристического маршрута. Класс должен содержать методы для добавления и удаления мест, а также подсчета общего количества пунктов назначения. Программа также должна создавать экземпляр класса \texttt{TravelItinerary}, добавлять места, удалять места и выводить информацию о маршруте на экран.

\begin{itemize}
    \item Создайте класс \texttt{TravelItinerary} с методом \texttt{\_\_init\_\_}, который создает пустой список мест.
    \item Создайте метод \texttt{add\_destination}, который принимает название города и количество дней пребывания в качестве аргументов и добавляет их в список.
    \item Создайте метод \texttt{remove\_destination}, который удаляет место из списка по названию города.
    \item Создайте метод \texttt{destination\_count}, который возвращает общее количество пунктов назначения в маршруте.
    \item Создайте экземпляр класса \texttt{TravelItinerary} и добавьте несколько городов.
    \item Выведите информацию о текущих местах на экран.
    \item Выведите общее количество пунктов назначения на экран.
    \item Удалите один из городов и выведите обновленный маршрут.
    \item Выведите общее количество пунктов назначения после удаления на экран.
\end{itemize}

\textbf{Пример использования:}

\begin{verbatim}
itinerary = TravelItinerary()
itinerary.add_destination("Париж", 4)
itinerary.add_destination("Рим", 3)
itinerary.add_destination("Барселона", 5)
print("Маршрут путешествия:")
for city, days in itinerary.destinations:
    print(city, "-", days, "дней")
count = itinerary.destination_count()
print("Всего пунктов:", count)
itinerary.remove_destination("Рим")
print("Маршрут после удаления Рима:")
for city, days in itinerary.destinations:
    print(city, "-", days, "дней")
count = itinerary.destination_count()
print("Всего пунктов:", count)
\end{verbatim}

\textbf{Вывод:}
\begin{verbatim}
Маршрут путешествия:
Париж - 4 дней
Рим - 3 дней
Барселона - 5 дней
Всего пунктов: 3
Маршрут после удаления Рима:
Париж - 4 дней
Барселона - 5 дней
Всего пунктов: 2
\end{verbatim}

\item[17] Написать программу на Python, которая создает класс \texttt{FitnessTracker} для представления тренировочного плана. Класс должен содержать методы для добавления и удаления упражнений, а также подсчета общего количества подходов. Программа также должна создавать экземпляр класса \texttt{FitnessTracker}, добавлять упражнения, удалять упражнения и выводить информацию о плане на экран.

\begin{itemize}
    \item Создайте класс \texttt{FitnessTracker} с методом \texttt{\_\_init\_\_}, который создает пустой список упражнений.
    \item Создайте метод \texttt{add\_exercise}, который принимает название упражнения и количество подходов в качестве аргументов и добавляет их в список.
    \item Создайте метод \texttt{remove\_exercise}, который удаляет упражнение из списка по его названию.
    \item Создайте метод \texttt{total\_sets}, который возвращает общее количество подходов по всем упражнениям.
    \item Создайте экземпляр класса \texttt{FitnessTracker} и добавьте несколько упражнений.
    \item Выведите информацию о текущих упражнениях на экран.
    \item Выведите общее количество подходов на экран.
    \item Удалите одно из упражнений и выведите обновленный план.
    \item Выведите общее количество подходов после удаления на экран.
\end{itemize}

\textbf{Пример использования:}

\begin{verbatim}
tracker = FitnessTracker()
tracker.add_exercise("Приседания", 4)
tracker.add_exercise("Отжимания", 3)
tracker.add_exercise("Подтягивания", 5)
print("Тренировочный план:")
for ex, sets in tracker.exercises:
    print(ex, "-", sets, "подходов")
total = tracker.total_sets()
print("Всего подходов:", total)
tracker.remove_exercise("Отжимания")
print("План после удаления отжиманий:")
for ex, sets in tracker.exercises:
    print(ex, "-", sets, "подходов")
total = tracker.total_sets()
print("Всего подходов:", total)
\end{verbatim}

\textbf{Вывод:}
\begin{verbatim}
Тренировочный план:
Приседания - 4 подходов
Отжимания - 3 подходов
Подтягивания - 5 подходов
Всего подходов: 12
План после удаления отжиманий:
Приседания - 4 подходов
Подтягивания - 5 подходов
Всего подходов: 9
\end{verbatim}

\item[18] Написать программу на Python, которая создает класс \texttt{ExpenseTracker} для представления расходов. Класс должен содержать методы для добавления и удаления трат, а также подсчета общей суммы расходов. Программа также должна создавать экземпляр класса \texttt{ExpenseTracker}, добавлять расходы, удалять расходы и выводить информацию о тратах на экран.

\begin{itemize}
    \item Создайте класс \texttt{ExpenseTracker} с методом \texttt{\_\_init\_\_}, который создает пустой список расходов.
    \item Создайте метод \texttt{add\_expense}, который принимает категорию и сумму в качестве аргументов и добавляет их в список.
    \item Создайте метод \texttt{remove\_expense}, который удаляет расход из списка по категории.
    \item Создайте метод \texttt{total\_expenses}, который возвращает общую сумму всех расходов.
    \item Создайте экземпляр класса \texttt{ExpenseTracker} и добавьте несколько расходов.
    \item Выведите информацию о текущих тратах на экран.
    \item Выведите общую сумму расходов на экран.
    \item Удалите один из расходов и выведите обновленный список.
    \item Выведите общую сумму расходов после удаления на экран.
\end{itemize}

\textbf{Пример использования:}

\begin{verbatim}
expenses = ExpenseTracker()
expenses.add_expense("Продукты", 2500)
expenses.add_expense("Транспорт", 800)
expenses.add_expense("Развлечения", 1200)
print("Расходы:")
for cat, amount in expenses.expenses:
    print(cat, "-", amount, "руб.")
total = expenses.total_expenses()
print("Общая сумма расходов:", total, "руб.")
expenses.remove_expense("Транспорт")
print("Расходы после удаления транспорта:")
for cat, amount in expenses.expenses:
    print(cat, "-", amount, "руб.")
total = expenses.total_expenses()
print("Общая сумма расходов:", total, "руб.")
\end{verbatim}

\textbf{Вывод:}
\begin{verbatim}
Расходы:
Продукты - 2500 руб.
Транспорт - 800 руб.
Развлечения - 1200 руб.
Общая сумма расходов: 4500 руб.
Расходы после удаления транспорта:
Продукты - 2500 руб.
Развлечения - 1200 руб.
Общая сумма расходов: 3700 руб.
\end{verbatim}

\item[19] Написать программу на Python, которая создает класс \texttt{ProjectTasks} для представления задач проекта. Класс должен содержать методы для добавления и удаления задач, а также подсчета общего количества задач. Программа также должна создавать экземпляр класса \texttt{ProjectTasks}, добавлять задачи, удалять задачи и выводить информацию о проекте на экран.

\begin{itemize}
    \item Создайте класс \texttt{ProjectTasks} с методом \texttt{\_\_init\_\_}, который создает пустой список задач.
    \item Создайте метод \texttt{add\_task}, который принимает описание задачи и срок выполнения (в днях) в качестве аргументов и добавляет их в список.
    \item Создайте метод \texttt{remove\_task}, который удаляет задачу из списка по её описанию.
    \item Создайте метод \texttt{task\_count}, который возвращает общее количество задач в проекте.
    \item Создайте экземпляр класса \texttt{ProjectTasks} и добавьте несколько задач.
    \item Выведите информацию о текущих задачах на экран.
    \item Выведите общее количество задач на экран.
    \item Удалите одну из задач и выведите обновленный список.
    \item Выведите общее количество задач после удаления на экран.
\end{itemize}

\textbf{Пример использования:}

\begin{verbatim}
project = ProjectTasks()
project.add_task("Разработка интерфейса", 5)
project.add_task("Тестирование", 3)
project.add_task("Документация", 2)
print("Задачи проекта:")
for desc, days in project.tasks:
    print(desc, "-", days, "дней")
count = project.task_count()
print("Всего задач:", count)
project.remove_task("Тестирование")
print("Задачи после удаления тестирования:")
for desc, days in project.tasks:
    print(desc, "-", days, "дней")
count = project.task_count()
print("Всего задач:", count)
\end{verbatim}

\textbf{Вывод:}
\begin{verbatim}
Задачи проекта:
Разработка интерфейса - 5 дней
Тестирование - 3 дней
Документация - 2 дней
Всего задач: 3
Задачи после удаления тестирования:
Разработка интерфейса - 5 дней
Документация - 2 дней
Всего задач: 2
\end{verbatim}

\item[20] Написать программу на Python, которая создает класс \texttt{EventSchedule} для представления расписания мероприятий. Класс должен содержать методы для добавления и удаления событий, а также подсчета общего количества мероприятий. Программа также должна создавать экземпляр класса \texttt{EventSchedule}, добавлять события, удалять события и выводить информацию о расписании на экран.

\begin{itemize}
    \item Создайте класс \texttt{EventSchedule} с методом \texttt{\_\_init\_\_}, который создает пустой список мероприятий.
    \item Создайте метод \texttt{add\_event}, который принимает название мероприятия и дату проведения в качестве аргументов и добавляет их в список.
    \item Создайте метод \texttt{remove\_event}, который удаляет мероприятие из списка по его названию.
    \item Создайте метод \texttt{event\_count}, который возвращает общее количество мероприятий в расписании.
    \item Создайте экземпляр класса \texttt{EventSchedule} и добавьте несколько мероприятий.
    \item Выведите информацию о текущих мероприятиях на экран.
    \item Выведите общее количество мероприятий на экран.
    \item Удалите одно из мероприятий и выведите обновленное расписание.
    \item Выведите общее количество мероприятий после удаления на экран.
\end{itemize}

\textbf{Пример использования:}

\begin{verbatim}
schedule = EventSchedule()
schedule.add_event("Конференция", "15.05.2024")
schedule.add_event("Воркшоп", "20.05.2024")
schedule.add_event("Выставка", "25.05.2024")
print("Расписание мероприятий:")
for name, date in schedule.events:
    print(name, "-", date)
count = schedule.event_count()
print("Всего мероприятий:", count)
schedule.remove_event("Воркшоп")
print("Расписание после удаления воркшопа:")
for name, date in schedule.events:
    print(name, "-", date)
count = schedule.event_count()
print("Всего мероприятий:", count)
\end{verbatim}

\textbf{Вывод:}
\begin{verbatim}
Расписание мероприятий:
Конференция - 15.05.2024
Воркшоп - 20.05.2024
Выставка - 25.05.2024
Всего мероприятий: 3
Расписание после удаления воркшопа:
Конференция - 15.05.2024
Выставка - 25.05.2024
Всего мероприятий: 2
\end{verbatim}

\item[21] Написать программу на Python, которая создает класс \texttt{GardenPlanner} для представления садового участка. Класс должен содержать методы для добавления и удаления растений, а также подсчета общего количества видов растений. Программа также должна создавать экземпляр класса \texttt{GardenPlanner}, добавлять растения, удалять растения и выводить информацию о саде на экран.

\begin{itemize}
    \item Создайте класс \texttt{GardenPlanner} с методом \texttt{\_\_init\_\_}, который создает пустой список растений.
    \item Создайте метод \texttt{add\_plant}, который принимает название растения и количество экземпляров в качестве аргументов и добавляет их в список.
    \item Создайте метод \texttt{remove\_plant}, который удаляет растение из списка по его названию.
    \item Создайте метод \texttt{plant\_count}, который возвращает общее количество различных видов растений в саду.
    \item Создайте экземпляр класса \texttt{GardenPlanner} и добавьте несколько растений.
    \item Выведите информацию о текущих растениях на экран.
    \item Выведите общее количество видов растений на экран.
    \item Удалите одно из растений и выведите обновленный список.
    \item Выведите общее количество видов растений после удаления на экран.
\end{itemize}

\textbf{Пример использования:}

\begin{verbatim}
garden = GardenPlanner()
garden.add_plant("Розы", 10)
garden.add_plant("Тюльпаны", 20)
garden.add_plant("Лаванда", 5)
print("Растения в саду:")
for name, qty in garden.plants:
    print(name, "-", qty)
count = garden.plant_count()
print("Всего видов растений:", count)
garden.remove_plant("Тюльпаны")
print("Растения после удаления тюльпанов:")
for name, qty in garden.plants:
    print(name, "-", qty)
count = garden.plant_count()
print("Всего видов растений:", count)
\end{verbatim}

\textbf{Вывод:}
\begin{verbatim}
Растения в саду:
Розы - 10
Тюльпаны - 20
Лаванда - 5
Всего видов растений: 3
Растения после удаления тюльпанов:
Розы - 10
Лаванда - 5
Всего видов растений: 2
\end{verbatim}

\item[22] Написать программу на Python, которая создает класс \texttt{Warehouse} для представления склада товаров. Класс должен содержать методы для добавления и удаления товаров, а также подсчета общего количества типов товаров. Программа также должна создавать экземпляр класса \texttt{Warehouse}, добавлять товары, удалять товары и выводить информацию о складе на экран.

\begin{itemize}
    \item Создайте класс \texttt{Warehouse} с методом \texttt{\_\_init\_\_}, который создает пустой список товаров.
    \item Создайте метод \texttt{add\_product}, который принимает название товара и количество единиц в качестве аргументов и добавляет их в список.
    \item Создайте метод \texttt{remove\_product}, который удаляет товар из списка по его названию.
    \item Создайте метод \texttt{product\_types}, который возвращает общее количество различных типов товаров на складе.
    \item Создайте экземпляр класса \texttt{Warehouse} и добавьте несколько товаров.
    \item Выведите информацию о текущих товарах на экран.
    \item Выведите общее количество типов товаров на экран.
    \item Удалите один из товаров и выведите обновленный список.
    \item Выведите общее количество типов товаров после удаления на экран.
\end{itemize}

\textbf{Пример использования:}

\begin{verbatim}
warehouse = Warehouse()
warehouse.add_product("Стулья", 50)
warehouse.add_product("Столы", 20)
warehouse.add_product("Лампы", 100)
print("Товары на складе:")
for name, qty in warehouse.products:
    print(name, "-", qty)
types = warehouse.product_types()
print("Всего типов товаров:", types)
warehouse.remove_product("Столы")
print("Товары после удаления столов:")
for name, qty in warehouse.products:
    print(name, "-", qty)
types = warehouse.product_types()
print("Всего типов товаров:", types)
\end{verbatim}

\textbf{Вывод:}
\begin{verbatim}
Товары на складе:
Стулья - 50
Столы - 20
Лампы - 100
Всего типов товаров: 3
Товары после удаления столов:
Стулья - 50
Лампы - 100
Всего типов товаров: 2
\end{verbatim}

\item[23] Написать программу на Python, которая создает класс \texttt{GameInventory} для представления инвентаря игрока. Класс должен содержать методы для добавления и удаления предметов, а также подсчета общего количества типов предметов. Программа также должна создавать экземпляр класса \texttt{GameInventory}, добавлять предметы, удалять предметы и выводить информацию об инвентаре на экран.

\begin{itemize}
    \item Создайте класс \texttt{GameInventory} с методом \texttt{\_\_init\_\_}, который создает пустой список предметов.
    \item Создайте метод \texttt{add\_item}, который принимает название предмета и количество в качестве аргументов и добавляет их в список.
    \item Создайте метод \texttt{remove\_item}, который удаляет предмет из списка по его названию.
    \item Создайте метод \texttt{item\_types}, который возвращает общее количество различных типов предметов в инвентаре.
    \item Создайте экземпляр класса \texttt{GameInventory} и добавьте несколько предметов.
    \item Выведите информацию о текущих предметах на экран.
    \item Выведите общее количество типов предметов на экран.
    \item Удалите один из предметов и выведите обновленный инвентарь.
    \item Выведите общее количество типов предметов после удаления на экран.
\end{itemize}

\textbf{Пример использования:}

\begin{verbatim}
inventory = GameInventory()
inventory.add_item("Меч", 1)
inventory.add_item("Зелье", 5)
inventory.add_item("Щит", 1)
print("Инвентарь игрока:")
for name, qty in inventory.items:
    print(name, "-", qty)
types = inventory.item_types()
print("Всего типов предметов:", types)
inventory.remove_item("Зелье")
print("Инвентарь после удаления зелий:")
for name, qty in inventory.items:
    print(name, "-", qty)
types = inventory.item_types()
print("Всего типов предметов:", types)
\end{verbatim}

\textbf{Вывод:}
\begin{verbatim}
Инвентарь игрока:
Меч - 1
Зелье - 5
Щит - 1
Всего типов предметов: 3
Инвентарь после удаления зелий:
Меч - 1
Щит - 1
Всего типов предметов: 2
\end{verbatim}

\item[24] Написать программу на Python, которая создает класс \texttt{MusicAlbum} для представления музыкального альбома. Класс должен содержать методы для добавления и удаления треков, а также подсчета общего количества треков. Программа также должна создавать экземпляр класса \texttt{MusicAlbum}, добавлять треки, удалять треки и выводить информацию об альбоме на экран.

\begin{itemize}
    \item Создайте класс \texttt{MusicAlbum} с методом \texttt{\_\_init\_\_}, который создает пустой список треков.
    \item Создайте метод \texttt{add\_track}, который принимает название трека и его длительность (в секундах) в качестве аргументов и добавляет их в список.
    \item Создайте метод \texttt{remove\_track}, который удаляет трек из списка по его названию.
    \item Создайте метод \texttt{track\_count}, который возвращает общее количество треков в альбоме.
    \item Создайте экземпляр класса \texttt{MusicAlbum} и добавьте несколько треков.
    \item Выведите информацию о текущих треках на экран.
    \item Выведите общее количество треков на экран.
    \item Удалите один из треков и выведите обновленный список.
    \item Выведите общее количество треков после удаления на экран.
\end{itemize}

\textbf{Пример использования:}

\begin{verbatim}
album = MusicAlbum()
album.add_track("Yesterday", 125)
album.add_track("Hey Jude", 431)
album.add_track("Let It Be", 243)
print("Треки в альбоме:")
for name, duration in album.tracks:
    print(name, "-", duration, "сек.")
count = album.track_count()
print("Всего треков:", count)
album.remove_track("Hey Jude")
print("Треки после удаления 'Hey Jude':")
for name, duration in album.tracks:
    print(name, "-", duration, "сек.")
count = album.track_count()
print("Всего треков:", count)
\end{verbatim}

\textbf{Вывод:}
\begin{verbatim}
Треки в альбоме:
Yesterday - 125 сек.
Hey Jude - 431 сек.
Let It Be - 243 сек.
Всего треков: 3
Треки после удаления 'Hey Jude':
Yesterday - 125 сек.
Let It Be - 243 сек.
Всего треков: 2
\end{verbatim}

\item[25] Написать программу на Python, которая создает класс \texttt{EmployeeRoster} для представления списка сотрудников. Класс должен содержать методы для добавления и удаления сотрудников, а также подсчета общего количества работников. Программа также должна создавать экземпляр класса \texttt{EmployeeRoster}, добавлять сотрудников, удалять сотрудников и выводить информацию о персонале на экран.

\begin{itemize}
    \item Создайте класс \texttt{EmployeeRoster} с методом \texttt{\_\_init\_\_}, который создает пустой список сотрудников.
    \item Создайте метод \texttt{hire\_employee}, который принимает имя сотрудника и его должность в качестве аргументов и добавляет их в список.
    \item Создайте метод \texttt{fire\_employee}, который удаляет сотрудника из списка по имени.
    \item Создайте метод \texttt{employee\_count}, который возвращает общее количество сотрудников.
    \item Создайте экземпляр класса \texttt{EmployeeRoster} и добавьте несколько сотрудников.
    \item Выведите информацию о текущих сотрудниках на экран.
    \item Выведите общее количество работников на экран.
    \item Удалите одного из сотрудников и выведите обновленный список.
    \item Выведите общее количество работников после удаления на экран.
\end{itemize}

\textbf{Пример использования:}

\begin{verbatim}
roster = EmployeeRoster()
roster.hire_employee("Елена", "Менеджер")
roster.hire_employee("Дмитрий", "Разработчик")
roster.hire_employee("Ольга", "Дизайнер")
print("Сотрудники компании:")
for name, position in roster.employees:
    print(name, "-", position)
count = roster.employee_count()
print("Всего сотрудников:", count)
roster.fire_employee("Дмитрий")
print("Сотрудники после увольнения Дмитрия:")
for name, position in roster.employees:
    print(name, "-", position)
count = roster.employee_count()
print("Всего сотрудников:", count)
\end{verbatim}

\textbf{Вывод:}
\begin{verbatim}
Сотрудники компании:
Елена - Менеджер
Дмитрий - Разработчик
Ольга - Дизайнер
Всего сотрудников: 3
Сотрудники после увольнения Дмитрия:
Елена - Менеджер
Ольга - Дизайнер
Всего сотрудников: 2
\end{verbatim}

\item[26] Написать программу на Python, которая создает класс \texttt{ShoppingWishlist} для представления списка желаний покупателя. Класс должен содержать методы для добавления и удаления товаров, а также подсчета общего количества позиций. Программа также должна создавать экземпляр класса \texttt{ShoppingWishlist}, добавлять товары, удалять товары и выводить информацию о списке желаний на экран.

\begin{itemize}
    \item Создайте класс \texttt{ShoppingWishlist} с методом \texttt{\_\_init\_\_}, который создает пустой список товаров.
    \item Создайте метод \texttt{add\_item}, который принимает название товара и его приоритет (от 1 до 5) в качестве аргументов и добавляет их в список.
    \item Создайте метод \texttt{remove\_item}, который удаляет товар из списка по его названию.
    \item Создайте метод \texttt{item\_count}, который возвращает общее количество товаров в списке желаний.
    \item Создайте экземпляр класса \texttt{ShoppingWishlist} и добавьте несколько товаров.
    \item Выведите информацию о текущих товарах на экран.
    \item Выведите общее количество позиций на экран.
    \item Удалите один из товаров и выведите обновленный список.
    \item Выведите общее количество позиций после удаления на экран.
\end{itemize}

\textbf{Пример использования:}

\begin{verbatim}
wishlist = ShoppingWishlist()
wishlist.add_item("Наушники", 5)
wishlist.add_item("Книга", 3)
wishlist.add_item("Флешка", 2)
print("Список желаний:")
for name, priority in wishlist.items:
    print(name, "(приоритет", priority, ")")
count = wishlist.item_count()
print("Всего позиций:", count)
wishlist.remove_item("Книга")
print("Список после удаления книги:")
for name, priority in wishlist.items:
    print(name, "(приоритет", priority, ")")
count = wishlist.item_count()
print("Всего позиций:", count)
\end{verbatim}

\textbf{Вывод:}
\begin{verbatim}
Список желаний:
Наушники (приоритет 5 )
Книга (приоритет 3 )
Флешка (приоритет 2 )
Всего позиций: 3
Список после удаления книги:
Наушники (приоритет 5 )
Флешка (приоритет 2 )
Всего позиций: 2
\end{verbatim}

\item[27] Написать программу на Python, которая создает класс \texttt{DietPlan} для представления плана питания. Класс должен содержать методы для добавления и удаления блюд, а также подсчета общего количества приемов пищи. Программа также должна создавать экземпляр класса \texttt{DietPlan}, добавлять блюда, удалять блюда и выводить информацию о плане на экран.

\begin{itemize}
    \item Создайте класс \texttt{DietPlan} с методом \texttt{\_\_init\_\_}, который создает пустой список блюд.
    \item Создайте метод \texttt{add\_meal}, который принимает название блюда и количество калорий в качестве аргументов и добавляет их в список.
    \item Создайте метод \texttt{remove\_meal}, который удаляет блюдо из списка по его названию.
    \item Создайте метод \texttt{meal\_count}, который возвращает общее количество блюд в плане.
    \item Создайте экземпляр класса \texttt{DietPlan} и добавьте несколько блюд.
    \item Выведите информацию о текущих блюдах на экран.
    \item Выведите общее количество приемов пищи на экран.
    \item Удалите одно из блюд и выведите обновленный план.
    \item Выведите общее количество приемов пищи после удаления на экран.
\end{itemize}

\textbf{Пример использования:}

\begin{verbatim}
diet = DietPlan()
diet.add_meal("Овсянка", 300)
diet.add_meal("Салат", 150)
diet.add_meal("Курица", 400)
print("План питания:")
for dish, calories in diet.meals:
    print(dish, "-", calories, "ккал")
count = diet.meal_count()
print("Всего приемов пищи:", count)
diet.remove_meal("Салат")
print("План после удаления салата:")
for dish, calories in diet.meals:
    print(dish, "-", calories, "ккал")
count = diet.meal_count()
print("Всего приемов пищи:", count)
\end{verbatim}

\textbf{Вывод:}
\begin{verbatim}
План питания:
Овсянка - 300 ккал
Салат - 150 ккал
Курица - 400 ккал
Всего приемов пищи: 3
План после удаления салата:
Овсянка - 300 ккал
Курица - 400 ккал
Всего приемов пищи: 2
\end{verbatim}

\item[28] Написать программу на Python, которая создает класс \texttt{PhotoAlbum} для представления фотоальбома. Класс должен содержать методы для добавления и удаления фотографий, а также подсчета общего количества снимков. Программа также должна создавать экземпляр класса \texttt{PhotoAlbum}, добавлять фотографии, удалять фотографии и выводить информацию об альбоме на экран.

\begin{itemize}
    \item Создайте класс \texttt{PhotoAlbum} с методом \texttt{\_\_init\_\_}, который создает пустой список фотографий.
    \item Создайте метод \texttt{add\_photo}, который принимает название фотографии и дату съемки в качестве аргументов и добавляет их в список.
    \item Создайте метод \texttt{remove\_photo}, который удаляет фотографию из списка по её названию.
    \item Создайте метод \texttt{photo\_count}, который возвращает общее количество фотографий в альбоме.
    \item Создайте экземпляр класса \texttt{PhotoAlbum} и добавьте несколько фотографий.
    \item Выведите информацию о текущих фотографиях на экран.
    \item Выведите общее количество снимков на экран.
    \item Удалите одну из фотографий и выведите обновленный альбом.
    \item Выведите общее количество снимков после удаления на экран.
\end{itemize}

\textbf{Пример использования:}

\begin{verbatim}
album = PhotoAlbum()
album.add_photo("Пляж", "2023-07-15")
album.add_photo("Горы", "2023-08-20")
album.add_photo("Семья", "2023-12-25")
print("Фотографии в альбоме:")
for name, date in album.photos:
    print(name, "-", date)
count = album.photo_count()
print("Всего фотографий:", count)
album.remove_photo("Горы")
print("Фотографии после удаления 'Горы':")
for name, date in album.photos:
    print(name, "-", date)
count = album.photo_count()
print("Всего фотографий:", count)
\end{verbatim}

\textbf{Вывод:}
\begin{verbatim}
Фотографии в альбоме:
Пляж - 2023-07-15
Горы - 2023-08-20
Семья - 2023-12-25
Всего фотографий: 3
Фотографии после удаления 'Горы':
Пляж - 2023-07-15
Семья - 2023-12-25
Всего фотографий: 2
\end{verbatim}

\item[29] Написать программу на Python, которая создает класс \texttt{StudyMaterials} для представления учебных материалов. Класс должен содержать методы для добавления и удаления материалов, а также подсчета общего количества ресурсов. Программа также должна создавать экземпляр класса \texttt{StudyMaterials}, добавлять материалы, удалять материалы и выводить информацию о ресурсах на экран.

\begin{itemize}
    \item Создайте класс \texttt{StudyMaterials} с методом \texttt{\_\_init\_\_}, который создает пустой список материалов.
    \item Создайте метод \texttt{add\_material}, который принимает название материала и тип (например, "книга", "видео", "статья") в качестве аргументов и добавляет их в список.
    \item Создайте метод \texttt{remove\_material}, который удаляет материал из списка по его названию.
    \item Создайте метод \texttt{material\_count}, который возвращает общее количество учебных материалов.
    \item Создайте экземпляр класса \texttt{StudyMaterials} и добавьте несколько материалов.
    \item Выведите информацию о текущих материалах на экран.
    \item Выведите общее количество ресурсов на экран.
    \item Удалите один из материалов и выведите обновленный список.
    \item Выведите общее количество ресурсов после удаления на экран.
\end{itemize}

\textbf{Пример использования:}

\begin{verbatim}
materials = StudyMaterials()
materials.add_material("Алгоритмы", "книга")
materials.add_material("Python для начинающих", "видео")
materials.add_material("Структуры данных", "статья")
print("Учебные материалы:")
for name, mtype in materials.materials:
    print(name, "-", mtype)
count = materials.material_count()
print("Всего материалов:", count)
materials.remove_material("Python для начинающих")
print("Материалы после удаления видео:")
for name, mtype in materials.materials:
    print(name, "-", mtype)
count = materials.material_count()
print("Всего материалов:", count)
\end{verbatim}

\textbf{Вывод:}
\begin{verbatim}
Учебные материалы:
Алгоритмы - книга
Python для начинающих - видео
Структуры данных - статья
Всего материалов: 3
Материалы после удаления видео:
Алгоритмы - книга
Структуры данных - статья
Всего материалов: 2
\end{verbatim}

\item[30] Написать программу на Python, которая создает класс \texttt{ArtCollection} для представления коллекции произведений искусства. Класс должен содержать методы для добавления и удаления работ, а также подсчета общего количества экспонатов. Программа также должна создавать экземпляр класса \texttt{ArtCollection}, добавлять работы, удалять работы и выводить информацию о коллекции на экран.

\begin{itemize}
    \item Создайте класс \texttt{ArtCollection} с методом \texttt{\_\_init\_\_}, который создает пустой список произведений.
    \item Создайте метод \texttt{add\_artwork}, который принимает название работы и имя художника в качестве аргументов и добавляет их в список.
    \item Создайте метод \texttt{remove\_artwork}, который удаляет работу из списка по её названию.
    \item Создайте метод \texttt{artwork\_count}, который возвращает общее количество произведений в коллекции.
    \item Создайте экземпляр класса \texttt{ArtCollection} и добавьте несколько работ.
    \item Выведите информацию о текущих произведениях на экран.
    \item Выведите общее количество экспонатов на экран.
    \item Удалите одну из работ и выведите обновленный список.
    \item Выведите общее количество экспонатов после удаления на экран.
\end{itemize}

\textbf{Пример использования:}

\begin{verbatim}
art = ArtCollection()
art.add_artwork("Звёздная ночь", "Ван Гог")
art.add_artwork("Мона Лиза", "Леонардо да Винчи")
art.add_artwork("Крик", "Мунк")
print("Произведения в коллекции:")
for title, artist in art.artworks:
    print(title, "—", artist)
count = art.artwork_count()
print("Всего экспонатов:", count)
art.remove_artwork("Мона Лиза")
print("Произведения после удаления 'Моны Лизы':")
for title, artist in art.artworks:
    print(title, "—", artist)
count = art.artwork_count()
print("Всего экспонатов:", count)
\end{verbatim}

\textbf{Вывод:}
\begin{verbatim}
Произведения в коллекции:
Звёздная ночь — Ван Гог
Мона Лиза — Леонардо да Винчи
Крик — Мунк
Всего экспонатов: 3
Произведения после удаления 'Моны Лизы':
Звёздная ночь — Ван Гог
Крик — Мунк
Всего экспонатов: 2
\end{verbatim}

\item[31] Написать программу на Python, которая создает класс \texttt{FlightSchedule} для представления расписания рейсов. Класс должен содержать методы для добавления и удаления рейсов, а также подсчета общего количества перелетов. Программа также должна создавать экземпляр класса \texttt{FlightSchedule}, добавлять рейсы, удалять рейсы и выводить информацию о расписании на экран.

\begin{itemize}
    \item Создайте класс \texttt{FlightSchedule} с методом \texttt{\_\_init\_\_}, который создает пустой список рейсов.
    \item Создайте метод \texttt{add\_flight}, который принимает номер рейса и пункт назначения в качестве аргументов и добавляет их в список.
    \item Создайте метод \texttt{remove\_flight}, который удаляет рейс из списка по его номеру.
    \item Создайте метод \texttt{flight\_count}, который возвращает общее количество рейсов в расписании.
    \item Создайте экземпляр класса \texttt{FlightSchedule} и добавьте несколько рейсов.
    \item Выведите информацию о текущих рейсах на экран.
    \item Выведите общее количество перелетов на экран.
    \item Удалите один из рейсов и выведите обновленное расписание.
    \item Выведите общее количество перелетов после удаления на экран.
\end{itemize}

\textbf{Пример использования:}

\begin{verbatim}
flights = FlightSchedule()
flights.add_flight("SU123", "Париж")
flights.add_flight("SU456", "Лондон")
flights.add_flight("SU789", "Рим")
print("Рейсы:")
for num, dest in flights.flights:
    print(num, "-", dest)
count = flights.flight_count()
print("Всего рейсов:", count)
flights.remove_flight("SU456")
print("Рейсы после отмены SU456:")
for num, dest in flights.flights:
    print(num, "-", dest)
count = flights.flight_count()
print("Всего рейсов:", count)
\end{verbatim}

\textbf{Вывод:}
\begin{verbatim}
Рейсы:
SU123 - Париж
SU456 - Лондон
SU789 - Рим
Всего рейсов: 3
Рейсы после отмены SU456:
SU123 - Париж
SU789 - Рим
Всего рейсов: 2
\end{verbatim}

\item[32] Написать программу на Python, которая создает класс \texttt{RecipeIngredients} для представления ингредиентов рецепта. Класс должен содержать методы для добавления и удаления ингредиентов, а также подсчета общего количества компонентов. Программа также должна создавать экземпляр класса \texttt{RecipeIngredients}, добавлять ингредиенты, удалять ингредиенты и выводить информацию о рецепте на экран.

\begin{itemize}
    \item Создайте класс \texttt{RecipeIngredients} с методом \texttt{\_\_init\_\_}, который создает пустой список ингредиентов.
    \item Создайте метод \texttt{add\_ingredient}, который принимает название ингредиента и количество (в граммах или штуках) в качестве аргументов и добавляет их в список.
    \item Создайте метод \texttt{remove\_ingredient}, который удаляет ингредиент из списка по его названию.
    \item Создайте метод \texttt{ingredient\_count}, который возвращает общее количество ингредиентов в рецепте.
    \item Создайте экземпляр класса \texttt{RecipeIngredients} и добавьте несколько ингредиентов.
    \item Выведите информацию о текущих ингредиентах на экран.
    \item Выведите общее количество компонентов на экран.
    \item Удалите один из ингредиентов и выведите обновленный список.
    \item Выведите общее количество компонентов после удаления на экран.
\end{itemize}

\textbf{Пример использования:}

\begin{verbatim}
recipe = RecipeIngredients()
recipe.add_ingredient("Мука", 200)
recipe.add_ingredient("Сахар", 100)
recipe.add_ingredient("Яйца", 2)
print("Ингредиенты рецепта:")
for name, qty in recipe.ingredients:
    print(name, "-", qty)
count = recipe.ingredient_count()
print("Всего ингредиентов:", count)
recipe.remove_ingredient("Сахар")
print("Ингредиенты после удаления сахара:")
for name, qty in recipe.ingredients:
    print(name, "-", qty)
count = recipe.ingredient_count()
print("Всего ингредиентов:", count)
\end{verbatim}

\textbf{Вывод:}
\begin{verbatim}
Ингредиенты рецепта:
Мука - 200
Сахар - 100
Яйца - 2
Всего ингредиентов: 3
Ингредиенты после удаления сахара:
Мука - 200
Яйца - 2
Всего ингредиентов: 2
\end{verbatim}

\item[33] Написать программу на Python, которая создает класс \texttt{WorkoutPlan} для представления плана тренировок. Класс должен содержать методы для добавления и удаления упражнений, а также подсчета общего количества упражнений. Программа также должна создавать экземпляр класса \texttt{WorkoutPlan}, добавлять упражнения, удалять упражнения и выводить информацию о плане на экран.

\begin{itemize}
    \item Создайте класс \texttt{WorkoutPlan} с методом \texttt{\_\_init\_\_}, который создает пустой список упражнений.
    \item Создайте метод \texttt{add\_exercise}, который принимает название упражнения и количество повторений в качестве аргументов и добавляет их в список.
    \item Создайте метод \texttt{remove\_exercise}, который удаляет упражнение из списка по его названию.
    \item Создайте метод \texttt{exercise\_count}, который возвращает общее количество упражнений в плане.
    \item Создайте экземпляр класса \texttt{WorkoutPlan} и добавьте несколько упражнений.
    \item Выведите информацию о текущих упражнениях на экран.
    \item Выведите общее количество упражнений на экран.
    \item Удалите одно из упражнений и выведите обновленный план.
    \item Выведите общее количество упражнений после удаления на экран.
\end{itemize}

\textbf{Пример использования:}

\begin{verbatim}
workout = WorkoutPlan()
workout.add_exercise("Бег", 30)
workout.add_exercise("Планка", 3)
workout.add_exercise("Приседания", 20)
print("План тренировки:")
for name, reps in workout.exercises:
    print(name, "-", reps)
count = workout.exercise_count()
print("Всего упражнений:", count)
workout.remove_exercise("Планка")
print("План после удаления планки:")
for name, reps in workout.exercises:
    print(name, "-", reps)
count = workout.exercise_count()
print("Всего упражнений:", count)
\end{verbatim}

\textbf{Вывод:}
\begin{verbatim}
План тренировки:
Бег - 30
Планка - 3
Приседания - 20
Всего упражнений: 3
План после удаления планки:
Бег - 30
Приседания - 20
Всего упражнений: 2
\end{verbatim}

\item[34] Написать программу на Python, которая создает класс \texttt{InventoryManager} для представления управления запасами. Класс должен содержать методы для добавления и удаления товаров, а также подсчета общего количества типов товаров. Программа также должна создавать экземпляр класса \texttt{InventoryManager}, добавлять товары, удалять товары и выводить информацию о запасах на экран.

\begin{itemize}
    \item Создайте класс \texttt{InventoryManager} с методом \texttt{\_\_init\_\_}, который создает пустой список товаров.
    \item Создайте метод \texttt{add\_product}, который принимает название товара и количество единиц в качестве аргументов и добавляет их в список.
    \item Создайте метод \texttt{remove\_product}, который удаляет товар из списка по его названию.
    \item Создайте метод \texttt{product\_types}, который возвращает общее количество различных типов товаров.
    \item Создайте экземпляр класса \texttt{InventoryManager} и добавьте несколько товаров.
    \item Выведите информацию о текущих товарах на экран.
    \item Выведите общее количество типов товаров на экран.
    \item Удалите один из товаров и выведите обновленный список.
    \item Выведите общее количество типов товаров после удаления на экран.
\end{itemize}

\textbf{Пример использования:}

\begin{verbatim}
inv = InventoryManager()
inv.add_product("Мыло", 100)
inv.add_product("Шампунь", 50)
inv.add_product("Зубная паста", 75)
print("Товары на складе:")
for name, qty in inv.products:
    print(name, "-", qty)
types = inv.product_types()
print("Всего типов товаров:", types)
inv.remove_product("Шампунь")
print("Товары после удаления шампуня:")
for name, qty in inv.products:
    print(name, "-", qty)
types = inv.product_types()
print("Всего типов товаров:", types)
\end{verbatim}

\textbf{Вывод:}
\begin{verbatim}
Товары на складе:
Мыло - 100
Шампунь - 50
Зубная паста - 75
Всего типов товаров: 3
Товары после удаления шампуня:
Мыло - 100
Зубная паста - 75
Всего типов товаров: 2
\end{verbatim}

\item[35] Написать программу на Python, которая создает класс \texttt{EventGuestList} для представления списка гостей мероприятия. Класс должен содержать методы для добавления и удаления гостей, а также подсчета общего количества приглашенных. Программа также должна создавать экземпляр класса \texttt{EventGuestList}, добавлять гостей, удалять гостей и выводить информацию о списке на экран.

\begin{itemize}
    \item Создайте класс \texttt{EventGuestList} с методом \texttt{\_\_init\_\_}, который создает пустой список гостей.
    \item Создайте метод \texttt{add\_guest}, который принимает имя гостя и его статус (например, "подтвержден", "ожидает") в качестве аргументов и добавляет их в список.
    \item Создайте метод \texttt{remove\_guest}, который удаляет гостя из списка по имени.
    \item Создайте метод \texttt{guest\_count}, который возвращает общее количество гостей в списке.
    \item Создайте экземпляр класса \texttt{EventGuestList} и добавьте несколько гостей.
    \item Выведите информацию о текущих гостях на экран.
    \item Выведите общее количество приглашенных на экран.
    \item Удалите одного из гостей и выведите обновленный список.
    \item Выведите общее количество приглашенных после удаления на экран.
\end{itemize}

\textbf{Пример использования:}

\begin{verbatim}
guests = EventGuestList()
guests.add_guest("Андрей", "подтвержден")
guests.add_guest("Светлана", "ожидает")
guests.add_guest("Михаил", "подтвержден")
print("Список гостей:")
for name, status in guests.guests:
    print(name, "-", status)
count = guests.guest_count()
print("Всего гостей:", count)
guests.remove_guest("Светлана")
print("Список после отмены Светланы:")
for name, status in guests.guests:
    print(name, "-", status)
count = guests.guest_count()
print("Всего гостей:", count)
\end{verbatim}

\textbf{Вывод:}
\begin{verbatim}
Список гостей:
Андрей - подтвержден
Светлана - ожидает
Михаил - подтвержден
Всего гостей: 3
Список после отмены Светланы:
Андрей - подтвержден
Михаил - подтвержден
Всего гостей: 2
\end{verbatim}
\end{enumerate}