\subsection{Семинар <<Классы-миксины и множественное наследование (продолжение)>> (2 часа)}

\begin{enumerate}
    \item[1] \textbf{1 часть}
    Написать класс Python \texttt{CarAcceleration}, выполняющий расчёт времени разгона автомобиля от 0 до 100 км/ч и от 100 до 200 км/ч.

    \textbf{Расчёт времени от 0 до 100 км/ч:}

    \begin{equation}
    t = \frac{m \cdot (\Delta V)^2}{2P_{\text{эфф}}},
    \end{equation}

    где:
    \begin{itemize}
        \item $t$ — время разгона, с;
        \item $m$ — масса автомобиля, кг;
        \item $\Delta V$ — изменение скорости, м/с;
        \item $P_{\text{эфф}}$ — эффективная мощность двигателя, Вт.
    \end{itemize}

    \textbf{Формула расчёта эффективной мощности двигателя для ускорения от 0 до 100 км/ч:}

    \begin{equation}
    P_{\text{эфф}} = P_{\text{макс}} \cdot k \cdot \eta,
    \end{equation}

    где:
    \begin{itemize}
        \item $P_{\text{макс}}$ — максимальная мощность двигателя, л. с.;
        \item $k = 0{,}25$ — коэффициент использования мощности;
        \item $\eta = 0{,}8$ — КПД трансмиссии;
        \item 1 л. с. = 735 Вт.
    \end{itemize}

    \textbf{Расчёт времени от 100 до 200 км/ч:}

    \begin{equation}
    t = \frac{m \cdot (V_2^2 - V_1^2)}{2P_{\text{эфф}}},
    \end{equation}

    где:
    \begin{itemize}
        \item $V_1$ — начальная скорость, м/с;
        \item $V_2$ — конечная скорость, м/с;
        \item $P_{\text{эфф}}$ — эффективная мощность двигателя, Вт.
    \end{itemize}

    \textbf{Формула расчёта эффективной мощности для ускорения от 100 до 200 км/ч:}

    \begin{equation}
    P_{\text{эфф}} = P_{\text{макс}} \cdot k \cdot \eta,
    \end{equation}

    где $k = 0{,}66$ — коэффициент использования мощности для высокоскоростного диапазона.

    \textbf{2 часть}

    На основе программного кода предыдущего практикума текущего курса ООП выполнить наследование класса \texttt{CarAcceleration} с использованием \texttt{super()} таким образом, чтобы через класс с множественным наследованием \texttt{Car} вывести на печать время ускорения до 100 км/ч и от 100 до 200 км/ч для трёх типов автомобилей при следующих параметрах:

    \begin{itemize}
        \item Sedan — масса 1100 кг, мощность двигателя 100 л. с.;
        \item Hatchback — масса 1200 кг, мощность двигателя 150 л. с.;
        \item SUV — масса 1700 кг, мощность двигателя 370 л. с.
    \end{itemize}

    \textbf{3 часть}

    На основе программного кода и параметров автомобилей из 2 части текущего практикума отобразить через класс с множественным наследованием \texttt{Car} расчёт времени разгона автомобилей от 100 до 200 км/ч через каждые 20 км/ч, то есть время разгона до 120, 140, 160, 180 и 200 км/ч, учитывая, что коэффициент мощности двигателя $k = 0{,}66$ увеличивается через каждые 20 км/ч на 0{,}08.

    \textbf{Коэффициенты мощности для различных диапазонов скоростей:}
    \begin{itemize}
        \item 100–120 км/ч: $k = 0{,}66$;
        \item 120–140 км/ч: $k = 0{,}74$;
        \item 140–160 км/ч: $k = 0{,}82$;
        \item 160–180 км/ч: $k = 0{,}90$;
        \item 180–200 км/ч: $k = 0{,}98$.
    \end{itemize}
Вот исправленные варианты 2–5 так, чтобы уровень детализации (структура, пояснения, формат формул и описаний) полностью соответствовал варианту 1. Я сохранил стиль изложения, структуру подзаголовков, полноту пояснений и формат описания физических величин.

---

\item[2] \textbf{1 часть}  
Написать класс Python \texttt{CycleAcceleration}, выполняющий расчёт времени разгона велосипеда от 0 до 20 км/ч и от 20 до 40 км/ч.

\textbf{Расчёт времени от 0 до 20 км/ч:}

\begin{equation}
t = \frac{m \cdot (\Delta V)^2}{2P_{\text{эфф}}},
\end{equation}

где:  
\begin{itemize}
    \item $t$ — время разгона, с;  
    \item $m$ — масса велосипеда с велосипедистом, кг;  
    \item $\Delta V$ — изменение скорости, м/с;  
    \item $P_{\text{эфф}}$ — эффективная мощность, Вт.  
\end{itemize}

\textbf{Формула расчёта эффективной мощности для разгона от 0 до 20 км/ч:}

\begin{equation}
P_{\text{эфф}} = P_{\text{макс}} \cdot k \cdot \eta,
\end{equation}

где:  
\begin{itemize}
    \item $P_{\text{макс}}$ — максимальная мощность велосипедиста или мотора, Вт;  
    \item $k = 0{,}30$ — коэффициент использования мощности;  
    \item $\eta = 0{,}95$ — КПД цепной передачи.  
\end{itemize}

\textbf{Расчёт времени от 20 до 40 км/ч:}

\begin{equation}
t = \frac{m \cdot (V_2^2 - V_1^2)}{2P_{\text{эфф}}},
\end{equation}

где:  
\begin{itemize}
    \item $V_1$ — начальная скорость, м/с;  
    \item $V_2$ — конечная скорость, м/с;  
    \item $P_{\text{эфф}}$ — эффективная мощность, Вт.  
\end{itemize}

\textbf{Формула расчёта эффективной мощности для ускорения от 20 до 40 км/ч:}

\begin{equation}
P_{\text{эфф}} = P_{\text{макс}} \cdot k \cdot \eta,
\end{equation}

где $k = 0{,}50$ — коэффициент использования мощности для высокоскоростного диапазона.

\textbf{2 часть}

На основе программного кода предыдущего практикума текущего курса ООП выполнить наследование класса \texttt{CycleAcceleration} с использованием \texttt{super()} таким образом, чтобы через класс с множественным наследованием \texttt{Cycle} вывести на печать время ускорения до 20 км/ч и от 20 до 40 км/ч для трёх типов велосипедов при следующих параметрах:

\begin{itemize}
    \item Bicycle — масса 80 кг, мощность велосипедиста 200 Вт;  
    \item Ebike — масса 25 кг, мощность мотора 250 Вт;  
    \item Tandem — масса 130 кг, суммарная мощность двух велосипедистов 350 Вт.  
\end{itemize}

\textbf{3 часть}

На основе программного кода и параметров велосипедов из 2 части текущего практикума отобразить через класс с множественным наследованием \texttt{Cycle} расчёт времени разгона велосипедов от 20 до 40 км/ч через каждые 5 км/ч, то есть время разгона до 25, 30, 35 и 40 км/ч, учитывая, что коэффициент мощности двигателя $k = 0{,}50$ увеличивается через каждые 5 км/ч на 0{,}04.

\textbf{Коэффициенты мощности для различных диапазонов скоростей:}
\begin{itemize}
    \item 20–25 км/ч: $k = 0{,}50$;  
    \item 25–30 км/ч: $k = 0{,}54$;  
    \item 30–35 км/ч: $k = 0{,}58$;  
    \item 35–40 км/ч: $k = 0{,}62$.  
\end{itemize}

---

\item[3] \textbf{1 часть}  
Написать класс Python \texttt{CycleAcceleration}, выполняющий расчёт времени разгона велосипеда от 0 до 25 км/ч и от 25 до 50 км/ч.

\textbf{Расчёт времени от 0 до 25 км/ч:}

\begin{equation}
t = \frac{m \cdot (\Delta V)^2}{2P_{\text{эфф}}},
\end{equation}

где:  
\begin{itemize}
    \item $t$ — время разгона, с;  
    \item $m$ — масса велосипеда с велосипедистом, кг;  
    \item $\Delta V$ — изменение скорости, м/с;  
    \item $P_{\text{эфф}}$ — эффективная мощность, Вт.  
\end{itemize}

\textbf{Формула расчёта эффективной мощности для ускорения от 0 до 25 км/ч:}

\begin{equation}
P_{\text{эфф}} = P_{\text{макс}} \cdot k \cdot \eta,
\end{equation}

где:  
\begin{itemize}
    \item $P_{\text{макс}}$ — мощность велосипедиста или мотора, Вт;  
    \item $k = 0{,}20$ — коэффициент использования мощности;  
    \item $\eta = 0{,}95$ — КПД цепной передачи.  
\end{itemize}

\textbf{Расчёт времени от 25 до 50 км/ч:}

\begin{equation}
t = \frac{m \cdot (V_2^2 - V_1^2)}{2P_{\text{эфф}}},
\end{equation}

где:  
\begin{itemize}
    \item $V_1$ — начальная скорость, м/с;  
    \item $V_2$ — конечная скорость, м/с;  
    \item $P_{\text{эфф}}$ — эффективная мощность, Вт.  
\end{itemize}

\textbf{Формула расчёта эффективной мощности для ускорения от 25 до 50 км/ч:}

\begin{equation}
P_{\text{эфф}} = P_{\text{макс}} \cdot k \cdot \eta,
\end{equation}

где $k = 0{,}45$ — коэффициент использования мощности для высокоскоростного диапазона.

\textbf{2 часть}

На основе программного кода предыдущего практикума текущего курса ООП выполнить наследование класса \texttt{CycleAcceleration} с использованием \texttt{super()} таким образом, чтобы через класс с множественным наследованием \texttt{Cycle} вывести на печать время ускорения до 25 км/ч и от 25 до 50 км/ч для трёх типов велосипедов при следующих параметрах:

\begin{itemize}
    \item Bicycle — масса 80 кг, мощность 200 Вт;  
    \item Ebike — масса 25 кг, мощность мотора 250 Вт;  
    \item Tandem — масса 130 кг, суммарная мощность двух велосипедистов 350 Вт.  
\end{itemize}

\textbf{3 часть}

На основе программного кода и параметров велосипедов из 2 части текущего практикума отобразить через класс с множественным наследованием \texttt{Cycle} расчёт времени разгона велосипедов от 25 до 50 км/ч через каждые 5 км/ч, то есть время разгона до 30, 35, 40, 45 и 50 км/ч, учитывая, что коэффициент мощности двигателя $k = 0{,}45$ увеличивается через каждые 5 км/ч на 0{,}05.

\textbf{Коэффициенты мощности для различных диапазонов скоростей:}
\begin{itemize}
    \item 25–30 км/ч: $k = 0{,}45$;  
    \item 30–35 км/ч: $k = 0{,}50$;  
    \item 35–40 км/ч: $k = 0{,}55$;  
    \item 40–45 км/ч: $k = 0{,}60$;  
    \item 45–50 км/ч: $k = 0{,}65$.  
\end{itemize}

---

\item[4] \textbf{1 часть}  
Написать класс Python \texttt{DroneAcceleration}, выполняющий расчёт времени набора высоты дроном от 0 до 30 м и от 30 до 60 м.

\textbf{Расчёт времени от 0 до 30 м:}

\begin{equation}
t = \frac{m \cdot (\Delta v)^2}{2P_{\text{эфф}}},
\end{equation}

где:  
\begin{itemize}
    \item $t$ — время подъёма, с;  
    \item $m$ — масса дрона, кг;  
    \item $\Delta v$ — изменение эквивалентной вертикальной скорости, м/с (рассчитывается как $v = \sqrt{2gh}$);  
    \item $P_{\text{эфф}}$ — эффективная мощность, Вт.  
\end{itemize}

\textbf{Формула расчёта эффективной мощности для набора высоты от 0 до 30 м:}

\begin{equation}
P_{\text{эфф}} = P_{\text{макс}} \cdot k \cdot \eta,
\end{equation}

где:  
\begin{itemize}
    \item $P_{\text{макс}}$ — суммарная мощность моторов, Вт;  
    \item $k = 0{,}28$ — коэффициент использования мощности;  
    \item $\eta = 0{,}85$ — КПД винтов.  
\end{itemize}

\textbf{Расчёт времени от 30 до 60 м:}

\begin{equation}
t = \frac{m \cdot (v_2^2 - v_1^2)}{2P_{\text{эфф}}},
\end{equation}

где:  
\begin{itemize}
    \item $v_1 = \sqrt{2g \cdot 30}$ — начальная эквивалентная скорость, м/с;  
    \item $v_2 = \sqrt{2g \cdot 60}$ — конечная эквивалентная скорость, м/с;  
    \item $P_{\text{эфф}}$ — эффективная мощность, Вт.  
\end{itemize}

\textbf{Формула расчёта эффективной мощности для набора высоты от 30 до 60 м:}

\begin{equation}
P_{\text{эфф}} = P_{\text{макс}} \cdot k \cdot \eta,
\end{equation}

где $k = 0{,}50$ — коэффициент использования мощности на большой высоте.

\textbf{2 часть}

На основе программного кода предыдущего практикума текущего курса ООП выполнить наследование класса \texttt{DroneAcceleration} с использованием \texttt{super()} таким образом, чтобы через класс с множественным наследованием \texttt{Rotorcraft} вывести на печать время набора высоты до 30 м и от 30 до 60 м для трёх типов дронов при следующих параметрах:

\begin{itemize}
    \item Consumer — масса 1.2 кг, мощность 1000 Вт;  
    \item Pro — масса 3.5 кг, мощность 2500 Вт;  
    \item Industrial — масса 8.0 кг, мощность 6000 Вт.  
\end{itemize}

\textbf{3 часть}

На основе программного кода и параметров дронов из 2 части текущего практикума отобразить через класс с множественным наследованием \texttt{Rotorcraft} расчёт времени набора высоты дронами от 30 до 60 м через каждые 5 м, то есть время подъёма до 35, 40, 45, 50, 55 и 60 м, учитывая, что коэффициент мощности $k = 0{,}50$ увеличивается через каждые 5 м на 0{,}04.

\textbf{Коэффициенты мощности для различных диапазонов высоты:}
\begin{itemize}
    \item 30–35 м: $k = 0{,}50$;  
    \item 35–40 м: $k = 0{,}54$;  
    \item 40–45 м: $k = 0{,}58$;  
    \item 45–50 м: $k = 0{,}62$;  
    \item 50–55 м: $k = 0{,}66$;  
    \item 55–60 м: $k = 0{,}70$.  
\end{itemize}

---

\item[5] \textbf{1 часть}  
Написать класс Python \texttt{CoolingAcceleration}, выполняющий расчёт времени охлаждения холодильника от 25\,°C до 4\,°C и от 4\,°C до -18\,°C.

\textbf{Расчёт времени от 25\,°C до 4\,°C:}

\begin{equation}
t = \frac{C \cdot (\Delta T)^2}{2P_{\text{эфф}}},
\end{equation}

где:  
\begin{itemize}
    \item $t$ — время охлаждения, с;  
    \item $C$ — теплоёмкость содержимого, Дж/°C;  
    \item $\Delta T$ — изменение температуры, °C;  
    \item $P_{\text{эфф}}$ — эффективная холодопроизводительность, Вт.  
\end{itemize}

\textbf{Формула расчёта эффективной холодопроизводительности для охлаждения от 25\,°C до 4\,°C:}

\begin{equation}
P_{\text{эфф}} = P_{\text{макс}} \cdot k \cdot \eta,
\end{equation}

где:  
\begin{itemize}
    \item $P_{\text{макс}}$ — номинальная мощность компрессора, Вт;  
    \item $k = 0{,}35$ — коэффициент использования мощности;  
    \item $\eta = 0{,}70$ — КПД холодильного цикла.  
\end{itemize}

\textbf{Расчёт времени от 4\,°C до -18\,°C:}

\begin{equation}
t = \frac{C \cdot (T_2^2 - T_1^2)}{2P_{\text{эфф}}},
\end{equation}

где:  
\begin{itemize}
    \item $T_1$ — начальная температура, °C;  
    \item $T_2$ — конечная температура, °C;  
    \item $P_{\text{эфф}}$ — эффективная холодопроизводительность, Вт.  
\end{itemize}

\textbf{Формула расчёта эффективной холодопроизводительности для охлаждения от 4\,°C до -18\,°C:}

\begin{equation}
P_{\text{эфф}} = P_{\text{макс}} \cdot k \cdot \eta,
\end{equation}

где $k = 0{,}60$ — коэффициент использования мощности в режиме заморозки.

\textbf{2 часть}

На основе программного кода предыдущего практикума текущего курса ООП выполнить наследование класса \texttt{CoolingAcceleration} с использованием \texttt{super()} таким образом, чтобы через класс с множественным наследованием \texttt{CoolingAppliance} вывести на печать время охлаждения до 4\,°C и от 4\,°C до -18\,°C для трёх типов устройств при следующих параметрах:

\begin{itemize}
    \item Refrigerator — $C = 8000$ Дж/°C, $P = 120$ Вт;  
    \item Freezer — $C = 6000$ Дж/°C, $P = 150$ Вт;  
    \item MiniFridge — $C = 3000$ Дж/°C, $P = 60$ Вт.  
\end{itemize}

\textbf{3 часть}

На основе программного кода и параметров устройств из 2 части текущего практикума отобразить через класс с множественным наследованием \texttt{CoolingAppliance} расчёт времени охлаждения от 4\,°C до -18\,°C через каждые 4.4\,°C, то есть до -0.4, -4.8, -9.2, -13.6 и -18\,°C, учитывая, что коэффициент мощности $k = 0{,}60$ увеличивается через каждые 4.4\,°C на 0{,}08.

\textbf{Коэффициенты мощности для различных диапазонов температур:}
\begin{itemize}
    \item 4 → -0.4\,°C: $k = 0{,}60$;  
    \item -0.4 → -4.8\,°C: $k = 0{,}68$;  
    \item -4.8 → -9.2\,°C: $k = 0{,}76$;  
    \item -9.2 → -13.6\,°C: $k = 0{,}84$;  
    \item -13.6 → -18\,°C: $k = 0{,}92$.  
\end{itemize}

\item[6] \textbf{1 часть}  
Написать класс Python \texttt{ScooterAcceleration}, выполняющий расчёт времени разгона электросамоката от 0 до 20 км/ч и от 20 до 40 км/ч.

\textbf{Расчёт времени от 0 до 20 км/ч:}

\begin{equation}
t = \frac{m \cdot (\Delta V)^2}{2P_{\text{эфф}}},
\end{equation}

где:  
\begin{itemize}
    \item $t$ — время разгона, с;  
    \item $m$ — масса самоката с водителем, кг;  
    \item $\Delta V$ — изменение скорости, м/с;  
    \item $P_{\text{эфф}}$ — эффективная мощность мотора, Вт.  
\end{itemize}

\textbf{Формула расчёта эффективной мощности для разгона от 0 до 20 км/ч:}

\begin{equation}
P_{\text{эфф}} = P_{\text{макс}} \cdot k \cdot \eta,
\end{equation}

где:  
\begin{itemize}
    \item $P_{\text{макс}}$ — номинальная мощность мотора, Вт;  
    \item $k = 0{,}30$ — коэффициент использования мощности;  
    \item $\eta = 0{,}90$ — КПД привода;  
    \item 1 км/ч = $\frac{1}{3{,}6}$ м/с.  
\end{itemize}

\textbf{Расчёт времени от 20 до 40 км/ч:}

\begin{equation}
t = \frac{m \cdot (V_2^2 - V_1^2)}{2P_{\text{эфф}}},
\end{equation}

где:  
\begin{itemize}
    \item $V_1$ — начальная скорость, м/с;  
    \item $V_2$ — конечная скорость, м/с;  
    \item $P_{\text{эфф}}$ — эффективная мощность, Вт.  
\end{itemize}

\textbf{Формула расчёта эффективной мощности для ускорения от 20 до 40 км/ч:}

\begin{equation}
P_{\text{эфф}} = P_{\text{макс}} \cdot k \cdot \eta,
\end{equation}

где $k = 0{,}55$ — коэффициент использования мощности для высокоскоростного диапазона.

\textbf{2 часть}

На основе программного кода предыдущего практикума текущего курса ООП выполнить наследование класса \texttt{ScooterAcceleration} с использованием \texttt{super()} таким образом, чтобы через класс с множественным наследованием \texttt{PersonalTransport} вывести на печать время ускорения до 20 км/ч и от 20 до 40 км/ч для трёх типов самокатов при следующих параметрах:

\begin{itemize}
    \item City — масса 85 кг, мощность мотора 350 Вт;  
    \item OffRoad — масса 110 кг, мощность мотора 750 Вт;  
    \item Folding — масса 70 кг, мощность мотора 250 Вт.  
\end{itemize}

\textbf{3 часть}

На основе программного кода и параметров самокатов из 2 части текущего практикума отобразить через класс с множественным наследованием \texttt{PersonalTransport} расчёт времени разгона самокатов от 20 до 40 км/ч через каждые 5 км/ч, то есть время разгона до 25, 30, 35 и 40 км/ч, учитывая, что коэффициент мощности $k = 0{,}55$ увеличивается через каждые 5 км/ч на 0{,}05.

\textbf{Коэффициенты мощности для различных диапазонов скоростей:}
\begin{itemize}
    \item 20–25 км/ч: $k = 0{,}55$;  
    \item 25–30 км/ч: $k = 0{,}60$;  
    \item 30–35 км/ч: $k = 0{,}65$;  
    \item 35–40 км/ч: $k = 0{,}70$.  
\end{itemize}


\item[7] \textbf{1 часть}  
Написать класс Python \texttt{AgriculturalAcceleration}, выполняющий расчёт времени выхода сельскохозяйственной техники на рабочую скорость от 0 до 8 км/ч и от 8 до 16 км/ч.

\textbf{Расчёт времени от 0 до 8 км/ч:}

\begin{equation}
t = \frac{m \cdot (\Delta V)^2}{2P_{\text{эфф}}},
\end{equation}

где:  
\begin{itemize}
    \item $t$ — время разгона, с;  
    \item $m$ — масса техники, кг;  
    \item $\Delta V$ — изменение скорости, м/с;  
    \item $P_{\text{эфф}}$ — эффективная мощность, Вт.  
\end{itemize}

\textbf{Формула расчёта эффективной мощности для разгона от 0 до 8 км/ч:}

\begin{equation}
P_{\text{эфф}} = P_{\text{макс}} \cdot k \cdot \eta,
\end{equation}

где:  
\begin{itemize}
    \item $P_{\text{макс}}$ — мощность двигателя, л. с.;  
    \item $k = 0{,}22$ — коэффициент использования мощности при старте;  
    \item $\eta = 0{,}82$ — КПД трансмиссии;  
    \item 1 л. с. = 735 Вт.  
\end{itemize}

\textbf{Расчёт времени от 8 до 16 км/ч:}

\begin{equation}
t = \frac{m \cdot (V_2^2 - V_1^2)}{2P_{\text{эфф}}},
\end{equation}

где:  
\begin{itemize}
    \item $V_1$ — начальная скорость, м/с;  
    \item $V_2$ — конечная скорость, м/с;  
    \item $P_{\text{эфф}}$ — эффективная мощность, Вт.  
\end{itemize}

\textbf{Формула расчёта эффективной мощности для ускорения от 8 до 16 км/ч:}

\begin{equation}
P_{\text{эфф}} = P_{\text{макс}} \cdot k \cdot \eta,
\end{equation}

где $k = 0{,}48$ — коэффициент использования мощности на рабочей скорости.

\textbf{2 часть}

На основе программного кода предыдущего практикума текущего курса ООП выполнить наследование класса \texttt{AgriculturalAcceleration} с использованием \texttt{super()} таким образом, чтобы через класс с множественным наследованием \texttt{AgriculturalMachine} вывести на печать время ускорения до 8 км/ч и от 8 до 16 км/ч для трёх типов техники при следующих параметрах:

\begin{itemize}
    \item Tractor — масса 3500 кг, мощность 120 л. с.;  
    \item Combine — масса 8500 кг, мощность 350 л. с.;  
    \item Sprayer — масса 2200 кг, мощность 90 л. с.  
\end{itemize}

\textbf{3 часть}

На основе программного кода и параметров техники из 2 части текущего практикума отобразить через класс с множественным наследованием \texttt{AgriculturalMachine} расчёт времени разгона от 8 до 16 км/ч через каждые 2 км/ч, то есть до 10, 12, 14 и 16 км/ч, учитывая, что коэффициент мощности $k = 0{,}48$ увеличивается через каждые 2 км/ч на 0{,}06.

\textbf{Коэффициенты мощности для различных диапазонов скоростей:}
\begin{itemize}
    \item 8–10 км/ч: $k = 0{,}48$;  
    \item 10–12 км/ч: $k = 0{,}54$;  
    \item 12–14 км/ч: $k = 0{,}60$;  
    \item 14–16 км/ч: $k = 0{,}66$.  
\end{itemize}

\item[8] \textbf{1 часть}  
Написать класс Python \texttt{ConstructionAcceleration}, выполняющий расчёт времени разгона строительной техники от 0 до 10 км/ч и от 10 до 20 км/ч.

\textbf{Расчёт времени от 0 до 10 км/ч:}

\begin{equation}
t = \frac{m \cdot (\Delta V)^2}{2P_{\text{эфф}}},
\end{equation}

где:  
\begin{itemize}
    \item $t$ — время разгона, с;  
    \item $m$ — масса техники, кг;  
    \item $\Delta V$ — изменение скорости, м/с;  
    \item $P_{\text{эфф}}$ — эффективная мощность, Вт.  
\end{itemize}

\textbf{Формула расчёта эффективной мощности для разгона от 0 до 10 км/ч:}

\begin{equation}
P_{\text{эфф}} = P_{\text{макс}} \cdot k \cdot \eta,
\end{equation}

где:  
\begin{itemize}
    \item $P_{\text{макс}}$ — мощность двигателя, л. с.;  
    \item $k = 0{,}20$ — коэффициент использования мощности;  
    \item $\eta = 0{,}78$ — КПД гидравлической трансмиссии;  
    \item 1 л. с. = 735 Вт.  
\end{itemize}

\textbf{Расчёт времени от 10 до 20 км/ч:}

\begin{equation}
t = \frac{m \cdot (V_2^2 - V_1^2)}{2P_{\text{эфф}}},
\end{equation}

где:  
\begin{itemize}
    \item $V_1$ — начальная скорость, м/с;  
    \item $V_2$ — конечная скорость, м/с;  
    \item $P_{\text{эфф}}$ — эффективная мощность, Вт.  
\end{itemize}

\textbf{Формула расчёта эффективной мощности для ускорения от 10 до 20 км/ч:}

\begin{equation}
P_{\text{эфф}} = P_{\text{макс}} \cdot k \cdot \eta,
\end{equation}

где $k = 0{,}52$ — коэффициент использования мощности при движении по площадке.

\textbf{2 часть}

На основе программного кода предыдущего практикума текущего курса ООП выполнить наследование класса \texttt{ConstructionAcceleration} с использованием \texttt{super()} таким образом, чтобы через класс с множественным наследованием \texttt{ConstructionEquipment} вывести на печать время ускорения до 10 км/ч и от 10 до 20 км/ч для трёх типов техники при следующих параметрах:

\begin{itemize}
    \item Excavator — масса 12000 кг, мощность 250 л. с.;  
    \item Bulldozer — масса 18000 кг, мощность 400 л. с.;  
    \item Crane — масса 9000 кг, мощность 200 л. с.  
\end{itemize}

\textbf{3 часть}

На основе программного кода и параметров техники из 2 части текущего практикума отобразить через класс с множественным наследованием \texttt{ConstructionEquipment} расчёт времени разгона от 10 до 20 км/ч через каждые 2.5 км/ч, то есть до 12.5, 15, 17.5 и 20 км/ч, учитывая, что коэффициент мощности $k = 0{,}52$ увеличивается через каждые 2.5 км/ч на 0{,}08.

\textbf{Коэффициенты мощности для различных диапазонов скоростей:}
\begin{itemize}
    \item 10–12.5 км/ч: $k = 0{,}52$;  
    \item 12.5–15 км/ч: $k = 0{,}60$;  
    \item 15–17.5 км/ч: $k = 0{,}68$;  
    \item 17.5–20 км/ч: $k = 0{,}76$.  
\end{itemize}

\item[9] \textbf{1 часть}  
Написать класс Python \texttt{EmergencyAcceleration}, выполняющий расчёт времени разгона служебного транспорта от 0 до 60 км/ч и от 60 до 120 км/ч.

\textbf{Расчёт времени от 0 до 60 км/ч:}

\begin{equation}
t = \frac{m \cdot (\Delta V)^2}{2P_{\text{эфф}}},
\end{equation}

где:  
\begin{itemize}
    \item $t$ — время разгона, с;  
    \item $m$ — масса транспортного средства, кг;  
    \item $\Delta V$ — изменение скорости, м/с;  
    \item $P_{\text{эфф}}$ — эффективная мощность, Вт.  
\end{itemize}

\textbf{Формула расчёта эффективной мощности для разгона от 0 до 60 км/ч:}

\begin{equation}
P_{\text{эфф}} = P_{\text{макс}} \cdot k \cdot \eta,
\end{equation}

где:  
\begin{itemize}
    \item $P_{\text{макс}}$ — мощность двигателя, л. с.;  
    \item $k = 0{,}28$ — коэффициент использования мощности при экстренном старте;  
    \item $\eta = 0{,}85$ — КПД трансмиссии;  
    \item 1 л. с. = 735 Вт.  
\end{itemize}

\textbf{Расчёт времени от 60 до 120 км/ч:}

\begin{equation}
t = \frac{m \cdot (V_2^2 - V_1^2)}{2P_{\text{эфф}}},
\end{equation}

где:  
\begin{itemize}
    \item $V_1$ — начальная скорость, м/с;  
    \item $V_2$ — конечная скорость, м/с;  
    \item $P_{\text{эфф}}$ — эффективная мощность, Вт.  
\end{itemize}

\textbf{Формула расчёта эффективной мощности для ускорения от 60 до 120 км/ч:}

\begin{equation}
P_{\text{эфф}} = P_{\text{макс}} \cdot k \cdot \eta,
\end{equation}

где $k = 0{,}64$ — коэффициент использования мощности при высокой скорости.

\textbf{2 часть}

На основе программного кода предыдущего практикума текущего курса ООП выполнить наследование класса \texttt{EmergencyAcceleration} с использованием \texttt{super()} таким образом, чтобы через класс с множественным наследованием \texttt{EmergencyVehicle} вывести на печать время ускорения до 60 км/ч и от 60 до 120 км/ч для трёх типов транспорта при следующих параметрах:

\begin{itemize}
    \item Ambulance — масса 2800 кг, мощность 180 л. с.;  
    \item FireTruck — масса 14000 кг, мощность 450 л. с.;  
    \item PoliceCar — масса 2000 кг, мощность 300 л. с.  
\end{itemize}

\textbf{3 часть}

На основе программного кода и параметров транспорта из 2 части текущего практикума отобразить через класс с множественным наследованием \texttt{EmergencyVehicle} расчёт времени разгона от 60 до 120 км/ч через каждые 15 км/ч, то есть до 75, 90, 105 и 120 км/ч, учитывая, что коэффициент мощности $k = 0{,}64$ увеличивается через каждые 15 км/ч на 0{,}08.

\textbf{Коэффициенты мощности для различных диапазонов скоростей:}
\begin{itemize}
    \item 60–75 км/ч: $k = 0{,}64$;  
    \item 75–90 км/ч: $k = 0{,}72$;  
    \item 90–105 км/ч: $k = 0{,}80$;  
    \item 105–120 км/ч: $k = 0{,}88$.  
\end{itemize}

\item[10] \textbf{1 часть}  
Написать класс Python \texttt{PassengerAcceleration}, выполняющий расчёт времени разгона пассажирского транспорта от 0 до 30 км/ч и от 30 до 60 км/ч.

\textbf{Расчёт времени от 0 до 30 км/ч:}

\begin{equation}
t = \frac{m \cdot (\Delta V)^2}{2P_{\text{эфф}}},
\end{equation}

где:  
\begin{itemize}
    \item $t$ — время разгона, с;  
    \item $m$ — масса транспортного средства с пассажирами, кг;  
    \item $\Delta V$ — изменение скорости, м/с;  
    \item $P_{\text{эфф}}$ — эффективная мощность, Вт.  
\end{itemize}

\textbf{Формула расчёта эффективной мощности для разгона от 0 до 30 км/ч:}

\begin{equation}
P_{\text{эфф}} = P_{\text{макс}} \cdot k \cdot \eta,
\end{equation}

где:  
\begin{itemize}
    \item $P_{\text{макс}}$ — мощность двигателя, л. с.;  
    \item $k = 0{,}26$ — коэффициент использования мощности;  
    \item $\eta = 0{,}83$ — КПД трансмиссии;  
    \item 1 л. с. = 735 Вт.  
\end{itemize}

\textbf{Расчёт времени от 30 до 60 км/ч:}

\begin{equation}
t = \frac{m \cdot (V_2^2 - V_1^2)}{2P_{\text{эфф}}},
\end{equation}

где:  
\begin{itemize}
    \item $V_1$ — начальная скорость, м/с;  
    \item $V_2$ — конечная скорость, м/с;  
    \item $P_{\text{эфф}}$ — эффективная мощность, Вт.  
\end{itemize}

\textbf{Формула расчёта эффективной мощности для ускорения от 30 до 60 км/ч:}

\begin{equation}
P_{\text{эфф}} = P_{\text{макс}} \cdot k \cdot \eta,
\end{equation}

где $k = 0{,}60$ — коэффициент использования мощности при движении в городском цикле.

\textbf{2 часть}

На основе программного кода предыдущего практикума текущего курса ООП выполнить наследование класса \texttt{PassengerAcceleration} с использованием \texttt{super()} таким образом, чтобы через класс с множественным наследованием \texttt{PassengerVehicle} вывести на печать время ускорения до 30 км/ч и от 30 до 60 км/ч для трёх типов транспорта при следующих параметрах:

\begin{itemize}
    \item Taxi — масса 1600 кг, мощность 120 л. с.;  
    \item RideShare — масса 1500 кг, мощность 110 л. с.;  
    \item Limousine — масса 2500 кг, мощность 200 л. с.  
\end{itemize}

\textbf{3 часть}

На основе программного кода и параметров транспорта из 2 части текущего практикума отобразить через класс с множественным наследованием \texttt{PassengerVehicle} расчёт времени разгона от 30 до 60 км/ч через каждые 7.5 км/ч, то есть до 37.5, 45, 52.5 и 60 км/ч, учитывая, что коэффициент мощности $k = 0{,}60$ увеличивается через каждые 7.5 км/ч на 0{,}08.

\textbf{Коэффициенты мощности для различных диапазонов скоростей:}
\begin{itemize}
    \item 30–37.5 км/ч: $k = 0{,}60$;  
    \item 37.5–45 км/ч: $k = 0{,}68$;  
    \item 45–52.5 км/ч: $k = 0{,}76$;  
    \item 52.5–60 км/ч: $k = 0{,}84$.  
\end{itemize}


\item[11] \textbf{1 часть}  
Написать класс Python \texttt{AgriculturalAcceleration}, выполняющий расчёт времени выхода сельскохозяйственной техники на рабочую скорость от 0 до 8 км/ч и от 8 до 16 км/ч.

\textbf{Расчёт времени от 0 до 8 км/ч:}

\begin{equation}
t = \frac{m \cdot (\Delta V)^2}{2P_{\text{эфф}}},
\end{equation}

где:  
\begin{itemize}
    \item $t$ — время разгона, с;  
    \item $m$ — масса техники, кг;  
    \item $\Delta V$ — изменение скорости, м/с;  
    \item $P_{\text{эфф}}$ — эффективная мощность, Вт.  
\end{itemize}

\textbf{Формула расчёта эффективной мощности для разгона от 0 до 8 км/ч:}

\begin{equation}
P_{\text{эфф}} = P_{\text{макс}} \cdot k \cdot \eta,
\end{equation}

где:  
\begin{itemize}
    \item $P_{\text{макс}}$ — мощность двигателя, л. с.;  
    \item $k = 0{,}22$ — коэффициент использования мощности при старте;  
    \item $\eta = 0{,}82$ — КПД трансмиссии;  
    \item 1 л. с. = 735 Вт.  
\end{itemize}

\textbf{Расчёт времени от 8 до 16 км/ч:}

\begin{equation}
t = \frac{m \cdot (V_2^2 - V_1^2)}{2P_{\text{эфф}}},
\end{equation}

где:  
\begin{itemize}
    \item $V_1$ — начальная скорость, м/с;  
    \item $V_2$ — конечная скорость, м/с;  
    \item $P_{\text{эфф}}$ — эффективная мощность, Вт.  
\end{itemize}

\textbf{Формула расчёта эффективной мощности для ускорения от 8 до 16 км/ч:}

\begin{equation}
P_{\text{эфф}} = P_{\text{макс}} \cdot k \cdot \eta,
\end{equation}

где $k = 0{,}48$ — коэффициент использования мощности на рабочей скорости.

\textbf{2 часть}

На основе программного кода предыдущего практикума текущего курса ООП выполнить наследование класса \texttt{AgriculturalAcceleration} с использованием \texttt{super()} таким образом, чтобы через класс с множественным наследованием \texttt{AgriculturalMachine} вывести на печать время ускорения до 8 км/ч и от 8 до 16 км/ч для трёх типов техники при следующих параметрах:

\begin{itemize}
    \item Tractor — масса 3500 кг, мощность 120 л. с.;  
    \item Combine — масса 8500 кг, мощность 350 л. с.;  
    \item Sprayer — масса 2200 кг, мощность 90 л. с.  
\end{itemize}

\textbf{3 часть}

На основе программного кода и параметров техники из 2 части текущего практикума отобразить через класс с множественным наследованием \texttt{AgriculturalMachine} расчёт времени разгона техники от 8 до 16 км/ч через каждые 2 км/ч, то есть время разгона до 10, 12, 14 и 16 км/ч, учитывая, что коэффициент мощности $k = 0{,}48$ увеличивается через каждые 2 км/ч на 0{,}06.

\textbf{Коэффициенты мощности для различных диапазонов скоростей:}
\begin{itemize}
    \item 8–10 км/ч: $k = 0{,}48$;  
    \item 10–12 км/ч: $k = 0{,}54$;  
    \item 12–14 км/ч: $k = 0{,}60$;  
    \item 14–16 км/ч: $k = 0{,}66$.  
\end{itemize}

\item[12] \textbf{1 часть}  
Написать класс Python \texttt{ConstructionAcceleration}, выполняющий расчёт времени разгона строительной техники от 0 до 10 км/ч и от 10 до 20 км/ч.

\textbf{Расчёт времени от 0 до 10 км/ч:}

\begin{equation}
t = \frac{m \cdot (\Delta V)^2}{2P_{\text{эфф}}},
\end{equation}

где:  
\begin{itemize}
    \item $t$ — время разгона, с;  
    \item $m$ — масса техники, кг;  
    \item $\Delta V$ — изменение скорости, м/с;  
    \item $P_{\text{эфф}}$ — эффективная мощность, Вт.  
\end{itemize}

\textbf{Формула расчёта эффективной мощности для разгона от 0 до 10 км/ч:}

\begin{equation}
P_{\text{эфф}} = P_{\text{макс}} \cdot k \cdot \eta,
\end{equation}

где:  
\begin{itemize}
    \item $P_{\text{макс}}$ — мощность двигателя, л. с.;  
    \item $k = 0{,}20$ — коэффициент использования мощности;  
    \item $\eta = 0{,}78$ — КПД гидравлической трансмиссии;  
    \item 1 л. с. = 735 Вт.  
\end{itemize}

\textbf{Расчёт времени от 10 до 20 км/ч:}

\begin{equation}
t = \frac{m \cdot (V_2^2 - V_1^2)}{2P_{\text{эфф}}},
\end{equation}

где:  
\begin{itemize}
    \item $V_1$ — начальная скорость, м/с;  
    \item $V_2$ — конечная скорость, м/с;  
    \item $P_{\text{эфф}}$ — эффективная мощность, Вт.  
\end{itemize}

\textbf{Формула расчёта эффективной мощности для ускорения от 10 до 20 км/ч:}

\begin{equation}
P_{\text{эфф}} = P_{\text{макс}} \cdot k \cdot \eta,
\end{equation}

где $k = 0{,}52$ — коэффициент использования мощности при движении по площадке.

\textbf{2 часть}

На основе программного кода предыдущего практикума текущего курса ООП выполнить наследование класса \texttt{ConstructionAcceleration} с использованием \texttt{super()} таким образом, чтобы через класс с множественным наследованием \texttt{ConstructionEquipment} вывести на печать время ускорения до 10 км/ч и от 10 до 20 км/ч для трёх типов техники при следующих параметрах:

\begin{itemize}
    \item Excavator — масса 12000 кг, мощность 250 л. с.;  
    \item Bulldozer — масса 18000 кг, мощность 400 л. с.;  
    \item Crane — масса 9000 кг, мощность 200 л. с.  
\end{itemize}

\textbf{3 часть}

На основе программного кода и параметров техники из 2 части текущего практикума отобразить через класс с множественным наследованием \texttt{ConstructionEquipment} расчёт времени разгона от 10 до 20 км/ч через каждые 2.5 км/ч, то есть до 12.5, 15, 17.5 и 20 км/ч, учитывая, что коэффициент мощности $k = 0{,}52$ увеличивается через каждые 2.5 км/ч на 0{,}08.

\textbf{Коэффициенты мощности для различных диапазонов скоростей:}
\begin{itemize}
    \item 10–12.5 км/ч: $k = 0{,}52$;  
    \item 12.5–15 км/ч: $k = 0{,}60$;  
    \item 15–17.5 км/ч: $k = 0{,}68$;  
    \item 17.5–20 км/ч: $k = 0{,}76$.  
\end{itemize}

\item[13] \textbf{1 часть}  
Написать класс Python \texttt{EmergencyAcceleration}, выполняющий расчёт времени разгона служебного транспорта от 0 до 60 км/ч и от 60 до 120 км/ч.

\textbf{Расчёт времени от 0 до 60 км/ч:}

\begin{equation}
t = \frac{m \cdot (\Delta V)^2}{2P_{\text{эфф}}},
\end{equation}

где:  
\begin{itemize}
    \item $t$ — время разгона, с;  
    \item $m$ — масса транспортного средства, кг;  
    \item $\Delta V$ — изменение скорости, м/с;  
    \item $P_{\text{эфф}}$ — эффективная мощность, Вт.  
\end{itemize}

\textbf{Формула расчёта эффективной мощности для разгона от 0 до 60 км/ч:}

\begin{equation}
P_{\text{эфф}} = P_{\text{макс}} \cdot k \cdot \eta,
\end{equation}

где:  
\begin{itemize}
    \item $P_{\text{макс}}$ — мощность двигателя, л. с.;  
    \item $k = 0{,}28$ — коэффициент использования мощности при экстренном старте;  
    \item $\eta = 0{,}85$ — КПД трансмиссии;  
    \item 1 л. с. = 735 Вт.  
\end{itemize}

\textbf{Расчёт времени от 60 до 120 км/ч:}

\begin{equation}
t = \frac{m \cdot (V_2^2 - V_1^2)}{2P_{\text{эфф}}},
\end{equation}

где:  
\begin{itemize}
    \item $V_1$ — начальная скорость, м/с;  
    \item $V_2$ — конечная скорость, м/с;  
    \item $P_{\text{эфф}}$ — эффективная мощность, Вт.  
\end{itemize}

\textbf{Формула расчёта эффективной мощности для ускорения от 60 до 120 км/ч:}

\begin{equation}
P_{\text{эфф}} = P_{\text{макс}} \cdot k \cdot \eta,
\end{equation}

где $k = 0{,}64$ — коэффициент использования мощности при высокой скорости.

\textbf{2 часть}

На основе программного кода предыдущего практикума текущего курса ООП выполнить наследование класса \texttt{EmergencyAcceleration} с использованием \texttt{super()} таким образом, чтобы через класс с множественным наследованием \texttt{EmergencyVehicle} вывести на печать время ускорения до 60 км/ч и от 60 до 120 км/ч для трёх типов транспорта при следующих параметрах:

\begin{itemize}
    \item Ambulance — масса 2800 кг, мощность 180 л. с.;  
    \item FireTruck — масса 14000 кг, мощность 450 л. с.;  
    \item PoliceCar — масса 2000 кг, мощность 300 л. с.  
\end{itemize}

\textbf{3 часть}

На основе программного кода и параметров транспорта из 2 части текущего практикума отобразить через класс с множественным наследованием \texttt{EmergencyVehicle} расчёт времени разгона от 60 до 120 км/ч через каждые 15 км/ч, то есть до 75, 90, 105 и 120 км/ч, учитывая, что коэффициент мощности $k = 0{,}64$ увеличивается через каждые 15 км/ч на 0{,}08.

\textbf{Коэффициенты мощности для различных диапазонов скоростей:}
\begin{itemize}
    \item 60–75 км/ч: $k = 0{,}64$;  
    \item 75–90 км/ч: $k = 0{,}72$;  
    \item 90–105 км/ч: $k = 0{,}80$;  
    \item 105–120 км/ч: $k = 0{,}88$.  
\end{itemize}

\item[14] \textbf{1 часть}  
Написать класс Python \texttt{PassengerAcceleration}, выполняющий расчёт времени разгона пассажирского транспорта от 0 до 30 км/ч и от 30 до 60 км/ч.

\textbf{Расчёт времени от 0 до 30 км/ч:}

\begin{equation}
t = \frac{m \cdot (\Delta V)^2}{2P_{\text{эфф}}},
\end{equation}

где:  
\begin{itemize}
    \item $t$ — время разгона, с;  
    \item $m$ — масса транспортного средства с пассажирами, кг;  
    \item $\Delta V$ — изменение скорости, м/с;  
    \item $P_{\text{эфф}}$ — эффективная мощность, Вт.  
\end{itemize}

\textbf{Формула расчёта эффективной мощности для разгона от 0 до 30 км/ч:}

\begin{equation}
P_{\text{эфф}} = P_{\text{макс}} \cdot k \cdot \eta,
\end{equation}

где:  
\begin{itemize}
    \item $P_{\text{макс}}$ — мощность двигателя, л. с.;  
    \item $k = 0{,}26$ — коэффициент использования мощности;  
    \item $\eta = 0{,}83$ — КПД трансмиссии;  
    \item 1 л. с. = 735 Вт.  
\end{itemize}

\textbf{Расчёт времени от 30 до 60 км/ч:}

\begin{equation}
t = \frac{m \cdot (V_2^2 - V_1^2)}{2P_{\text{эфф}}},
\end{equation}

где:  
\begin{itemize}
    \item $V_1$ — начальная скорость, м/с;  
    \item $V_2$ — конечная скорость, м/с;  
    \item $P_{\text{эфф}}$ — эффективная мощность, Вт.  
\end{itemize}

\textbf{Формула расчёта эффективной мощности для ускорения от 30 до 60 км/ч:}

\begin{equation}
P_{\text{эфф}} = P_{\text{макс}} \cdot k \cdot \eta,
\end{equation}

где $k = 0{,}60$ — коэффициент использования мощности при движении в городском цикле.

\textbf{2 часть}

На основе программного кода предыдущего практикума текущего курса ООП выполнить наследование класса \texttt{PassengerAcceleration} с использованием \texttt{super()} таким образом, чтобы через класс с множественным наследованием \texttt{PassengerVehicle} вывести на печать время ускорения до 30 км/ч и от 30 до 60 км/ч для трёх типов транспорта при следующих параметрах:

\begin{itemize}
    \item Taxi — масса 1600 кг, мощность 120 л. с.;  
    \item RideShare — масса 1500 кг, мощность 110 л. с.;  
    \item Limousine — масса 2500 кг, мощность 200 л. с.  
\end{itemize}

\textbf{3 часть}

На основе программного кода и параметров транспорта из 2 части текущего практикума отобразить через класс с множественным наследованием \texttt{PassengerVehicle} расчёт времени разгона от 30 до 60 км/ч через каждые 7.5 км/ч, то есть до 37.5, 45, 52.5 и 60 км/ч, учитывая, что коэффициент мощности $k = 0{,}60$ увеличивается через каждые 7.5 км/ч на 0{,}08.

\textbf{Коэффициенты мощности для различных диапазонов скоростей:}
\begin{itemize}
    \item 30–37.5 км/ч: $k = 0{,}60$;  
    \item 37.5–45 км/ч: $k = 0{,}68$;  
    \item 45–52.5 км/ч: $k = 0{,}76$;  
    \item 52.5–60 км/ч: $k = 0{,}84$.  
\end{itemize}

\item[15] \textbf{1 часть}  
Написать класс Python \texttt{BusAcceleration}, выполняющий расчёт времени разгона автобуса от 0 до 25 км/ч и от 25 до 50 км/ч.

\textbf{Расчёт времени от 0 до 25 км/ч:}

\begin{equation}
t = \frac{m \cdot (\Delta V)^2}{2P_{\text{эфф}}},
\end{equation}

где:  
\begin{itemize}
    \item $t$ — время разгона, с;  
    \item $m$ — масса автобуса с пассажирами, кг;  
    \item $\Delta V$ — изменение скорости, м/с;  
    \item $P_{\text{эфф}}$ — эффективная мощность, Вт.  
\end{itemize}

\textbf{Формула расчёта эффективной мощности для разгона от 0 до 25 км/ч:}

\begin{equation}
P_{\text{эфф}} = P_{\text{макс}} \cdot k \cdot \eta,
\end{equation}

где:  
\begin{itemize}
    \item $P_{\text{макс}}$ — мощность двигателя, л. с.;  
    \item $k = 0{,}24$ — коэффициент использования мощности при старте с остановки;  
    \item $\eta = 0{,}81$ — КПД трансмиссии;  
    \item 1 л. с. = 735 Вт.  
\end{itemize}

\textbf{Расчёт времени от 25 до 50 км/ч:}

\begin{equation}
t = \frac{m \cdot (V_2^2 - V_1^2)}{2P_{\text{эфф}}},
\end{equation}

где:  
\begin{itemize}
    \item $V_1$ — начальная скорость, м/с;  
    \item $V_2$ — конечная скорость, м/с;  
    \item $P_{\text{эфф}}$ — эффективная мощность, Вт.  
\end{itemize}

\textbf{Формула расчёта эффективной мощности для ускорения от 25 до 50 км/ч:}

\begin{equation}
P_{\text{эфф}} = P_{\text{макс}} \cdot k \cdot \eta,
\end{equation}

где $k = 0{,}58$ — коэффициент использования мощности при движении по маршруту.

\textbf{2 часть}

На основе программного кода предыдущего практикума текущего курса ООП выполнить наследование класса \texttt{BusAcceleration} с использованием \texttt{super()} таким образом, чтобы через класс с множественным наследованием \texttt{Bus} вывести на печать время ускорения до 25 км/ч и от 25 до 50 км/ч для трёх типов автобусов при следующих параметрах:

\begin{itemize}
    \item SchoolBus — масса 7500 кг, мощность 220 л. с.;  
    \item Coach — масса 12500 кг, мощность 380 л. с.;  
    \item Minibus — масса 3200 кг, мощность 130 л. с.  
\end{itemize}

\textbf{3 часть}

На основе программного кода и параметров автобусов из 2 части текущего практикума отобразить через класс с множественным наследованием \texttt{Bus} расчёт времени разгона от 25 до 50 км/ч через каждые 5 км/ч, то есть до 30, 35, 40, 45 и 50 км/ч, учитывая, что коэффициент мощности $k = 0{,}58$ увеличивается через каждые 5 км/ч на 0{,}06.

\textbf{Коэффициенты мощности для различных диапазонов скоростей:}
\begin{itemize}
    \item 25–30 км/ч: $k = 0{,}58$;  
    \item 30–35 км/ч: $k = 0{,}64$;  
    \item 35–40 км/ч: $k = 0{,}70$;  
    \item 40–45 км/ч: $k = 0{,}76$;  
    \item 45–50 км/ч: $k = 0{,}82$.  
\end{itemize}


\item[16] \textbf{1 часть}  
Написать класс Python \texttt{FreightAcceleration}, выполняющий расчёт времени разгона грузового транспорта от 0 до 40 км/ч и от 40 до 80 км/ч.

\textbf{Расчёт времени от 0 до 40 км/ч:}

\begin{equation}
t = \frac{m \cdot (\Delta V)^2}{2P_{\text{эфф}}},
\end{equation}

где:  
\begin{itemize}
    \item $t$ — время разгона, с;  
    \item $m$ — масса транспорта с грузом, кг;  
    \item $\Delta V$ — изменение скорости, м/с;  
    \item $P_{\text{эфф}}$ — эффективная мощность, Вт.  
\end{itemize}

\textbf{Формула расчёта эффективной мощности для разгона от 0 до 40 км/ч:}

\begin{equation}
P_{\text{эфф}} = P_{\text{макс}} \cdot k \cdot \eta,
\end{equation}

где:  
\begin{itemize}
    \item $P_{\text{макс}}$ — мощность двигателя, л. с.;  
    \item $k = 0{,}21$ — коэффициент использования мощности при старте с грузом;  
    \item $\eta = 0{,}79$ — КПД трансмиссии;  
    \item 1 л. с. = 735 Вт.  
\end{itemize}

\textbf{Расчёт времени от 40 до 80 км/ч:}

\begin{equation}
t = \frac{m \cdot (V_2^2 - V_1^2)}{2P_{\text{эфф}}},
\end{equation}

где:  
\begin{itemize}
    \item $V_1$ — начальная скорость, м/с;  
    \item $V_2$ — конечная скорость, м/с;  
    \item $P_{\text{эфф}}$ — эффективная мощность, Вт.  
\end{itemize}

\textbf{Формула расчёта эффективной мощности для ускорения от 40 до 80 км/ч:}

\begin{equation}
P_{\text{эфф}} = P_{\text{макс}} \cdot k \cdot \eta,
\end{equation}

где $k = 0{,}54$ — коэффициент использования мощности на трассе.

\textbf{2 часть}

На основе программного кода предыдущего практикума текущего курса ООП выполнить наследование класса \texttt{FreightAcceleration} с использованием \texttt{super()} таким образом, чтобы через класс с множественным наследованием \texttt{FreightVehicle} вывести на печать время ускорения до 40 км/ч и от 40 до 80 км/ч для трёх типов транспорта при следующих параметрах:

\begin{itemize}
    \item Truck — масса 12000 кг, мощность 300 л. с.;  
    \item Semi — масса 36000 кг, мощность 500 л. с.;  
    \item DumpTruck — масса 25000 кг, мощность 400 л. с.  
\end{itemize}

\textbf{3 часть}

На основе программного кода и параметров транспорта из 2 части текущего практикума отобразить через класс с множественным наследованием \texttt{FreightVehicle} расчёт времени разгона от 40 до 80 км/ч через каждые 10 км/ч, то есть до 50, 60, 70 и 80 км/ч, учитывая, что коэффициент мощности $k = 0{,}54$ увеличивается через каждые 10 км/ч на 0{,}08.

\textbf{Коэффициенты мощности для различных диапазонов скоростей:}
\begin{itemize}
    \item 40–50 км/ч: $k = 0{,}54$;  
    \item 50–60 км/ч: $k = 0{,}62$;  
    \item 60–70 км/ч: $k = 0{,}70$;  
    \item 70–80 км/ч: $k = 0{,}78$.  
\end{itemize}


\item[17] \textbf{1 часть}  
Написать класс Python \texttt{SpacecraftAcceleration}, выполняющий расчёт времени набора скорости космического аппарата от 0 до 1000 м/с и от 1000 до 3000 м/с.

\textbf{Расчёт времени от 0 до 1000 м/с:}

\begin{equation}
t = \frac{m \cdot (\Delta V)^2}{2P_{\text{эфф}}},
\end{equation}

где:  
\begin{itemize}
    \item $t$ — время ускорения, с;  
    \item $m$ — масса аппарата, кг;  
    \item $\Delta V$ — изменение скорости, м/с;  
    \item $P_{\text{эфф}}$ — эффективная мощность тяги, Вт (условно, 1 Н·м/с = 1 Вт).  
\end{itemize}

\textbf{Формула расчёта эффективной мощности для ускорения от 0 до 1000 м/с:}

\begin{equation}
P_{\text{эфф}} = P_{\text{макс}} \cdot k \cdot \eta,
\end{equation}

где:  
\begin{itemize}
    \item $P_{\text{макс}}$ — номинальная мощность двигателя, МВт;  
    \item $k = 0{,}23$ — коэффициент использования тяги на старте;  
    \item $\eta = 0{,}92$ — эффективность ракетного сопла;  
    \item 1 МВт = $10^6$ Вт.  
\end{itemize}

\textbf{Расчёт времени от 1000 до 3000 м/с:}

\begin{equation}
t = \frac{m \cdot (V_2^2 - V_1^2)}{2P_{\text{эфф}}},
\end{equation}

где:  
\begin{itemize}
    \item $V_1 = 1000$ м/с, $V_2 = 3000$ м/с;  
    \item $P_{\text{эфф}}$ — эффективная мощность, Вт.  
\end{itemize}

\textbf{Формула расчёта эффективной мощности для ускорения от 1000 до 3000 м/с:}

\begin{equation}
P_{\text{эфф}} = P_{\text{макс}} \cdot k \cdot \eta,
\end{equation}

где $k = 0{,}62$ — коэффициент использования тяги в вакууме.

\textbf{2 часть}

На основе программного кода предыдущего практикума текущего курса ООП выполнить наследование класса \texttt{SpacecraftAcceleration} с использованием \texttt{super()} таким образом, чтобы через класс с множественным наследованием \texttt{Spacecraft} вывести на печать время ускорения до 1000 м/с и от 1000 до 3000 м/с для трёх типов аппаратов при следующих параметрах:

\begin{itemize}
    \item Rocket — масса 200000 кг, мощность 25 МВт;  
    \item Spaceplane — масса 80000 кг, мощность 18 МВт;  
    \item Lander — масса 15000 кг, мощность 5 МВт.  
\end{itemize}

\textbf{3 часть}

На основе программного кода и параметров аппаратов из 2 части текущего практикума отобразить через класс с множественным наследованием \texttt{Spacecraft} расчёт времени ускорения от 1000 до 3000 м/с через каждые 500 м/с, то есть до 1500, 2000, 2500 и 3000 м/с, учитывая, что коэффициент мощности $k = 0{,}62$ увеличивается через каждые 500 м/с на 0{,}08.

\textbf{Коэффициенты мощности для различных диапазонов скоростей:}
\begin{itemize}
    \item 1000–1500 м/с: $k = 0{,}62$;  
    \item 1500–2000 м/с: $k = 0{,}70$;  
    \item 2000–2500 м/с: $k = 0{,}78$;  
    \item 2500–3000 м/с: $k = 0{,}86$.  
\end{itemize}


\item[18] \textbf{1 часть}  
Написать класс Python \texttt{WearableAcceleration}, выполняющий расчёт времени перехода носимого устройства из спящего режима в активный и далее до полной функциональности.

\textbf{Расчёт времени от спящего режима до активного:}

\begin{equation}
t = \frac{C \cdot (\Delta V)^2}{2P_{\text{эфф}}},
\end{equation}

где:  
\begin{itemize}
    \item $t$ — время пробуждения, с;  
    \item $C$ — эквивалентная ёмкость системы (моделирует инерцию старта), Дж/В²;  
    \item $\Delta V$ — изменение уровня активности (условные единицы, от 0 до 1);  
    \item $P_{\text{эфф}}$ — эффективная мощность, Вт.  
\end{itemize}

\textbf{Формула расчёта эффективной мощности для пробуждения:}

\begin{equation}
P_{\text{эфф}} = P_{\text{макс}} \cdot k \cdot \eta,
\end{equation}

где:  
\begin{itemize}
    \item $P_{\text{макс}}$ — пиковая мощность процессора, Вт;  
    \item $k = 0{,}27$ — коэффициент использования при старте;  
    \item $\eta = 0{,}95$ — КПД энергосистемы.  
\end{itemize}

\textbf{Расчёт времени от активного до полной функциональности:}

\begin{equation}
t = \frac{C \cdot (V_2^2 - V_1^2)}{2P_{\text{эфф}}},
\end{equation}

где:  
\begin{itemize}
    \item $V_1 = 0{,}3$, $V_2 = 1{,}0$ — уровни активности;  
    \item $P_{\text{эфф}}$ — эффективная мощность, Вт.  
\end{itemize}

\textbf{Формула расчёта эффективной мощности для полной активации:}

\begin{equation}
P_{\text{эфф}} = P_{\text{макс}} \cdot k \cdot \eta,
\end{equation}

где $k = 0{,}56$ — коэффициент при полной нагрузке.

\textbf{2 часть}

На основе программного кода предыдущего практикума текущего курса ООП выполнить наследование класса \texttt{WearableAcceleration} с использованием \texttt{super()} таким образом, чтобы через класс с множественным наследованием \texttt{WearableDevice} вывести на печать время перехода в активный режим и до полной функциональности для трёх типов устройств при следующих параметрах:

\begin{itemize}
    \item Smartwatch — $C = 0{,}02$ Дж/В², $P = 2{,}5$ Вт;  
    \item FitnessTracker — $C = 0{,}015$ Дж/В², $P = 1{,}8$ Вт;  
    \item AR\_Glasses — $C = 0{,}03$ Дж/В², $P = 4{,}0$ Вт.  
\end{itemize}

\textbf{3 часть}

На основе программного кода и параметров устройств из 2 части текущего практикума отобразить через класс с множественным наследованием \texttt{WearableDevice} расчёт времени перехода от активного состояния до полной функциональности через каждые 0.15 условных единиц, то есть до 0.45, 0.60, 0.75, 0.90 и 1.00, учитывая, что коэффициент мощности $k = 0{,}56$ увеличивается через каждые 0.15 на 0{,}08.

\textbf{Коэффициенты мощности для различных диапазонов активности:}
\begin{itemize}
    \item 0.30–0.45: $k = 0{,}56$;  
    \item 0.45–0.60: $k = 0{,}64$;  
    \item 0.60–0.75: $k = 0{,}72$;  
    \item 0.75–0.90: $k = 0{,}80$;  
    \item 0.90–1.00: $k = 0{,}88$.  
\end{itemize}


\item[19] \textbf{1 часть}  
Написать класс Python \texttt{MobileAcceleration}, выполняющий расчёт времени запуска мобильного устройства от нажатия кнопки до готовности к работе и до запуска сложного приложения.

\textbf{Расчёт времени от нажатия кнопки до готовности:}

\begin{equation}
t = \frac{C \cdot (\Delta V)^2}{2P_{\text{эфф}}},
\end{equation}

где:  
\begin{itemize}
    \item $t$ — время загрузки, с;  
    \item $C$ — эквивалентная ёмкость системы (модель инициализации), Дж/В²;  
    \item $\Delta V$ — изменение уровня готовности (от 0 до 1);  
    \item $P_{\text{эфф}}$ — эффективная мощность, Вт.  
\end{itemize}

\textbf{Формула расчёта эффективной мощности для загрузки ОС:}

\begin{equation}
P_{\text{эфф}} = P_{\text{макс}} \cdot k \cdot \eta,
\end{equation}

где:  
\begin{itemize}
    \item $P_{\text{макс}}$ — пиковая мощность, Вт;  
    \item $k = 0{,}29$ — коэффициент при старте;  
    \item $\eta = 0{,}93$ — КПД энергосистемы.  
\end{itemize}

\textbf{Расчёт времени от готовности до запуска приложения:}

\begin{equation}
t = \frac{C \cdot (V_2^2 - V_1^2)}{2P_{\text{эфф}}},
\end{equation}

где:  
\begin{itemize}
    \item $V_1 = 0{,}6$, $V_2 = 1{,}0$ — уровни загрузки;  
    \item $P_{\text{эфф}}$ — эффективная мощность, Вт.  
\end{itemize}

\textbf{Формула расчёта эффективной мощности для запуска приложения:}

\begin{equation}
P_{\text{эфф}} = P_{\text{макс}} \cdot k \cdot \eta,
\end{equation}

где $k = 0{,}59$ — коэффициент при полной загрузке.

\textbf{2 часть}

На основе программного кода предыдущего практикума текущего курса ООП выполнить наследование класса \texttt{MobileAcceleration} с использованием \texttt{super()} таким образом, чтобы через класс с множественным наследованием \texttt{MobileDevice} вывести на печать время загрузки и запуска приложения для трёх типов устройств при следующих параметрах:

\begin{itemize}
    \item Smartphone — $C = 0{,}018$ Дж/В², $P = 5{,}0$ Вт;  
    \item Tablet — $C = 0{,}025$ Дж/В², $P = 8{,}0$ Вт;  
    \item Ereader — $C = 0{,}008$ Дж/В², $P = 1{,}5$ Вт.  
\end{itemize}

\textbf{3 часть}

На основе программного кода и параметров устройств из 2 части текущего практикума отобразить через класс с множественным наследованием \texttt{MobileDevice} расчёт времени запуска приложения от уровня 0.6 до 1.0 через каждые 0.1, учитывая, что коэффициент мощности $k = 0{,}59$ увеличивается через каждые 0.1 на 0{,}07.

\textbf{Коэффициенты мощности для различных диапазонов загрузки:}
\begin{itemize}
    \item 0.6–0.7: $k = 0{,}59$;  
    \item 0.7–0.8: $k = 0{,}66$;  
    \item 0.8–0.9: $k = 0{,}73$;  
    \item 0.9–1.0: $k = 0{,}80$.  
\end{itemize}


\item[20] \textbf{1 часть}  
Написать класс Python \texttt{ComputerAcceleration}, выполняющий расчёт времени запуска вычислительного устройства от включения питания до полной готовности и до выполнения ресурсоёмкой задачи.

\textbf{Расчёт времени от включения до готовности:}

\begin{equation}
t = \frac{C \cdot (\Delta V)^2}{2P_{\text{эфф}}},
\end{equation}

где:  
\begin{itemize}
    \item $t$ — время запуска, с;  
    \item $C$ — эквивалентная ёмкость системы, Дж/В²;  
    \item $\Delta V$ — изменение уровня активности (от 0 до 1);  
    \item $P_{\text{эфф}}$ — эффективная мощность, Вт.  
\end{itemize}

\textbf{Формула расчёта эффективной мощности для загрузки:}

\begin{equation}
P_{\text{эфф}} = P_{\text{макс}} \cdot k \cdot \eta,
\end{equation}

где:  
\begin{itemize}
    \item $P_{\text{макс}}$ — пиковая мощность, Вт;  
    \item $k = 0{,}25$ — коэффициент при старте;  
    \item $\eta = 0{,}90$ — КПД блока питания.  
\end{itemize}

\textbf{Расчёт времени от готовности до выполнения задачи:}

\begin{equation}
t = \frac{C \cdot (V_2^2 - V_1^2)}{2P_{\text{эфф}}},
\end{equation}

где:  
\begin{itemize}
    \item $V_1 = 0{,}5$, $V_2 = 1{,}0$ — уровни нагрузки;  
    \item $P_{\text{эфф}}$ — эффективная мощность, Вт.  
\end{itemize}

\textbf{Формула расчёта эффективной мощности для выполнения задачи:}

\begin{equation}
P_{\text{эфф}} = P_{\text{макс}} \cdot k \cdot \eta,
\end{equation}

где $k = 0{,}61$ — коэффициент при высокой нагрузке.

\textbf{2 часть}

На основе программного кода предыдущего практикума текущего курса ООП выполнить наследование класса \texttt{ComputerAcceleration} с использованием \texttt{super()} таким образом, чтобы через класс с множественным наследованием \texttt{Computer} вывести на печать время запуска и выполнения задачи для трёх типов устройств при следующих параметрах:

\begin{itemize}
    \item Laptop — $C = 0{,}03$ Дж/В², $P = 65$ Вт;  
    \item Desktop — $C = 0{,}08$ Дж/В², $P = 650$ Вт;  
    \item Workstation — $C = 0{,}12$ Дж/В², $P = 1200$ Вт.  
\end{itemize}

\textbf{3 часть}

На основе программного кода и параметров устройств из 2 части текущего практикума отобразить через класс с множественным наследованием \texttt{Computer} расчёт времени выполнения задачи от уровня 0.5 до 1.0 через каждые 0.125, учитывая, что коэффициент мощности $k = 0{,}61$ увеличивается через каждые 0.125 на 0{,}07.

\textbf{Коэффициенты мощности для различных диапазонов нагрузки:}
\begin{itemize}
    \item 0.50–0.625: $k = 0{,}61$;  
    \item 0.625–0.75: $k = 0{,}68$;  
    \item 0.75–0.875: $k = 0{,}75$;  
    \item 0.875–1.00: $k = 0{,}82$.  
\end{itemize}


\item[21] \textbf{1 часть}  
Написать класс Python \texttt{NetworkAcceleration}, выполняющий расчёт времени подключения сетевого устройства к сети и времени полной активации всех сервисов.

\textbf{Расчёт времени от выключения до подключения к сети:}

\begin{equation}
t = \frac{C \cdot (\Delta V)^2}{2P_{\text{эфф}}},
\end{equation}

где:  
\begin{itemize}
    \item $t$ — время подключения, с;  
    \item $C$ — эквивалентная ёмкость системы (модель инициализации), Дж/В²;  
    \item $\Delta V$ — изменение уровня активности от 0 до 0.6;  
    \item $P_{\text{эфф}}$ — эффективная мощность, Вт.  
\end{itemize}

\textbf{Формула расчёта эффективной мощности для подключения к сети:}

\begin{equation}
P_{\text{эфф}} = P_{\text{макс}} \cdot k \cdot \eta,
\end{equation}

где:  
\begin{itemize}
    \item $P_{\text{макс}}$ — пиковая мощность устройства, Вт;  
    \item $k = 0{,}25$ — коэффициент использования при старте;  
    \item $\eta = 0{,}94$ — КПД блока питания.  
\end{itemize}

\textbf{Расчёт времени от подключения до полной активации сервисов:}

\begin{equation}
t = \frac{C \cdot (V_2^2 - V_1^2)}{2P_{\text{эфф}}},
\end{equation}

где:  
\begin{itemize}
    \item $V_1 = 0{,}6$, $V_2 = 1{,}0$ — уровни активации;  
    \item $P_{\text{эфф}}$ — эффективная мощность, Вт.  
\end{itemize}

\textbf{Формула расчёта эффективной мощности для полной активации:}

\begin{equation}
P_{\text{эфф}} = P_{\text{макс}} \cdot k \cdot \eta,
\end{equation}

где $k = 0{,}60$ — коэффициент использования при полной нагрузке.

\textbf{2 часть}

На основе программного кода предыдущего практикума текущего курса ООП выполнить наследование класса \texttt{NetworkAcceleration} с использованием \texttt{super()} таким образом, чтобы через класс с множественным наследованием \texttt{NetworkDevice} вывести на печать время подключения к сети и время полной активации сервисов для трёх типов устройств при следующих параметрах:

\begin{itemize}
    \item Router — $C = 0{,}008$ Дж/В², $P = 12$ Вт;  
    \item Switch — $C = 0{,}012$ Дж/В², $P = 25$ Вт;  
    \item Firewall — $C = 0{,}018$ Дж/В², $P = 40$ Вт.  
\end{itemize}

\textbf{3 часть}

На основе программного кода и параметров устройств из 2 части текущего практикума отобразить через класс с множественным наследованием \texttt{NetworkDevice} расчёт времени активации сервисов от уровня 0.6 до 1.0 через каждые 0.1, учитывая, что коэффициент мощности $k = 0{,}60$ увеличивается через каждые 0.1 на 0{,}07.

\textbf{Коэффициенты мощности для различных диапазонов активации:}
\begin{itemize}
    \item 0.6–0.7: $k = 0{,}60$;  
    \item 0.7–0.8: $k = 0{,}67$;  
    \item 0.8–0.9: $k = 0{,}74$;  
    \item 0.9–1.0: $k = 0{,}81$.  
\end{itemize}


\item[22] \textbf{1 часть}  
Написать класс Python \texttt{OfficeAcceleration}, выполняющий расчёт времени готовности офисной техники к работе и до полной функциональности.

\textbf{Расчёт времени от выключения до готовности:}

\begin{equation}
t = \frac{C \cdot (\Delta V)^2}{2P_{\text{эфф}}},
\end{equation}

где:  
\begin{itemize}
    \item $t$ — время готовности, с;  
    \item $C$ — эквивалентная ёмкость системы, Дж/В²;  
    \item $\Delta V$ — изменение уровня активности от 0 до 0.5;  
    \item $P_{\text{эфф}}$ — эффективная мощность, Вт.  
\end{itemize}

\textbf{Формула расчёта эффективной мощности для готовности:}

\begin{equation}
P_{\text{эфф}} = P_{\text{макс}} \cdot k \cdot \eta,
\end{equation}

где:  
\begin{itemize}
    \item $P_{\text{макс}}$ — пиковая мощность, Вт;  
    \item $k = 0{,}24$ — коэффициент при старте;  
    \item $\eta = 0{,}92$ — КПД блока питания.  
\end{itemize}

\textbf{Расчёт времени от готовности до полной функциональности:}

\begin{equation}
t = \frac{C \cdot (V_2^2 - V_1^2)}{2P_{\text{эфф}}},
\end{equation}

где:  
\begin{itemize}
    \item $V_1 = 0{,}5$, $V_2 = 1{,}0$;  
    \item $P_{\text{эфф}}$ — эффективная мощность, Вт.  
\end{itemize}

\textbf{Формула расчёта эффективной мощности для полной функциональности:}

\begin{equation}
P_{\text{эфф}} = P_{\text{макс}} \cdot k \cdot \eta,
\end{equation}

где $k = 0{,}58$ — коэффициент при полной загрузке.

\textbf{2 часть}

На основе программного кода предыдущего практикума текущего курса ООП выполнить наследование класса \texttt{OfficeAcceleration} с использованием \texttt{super()} таким образом, чтобы через класс с множественным наследованием \texttt{OfficeDevice} вывести на печать время готовности и полной функциональности для трёх типов устройств при следующих параметрах:

\begin{itemize}
    \item Printer — $C = 0{,}015$ Дж/В², $P = 300$ Вт;  
    \item Scanner — $C = 0{,}009$ Дж/В², $P = 20$ Вт;  
    \item Copier — $C = 0{,}022$ Дж/В², $P = 500$ Вт.  
\end{itemize}

\textbf{3 часть}

На основе программного кода и параметров устройств из 2 части текущего практикума отобразить через класс с множественным наследованием \texttt{OfficeDevice} расчёт времени полной функциональности от уровня 0.5 до 1.0 через каждые 0.125, учитывая, что коэффициент мощности $k = 0{,}58$ увеличивается через каждые 0.125 на 0{,}06.

\textbf{Коэффициенты мощности для различных диапазонов активации:}
\begin{itemize}
    \item 0.50–0.625: $k = 0{,}58$;  
    \item 0.625–0.75: $k = 0{,}64$;  
    \item 0.75–0.875: $k = 0{,}70$;  
    \item 0.875–1.00: $k = 0{,}76$.  
\end{itemize}


\item[23] \textbf{1 часть}  
Написать класс Python \texttt{ImagingAcceleration}, выполняющий расчёт времени готовности устройства захвата изображения к съёмке и до начала записи видео.

\textbf{Расчёт времени от выключения до готовности к фото:}

\begin{equation}
t = \frac{C \cdot (\Delta V)^2}{2P_{\text{эфф}}},
\end{equation}

где:  
\begin{itemize}
    \item $t$ — время готовности, с;  
    \item $C$ — эквивалентная ёмкость системы, Дж/В²;  
    \item $\Delta V = 0{,}5$ — изменение уровня активности;  
    \item $P_{\text{эфф}}$ — эффективная мощность, Вт.  
\end{itemize}

\textbf{Формула расчёта эффективной мощности для фото:}

\begin{equation}
P_{\text{эфф}} = P_{\text{макс}} \cdot k \cdot \eta,
\end{equation}

где:  
\begin{itemize}
    \item $P_{\text{макс}}$ — пиковая мощность, Вт;  
    \item $k = 0{,}26$ — коэффициент при старте;  
    \item $\eta = 0{,}91$ — КПД энергосистемы.  
\end{itemize}

\textbf{Расчёт времени от фото до записи видео:}

\begin{equation}
t = \frac{C \cdot (V_2^2 - V_1^2)}{2P_{\text{эфф}}},
\end{equation}

где:  
\begin{itemize}
    \item $V_1 = 0{,}5$, $V_2 = 1{,}0$;  
    \item $P_{\text{эфф}}$ — эффективная мощность, Вт.  
\end{itemize}

\textbf{Формула расчёта эффективной мощности для видео:}

\begin{equation}
P_{\text{эфф}} = P_{\text{макс}} \cdot k \cdot \eta,
\end{equation}

где $k = 0{,}61$ — коэффициент при видеозаписи.

\textbf{2 часть}

На основе программного кода предыдущего практикума текущего курса ООП выполнить наследование класса \texttt{ImagingAcceleration} с использованием \texttt{super()} таким образом, чтобы через класс с множественным наследованием \texttt{ImagingDevice} вывести на печать время готовности и начала записи для трёх типов устройств при следующих параметрах:

\begin{itemize}
    \item Camera — $C = 0{,}007$ Дж/В², $P = 10$ Вт;  
    \item Camcorder — $C = 0{,}014$ Дж/В², $P = 25$ Вт;  
    \item DroneCam — $C = 0{,}011$ Дж/В², $P = 18$ Вт.  
\end{itemize}

\textbf{3 часть}

На основе программного кода и параметров устройств из 2 части текущего практикума отобразить через класс с множественным наследованием \texttt{ImagingDevice} расчёт времени записи видео от уровня 0.5 до 1.0 через каждые 0.125, учитывая, что коэффициент мощности $k = 0{,}61$ увеличивается через каждые 0.125 на 0{,}07.

\textbf{Коэффициенты мощности для различных диапазонов активации:}
\begin{itemize}
    \item 0.50–0.625: $k = 0{,}61$;  
    \item 0.625–0.75: $k = 0{,}68$;  
    \item 0.75–0.875: $k = 0{,}75$;  
    \item 0.875–1.00: $k = 0{,}82$.  
\end{itemize}


\item[24] \textbf{1 часть}  
Написать класс Python \texttt{CookingAcceleration}, выполняющий расчёт времени выхода кухонной техники на рабочую температуру и до полного цикла приготовления.

\textbf{Расчёт времени от включения до рабочей температуры:}

\begin{equation}
t = \frac{C \cdot (\Delta T)^2}{2P_{\text{эфф}}},
\end{equation}

где:  
\begin{itemize}
    \item $t$ — время нагрева, с;  
    \item $C$ — теплоёмкость камеры, Дж/°C;  
    \item $\Delta T$ — изменение температуры, °C;  
    \item $P_{\text{эфф}}$ — эффективная мощность, Вт.  
\end{itemize}

\textbf{Формула расчёта эффективной мощности для нагрева:}

\begin{equation}
P_{\text{эфф}} = P_{\text{макс}} \cdot k \cdot \eta,
\end{equation}

где:  
\begin{itemize}
    \item $P_{\text{макс}}$ — номинальная мощность, Вт;  
    \item $k = 0{,}27$ — коэффициент при старте;  
    \item $\eta = 0{,}88$ — КПД нагревательного элемента.  
\end{itemize}

\textbf{Расчёт времени от рабочей температуры до завершения цикла:}

\begin{equation}
t = \frac{C \cdot (T_2^2 - T_1^2)}{2P_{\text{эфф}}},
\end{equation}

где:  
\begin{itemize}
    \item $T_1 = 150$, $T_2 = 250$ °C;  
    \item $P_{\text{эфф}}$ — эффективная мощность, Вт.  
\end{itemize}

\textbf{Формула расчёта эффективной мощности для приготовления:}

\begin{equation}
P_{\text{эфф}} = P_{\text{макс}} \cdot k \cdot \eta,
\end{equation}

где $k = 0{,}62$ — коэффициент при рабочем режиме.

\textbf{2 часть}

На основе программного кода предыдущего практикума текущего курса ООП выполнить наследование класса \texttt{CookingAcceleration} с использованием \texttt{super()} таким образом, чтобы через класс с множественным наследованием \texttt{CookingAppliance} вывести на печать время нагрева и приготовления для трёх типов приборов при следующих параметрах:

\begin{itemize}
    \item Microwave — $C = 120$ Дж/°C, $P = 1000$ Вт;  
    \item Oven — $C = 450$ Дж/°C, $P = 2500$ Вт;  
    \item AirFryer — $C = 200$ Дж/°C, $P = 1500$ Вт.  
\end{itemize}

\textbf{3 часть}

На основе программного кода и параметров приборов из 2 части текущего практикума отобразить через класс с множественным наследованием \texttt{CookingAppliance} расчёт времени приготовления от 150 °C до 250 °C через каждые 25 °C, учитывая, что коэффициент мощности $k = 0{,}62$ увеличивается через каждые 25 °C на 0{,}08.

\textbf{Коэффициенты мощности для различных диапазонов температур:}
\begin{itemize}
    \item 150–175 °C: $k = 0{,}62$;  
    \item 175–200 °C: $k = 0{,}70$;  
    \item 200–225 °C: $k = 0{,}78$;  
    \item 225–250 °C: $k = 0{,}86$.  
\end{itemize}


\item[25] \textbf{1 часть}  
Написать класс Python \texttt{CoolingAcceleration}, выполняющий расчёт времени охлаждения холодильника от 25 °C до 4 °C и от 4 °C до -18 °C.

\textbf{Расчёт времени от 25 °C до 4 °C:}

\begin{equation}
t = \frac{C \cdot (\Delta T)^2}{2P_{\text{эфф}}},
\end{equation}

где:  
\begin{itemize}
    \item $t$ — время охлаждения, с;  
    \item $C$ — теплоёмкость содержимого, Дж/°C;  
    \item $\Delta T = 21$ °C;  
    \item $P_{\text{эфф}}$ — эффективная холодопроизводительность, Вт.  
\end{itemize}

\textbf{Формула расчёта эффективной холодопроизводительности для охлаждения до 4 °C:}

\begin{equation}
P_{\text{эфф}} = P_{\text{макс}} \cdot k \cdot \eta,
\end{equation}

где:  
\begin{itemize}
    \item $P_{\text{макс}}$ — номинальная мощность компрессора, Вт;  
    \item $k = 0{,}35$ — коэффициент использования мощности;  
    \item $\eta = 0{,}70$ — КПД холодильного цикла.  
\end{itemize}

\textbf{Расчёт времени от 4 °C до -18 °C:}

\begin{equation}
t = \frac{C \cdot (T_2^2 - T_1^2)}{2P_{\text{эфф}}},
\end{equation}

где:  
\begin{itemize}
    \item $T_1 = 4$, $T_2 = -18$ (в °C, но используется квадрат разности);  
    \item $P_{\text{эфф}}$ — эффективная холодопроизводительность, Вт.  
\end{itemize}

\textbf{Формула расчёта эффективной холодопроизводительности для заморозки:}

\begin{equation}
P_{\text{эфф}} = P_{\text{макс}} \cdot k \cdot \eta,
\end{equation}

где $k = 0{,}60$ — коэффициент использования мощности в режиме заморозки.

\textbf{2 часть}

На основе программного кода предыдущего практикума текущего курса ООП выполнить наследование класса \texttt{CoolingAcceleration} с использованием \texttt{super()} таким образом, чтобы через класс с множественным наследованием \texttt{CoolingAppliance} вывести на печать время охлаждения до 4 °C и от 4 °C до -18 °C для трёх типов устройств при следующих параметрах:

\begin{itemize}
    \item Refrigerator — $C = 8000$ Дж/°C, $P = 120$ Вт;  
    \item Freezer — $C = 6000$ Дж/°C, $P = 150$ Вт;  
    \item MiniFridge — $C = 3000$ Дж/°C, $P = 60$ Вт.  
\end{itemize}

\textbf{3 часть}

На основе программного кода и параметров устройств из 2 части текущего практикума отобразить через класс с множественным наследованием \texttt{CoolingAppliance} расчёт времени охлаждения от 4 °C до -18 °C через каждые 4.4 °C, то есть до -0.4, -4.8, -9.2, -13.6 и -18 °C, учитывая, что коэффициент мощности $k = 0{,}60$ увеличивается через каждые 4.4 °C на 0{,}08.

\textbf{Коэффициенты мощности для различных диапазонов температур:}
\begin{itemize}
    \item 4 → -0.4 °C: $k = 0{,}60$;  
    \item -0.4 → -4.8 °C: $k = 0{,}68$;  
    \item -4.8 → -9.2 °C: $k = 0{,}76$;  
    \item -9.2 → -13.6 °C: $k = 0{,}84$;  
    \item -13.6 → -18 °C: $k = 0{,}92$.  
\end{itemize}


\item[26] \textbf{1 часть}  
Написать класс Python \texttt{LaundryAcceleration}, выполняющий расчёт времени запуска стиральной машины от включения до начала стирки и до завершения полного цикла.

\textbf{Расчёт времени от включения до начала стирки:}

\begin{equation}
t = \frac{C \cdot (\Delta V)^2}{2P_{\text{эфф}}},
\end{equation}

где:  
\begin{itemize}
    \item $t$ — время запуска, с;  
    \item $C$ — эквивалентная ёмкость системы, Дж/В²;  
    \item $\Delta V = 0{,}4$ — изменение уровня активности (от 0 до 0.4);  
    \item $P_{\text{эфф}}$ — эффективная мощность, Вт.  
\end{itemize}

\textbf{Формула расчёта эффективной мощности для запуска:}

\begin{equation}
P_{\text{эфф}} = P_{\text{макс}} \cdot k \cdot \eta,
\end{equation}

где:  
\begin{itemize}
    \item $P_{\text{макс}}$ — номинальная мощность, Вт;  
    \item $k = 0{,}26$ — коэффициент при старте;  
    \item $\eta = 0{,}89$ — КПД электродвигателя.  
\end{itemize}

\textbf{Расчёт времени от начала стирки до завершения цикла:}

\begin{equation}
t = \frac{C \cdot (V_2^2 - V_1^2)}{2P_{\text{эфф}}},
\end{equation}

где:  
\begin{itemize}
    \item $V_1 = 0{,}4$, $V_2 = 1{,}0$;  
    \item $P_{\text{эфф}}$ — эффективная мощность, Вт.  
\end{itemize}

\textbf{Формула расчёта эффективной мощности для полного цикла:}

\begin{equation}
P_{\text{эфф}} = P_{\text{макс}} \cdot k \cdot \eta,
\end{equation}

где $k = 0{,}59$ — коэффициент при полной нагрузке.

\textbf{2 часть}

На основе программного кода предыдущего практикума текущего курса ООП выполнить наследование класса \texttt{LaundryAcceleration} с использованием \texttt{super()} таким образом, чтобы через класс с множественным наследованием \texttt{LaundryAppliance} вывести на печать время запуска и завершения цикла для трёх типов устройств при следующих параметрах:

\begin{itemize}
    \item WashingMachine — $C = 0{,}018$ Дж/В², $P = 2000$ Вт;  
    \item Dryer — $C = 0{,}022$ Дж/В², $P = 2500$ Вт;  
    \item Iron — $C = 0{,}006$ Дж/В², $P = 1200$ Вт.  
\end{itemize}

\textbf{3 часть}

На основе программного кода и параметров устройств из 2 части текущего практикума отобразить через класс с множественным наследованием \texttt{LaundryAppliance} расчёт времени завершения цикла от уровня 0.4 до 1.0 через каждые 0.15, учитывая, что коэффициент мощности $k = 0{,}59$ увеличивается через каждые 0.15 на 0{,}07.

\textbf{Коэффициенты мощности для различных диапазонов активности:}
\begin{itemize}
    \item 0.40–0.55: $k = 0{,}59$;  
    \item 0.55–0.70: $k = 0{,}66$;  
    \item 0.70–0.85: $k = 0{,}73$;  
    \item 0.85–1.00: $k = 0{,}80$.  
\end{itemize}


\item[27] \textbf{1 часть}  
Написать класс Python \texttt{CleaningAcceleration}, выполняющий расчёт времени готовности уборочной техники к работе и до завершения полного цикла уборки.

\textbf{Расчёт времени от выключения до готовности:}

\begin{equation}
t = \frac{C \cdot (\Delta V)^2}{2P_{\text{эфф}}},
\end{equation}

где:  
\begin{itemize}
    \item $t$ — время готовности, с;  
    \item $C$ — эквивалентная ёмкость системы, Дж/В²;  
    \item $\Delta V = 0{,}5$;  
    \item $P_{\text{эфф}}$ — эффективная мощность, Вт.  
\end{itemize}

\textbf{Формула расчёта эффективной мощности для готовности:}

\begin{equation}
P_{\text{эфф}} = P_{\text{макс}} \cdot k \cdot \eta,
\end{equation}

где:  
\begin{itemize}
    \item $P_{\text{макс}}$ — пиковая мощность, Вт;  
    \item $k = 0{,}25$;  
    \item $\eta = 0{,}90$.  
\end{itemize}

\textbf{Расчёт времени от готовности до завершения уборки:}

\begin{equation}
t = \frac{C \cdot (V_2^2 - V_1^2)}{2P_{\text{эфф}}},
\end{equation}

где:  
\begin{itemize}
    \item $V_1 = 0{,}5$, $V_2 = 1{,}0$;  
    \item $P_{\text{эфф}}$ — эффективная мощность, Вт.  
\end{itemize}

\textbf{Формула расчёта эффективной мощности для полной уборки:}

\begin{equation}
P_{\text{эфф}} = P_{\text{макс}} \cdot k \cdot \eta,
\end{equation}

где $k = 0{,}60$ — коэффициент при полной нагрузке.

\textbf{2 часть}

На основе программного кода предыдущего практикума текущего курса ООП выполнить наследование класса \texttt{CleaningAcceleration} с использованием \texttt{super()} таким образом, чтобы через класс с множественным наследованием \texttt{CleaningDevice} вывести на печать время готовности и завершения уборки для трёх типов устройств при следующих параметрах:

\begin{itemize}
    \item Vacuum — $C = 0{,}009$ Дж/В², $P = 200$ Вт;  
    \item MopRobot — $C = 0{,}014$ Дж/В², $P = 150$ Вт;  
    \item SteamCleaner — $C = 0{,}011$ Дж/В², $P = 1800$ Вт.  
\end{itemize}

\textbf{3 часть}

На основе программного кода и параметров устройств из 2 части текущего практикума отобразить через класс с множественным наследованием \texttt{CleaningDevice} расчёт времени уборки от уровня 0.5 до 1.0 через каждые 0.125, учитывая, что коэффициент мощности $k = 0{,}60$ увеличивается через каждые 0.125 на 0{,}06.

\textbf{Коэффициенты мощности для различных диапазонов активности:}
\begin{itemize}
    \item 0.50–0.625: $k = 0{,}60$;  
    \item 0.625–0.75: $k = 0{,}66$;  
    \item 0.75–0.875: $k = 0{,}72$;  
    \item 0.875–1.00: $k = 0{,}78$.  
\end{itemize}


\item[28] \textbf{1 часть}  
Написать класс Python \texttt{ClimateAcceleration}, выполняющий расчёт времени выхода климатической техники на заданный режим и до полной стабилизации микроклимата.

\textbf{Расчёт времени от включения до заданного режима:}

\begin{equation}
t = \frac{C \cdot (\Delta V)^2}{2P_{\text{эфф}}},
\end{equation}

где:  
\begin{itemize}
    \item $t$ — время установки режима, с;  
    \item $C$ — эквивалентная ёмкость системы, Дж/В²;  
    \item $\Delta V = 0{,}5$;  
    \item $P_{\text{эфф}}$ — эффективная мощность, Вт.  
\end{itemize}

\textbf{Формула расчёта эффективной мощности для установки режима:}

\begin{equation}
P_{\text{эфф}} = P_{\text{макс}} \cdot k \cdot \eta,
\end{equation}

где:  
\begin{itemize}
    \item $P_{\text{макс}}$ — номинальная мощность, Вт;  
    \item $k = 0{,}24$;  
    \item $\eta = 0{,}91$.  
\end{itemize}

\textbf{Расчёт времени от заданного режима до стабилизации:}

\begin{equation}
t = \frac{C \cdot (V_2^2 - V_1^2)}{2P_{\text{эфф}}},
\end{equation}

где:  
\begin{itemize}
    \item $V_1 = 0{,}5$, $V_2 = 1{,}0$;  
    \item $P_{\text{эфф}}$ — эффективная мощность, Вт.  
\end{itemize}

\textbf{Формула расчёта эффективной мощности для стабилизации:}

\begin{equation}
P_{\text{эфф}} = P_{\text{макс}} \cdot k \cdot \eta,
\end{equation}

где $k = 0{,}58$ — коэффициент при полной нагрузке.

\textbf{2 часть}

На основе программного кода предыдущего практикума текущего курса ООП выполнить наследование класса \texttt{ClimateAcceleration} с использованием \texttt{super()} таким образом, чтобы через класс с множественным наследованием \texttt{ClimateDevice} вывести на печать время установки режима и стабилизации для трёх типов устройств при следующих параметрах:

\begin{itemize}
    \item Thermostat — $C = 0{,}005$ Дж/В², $P = 5$ Вт;  
    \item Humidifier — $C = 0{,}008$ Дж/В², $P = 30$ Вт;  
    \item AirPurifier — $C = 0{,}007$ Дж/В², $P = 45$ Вт.  
\end{itemize}

\textbf{3 часть}

На основе программного кода и параметров устройств из 2 части текущего практикума отобразить через класс с множественным наследованием \texttt{ClimateDevice} расчёт времени стабилизации от уровня 0.5 до 1.0 через каждые 0.125, учитывая, что коэффициент мощности $k = 0{,}58$ увеличивается через каждые 0.125 на 0{,}06.

\textbf{Коэффициенты мощности для различных диапазонов активности:}
\begin{itemize}
    \item 0.50–0.625: $k = 0{,}58$;  
    \item 0.625–0.75: $k = 0{,}64$;  
    \item 0.75–0.875: $k = 0{,}70$;  
    \item 0.875–1.00: $k = 0{,}76$.  
\end{itemize}


\item[29] \textbf{1 часть}  
Написать класс Python \texttt{SecurityAcceleration}, выполняющий расчёт времени активации системы безопасности и до полной готовности всех модулей.

\textbf{Расчёт времени от выключения до активации:}

\begin{equation}
t = \frac{C \cdot (\Delta V)^2}{2P_{\text{эфф}}},
\end{equation}

где:  
\begin{itemize}
    \item $t$ — время активации, с;  
    \item $C$ — эквивалентная ёмкость системы, Дж/В²;  
    \item $\Delta V = 0{,}4$;  
    \item $P_{\text{эфф}}$ — эффективная мощность, Вт.  
\end{itemize}

\textbf{Формула расчёта эффективной мощности для активации:}

\begin{equation}
P_{\text{эфф}} = P_{\text{макс}} \cdot k \cdot \eta,
\end{equation}

где:  
\begin{itemize}
    \item $P_{\text{макс}}$ — пиковая мощность, Вт;  
    \item $k = 0{,}23$;  
    \item $\eta = 0{,}93$.  
\end{itemize}

\textbf{Расчёт времени от активации до полной готовности:}

\begin{equation}
t = \frac{C \cdot (V_2^2 - V_1^2)}{2P_{\text{эфф}}},
\end{equation}

где:  
\begin{itemize}
    \item $V_1 = 0{,}4$, $V_2 = 1{,}0$;  
    \item $P_{\text{эфф}}$ — эффективная мощность, Вт.  
\end{itemize}

\textbf{Формула расчёта эффективной мощности для полной готовности:}

\begin{equation}
P_{\text{эфф}} = P_{\text{макс}} \cdot k \cdot \eta,
\end{equation}

где $k = 0{,}59$ — коэффициент при полной нагрузке.

\textbf{2 часть}

На основе программного кода предыдущего практикума текущего курса ООП выполнить наследование класса \texttt{SecurityAcceleration} с использованием \texttt{super()} таким образом, чтобы через класс с множественным наследованием \texttt{SecurityDevice} вывести на печать время активации и готовности для трёх типов устройств при следующих параметрах:

\begin{itemize}
    \item SmartLock — $C = 0{,}004$ Дж/В², $P = 3$ Вт;  
    \item SecurityCamera — $C = 0{,}009$ Дж/В², $P = 12$ Вт;  
    \item AlarmSystem — $C = 0{,}007$ Дж/В², $P = 25$ Вт.  
\end{itemize}

\textbf{3 часть}

На основе программного кода и параметров устройств из 2 части текущего практикума отобразить через класс с множественным наследованием \texttt{SecurityDevice} расчёт времени готовности от уровня 0.4 до 1.0 через каждые 0.15, учитывая, что коэффициент мощности $k = 0{,}59$ увеличивается через каждые 0.15 на 0{,}07.

\textbf{Коэффициенты мощности для различных диапазонов активности:}
\begin{itemize}
    \item 0.40–0.55: $k = 0{,}59$;  
    \item 0.55–0.70: $k = 0{,}66$;  
    \item 0.70–0.85: $k = 0{,}73$;  
    \item 0.85–1.00: $k = 0{,}80$.  
\end{itemize}


\item[30] \textbf{1 часть}  
Написать класс Python \texttt{LightingAcceleration}, выполняющий расчёт времени включения осветительной системы и до достижения полной яркости с анимацией.

\textbf{Расчёт времени от выключения до включения:}

\begin{equation}
t = \frac{C \cdot (\Delta V)^2}{2P_{\text{эфф}}},
\end{equation}

где:  
\begin{itemize}
    \item $t$ — время включения, с;  
    \item $C$ — эквивалентная ёмкость системы, Дж/В²;  
    \item $\Delta V = 0{,}3$;  
    \item $P_{\text{эфф}}$ — эффективная мощность, Вт.  
\end{itemize}

\textbf{Формула расчёта эффективной мощности для включения:}

\begin{equation}
P_{\text{эфф}} = P_{\text{макс}} \cdot k \cdot \eta,
\end{equation}

где:  
\begin{itemize}
    \item $P_{\text{макс}}$ — номинальная мощность, Вт;  
    \item $k = 0{,}22$;  
    \item $\eta = 0{,}95$.  
\end{itemize}

\textbf{Расчёт времени от включения до полной яркости с анимацией:}

\begin{equation}
t = \frac{C \cdot (V_2^2 - V_1^2)}{2P_{\text{эфф}}},
\end{equation}

где:  
\begin{itemize}
    \item $V_1 = 0{,}3$, $V_2 = 1{,}0$;  
    \item $P_{\text{эфф}}$ — эффективная мощность, Вт.  
\end{itemize}

\textbf{Формула расчёта эффективной мощности для полной яркости:}

\begin{equation}
P_{\text{эфф}} = P_{\text{макс}} \cdot k \cdot \eta,
\end{equation}

где $k = 0{,}57$ — коэффициент при полной нагрузке.

\textbf{2 часть}

На основе программного кода предыдущего практикума текущего курса ООП выполнить наследование класса \texttt{LightingAcceleration} с использованием \texttt{super()} таким образом, чтобы через класс с множественным наследованием \texttt{LightingSystem} вывести на печать время включения и достижения полной яркости для трёх типов устройств при следующих параметрах:

\begin{itemize}
    \item SmartBulb — $C = 0{,}002$ Дж/В², $P = 10$ Вт;  
    \item SmartPlug — $C = 0{,}003$ Дж/В², $P = 2000$ Вт;  
    \item LightStrip — $C = 0{,}006$ Дж/В², $P = 60$ Вт.  
\end{itemize}

\textbf{3 часть}

На основе программного кода и параметров устройств из 2 части текущего практикума отобразить через класс с множественным наследованием \texttt{LightingSystem} расчёт времени достижения полной яркости от уровня 0.3 до 1.0 через каждые 0.175, учитывая, что коэффициент мощности $k = 0{,}57$ увеличивается через каждые 0.175 на 0{,}07.

\textbf{Коэффициенты мощности для различных диапазонов активности:}
\begin{itemize}
    \item 0.30–0.475: $k = 0{,}57$;  
    \item 0.475–0.65: $k = 0{,}64$;  
    \item 0.65–0.825: $k = 0{,}71$;  
    \item 0.825–1.00: $k = 0{,}78$.  
\end{itemize}

\item[31] \textbf{1 часть}  
Написать класс Python \texttt{GamingAcceleration}, выполняющий расчёт времени запуска игрового устройства от включения до готовности к игре и до полной загрузки игры.

\textbf{Расчёт времени от включения до готовности:}

\begin{equation}
t = \frac{C \cdot (\Delta V)^2}{2P_{\text{эфф}}},
\end{equation}

где:  
\begin{itemize}
    \item $t$ — время запуска, с;  
    \item $C$ — эквивалентная ёмкость системы, Дж/В²;  
    \item $\Delta V = 0{,}4$ — изменение уровня активности;  
    \item $P_{\text{эфф}}$ — эффективная мощность, Вт.  
\end{itemize}

\textbf{Формула расчёта эффективной мощности для запуска:}

\begin{equation}
P_{\text{эфф}} = P_{\text{макс}} \cdot k \cdot \eta,
\end{equation}

где:  
\begin{itemize}
    \item $P_{\text{макс}}$ — пиковая мощность, Вт;  
    \item $k = 0{,}25$ — коэффициент при старте;  
    \item $\eta = 0{,}90$ — КПД блока питания.  
\end{itemize}

\textbf{Расчёт времени от готовности до полной загрузки игры:}

\begin{equation}
t = \frac{C \cdot (V_2^2 - V_1^2)}{2P_{\text{эфф}}},
\end{equation}

где:  
\begin{itemize}
    \item $V_1 = 0{,}4$, $V_2 = 1{,}0$;  
    \item $P_{\text{эфф}}$ — эффективная мощность, Вт.  
\end{itemize}

\textbf{Формула расчёта эффективной мощности для загрузки игры:}

\begin{equation}
P_{\text{эфф}} = P_{\text{макс}} \cdot k \cdot \eta,
\end{equation}

где $k = 0{,}60$ — коэффициент при полной нагрузке.

\textbf{2 часть}

На основе программного кода предыдущего практикума текущего курса ООП выполнить наследование класса \texttt{GamingAcceleration} с использованием \texttt{super()} таким образом, чтобы через класс с множественным наследованием \texttt{GamingDevice} вывести на печать время запуска и загрузки игры для трёх типов устройств при следующих параметрах:

\begin{itemize}
    \item GameConsole — $C = 0{,}025$ Дж/В², $P = 200$ Вт;  
    \item Handheld — $C = 0{,}008$ Дж/В², $P = 15$ Вт;  
    \item VR\_Headset — $C = 0{,}012$ Дж/В², $P = 35$ Вт.  
\end{itemize}

\textbf{3 часть}

На основе программного кода и параметров устройств из 2 части текущего практикума отобразить через класс с множественным наследованием \texttt{GamingDevice} расчёт времени загрузки игры от уровня 0.4 до 1.0 через каждые 0.15, учитывая, что коэффициент мощности $k = 0{,}60$ увеличивается через каждые 0.15 на 0{,}06.

\textbf{Коэффициенты мощности для различных диапазонов активности:}
\begin{itemize}
    \item 0.40–0.55: $k = 0{,}60$;  
    \item 0.55–0.70: $k = 0{,}66$;  
    \item 0.70–0.85: $k = 0{,}72$;  
    \item 0.85–1.00: $k = 0{,}78$.  
\end{itemize}


\item[32] \textbf{1 часть}  
Написать класс Python \texttt{AudioAcceleration}, выполняющий расчёт времени запуска электронного музыкального инструмента от включения до готовности к исполнению и до полной настройки звука.

\textbf{Расчёт времени от включения до готовности:}

\begin{equation}
t = \frac{C \cdot (\Delta V)^2}{2P_{\text{эфф}}},
\end{equation}

где:  
\begin{itemize}
    \item $t$ — время запуска, с;  
    \item $C$ — эквивалентная ёмкость системы, Дж/В²;  
    \item $\Delta V = 0{,}3$;  
    \item $P_{\text{эфф}}$ — эффективная мощность, Вт.  
\end{itemize}

\textbf{Формула расчёта эффективной мощности для запуска:}

\begin{equation}
P_{\text{эфф}} = P_{\text{макс}} \cdot k \cdot \eta,
\end{equation}

где:  
\begin{itemize}
    \item $P_{\text{макс}}$ — номинальная мощность, Вт;  
    \item $k = 0{,}24$;  
    \item $\eta = 0{,}88$.  
\end{itemize}

\textbf{Расчёт времени от готовности до полной настройки звука:}

\begin{equation}
t = \frac{C \cdot (V_2^2 - V_1^2)}{2P_{\text{эфф}}},
\end{equation}

где:  
\begin{itemize}
    \item $V_1 = 0{,}3$, $V_2 = 1{,}0$;  
    \item $P_{\text{эфф}}$ — эффективная мощность, Вт.  
\end{itemize}

\textbf{Формула расчёта эффективной мощности для настройки звука:}

\begin{equation}
P_{\text{эфф}} = P_{\text{макс}} \cdot k \cdot \eta,
\end{equation}

где $k = 0{,}59$ — коэффициент при полной настройке.

\textbf{2 часть}

На основе программного кода предыдущего практикума текущего курса ООП выполнить наследование класса \texttt{AudioAcceleration} с использованием \texttt{super()} таким образом, чтобы через класс с множественным наследованием \texttt{ElectronicInstrument} вывести на печать время запуска и настройки для трёх типов инструментов при следующих параметрах:

\begin{itemize}
    \item ElectricGuitar — $C = 0{,}006$ Дж/В², $P = 50$ Вт;  
    \item Synthesizer — $C = 0{,}014$ Дж/В², $P = 80$ Вт;  
    \item DrumMachine — $C = 0{,}010$ Дж/В², $P = 60$ Вт.  
\end{itemize}

\textbf{3 часть}

На основе программного кода и параметров инструментов из 2 части текущего практикума отобразить через класс с множественным наследованием \texttt{ElectronicInstrument} расчёт времени настройки от уровня 0.3 до 1.0 через каждые 0.175, учитывая, что коэффициент мощности $k = 0{,}59$ увеличивается через каждые 0.175 на 0{,}07.

\textbf{Коэффициенты мощности для различных диапазонов активности:}
\begin{itemize}
    \item 0.30–0.475: $k = 0{,}59$;  
    \item 0.475–0.65: $k = 0{,}66$;  
    \item 0.65–0.825: $k = 0{,}73$;  
    \item 0.825–1.00: $k = 0{,}80$.  
\end{itemize}


\item[33] \textbf{1 часть}  
Написать класс Python \texttt{SmartHomeAcceleration}, выполняющий расчёт времени активации системы умного дома от включения до готовности и до полной автоматизации.

\textbf{Расчёт времени от включения до готовности:}

\begin{equation}
t = \frac{C \cdot (\Delta V)^2}{2P_{\text{эфф}}},
\end{equation}

где:  
\begin{itemize}
    \item $t$ — время активации, с;  
    \item $C$ — эквивалентная ёмкость системы, Дж/В²;  
    \item $\Delta V = 0{,}4$;  
    \item $P_{\text{эфф}}$ — эффективная мощность, Вт.  
\end{itemize}

\textbf{Формула расчёта эффективной мощности для активации:}

\begin{equation}
P_{\text{эфф}} = P_{\text{макс}} \cdot k \cdot \eta,
\end{equation}

где:  
\begin{itemize}
    \item $P_{\text{макс}}$ — пиковая мощность, Вт;  
    \item $k = 0{,}23$;  
    \item $\eta = 0{,}92$.  
\end{itemize}

\textbf{Расчёт времени от готовности до полной автоматизации:}

\begin{equation}
t = \frac{C \cdot (V_2^2 - V_1^2)}{2P_{\text{эфф}}},
\end{equation}

где:  
\begin{itemize}
    \item $V_1 = 0{,}4$, $V_2 = 1{,}0$;  
    \item $P_{\text{эфф}}$ — эффективная мощность, Вт.  
\end{itemize}

\textbf{Формула расчёта эффективной мощности для автоматизации:}

\begin{equation}
P_{\text{эфф}} = P_{\text{макс}} \cdot k \cdot \eta,
\end{equation}

где $k = 0{,}58$ — коэффициент при полной автоматизации.

\textbf{2 часть}

На основе программного кода предыдущего практикума текущего курса ООП выполнить наследование класса \texttt{SmartHomeAcceleration} с использованием \texttt{super()} таким образом, чтобы через класс с множественным наследованием \texttt{SmartHomeSystem} вывести на печать время активации и автоматизации для трёх типов устройств при следующих параметрах:

\begin{itemize}
    \item SmartThermostat — $C = 0{,}005$ Дж/В², $P = 8$ Вт;  
    \item SmartBlinds — $C = 0{,}007$ Дж/В², $P = 12$ Вт;  
    \item SmartSprinkler — $C = 0{,}006$ Дж/В², $P = 10$ Вт.  
\end{itemize}

\textbf{3 часть}

На основе программного кода и параметров устройств из 2 части текущего практикума отобразить через класс с множественным наследованием \texttt{SmartHomeSystem} расчёт времени автоматизации от уровня 0.4 до 1.0 через каждые 0.15, учитывая, что коэффициент мощности $k = 0{,}58$ увеличивается через каждые 0.15 на 0{,}06.

\textbf{Коэффициенты мощности для различных диапазонов активности:}
\begin{itemize}
    \item 0.40–0.55: $k = 0{,}58$;  
    \item 0.55–0.70: $k = 0{,}64$;  
    \item 0.70–0.85: $k = 0{,}70$;  
    \item 0.85–1.00: $k = 0{,}76$.  
\end{itemize}


\item[34] \textbf{1 часть}  
Написать класс Python \texttt{PersonalTransportAcceleration}, выполняющий расчёт времени готовности персонального электротранспорта к поездке и до достижения максимальной скорости.

\textbf{Расчёт времени от выключения до готовности:}

\begin{equation}
t = \frac{m \cdot (\Delta V)^2}{2P_{\text{эфф}}},
\end{equation}

где:  
\begin{itemize}
    \item $t$ — время готовности, с;  
    \item $m$ — масса транспорта с водителем, кг;  
    \item $\Delta V$ — изменение скорости от 0 до 5 км/ч (в м/с);  
    \item $P_{\text{эфф}}$ — эффективная мощность, Вт.  
\end{itemize}

\textbf{Формула расчёта эффективной мощности для готовности:}

\begin{equation}
P_{\text{эфф}} = P_{\text{макс}} \cdot k \cdot \eta,
\end{equation}

где:  
\begin{itemize}
    \item $P_{\text{макс}}$ — номинальная мощность мотора, Вт;  
    \item $k = 0{,}26$;  
    \item $\eta = 0{,}89$.  
\end{itemize}

\textbf{Расчёт времени от готовности до максимальной скорости:}

\begin{equation}
t = \frac{m \cdot (V_2^2 - V_1^2)}{2P_{\text{эфф}}},
\end{equation}

где:  
\begin{itemize}
    \item $V_1 = 5$ км/ч, $V_2 = 25$ км/ч (в м/с);  
    \item $P_{\text{эфф}}$ — эффективная мощность, Вт.  
\end{itemize}

\textbf{Формула расчёта эффективной мощности для максимальной скорости:}

\begin{equation}
P_{\text{эфф}} = P_{\text{макс}} \cdot k \cdot \eta,
\end{equation}

где $k = 0{,}61$ — коэффициент при высокой скорости.

\textbf{2 часть}

На основе программного кода предыдущего практикума текущего курса ООП выполнить наследование класса \texttt{PersonalTransportAcceleration} с использованием \texttt{super()} таким образом, чтобы через класс с множественным наследованием \texttt{PersonalTransport} вывести на печать время готовности и достижения максимальной скорости для трёх типов транспорта при следующих параметрах:

\begin{itemize}
    \item ElectricScooter — масса 80 кг, мощность 350 Вт;  
    \item Segway — масса 90 кг, мощность 600 Вт;  
    \item Hoverboard — масса 70 кг, мощность 700 Вт.  
\end{itemize}

\textbf{3 часть}

На основе программного кода и параметров транспорта из 2 части текущего практикума отобразить через класс с множественным наследованием \texttt{PersonalTransport} расчёт времени разгона от 5 до 25 км/ч через каждые 5 км/ч, учитывая, что коэффициент мощности $k = 0{,}61$ увеличивается через каждые 5 км/ч на 0{,}08.

\textbf{Коэффициенты мощности для различных диапазонов скоростей:}
\begin{itemize}
    \item 5–10 км/ч: $k = 0{,}61$;  
    \item 10–15 км/ч: $k = 0{,}69$;  
    \item 15–20 км/ч: $k = 0{,}77$;  
    \item 20–25 км/ч: $k = 0{,}85$.  
\end{itemize}


\item[35] \textbf{1 часть}  
Написать класс Python \texttt{BeverageAcceleration}, выполняющий расчёт времени готовности техники для напитков от включения до подачи напитка и до завершения цикла.

\textbf{Расчёт времени от включения до подачи напитка:}

\begin{equation}
t = \frac{C \cdot (\Delta T)^2}{2P_{\text{эфф}}},
\end{equation}

где:  
\begin{itemize}
    \item $t$ — время готовности, с;  
    \item $C$ — теплоёмкость системы, Дж/°C;  
    \item $\Delta T = 60$ °C (от 20 °C до 80 °C);  
    \item $P_{\text{эфф}}$ — эффективная мощность, Вт.  
\end{itemize}

\textbf{Формула расчёта эффективной мощности для подачи:}

\begin{equation}
P_{\text{эфф}} = P_{\text{макс}} \cdot k \cdot \eta,
\end{equation}

где:  
\begin{itemize}
    \item $P_{\text{макс}}$ — номинальная мощность, Вт;  
    \item $k = 0{,}27$;  
    \item $\eta = 0{,}87$.  
\end{itemize}

\textbf{Расчёт времени от подачи до завершения цикла:}

\begin{equation}
t = \frac{C \cdot (T_2^2 - T_1^2)}{2P_{\text{эфф}}},
\end{equation}

где:  
\begin{itemize}
    \item $T_1 = 80$, $T_2 = 95$ °C;  
    \item $P_{\text{эфф}}$ — эффективная мощность, Вт.  
\end{itemize}

\textbf{Формула расчёта эффективной мощности для завершения:}

\begin{equation}
P_{\text{эфф}} = P_{\text{макс}} \cdot k \cdot \eta,
\end{equation}

где $k = 0{,}62$ — коэффициент при полном цикле.

\textbf{2 часть}

На основе программного кода предыдущего практикума текущего курса ООП выполнить наследование класса \texttt{BeverageAcceleration} с использованием \texttt{super()} таким образом, чтобы через класс с множественным наследованием \texttt{BeverageAppliance} вывести на печать время подачи и завершения цикла для трёх типов приборов при следующих параметрах:

\begin{itemize}
    \item CoffeeMachine — $C = 300$ Дж/°C, $P = 1500$ Вт;  
    \item Teapot — $C = 250$ Дж/°C, $P = 2000$ Вт;  
    \item Juicer — $C = 180$ Дж/°C, $P = 1200$ Вт.  
\end{itemize}

\textbf{3 часть}

На основе программного кода и параметров приборов из 2 части текущего практикума отобразить через класс с множественным наследованием \texttt{BeverageAppliance} расчёт времени завершения цикла от 80 °C до 95 °C через каждые 3.75 °C, учитывая, что коэффициент мощности $k = 0{,}62$ увеличивается через каждые 3.75 °C на 0{,}08.

\textbf{Коэффициенты мощности для различных диапазонов температур:}
\begin{itemize}
    \item 80–83.75 °C: $k = 0{,}62$;  
    \item 83.75–87.5 °C: $k = 0{,}70$;  
    \item 87.5–91.25 °C: $k = 0{,}78$;  
    \item 91.25–95 °C: $k = 0{,}86$.  
\end{itemize}

\end{enumerate}
