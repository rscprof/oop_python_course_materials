\subsection{Семинар <<Разработка метакласса и классов для создания таблиц в 
базах данных объектным способом>>}

Необходимо разработать программу на основе метакласса, которая позволяет создавать 
модели (таблицы) данных, используя технологию ORM. 
ORM (Object-Relational Mapping, объектно-реляционное отображение) — это технология 
ООП программирования, которая позволяет разработчикам взаимодействовать 
с реляционными базами данных, работая с ними как с обычными 
объектами ООП в памяти программы, а не напрямую с SQL-запросами. 
Основная цель ORM — абстракция доступа к данным, что делает работу с 
базами данных проще и безопаснее. Примером реализации такой 
технологии являются классы \texttt{models.Model} веб-фреймворка Django.

\textbf{Важно:} В рамках данного практического задания 
\textbf{не предполагается подключение к реальной базе данных}. Программа должна 
генерировать и \textbf{выводить SQL-запросы в консоль}, демонстрируя корректность 
формирования DDL (Data Definition Language) и DML (Data Manipulation Language) команд.

Для реализации этой задачи необходимо создать 
метакласс \texttt{ModelBase}, который будет служить основой для создания 
классов моделей данных. Этот метакласс должен обеспечивать следующие возможности:

\begin{enumerate}
    \item \textbf{Сбор полей модели:} Метакласс должен уметь анализировать 
    пространство имён создаваемого класса и собирать все поля, 
    которые являются объектами класса \texttt{Field}. Собранные поля 
    должны сохраняться в специальном атрибуте \texttt{\_fields} нового 
    класса, а сами поля должны удаляться из исходного пространства 
    имён, чтобы избежать конфликтов.
    \item Необходимо реализовать классы-типы данных, которые будут представлять 
    различные типы полей в модели:
    \begin{itemize}
        \item \texttt{CharField}: Представляет строку символов с возможностью 
        указания максимальной длины и ограничений на допустимые значения.
        \item \texttt{IntegerField}: Представляет целое число с возможностью 
        задания ограничений на допустимые значения.
        \item \texttt{DateField}: Представляет дату с возможностью задания 
        ограничений на допустимые значения.
    \end{itemize}
    Эти классы должны наследоваться от общего класса \texttt{Field} и переопределять 
    метод \texttt{to\_sql()} для генерации соответствующей части SQL-запроса, 
    учитывая особенности каждого типа данных.
    \item \textbf{Создание класса:} Метакласс должен формировать новый класс 
    на основе собранной информации и возвращать его. Новый класс 
    должен содержать собранные поля в виде словаря \texttt{\_fields}.
    \item \textbf{Генерация SQL-запросов:} Класс модели должен предоставлять метод для 
    генерации SQL-запроса, который <<создаёт>> таблицу в базе данных на основе 
    собранных полей. Этот запрос должен учитывать типы данных и ограничения, 
    заданные в соответствующих полях.
\end{enumerate}

\subsubsection*{Алгоритм программирования задачи}

Алгоритм программирования задачи, позволяющей создавать классы для описания 
таблиц баз данных с помощью метакласса \texttt{ModelBase} и других классов, 
таких как \texttt{Field}, \texttt{CharField}, \texttt{IntegerField}, 
\texttt{DateField}, \texttt{Model}, следующий:

\begin{enumerate}
    \item \textbf{Определение класса \texttt{Field}}: Определяется базовый 
    класс \texttt{Field}, который будет представлять различные типы данных в модели.
    \begin{enumerate}
        \item Импортируйте модуль \texttt{datetime}.
        \item Определите класс \texttt{Field}.
        \item В конструкторе \texttt{\_\_init\_\_} примите 
        аргументы: \texttt{field\_type}, \texttt{max\_length=None}, \texttt{null=False}, \texttt{default=None}.
        \item Сохраните их как атрибуты экземпляра.
        \item Добавьте метод \texttt{to\_sql(self, name)}, который генерирует 
        SQL-представление столбца:
            \begin{itemize}
                \item Сопоставьте Python-типы с SQL-типами 
                (например, \texttt{str} → \texttt{VARCHAR}, 
                \texttt{int} → \texttt{INTEGER}, 
                \texttt{datetime.date} → \texttt{DATE}).
                \item Если \texttt{max\_length} задано и тип — строковый, 
                используйте \texttt{VARCHAR(max\_length)}.
                \item Добавьте \texttt{NOT NULL}, если \texttt{null=False}.
                \item Укажите \texttt{DEFAULT} в соответствии со значением 
                \texttt{default}.\\
                \textbf{Важно:} На данном этапе \texttt{default} используется 
                только для генерации DDL. Для вызываемых значений 
                (callable defaults, например, \texttt{datetime.date.today}) в DDL 
                следует подставлять конкретное значение, полученное при 
                создании модели (см. ниже).
            \end{itemize}
    \end{enumerate}
    
    \item \textbf{Реализация классов-типов данных}:
    \begin{enumerate}
        \item \texttt{CharField}:  
        \begin{itemize}
            \item В \texttt{\_\_init\_\_} установите: \texttt{max\_length=255} 
            (по умолчанию), \texttt{null=False}, \texttt{default=''}.
            \item Вызовите \texttt{super().\_\_init\_\_(str, max\_length, null, default)}.
        \end{itemize}
        \item \texttt{IntegerField}:  
        \begin{itemize}
            \item В \texttt{\_\_init\_\_} установите: \texttt{null=False}, \texttt{default=0}.
            \item Вызовите \texttt{super().\_\_init\_\_(int, None, null, default)}.
        \end{itemize}
        \item \texttt{DateField}:  
        \begin{itemize}
            \item В \texttt{\_\_init\_\_} установите: \texttt{null=False}, 
            \texttt{default=datetime.date.today} \textbf{(без скобок!)}.
            \item Вызовите \texttt{super().\_\_init\_\_(datetime.date, None, null, default)}.
            \item \textbf{Пояснение:} Передача \texttt{datetime.date.today} как callable 
            позволяет вычислять «сегодняшнюю дату» на момент создания 
            объекта модели, а не на момент определения класса.
        \end{itemize}
    \end{enumerate}
    
    \item \textbf{Создание метакласса \texttt{ModelBase}}:
    \begin{enumerate}
        \item Определите \texttt{class ModelBase(type)}.
        \item Реализуйте \texttt{\_\_new\_\_(cls, name, bases, attrs)}:
            \begin{itemize}
                \item Создайте пустой словарь \texttt{fields}.
                \item Пройдитесь по \texttt{attrs}, соберите все значения, 
                являющиеся экземплярами \texttt{Field}, в \texttt{fields}, и удалите 
                их из \texttt{attrs}.
                \item Вызовите \texttt{super().\_\_new\_\_(cls, name, bases, attrs)}.
                \item Присвойте \texttt{new\_class.\_fields = fields}.
                \item Верните \texttt{new\_class}.
            \end{itemize}
        \end{enumerate}
        
    \item \textbf{Создание базового класса \texttt{Model}}:
    \begin{enumerate}
        \item Определите \texttt{class Model(metaclass=ModelBase)}.
        \item В \texttt{\_\_init\_\_(self, **kwargs)}:
            \begin{itemize}
                \item Для каждого поля из \texttt{self.\_fields}:
                    \begin{itemize}
                        \item Если имя поля есть в \texttt{kwargs}, используйте 
                        переданное значение.
                        \item Иначе, если у поля есть \texttt{default}:
                            \begin{itemize}
                                \item Если \texttt{default} — callable (например, 
                                \texttt{datetime.date.today}), вызовите его.
                                \item Иначе — используйте значение напрямую.
                            \end{itemize}
                        \item Установите значение через \texttt{setattr(self, name, value)}.
                    \end{itemize}
            \end{itemize}
        \item Реализуйте метод 
        класса \texttt{@classmethod def create\_table(cls, table\_name)}:
            \begin{itemize}
                \item Соберите SQL-представление каждого поля: \texttt{field.to\_sql(name)}.
                \item Сформируйте команду \texttt{CREATE TABLE IF NOT EXISTS \{table\_name\} (...)}.
                \item \textbf{Выведите её на печать.}
            \end{itemize}
        \item Реализуйте метод \texttt{save(self)}:
            \begin{itemize}
                \item Получите значения всех полей 
                из \texttt{self.\_fields} через \texttt{getattr(self, name)}.
                \item \textbf{Важно:} Для безопасной работы с данными 
                в реальных ORM используются \textbf{параметризованные запросы}, чтобы 
                предотвратить SQL-инъекции. В данном учебном задании, 
                поскольку мы только выводим SQL, \textbf{значения должны 
                быть корректно экранированы или преобразованы в SQL-литералы} (например, 
                строки — в кавычки, числа — без кавычек, даты — в формате 'ГГГГ-ММ-ДД'; 
                не забудьте про то, что внутри строки могут быть кавычки, переносы строк и 
                другие особенности). 
                \item Сформируйте \texttt{INSERT INTO ... VALUES (...)}.
                \item Выведите запрос на печать.
            \end{itemize}
    \end{enumerate}
\end{enumerate}

\subsubsection{Тестирование созданной программы (по вариантам)}

\begin{enumerate}
    \item[1]


    \begin{enumerate}
        \item Определите класс \texttt{Book}:
        \begin{verbatim}
class Book(Model):
    title = CharField(max_length=255)
    author = CharField(max_length=100)
    published_date = DateField()
    year = IntegerField()
        \end{verbatim}
        \item Вызовите: \texttt{Book.create\_table('books')}
        \item Создайте экземпляр:
        \begin{verbatim}
book = Book(
    title='Python Cookbook',
    author='David Beazley',
    published_date=datetime.date(2013, 5, 10),
    year=2012,
)
book.save()
        \end{verbatim}
    \end{enumerate}
    Программа должна вывести следующие SQL-запросы:
    
    \begin{enumerate}
        \item Запрос на создание таблицы (пример для даты выполнения 2025-04-05):
        \begin{verbatim}
CREATE TABLE IF NOT EXISTS books (
    title VARCHAR(255) NOT NULL DEFAULT '',
    author VARCHAR(100) NOT NULL DEFAULT '',
    published_date DATE NOT NULL DEFAULT '2025-04-05',
    year INTEGER NOT NULL DEFAULT 0
);
        \end{verbatim}
        
        \item Запрос на вставку данных:
        \begin{verbatim}
INSERT INTO books (title, author, published_date, year) VALUES
('Python Cookbook', 'David Beazley', '2013-05-10', 2012);
        \end{verbatim}
    \end{enumerate}
    \item[2]
    \begin{enumerate}
        \item Определите класс \texttt{Student}:
        \begin{verbatim}
class Student(Model):
    name = CharField(max_length=100)
    surname = CharField(max_length=100)
    birth_date = DateField()
    course = IntegerField()
        \end{verbatim}
        \item Вызовите: \texttt{Student.create\_table('students')}
        \item Создайте экземпляр:
        \begin{verbatim}
student = Student(
    name='Иван',
    surname='Петров',
    birth_date=datetime.date(2002, 8, 15),
    course=3,
)
student.save()
        \end{verbatim}
    \end{enumerate}
    Программа должна вывести следующие SQL-запросы:
    
    \begin{enumerate}
        \item Запрос на создание таблицы (пример для даты выполнения 2025-04-05):
        \begin{verbatim}
CREATE TABLE IF NOT EXISTS students (
    name VARCHAR(100) NOT NULL DEFAULT '',
    surname VARCHAR(100) NOT NULL DEFAULT '',
    birth_date DATE NOT NULL DEFAULT '2025-04-05',
    course INTEGER NOT NULL DEFAULT 0
);
        \end{verbatim}
        
        \item Запрос на вставку данных:
        \begin{verbatim}
INSERT INTO students (name, surname, birth_date, course) VALUES
('Иван', 'Петров', '2002-08-15', 3);
        \end{verbatim}
    \end{enumerate}

    \item[3]
    \begin{enumerate}
        \item Определите класс \texttt{Product}:
        \begin{verbatim}
class Product(Model):
    name = CharField(max_length=200)
    brand = CharField(max_length=100)
    release_date = DateField()
    price_cents = IntegerField()
        \end{verbatim}
        \item Вызовите: \texttt{Product.create\_table('products')}
        \item Создайте экземпляр:
        \begin{verbatim}
product = Product(
    name='Wireless Headphones',
    brand='SoundMax',
    release_date=datetime.date(2023, 11, 1),
    price_cents=5999,
)
product.save()
        \end{verbatim}
    \end{enumerate}
    Программа должна вывести следующие SQL-запросы:
    
    \begin{enumerate}
        \item Запрос на создание таблицы (пример для даты выполнения 2025-04-05):
        \begin{verbatim}
CREATE TABLE IF NOT EXISTS products (
    name VARCHAR(200) NOT NULL DEFAULT '',
    brand VARCHAR(100) NOT NULL DEFAULT '',
    release_date DATE NOT NULL DEFAULT '2025-04-05',
    price_cents INTEGER NOT NULL DEFAULT 0
);
        \end{verbatim}
        
        \item Запрос на вставку данных:
        \begin{verbatim}
INSERT INTO products (name, brand, release_date, price_cents) VALUES
('Wireless Headphones', 'SoundMax', '2023-11-01', 5999);
        \end{verbatim}
    \end{enumerate}

    \item[4]
    \begin{enumerate}
        \item Определите класс \texttt{Employee}:
        \begin{verbatim}
class Employee(Model):
    first_name = CharField(max_length=50)
    last_name = CharField(max_length=50)
    hire_date = DateField()
    department_id = IntegerField()
        \end{verbatim}
        \item Вызовите: \texttt{Employee.create\_table('employees')}
        \item Создайте экземпляр:
        \begin{verbatim}
employee = Employee(
    first_name='Анна',
    last_name='Смирнова',
    hire_date=datetime.date(2021, 3, 12),
    department_id=7,
)
employee.save()
        \end{verbatim}
    \end{enumerate}
    Программа должна вывести следующие SQL-запросы:
    
    \begin{enumerate}
        \item Запрос на создание таблицы (пример для даты выполнения 2025-04-05):
        \begin{verbatim}
CREATE TABLE IF NOT EXISTS employees (
    first_name VARCHAR(50) NOT NULL DEFAULT '',
    last_name VARCHAR(50) NOT NULL DEFAULT '',
    hire_date DATE NOT NULL DEFAULT '2025-04-05',
    department_id INTEGER NOT NULL DEFAULT 0
);
        \end{verbatim}
        
        \item Запрос на вставку данных:
        \begin{verbatim}
INSERT INTO employees (first_name, last_name, hire_date, department_id) VALUES
('Анна', 'Смирнова', '2021-03-12', 7);
        \end{verbatim}
    \end{enumerate}

    \item[5]
    \begin{enumerate}
        \item Определите класс \texttt{Movie}:
        \begin{verbatim}
class Movie(Model):
    title = CharField(max_length=150)
    director = CharField(max_length=100)
    release_date = DateField()
    duration_minutes = IntegerField()
        \end{verbatim}
        \item Вызовите: \texttt{Movie.create\_table('movies')}
        \item Создайте экземпляр:
        \begin{verbatim}
movie = Movie(
    title='Inception',
    director='Christopher Nolan',
    release_date=datetime.date(2010, 7, 16),
    duration_minutes=148,
)
movie.save()
        \end{verbatim}
    \end{enumerate}
    Программа должна вывести следующие SQL-запросы:
    
    \begin{enumerate}
        \item Запрос на создание таблицы (пример для даты выполнения 2025-04-05):
        \begin{verbatim}
CREATE TABLE IF NOT EXISTS movies (
    title VARCHAR(150) NOT NULL DEFAULT '',
    director VARCHAR(100) NOT NULL DEFAULT '',
    release_date DATE NOT NULL DEFAULT '2025-04-05',
    duration_minutes INTEGER NOT NULL DEFAULT 0
);
        \end{verbatim}
        
        \item Запрос на вставку данных:
        \begin{verbatim}
INSERT INTO movies (title, director, release_date, duration_minutes) VALUES
('Inception', 'Christopher Nolan', '2010-07-16', 148);
        \end{verbatim}
    \end{enumerate}

    \item[6]
    \begin{enumerate}
        \item Определите класс \texttt{Car}:
        \begin{verbatim}
class Car(Model):
    make = CharField(max_length=50)
    model = CharField(max_length=50)
    production_date = DateField()
    horsepower = IntegerField()
        \end{verbatim}
        \item Вызовите: \texttt{Car.create\_table('cars')}
        \item Создайте экземпляр:
        \begin{verbatim}
car = Car(
    make='Toyota',
    model='Corolla',
    production_date=datetime.date(2022, 9, 5),
    horsepower=139,
)
car.save()
        \end{verbatim}
    \end{enumerate}
    Программа должна вывести следующие SQL-запросы:
    
    \begin{enumerate}
        \item Запрос на создание таблицы (пример для даты выполнения 2025-04-05):
        \begin{verbatim}
CREATE TABLE IF NOT EXISTS cars (
    make VARCHAR(50) NOT NULL DEFAULT '',
    model VARCHAR(50) NOT NULL DEFAULT '',
    production_date DATE NOT NULL DEFAULT '2025-04-05',
    horsepower INTEGER NOT NULL DEFAULT 0
);
        \end{verbatim}
        
        \item Запрос на вставку данных:
        \begin{verbatim}
INSERT INTO cars (make, model, production_date, horsepower) VALUES
('Toyota', 'Corolla', '2022-09-05', 139);
        \end{verbatim}
    \end{enumerate}

    \item[7]
    \begin{enumerate}
        \item Определите класс \texttt{Task}:
        \begin{verbatim}
class Task(Model):
    title = CharField(max_length=200)
    assignee = CharField(max_length=100)
    due_date = DateField()
    priority = IntegerField()
        \end{verbatim}
        \item Вызовите: \texttt{Task.create\_table('tasks')}
        \item Создайте экземпляр:
        \begin{verbatim}
task = Task(
    title='Refactor ORM module',
    assignee='developer@example.com',
    due_date=datetime.date(2025, 4, 20),
    priority=1,
)
task.save()
        \end{verbatim}
    \end{enumerate}
    Программа должна вывести следующие SQL-запросы:
    
    \begin{enumerate}
        \item Запрос на создание таблицы (пример для даты выполнения 2025-04-05):
        \begin{verbatim}
CREATE TABLE IF NOT EXISTS tasks (
    title VARCHAR(200) NOT NULL DEFAULT '',
    assignee VARCHAR(100) NOT NULL DEFAULT '',
    due_date DATE NOT NULL DEFAULT '2025-04-05',
    priority INTEGER NOT NULL DEFAULT 0
);
        \end{verbatim}
        
        \item Запрос на вставку данных:
        \begin{verbatim}
INSERT INTO tasks (title, assignee, due_date, priority) VALUES
('Refactor ORM module', 'developer@example.com', '2025-04-20', 1);
        \end{verbatim}
    \end{enumerate}

    \item[8]
    \begin{enumerate}
        \item Определите класс \texttt{Course}:
        \begin{verbatim}
class Course(Model):
    name = CharField(max_length=255)
    instructor = CharField(max_length=100)
    start_date = DateField()
    credits = IntegerField()
        \end{verbatim}
        \item Вызовите: \texttt{Course.create\_table('courses')}
        \item Создайте экземпляр:
        \begin{verbatim}
course = Course(
    name='Advanced Python Programming',
    instructor='Dr. Elena Volkova',
    start_date=datetime.date(2025, 2, 10),
    credits=6,
)
course.save()
        \end{verbatim}
    \end{enumerate}
    Программа должна вывести следующие SQL-запросы:
    
    \begin{enumerate}
        \item Запрос на создание таблицы (пример для даты выполнения 2025-04-05):
        \begin{verbatim}
CREATE TABLE IF NOT EXISTS courses (
    name VARCHAR(255) NOT NULL DEFAULT '',
    instructor VARCHAR(100) NOT NULL DEFAULT '',
    start_date DATE NOT NULL DEFAULT '2025-04-05',
    credits INTEGER NOT NULL DEFAULT 0
);
        \end{verbatim}
        
        \item Запрос на вставку данных:
        \begin{verbatim}
INSERT INTO courses (name, instructor, start_date, credits) VALUES
('Advanced Python Programming', 'Dr. Elena Volkova', '2025-02-10', 6);
        \end{verbatim}
    \end{enumerate}

    \item[9]
    \begin{enumerate}
        \item Определите класс \texttt{Order}:
        \begin{verbatim}
class Order(Model):
    customer_name = CharField(max_length=150)
    product = CharField(max_length=200)
    order_date = DateField()
    quantity = IntegerField()
        \end{verbatim}
        \item Вызовите: \texttt{Order.create\_table('orders')}
        \item Создайте экземпляр:
        \begin{verbatim}
order = Order(
    customer_name='John Doe',
    product='Laptop Stand',
    order_date=datetime.date(2025, 3, 22),
    quantity=2,
)
order.save()
        \end{verbatim}
    \end{enumerate}
    Программа должна вывести следующие SQL-запросы:
    
    \begin{enumerate}
        \item Запрос на создание таблицы (пример для даты выполнения 2025-04-05):
        \begin{verbatim}
CREATE TABLE IF NOT EXISTS orders (
    customer_name VARCHAR(150) NOT NULL DEFAULT '',
    product VARCHAR(200) NOT NULL DEFAULT '',
    order_date DATE NOT NULL DEFAULT '2025-04-05',
    quantity INTEGER NOT NULL DEFAULT 0
);
        \end{verbatim}
        
        \item Запрос на вставку данных:
        \begin{verbatim}
INSERT INTO orders (customer_name, product, order_date, quantity) VALUES
('John Doe', 'Laptop Stand', '2025-03-22', 2);
        \end{verbatim}
    \end{enumerate}

    \item[10]
    \begin{enumerate}
        \item Определите класс \texttt{Event}:
        \begin{verbatim}
class Event(Model):
    title = CharField(max_length=300)
    location = CharField(max_length=200)
    event_date = DateField()
    attendees = IntegerField()
        \end{verbatim}
        \item Вызовите: \texttt{Event.create\_table('events')}
        \item Создайте экземпляр:
        \begin{verbatim}
event = Event(
    title='Tech Conference 2025',
    location='Moscow Expo Center',
    event_date=datetime.date(2025, 6, 15),
    attendees=350,
)
event.save()
        \end{verbatim}
    \end{enumerate}
    Программа должна вывести следующие SQL-запросы:
    
    \begin{enumerate}
        \item Запрос на создание таблицы (пример для даты выполнения 2025-04-05):
        \begin{verbatim}
CREATE TABLE IF NOT EXISTS events (
    title VARCHAR(300) NOT NULL DEFAULT '',
    location VARCHAR(200) NOT NULL DEFAULT '',
    event_date DATE NOT NULL DEFAULT '2025-04-05',
    attendees INTEGER NOT NULL DEFAULT 0
);
        \end{verbatim}
        
        \item Запрос на вставку данных:
        \begin{verbatim}
INSERT INTO events (title, location, event_date, attendees) VALUES
('Tech Conference 2025', 'Moscow Expo Center', '2025-06-15', 350);
        \end{verbatim}
    \end{enumerate}

    \item[11]
    \begin{enumerate}
        \item Определите класс \texttt{Animal}:
        \begin{verbatim}
class Animal(Model):
    species = CharField(max_length=100)
    name = CharField(max_length=50)
    birth_date = DateField()
    weight_kg = IntegerField()
        \end{verbatim}
        \item Вызовите: \texttt{Animal.create\_table('animals')}
        \item Создайте экземпляр:
        \begin{verbatim}
animal = Animal(
    species='Dog',
    name='Buddy',
    birth_date=datetime.date(2020, 11, 3),
    weight_kg=25,
)
animal.save()
        \end{verbatim}
    \end{enumerate}
    Программа должна вывести следующие SQL-запросы:
    
    \begin{enumerate}
        \item Запрос на создание таблицы (пример для даты выполнения 2025-04-05):
        \begin{verbatim}
CREATE TABLE IF NOT EXISTS animals (
    species VARCHAR(100) NOT NULL DEFAULT '',
    name VARCHAR(50) NOT NULL DEFAULT '',
    birth_date DATE NOT NULL DEFAULT '2025-04-05',
    weight_kg INTEGER NOT NULL DEFAULT 0
);
        \end{verbatim}
        
        \item Запрос на вставку данных:
        \begin{verbatim}
INSERT INTO animals (species, name, birth_date, weight_kg) VALUES
('Dog', 'Buddy', '2020-11-03', 25);
        \end{verbatim}
    \end{enumerate}

    \item[12]
    \begin{enumerate}
        \item Определите класс \texttt{Recipe}:
        \begin{verbatim}
class Recipe(Model):
    dish_name = CharField(max_length=200)
    chef = CharField(max_length=100)
    created_date = DateField()
    difficulty = IntegerField()
        \end{verbatim}
        \item Вызовите: \texttt{Recipe.create\_table('recipes')}
        \item Создайте экземпляр:
        \begin{verbatim}
recipe = Recipe(
    dish_name='Beef Stroganoff',
    chef='Maria Ivanova',
    created_date=datetime.date(2023, 12, 1),
    difficulty=3,
)
recipe.save()
        \end{verbatim}
    \end{enumerate}
    Программа должна вывести следующие SQL-запросы:
    
    \begin{enumerate}
        \item Запрос на создание таблицы (пример для даты выполнения 2025-04-05):
        \begin{verbatim}
CREATE TABLE IF NOT EXISTS recipes (
    dish_name VARCHAR(200) NOT NULL DEFAULT '',
    chef VARCHAR(100) NOT NULL DEFAULT '',
    created_date DATE NOT NULL DEFAULT '2025-04-05',
    difficulty INTEGER NOT NULL DEFAULT 0
);
        \end{verbatim}
        
        \item Запрос на вставку данных:
        \begin{verbatim}
INSERT INTO recipes (dish_name, chef, created_date, difficulty) VALUES
('Beef Stroganoff', 'Maria Ivanova', '2023-12-01', 3);
        \end{verbatim}
    \end{enumerate}

    \item[13]
    \begin{enumerate}
        \item Определите класс \texttt{Flight}:
        \begin{verbatim}
class Flight(Model):
    airline = CharField(max_length=100)
    flight_number = CharField(max_length=10)
    departure_date = DateField()
    seats = IntegerField()
        \end{verbatim}
        \item Вызовите: \texttt{Flight.create\_table('flights')}
        \item Создайте экземпляр:
        \begin{verbatim}
flight = Flight(
    airline='Aeroflot',
    flight_number='SU1234',
    departure_date=datetime.date(2025, 5, 8),
    seats=180,
)
flight.save()
        \end{verbatim}
    \end{enumerate}
    Программа должна вывести следующие SQL-запросы:
    
    \begin{enumerate}
        \item Запрос на создание таблицы (пример для даты выполнения 2025-04-05):
        \begin{verbatim}
CREATE TABLE IF NOT EXISTS flights (
    airline VARCHAR(100) NOT NULL DEFAULT '',
    flight_number VARCHAR(10) NOT NULL DEFAULT '',
    departure_date DATE NOT NULL DEFAULT '2025-04-05',
    seats INTEGER NOT NULL DEFAULT 0
);
        \end{verbatim}
        
        \item Запрос на вставку данных:
        \begin{verbatim}
INSERT INTO flights (airline, flight_number, departure_date, seats) VALUES
('Aeroflot', 'SU1234', '2025-05-08', 180);
        \end{verbatim}
    \end{enumerate}

    \item[14]
    \begin{enumerate}
        \item Определите класс \texttt{User}:
        \begin{verbatim}
class User(Model):
    username = CharField(max_length=50)
    email = CharField(max_length=150)
    registration_date = DateField()
    score = IntegerField()
        \end{verbatim}
        \item Вызовите: \texttt{User.create\_table('users')}
        \item Создайте экземпляр:
        \begin{verbatim}
user = User(
    username='coder42',
    email='coder42@example.com',
    registration_date=datetime.date(2024, 1, 10),
    score=1250,
)
user.save()
        \end{verbatim}
    \end{enumerate}
    Программа должна вывести следующие SQL-запросы:
    
    \begin{enumerate}
        \item Запрос на создание таблицы (пример для даты выполнения 2025-04-05):
        \begin{verbatim}
CREATE TABLE IF NOT EXISTS users (
    username VARCHAR(50) NOT NULL DEFAULT '',
    email VARCHAR(150) NOT NULL DEFAULT '',
    registration_date DATE NOT NULL DEFAULT '2025-04-05',
    score INTEGER NOT NULL DEFAULT 0
);
        \end{verbatim}
        
        \item Запрос на вставку данных:
        \begin{verbatim}
INSERT INTO users (username, email, registration_date, score) VALUES
('coder42', 'coder42@example.com', '2024-01-10', 1250);
        \end{verbatim}
    \end{enumerate}

    \item[15]
    \begin{enumerate}
        \item Определите класс \texttt{Project}:
        \begin{verbatim}
class Project(Model):
    name = CharField(max_length=200)
    manager = CharField(max_length=100)
    start_date = DateField()
    budget_thousands = IntegerField()
        \end{verbatim}
        \item Вызовите: \texttt{Project.create\_table('projects')}
        \item Создайте экземпляр:
        \begin{verbatim}
project = Project(
    name='ORM Framework Development',
    manager='Alex Johnson',
    start_date=datetime.date(2025, 1, 15),
    budget_thousands=250,
)
project.save()
        \end{verbatim}
    \end{enumerate}
    Программа должна вывести следующие SQL-запросы:
    
    \begin{enumerate}
        \item Запрос на создание таблицы (пример для даты выполнения 2025-04-05):
        \begin{verbatim}
CREATE TABLE IF NOT EXISTS projects (
    name VARCHAR(200) NOT NULL DEFAULT '',
    manager VARCHAR(100) NOT NULL DEFAULT '',
    start_date DATE NOT NULL DEFAULT '2025-04-05',
    budget_thousands INTEGER NOT NULL DEFAULT 0
);
        \end{verbatim}
        
        \item Запрос на вставку данных:
        \begin{verbatim}
INSERT INTO projects (name, manager, start_date, budget_thousands) VALUES
('ORM Framework Development', 'Alex Johnson', '2025-01-15', 250);
        \end{verbatim}
    \end{enumerate}

    \item[16]
    \begin{enumerate}
        \item Определите класс \texttt{Invoice}:
        \begin{verbatim}
class Invoice(Model):
    client = CharField(max_length=150)
    description = CharField(max_length=300)
    issue_date = DateField()
    amount = IntegerField()
        \end{verbatim}
        \item Вызовите: \texttt{Invoice.create\_table('invoices')}
        \item Создайте экземпляр:
        \begin{verbatim}
invoice = Invoice(
    client='ABC Corp',
    description='Software License',
    issue_date=datetime.date(2025, 3, 1),
    amount=4500,
)
invoice.save()
        \end{verbatim}
    \end{enumerate}
    Программа должна вывести следующие SQL-запросы:
    
    \begin{enumerate}
        \item Запрос на создание таблицы (пример для даты выполнения 2025-04-05):
        \begin{verbatim}
CREATE TABLE IF NOT EXISTS invoices (
    client VARCHAR(150) NOT NULL DEFAULT '',
    description VARCHAR(300) NOT NULL DEFAULT '',
    issue_date DATE NOT NULL DEFAULT '2025-04-05',
    amount INTEGER NOT NULL DEFAULT 0
);
        \end{verbatim}
        
        \item Запрос на вставку данных:
        \begin{verbatim}
INSERT INTO invoices (client, description, issue_date, amount) VALUES
('ABC Corp', 'Software License', '2025-03-01', 4500);
        \end{verbatim}
    \end{enumerate}

    \item[17]
    \begin{enumerate}
        \item Определите класс \texttt{License}:
        \begin{verbatim}
class License(Model):
    software = CharField(max_length=100)
    owner = CharField(max_length=150)
    valid_from = DateField()
    duration_days = IntegerField()
        \end{verbatim}
        \item Вызовите: \texttt{License.create\_table('licenses')}
        \item Создайте экземпляр:
        \begin{verbatim}
license = License(
    software='TextEditor Pro',
    owner='Jane Smith',
    valid_from=datetime.date(2025, 2, 20),
    duration_days=365,
)
license.save()
        \end{verbatim}
    \end{enumerate}
    Программа должна вывести следующие SQL-запросы:
    
    \begin{enumerate}
        \item Запрос на создание таблицы (пример для даты выполнения 2025-04-05):
        \begin{verbatim}
CREATE TABLE IF NOT EXISTS licenses (
    software VARCHAR(100) NOT NULL DEFAULT '',
    owner VARCHAR(150) NOT NULL DEFAULT '',
    valid_from DATE NOT NULL DEFAULT '2025-04-05',
    duration_days INTEGER NOT NULL DEFAULT 0
);
        \end{verbatim}
        
        \item Запрос на вставку данных:
        \begin{verbatim}
INSERT INTO licenses (software, owner, valid_from, duration_days) VALUES
('TextEditor Pro', 'Jane Smith', '2025-02-20', 365);
        \end{verbatim}
    \end{enumerate}

    \item[18]
    \begin{enumerate}
        \item Определите класс \texttt{Sensor}:
        \begin{verbatim}
class Sensor(Model):
    model = CharField(max_length=50)
    location = CharField(max_length=200)
    install_date = DateField()
    reading_interval_sec = IntegerField()
        \end{verbatim}
        \item Вызовите: \texttt{Sensor.create\_table('sensors')}
        \item Создайте экземпляр:
        \begin{verbatim}
sensor = Sensor(
    model='TempX200',
    location='Warehouse A',
    install_date=datetime.date(2024, 10, 5),
    reading_interval_sec=30,
)
sensor.save()
        \end{verbatim}
    \end{enumerate}
    Программа должна вывести следующие SQL-запросы:
    
    \begin{enumerate}
        \item Запрос на создание таблицы (пример для даты выполнения 2025-04-05):
        \begin{verbatim}
CREATE TABLE IF NOT EXISTS sensors (
    model VARCHAR(50) NOT NULL DEFAULT '',
    location VARCHAR(200) NOT NULL DEFAULT '',
    install_date DATE NOT NULL DEFAULT '2025-04-05',
    reading_interval_sec INTEGER NOT NULL DEFAULT 0
);
        \end{verbatim}
        
        \item Запрос на вставку данных:
        \begin{verbatim}
INSERT INTO sensors (model, location, install_date, reading_interval_sec) VALUES
('TempX200', 'Warehouse A', '2024-10-05', 30);
        \end{verbatim}
    \end{enumerate}

    \item[19]
    \begin{enumerate}
        \item Определите класс \texttt{Membership}:
        \begin{verbatim}
class Membership(Model):
    member_name = CharField(max_length=100)
    club = CharField(max_length=150)
    join_date = DateField()
    fee_paid = IntegerField()
        \end{verbatim}
        \item Вызовите: \texttt{Membership.create\_table('memberships')}
        \item Создайте экземпляр:
        \begin{verbatim}
membership = Membership(
    member_name='Olga Petrova',
    club='Photography Club',
    join_date=datetime.date(2023, 9, 12),
    fee_paid=1200,
)
membership.save()
        \end{verbatim}
    \end{enumerate}
    Программа должна вывести следующие SQL-запросы:
    
    \begin{enumerate}
        \item Запрос на создание таблицы (пример для даты выполнения 2025-04-05):
        \begin{verbatim}
CREATE TABLE IF NOT EXISTS memberships (
    member_name VARCHAR(100) NOT NULL DEFAULT '',
    club VARCHAR(150) NOT NULL DEFAULT '',
    join_date DATE NOT NULL DEFAULT '2025-04-05',
    fee_paid INTEGER NOT NULL DEFAULT 0
);
        \end{verbatim}
        
        \item Запрос на вставку данных:
        \begin{verbatim}
INSERT INTO memberships (member_name, club, join_date, fee_paid) VALUES
('Olga Petrova', 'Photography Club', '2023-09-12', 1200);
        \end{verbatim}
    \end{enumerate}

    \item[20]
    \begin{enumerate}
        \item Определите класс \texttt{Document}:
        \begin{verbatim}
class Document(Model):
    title = CharField(max_length=255)
    author = CharField(max_length=100)
    created_date = DateField()
    pages = IntegerField()
        \end{verbatim}
        \item Вызовите: \texttt{Document.create\_table('documents')}
        \item Создайте экземпляр:
        \begin{verbatim}
doc = Document(
    title='Annual Report 2024',
    author='Finance Team',
    created_date=datetime.date(2025, 1, 25),
    pages=42,
)
doc.save()
        \end{verbatim}
    \end{enumerate}
    Программа должна вывести следующие SQL-запросы:
    
    \begin{enumerate}
        \item Запрос на создание таблицы (пример для даты выполнения 2025-04-05):
        \begin{verbatim}
CREATE TABLE IF NOT EXISTS documents (
    title VARCHAR(255) NOT NULL DEFAULT '',
    author VARCHAR(100) NOT NULL DEFAULT '',
    created_date DATE NOT NULL DEFAULT '2025-04-05',
    pages INTEGER NOT NULL DEFAULT 0
);
        \end{verbatim}
        
        \item Запрос на вставку данных:
        \begin{verbatim}
INSERT INTO documents (title, author, created_date, pages) VALUES
('Annual Report 2024', 'Finance Team', '2025-01-25', 42);
        \end{verbatim}
    \end{enumerate}

    \item[21]
    \begin{enumerate}
        \item Определите класс \texttt{Exam}:
        \begin{verbatim}
class Exam(Model):
    subject = CharField(max_length=100)
    teacher = CharField(max_length=100)
    exam_date = DateField()
    max_score = IntegerField()
        \end{verbatim}
        \item Вызовите: \texttt{Exam.create\_table('exams')}
        \item Создайте экземпляр:
        \begin{verbatim}
exam = Exam(
    subject='Mathematics',
    teacher='Prof. Ivanov',
    exam_date=datetime.date(2025, 5, 20),
    max_score=100,
)
exam.save()
        \end{verbatim}
    \end{enumerate}
    Программа должна вывести следующие SQL-запросы:
    
    \begin{enumerate}
        \item Запрос на создание таблицы (пример для даты выполнения 2025-04-05):
        \begin{verbatim}
CREATE TABLE IF NOT EXISTS exams (
    subject VARCHAR(100) NOT NULL DEFAULT '',
    teacher VARCHAR(100) NOT NULL DEFAULT '',
    exam_date DATE NOT NULL DEFAULT '2025-04-05',
    max_score INTEGER NOT NULL DEFAULT 0
);
        \end{verbatim}
        
        \item Запрос на вставку данных:
        \begin{verbatim}
INSERT INTO exams (subject, teacher, exam_date, max_score) VALUES
('Mathematics', 'Prof. Ivanov', '2025-05-20', 100);
        \end{verbatim}
    \end{enumerate}

    \item[22]
    \begin{enumerate}
        \item Определите класс \texttt{Game}:
        \begin{verbatim}
class Game(Model):
    name = CharField(max_length=200)
    developer = CharField(max_length=150)
    release_date = DateField()
    rating = IntegerField()
        \end{verbatim}
        \item Вызовите: \texttt{Game.create\_table('games')}
        \item Создайте экземпляр:
        \begin{verbatim}
game = Game(
    name='Mystic Forest',
    developer='Pixel Studios',
    release_date=datetime.date(2024, 11, 30),
    rating=92,
)
game.save()
        \end{verbatim}
    \end{enumerate}
    Программа должна вывести следующие SQL-запросы:
    
    \begin{enumerate}
        \item Запрос на создание таблицы (пример для даты выполнения 2025-04-05):
        \begin{verbatim}
CREATE TABLE IF NOT EXISTS games (
    name VARCHAR(200) NOT NULL DEFAULT '',
    developer VARCHAR(150) NOT NULL DEFAULT '',
    release_date DATE NOT NULL DEFAULT '2025-04-05',
    rating INTEGER NOT NULL DEFAULT 0
);
        \end{verbatim}
        
        \item Запрос на вставку данных:
        \begin{verbatim}
INSERT INTO games (name, developer, release_date, rating) VALUES
('Mystic Forest', 'Pixel Studios', '2024-11-30', 92);
        \end{verbatim}
    \end{enumerate}

    \item[23]
    \begin{enumerate}
        \item Определите класс \texttt{Building}:
        \begin{verbatim}
class Building(Model):
    name = CharField(max_length=150)
    architect = CharField(max_length=100)
    completion_date = DateField()
    floors = IntegerField()
        \end{verbatim}
        \item Вызовите: \texttt{Building.create\_table('buildings')}
        \item Создайте экземпляр:
        \begin{verbatim}
building = Building(
    name='Sky Tower',
    architect='Mikhail Sokolov',
    completion_date=datetime.date(2022, 7, 14),
    floors=45,
)
building.save()
        \end{verbatim}
    \end{enumerate}
    Программа должна вывести следующие SQL-запросы:
    
    \begin{enumerate}
        \item Запрос на создание таблицы (пример для даты выполнения 2025-04-05):
        \begin{verbatim}
CREATE TABLE IF NOT EXISTS buildings (
    name VARCHAR(150) NOT NULL DEFAULT '',
    architect VARCHAR(100) NOT NULL DEFAULT '',
    completion_date DATE NOT NULL DEFAULT '2025-04-05',
    floors INTEGER NOT NULL DEFAULT 0
);
        \end{verbatim}
        
        \item Запрос на вставку данных:
        \begin{verbatim}
INSERT INTO buildings (name, architect, completion_date, floors) VALUES
('Sky Tower', 'Mikhail Sokolov', '2022-07-14', 45);
        \end{verbatim}
    \end{enumerate}

    \item[24]
    \begin{enumerate}
        \item Определите класс \texttt{Subscription}:
        \begin{verbatim}
class Subscription(Model):
    service = CharField(max_length=100)
    subscriber = CharField(max_length=150)
    start_date = DateField()
    monthly_fee = IntegerField()
        \end{verbatim}
        \item Вызовите: \texttt{Subscription.create\_table('subscriptions')}
        \item Создайте экземпляр:
        \begin{verbatim}
sub = Subscription(
    service='Cloud Storage Plus',
    subscriber='user@domain.com',
    start_date=datetime.date(2025, 4, 1),
    monthly_fee=250,
)
sub.save()
        \end{verbatim}
    \end{enumerate}
    Программа должна вывести следующие SQL-запросы:
    
    \begin{enumerate}
        \item Запрос на создание таблицы (пример для даты выполнения 2025-04-05):
        \begin{verbatim}
CREATE TABLE IF NOT EXISTS subscriptions (
    service VARCHAR(100) NOT NULL DEFAULT '',
    subscriber VARCHAR(150) NOT NULL DEFAULT '',
    start_date DATE NOT NULL DEFAULT '2025-04-05',
    monthly_fee INTEGER NOT NULL DEFAULT 0
);
        \end{verbatim}
        
        \item Запрос на вставку данных:
        \begin{verbatim}
INSERT INTO subscriptions (service, subscriber, start_date, monthly_fee) VALUES
('Cloud Storage Plus', 'user@domain.com', '2025-04-01', 250);
        \end{verbatim}
    \end{enumerate}

    \item[25]
    \begin{enumerate}
        \item Определите класс \texttt{Vehicle}:
        \begin{verbatim}
class Vehicle(Model):
    brand = CharField(max_length=50)
    model = CharField(max_length=50)
    manufacture_date = DateField()
    mileage = IntegerField()
        \end{verbatim}
        \item Вызовите: \texttt{Vehicle.create\_table('vehicles')}
        \item Создайте экземпляр:
        \begin{verbatim}
vehicle = Vehicle(
    brand='BMW',
    model='X5',
    manufacture_date=datetime.date(2021, 8, 22),
    mileage=45000,
)
vehicle.save()
        \end{verbatim}
    \end{enumerate}
    Программа должна вывести следующие SQL-запросы:
    
    \begin{enumerate}
        \item Запрос на создание таблицы (пример для даты выполнения 2025-04-05):
        \begin{verbatim}
CREATE TABLE IF NOT EXISTS vehicles (
    brand VARCHAR(50) NOT NULL DEFAULT '',
    model VARCHAR(50) NOT NULL DEFAULT '',
    manufacture_date DATE NOT NULL DEFAULT '2025-04-05',
    mileage INTEGER NOT NULL DEFAULT 0
);
        \end{verbatim}
        
        \item Запрос на вставку данных:
        \begin{verbatim}
INSERT INTO vehicles (brand, model, manufacture_date, mileage) VALUES
('BMW', 'X5', '2021-08-22', 45000);
        \end{verbatim}
    \end{enumerate}

    \item[26]
    \begin{enumerate}
        \item Определите класс \texttt{Conference}:
        \begin{verbatim}
class Conference(Model):
    title = CharField(max_length=300)
    organizer = CharField(max_length=150)
    event_date = DateField()
    participants = IntegerField()
        \end{verbatim}
        \item Вызовите: \texttt{Conference.create\_table('conferences')}
        \item Создайте экземпляр:
        \begin{verbatim}
conf = Conference(
    title='AI Ethics Summit',
    organizer='Global Tech Ethics',
    event_date=datetime.date(2025, 9, 10),
    participants=200,
)
conf.save()
        \end{verbatim}
    \end{enumerate}
    Программа должна вывести следующие SQL-запросы:
    
    \begin{enumerate}
        \item Запрос на создание таблицы (пример для даты выполнения 2025-04-05):
        \begin{verbatim}
CREATE TABLE IF NOT EXISTS conferences (
    title VARCHAR(300) NOT NULL DEFAULT '',
    organizer VARCHAR(150) NOT NULL DEFAULT '',
    event_date DATE NOT NULL DEFAULT '2025-04-05',
    participants INTEGER NOT NULL DEFAULT 0
);
        \end{verbatim}
        
        \item Запрос на вставку данных:
        \begin{verbatim}
INSERT INTO conferences (title, organizer, event_date, participants) VALUES
('AI Ethics Summit', 'Global Tech Ethics', '2025-09-10', 200);
        \end{verbatim}
    \end{enumerate}

    \item[27]
    \begin{enumerate}
        \item Определите класс \texttt{LogEntry}:
        \begin{verbatim}
class LogEntry(Model):
    level = CharField(max_length=20)
    message = CharField(max_length=500)
    timestamp_date = DateField()
    code = IntegerField()
        \end{verbatim}
        \item Вызовите: \texttt{LogEntry.create\_table('log\_entries')}
        \item Создайте экземпляр:
        \begin{verbatim}
log = LogEntry(
    level='ERROR',
    message='Database connection failed',
    timestamp_date=datetime.date(2025, 4, 4),
    code=5001,
)
log.save()
        \end{verbatim}
    \end{enumerate}
    Программа должна вывести следующие SQL-запросы:
    
    \begin{enumerate}
        \item Запрос на создание таблицы (пример для даты выполнения 2025-04-05):
        \begin{verbatim}
CREATE TABLE IF NOT EXISTS log_entries (
    level VARCHAR(20) NOT NULL DEFAULT '',
    message VARCHAR(500) NOT NULL DEFAULT '',
    timestamp_date DATE NOT NULL DEFAULT '2025-04-05',
    code INTEGER NOT NULL DEFAULT 0
);
        \end{verbatim}
        
        \item Запрос на вставку данных:
        \begin{verbatim}
INSERT INTO log_entries (level, message, timestamp_date, code) VALUES
('ERROR', 'Database connection failed', '2025-04-04', 5001);
        \end{verbatim}
    \end{enumerate}

    \item[28]
    \begin{enumerate}
        \item Определите класс \texttt{Certificate}:
        \begin{verbatim}
class Certificate(Model):
    course_name = CharField(max_length=200)
    student = CharField(max_length=150)
    issue_date = DateField()
    grade = IntegerField()
        \end{verbatim}
        \item Вызовите: \texttt{Certificate.create\_table('certificates')}
        \item Создайте экземпляр:
        \begin{verbatim}
cert = Certificate(
    course_name='Web Development Fundamentals',
    student='Nikolai Smirnov',
    issue_date=datetime.date(2025, 3, 28),
    grade=88,
)
cert.save()
        \end{verbatim}
    \end{enumerate}
    Программа должна вывести следующие SQL-запросы:
    
    \begin{enumerate}
        \item Запрос на создание таблицы (пример для даты выполнения 2025-04-05):
        \begin{verbatim}
CREATE TABLE IF NOT EXISTS certificates (
    course_name VARCHAR(200) NOT NULL DEFAULT '',
    student VARCHAR(150) NOT NULL DEFAULT '',
    issue_date DATE NOT NULL DEFAULT '2025-04-05',
    grade INTEGER NOT NULL DEFAULT 0
);
        \end{verbatim}
        
        \item Запрос на вставку данных:
        \begin{verbatim}
INSERT INTO certificates (course_name, student, issue_date, grade) VALUES
('Web Development Fundamentals', 'Nikolai Smirnov', '2025-03-28', 88);
        \end{verbatim}
    \end{enumerate}

    \item[29]
    \begin{enumerate}
        \item Определите класс \texttt{Device}:
        \begin{verbatim}
class Device(Model):
    serial = CharField(max_length=100)
    manufacturer = CharField(max_length=100)
    purchase_date = DateField()
    warranty_months = IntegerField()
        \end{verbatim}
        \item Вызовите: \texttt{Device.create\_table('devices')}
        \item Создайте экземпляр:
        \begin{verbatim}
device = Device(
    serial='SN123456789',
    manufacturer='TechCorp',
    purchase_date=datetime.date(2024, 12, 15),
    warranty_months=24,
)
device.save()
        \end{verbatim}
    \end{enumerate}
    Программа должна вывести следующие SQL-запросы:
    
    \begin{enumerate}
        \item Запрос на создание таблицы (пример для даты выполнения 2025-04-05):
        \begin{verbatim}
CREATE TABLE IF NOT EXISTS devices (
    serial VARCHAR(100) NOT NULL DEFAULT '',
    manufacturer VARCHAR(100) NOT NULL DEFAULT '',
    purchase_date DATE NOT NULL DEFAULT '2025-04-05',
    warranty_months INTEGER NOT NULL DEFAULT 0
);
        \end{verbatim}
        
        \item Запрос на вставку данных:
        \begin{verbatim}
INSERT INTO devices (serial, manufacturer, purchase_date, warranty_months) VALUES
('SN123456789', 'TechCorp', '2024-12-15', 24);
        \end{verbatim}
    \end{enumerate}

    \item[30]
    \begin{enumerate}
        \item Определите класс \texttt{Transaction}:
        \begin{verbatim}
class Transaction(Model):
    account = CharField(max_length=50)
    description = CharField(max_length=200)
    date = DateField()
    amount_cents = IntegerField()
        \end{verbatim}
        \item Вызовите: \texttt{Transaction.create\_table('transactions')}
        \item Создайте экземпляр:
        \begin{verbatim}
txn = Transaction(
    account='ACCT-8899',
    description='Online Purchase',
    date=datetime.date(2025, 4, 3),
    amount_cents=-12500,
)
txn.save()
        \end{verbatim}
    \end{enumerate}
    Программа должна вывести следующие SQL-запросы:
    
    \begin{enumerate}
        \item Запрос на создание таблицы (пример для даты выполнения 2025-04-05):
        \begin{verbatim}
CREATE TABLE IF NOT EXISTS transactions (
    account VARCHAR(50) NOT NULL DEFAULT '',
    description VARCHAR(200) NOT NULL DEFAULT '',
    date DATE NOT NULL DEFAULT '2025-04-05',
    amount_cents INTEGER NOT NULL DEFAULT 0
);
        \end{verbatim}
        
        \item Запрос на вставку данных:
        \begin{verbatim}
INSERT INTO transactions (account, description, date, amount_cents) VALUES
('ACCT-8899', 'Online Purchase', '2025-04-03', -12500);
        \end{verbatim}
    \end{enumerate}

    \item[31]
    \begin{enumerate}
        \item Определите класс \texttt{Publication}:
        \begin{verbatim}
class Publication(Model):
    journal = CharField(max_length=200)
    author = CharField(max_length=150)
    publish_date = DateField()
    citations = IntegerField()
        \end{verbatim}
        \item Вызовите: \texttt{Publication.create\_table('publications')}
        \item Создайте экземпляр:
        \begin{verbatim}
pub = Publication(
    journal='Journal of AI Research',
    author='Dr. Anna Volkova',
    publish_date=datetime.date(2024, 6, 12),
    citations=34,
)
pub.save()
        \end{verbatim}
    \end{enumerate}
    Программа должна вывести следующие SQL-запросы:
    
    \begin{enumerate}
        \item Запрос на создание таблицы (пример для даты выполнения 2025-04-05):
        \begin{verbatim}
CREATE TABLE IF NOT EXISTS publications (
    journal VARCHAR(200) NOT NULL DEFAULT '',
    author VARCHAR(150) NOT NULL DEFAULT '',
    publish_date DATE NOT NULL DEFAULT '2025-04-05',
    citations INTEGER NOT NULL DEFAULT 0
);
        \end{verbatim}
        
        \item Запрос на вставку данных:
        \begin{verbatim}
INSERT INTO publications (journal, author, publish_date, citations) VALUES
('Journal of AI Research', 'Dr. Anna Volkova', '2024-06-12', 34);
        \end{verbatim}
    \end{enumerate}

    \item[32]
    \begin{enumerate}
        \item Определите класс \texttt{Appointment}:
        \begin{verbatim}
class Appointment(Model):
    patient = CharField(max_length=100)
    doctor = CharField(max_length=100)
    appointment_date = DateField()
    duration_minutes = IntegerField()
        \end{verbatim}
        \item Вызовите: \texttt{Appointment.create\_table('appointments')}
        \item Создайте экземпляр:
        \begin{verbatim}
appt = Appointment(
    patient='Sergey Orlov',
    doctor='Dr. Elena Kuznetsova',
    appointment_date=datetime.date(2025, 4, 7),
    duration_minutes=30,
)
appt.save()
        \end{verbatim}
    \end{enumerate}
    Программа должна вывести следующие SQL-запросы:
    
    \begin{enumerate}
        \item Запрос на создание таблицы (пример для даты выполнения 2025-04-05):
        \begin{verbatim}
CREATE TABLE IF NOT EXISTS appointments (
    patient VARCHAR(100) NOT NULL DEFAULT '',
    doctor VARCHAR(100) NOT NULL DEFAULT '',
    appointment_date DATE NOT NULL DEFAULT '2025-04-05',
    duration_minutes INTEGER NOT NULL DEFAULT 0
);
        \end{verbatim}
        
        \item Запрос на вставку данных:
        \begin{verbatim}
INSERT INTO appointments (patient, doctor, appointment_date, duration_minutes) VALUES
('Sergey Orlov', 'Dr. Elena Kuznetsova', '2025-04-07', 30);
        \end{verbatim}
    \end{enumerate}

    \item[33]
    \begin{enumerate}
        \item Определите класс \texttt{Contract}:
        \begin{verbatim}
class Contract(Model):
    party_a = CharField(max_length=150)
    party_b = CharField(max_length=150)
    signed_date = DateField()
    term_months = IntegerField()
        \end{verbatim}
        \item Вызовите: \texttt{Contract.create\_table('contracts')}
        \item Создайте экземпляр:
        \begin{verbatim}
contract = Contract(
    party_a='Company LLC',
    party_b='Freelancer Inc.',
    signed_date=datetime.date(2025, 2, 28),
    term_months=12,
)
contract.save()
        \end{verbatim}
    \end{enumerate}
    Программа должна вывести следующие SQL-запросы:
    
    \begin{enumerate}
        \item Запрос на создание таблицы (пример для даты выполнения 2025-04-05):
        \begin{verbatim}
CREATE TABLE IF NOT EXISTS contracts (
    party_a VARCHAR(150) NOT NULL DEFAULT '',
    party_b VARCHAR(150) NOT NULL DEFAULT '',
    signed_date DATE NOT NULL DEFAULT '2025-04-05',
    term_months INTEGER NOT NULL DEFAULT 0
);
        \end{verbatim}
        
        \item Запрос на вставку данных:
        \begin{verbatim}
INSERT INTO contracts (party_a, party_b, signed_date, term_months) VALUES
('Company LLC', 'Freelancer Inc.', '2025-02-28', 12);
        \end{verbatim}
    \end{enumerate}

    \item[34]
    \begin{enumerate}
        \item Определите класс \texttt{Award}:
        \begin{verbatim}
class Award(Model):
    title = CharField(max_length=200)
    recipient = CharField(max_length=150)
    award_date = DateField()
    rank = IntegerField()
        \end{verbatim}
        \item Вызовите: \texttt{Award.create\_table('awards')}
        \item Создайте экземпляр:
        \begin{verbatim}
award = Award(
    title='Best Open Source Project',
    recipient='Team ORM',
    award_date=datetime.date(2024, 10, 15),
    rank=1,
)
award.save()
        \end{verbatim}
    \end{enumerate}
    Программа должна вывести следующие SQL-запросы:
    
    \begin{enumerate}
        \item Запрос на создание таблицы (пример для даты выполнения 2025-04-05):
        \begin{verbatim}
CREATE TABLE IF NOT EXISTS awards (
    title VARCHAR(200) NOT NULL DEFAULT '',
    recipient VARCHAR(150) NOT NULL DEFAULT '',
    award_date DATE NOT NULL DEFAULT '2025-04-05',
    rank INTEGER NOT NULL DEFAULT 0
);
        \end{verbatim}
        
        \item Запрос на вставку данных:
        \begin{verbatim}
INSERT INTO awards (title, recipient, award_date, rank) VALUES
('Best Open Source Project', 'Team ORM', '2024-10-15', 1);
        \end{verbatim}
    \end{enumerate}

    \item[35]
    \begin{enumerate}
        \item Определите класс \texttt{Playlist}:
        \begin{verbatim}
class Playlist(Model):
    name = CharField(max_length=150)
    creator = CharField(max_length=100)
    created_date = DateField()
    track_count = IntegerField()
        \end{verbatim}
        \item Вызовите: \texttt{Playlist.create\_table('playlists')}
        \item Создайте экземпляр:
        \begin{verbatim}
playlist = Playlist(
    name='Coding Focus',
    creator='user123',
    created_date=datetime.date(2025, 3, 30),
    track_count=18,
)
playlist.save()
        \end{verbatim}
    \end{enumerate}
    Программа должна вывести следующие SQL-запросы:
    
    \begin{enumerate}
        \item Запрос на создание таблицы (пример для даты выполнения 2025-04-05):
        \begin{verbatim}
CREATE TABLE IF NOT EXISTS playlists (
    name VARCHAR(150) NOT NULL DEFAULT '',
    creator VARCHAR(100) NOT NULL DEFAULT '',
    created_date DATE NOT NULL DEFAULT '2025-04-05',
    track_count INTEGER NOT NULL DEFAULT 0
);
        \end{verbatim}
        
        \item Запрос на вставку данных:
        \begin{verbatim}
INSERT INTO playlists (name, creator, created_date, track_count) VALUES
('Coding Focus', 'user123', '2025-03-30', 18);
        \end{verbatim}
    \end{enumerate}
\end{enumerate}


\end{document}