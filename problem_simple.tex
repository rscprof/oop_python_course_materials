\subsection{Семинар <<Правила формирования класса для программирования в IDE PyCharm. Отработка навыков создания простых классов и объектов класса>> 
(2 очных часа)}

В ходе работы создайте 5 классов с соответствующими методами, описанными в индивидуальном задании. 
Предполагается, что пользователь класса не имеет права обращаться к свойствам напрямую 
(соблюдая принцип инкапсуляции), а должен использовать методы. Важно: в задании не всегда указаны 
все необходимые методы и свойства, при необходимости вам надо самостоятельно их добавить.
Продемонстрируйте работоспособность всех методов (из задания) посредством создания запускаемых файлов, где осуществляется вызов методов для разных ситуаций (без ручного ввода, но с выводом результатов в консоль). 
Каждый класс должен сохраняться в отдельном исходном файле. Необходимо соблюдать все стандартные требования к качеству кода (отступы, именования переменных, классов, методов, проверка корректности входных данных).
Для каждого класса создайте отдельный запускаемый файл для проверки всех его методов 
(допускается использование других классов в этих тестах).

Все предлагаемые классы в заданиях упрощенные; для использования в production-окружении они требуют серьезной доработки. Суть задания — в отработке базовых навыков, а не в идеальном моделировании предложенных ситуаций.

Для сдачи работы будьте готовы пояснить или аналогично заданию модифицировать любую часть кода, а также ответить на вопросы:
\begin{enumerate}
    \item Кратко опишите парадигму объектно-ориентированного программирования (ООП).
    \item Что такое класс в парадигме ООП?
    \item Что такое объект (экземпляр) в парадигме ООП?
    \item Что обозначает свойство инкапсуляции в парадигме ООП?
    \item Синтаксис классов в Python (в рамках выполненной работы), создание и работа с объектами в Python.
\end{enumerate}

При выполнении задания предполагается самое простое базовое описание классов, соответствующее следующему 
примеру (вы можете использовать то, что вы ЗНАЕТЕ дополнительно, но это остается на ваше усмотрение):

Если вы нашли в задачнике ошибки, опечатки и другие недостатки, то вы можете сделать pull-request. 

\begin{lstlisting}
class Worker:
    def set_last_name(self, last_name):
        self.last_name = last_name

    def print_last_name(self):
        print (f"Фамилия: {self.last_name}")

    def get_last_name(self):
        return last_name

worker = Worker()
worker.set_last_name(self,"Иванов")
worker.print_last_name()
print(worker.get_last_name())
\end{lstlisting}

\textbf{Срок сдачи работы (начала сдачи):} следующее занятие после его выдачи. В последующие сроки оценка будет снижаться (при отсутствии оправдывающих документов).

\begin{enumerate}

\item
\textbf{Описание ситуации:}
Рассмотрим работу грузовой железнодорожной станции. На станции есть несколько путей, по которым поезда могут прибывать и отправляться. Каждый путь имеет свой номер и может вместить несколько поездов. Поезда формируются из вагонов, каждый из которых может перевозить разные грузы. Работники станции отвечают за диспетчерское управление маневровыми локомотивами, осмотр вагонов, выполнение погрузочно-разгрузочных работ, прием груза к перевозке, ремонт путей, обеспечение безопасности и т.п. Они используют радиостанции для связи друг с другом и для отслеживания положения поездов и передвижения вагонов.

\textbf{Создаваемые классы:} `Путь`, `Поезд`, `Вагон`, `Станция`, `РаботникСтанции`.

Для классов реализовать следующие простые методы (ниже приведен не исчерпывающий список методов; для демонстрации работы классов вам потребуются дополнительные методы, позволяющие отследить состояние объектов), используя для хранения данных списки (`[]`) Python:
\begin{enumerate}
    \item \textbf{Путь:} добавить поезд на путь, убрать поезд с пути, получить список поездов на конкретном пути.
    \item \textbf{Поезд:} прицепить вагон к поезду, отцепить вагон от поезда, получить (распечатать) список вагонов в поезде, вывести информацию о грузе в поезде.
    \item \textbf{Вагон:} добавить номер поезда, в который включался конкретный вагон, удалить номер поезда из истории, отобразить историю поездов для конкретного вагона.
    \item \textbf{РаботникСтанции:} класс, представляющий отдельного работника на станции, имеющий идентификатор, информацию о персональной радиостанции, список закрепленных за ним поездов для осмотра, ФИО, должность.
    \item \textbf{Станция:} добавить станционный путь, добавить поезд на станцию, нанять работника станции, вывести информацию о всех путях, поездах, работниках, удалить путь, удалить поезд, уволить работника.
\end{enumerate}

\item
\textbf{Описание ситуации:}
Рассмотрим работу крупного логистического терминала для обработки грузовых автомобилей. На терминале есть несколько доков (рамп), куда фуры прибывают для проведения погрузочно-разгрузочных работ. Каждый док имеет свой номер и может одновременно обслуживать одну машину. Грузовики перевозят паллеты, каждая из которых содержит определенный товар. Сотрудники терминала отвечают за прием грузовиков, управление погрузочной техникой, проверку сопроводительных документов, приемку и отгрузку товара, а также техническое обслуживание доков. Они используют портативные рации для координации действий и отслеживания статуса обработки автомобилей.

\textbf{Создаваемые классы:} `Док`, `Грузовик`, `Паллета`, `Терминал`, `Сотрудник`.

Для классов реализовать следующие простые методы, используя для хранения данных списки (`[]`) Python:
\begin{enumerate}
    \item \textbf{Док:} занять док конкретным грузовиком, освободить док, получить информацию о грузовике, который сейчас находится на доке.
    \item \textbf{Грузовик:} добавить паллету в грузовик, выгрузить паллету из грузовика, получить (распечатать) список паллет в грузовике, вывести информацию о товарах в грузовике.
    \item \textbf{Паллета:} добавить номер грузовика, в который загружалась конкретная паллета, удалить номер грузовика из истории, отобразить историю перевозок (номера грузовиков) для конкретной паллеты.
    \item \textbf{Сотрудник:} класс, представляющий отдельного сотрудника терминала, имеющий идентификатор, номер рации, список доков, за которые он отвечает, ФИО, должность.
    \item \textbf{Терминал:} добавить новый док на терминале, зарегистрировать прибытие грузовика, нанять нового сотрудника, вывести список всех доков, грузовиков на территории, сотрудников, удалить док, удалить грузовик, уволить сотрудника.
\end{enumerate}

\item
\textbf{Описание ситуации:}
Рассмотрим работу аэропорта. В аэропорту есть несколько взлетно-посадочных полос (ВПП), которые принимают и отправляют рейсы. Каждая ВПП имеет свой номер, длину и статус доступности. Самолеты перевозят пассажиров и их ручную кладь, размещенную в салоне. Авиадиспетчеры управляют движением самолетов, назначают полосы для взлета и посадки, следят за воздушной обстановкой и координируют действия с помощью радиосвязи.

\textbf{Создаваемые классы:} `ВПП`, `Самолет`, `Пассажир`, `Аэропорт`, `Авиадиспетчер`.

Для классов реализовать следующие простые методы, используя для хранения данных списки (`[]`) Python:
\begin{enumerate}
    \item \textbf{ВПП:} занять полосу для взлета/посадки, освободить полосу, получить список рейсов, использовавших полосу.
    \item \textbf{Самолет:} добавить пассажира на борт (включая вес его ручной клади), высадить пассажира, получить (распечатать) список пассажиров на борту, рассчитать общий вес ручной клади.
    \item \textbf{Пассажир:} добавить рейс в историю перелетов пассажира, удалить рейс из истории (ошибка бронирования), отобразить всю историю перелетов.
    \item \textbf{Авиадиспетчер:} класс, представляющий диспетчера, имеющий идентификатор, рабочую частоту, график работы (список интервалов времени в сутках), ФИО.
    \item \textbf{Аэропорт:} добавить новую ВПП, зарегистрировать прибытие самолета, нанять диспетчера, вывести список всех ВПП, самолетов в аэропорту, диспетчеров, удалить ВПП (на ремонт), списать самолет, уволить диспетчера.
\end{enumerate}

\item
\textbf{Описание ситуации:}
Рассмотрим работу речного порта. В порту есть несколько причалов для швартовки грузовых барж и буксиров. Каждый причал имеет уникальный номер и максимальную глубину, определяющую осадку судов, которые могут к нему подойти. Баржи перевозят контейнеры с различными грузами. 
Их характеризуют вес судна, максимальная грузоподъемность и осадка (как без груза, так и с максимальным грузом). Портовые рабочие отвечают за швартовку судов, управление портовыми кранами для погрузки/разгрузки контейнеров, оформление документов и поддержание порядка на территории.

\textbf{Создаваемые классы:} `Причал`, `Баржа`, `Контейнер`, `Порт`, `ПортовыйРабочий`.

Для классов реализовать следующие простые методы, используя для хранения данных списки (`[]`) Python:
\begin{enumerate}
    \item \textbf{Причал:} пришвартовать баржу к причалу, отшвартовать баржу, получить список барж, находящихся у причала.
    \item \textbf{Баржа:} загрузить контейнер на баржу (с указанием веса контейнера), разгрузить контейнер с баржи,
     получить (распечатать) список контейнеров на барже, рассчитать текущую осадку судна 
     (предполагается линейная зависимость осадки от суммарного веса груза и баржи).
    \item \textbf{Контейнер:} добавить номер баржи, на которую погрузили контейнер, удалить номер баржи, отобразить историю перемещений контейнера между баржами.
    \item \textbf{ПортовыйРабочий:} класс, представляющий рабочего, имеющий идентификатор, допуск к работе с краном, список закрепленных причалов, ФИО, должность.
    \item \textbf{Порт:} ввести новый причал в эксплуатацию, принять баржу в акваторию порта, принять на работу рабочего, вывести список причалов, барж в акватории, рабочих, списать причал, отправить баржу, уволить рабочего.
\end{enumerate}

\item
\textbf{Описание ситуации:}
Рассмотрим работу автобусного парка. В парке есть несколько маршрутов, которые обслуживаются автобусами. Каждый маршрут имеет номер и список остановок. Автобусы имеют государственный номер, количество мест и текущий пробег. Водители закреплены за автобусами и маршрутами. Диспетчеры автопарка составляют расписание, следят за выходами автобусов на линию, учетом пробега и техническим состоянием.

\textbf{Создаваемые классы:} `Маршрут`, `Автобус`, `Остановка`, `Автопарк`, `Водитель`.

Для классов реализовать следующие простые методы, используя для хранения данных списки (`[]`) Python:
\begin{enumerate}
    \item \textbf{Маршрут:} добавить остановку в маршрут, удалить остановку из маршрута, получить список всех остановок на маршруте.
    \item \textbf{Автобус:} назначить автобус на маршрут, снять с маршрута, увеличить пробег на заданное значение, получить текущий пробег.
    \item \textbf{Остановка:} добавить маршрут, проходящий через остановку, удалить маршрут, отобразить список всех маршрутов, проходящих через данную остановку.
    \item \textbf{Водитель:} класс, представляющий водителя, имеющий идентификатор, права категории, закрепленный автобус, ФИО, график работы.
    \item \textbf{Автопарк:} добавить новый маршрут, приобрести новый автобус, принять на работу водителя, вывести список маршрутов, автобусов (с указанием их состояния), водителей, списать автобус, уволить водителя.
\end{enumerate}

\item
\textbf{Описание ситуации:}
Рассмотрим работу метрополитена. В метро есть линии, состоящие из станций и тоннелей между ними. Составы из вагонов перемещаются по линиям. Каждая станция имеет название и может быть точкой пересадки на другие линии. Машинисты управляют поездами. Дежурные по станции следят за порядком на платформах и работой оборудования. Управление метрополитеном координирует движение составов.

\textbf{Создаваемые классы:} `ЛинияМетро`, `ПоездМетро`, `Станция`, `УправлениеМетрополитеном`, `Машинист`.

Для классов реализовать следующие простые методы, используя для хранения данных списки (`[]`) Python:
\begin{enumerate}
    \item \textbf{ЛинияМетро:} добавить станцию на линию, получить список станций на линии, получить список поездов на линии.
    \item \textbf{ПоездМетро:} добавить вагон в состав, отцепить вагон, назначить машиниста на поезд.
    \item \textbf{Станция:} добавить линию, проходящую через станцию (для моделирования пересадочных узлов), получить список линий на станции.
    \item \textbf{Машинист:} класс, представляющий машиниста, имеющий идентификатор, допуск к управлению, закрепленный поезд, ФИО, стаж.
    \item \textbf{УправлениеМетрополитеном:} открыть новую линию, ввести новый поезд в эксплуатацию, принять на работу машиниста, вывести список линий, поездов (в депо и на линиях), машинистов, закрыть линию на техобслуживание, списать поезд, вывести полную схему метро (в текстовом виде).
\end{enumerate}

\item
\textbf{Описание ситуации:}
Рассмотрим работу службы доставки пиццы. В службе есть несколько филиалов. Каждый филиал обслуживает определенный район и имеет курьеров. Заказы формируются из позиций меню. Курьеры используют скутеры для доставки. Менеджеры филиалов принимают заказы, назначают курьеров и следят за выполнением заказов.

\textbf{Создаваемые классы:} `Филиал`, `Заказ`, `Курьер`, `Скутер`, `Менеджер`.

Для классов реализовать следующие простые методы, используя для хранения данных списки (`[]`) Python:
\begin{enumerate}
    \item \textbf{Филиал:} добавить курьера в филиал, уволить курьера, получить список активных заказов филиала.
    \item \textbf{Заказ:} добавить позицию в заказ (название + цена), удалить позицию, рассчитать стоимость заказа, изменить статус заказа (принят, готовится, в пути, доставлен).
    \item \textbf{Курьер:} назначить заказ курьеру, завершить доставку заказа, получить список доставленных заказов за смену, закрепить скутер за курьером.
    \item \textbf{Менеджер:} класс, представляющий менеджера, имеющий идентификатор, закрепленный филиал, ФИО, смену.
    \item \textbf{Скутер:} отправить на зарядку, вернуть в строй, увеличить пробег, получить текущий пробег.
\end{enumerate}

\textbf{Описание ситуации:}
Рассмотрим работу трамвайного депо. В депо есть несколько маршрутов, обслуживаемых трамвайными вагонами. Каждый трамвайный вагон имеет бортовой номер, вместимость и текущий пробег. Маршруты состоят из остановок и имеют определенный график движения. Водители трамваев закреплены за конкретными вагонами и маршрутами. Диспетчеры управляют выпуском трамваев на линию и ведут учет технического состояния.

\textbf{Создаваемые классы:} Маршрут, Трамвай, Остановка, Депо, Водитель.

Для классов реализовать следующие простые методы, используя для хранения данных списки ([]) Python:
\begin{enumerate}
\item \textbf{Маршрут:} добавить остановку в маршрут, удалить остановку из маршрута, получить список всех остановок на маршруте.
\item \textbf{Трамвай:} назначить трамвай на маршрут, снять с маршрута, увеличить пробег на заданное значение, получить текущий пробег.
\item \textbf{Остановка:} добавить маршрут, проходящий через остановку, удалить маршрут, отобразить список всех маршрутов, проходящих через данную остановку.
\item \textbf{Водитель:} класс, представляющий водителя, имеющий идентификатор, права категории, закрепленный трамвай, ФИО, график работы.
\item \textbf{Депо:} добавить новый маршрут, принять новый трамвай в депо, принять на работу водителя, выполнить вывод списка маршрутов, трамваев (с указанием их состояния), водителей, списать трамвай, уволить водителя.
\end{enumerate}

\item
\textbf{Описание ситуации:}
Рассмотрим работу морского порта для приёма пассажирских паромов. 
В порту есть несколько причалов, каждый из которых обслуживает один паром за раз. 
Паромы перевозят пассажиров и автомобили. 
Пассажиры покупают билеты, автомобили записываются в список грузовой палубы. 
Сотрудники порта координируют погрузку, проверку билетов и безопасность.
\textbf{Создаваемые классы:} Причал, Паром, Пассажир, Автомобиль, СотрудникПорта.
\begin{enumerate}
    \item \textbf{Причал:} пришвартовать паром, освободить причал, получить информацию о пароме у причала.
    \item \textbf{Паром:} добавить пассажира, добавить автомобиль, высадить пассажира, выгрузить автомобиль.
    \item \textbf{Пассажир:} добавить рейс в историю поездок, удалить рейс из истории, 
    вывести историю поездок.
    \item \textbf{Автомобиль:} зарегистрировать номер парома, удалить номер парома, вывести историю перевозок.
    \item \textbf{СотрудникПорта:} идентификатор, должность, ФИО, список закреплённых причалов.
\end{enumerate}

\item
\textbf{Описание ситуации:}
Рассмотрим работу пригородной электрички. В системе есть станции, 
между которыми курсируют электрички. У каждой электрички есть номер, 
список вагонов и машинист. Пассажиры покупают билеты и 
занимают места в вагонах. Диспетчеры контролируют движение электричек.
\textbf{Создаваемые классы:} Станция, Электричка, Вагон, Пассажир, Диспетчер.
\begin{enumerate}
    \item \textbf{Станция:} принять электричку, отправить электричку, вывести список электричек на станции.
    \item \textbf{Электричка:} добавить вагон, отцепить вагон, получить список вагонов.
    \item \textbf{Вагон:} посадить пассажира, высадить пассажира, вывести список пассажиров.
    \item \textbf{Пассажир:} добавить поездку в историю, удалить поездку, показать историю поездок.
    \item \textbf{Диспетчер:} идентификатор, ФИО, рабочая смена, список контролируемых электричек.
\end{enumerate}

\item
\textbf{Описание ситуации:}
Рассмотрим работу таксопарка. В таксопарке есть автомобили, 
водители и диспетчеры. Автомобиль закрепляется за водителем. 
Диспетчеры принимают заказы и назначают их водителям. Пассажиры совершают поездки.
\textbf{Создаваемые классы:} Таксопарк, Автомобиль, Водитель, Заказ, Диспетчер.
\begin{enumerate}
    \item \textbf{Таксопарк:} добавить автомобиль, принять водителя, вывести список машин и водителей, 
    уволить водителя.
    \item \textbf{Автомобиль:} назначить водителя, снять водителя, увеличить пробег, получить пробег.
    \item \textbf{Водитель:} назначить заказ, завершить заказ, 
    вывести список выполненных заказов.
    \item \textbf{Заказ:} назначить пассажира, завершить поездку, вывести информацию о заказе.
    \item \textbf{Диспетчер:} идентификатор, ФИО, список назначенных заказов.
\end{enumerate}

\item
\textbf{Описание ситуации:}
Рассмотрим работу грузового аэропорта. Самолёты перевозят контейнеры. 
В аэропорту есть ангары для хранения самолётов и площадки для погрузки. 
Работники аэропорта координируют загрузку и выгрузку контейнеров.
\textbf{Создаваемые классы:} Самолёт, Контейнер, Ангар, РаботникАэропорта, Аэропорт.
\begin{enumerate}
    \item \textbf{Самолёт:} загрузить контейнер, выгрузить контейнер, вывести список контейнеров.
    \item \textbf{Контейнер:} добавить номер самолёта, удалить номер самолёта, вывести историю перевозок.
    \item \textbf{Ангар:} принять самолёт, вывести список самолётов, освободить ангар.
    \item \textbf{РаботникАэропорта:} идентификатор, ФИО, должность, список самолётов в обслуживании.
    \item \textbf{Аэропорт:} принять самолёт, убрать самолёт, принять раотника, уволить работника, 
    вывести список самолётов и работников.
\end{enumerate}

\item
\textbf{Описание ситуации:}
Рассмотрим работу велопроката. В прокате есть велосипеды, станции для их хранения, 
клиенты и сотрудники. Клиенты арендуют велосипеды и возвращают их на станцию.
\textbf{Создаваемые классы:} Велосипед, СтанцияПроката, Клиент, Сотрудник, Прокат.
\begin{enumerate}
    \item \textbf{Велосипед:} выдать в аренду, вернуть на станцию, получить пробег.
    \item \textbf{СтанцияПроката:} добавить велосипед, убрать велосипед, вывести список велосипедов.
    \item \textbf{Клиент:} арендовать велосипед, вернуть велосипед, вывести историю аренд.
    \item \textbf{Сотрудник:} идентификатор, ФИО, должность, список закреплённых станций.
    \item \textbf{Прокат:} добавить станцию, демонтировать станцию, 
    вывести список станций и велосипедов, уволить сотрудника, нанять сотрудника, вывести список сотрудников.
\end{enumerate}

\item
\textbf{Описание ситуации:}
Рассмотрим работу речных теплоходов. У каждого теплохода есть рейсы и список пассажиров. 
Пассажиры покупают билеты. Работники пристани обслуживают теплоходы.
\textbf{Создаваемые классы:} Теплоход, Рейс, Пассажир, Пристань, РаботникПристани.
\begin{enumerate}
    \item \textbf{Теплоход:} добавить рейс, убрать рейс, вывести список рейсов.
    \item \textbf{Рейс:} добавить пассажира, удалить пассажира, вывести список пассажиров.
    \item \textbf{Пассажир:} добавить рейс в историю, удалить рейс, вывести историю.
    \item \textbf{Пристань:} принять теплоход, отправить теплоход, вывести список теплоходов.
    \item \textbf{РаботникПристани:} идентификатор, ФИО, должность, закреплённые рейсы.
\end{enumerate}

\item
\textbf{Описание ситуации:}
Рассмотрим работу каршеринга. В системе есть автомобили, клиенты и диспетчеры. 
Автомобили бронируются клиентами и возвращаются после поездки. Диспетчеры контролируют состояние машин.
\textbf{Создаваемые классы:} Автомобиль, Клиент, Диспетчер, Заказ, Каршеринг.
\begin{enumerate}
    \item \textbf{Автомобиль:} выдать клиенту, вернуть, увеличить пробег, вывести пробег.
    \item \textbf{Клиент:} арендовать автомобиль, завершить аренду, вывести историю аренд.
    \item \textbf{Диспетчер:} идентификатор, ФИО, список автомобилей под контролем.
    \item \textbf{Заказ:} назначить автомобиль, завершить поездку, вывести данные заказа.
    \item \textbf{Каршеринг:} добавить автомобиль, списать автомобиль, добавить клиента, удалить клиента, 
    добавить диспетчера, удалить диспетчера,
    вывести список клиентов, диспетчеров и машин.
\end{enumerate}

\item
\textbf{Описание ситуации:}
Рассмотрим работу железнодорожного музея. 
В музее есть экспонаты (локомотивы и вагоны), 
экскурсии и экскурсоводы. Посетители записываются на экскурсии.
\textbf{Создаваемые классы:} Экспонат, Экскурсия, Экскурсовод, Посетитель, Музей.
\begin{enumerate}
    \item \textbf{Экспонат:} добавить к экскурсии, убрать, вывести список экскурсий.
    \item \textbf{Экскурсия:} записать посетителя, удалить, вывести список посетителей.
    \item \textbf{Экскурсовод:} идентификатор, ФИО, список экскурсий.
    \item \textbf{Посетитель:} записаться на экскурсию, отменить запись, вывести историю.
    \item \textbf{Музей:} добавить экспонат, списать экспонат, добавить экскурсовода, уволить экскурсовода, 
    провести экскурсию, вывести список всех экскурсий и экскурсоводов.
\end{enumerate}

\item
\textbf{Описание ситуации:}
Рассмотрим работу автозаправочной станции. На станции есть топливо, 
колонки и операторы. Автомобили приезжают заправляться.
\textbf{Создаваемые классы:} Колонка, Автомобиль, Оператор, Топливо, АЗС.
\begin{enumerate}
    \item \textbf{Колонка:} заправить автомобиль, освободить колонку, вывести статус.
    \item \textbf{Автомобиль:} получить заправку, вывести историю заправок.
    \item \textbf{Оператор:} идентификатор, ФИО, список закреплённых колонок.
    \item \textbf{Топливо:} уменьшить количество, увеличить количество, вывести остаток.
    \item \textbf{АЗС:} добавить колонку, нанять оператора, уволить оператора, демонтировать колонку, 
    вывести список машин, операторов и колонок.
\end{enumerate}


\item \textbf{Описание ситуации:} Рассмотрим работу сортировочного центра курьерской службы. 
В центре есть зоны обработки посылок, конвейерные линии и сотрудники. 
Каждая посылка имеет трек-номер и проходит через несколько этапов обработки. 
Сотрудники сканируют посылки, сортируют их по направлениям и загружают в 
транспортировочные контейнеры. Менеджеры контролируют процесс сортировки и работу оборудования.

\textbf{Создаваемые классы:} `ЗонаОбработки`, `Посылка`, `Конвейер`, `СотрудникЦентра`, `СортировочныйЦентр`.

Для классов реализовать следующие простые методы, используя для хранения данных списки (`[]`) Python:
\begin{enumerate}
    \item \textbf{ЗонаОбработки:} добавить посылку в зону, удалить посылку из зоны, 
    получить список посылок в зоне.
    \item \textbf{Посылка:} добавить статус обработки (принята, сортируется, отправлена), 
    удалить ошибочный статус, отобразить историю статусов обработки.
    \item \textbf{Конвейер:} запустить конвейерную ленту, остановить конвейер, 
    добавить посылку на конвейер, снять посылку с конвейера.
    \item \textbf{СотрудникЦентра:} класс, представляющий сотрудника, имеющий идентификатор, 
    смену, список закрепленных зон обработки, ФИО, должность.
    \item \textbf{СортировочныйЦентр:} добавить новую зону обработки, 
    ввести в эксплуатацию конвейер, нанять сотрудника, вывести список всех зон, конвейеров, 
    сотрудников, удалить зону, вывести из эксплуатации конвейер, уволить сотрудника.
\end{enumerate}

\item \textbf{Описание ситуации:} Рассмотрим работу диспетчерской службы 
городского пассажирского транспорта. 
Диспетчеры отслеживают движение автобусов, троллейбусов и трамваев 
на маршрутах, регулируют интервалы движения, фиксируют отклонения от 
графика. Транспортные средства оснащены GPS-трекерами для передачи местоположения.

\textbf{Создаваемые классы:} `Маршрут`, `ТранспортноеСредство`, `Диспетчер`, `Остановка`, `ДиспетчерскаяСлужба`.

Для классов реализовать следующие простые методы, используя для хранения данных списки (`[]`) Python:
\begin{enumerate}
    \item \textbf{Маршрут:} добавить транспортное средство на маршрут, 
    снять с маршрута, получить список транспорта на маршруте.
    \item \textbf{ТранспортноеСредство:} обновить местоположение (координаты), 
    получить текущее местоположение, добавить информацию о задержке/опережении графика.
    \item \textbf{Диспетчер:} класс, представляющий диспетчера, 
    имеющий идентификатор, смену, список контролируемых маршрутов, ФИО.
    \item \textbf{Остановка:} добавить маршрут, проходящий через остановку, 
    удалить маршрут, получить список маршрутов на остановке.
    \item \textbf{ДиспетчерскаяСлужба:} добавить новый маршрут, зарегистрировать транспортное средство, 
    нанять диспетчера, вывести информацию о всех маршрутах, транспорте, 
    диспетчерах, удалить маршрут, списать транспорт, уволить диспетчера.
\end{enumerate}

\item \textbf{Описание ситуации:} Рассмотрим работу центра технического обслуживания 
городского транспорта. В центре есть ремонтные зоны для разных видов 
транспорта, запасы запчастей и бригады механиков. Транспортные средства 
проходят плановое ТО и внеплановый ремонт.

\textbf{Создаваемые классы:} `РемонтнаяЗона`, `ТранспортноеСредство`, `Запчасть`, `Механик`, `ЦентрТехОбслуживания`.

Для классов реализовать следующие простые методы, используя для хранения данных списки (`[]`) Python:
\begin{enumerate}
    \item \textbf{РемонтнаяЗона:} поставить транспорт на ремонт, 
    завершить ремонт, получить список транспорта в ремонте.
    \item \textbf{ТранспортноеСредство:} добавить запись о ремонте (дата, вид работ), 
    удалить ошибочную запись, отобразить историю ремонтов.
    \item \textbf{Запчасть:} уменьшить количество на складе, увеличить количество, 
    получить текущий остаток.
    \item \textbf{Механик:} класс, представляющий механика, имеющий идентификатор, 
    квалификацию, список закрепленных ремонтных зон, ФИО.
    \item \textbf{ЦентрТехОбслуживания:} добавить ремонтную зону, закупить запчасти, 
    нанять механика, вывести информацию о зонах, запчастях, механиках, удалить зону, уволить механика.
\end{enumerate}

\item \textbf{Описание ситуации:}
Рассмотрим работу логистического центра междугородных автобусных перевозок. 
Автобусы совершают рейсы между городами по определенным маршрутам, 
перевозя пассажиров и их багаж. Диспетчеры формируют расписание, 
продают билеты и контролируют отправление автобусов.

\textbf{Создаваемые классы:} `Автобус`, `Маршрут`, `Пассажир`, `Диспетчер`, `ЛогистическийЦентр`.

Для классов реализовать следующие простые методы, используя для хранения данных списки (`[]`) Python:
\begin{enumerate}
    \item \textbf{Автобус:} назначить на маршрут, снять с маршрута, 
    добавить пассажира, высадить пассажира, получить список пассажиров.
    \item \textbf{Маршрут:} добавить город в маршрут, удалить город, 
    получить список всех городов на маршруте.
    \item \textbf{Пассажир:} купить билет (добавить маршрут в историю), 
    сдать билет (удалить маршрут), показать историю поездок.
    \item \textbf{Диспетчер:} класс, представляющий диспетчера, 
    имеющий идентификатор, список закрепленных маршрутов, ФИО, график работы.
    \item \textbf{ЛогистическийЦентр:} добавить автобус в парк, 
    добавить маршрут, нанять диспетчера, вывести список автобусов, 
    маршрутов и диспетчеров, списать автобус, уволить диспетчера.
\end{enumerate}

\item \textbf{Описание ситуации:} Рассмотрим работу центра управления интеллектуальной 
транспортной системой города. Система включает в себя управление светофорами, 
камеры видеонаблюдения, датчики транспортного потока. Операторы следят 
за дорожной ситуацией и оперативно реагируют на инциденты.

\textbf{Создаваемые классы:} `Перекресток`, `Светофор`, `КамераНаблюдения`, `ОператорИТС`, `ЦентрУправления`.

Для классов реализовать следующие простые методы, используя для хранения данных списки (`[]`) Python:
\begin{enumerate}
    \item \textbf{Перекресток:} добавить светофор к перекрестку, удалить светофор, 
    получить список светофоров на перекрестке.
    \item \textbf{Светофор:} изменить режим работы (красный/желтый/зеленый), 
    получить текущий режим, добавить информацию о неисправности, вывести список неисправностей.
    \item \textbf{КамераНаблюдения:} включить запись, выключить запись, 
    получить статус работы, зафиксировать нарушение ПДД, вывести список нарушений.
    \item \textbf{ОператорИТС:} класс, представляющий оператора, 
    имеющий идентификатор, смену, список контролируемых перекрестков, ФИО.
    \item \textbf{ЦентрУправления:} добавить новый перекресток в систему, 
    установить светофор, установить камеру, нанять оператора, вывести информацию о перекрестках, 
    светофорах, камерах, операторах, удалить перекресток, уволить оператора, снять камеру, снять светофор.
\end{enumerate}

\item \textbf{Описание ситуации:} Рассмотрим работу службы эвакуации аварийных транспортных средств. 
Эвакуаторы дежурят на специальных парковках и выезжают по вызову на места ДТП 
или поломок. Диспетчеры принимают вызовы и направляют ближайший свободный эвакуатор.

\textbf{Создаваемые классы:} `Эвакуатор`, `Вызов`, `ПарковкаЭвакуаторов`, `ДиспетчерЭвакуации`, `СлужбаЭвакуации`.

Для классов реализовать следующие простые методы, используя для хранения данных списки (`[]`) Python:
\begin{enumerate}
    \item \textbf{Эвакуатор:} принять вызов, завершить вызов, 
    получить текущий статус (свободен/занят), обновить местоположение.
    \item \textbf{Вызов:} зафиксировать время принятия, время выполнения, 
    получить статус выполнения.
    \item \textbf{ПарковкаЭвакуаторов:} принять эвакуатор на парковку, 
    выпустить эвакуатор с парковки, 
    получить список эвакуаторов на парковке.
    \item \textbf{ДиспетчерЭвакуации:} класс, представляющий диспетчера, 
    имеющий идентификатор, смену, список обработанных вызовов, ФИО.
    \item \textbf{СлужбаЭвакуации:} добавить эвакуатор в парк, 
    списать эвакуатор, 
    нанять диспетчера, вывести информацию о эвакуаторах, вызовах, диспетчерах, уволить диспетчера.
\end{enumerate}

\item \textbf{Описание ситуации:} Рассмотрим работу центра контроля коммерческих грузоперевозок. 
Система отслеживает движение грузовых автомобилей, контролирует соблюдение маршрутов, 
норм труда водителей и расход топлива. Менеджеры по логистике планируют маршруты 
и анализируют отчеты.

\textbf{Создаваемые классы:} `ГрузовойАвтомобиль`, `МаршрутПеревозки`, `Водитель`, `Рейс`, `МенеджерЛогистики`.

Для классов реализовать следующие простые методы, используя для хранения данных списки (`[]`) Python:
\begin{enumerate}
    \item \textbf{ГрузовойАвтомобиль:} начать рейс, завершить рейс, получить текущий статус, 
    зафиксировать расход топлива.
    \item \textbf{МаршрутПеревозки:} добавить точку маршрута (город, склад), 
    удалить точку, получить полный маршрут.
    \item \textbf{Водитель:} класс, представляющий водителя, 
    имеющий идентификатор, права, график работы, ФИО, стаж.
    \item \textbf{Рейс:} закрепить автомобиль за рейсом, закрепить водителя за рейсом, 
    открепить автомобиль, снять водителя, получить информацию о рейсе.
    \item \textbf{МенеджерЛогистики:} класс, представляющий менеджера, 
    имеющий идентификатор, список контролируемых маршрутов, ФИО.
\end{enumerate}

\item \textbf{Описание ситуации:} Рассмотрим работу службы парковки аэропорта. 
На территории аэропорта есть несколько парковочных зон для разных типов 
транспорта (краткосрочная, долгосрочная, VIP). 
Операторы контролируют занятость мест, прием оплаты и работу шлагбаумов.

\textbf{Создаваемые классы:} `ПарковочнаяЗона`, `ПарковочноеМесто`, `Автомобиль`, `ОператорПарковки`, `СлужбаПарковки`.

Для классов реализовать следующие простые методы, используя для хранения данных списки (`[]`) Python:
\begin{enumerate}
    \item \textbf{ПарковочнаяЗона:} добавить парковочное место, 
    удалить место, получить список мест в зоне, получить список всех автомобилей. Так же парковочной зоне 
    соответсвует стоимость часа стоянки.
    \item \textbf{ПарковочноеМесто:} занять место автомобилем, 
    освободить место, получить текущий статус (свободно/занято).
    \item \textbf{Автомобиль:} зафиксировать время въезда, время выезда + 
    рассчитать стоимость парковки (с учетом стоимости часа), получить историю.
    \item \textbf{ОператорПарковки:} класс, представляющий оператора, 
    имеющий идентификатор, смену, список контролируемых зон, ФИО.
    \item \textbf{СлужбаПарковки:} добавить новую парковочную зону, 
    нанять оператора, вывести информацию о зонах, местах, операторах, удалить зону, уволить оператора.
\end{enumerate}

\item \textbf{Описание ситуации:} Рассмотрим работу центра управления речным судоходством. 
Диспетчеры следят за движением судов по фарватеру, 
распределяют шлюзы, контролируют соблюдение графика движения 
и обеспечивают безопасность судоходства.

\textbf{Создаваемые классы:} `Судно`, `Шлюз`, `Фарватер`, `ДиспетчерСудоходства`, `ЦентрУправления`.

Для классов реализовать следующие простые методы, используя для хранения данных списки (`[]`) Python:
\begin{enumerate}
    \item \textbf{Судно:} начать движение по фарватеру, завершить движение, 
    получить текущее местоположение, зафиксировать прохождение шлюза.
    \item \textbf{Шлюз:} принять судно для шлюзования, 
    завершить шлюзование, получить текущий статус (свободен/занят).
    \item \textbf{Фарватер:} добавить участок фарватера, 
    удалить участок, получить список судов на фарватере.
    \item \textbf{ДиспетчерСудоходства:} класс, представляющий диспетчера, 
    имеющий идентификатор, смену, список контролируемых шлюзов, ФИО.
    \item \textbf{ЦентрУправления:} добавить шлюз в систему, 
    зарегистрировать судно, нанять диспетчера, 
    вывести информацию о шлюзах, фарватерах, судах, диспетчерах, удалить шлюз, уволить диспетчера.
\end{enumerate}

\item \textbf{Описание ситуации:} Рассмотрим работу службы технического контроля метрополитена. 
Инспекторы проверяют состояние путей, тоннелей, подвижного состава и оборудования станций. 
Дефекты фиксируются в системе для оперативного устранения ремонтными бригадами.

\textbf{Создаваемые классы:} `УчастокПути`, `ПодвижнойСостав`, `Инспектор`, `Дефект`, `СлужбаКонтроля`.

Для классов реализовать следующие простые методы, использующие для хранения данных списки (`[]`) Python:
\begin{enumerate}
    \item \textbf{УчастокПути:} добавить информацию о дефекте, 
    получить список неустраненных дефектов на участке.
    \item \textbf{ПодвижнойСостав:} добавить запись о техническом осмотре, 
    удалить ошибочную запись, отобразить историю осмотров.
    \item \textbf{Инспектор:} класс, представляющий инспектора, 
    имеющий идентификатор, квалификацию, список закрепленных участков, ФИО.
    \item \textbf{Дефект:} зафиксировать время обнаружения, 
    время устранения, получить статус устранения.
    \item \textbf{СлужбаКонтроля:} добавить участок пути в систему, 
    зарегистрировать подвижной состав, нанять инспектора, 
    вывести информацию об участках, составе, инспекторах, дефектах, удалить участок, уволить инспектора, 
    снять с эксплуатации подвижной состав.
\end{enumerate}

\item
\textbf{Описание ситуации:}
Рассмотрим работу центра управления умными светофорами на перекрестках. 
Умные светофоры адаптивно меняют режим работы в зависимости от транспортного потока, 
приоритизируя общественный транспорт и спецтранспорт. 
Система анализирует данные с датчиков и камер, оптимизируя пропускную способность перекрестков.

\textbf{Создаваемые классы:} УмныйСветофор, Перекресток, ДатчикТранспортногоПотока, ИнженерАТС, ЦентрУправленияСветофорами.

Для классов реализовать следующие простые методы, используя для хранения данных списки ([]) Python:
\begin{enumerate}
\item \textbf{УмныйСветофор:} изменить длительность фаз (красный/зеленый), 
перейти в аварийный режим, получить текущий режим работы.
\item \textbf{Перекресток:} добавить светофор к перекрестку, удалить светофор, 
получить список всех светофоров перекрестка.
\item \textbf{ДатчикТранспортногоПотока:} установить текущие данные о интенсивности движения, 
получить текущие показания, получить историю показаний.
\item \textbf{ИнженерАТС:} класс, представляющий инженера автоматизированной транспортной системы, 
имеющий идентификатор, квалификацию, список закрепленных перекрестков, ФИО.
\item \textbf{ЦентрУправленияСветофорами:} добавить новый перекресток в систему, 
установить умный светофор, нанять инженера, вывести информацию о перекрестках, светофорах, инженерах, 
удалить перекресток, уволить инженера, снять умный светофор.
\end{enumerate}

\item
\textbf{Описание ситуации:}
Рассмотрим работу монорельсовой транспортной системы. 
Монорельс движется по эстакаде, состоящей из станций и перегонов. 
Составы имеют фиксированное количество вагонов. 
Операторы управляют движением составов, следят за соблюдением графика и безопасностью пассажиров.

\textbf{Создаваемые классы:} СтанцияМонорельса, СоставМонорельса, ВагонМонорельса, ОператорСистемы, УправлениеМонорельсом.

Для классов реализовать следующие простые методы, используя для хранения данных списки ([]) Python:
\begin{enumerate}
\item \textbf{СтанцияМонорельса:} принять состав, отправить состав, 
получить список составов на станции.
\item \textbf{СоставМонорельса:} добавить вагон в состав (при техническом обслуживании), 
удалить вагон, получить список вагонов.
\item \textbf{ВагонМонорельса:} зафиксировать текущий пробег, 
провести техническое обслуживание, получить историю обслуживаний.
\item \textbf{ОператорСистемы:} класс, представляющий оператора, 
имеющий идентификатор, смену, список закрепленных станций, ФИО.
\item \textbf{УправлениеМонорельсом:} добавить новую станцию, ввести состав в эксплуатацию, 
нанять оператора, вывести информацию о станциях, составах, операторах, закрыть станцию на ремонт, 
списать состав, уволить оператора.
\end{enumerate}

\item
\textbf{Описание ситуации:}
Рассмотрим работу канатной дороги. Канатная дорога состоит из линий с опорами и кабинок, 
перемещающихся между станциями. 
Кабинки имеют ограниченную вместимость. Техники обслуживают механизмы и следят за безопасностью.

\textbf{Создаваемые классы:} ЛинияКанатнойДороги, Кабинка, СтанцияКанатнойДороги, Техник, УправлениеКанатнойДорогой.

Для классов реализовать следующие простые методы, используя для хранения данных списки ([]) Python:
\begin{enumerate}
\item \textbf{ЛинияКанатнойДороги:} добавить кабинку на линию, снять кабинку, 
получить список кабинок на линии.
\item \textbf{Кабинка:} запустить в движение, остановить для посадки/высадки, 
получить текущий статус (движется/стоит).
\item \textbf{СтанцияКанатнойДороги:} принять кабинку, отправить кабинку, 
получить список кабинок на станции.
\item \textbf{Техник:} класс, представляющий техника, имеющий идентификатор, 
квалификацию, список закрепленных линий, ФИО.
\item \textbf{УправлениеКанатнойДорогой:} добавить новую линию, 
ввести кабинку в эксплуатацию, нанять техника, вывести информацию о линиях, кабинках, техниках, 
закрыть линию на обслуживание, списать кабинку, уволить техника.
\end{enumerate}

\item
\textbf{Описание ситуации:}
Рассмотрим работу службы доставки с использованием дронов. 
Дроны осуществляют доставку небольших грузов между пунктами выдачи. 
Каждый дрон имеет грузоподъемность и дальность полета. 
Операторы управляют полетами дронов и обслуживают пункты выдачи.

\textbf{Создаваемые классы:} ПунктВыдачи, Дрон, Груз, ОператорДронов, СлужбаДоставки.

Для классов реализовать следующие простые методы, используя для хранения данных списки ([]) Python:
\begin{enumerate}
\item \textbf{ПунктВыдачи:} принять дрон с грузом, отправить дрон, получить список дронов в пункте.
\item \textbf{Дрон:} загрузить груз, выгрузить груз, 
начать полет, завершить полет, получить текущий статус (в полете/на земле).
\item \textbf{Груз:} зарегистрировать отправку, зарегистрировать доставку, 
получить историю перемещений.
\item \textbf{ОператорДронов:} класс, представляющий оператора, 
имеющий идентификатор, смену, список закрепленных пунктов выдачи, ФИО.
\item \textbf{СлужбаДоставки:} добавить новый пункт выдачи, 
ввести дрон в эксплуатацию, нанять оператора, вывести информацию о пунктах, 
дронах, операторах, закрыть пункт, списать дрон, уволить оператора.
\end{enumerate}

\end{enumerate}