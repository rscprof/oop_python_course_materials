\subsection{Семинар <<Ограничения доступа и Unit-тестирование>>  
(2 часа)}

В ходе работы решите 2 задачи. 

Первое задание предполагает просто описание способов доступа к свойствам и 
методам различными способами.

Второе задание -- реализацию простого класса и unit-тестов для него.

\subsubsection{Принципы unit-тестирования в Python}

Unit-тестирование позволяет проверять отдельные части кода — функции, методы или классы. Основные принципы:

\begin{itemize}
    \item Каждый тест проверяет \textbf{одну конкретную функциональность}.
    \item Тесты должны покрывать \textbf{все важные сценарии использования}, включая краевые и граничные значения.
    \item Тесты должны быть \textbf{повторяемыми и независимыми} друг от друга.
    \item Используются утверждения: \texttt{assertEqual}, \texttt{assertTrue}, \texttt{assertFalse}, \texttt{assertRaises}.
\end{itemize}

\subsubsection{Как анализировать код для тестирования всех случаев}

При разработке unit-тестов важно систематически анализировать код и выявлять все ветви и варианты поведения:

\begin{enumerate}
    \item \textbf{Анализ условных операторов (if/else)}:  
    Для каждого условия нужно проверить как «истинный» путь, так и «ложный».  
    Пример:  
    \begin{verbatim}
def divide(a, b):
    if b == 0:
        raise ValueError("Division by zero")
    return a / b
    \end{verbatim}  
    Тесты должны проверять:
    \begin{itemize}
        \item деление на ненулевое число (if=False)
        \item деление на ноль (if=True)
    \end{itemize}

    \item \textbf{Анализ циклов (for/while)}:  
    Циклы проверяются на:
    \begin{itemize}
        \item пустой вход (0 итераций)
        \item одну итерацию
        \item несколько итераций
        \item граничные случаи (максимально допустимое число элементов)
    \end{itemize}

    \item \textbf{Граничные значения (boundary values)}:  
    Любой метод, работающий с числами или индексами, должен проверяться на:
    \begin{itemize}
        \item минимальные допустимые значения
        \item максимальные допустимые значения
        \item ноль и отрицательные значения (если применимо)
    \end{itemize}

    \item \textbf{Исключения и ошибки}:  
    Нужно проверять, что код корректно реагирует на некорректные входные данные, выбрасывая ожидаемые исключения.

    \item \textbf{Комбинации входных данных}:  
    Для методов с несколькими параметрами важно проверять сочетания «нормальных» и «краевых» значений.
\end{enumerate}

\subsubsection{Пример простого класса с unit-тестами}

Рассмотрим класс \texttt{Calculator}, который выполняет сложение и деление чисел:

\begin{verbatim}
# calculator.py
class Calculator:
    def add(self, a, b):
        return a + b

    def divide(self, a, b):
        if b == 0:
            raise ValueError("Division by zero")
        return a / b
\end{verbatim}

\begin{verbatim}
# test_calculator.py
import unittest
from calculator import Calculator

class TestCalculator(unittest.TestCase):

    def setUp(self):
        self.calc = Calculator()

    # Проверка всех важных случаев для сложения
    def test_add_positive_numbers(self):
        self.assertEqual(self.calc.add(2, 3), 5)

    def test_add_negative_numbers(self):
        self.assertEqual(self.calc.add(-2, -3), -5)

    def test_add_zero(self):
        self.assertEqual(self.calc.add(0, 5), 5)
        self.assertEqual(self.calc.add(5, 0), 5)

    # Проверка всех важных случаев для деления
    def test_divide_normal(self):
        self.assertEqual(self.calc.divide(10, 2), 5)

    def test_divide_fraction(self):
        self.assertEqual(self.calc.divide(1, 2), 0.5)

    def test_divide_by_zero(self):
        with self.assertRaises(ValueError):
            self.calc.divide(5, 0)

if __name__ == '__main__':
    unittest.main()
\end{verbatim}

\textbf{Объяснение:}  

\begin{itemize}
    \item \texttt{setUp()} создает объект перед каждым тестом.
    \item Каждый метод, имя которого начинается с \texttt{test\_}, проверяет отдельный сценарий.
    \item Мы покрыли:
    \begin{itemize}
        \item положительные и отрицательные числа
        \item ноль
        \item дробные значения
        \item исключения (деление на ноль)
    \end{itemize}
    \item Для более сложного кода нужно аналогично анализировать все условия и ветвления.
\end{itemize}

Для сдачи работы будьте готовы пояснить или модифицировать любую часть кода, а также ответить на вопросы:

\begin{enumerate}
    \item Как можно обеспечить инкапсуляцию в Python (перечислите все варианты)
\end{enumerate}

Если вы нашли в задачнике ошибки, опечатки и другие недостатки, то вы можете сделать pull-request.  

\subsubsection{Задача 1}

\input{problem_access_1}

\subsubsection{Задача 2}



\textbf{Инструкция:} Напишите функцию и соответствующие unit-тесты, покрывающие все важные случаи. Для заданий с ветвлениями (\texttt{if}/\texttt{elif}/\texttt{else}) обязательно проверяйте все ветви.  

\textbf{Пояснения:}  
\begin{itemize}
    \item \textbf{Triangle types:} 
        \begin{itemize}
            \item \texttt{equilateral} — все стороны равны  
            \item \texttt{isosceles} — две стороны равны  
            \item \texttt{scalene} — все стороны разные  
            \item \texttt{invalid} — невозможно построить треугольник  
        \end{itemize}
    \item \textbf{BMI (Body Mass Index):} индекс массы тела. Категории: \texttt{Underweight}, \texttt{Normal}, \texttt{Overweight}, \texttt{Obese}  
    \item \textbf{Palindrome:} строка или число, читающееся одинаково слева направо и справа налево  
    \item \textbf{Perfect number:} число, равное сумме своих делителей, исключая само число  
    \item \textbf{Triangle angles:} 
        \begin{itemize}
            \item \texttt{acute} — все углы < 90°  
            \item \texttt{right} — один угол = 90°  
            \item \texttt{obtuse} — один угол > 90°  
            \item \texttt{invalid} — треугольник не существует  
        \end{itemize}
    \item \textbf{Traffic fine:} штраф за превышение скорости. Функция должна учитывать разные зоны (\texttt{residential}, \texttt{city}, \texttt{highway}) и уровни превышения скорости.
\end{itemize}

\begin{enumerate}
    \item \texttt{classify\_triangle(a, b, c)} — возвращает тип треугольника.  
    \item \texttt{classify\_number(n)} — возвращает `"positive even"`, `"positive odd"`, `"negative even"`, `"negative odd"`, `"zero"`.  
    \item \texttt{middle\_value(a, b, c)} — возвращает среднее (не арифметическое) число среди трёх, через сравнения.  
    \item \texttt{median\_of\_three(a, b, c)} — медиана трёх чисел через if/elif/else.  
    \item \texttt{is\_leap\_year(year)} — проверяет високосный год (делится на 4, но не на 100, или на 400).  
    \item \texttt{bmi\_category(weight, height)} — возвращает категорию BMI.  
    \item \texttt{categorize\_temperature(temp)} — диапазоны: `"freezing"` $\leqslant$ 0°C, `"cold"` 1–10°C, `"cool"` 11–20°C, `"warm"` 21–30°C, `"hot"` >30°C.  
    \item \texttt{triangle\_area\_type(a, b, c)} — возвращает `"acute"`, `"right"`, `"obtuse"` или `"invalid"`.  
    \item \texttt{quadrant(x, y)} — возвращает номер четверти (1–4) или `"origin"`/`"axis"`.  
    \item \texttt{days\_in\_month(month, leap)} — возвращает число дней в месяце; \texttt{leap} = True для високосного года.  
    \item \texttt{traffic\_fine(speed, zone)} — вычисляет штраф за превышение скорости.  
    \begin{itemize}
        \item \textbf{speed} — скорость автомобиля (км/ч)  
        \item \textbf{zone} — тип зоны: `"residential"`, `"city"`, `"highway"`  
        \item \textbf{правила:}  
        \begin{itemize}
            \item `"residential"`: превышение >20 км/ч → 200, >10 км/ч → 100, иначе 0  
            \item `"city"`: превышение >30 → 150, >15 → 75, иначе 0  
            \item `"highway"`: превышение >40 → 100, >20 → 50, иначе 0  
            \item некорректная зона → `"invalid zone"`  
        \end{itemize}
        \item \textbf{требования:} использовать ветвления if/elif/else, проверить все сценарии превышения и отсутствие превышения.  
    \end{itemize}
    \item \texttt{compare\_three\_numbers(a, b, c)} — возвращает `"all equal"`, `"all different"`, или `"two equal"`.  
    \item \texttt{max\_digit(n)} — наибольшая цифра числа.  
    \item \texttt{triangle\_angle\_category(a, b, c)} — вычисляет углы через теорему косинусов и возвращает `"acute"`, `"right"`, `"obtuse"`, `"invalid"`.  
    \item \texttt{next\_day(day, month, leap)} — возвращает следующий день месяца, учитывая количество дней.  
    \item \texttt{is\_inside\_rectangle(x, y, x1, y1, x2, y2)} — проверка попадания точки в прямоугольник.  
    \item \texttt{discount(price)} — возврат цены со скидкой по условию (>1000 → 10%, >500 → 5%, иначе 0%).  
    \item \texttt{closest\_to\_zero(lst)} — элемент списка, ближайший к нулю.  
    \item \texttt{season(month)} — `"Winter"`, `"Spring"`, `"Summer"`, `"Autumn"`.  
    \item \texttt{simple\_calculator(a, b, op)} — +, -, *, /; деление на 0 → `"error"`.  
    \item \texttt{sum\_positive(lst)} — сумма положительных чисел.  
    \item \texttt{sum\_even(lst)} — сумма чётных чисел.  
    \item \texttt{sum\_odd(lst)} — сумма нечётных чисел.  
    \item \texttt{reverse\_signs(lst)} — меняет знак всех элементов.  
    \item \texttt{nearest\_multiple(n, m)} — ближайшее к n кратное m.  
    \item \texttt{sort\_three(a, b, c)} — тройка чисел в порядке возрастания через if/elif/else.  
    \item \texttt{validate\_password(password)} — True, если длина $\geqslant$8, есть цифра и заглавная буква.  
    \item \texttt{is\_perfect(n)} — True, если число совершенное.  
    \item \texttt{sum\_digits(n)} — сумма цифр числа.  
    \item \texttt{count\_vowels(s)} — количество гласных.  
    \item \texttt{is\_palindrome(s)} — True, если строка читается одинаково слева направо и справа налево.  
    \item \texttt{remove\_duplicates(lst)} — возвращает список без повторов.  
    \item \texttt{factorial\_iterative(n)} — факториал через цикл, без рекурсии.  
    \item \texttt{fizz\_buzz(n)} — числа от 1 до n с заменой кратных 3 → `"Fizz"`, кратных 5 → `"Buzz"`, кратных 15 → `"FizzBuzz"`.  
\end{enumerate}

