\subsection{Семинар <<Ограничения доступа и Unit-тестирование>>  
(2 часа)}

В ходе работы решите 2 задачи. 

Первое задание предполагает просто описание способов доступа к свойствам и 
методам различными способами.

Второе задание -- реализацию простого класса и unit-тестов для него.

\subsubsection{Принципы unit-тестирования в Python}

Unit-тестирование позволяет проверять отдельные части кода — функции, методы или классы. Основные принципы:

\begin{itemize}
    \item Каждый тест проверяет \textbf{одну конкретную функциональность}.
    \item Тесты должны покрывать \textbf{все важные сценарии использования}, включая краевые и граничные значения.
    \item Тесты должны быть \textbf{повторяемыми и независимыми} друг от друга.
    \item Используются утверждения: \texttt{assertEqual}, \texttt{assertTrue}, \texttt{assertFalse}, \texttt{assertRaises}.
\end{itemize}

\subsubsection{Как анализировать код для тестирования всех случаев}

При разработке unit-тестов важно систематически анализировать код и выявлять все ветви и варианты поведения:

\begin{enumerate}
    \item \textbf{Анализ условных операторов (if/else)}:  
    Для каждого условия нужно проверить как «истинный» путь, так и «ложный».  
    Пример:  
    \begin{verbatim}
def divide(a, b):
    if b == 0:
        raise ValueError("Division by zero")
    return a / b
    \end{verbatim}  
    Тесты должны проверять:
    \begin{itemize}
        \item деление на ненулевое число (if=False)
        \item деление на ноль (if=True)
    \end{itemize}

    \item \textbf{Анализ циклов (for/while)}:  
    Циклы проверяются на:
    \begin{itemize}
        \item пустой вход (0 итераций)
        \item одну итерацию
        \item несколько итераций
        \item граничные случаи (максимально допустимое число элементов)
    \end{itemize}

    \item \textbf{Граничные значения (boundary values)}:  
    Любой метод, работающий с числами или индексами, должен проверяться на:
    \begin{itemize}
        \item минимальные допустимые значения
        \item максимальные допустимые значения
        \item ноль и отрицательные значения (если применимо)
    \end{itemize}

    \item \textbf{Исключения и ошибки}:  
    Нужно проверять, что код корректно реагирует на некорректные входные данные, выбрасывая ожидаемые исключения.

    \item \textbf{Комбинации входных данных}:  
    Для методов с несколькими параметрами важно проверять сочетания «нормальных» и «краевых» значений.
\end{enumerate}

\subsubsection{Пример простого класса с unit-тестами}

Рассмотрим класс \texttt{Calculator}, который выполняет сложение и деление чисел:

\begin{verbatim}
# calculator.py
class Calculator:
    def add(self, a, b):
        return a + b

    def divide(self, a, b):
        if b == 0:
            raise ValueError("Division by zero")
        return a / b
\end{verbatim}

\begin{verbatim}
# test_calculator.py
import unittest
from calculator import Calculator

class TestCalculator(unittest.TestCase):

    def setUp(self):
        self.calc = Calculator()

    # Проверка всех важных случаев для сложения
    def test_add_positive_numbers(self):
        self.assertEqual(self.calc.add(2, 3), 5)

    def test_add_negative_numbers(self):
        self.assertEqual(self.calc.add(-2, -3), -5)

    def test_add_zero(self):
        self.assertEqual(self.calc.add(0, 5), 5)
        self.assertEqual(self.calc.add(5, 0), 5)

    # Проверка всех важных случаев для деления
    def test_divide_normal(self):
        self.assertEqual(self.calc.divide(10, 2), 5)

    def test_divide_fraction(self):
        self.assertEqual(self.calc.divide(1, 2), 0.5)

    def test_divide_by_zero(self):
        with self.assertRaises(ValueError):
            self.calc.divide(5, 0)

if __name__ == '__main__':
    unittest.main()
\end{verbatim}

\textbf{Объяснение:}  

\begin{itemize}
    \item \texttt{setUp()} создает объект перед каждым тестом.
    \item Каждый метод, имя которого начинается с \texttt{test\_}, проверяет отдельный сценарий.
    \item Мы покрыли:
    \begin{itemize}
        \item положительные и отрицательные числа
        \item ноль
        \item дробные значения
        \item исключения (деление на ноль)
    \end{itemize}
    \item Для более сложного кода нужно аналогично анализировать все условия и ветвления.
\end{itemize}

Для сдачи работы будьте готовы пояснить или модифицировать любую часть кода, а также ответить на вопросы:

\begin{enumerate}
    \item Как можно обеспечить инкапсуляцию в Python (перечислите все варианты)
\end{enumerate}

Если вы нашли в задачнике ошибки, опечатки и другие недостатки, то вы можете сделать pull-request.  

\subsubsection{Задача 1}

\begin{enumerate}
\item[1] Разработать класс \texttt{Bus}, который будет описывать модель автобуса. В классе должны быть следующие поля с доступом уровня \textbf{private} (только внутри класса):
\begin{itemize}
    \item \texttt{\_\_speed}: скорость движения автобуса  
    \item \texttt{\_\_distance}: расстояние, которое автобус проехал  
    \item \texttt{\_\_max\_speed}: максимальная разрешённая скорость движения автобуса  
    \item \texttt{\_\_passengers}: список пассажиров  
    \item \texttt{\_\_capacity}: максимальная вместимость пассажиров в автобусе  
    \item \texttt{\_\_empty\_seats}: число свободных мест  
    \item \texttt{\_\_seats\_occupied}: число занятых мест в автобусе  
    \item \texttt{\_\_fuel\_tank}: объём топливного бака  
    \item \texttt{\_\_fuel}: количество топлива в литрах  
    \item \texttt{\_\_engine\_oil\_capacity}: объём картера масла двигателя (литры)  
    \item \texttt{\_\_engine\_oil}: количество моторного масла в литрах  
    \item \texttt{\_\_luggage\_spaces}: количество багажных мест  
    \item \texttt{\_\_luggage}: багаж автобуса  
\end{itemize}
Уровень доступа к полям должен быть следующим:
\begin{itemize}
    \item \texttt{\_\_max\_speed}, \texttt{\_\_capacity}, \texttt{\_\_fuel\_tank}, \texttt{\_\_engine\_oil\_capacity}, \texttt{\_\_luggage\_spaces}: \textbf{только чтение} (через геттеры)  
    \item \texttt{\_\_speed}, \texttt{\_\_distance}, \texttt{\_\_passengers}, \texttt{\_\_empty\_seats}, \texttt{\_\_seats\_occupied}, \texttt{\_\_fuel}, \texttt{\_\_engine\_oil}, \texttt{\_\_luggage}: \textbf{чтение и запись} (через геттеры и сеттеры)
\end{itemize}
Требования к сеттерам:
\begin{itemize}
    \item Для полей \texttt{\_\_empty\_seats} и \texttt{\_\_seats\_occupied} в сеттерах необходимо проверять, что передаваемое значение не превышает \texttt{\_\_capacity} и неотрицательно.  
    \item Для поля \texttt{\_\_passengers} в сеттере необходимо проверять, что количество пассажиров (длина списка) не превышает \texttt{\_\_capacity}.  
    \item Для поля \texttt{\_\_speed} в сеттере необходимо проверять, что заданная скорость не превышает \texttt{\_\_max\_speed} и неотрицательна.  
    \item Для поля \texttt{\_\_luggage} в сеттере необходимо проверять, что количество единиц багажа не превышает \texttt{\_\_luggage\_spaces}.
    \item Для полей \texttt{\_\_fuel} и \texttt{\_\_engine\_oil} значения не должны превышать соответствующие ёмкости (\texttt{\_\_fuel\_tank} и \texttt{\_\_engine\_oil\_capacity}) и должны быть неотрицательными.
\end{itemize}
Реализовать метод вывода всех установленных через сеттеры значений закрытых полей экземпляра класса.
На основе этого класса реализовать три подхода к управлению доступом:
\begin{enumerate}
    \item \textbf{С использованием объекта \texttt{property}}:  
    Для каждого поля определить отдельные методы-геттеры и сеттеры (например, \texttt{get\_speed}, \texttt{set\_speed}), а затем создать свойство:  
    \begin{verbatim}
speed = property(get_speed, set_speed)
    \end{verbatim}  
    Этот код должен располагаться после определения соответствующих методов. Первый аргумент — геттер, второй — сеттер.  
    Продемонстрировать работу на трёх экземплярах класса: создать \texttt{mybus1}, \texttt{mybus2}, \texttt{mybus3}, установить значения через свойства и вывести их.
    \item \textbf{С использованием декораторов \texttt{@property} и \texttt{@<имя>.setter}}:  
    Создать новую версию класса, в которой геттеры оформляются с декоратором \texttt{@property}, а сеттеры — с декоратором вида \texttt{@speed.setter}. Имена методов должны совпадать и не содержать префиксов \texttt{get\_}/\texttt{set\_}.  
    Пример:  
    \begin{verbatim}
@property
def speed(self):
    return self.__speed
@speed.setter
def speed(self, value):
    if 0 <= value <= self.__max_speed:
        self.__speed = value
    else:
        raise ValueError("Недопустимая скорость")
    \end{verbatim}  
    Продемонстрировать работу на трёх экземплярах и сделать выводы об оптимизации кода по сравнению с первым подходом.
    \item \textbf{С использованием модуля \texttt{accessify}}:  
    Установить модуль командой \texttt{pip install accessify} и импортировать:  
    \begin{verbatim}
from accessify import private, protected
    \end{verbatim}  
    Сделать поля \texttt{max\_speed}, \texttt{capacity}, \texttt{fuel\_tank}, \texttt{engine\_oil\_capacity}, \texttt{luggage\_spaces} по-настоящему приватными с помощью функции \texttt{private} (например, как атрибуты класса до \texttt{\_\_init\_\_}). Удалить их из инициализатора.  
    Проверки в сеттерах реализовать через вспомогательные методы, помеченные декоратором \texttt{@private}.  
    Учитывать, что методы с \texttt{@private} нельзя вызывать из методов, использующих \texttt{@property}, поэтому для этой версии использовать только классические геттеры и сеттеры (\texttt{get\_...}, \texttt{set\_...}).  
    Продемонстрировать, что попытка доступа извне (включая \texttt{mybus3.\_Bus\_\_max\_speed}) \textbf{не даёт результата}, а вызов приватного метода или чтение приватного поля вызывает ошибку доступа.
\end{enumerate}
Для всех трёх подходов создать по три экземпляра автобуса, установить значения полей с учётом всех ограничений и вывести текущие значения всех полей каждого экземпляра.
\item[2] Разработать класс \texttt{Train}, который будет описывать модель поезда. В классе должны быть следующие поля с доступом уровня \textbf{private} (только внутри класса):
\begin{itemize}
    \item \texttt{\_\_speed}: скорость движения поезда  
    \item \texttt{\_\_distance}: расстояние, которое поезд проехал  
    \item \texttt{\_\_max\_speed}: максимальная разрешённая скорость движения поезда  
    \item \texttt{\_\_passengers}: список пассажиров  
    \item \texttt{\_\_capacity}: максимальная вместимость пассажиров в поезде  
    \item \texttt{\_\_empty\_seats}: число свободных мест  
    \item \texttt{\_\_seats\_occupied}: число занятых мест в поезде  
    \item \texttt{\_\_fuel\_tank}: объём топливного бака (для дизельных поездов; для электрических — не применимо, но оставлено для единообразия)  
    \item \texttt{\_\_fuel}: количество топлива в литрах  
    \item \texttt{\_\_engine\_oil\_capacity}: объём картера масла двигателя (литры)  
    \item \texttt{\_\_engine\_oil}: количество моторного масла в литрах  
    \item \texttt{\_\_luggage\_spaces}: количество багажных мест  
    \item \texttt{\_\_luggage}: багаж поезда  
\end{itemize}
Уровень доступа к полям должен быть следующим:
\begin{itemize}
    \item \texttt{\_\_max\_speed}, \texttt{\_\_capacity}, \texttt{\_\_fuel\_tank}, \texttt{\_\_engine\_oil\_capacity}, \texttt{\_\_luggage\_spaces}: \textbf{только чтение} (через геттеры)  
    \item \texttt{\_\_speed}, \texttt{\_\_distance}, \texttt{\_\_passengers}, \texttt{\_\_empty\_seats}, \texttt{\_\_seats\_occupied}, \texttt{\_\_fuel}, \texttt{\_\_engine\_oil}, \texttt{\_\_luggage}: \textbf{чтение и запись} (через геттеры и сеттеры)
\end{itemize}
Требования к сеттерам:
\begin{itemize}
    \item Для полей \texttt{\_\_empty\_seats} и \texttt{\_\_seats\_occupied} в сеттерах необходимо проверять, что передаваемое значение не превышает \texttt{\_\_capacity} и неотрицательно.  
    \item Для поля \texttt{\_\_passengers} в сеттере необходимо проверять, что количество пассажиров (длина списка) не превышает \texttt{\_\_capacity}.  
    \item Для поля \texttt{\_\_speed} в сеттере необходимо проверять, что заданная скорость не превышает \texttt{\_\_max\_speed} и неотрицательна.  
    \item Для поля \texttt{\_\_luggage} в сеттере необходимо проверять, что количество единиц багажа не превышает \texttt{\_\_luggage\_spaces}.
    \item Для полей \texttt{\_\_fuel} и \texttt{\_\_engine\_oil} значения не должны превышать соответствующие ёмкости (\texttt{\_\_fuel\_tank} и \texttt{\_\_engine\_oil\_capacity}) и должны быть неотрицательными.
\end{itemize}
Реализовать метод вывода всех установленных через сеттеры значений закрытых полей экземпляра класса.
На основе этого класса реализовать три подхода к управлению доступом:
\begin{enumerate}
    \item \textbf{С использованием объекта \texttt{property}}:  
    Для каждого поля определить отдельные методы-геттеры и сеттеры (например, \texttt{get\_speed}, \texttt{set\_speed}), а затем создать свойство:  
    \begin{verbatim}
speed = property(get_speed, set_speed)
    \end{verbatim}  
    Этот код должен располагаться после определения соответствующих методов. Первый аргумент — геттер, второй — сеттер.  
    Продемонстрировать работу на трёх экземплярах класса: создать \texttt{mytrain1}, \texttt{mytrain2}, \texttt{mytrain3}, установить значения через свойства и вывести их.
    \item \textbf{С использованием декораторов \texttt{@property} и \texttt{@<имя>.setter}}:  
    Создать новую версию класса, в которой геттеры оформляются с декоратором \texttt{@property}, а сеттеры — с декоратором вида \texttt{@speed.setter}. Имена методов должны совпадать и не содержать префиксов \texttt{get\_}/\texttt{set\_}.  
    Пример:  
    \begin{verbatim}
@property
def speed(self):
    return self.__speed
@speed.setter
def speed(self, value):
    if 0 <= value <= self.__max_speed:
        self.__speed = value
    else:
        raise ValueError("Недопустимая скорость")
    \end{verbatim}  
    Продемонстрировать работу на трёх экземплярах и сделать выводы об оптимизации кода по сравнению с первым подходом.
    \item \textbf{С использованием модуля \texttt{accessify}}:  
    Установить модуль командой \texttt{pip install accessify} и импортировать:  
    \begin{verbatim}
from accessify import private, protected
    \end{verbatim}  
    Сделать поля \texttt{max\_speed}, \texttt{capacity}, \texttt{fuel\_tank}, \texttt{engine\_oil\_capacity}, \texttt{luggage\_spaces} по-настоящему приватными с помощью функции \texttt{private} (например, как атрибуты класса до \texttt{\_\_init\_\_}). Удалить их из инициализатора.  
    Проверки в сеттерах реализовать через вспомогательные методы, помеченные декоратором \texttt{@private}.  
    Учитывать, что методы с \texttt{@private} нельзя вызывать из методов, использующих \texttt{@property}, поэтому для этой версии использовать только классические геттеры и сеттеры (\texttt{get\_...}, \texttt{set\_...}).  
    Продемонстрировать, что попытка доступа извне (включая \texttt{mytrain3.\_Train\_\_max\_speed}) \textbf{не даёт результата}, а вызов приватного метода или чтение приватного поля вызывает ошибку доступа.
\end{enumerate}
Для всех трёх подходов создать по три экземпляра поезда, установить значения полей с учётом всех ограничений и вывести текущие значения всех полей каждого экземпляра.
\item[3] Разработать класс \texttt{Airplane}, который будет описывать модель самолёта. В классе должны быть следующие поля с доступом уровня \textbf{private} (только внутри класса):
\begin{itemize}
    \item \texttt{\_\_speed}: скорость движения самолёта  
    \item \texttt{\_\_distance}: расстояние, которое самолёт пролетел  
    \item \texttt{\_\_max\_speed}: максимальная разрешённая скорость движения самолёта  
    \item \texttt{\_\_passengers}: список пассажиров  
    \item \texttt{\_\_capacity}: максимальная вместимость пассажиров в самолёте  
    \item \texttt{\_\_empty\_seats}: число свободных мест  
    \item \texttt{\_\_seats\_occupied}: число занятых мест в самолёте  
    \item \texttt{\_\_fuel\_tank}: объём топливного бака  
    \item \texttt{\_\_fuel}: количество топлива в литрах  
    \item \texttt{\_\_engine\_oil\_capacity}: объём картера масла двигателя (литры)  
    \item \texttt{\_\_engine\_oil}: количество моторного масла в литрах  
    \item \texttt{\_\_luggage\_spaces}: количество багажных мест  
    \item \texttt{\_\_luggage}: багаж самолёта  
\end{itemize}
Уровень доступа к полям должен быть следующим:
\begin{itemize}
    \item \texttt{\_\_max\_speed}, \texttt{\_\_capacity}, \texttt{\_\_fuel\_tank}, \texttt{\_\_engine\_oil\_capacity}, \texttt{\_\_luggage\_spaces}: \textbf{только чтение} (через геттеры)  
    \item \texttt{\_\_speed}, \texttt{\_\_distance}, \texttt{\_\_passengers}, \texttt{\_\_empty\_seats}, \texttt{\_\_seats\_occupied}, \texttt{\_\_fuel}, \texttt{\_\_engine\_oil}, \texttt{\_\_luggage}: \textbf{чтение и запись} (через геттеры и сеттеры)
\end{itemize}
Требования к сеттерам:
\begin{itemize}
    \item Для полей \texttt{\_\_empty\_seats} и \texttt{\_\_seats\_occupied} в сеттерах необходимо проверять, что передаваемое значение не превышает \texttt{\_\_capacity} и неотрицательно.  
    \item Для поля \texttt{\_\_passengers} в сеттере необходимо проверять, что количество пассажиров (длина списка) не превышает \texttt{\_\_capacity}.  
    \item Для поля \texttt{\_\_speed} в сеттере необходимо проверять, что заданная скорость не превышает \texttt{\_\_max\_speed} и неотрицательна.  
    \item Для поля \texttt{\_\_luggage} в сеттере необходимо проверять, что количество единиц багажа не превышает \texttt{\_\_luggage\_spaces}.
    \item Для полей \texttt{\_\_fuel} и \texttt{\_\_engine\_oil} значения не должны превышать соответствующие ёмкости (\texttt{\_\_fuel\_tank} и \texttt{\_\_engine\_oil\_capacity}) и должны быть неотрицательными.
\end{itemize}
Реализовать метод вывода всех установленных через сеттеры значений закрытых полей экземпляра класса.
На основе этого класса реализовать три подхода к управлению доступом:
\begin{enumerate}
    \item \textbf{С использованием объекта \texttt{property}}:  
    Для каждого поля определить отдельные методы-геттеры и сеттеры (например, \texttt{get\_speed}, \texttt{set\_speed}), а затем создать свойство:  
    \begin{verbatim}
speed = property(get_speed, set_speed)
    \end{verbatim}  
    Этот код должен располагаться после определения соответствующих методов. Первый аргумент — геттер, второй — сеттер.  
    Продемонстрировать работу на трёх экземплярах класса: создать \texttt{myplane1}, \texttt{myplane2}, \texttt{myplane3}, установить значения через свойства и вывести их.
    \item \textbf{С использованием декораторов \texttt{@property} и \texttt{@<имя>.setter}}:  
    Создать новую версию класса, в которой геттеры оформляются с декоратором \texttt{@property}, а сеттеры — с декоратором вида \texttt{@speed.setter}. Имена методов должны совпадать и не содержать префиксов \texttt{get\_}/\texttt{set\_}.  
    Пример:  
    \begin{verbatim}
@property
def speed(self):
    return self.__speed
@speed.setter
def speed(self, value):
    if 0 <= value <= self.__max_speed:
        self.__speed = value
    else:
        raise ValueError("Недопустимая скорость")
    \end{verbatim}  
    Продемонстрировать работу на трёх экземплярах и сделать выводы об оптимизации кода по сравнению с первым подходом.
    \item \textbf{С использованием модуля \texttt{accessify}}:  
    Установить модуль командой \texttt{pip install accessify} и импортировать:  
    \begin{verbatim}
from accessify import private, protected
    \end{verbatim}  
    Сделать поля \texttt{max\_speed}, \texttt{capacity}, \texttt{fuel\_tank}, \texttt{engine\_oil\_capacity}, \texttt{luggage\_spaces} по-настоящему приватными с помощью функции \texttt{private} (например, как атрибуты класса до \texttt{\_\_init\_\_}). Удалить их из инициализатора.  
    Проверки в сеттерах реализовать через вспомогательные методы, помеченные декоратором \texttt{@private}.  
    Учитывать, что методы с \texttt{@private} нельзя вызывать из методов, использующих \texttt{@property}, поэтому для этой версии использовать только классические геттеры и сеттеры (\texttt{get\_...}, \texttt{set\_...}).  
    Продемонстрировать, что попытка доступа извне (включая \texttt{myplane3.\_Airplane\_\_max\_speed}) \textbf{не даёт результата}, а вызов приватного метода или чтение приватного поля вызывает ошибку доступа.
\end{enumerate}
Для всех трёх подходов создать по три экземпляра самолёта, установить значения полей с учётом всех ограничений и вывести текущие значения всех полей каждого экземпляра.
\item[4] Разработать класс \texttt{Ship}, который будет описывать модель корабля. В классе должны быть следующие поля с доступом уровня \textbf{private} (только внутри класса):
\begin{itemize}
    \item \texttt{\_\_speed}: скорость движения корабля  
    \item \texttt{\_\_distance}: расстояние, которое корабль прошёл  
    \item \texttt{\_\_max\_speed}: максимальная разрешённая скорость движения корабля  
    \item \texttt{\_\_passengers}: список пассажиров  
    \item \texttt{\_\_capacity}: максимальная вместимость пассажиров на корабле  
    \item \texttt{\_\_empty\_seats}: число свободных мест  
    \item \texttt{\_\_seats\_occupied}: число занятых мест на корабле  
    \item \texttt{\_\_fuel\_tank}: объём топливного бака  
    \item \texttt{\_\_fuel}: количество топлива в литрах  
    \item \texttt{\_\_engine\_oil\_capacity}: объём картера масла двигателя (литры)  
    \item \texttt{\_\_engine\_oil}: количество моторного масла в литрах  
    \item \texttt{\_\_luggage\_spaces}: количество багажных мест  
    \item \texttt{\_\_luggage}: багаж корабля  
\end{itemize}
Уровень доступа к полям должен быть следующим:
\begin{itemize}
    \item \texttt{\_\_max\_speed}, \texttt{\_\_capacity}, \texttt{\_\_fuel\_tank}, \texttt{\_\_engine\_oil\_capacity}, \texttt{\_\_luggage\_spaces}: \textbf{только чтение} (через геттеры)  
    \item \texttt{\_\_speed}, \texttt{\_\_distance}, \texttt{\_\_passengers}, \texttt{\_\_empty\_seats}, \texttt{\_\_seats\_occupied}, \texttt{\_\_fuel}, \texttt{\_\_engine\_oil}, \texttt{\_\_luggage}: \textbf{чтение и запись} (через геттеры и сеттеры)
\end{itemize}
Требования к сеттерам:
\begin{itemize}
    \item Для полей \texttt{\_\_empty\_seats} и \texttt{\_\_seats\_occupied} в сеттерах необходимо проверять, что передаваемое значение не превышает \texttt{\_\_capacity} и неотрицательно.  
    \item Для поля \texttt{\_\_passengers} в сеттере необходимо проверять, что количество пассажиров (длина списка) не превышает \texttt{\_\_capacity}.  
    \item Для поля \texttt{\_\_speed} в сеттере необходимо проверять, что заданная скорость не превышает \texttt{\_\_max\_speed} и неотрицательна.  
    \item Для поля \texttt{\_\_luggage} в сеттере необходимо проверять, что количество единиц багажа не превышает \texttt{\_\_luggage\_spaces}.
    \item Для полей \texttt{\_\_fuel} и \texttt{\_\_engine\_oil} значения не должны превышать соответствующие ёмкости (\texttt{\_\_fuel\_tank} и \texttt{\_\_engine\_oil\_capacity}) и должны быть неотрицательными.
\end{itemize}
Реализовать метод вывода всех установленных через сеттеры значений закрытых полей экземпляра класса.
На основе этого класса реализовать три подхода к управлению доступом:
\begin{enumerate}
    \item \textbf{С использованием объекта \texttt{property}}:  
    Для каждого поля определить отдельные методы-геттеры и сеттеры (например, \texttt{get\_speed}, \texttt{set\_speed}), а затем создать свойство:  
    \begin{verbatim}
speed = property(get_speed, set_speed)
    \end{verbatim}  
    Этот код должен располагаться после определения соответствующих методов. Первый аргумент — геттер, второй — сеттер.  
    Продемонстрировать работу на трёх экземплярах класса: создать \texttt{myship1}, \texttt{myship2}, \texttt{myship3}, установить значения через свойства и вывести их.
    \item \textbf{С использованием декораторов \texttt{@property} и \texttt{@<имя>.setter}}:  
    Создать новую версию класса, в которой геттеры оформляются с декоратором \texttt{@property}, а сеттеры — с декоратором вида \texttt{@speed.setter}. Имена методов должны совпадать и не содержать префиксов \texttt{get\_}/\texttt{set\_}.  
    Пример:  
    \begin{verbatim}
@property
def speed(self):
    return self.__speed
@speed.setter
def speed(self, value):
    if 0 <= value <= self.__max_speed:
        self.__speed = value
    else:
        raise ValueError("Недопустимая скорость")
    \end{verbatim}  
    Продемонстрировать работу на трёх экземплярах и сделать выводы об оптимизации кода по сравнению с первым подходом.
    \item \textbf{С использованием модуля \texttt{accessify}}:  
    Установить модуль командой \texttt{pip install accessify} и импортировать:  
    \begin{verbatim}
from accessify import private, protected
    \end{verbatim}  
    Сделать поля \texttt{max\_speed}, \texttt{capacity}, \texttt{fuel\_tank}, \texttt{engine\_oil\_capacity}, \texttt{luggage\_spaces} по-настоящему приватными с помощью функции \texttt{private} (например, как атрибуты класса до \texttt{\_\_init\_\_}). Удалить их из инициализатора.  
    Проверки в сеттерах реализовать через вспомогательные методы, помеченные декоратором \texttt{@private}.  
    Учитывать, что методы с \texttt{@private} нельзя вызывать из методов, использующих \texttt{@property}, поэтому для этой версии использовать только классические геттеры и сеттеры (\texttt{get\_...}, \texttt{set\_...}).  
    Продемонстрировать, что попытка доступа извне (включая \texttt{myship3.\_Ship\_\_max\_speed}) \textbf{не даёт результата}, а вызов приватного метода или чтение приватного поля вызывает ошибку доступа.
\end{enumerate}
Для всех трёх подходов создать по три экземпляра корабля, установить значения полей с учётом всех ограничений и вывести текущие значения всех полей каждого экземпляра.
\item[5] Разработать класс \texttt{Truck}, который будет описывать модель грузовика. В классе должны быть следующие поля с доступом уровня \textbf{private} (только внутри класса):
\begin{itemize}
    \item \texttt{\_\_speed}: скорость движения грузовика  
    \item \texttt{\_\_distance}: расстояние, которое грузовик проехал  
    \item \texttt{\_\_max\_speed}: максимальная разрешённая скорость движения грузовика  
    \item \texttt{\_\_passengers}: список пассажиров  
    \item \texttt{\_\_capacity}: максимальная вместимость пассажиров в грузовике  
    \item \texttt{\_\_empty\_seats}: число свободных мест  
    \item \texttt{\_\_seats\_occupied}: число занятых мест в грузовике  
    \item \texttt{\_\_fuel\_tank}: объём топливного бака  
    \item \texttt{\_\_fuel}: количество топлива в литрах  
    \item \texttt{\_\_engine\_oil\_capacity}: объём картера масла двигателя (литры)  
    \item \texttt{\_\_engine\_oil}: количество моторного масла в литрах  
    \item \texttt{\_\_luggage\_spaces}: количество багажных мест  
    \item \texttt{\_\_luggage}: багаж грузовика  
\end{itemize}
Уровень доступа к полям должен быть следующим:
\begin{itemize}
    \item \texttt{\_\_max\_speed}, \texttt{\_\_capacity}, \texttt{\_\_fuel\_tank}, \texttt{\_\_engine\_oil\_capacity}, \texttt{\_\_luggage\_spaces}: \textbf{только чтение} (через геттеры)  
    \item \texttt{\_\_speed}, \texttt{\_\_distance}, \texttt{\_\_passengers}, \texttt{\_\_empty\_seats}, \texttt{\_\_seats\_occupied}, \texttt{\_\_fuel}, \texttt{\_\_engine\_oil}, \texttt{\_\_luggage}: \textbf{чтение и запись} (через геттеры и сеттеры)
\end{itemize}
Требования к сеттерам:
\begin{itemize}
    \item Для полей \texttt{\_\_empty\_seats} и \texttt{\_\_seats\_occupied} в сеттерах необходимо проверять, что передаваемое значение не превышает \texttt{\_\_capacity} и неотрицательно.  
    \item Для поля \texttt{\_\_passengers} в сеттере необходимо проверять, что количество пассажиров (длина списка) не превышает \texttt{\_\_capacity}.  
    \item Для поля \texttt{\_\_speed} в сеттере необходимо проверять, что заданная скорость не превышает \texttt{\_\_max\_speed} и неотрицательна.  
    \item Для поля \texttt{\_\_luggage} в сеттере необходимо проверять, что количество единиц багажа не превышает \texttt{\_\_luggage\_spaces}.
    \item Для полей \texttt{\_\_fuel} и \texttt{\_\_engine\_oil} значения не должны превышать соответствующие ёмкости (\texttt{\_\_fuel\_tank} и \texttt{\_\_engine\_oil\_capacity}) и должны быть неотрицательными.
\end{itemize}
Реализовать метод вывода всех установленных через сеттеры значений закрытых полей экземпляра класса.
На основе этого класса реализовать три подхода к управлению доступом:
\begin{enumerate}
    \item \textbf{С использованием объекта \texttt{property}}:  
    Для каждого поля определить отдельные методы-геттеры и сеттеры (например, \texttt{get\_speed}, \texttt{set\_speed}), а затем создать свойство:  
    \begin{verbatim}
speed = property(get_speed, set_speed)
    \end{verbatim}  
    Этот код должен располагаться после определения соответствующих методов. Первый аргумент — геттер, второй — сеттер.  
    Продемонстрировать работу на трёх экземплярах класса: создать \texttt{mytruck1}, \texttt{mytruck2}, \texttt{mytruck3}, установить значения через свойства и вывести их.
    \item \textbf{С использованием декораторов \texttt{@property} и \texttt{@<имя>.setter}}:  
    Создать новую версию класса, в которой геттеры оформляются с декоратором \texttt{@property}, а сеттеры — с декоратором вида \texttt{@speed.setter}. Имена методов должны совпадать и не содержать префиксов \texttt{get\_}/\texttt{set\_}.  
    Пример:  
    \begin{verbatim}
@property
def speed(self):
    return self.__speed
@speed.setter
def speed(self, value):
    if 0 <= value <= self.__max_speed:
        self.__speed = value
    else:
        raise ValueError("Недопустимая скорость")
    \end{verbatim}  
    Продемонстрировать работу на трёх экземплярах и сделать выводы об оптимизации кода по сравнению с первым подходом.
    \item \textbf{С использованием модуля \texttt{accessify}}:  
    Установить модуль командой \texttt{pip install accessify} и импортировать:  
    \begin{verbatim}
from accessify import private, protected
    \end{verbatim}  
    Сделать поля \texttt{max\_speed}, \texttt{capacity}, \texttt{fuel\_tank}, \texttt{engine\_oil\_capacity}, \texttt{luggage\_spaces} по-настоящему приватными с помощью функции \texttt{private} (например, как атрибуты класса до \texttt{\_\_init\_\_}). Удалить их из инициализатора.  
    Проверки в сеттерах реализовать через вспомогательные методы, помеченные декоратором \texttt{@private}.  
    Учитывать, что методы с \texttt{@private} нельзя вызывать из методов, использующих \texttt{@property}, поэтому для этой версии использовать только классические геттеры и сеттеры (\texttt{get\_...}, \texttt{set\_...}).  
    Продемонстрировать, что попытка доступа извне (включая \texttt{mytruck3.\_Truck\_\_max\_speed}) \textbf{не даёт результата}, а вызов приватного метода или чтение приватного поля вызывает ошибку доступа.
\end{enumerate}
Для всех трёх подходов создать по три экземпляра грузовика, установить значения полей с учётом всех ограничений и вывести текущие значения всех полей каждого экземпляра.
\item[6] Разработать класс \texttt{Motorcycle}, который будет описывать модель мотоцикла. В классе должны быть следующие поля с доступом уровня \textbf{private} (только внутри класса):
\begin{itemize}
    \item \texttt{\_\_speed}: скорость движения мотоцикла  
    \item \texttt{\_\_distance}: расстояние, которое мотоцикл проехал  
    \item \texttt{\_\_max\_speed}: максимальная разрешённая скорость движения мотоцикла  
    \item \texttt{\_\_passengers}: список пассажиров  
    \item \texttt{\_\_capacity}: максимальная вместимость пассажиров на мотоцикле (обычно 1–2)  
    \item \texttt{\_\_empty\_seats}: число свободных мест  
    \item \texttt{\_\_seats\_occupied}: число занятых мест на мотоцикле  
    \item \texttt{\_\_fuel\_tank}: объём топливного бака  
    \item \texttt{\_\_fuel}: количество топлива в литрах  
    \item \texttt{\_\_engine\_oil\_capacity}: объём картера масла двигателя (литры)  
    \item \texttt{\_\_engine\_oil}: количество моторного масла в литрах  
    \item \texttt{\_\_luggage\_spaces}: количество багажных мест (обычно сумки/кофры)  
    \item \texttt{\_\_luggage}: багаж мотоцикла  
\end{itemize}
Уровень доступа к полям должен быть следующим:
\begin{itemize}
    \item \texttt{\_\_max\_speed}, \texttt{\_\_capacity}, \texttt{\_\_fuel\_tank}, \texttt{\_\_engine\_oil\_capacity}, \texttt{\_\_luggage\_spaces}: \textbf{только чтение} (через геттеры)  
    \item \texttt{\_\_speed}, \texttt{\_\_distance}, \texttt{\_\_passengers}, \texttt{\_\_empty\_seats}, \texttt{\_\_seats\_occupied}, \texttt{\_\_fuel}, \texttt{\_\_engine\_oil}, \texttt{\_\_luggage}: \textbf{чтение и запись} (через геттеры и сеттеры)
\end{itemize}
Требования к сеттерам:
\begin{itemize}
    \item Для полей \texttt{\_\_empty\_seats} и \texttt{\_\_seats\_occupied} в сеттерах необходимо проверять, что передаваемое значение не превышает \texttt{\_\_capacity} и неотрицательно.  
    \item Для поля \texttt{\_\_passengers} в сеттере необходимо проверять, что количество пассажиров (длина списка) не превышает \texttt{\_\_capacity}.  
    \item Для поля \texttt{\_\_speed} в сеттере необходимо проверять, что заданная скорость не превышает \texttt{\_\_max\_speed} и неотрицательна.  
    \item Для поля \texttt{\_\_luggage} в сеттере необходимо проверять, что количество единиц багажа не превышает \texttt{\_\_luggage\_spaces}.
    \item Для полей \texttt{\_\_fuel} и \texttt{\_\_engine\_oil} значения не должны превышать соответствующие ёмкости (\texttt{\_\_fuel\_tank} и \texttt{\_\_engine\_oil\_capacity}) и должны быть неотрицательными.
\end{itemize}
Реализовать метод вывода всех установленных через сеттеры значений закрытых полей экземпляра класса.
На основе этого класса реализовать три подхода к управлению доступом:
\begin{enumerate}
    \item \textbf{С использованием объекта \texttt{property}}:  
    Для каждого поля определить отдельные методы-геттеры и сеттеры (например, \texttt{get\_speed}, \texttt{set\_speed}), а затем создать свойство:  
    \begin{verbatim}
speed = property(get_speed, set_speed)
    \end{verbatim}  
    Этот код должен располагаться после определения соответствующих методов. Первый аргумент — геттер, второй — сеттер.  
    Продемонстрировать работу на трёх экземплярах класса: создать \texttt{mymoto1}, \texttt{mymoto2}, \texttt{mymoto3}, установить значения через свойства и вывести их.
    \item \textbf{С использованием декораторов \texttt{@property} и \texttt{@<имя>.setter}}:  
    Создать новую версию класса, в которой геттеры оформляются с декоратором \texttt{@property}, а сеттеры — с декоратором вида \texttt{@speed.setter}. Имена методов должны совпадать и не содержать префиксов \texttt{get\_}/\texttt{set\_}.  
    Пример:  
    \begin{verbatim}
@property
def speed(self):
    return self.__speed
@speed.setter
def speed(self, value):
    if 0 <= value <= self.__max_speed:
        self.__speed = value
    else:
        raise ValueError("Недопустимая скорость")
    \end{verbatim}  
    Продемонстрировать работу на трёх экземплярах и сделать выводы об оптимизации кода по сравнению с первым подходом.
    \item \textbf{С использованием модуля \texttt{accessify}}:  
    Установить модуль командой \texttt{pip install accessify} и импортировать:  
    \begin{verbatim}
from accessify import private, protected
    \end{verbatim}  
    Сделать поля \texttt{max\_speed}, \texttt{capacity}, \texttt{fuel\_tank}, \texttt{engine\_oil\_capacity}, \texttt{luggage\_spaces} по-настоящему приватными с помощью функции \texttt{private} (например, как атрибуты класса до \texttt{\_\_init\_\_}). Удалить их из инициализатора.  
    Проверки в сеттерах реализовать через вспомогательные методы, помеченные декоратором \texttt{@private}.  
    Учитывать, что методы с \texttt{@private} нельзя вызывать из методов, использующих \texttt{@property}, поэтому для этой версии использовать только классические геттеры и сеттеры (\texttt{get\_...}, \texttt{set\_...}).  
    Продемонстрировать, что попытка доступа извне (включая \texttt{mymoto3.\_Motorcycle\_\_max\_speed}) \textbf{не даёт результата}, а вызов приватного метода или чтение приватного поля вызывает ошибку доступа.
\end{enumerate}
Для всех трёх подходов создать по три экземпляра мотоцикла, установить значения полей с учётом всех ограничений и вывести текущие значения всех полей каждого экземпляра.
\item[7] Разработать класс \texttt{Bicycle}, который будет описывать модель велосипеда. В классе должны быть следующие поля с доступом уровня \textbf{private} (только внутри класса):
\begin{itemize}
    \item \texttt{\_\_speed}: скорость движения велосипеда  
    \item \texttt{\_\_distance}: расстояние, которое велосипед проехал  
    \item \texttt{\_\_max\_speed}: максимальная разрешённая скорость движения велосипеда  
    \item \texttt{\_\_passengers}: список пассажиров  
    \item \texttt{\_\_capacity}: максимальная вместимость пассажиров на велосипеде (обычно 1, иногда 2)  
    \item \texttt{\_\_empty\_seats}: число свободных мест  
    \item \texttt{\_\_seats\_occupied}: число занятых мест на велосипеде  
    \item \texttt{\_\_fuel\_tank}: объём топливного бака (не применимо к обычному велосипеду; оставлено для единообразия)  
    \item \texttt{\_\_fuel}: количество топлива в литрах (обычно 0 для обычного велосипеда)  
    \item \texttt{\_\_engine\_oil\_capacity}: объём картера масла двигателя (не применимо; оставлено для единообразия)  
    \item \texttt{\_\_engine\_oil}: количество моторного масла в литрах (обычно 0)  
    \item \texttt{\_\_luggage\_spaces}: количество багажных мест  
    \item \texttt{\_\_luggage}: багаж велосипеда  
\end{itemize}
Уровень доступа к полям должен быть следующим:
\begin{itemize}
    \item \texttt{\_\_max\_speed}, \texttt{\_\_capacity}, \texttt{\_\_fuel\_tank}, \texttt{\_\_engine\_oil\_capacity}, \texttt{\_\_luggage\_spaces}: \textbf{только чтение} (через геттеры)  
    \item \texttt{\_\_speed}, \texttt{\_\_distance}, \texttt{\_\_passengers}, \texttt{\_\_empty\_seats}, \texttt{\_\_seats\_occupied}, \texttt{\_\_fuel}, \texttt{\_\_engine\_oil}, \texttt{\_\_luggage}: \textbf{чтение и запись} (через геттеры и сеттеры)
\end{itemize}
Требования к сеттерам:
\begin{itemize}
    \item Для полей \texttt{\_\_empty\_seats} и \texttt{\_\_seats\_occupied} в сеттерах необходимо проверять, что передаваемое значение не превышает \texttt{\_\_capacity} и неотрицательно.  
    \item Для поля \texttt{\_\_passengers} в сеттере необходимо проверять, что количество пассажиров (длина списка) не превышает \texttt{\_\_capacity}.  
    \item Для поля \texttt{\_\_speed} в сеттере необходимо проверять, что заданная скорость не превышает \texttt{\_\_max\_speed} и неотрицательна.  
    \item Для поля \texttt{\_\_luggage} в сеттере необходимо проверять, что количество единиц багажа не превышает \texttt{\_\_luggage\_spaces}.
    \item Для полей \texttt{\_\_fuel} и \texttt{\_\_engine\_oil} значения не должны превышать соответствующие ёмкости (\texttt{\_\_fuel\_tank} и \texttt{\_\_engine\_oil\_capacity}) и должны быть неотрицательными (обычно 0).
\end{itemize}
Реализовать метод вывода всех установленных через сеттеры значений закрытых полей экземпляра класса.
На основе этого класса реализовать три подхода к управлению доступом:
\begin{enumerate}
    \item \textbf{С использованием объекта \texttt{property}}:  
    Для каждого поля определить отдельные методы-геттеры и сеттеры (например, \texttt{get\_speed}, \texttt{set\_speed}), а затем создать свойство:  
    \begin{verbatim}
speed = property(get_speed, set_speed)
    \end{verbatim}  
    Этот код должен располагаться после определения соответствующих методов. Первый аргумент — геттер, второй — сеттер.  
    Продемонстрировать работу на трёх экземплярах класса: создать \texttt{mybike1}, \texttt{mybike2}, \texttt{mybike3}, установить значения через свойства и вывести их.
    \item \textbf{С использованием декораторов \texttt{@property} и \texttt{@<имя>.setter}}:  
    Создать новую версию класса, в которой геттеры оформляются с декоратором \texttt{@property}, а сеттеры — с декоратором вида \texttt{@speed.setter}. Имена методов должны совпадать и не содержать префиксов \texttt{get\_}/\texttt{set\_}.  
    Пример:  
    \begin{verbatim}
@property
def speed(self):
    return self.__speed
@speed.setter
def speed(self, value):
    if 0 <= value <= self.__max_speed:
        self.__speed = value
    else:
        raise ValueError("Недопустимая скорость")
    \end{verbatim}  
    Продемонстрировать работу на трёх экземплярах и сделать выводы об оптимизации кода по сравнению с первым подходом.
    \item \textbf{С использованием модуля \texttt{accessify}}:  
    Установить модуль командой \texttt{pip install accessify} и импортировать:  
    \begin{verbatim}
from accessify import private, protected
    \end{verbatim}  
    Сделать поля \texttt{max\_speed}, \texttt{capacity}, \texttt{fuel\_tank}, \texttt{engine\_oil\_capacity}, \texttt{luggage\_spaces} по-настоящему приватными с помощью функции \texttt{private} (например, как атрибуты класса до \texttt{\_\_init\_\_}). Удалить их из инициализатора.  
    Проверки в сеттерах реализовать через вспомогательные методы, помеченные декоратором \texttt{@private}.  
    Учитывать, что методы с \texttt{@private} нельзя вызывать из методов, использующих \texttt{@property}, поэтому для этой версии использовать только классические геттеры и сеттеры (\texttt{get\_...}, \texttt{set\_...}).  
    Продемонстрировать, что попытка доступа извне (включая \texttt{mybike3.\_Bicycle\_\_max\_speed}) \textbf{не даёт результата}, а вызов приватного метода или чтение приватного поля вызывает ошибку доступа.
\end{enumerate}
Для всех трёх подходов создать по три экземпляра велосипеда, установить значения полей с учётом всех ограничений и вывести текущие значения всех полей каждого экземпляра.
\item[8] Разработать класс \texttt{Helicopter}, который будет описывать модель вертолёта. В классе должны быть следующие поля с доступом уровня \textbf{private} (только внутри класса):
\begin{itemize}
    \item \texttt{\_\_speed}: скорость движения вертолёта  
    \item \texttt{\_\_distance}: расстояние, которое вертолёт пролетел  
    \item \texttt{\_\_max\_speed}: максимальная разрешённая скорость движения вертолёта  
    \item \texttt{\_\_passengers}: список пассажиров  
    \item \texttt{\_\_capacity}: максимальная вместимость пассажиров в вертолёте  
    \item \texttt{\_\_empty\_seats}: число свободных мест  
    \item \texttt{\_\_seats\_occupied}: число занятых мест в вертолёте  
    \item \texttt{\_\_fuel\_tank}: объём топливного бака  
    \item \texttt{\_\_fuel}: количество топлива в литрах  
    \item \texttt{\_\_engine\_oil\_capacity}: объём картера масла двигателя (литры)  
    \item \texttt{\_\_engine\_oil}: количество моторного масла в литрах  
    \item \texttt{\_\_luggage\_spaces}: количество багажных мест  
    \item \texttt{\_\_luggage}: багаж вертолёта  
\end{itemize}
Уровень доступа к полям должен быть следующим:
\begin{itemize}
    \item \texttt{\_\_max\_speed}, \texttt{\_\_capacity}, \texttt{\_\_fuel\_tank}, \texttt{\_\_engine\_oil\_capacity}, \texttt{\_\_luggage\_spaces}: \textbf{только чтение} (через геттеры)  
    \item \texttt{\_\_speed}, \texttt{\_\_distance}, \texttt{\_\_passengers}, \texttt{\_\_empty\_seats}, \texttt{\_\_seats\_occupied}, \texttt{\_\_fuel}, \texttt{\_\_engine\_oil}, \texttt{\_\_luggage}: \textbf{чтение и запись} (через геттеры и сеттеры)
\end{itemize}
Требования к сеттерам:
\begin{itemize}
    \item Для полей \texttt{\_\_empty\_seats} и \texttt{\_\_seats\_occupied} в сеттерах необходимо проверять, что передаваемое значение не превышает \texttt{\_\_capacity} и неотрицательно.  
    \item Для поля \texttt{\_\_passengers} в сеттере необходимо проверять, что количество пассажиров (длина списка) не превышает \texttt{\_\_capacity}.  
    \item Для поля \texttt{\_\_speed} в сеттере необходимо проверять, что заданная скорость не превышает \texttt{\_\_max\_speed} и неотрицательна.  
    \item Для поля \texttt{\_\_luggage} в сеттере необходимо проверять, что количество единиц багажа не превышает \texttt{\_\_luggage\_spaces}.
    \item Для полей \texttt{\_\_fuel} и \texttt{\_\_engine\_oil} значения не должны превышать соответствующие ёмкости (\texttt{\_\_fuel\_tank} и \texttt{\_\_engine\_oil\_capacity}) и должны быть неотрицательными.
\end{itemize}
Реализовать метод вывода всех установленных через сеттеры значений закрытых полей экземпляра класса.
На основе этого класса реализовать три подхода к управлению доступом:
\begin{enumerate}
    \item \textbf{С использованием объекта \texttt{property}}:  
    Для каждого поля определить отдельные методы-геттеры и сеттеры (например, \texttt{get\_speed}, \texttt{set\_speed}), а затем создать свойство:  
    \begin{verbatim}
speed = property(get_speed, set_speed)
    \end{verbatim}  
    Этот код должен располагаться после определения соответствующих методов. Первый аргумент — геттер, второй — сеттер.  
    Продемонстрировать работу на трёх экземплярах класса: создать \texttt{myheli1}, \texttt{myheli2}, \texttt{myheli3}, установить значения через свойства и вывести их.
    \item \textbf{С использованием декораторов \texttt{@property} и \texttt{@<имя>.setter}}:  
    Создать новую версию класса, в которой геттеры оформляются с декоратором \texttt{@property}, а сеттеры — с декоратором вида \texttt{@speed.setter}. Имена методов должны совпадать и не содержать префиксов \texttt{get\_}/\texttt{set\_}.  
    Пример:  
    \begin{verbatim}
@property
def speed(self):
    return self.__speed
@speed.setter
def speed(self, value):
    if 0 <= value <= self.__max_speed:
        self.__speed = value
    else:
        raise ValueError("Недопустимая скорость")
    \end{verbatim}  
    Продемонстрировать работу на трёх экземплярах и сделать выводы об оптимизации кода по сравнению с первым подходом.
    \item \textbf{С использованием модуля \texttt{accessify}}:  
    Установить модуль командой \texttt{pip install accessify} и импортировать:  
    \begin{verbatim}
from accessify import private, protected
    \end{verbatim}  
    Сделать поля \texttt{max\_speed}, \texttt{capacity}, \texttt{fuel\_tank}, \texttt{engine\_oil\_capacity}, \texttt{luggage\_spaces} по-настоящему приватными с помощью функции \texttt{private} (например, как атрибуты класса до \texttt{\_\_init\_\_}). Удалить их из инициализатора.  
    Проверки в сеттерах реализовать через вспомогательные методы, помеченные декоратором \texttt{@private}.  
    Учитывать, что методы с \texttt{@private} нельзя вызывать из методов, использующих \texttt{@property}, поэтому для этой версии использовать только классические геттеры и сеттеры (\texttt{get\_...}, \texttt{set\_...}).  
    Продемонстрировать, что попытка доступа извне (включая \texttt{myheli3.\_Helicopter\_\_max\_speed}) \textbf{не даёт результата}, а вызов приватного метода или чтение приватного поля вызывает ошибку доступа.
\end{enumerate}
Для всех трёх подходов создать по три экземпляра вертолёта, установить значения полей с учётом всех ограничений и вывести текущие значения всех полей каждого экземпляра.
\item[9] Разработать класс \texttt{Submarine}, который будет описывать модель подводной лодки. В классе должны быть следующие поля с доступом уровня \textbf{private} (только внутри класса):
\begin{itemize}
    \item \texttt{\_\_speed}: скорость движения подводной лодки  
    \item \texttt{\_\_distance}: расстояние, которое подводная лодка прошла  
    \item \texttt{\_\_max\_speed}: максимальная разрешённая скорость движения подводной лодки  
    \item \texttt{\_\_passengers}: список пассажиров (обычно экипаж и, возможно, пассажиры)  
    \item \texttt{\_\_capacity}: максимальная вместимость пассажиров в подводной лодке  
    \item \texttt{\_\_empty\_seats}: число свободных мест  
    \item \texttt{\_\_seats\_occupied}: число занятых мест в подводной лодке  
    \item \texttt{\_\_fuel\_tank}: объём топливного бака (для дизель-электрических; для атомных — не применимо, но оставлено для единообразия)  
    \item \texttt{\_\_fuel}: количество топлива в литрах  
    \item \texttt{\_\_engine\_oil\_capacity}: объём картера масла двигателя (литры)  
    \item \texttt{\_\_engine\_oil}: количество моторного масла в литрах  
    \item \texttt{\_\_luggage\_spaces}: количество багажных мест  
    \item \texttt{\_\_luggage}: багаж подводной лодки  
\end{itemize}
Уровень доступа к полям должен быть следующим:
\begin{itemize}
    \item \texttt{\_\_max\_speed}, \texttt{\_\_capacity}, \texttt{\_\_fuel\_tank}, \texttt{\_\_engine\_oil\_capacity}, \texttt{\_\_luggage\_spaces}: \textbf{только чтение} (через геттеры)  
    \item \texttt{\_\_speed}, \texttt{\_\_distance}, \texttt{\_\_passengers}, \texttt{\_\_empty\_seats}, \texttt{\_\_seats\_occupied}, \texttt{\_\_fuel}, \texttt{\_\_engine\_oil}, \texttt{\_\_luggage}: \textbf{чтение и запись} (через геттеры и сеттеры)
\end{itemize}
Требования к сеттерам:
\begin{itemize}
    \item Для полей \texttt{\_\_empty\_seats} и \texttt{\_\_seats\_occupied} в сеттерах необходимо проверять, что передаваемое значение не превышает \texttt{\_\_capacity} и неотрицательно.  
    \item Для поля \texttt{\_\_passengers} в сеттере необходимо проверять, что количество пассажиров (длина списка) не превышает \texttt{\_\_capacity}.  
    \item Для поля \texttt{\_\_speed} в сеттере необходимо проверять, что заданная скорость не превышает \texttt{\_\_max\_speed} и неотрицательна.  
    \item Для поля \texttt{\_\_luggage} в сеттере необходимо проверять, что количество единиц багажа не превышает \texttt{\_\_luggage\_spaces}.
    \item Для полей \texttt{\_\_fuel} и \texttt{\_\_engine\_oil} значения не должны превышать соответствующие ёмкости (\texttt{\_\_fuel\_tank} и \texttt{\_\_engine\_oil\_capacity}) и должны быть неотрицательными.
\end{itemize}
Реализовать метод вывода всех установленных через сеттеры значений закрытых полей экземпляра класса.
На основе этого класса реализовать три подхода к управлению доступом:
\begin{enumerate}
    \item \textbf{С использованием объекта \texttt{property}}:  
    Для каждого поля определить отдельные методы-геттеры и сеттеры (например, \texttt{get\_speed}, \texttt{set\_speed}), а затем создать свойство:  
    \begin{verbatim}
speed = property(get_speed, set_speed)
    \end{verbatim}  
    Этот код должен располагаться после определения соответствующих методов. Первый аргумент — геттер, второй — сеттер.  
    Продемонстрировать работу на трёх экземплярах класса: создать \texttt{mysub1}, \texttt{mysub2}, \texttt{mysub3}, установить значения через свойства и вывести их.
    \item \textbf{С использованием декораторов \texttt{@property} и \texttt{@<имя>.setter}}:  
    Создать новую версию класса, в которой геттеры оформляются с декоратором \texttt{@property}, а сеттеры — с декоратором вида \texttt{@speed.setter}. Имена методов должны совпадать и не содержать префиксов \texttt{get\_}/\texttt{set\_}.  
    Пример:  
    \begin{verbatim}
@property
def speed(self):
    return self.__speed
@speed.setter
def speed(self, value):
    if 0 <= value <= self.__max_speed:
        self.__speed = value
    else:
        raise ValueError("Недопустимая скорость")
    \end{verbatim}  
    Продемонстрировать работу на трёх экземплярах и сделать выводы об оптимизации кода по сравнению с первым подходом.
    \item \textbf{С использованием модуля \texttt{accessify}}:  
    Установить модуль командой \texttt{pip install accessify} и импортировать:  
    \begin{verbatim}
from accessify import private, protected
    \end{verbatim}  
    Сделать поля \texttt{max\_speed}, \texttt{capacity}, \texttt{fuel\_tank}, \texttt{engine\_oil\_capacity}, \texttt{luggage\_spaces} по-настоящему приватными с помощью функции \texttt{private} (например, как атрибуты класса до \texttt{\_\_init\_\_}). Удалить их из инициализатора.  
    Проверки в сеттерах реализовать через вспомогательные методы, помеченные декоратором \texttt{@private}.  
    Учитывать, что методы с \texttt{@private} нельзя вызывать из методов, использующих \texttt{@property}, поэтому для этой версии использовать только классические геттеры и сеттеры (\texttt{get\_...}, \texttt{set\_...}).  
    Продемонстрировать, что попытка доступа извне (включая \texttt{mysub3.\_Submarine\_\_max\_speed}) \textbf{не даёт результата}, а вызов приватного метода или чтение приватного поля вызывает ошибку доступа.
\end{enumerate}
Для всех трёх подходов создать по три экземпляра подводной лодки, установить значения полей с учётом всех ограничений и вывести текущие значения всех полей каждого экземпляра.
\item[10] Разработать класс \texttt{Spaceship}, который будет описывать модель космического корабля. В классе должны быть следующие поля с доступом уровня \textbf{private} (только внутри класса):
\begin{itemize}
    \item \texttt{\_\_speed}: скорость движения космического корабля  
    \item \texttt{\_\_distance}: расстояние, которое космический корабль пролетел  
    \item \texttt{\_\_max\_speed}: максимальная разрешённая скорость движения космического корабля  
    \item \texttt{\_\_passengers}: список пассажиров  
    \item \texttt{\_\_capacity}: максимальная вместимость пассажиров в космическом корабле  
    \item \texttt{\_\_empty\_seats}: число свободных мест  
    \item \texttt{\_\_seats\_occupied}: число занятых мест в космическом корабле  
    \item \texttt{\_\_fuel\_tank}: объём топливного бака (для ракетного топлива)  
    \item \texttt{\_\_fuel}: количество топлива в литрах (или в условных единицах)  
    \item \texttt{\_\_engine\_oil\_capacity}: объём картера масла двигателя (не применимо к большинству космических двигателей; оставлено для единообразия)  
    \item \texttt{\_\_engine\_oil}: количество моторного масла в литрах (обычно 0)  
    \item \texttt{\_\_luggage\_spaces}: количество багажных мест  
    \item \texttt{\_\_luggage}: багаж космического корабля  
\end{itemize}
Уровень доступа к полям должен быть следующим:
\begin{itemize}
    \item \texttt{\_\_max\_speed}, \texttt{\_\_capacity}, \texttt{\_\_fuel\_tank}, \texttt{\_\_engine\_oil\_capacity}, \texttt{\_\_luggage\_spaces}: \textbf{только чтение} (через геттеры)  
    \item \texttt{\_\_speed}, \texttt{\_\_distance}, \texttt{\_\_passengers}, \texttt{\_\_empty\_seats}, \texttt{\_\_seats\_occupied}, \texttt{\_\_fuel}, \texttt{\_\_engine\_oil}, \texttt{\_\_luggage}: \textbf{чтение и запись} (через геттеры и сеттеры)
\end{itemize}
Требования к сеттерам:
\begin{itemize}
    \item Для полей \texttt{\_\_empty\_seats} и \texttt{\_\_seats\_occupied} в сеттерах необходимо проверять, что передаваемое значение не превышает \texttt{\_\_capacity} и неотрицательно.  
    \item Для поля \texttt{\_\_passengers} в сеттере необходимо проверять, что количество пассажиров (длина списка) не превышает \texttt{\_\_capacity}.  
    \item Для поля \texttt{\_\_speed} в сеттере необходимо проверять, что заданная скорость не превышает \texttt{\_\_max\_speed} и неотрицательна.  
    \item Для поля \texttt{\_\_luggage} в сеттере необходимо проверять, что количество единиц багажа не превышает \texttt{\_\_luggage\_spaces}.
    \item Для полей \texttt{\_\_fuel} и \texttt{\_\_engine\_oil} значения не должны превышать соответствующие ёмкости (\texttt{\_\_fuel\_tank} и \texttt{\_\_engine\_oil\_capacity}) и должны быть неотрицательными.
\end{itemize}
Реализовать метод вывода всех установленных через сеттеры значений закрытых полей экземпляра класса.
На основе этого класса реализовать три подхода к управлению доступом:
\begin{enumerate}
    \item \textbf{С использованием объекта \texttt{property}}:  
    Для каждого поля определить отдельные методы-геттеры и сеттеры (например, \texttt{get\_speed}, \texttt{set\_speed}), а затем создать свойство:  
    \begin{verbatim}
speed = property(get_speed, set_speed)
    \end{verbatim}  
    Этот код должен располагаться после определения соответствующих методов. Первый аргумент — геттер, второй — сеттер.  
    Продемонстрировать работу на трёх экземплярах класса: создать \texttt{myspace1}, \texttt{myspace2}, \texttt{myspace3}, установить значения через свойства и вывести их.
    \item \textbf{С использованием декораторов \texttt{@property} и \texttt{@<имя>.setter}}:  
    Создать новую версию класса, в которой геттеры оформляются с декоратором \texttt{@property}, а сеттеры — с декоратором вида \texttt{@speed.setter}. Имена методов должны совпадать и не содержать префиксов \texttt{get\_}/\texttt{set\_}.  
    Пример:  
    \begin{verbatim}
@property
def speed(self):
    return self.__speed
@speed.setter
def speed(self, value):
    if 0 <= value <= self.__max_speed:
        self.__speed = value
    else:
        raise ValueError("Недопустимая скорость")
    \end{verbatim}  
    Продемонстрировать работу на трёх экземплярах и сделать выводы об оптимизации кода по сравнению с первым подходом.
    \item \textbf{С использованием модуля \texttt{accessify}}:  
    Установить модуль командой \texttt{pip install accessify} и импортировать:  
    \begin{verbatim}
from accessify import private, protected
    \end{verbatim}  
    Сделать поля \texttt{max\_speed}, \texttt{capacity}, \texttt{fuel\_tank}, \texttt{engine\_oil\_capacity}, \texttt{luggage\_spaces} по-настоящему приватными с помощью функции \texttt{private} (например, как атрибуты класса до \texttt{\_\_init\_\_}). Удалить их из инициализатора.  
    Проверки в сеттерах реализовать через вспомогательные методы, помеченные декоратором \texttt{@private}.  
    Учитывать, что методы с \texttt{@private} нельзя вызывать из методов, использующих \texttt{@property}, поэтому для этой версии использовать только классические геттеры и сеттеры (\texttt{get\_...}, \texttt{set\_...}).  
    Продемонстрировать, что попытка доступа извне (включая \texttt{myspace3.\_Spaceship\_\_max\_speed}) \textbf{не даёт результата}, а вызов приватного метода или чтение приватного поля вызывает ошибку доступа.
\end{enumerate}
Для всех трёх подходов создать по три экземпляра космического корабля, установить значения полей с учётом всех ограничений и вывести текущие значения всех полей каждого экземпляра.
\item[11] Разработать класс \texttt{Drone}, который будет описывать модель дрона. В классе должны быть следующие поля с доступом уровня \textbf{private} (только внутри класса):
\begin{itemize}
    \item \texttt{\_\_speed}: скорость движения дрона  
    \item \texttt{\_\_distance}: расстояние, которое дрон пролетел  
    \item \texttt{\_\_max\_speed}: максимальная разрешённая скорость движения дрона  
    \item \texttt{\_\_passengers}: список пассажиров (обычно пустой; оставлено для единообразия)  
    \item \texttt{\_\_capacity}: максимальная вместимость пассажиров на дроне (обычно 0)  
    \item \texttt{\_\_empty\_seats}: число свободных мест (обычно 0)  
    \item \texttt{\_\_seats\_occupied}: число занятых мест на дроне (обычно 0)  
    \item \texttt{\_\_fuel\_tank}: объём топливного бака (для топливных дронов; для электрических — не применимо)  
    \item \texttt{\_\_fuel}: количество топлива в литрах  
    \item \texttt{\_\_engine\_oil\_capacity}: объём картера масла двигателя (обычно 0)  
    \item \texttt{\_\_engine\_oil}: количество моторного масла в литрах (обычно 0)  
    \item \texttt{\_\_luggage\_spaces}: количество багажных мест (для грузовых дронов)  
    \item \texttt{\_\_luggage}: багаж дрона  
\end{itemize}
Уровень доступа к полям должен быть следующим:
\begin{itemize}
    \item \texttt{\_\_max\_speed}, \texttt{\_\_capacity}, \texttt{\_\_fuel\_tank}, \texttt{\_\_engine\_oil\_capacity}, \texttt{\_\_luggage\_spaces}: \textbf{только чтение} (через геттеры)  
    \item \texttt{\_\_speed}, \texttt{\_\_distance}, \texttt{\_\_passengers}, \texttt{\_\_empty\_seats}, \texttt{\_\_seats\_occupied}, \texttt{\_\_fuel}, \texttt{\_\_engine\_oil}, \texttt{\_\_luggage}: \textbf{чтение и запись} (через геттеры и сеттеры)
\end{itemize}
Требования к сеттерам:
\begin{itemize}
    \item Для полей \texttt{\_\_empty\_seats} и \texttt{\_\_seats\_occupied} в сеттерах необходимо проверять, что передаваемое значение не превышает \texttt{\_\_capacity} и неотрицательно.  
    \item Для поля \texttt{\_\_passengers} в сеттере необходимо проверять, что количество пассажиров (длина списка) не превышает \texttt{\_\_capacity}.  
    \item Для поля \texttt{\_\_speed} в сеттере необходимо проверять, что заданная скорость не превышает \texttt{\_\_max\_speed} и неотрицательна.  
    \item Для поля \texttt{\_\_luggage} в сеттере необходимо проверять, что количество единиц багажа не превышает \texttt{\_\_luggage\_spaces}.
    \item Для полей \texttt{\_\_fuel} и \texttt{\_\_engine\_oil} значения не должны превышать соответствующие ёмкости (\texttt{\_\_fuel\_tank} и \texttt{\_\_engine\_oil\_capacity}) и должны быть неотрицательными.
\end{itemize}
Реализовать метод вывода всех установленных через сеттеры значений закрытых полей экземпляра класса.
На основе этого класса реализовать три подхода к управлению доступом:
\begin{enumerate}
    \item \textbf{С использованием объекта \texttt{property}}:  
    Для каждого поля определить отдельные методы-геттеры и сеттеры (например, \texttt{get\_speed}, \texttt{set\_speed}), а затем создать свойство:  
    \begin{verbatim}
speed = property(get_speed, set_speed)
    \end{verbatim}  
    Этот код должен располагаться после определения соответствующих методов. Первый аргумент — геттер, второй — сеттер.  
    Продемонстрировать работу на трёх экземплярах класса: создать \texttt{mydrone1}, \texttt{mydrone2}, \texttt{mydrone3}, установить значения через свойства и вывести их.
    \item \textbf{С использованием декораторов \texttt{@property} и \texttt{@<имя>.setter}}:  
    Создать новую версию класса, в которой геттеры оформляются с декоратором \texttt{@property}, а сеттеры — с декоратором вида \texttt{@speed.setter}. Имена методов должны совпадать и не содержать префиксов \texttt{get\_}/\texttt{set\_}.  
    Пример:  
    \begin{verbatim}
@property
def speed(self):
    return self.__speed
@speed.setter
def speed(self, value):
    if 0 <= value <= self.__max_speed:
        self.__speed = value
    else:
        raise ValueError("Недопустимая скорость")
    \end{verbatim}  
    Продемонстрировать работу на трёх экземплярах и сделать выводы об оптимизации кода по сравнению с первым подходом.
    \item \textbf{С использованием модуля \texttt{accessify}}:  
    Установить модуль командой \texttt{pip install accessify} и импортировать:  
    \begin{verbatim}
from accessify import private, protected
    \end{verbatim}  
    Сделать поля \texttt{max\_speed}, \texttt{capacity}, \texttt{fuel\_tank}, \texttt{engine\_oil\_capacity}, \texttt{luggage\_spaces} по-настоящему приватными с помощью функции \texttt{private} (например, как атрибуты класса до \texttt{\_\_init\_\_}). Удалить их из инициализатора.  
    Проверки в сеттерах реализовать через вспомогательные методы, помеченные декоратором \texttt{@private}.  
    Учитывать, что методы с \texttt{@private} нельзя вызывать из методов, использующих \texttt{@property}, поэтому для этой версии использовать только классические геттеры и сеттеры (\texttt{get\_...}, \texttt{set\_...}).  
    Продемонстрировать, что попытка доступа извне (включая \texttt{mydrone3.\_Drone\_\_max\_speed}) \textbf{не даёт результата}, а вызов приватного метода или чтение приватного поля вызывает ошибку доступа.
\end{enumerate}
Для всех трёх подходов создать по три экземпляра дрона, установить значения полей с учётом всех ограничений и вывести текущие значения всех полей каждого экземпляра.
\item[12] Разработать класс \texttt{Scooter}, который будет описывать модель скутера. В классе должны быть следующие поля с доступом уровня \textbf{private} (только внутри класса):
\begin{itemize}
    \item \texttt{\_\_speed}: скорость движения скутера  
    \item \texttt{\_\_distance}: расстояние, которое скутер проехал  
    \item \texttt{\_\_max\_speed}: максимальная разрешённая скорость движения скутера  
    \item \texttt{\_\_passengers}: список пассажиров  
    \item \texttt{\_\_capacity}: максимальная вместимость пассажиров на скутере (обычно 1–2)  
    \item \texttt{\_\_empty\_seats}: число свободных мест  
    \item \texttt{\_\_seats\_occupied}: число занятых мест на скутере  
    \item \texttt{\_\_fuel\_tank}: объём топливного бака  
    \item \texttt{\_\_fuel}: количество топлива в литрах  
    \item \texttt{\_\_engine\_oil\_capacity}: объём картера масла двигателя (литры)  
    \item \texttt{\_\_engine\_oil}: количество моторного масла в литрах  
    \item \texttt{\_\_luggage\_spaces}: количество багажных мест  
    \item \texttt{\_\_luggage}: багаж скутера  
\end{itemize}
Уровень доступа к полям должен быть следующим:
\begin{itemize}
    \item \texttt{\_\_max\_speed}, \texttt{\_\_capacity}, \texttt{\_\_fuel\_tank}, \texttt{\_\_engine\_oil\_capacity}, \texttt{\_\_luggage\_spaces}: \textbf{только чтение} (через геттеры)  
    \item \texttt{\_\_speed}, \texttt{\_\_distance}, \texttt{\_\_passengers}, \texttt{\_\_empty\_seats}, \texttt{\_\_seats\_occupied}, \texttt{\_\_fuel}, \texttt{\_\_engine\_oil}, \texttt{\_\_luggage}: \textbf{чтение и запись} (через геттеры и сеттеры)
\end{itemize}
Требования к сеттерам:
\begin{itemize}
    \item Для полей \texttt{\_\_empty\_seats} и \texttt{\_\_seats\_occupied} в сеттерах необходимо проверять, что передаваемое значение не превышает \texttt{\_\_capacity} и неотрицательно.  
    \item Для поля \texttt{\_\_passengers} в сеттере необходимо проверять, что количество пассажиров (длина списка) не превышает \texttt{\_\_capacity}.  
    \item Для поля \texttt{\_\_speed} в сеттере необходимо проверять, что заданная скорость не превышает \texttt{\_\_max\_speed} и неотрицательна.  
    \item Для поля \texttt{\_\_luggage} в сеттере необходимо проверять, что количество единиц багажа не превышает \texttt{\_\_luggage\_spaces}.
    \item Для полей \texttt{\_\_fuel} и \texttt{\_\_engine\_oil} значения не должны превышать соответствующие ёмкости (\texttt{\_\_fuel\_tank} и \texttt{\_\_engine\_oil\_capacity}) и должны быть неотрицательными.
\end{itemize}
Реализовать метод вывода всех установленных через сеттеры значений закрытых полей экземпляра класса.
На основе этого класса реализовать три подхода к управлению доступом:
\begin{enumerate}
    \item \textbf{С использованием объекта \texttt{property}}:  
    Для каждого поля определить отдельные методы-геттеры и сеттеры (например, \texttt{get\_speed}, \texttt{set\_speed}), а затем создать свойство:  
    \begin{verbatim}
speed = property(get_speed, set_speed)
    \end{verbatim}  
    Этот код должен располагаться после определения соответствующих методов. Первый аргумент — геттер, второй — сеттер.  
    Продемонстрировать работу на трёх экземплярах класса: создать \texttt{myscoot1}, \texttt{myscoot2}, \texttt{myscoot3}, установить значения через свойства и вывести их.
    \item \textbf{С использованием декораторов \texttt{@property} и \texttt{@<имя>.setter}}:  
    Создать новую версию класса, в которой геттеры оформляются с декоратором \texttt{@property}, а сеттеры — с декоратором вида \texttt{@speed.setter}. Имена методов должны совпадать и не содержать префиксов \texttt{get\_}/\texttt{set\_}.  
    Пример:  
    \begin{verbatim}
@property
def speed(self):
    return self.__speed
@speed.setter
def speed(self, value):
    if 0 <= value <= self.__max_speed:
        self.__speed = value
    else:
        raise ValueError("Недопустимая скорость")
    \end{verbatim}  
    Продемонстрировать работу на трёх экземплярах и сделать выводы об оптимизации кода по сравнению с первым подходом.
    \item \textbf{С использованием модуля \texttt{accessify}}:  
    Установить модуль командой \texttt{pip install accessify} и импортировать:  
    \begin{verbatim}
from accessify import private, protected
    \end{verbatim}  
    Сделать поля \texttt{max\_speed}, \texttt{capacity}, \texttt{fuel\_tank}, \texttt{engine\_oil\_capacity}, \texttt{luggage\_spaces} по-настоящему приватными с помощью функции \texttt{private} (например, как атрибуты класса до \texttt{\_\_init\_\_}). Удалить их из инициализатора.  
    Проверки в сеттерах реализовать через вспомогательные методы, помеченные декоратором \texttt{@private}.  
    Учитывать, что методы с \texttt{@private} нельзя вызывать из методов, использующих \texttt{@property}, поэтому для этой версии использовать только классические геттеры и сеттеры (\texttt{get\_...}, \texttt{set\_...}).  
    Продемонстрировать, что попытка доступа извне (включая \texttt{myscoot3.\_Scooter\_\_max\_speed}) \textbf{не даёт результата}, а вызов приватного метода или чтение приватного поля вызывает ошибку доступа.
\end{enumerate}
Для всех трёх подходов создать по три экземпляра скутера, установить значения полей с учётом всех ограничений и вывести текущие значения всех полей каждого экземпляра.
\item[13] Разработать класс \texttt{Taxi}, который будет описывать модель такси. В классе должны быть следующие поля с доступом уровня \textbf{private} (только внутри класса):
\begin{itemize}
    \item \texttt{\_\_speed}: скорость движения такси  
    \item \texttt{\_\_distance}: расстояние, которое такси проехало  
    \item \texttt{\_\_max\_speed}: максимальная разрешённая скорость движения такси  
    \item \texttt{\_\_passengers}: список пассажиров  
    \item \texttt{\_\_capacity}: максимальная вместимость пассажиров в такси  
    \item \texttt{\_\_empty\_seats}: число свободных мест  
    \item \texttt{\_\_seats\_occupied}: число занятых мест в такси  
    \item \texttt{\_\_fuel\_tank}: объём топливного бака  
    \item \texttt{\_\_fuel}: количество топлива в литрах  
    \item \texttt{\_\_engine\_oil\_capacity}: объём картера масла двигателя (литры)  
    \item \texttt{\_\_engine\_oil}: количество моторного масла в литрах  
    \item \texttt{\_\_luggage\_spaces}: количество багажных мест  
    \item \texttt{\_\_luggage}: багаж такси  
\end{itemize}
Уровень доступа к полям должен быть следующим:
\begin{itemize}
    \item \texttt{\_\_max\_speed}, \texttt{\_\_capacity}, \texttt{\_\_fuel\_tank}, \texttt{\_\_engine\_oil\_capacity}, \texttt{\_\_luggage\_spaces}: \textbf{только чтение} (через геттеры)  
    \item \texttt{\_\_speed}, \texttt{\_\_distance}, \texttt{\_\_passengers}, \texttt{\_\_empty\_seats}, \texttt{\_\_seats\_occupied}, \texttt{\_\_fuel}, \texttt{\_\_engine\_oil}, \texttt{\_\_luggage}: \textbf{чтение и запись} (через геттеры и сеттеры)
\end{itemize}
Требования к сеттерам:
\begin{itemize}
    \item Для полей \texttt{\_\_empty\_seats} и \texttt{\_\_seats\_occupied} в сеттерах необходимо проверять, что передаваемое значение не превышает \texttt{\_\_capacity} и неотрицательно.  
    \item Для поля \texttt{\_\_passengers} в сеттере необходимо проверять, что количество пассажиров (длина списка) не превышает \texttt{\_\_capacity}.  
    \item Для поля \texttt{\_\_speed} в сеттере необходимо проверять, что заданная скорость не превышает \texttt{\_\_max\_speed} и неотрицательна.  
    \item Для поля \texttt{\_\_luggage} в сеттере необходимо проверять, что количество единиц багажа не превышает \texttt{\_\_luggage\_spaces}.
    \item Для полей \texttt{\_\_fuel} и \texttt{\_\_engine\_oil} значения не должны превышать соответствующие ёмкости (\texttt{\_\_fuel\_tank} и \texttt{\_\_engine\_oil\_capacity}) и должны быть неотрицательными.
\end{itemize}
Реализовать метод вывода всех установленных через сеттеры значений закрытых полей экземпляра класса.
На основе этого класса реализовать три подхода к управлению доступом:
\begin{enumerate}
    \item \textbf{С использованием объекта \texttt{property}}:  
    Для каждого поля определить отдельные методы-геттеры и сеттеры (например, \texttt{get\_speed}, \texttt{set\_speed}), а затем создать свойство:  
    \begin{verbatim}
speed = property(get_speed, set_speed)
    \end{verbatim}  
    Этот код должен располагаться после определения соответствующих методов. Первый аргумент — геттер, второй — сеттер.  
    Продемонстрировать работу на трёх экземплярах класса: создать \texttt{mytaxi1}, \texttt{mytaxi2}, \texttt{mytaxi3}, установить значения через свойства и вывести их.
    \item \textbf{С использованием декораторов \texttt{@property} и \texttt{@<имя>.setter}}:  
    Создать новую версию класса, в которой геттеры оформляются с декоратором \texttt{@property}, а сеттеры — с декоратором вида \texttt{@speed.setter}. Имена методов должны совпадать и не содержать префиксов \texttt{get\_}/\texttt{set\_}.  
    Пример:  
    \begin{verbatim}
@property
def speed(self):
    return self.__speed
@speed.setter
def speed(self, value):
    if 0 <= value <= self.__max_speed:
        self.__speed = value
    else:
        raise ValueError("Недопустимая скорость")
    \end{verbatim}  
    Продемонстрировать работу на трёх экземплярах и сделать выводы об оптимизации кода по сравнению с первым подходом.
    \item \textbf{С использованием модуля \texttt{accessify}}:  
    Установить модуль командой \texttt{pip install accessify} и импортировать:  
    \begin{verbatim}
from accessify import private, protected
    \end{verbatim}  
    Сделать поля \texttt{max\_speed}, \texttt{capacity}, \texttt{fuel\_tank}, \texttt{engine\_oil\_capacity}, \texttt{luggage\_spaces} по-настоящему приватными с помощью функции \texttt{private} (например, как атрибуты класса до \texttt{\_\_init\_\_}). Удалить их из инициализатора.  
    Проверки в сеттерах реализовать через вспомогательные методы, помеченные декоратором \texttt{@private}.  
    Учитывать, что методы с \texttt{@private} нельзя вызывать из методов, использующих \texttt{@property}, поэтому для этой версии использовать только классические геттеры и сеттеры (\texttt{get\_...}, \texttt{set\_...}).  
    Продемонстрировать, что попытка доступа извне (включая \texttt{mytaxi3.\_Taxi\_\_max\_speed}) \textbf{не даёт результата}, а вызов приватного метода или чтение приватного поля вызывает ошибку доступа.
\end{enumerate}
Для всех трёх подходов создать по три экземпляра такси, установить значения полей с учётом всех ограничений и вывести текущие значения всех полей каждого экземпляра.
\item[14] Разработать класс \texttt{Ambulance}, который будет описывать модель скорой помощи. В классе должны быть следующие поля с доступом уровня \textbf{private} (только внутри класса):
\begin{itemize}
    \item \texttt{\_\_speed}: скорость движения скорой помощи  
    \item \texttt{\_\_distance}: расстояние, которое скорая помощь проехала  
    \item \texttt{\_\_max\_speed}: максимальная разрешённая скорость движения скорой помощи  
    \item \texttt{\_\_passengers}: список пассажиров  
    \item \texttt{\_\_capacity}: максимальная вместимость пассажиров в скорой помощи  
    \item \texttt{\_\_empty\_seats}: число свободных мест  
    \item \texttt{\_\_seats\_occupied}: число занятых мест в скорой помощи  
    \item \texttt{\_\_fuel\_tank}: объём топливного бака  
    \item \texttt{\_\_fuel}: количество топлива в литрах  
    \item \texttt{\_\_engine\_oil\_capacity}: объём картера масла двигателя (литры)  
    \item \texttt{\_\_engine\_oil}: количество моторного масла в литрах  
    \item \texttt{\_\_luggage\_spaces}: количество багажных мест  
    \item \texttt{\_\_luggage}: багаж скорой помощи  
\end{itemize}
Уровень доступа к полям должен быть следующим:
\begin{itemize}
    \item \texttt{\_\_max\_speed}, \texttt{\_\_capacity}, \texttt{\_\_fuel\_tank}, \texttt{\_\_engine\_oil\_capacity}, \texttt{\_\_luggage\_spaces}: \textbf{только чтение} (через геттеры)  
    \item \texttt{\_\_speed}, \texttt{\_\_distance}, \texttt{\_\_passengers}, \texttt{\_\_empty\_seats}, \texttt{\_\_seats\_occupied}, \texttt{\_\_fuel}, \texttt{\_\_engine\_oil}, \texttt{\_\_luggage}: \textbf{чтение и запись} (через геттеры и сеттеры)
\end{itemize}
Требования к сеттерам:
\begin{itemize}
    \item Для полей \texttt{\_\_empty\_seats} и \texttt{\_\_seats\_occupied} в сеттерах необходимо проверять, что передаваемое значение не превышает \texttt{\_\_capacity} и неотрицательно.  
    \item Для поля \texttt{\_\_passengers} в сеттере необходимо проверять, что количество пассажиров (длина списка) не превышает \texttt{\_\_capacity}.  
    \item Для поля \texttt{\_\_speed} в сеттере необходимо проверять, что заданная скорость не превышает \texttt{\_\_max\_speed} и неотрицательна.  
    \item Для поля \texttt{\_\_luggage} в сеттере необходимо проверять, что количество единиц багажа не превышает \texttt{\_\_luggage\_spaces}.
    \item Для полей \texttt{\_\_fuel} и \texttt{\_\_engine\_oil} значения не должны превышать соответствующие ёмкости (\texttt{\_\_fuel\_tank} и \texttt{\_\_engine\_oil\_capacity}) и должны быть неотрицательными.
\end{itemize}
Реализовать метод вывода всех установленных через сеттеры значений закрытых полей экземпляра класса.
На основе этого класса реализовать три подхода к управлению доступом:
\begin{enumerate}
    \item \textbf{С использованием объекта \texttt{property}}:  
    Для каждого поля определить отдельные методы-геттеры и сеттеры (например, \texttt{get\_speed}, \texttt{set\_speed}), а затем создать свойство:  
    \begin{verbatim}
speed = property(get_speed, set_speed)
    \end{verbatim}  
    Этот код должен располагаться после определения соответствующих методов. Первый аргумент — геттер, второй — сеттер.  
    Продемонстрировать работу на трёх экземплярах класса: создать \texttt{myamb1}, \texttt{myamb2}, \texttt{myamb3}, установить значения через свойства и вывести их.
    \item \textbf{С использованием декораторов \texttt{@property} и \texttt{@<имя>.setter}}:  
    Создать новую версию класса, в которой геттеры оформляются с декоратором \texttt{@property}, а сеттеры — с декоратором вида \texttt{@speed.setter}. Имена методов должны совпадать и не содержать префиксов \texttt{get\_}/\texttt{set\_}.  
    Пример:  
    \begin{verbatim}
@property
def speed(self):
    return self.__speed
@speed.setter
def speed(self, value):
    if 0 <= value <= self.__max_speed:
        self.__speed = value
    else:
        raise ValueError("Недопустимая скорость")
    \end{verbatim}  
    Продемонстрировать работу на трёх экземплярах и сделать выводы об оптимизации кода по сравнению с первым подходом.
    \item \textbf{С использованием модуля \texttt{accessify}}:  
    Установить модуль командой \texttt{pip install accessify} и импортировать:  
    \begin{verbatim}
from accessify import private, protected
    \end{verbatim}  
    Сделать поля \texttt{max\_speed}, \texttt{capacity}, \texttt{fuel\_tank}, \texttt{engine\_oil\_capacity}, \texttt{luggage\_spaces} по-настоящему приватными с помощью функции \texttt{private} (например, как атрибуты класса до \texttt{\_\_init\_\_}). Удалить их из инициализатора.  
    Проверки в сеттерах реализовать через вспомогательные методы, помеченные декоратором \texttt{@private}.  
    Учитывать, что методы с \texttt{@private} нельзя вызывать из методов, использующих \texttt{@property}, поэтому для этой версии использовать только классические геттеры и сеттеры (\texttt{get\_...}, \texttt{set\_...}).  
    Продемонстрировать, что попытка доступа извне (включая \texttt{myamb3.\_Ambulance\_\_max\_speed}) \textbf{не даёт результата}, а вызов приватного метода или чтение приватного поля вызывает ошибку доступа.
\end{enumerate}
Для всех трёх подходов создать по три экземпляра скорой помощи, установить значения полей с учётом всех ограничений и вывести текущие значения всех полей каждого экземпляра.
\item[15] Разработать класс \texttt{FireTruck}, который будет описывать модель пожарной машины. В классе должны быть следующие поля с доступом уровня \textbf{private} (только внутри класса):
\begin{itemize}
    \item \texttt{\_\_speed}: скорость движения пожарной машины  
    \item \texttt{\_\_distance}: расстояние, которое пожарная машина проехала  
    \item \texttt{\_\_max\_speed}: максимальная разрешённая скорость движения пожарной машины  
    \item \texttt{\_\_passengers}: список пассажиров  
    \item \texttt{\_\_capacity}: максимальная вместимость пассажиров в пожарной машине  
    \item \texttt{\_\_empty\_seats}: число свободных мест  
    \item \texttt{\_\_seats\_occupied}: число занятых мест в пожарной машине  
    \item \texttt{\_\_fuel\_tank}: объём топливного бака  
    \item \texttt{\_\_fuel}: количество топлива в литрах  
    \item \texttt{\_\_engine\_oil\_capacity}: объём картера масла двигателя (литры)  
    \item \texttt{\_\_engine\_oil}: количество моторного масла в литрах  
    \item \texttt{\_\_luggage\_spaces}: количество багажных мест  
    \item \texttt{\_\_luggage}: багаж пожарной машины  
\end{itemize}
Уровень доступа к полям должен быть следующим:
\begin{itemize}
    \item \texttt{\_\_max\_speed}, \texttt{\_\_capacity}, \texttt{\_\_fuel\_tank}, \texttt{\_\_engine\_oil\_capacity}, \texttt{\_\_luggage\_spaces}: \textbf{только чтение} (через геттеры)  
    \item \texttt{\_\_speed}, \texttt{\_\_distance}, \texttt{\_\_passengers}, \texttt{\_\_empty\_seats}, \texttt{\_\_seats\_occupied}, \texttt{\_\_fuel}, \texttt{\_\_engine\_oil}, \texttt{\_\_luggage}: \textbf{чтение и запись} (через геттеры и сеттеры)
\end{itemize}
Требования к сеттерам:
\begin{itemize}
    \item Для полей \texttt{\_\_empty\_seats} и \texttt{\_\_seats\_occupied} в сеттерах необходимо проверять, что передаваемое значение не превышает \texttt{\_\_capacity} и неотрицательно.  
    \item Для поля \texttt{\_\_passengers} в сеттере необходимо проверять, что количество пассажиров (длина списка) не превышает \texttt{\_\_capacity}.  
    \item Для поля \texttt{\_\_speed} в сеттере необходимо проверять, что заданная скорость не превышает \texttt{\_\_max\_speed} и неотрицательна.  
    \item Для поля \texttt{\_\_luggage} в сеттере необходимо проверять, что количество единиц багажа не превышает \texttt{\_\_luggage\_spaces}.
    \item Для полей \texttt{\_\_fuel} и \texttt{\_\_engine\_oil} значения не должны превышать соответствующие ёмкости (\texttt{\_\_fuel\_tank} и \texttt{\_\_engine\_oil\_capacity}) и должны быть неотрицательными.
\end{itemize}
Реализовать метод вывода всех установленных через сеттеры значений закрытых полей экземпляра класса.
На основе этого класса реализовать три подхода к управлению доступом:
\begin{enumerate}
    \item \textbf{С использованием объекта \texttt{property}}:  
    Для каждого поля определить отдельные методы-геттеры и сеттеры (например, \texttt{get\_speed}, \texttt{set\_speed}), а затем создать свойство:  
    \begin{verbatim}
speed = property(get_speed, set_speed)
    \end{verbatim}  
    Этот код должен располагаться после определения соответствующих методов. Первый аргумент — геттер, второй — сеттер.  
    Продемонстрировать работу на трёх экземплярах класса: создать \texttt{myfire1}, \texttt{myfire2}, \texttt{myfire3}, установить значения через свойства и вывести их.
    \item \textbf{С использованием декораторов \texttt{@property} и \texttt{@<имя>.setter}}:  
    Создать новую версию класса, в которой геттеры оформляются с декоратором \texttt{@property}, а сеттеры — с декоратором вида \texttt{@speed.setter}. Имена методов должны совпадать и не содержать префиксов \texttt{get\_}/\texttt{set\_}.  
    Пример:  
    \begin{verbatim}
@property
def speed(self):
    return self.__speed
@speed.setter
def speed(self, value):
    if 0 <= value <= self.__max_speed:
        self.__speed = value
    else:
        raise ValueError("Недопустимая скорость")
    \end{verbatim}  
    Продемонстрировать работу на трёх экземплярах и сделать выводы об оптимизации кода по сравнению с первым подходом.
    \item \textbf{С использованием модуля \texttt{accessify}}:  
    Установить модуль командой \texttt{pip install accessify} и импортировать:  
    \begin{verbatim}
from accessify import private, protected
    \end{verbatim}  
    Сделать поля \texttt{max\_speed}, \texttt{capacity}, \texttt{fuel\_tank}, \texttt{engine\_oil\_capacity}, \texttt{luggage\_spaces} по-настоящему приватными с помощью функции \texttt{private} (например, как атрибуты класса до \texttt{\_\_init\_\_}). Удалить их из инициализатора.  
    Проверки в сеттерах реализовать через вспомогательные методы, помеченные декоратором \texttt{@private}.  
    Учитывать, что методы с \texttt{@private} нельзя вызывать из методов, использующих \texttt{@property}, поэтому для этой версии использовать только классические геттеры и сеттеры (\texttt{get\_...}, \texttt{set\_...}).  
    Продемонстрировать, что попытка доступа извне (включая \texttt{myfire3.\_FireTruck\_\_max\_speed}) \textbf{не даёт результата}, а вызов приватного метода или чтение приватного поля вызывает ошибку доступа.
\end{enumerate}
Для всех трёх подходов создать по три экземпляра пожарной машины, установить значения полей с учётом всех ограничений и вывести текущие значения всех полей каждого экземпляра.
\item[16] Разработать класс \texttt{PoliceCar}, который будет описывать модель полицейского автомобиля. В классе должны быть следующие поля с доступом уровня \textbf{private} (только внутри класса):
\begin{itemize}
    \item \texttt{\_\_speed}: скорость движения полицейского автомобиля  
    \item \texttt{\_\_distance}: расстояние, которое полицейский автомобиль проехал  
    \item \texttt{\_\_max\_speed}: максимальная разрешённая скорость движения полицейского автомобиля  
    \item \texttt{\_\_passengers}: список пассажиров  
    \item \texttt{\_\_capacity}: максимальная вместимость пассажиров в полицейском автомобиле  
    \item \texttt{\_\_empty\_seats}: число свободных мест  
    \item \texttt{\_\_seats\_occupied}: число занятых мест в полицейском автомобиле  
    \item \texttt{\_\_fuel\_tank}: объём топливного бака  
    \item \texttt{\_\_fuel}: количество топлива в литрах  
    \item \texttt{\_\_engine\_oil\_capacity}: объём картера масла двигателя (литры)  
    \item \texttt{\_\_engine\_oil}: количество моторного масла в литрах  
    \item \texttt{\_\_luggage\_spaces}: количество багажных мест  
    \item \texttt{\_\_luggage}: багаж полицейского автомобиля  
\end{itemize}
Уровень доступа к полям должен быть следующим:
\begin{itemize}
    \item \texttt{\_\_max\_speed}, \texttt{\_\_capacity}, \texttt{\_\_fuel\_tank}, \texttt{\_\_engine\_oil\_capacity}, \texttt{\_\_luggage\_spaces}: \textbf{только чтение} (через геттеры)  
    \item \texttt{\_\_speed}, \texttt{\_\_distance}, \texttt{\_\_passengers}, \texttt{\_\_empty\_seats}, \texttt{\_\_seats\_occupied}, \texttt{\_\_fuel}, \texttt{\_\_engine\_oil}, \texttt{\_\_luggage}: \textbf{чтение и запись} (через геттеры и сеттеры)
\end{itemize}
Требования к сеттерам:
\begin{itemize}
    \item Для полей \texttt{\_\_empty\_seats} и \texttt{\_\_seats\_occupied} в сеттерах необходимо проверять, что передаваемое значение не превышает \texttt{\_\_capacity} и неотрицательно.  
    \item Для поля \texttt{\_\_passengers} в сеттере необходимо проверять, что количество пассажиров (длина списка) не превышает \texttt{\_\_capacity}.  
    \item Для поля \texttt{\_\_speed} в сеттере необходимо проверять, что заданная скорость не превышает \texttt{\_\_max\_speed} и неотрицательна.  
    \item Для поля \texttt{\_\_luggage} в сеттере необходимо проверять, что количество единиц багажа не превышает \texttt{\_\_luggage\_spaces}.
    \item Для полей \texttt{\_\_fuel} и \texttt{\_\_engine\_oil} значения не должны превышать соответствующие ёмкости (\texttt{\_\_fuel\_tank} и \texttt{\_\_engine\_oil\_capacity}) и должны быть неотрицательными.
\end{itemize}
Реализовать метод вывода всех установленных через сеттеры значений закрытых полей экземпляра класса.
На основе этого класса реализовать три подхода к управлению доступом:
\begin{enumerate}
    \item \textbf{С использованием объекта \texttt{property}}:  
    Для каждого поля определить отдельные методы-геттеры и сеттеры (например, \texttt{get\_speed}, \texttt{set\_speed}), а затем создать свойство:  
    \begin{verbatim}
speed = property(get_speed, set_speed)
    \end{verbatim}  
    Этот код должен располагаться после определения соответствующих методов. Первый аргумент — геттер, второй — сеттер.  
    Продемонстрировать работу на трёх экземплярах класса: создать \texttt{mypolice1}, \texttt{mypolice2}, \texttt{mypolice3}, установить значения через свойства и вывести их.
    \item \textbf{С использованием декораторов \texttt{@property} и \texttt{@<имя>.setter}}:  
    Создать новую версию класса, в которой геттеры оформляются с декоратором \texttt{@property}, а сеттеры — с декоратором вида \texttt{@speed.setter}. Имена методов должны совпадать и не содержать префиксов \texttt{get\_}/\texttt{set\_}.  
    Пример:  
    \begin{verbatim}
@property
def speed(self):
    return self.__speed
@speed.setter
def speed(self, value):
    if 0 <= value <= self.__max_speed:
        self.__speed = value
    else:
        raise ValueError("Недопустимая скорость")
    \end{verbatim}  
    Продемонстрировать работу на трёх экземплярах и сделать выводы об оптимизации кода по сравнению с первым подходом.
    \item \textbf{С использованием модуля \texttt{accessify}}:  
    Установить модуль командой \texttt{pip install accessify} и импортировать:  
    \begin{verbatim}
from accessify import private, protected
    \end{verbatim}  
    Сделать поля \texttt{max\_speed}, \texttt{capacity}, \texttt{fuel\_tank}, \texttt{engine\_oil\_capacity}, \texttt{luggage\_spaces} по-настоящему приватными с помощью функции \texttt{private} (например, как атрибуты класса до \texttt{\_\_init\_\_}). Удалить их из инициализатора.  
    Проверки в сеттерах реализовать через вспомогательные методы, помеченные декоратором \texttt{@private}.  
    Учитывать, что методы с \texttt{@private} нельзя вызывать из методов, использующих \texttt{@property}, поэтому для этой версии использовать только классические геттеры и сеттеры (\texttt{get\_...}, \texttt{set\_...}).  
    Продемонстрировать, что попытка доступа извне (включая \texttt{mypolice3.\_PoliceCar\_\_max\_speed}) \textbf{не даёт результата}, а вызов приватного метода или чтение приватного поля вызывает ошибку доступа.
\end{enumerate}
Для всех трёх подходов создать по три экземпляра полицейского автомобиля, установить значения полей с учётом всех ограничений и вывести текущие значения всех полей каждого экземпляра.
\item[17] Разработать класс \texttt{Crane}, который будет описывать модель подъёмного крана. В классе должны быть следующие поля с доступом уровня \textbf{private} (только внутри класса):
\begin{itemize}
    \item \texttt{\_\_speed}: скорость движения крана  
    \item \texttt{\_\_distance}: расстояние, которое кран проехал  
    \item \texttt{\_\_max\_speed}: максимальная разрешённая скорость движения крана  
    \item \texttt{\_\_passengers}: список пассажиров  
    \item \texttt{\_\_capacity}: максимальная вместимость пассажиров в кране  
    \item \texttt{\_\_empty\_seats}: число свободных мест  
    \item \texttt{\_\_seats\_occupied}: число занятых мест в кране  
    \item \texttt{\_\_fuel\_tank}: объём топливного бака  
    \item \texttt{\_\_fuel}: количество топлива в литрах  
    \item \texttt{\_\_engine\_oil\_capacity}: объём картера масла двигателя (литры)  
    \item \texttt{\_\_engine\_oil}: количество моторного масла в литрах  
    \item \texttt{\_\_luggage\_spaces}: количество багажных мест  
    \item \texttt{\_\_luggage}: багаж крана  
\end{itemize}
Уровень доступа к полям должен быть следующим:
\begin{itemize}
    \item \texttt{\_\_max\_speed}, \texttt{\_\_capacity}, \texttt{\_\_fuel\_tank}, \texttt{\_\_engine\_oil\_capacity}, \texttt{\_\_luggage\_spaces}: \textbf{только чтение} (через геттеры)  
    \item \texttt{\_\_speed}, \texttt{\_\_distance}, \texttt{\_\_passengers}, \texttt{\_\_empty\_seats}, \texttt{\_\_seats\_occupied}, \texttt{\_\_fuel}, \texttt{\_\_engine\_oil}, \texttt{\_\_luggage}: \textbf{чтение и запись} (через геттеры и сеттеры)
\end{itemize}
Требования к сеттерам:
\begin{itemize}
    \item Для полей \texttt{\_\_empty\_seats} и \texttt{\_\_seats\_occupied} в сеттерах необходимо проверять, что передаваемое значение не превышает \texttt{\_\_capacity} и неотрицательно.  
    \item Для поля \texttt{\_\_passengers} в сеттере необходимо проверять, что количество пассажиров (длина списка) не превышает \texttt{\_\_capacity}.  
    \item Для поля \texttt{\_\_speed} в сеттере необходимо проверять, что заданная скорость не превышает \texttt{\_\_max\_speed} и неотрицательна.  
    \item Для поля \texttt{\_\_luggage} в сеттере необходимо проверять, что количество единиц багажа не превышает \texttt{\_\_luggage\_spaces}.
    \item Для полей \texttt{\_\_fuel} и \texttt{\_\_engine\_oil} значения не должны превышать соответствующие ёмкости (\texttt{\_\_fuel\_tank} и \texttt{\_\_engine\_oil\_capacity}) и должны быть неотрицательными.
\end{itemize}
Реализовать метод вывода всех установленных через сеттеры значений закрытых полей экземпляра класса.
На основе этого класса реализовать три подхода к управлению доступом:
\begin{enumerate}
    \item \textbf{С использованием объекта \texttt{property}}:  
    Для каждого поля определить отдельные методы-геттеры и сеттеры (например, \texttt{get\_speed}, \texttt{set\_speed}), а затем создать свойство:  
    \begin{verbatim}
speed = property(get_speed, set_speed)
    \end{verbatim}  
    Этот код должен располагаться после определения соответствующих методов. Первый аргумент — геттер, второй — сеттер.  
    Продемонстрировать работу на трёх экземплярах класса: создать \texttt{mycrane1}, \texttt{mycrane2}, \texttt{mycrane3}, установить значения через свойства и вывести их.
    \item \textbf{С использованием декораторов \texttt{@property} и \texttt{@<имя>.setter}}:  
    Создать новую версию класса, в которой геттеры оформляются с декоратором \texttt{@property}, а сеттеры — с декоратором вида \texttt{@speed.setter}. Имена методов должны совпадать и не содержать префиксов \texttt{get\_}/\texttt{set\_}.  
    Пример:  
    \begin{verbatim}
@property
def speed(self):
    return self.__speed
@speed.setter
def speed(self, value):
    if 0 <= value <= self.__max_speed:
        self.__speed = value
    else:
        raise ValueError("Недопустимая скорость")
    \end{verbatim}  
    Продемонстрировать работу на трёх экземплярах и сделать выводы об оптимизации кода по сравнению с первым подходом.
    \item \textbf{С использованием модуля \texttt{accessify}}:  
    Установить модуль командой \texttt{pip install accessify} и импортировать:  
    \begin{verbatim}
from accessify import private, protected
    \end{verbatim}  
    Сделать поля \texttt{max\_speed}, \texttt{capacity}, \texttt{fuel\_tank}, \texttt{engine\_oil\_capacity}, \texttt{luggage\_spaces} по-настоящему приватными с помощью функции \texttt{private} (например, как атрибуты класса до \texttt{\_\_init\_\_}). Удалить их из инициализатора.  
    Проверки в сеттерах реализовать через вспомогательные методы, помеченные декоратором \texttt{@private}.  
    Учитывать, что методы с \texttt{@private} нельзя вызывать из методов, использующих \texttt{@property}, поэтому для этой версии использовать только классические геттеры и сеттеры (\texttt{get\_...}, \texttt{set\_...}).  
    Продемонстрировать, что попытка доступа извне (включая \texttt{mycrane3.\_Crane\_\_max\_speed}) \textbf{не даёт результата}, а вызов приватного метода или чтение приватного поля вызывает ошибку доступа.
\end{enumerate}
Для всех трёх подходов создать по три экземпляра подъёмного крана, установить значения полей с учётом всех ограничений и вывести текущие значения всех полей каждого экземпляра.
\item[18] Разработать класс \texttt{Excavator}, который будет описывать модель экскаватора. В классе должны быть следующие поля с доступом уровня \textbf{private} (только внутри класса):
\begin{itemize}
    \item \texttt{\_\_speed}: скорость движения экскаватора  
    \item \texttt{\_\_distance}: расстояние, которое экскаватор проехал  
    \item \texttt{\_\_max\_speed}: максимальная разрешённая скорость движения экскаватора  
    \item \texttt{\_\_passengers}: список пассажиров  
    \item \texttt{\_\_capacity}: максимальная вместимость пассажиров в экскаваторе  
    \item \texttt{\_\_empty\_seats}: число свободных мест  
    \item \texttt{\_\_seats\_occupied}: число занятых мест в экскаваторе  
    \item \texttt{\_\_fuel\_tank}: объём топливного бака  
    \item \texttt{\_\_fuel}: количество топлива в литрах  
    \item \texttt{\_\_engine\_oil\_capacity}: объём картера масла двигателя (литры)  
    \item \texttt{\_\_engine\_oil}: количество моторного масла в литрах  
    \item \texttt{\_\_luggage\_spaces}: количество багажных мест  
    \item \texttt{\_\_luggage}: багаж экскаватора  
\end{itemize}
Уровень доступа к полям должен быть следующим:
\begin{itemize}
    \item \texttt{\_\_max\_speed}, \texttt{\_\_capacity}, \texttt{\_\_fuel\_tank}, \texttt{\_\_engine\_oil\_capacity}, \texttt{\_\_luggage\_spaces}: \textbf{только чтение} (через геттеры)  
    \item \texttt{\_\_speed}, \texttt{\_\_distance}, \texttt{\_\_passengers}, \texttt{\_\_empty\_seats}, \texttt{\_\_seats\_occupied}, \texttt{\_\_fuel}, \texttt{\_\_engine\_oil}, \texttt{\_\_luggage}: \textbf{чтение и запись} (через геттеры и сеттеры)
\end{itemize}
Требования к сеттерам:
\begin{itemize}
    \item Для полей \texttt{\_\_empty\_seats} и \texttt{\_\_seats\_occupied} в сеттерах необходимо проверять, что передаваемое значение не превышает \texttt{\_\_capacity} и неотрицательно.  
    \item Для поля \texttt{\_\_passengers} в сеттере необходимо проверять, что количество пассажиров (длина списка) не превышает \texttt{\_\_capacity}.  
    \item Для поля \texttt{\_\_speed} в сеттере необходимо проверять, что заданная скорость не превышает \texttt{\_\_max\_speed} и неотрицательна.  
    \item Для поля \texttt{\_\_luggage} в сеттере необходимо проверять, что количество единиц багажа не превышает \texttt{\_\_luggage\_spaces}.
    \item Для полей \texttt{\_\_fuel} и \texttt{\_\_engine\_oil} значения не должны превышать соответствующие ёмкости (\texttt{\_\_fuel\_tank} и \texttt{\_\_engine\_oil\_capacity}) и должны быть неотрицательными.
\end{itemize}
Реализовать метод вывода всех установленных через сеттеры значений закрытых полей экземпляра класса.
На основе этого класса реализовать три подхода к управлению доступом:
\begin{enumerate}
    \item \textbf{С использованием объекта \texttt{property}}:  
    Для каждого поля определить отдельные методы-геттеры и сеттеры (например, \texttt{get\_speed}, \texttt{set\_speed}), а затем создать свойство:  
    \begin{verbatim}
speed = property(get_speed, set_speed)
    \end{verbatim}  
    Этот код должен располагаться после определения соответствующих методов. Первый аргумент — геттер, второй — сеттер.  
    Продемонстрировать работу на трёх экземплярах класса: создать \texttt{myex1}, \texttt{myex2}, \texttt{myex3}, установить значения через свойства и вывести их.
    \item \textbf{С использованием декораторов \texttt{@property} и \texttt{@<имя>.setter}}:  
    Создать новую версию класса, в которой геттеры оформляются с декоратором \texttt{@property}, а сеттеры — с декоратором вида \texttt{@speed.setter}. Имена методов должны совпадать и не содержать префиксов \texttt{get\_}/\texttt{set\_}.  
    Пример:  
    \begin{verbatim}
@property
def speed(self):
    return self.__speed
@speed.setter
def speed(self, value):
    if 0 <= value <= self.__max_speed:
        self.__speed = value
    else:
        raise ValueError("Недопустимая скорость")
    \end{verbatim}  
    Продемонстрировать работу на трёх экземплярах и сделать выводы об оптимизации кода по сравнению с первым подходом.
    \item \textbf{С использованием модуля \texttt{accessify}}:  
    Установить модуль командой \texttt{pip install accessify} и импортировать:  
    \begin{verbatim}
from accessify import private, protected
    \end{verbatim}  
    Сделать поля \texttt{max\_speed}, \texttt{capacity}, \texttt{fuel\_tank}, \texttt{engine\_oil\_capacity}, \texttt{luggage\_spaces} по-настоящему приватными с помощью функции \texttt{private} (например, как атрибуты класса до \texttt{\_\_init\_\_}). Удалить их из инициализатора.  
    Проверки в сеттерах реализовать через вспомогательные методы, помеченные декоратором \texttt{@private}.  
    Учитывать, что методы с \texttt{@private} нельзя вызывать из методов, использующих \texttt{@property}, поэтому для этой версии использовать только классические геттеры и сеттеры (\texttt{get\_...}, \texttt{set\_...}).  
    Продемонстрировать, что попытка доступа извне (включая \texttt{myex3.\_Excavator\_\_max\_speed}) \textbf{не даёт результата}, а вызов приватного метода или чтение приватного поля вызывает ошибку доступа.
\end{enumerate}
Для всех трёх подходов создать по три экземпляра экскаватора, установить значения полей с учётом всех ограничений и вывести текущие значения всех полей каждого экземпляра.
\item[19] Разработать класс \texttt{Tractor}, который будет описывать модель трактора. В классе должны быть следующие поля с доступом уровня \textbf{private} (только внутри класса):
\begin{itemize}
    \item \texttt{\_\_speed}: скорость движения трактора  
    \item \texttt{\_\_distance}: расстояние, которое трактор проехал  
    \item \texttt{\_\_max\_speed}: максимальная разрешённая скорость движения трактора  
    \item \texttt{\_\_passengers}: список пассажиров  
    \item \texttt{\_\_capacity}: максимальная вместимость пассажиров в тракторе  
    \item \texttt{\_\_empty\_seats}: число свободных мест  
    \item \texttt{\_\_seats\_occupied}: число занятых мест в тракторе  
    \item \texttt{\_\_fuel\_tank}: объём топливного бака  
    \item \texttt{\_\_fuel}: количество топлива в литрах  
    \item \texttt{\_\_engine\_oil\_capacity}: объём картера масла двигателя (литры)  
    \item \texttt{\_\_engine\_oil}: количество моторного масла в литрах  
    \item \texttt{\_\_luggage\_spaces}: количество багажных мест  
    \item \texttt{\_\_luggage}: багаж трактора  
\end{itemize}
Уровень доступа к полям должен быть следующим:
\begin{itemize}
    \item \texttt{\_\_max\_speed}, \texttt{\_\_capacity}, \texttt{\_\_fuel\_tank}, \texttt{\_\_engine\_oil\_capacity}, \texttt{\_\_luggage\_spaces}: \textbf{только чтение} (через геттеры)  
    \item \texttt{\_\_speed}, \texttt{\_\_distance}, \texttt{\_\_passengers}, \texttt{\_\_empty\_seats}, \texttt{\_\_seats\_occupied}, \texttt{\_\_fuel}, \texttt{\_\_engine\_oil}, \texttt{\_\_luggage}: \textbf{чтение и запись} (через геттеры и сеттеры)
\end{itemize}
Требования к сеттерам:
\begin{itemize}
    \item Для полей \texttt{\_\_empty\_seats} и \texttt{\_\_seats\_occupied} в сеттерах необходимо проверять, что передаваемое значение не превышает \texttt{\_\_capacity} и неотрицательно.  
    \item Для поля \texttt{\_\_passengers} в сеттере необходимо проверять, что количество пассажиров (длина списка) не превышает \texttt{\_\_capacity}.  
    \item Для поля \texttt{\_\_speed} в сеттере необходимо проверять, что заданная скорость не превышает \texttt{\_\_max\_speed} и неотрицательна.  
    \item Для поля \texttt{\_\_luggage} в сеттере необходимо проверять, что количество единиц багажа не превышает \texttt{\_\_luggage\_spaces}.
    \item Для полей \texttt{\_\_fuel} и \texttt{\_\_engine\_oil} значения не должны превышать соответствующие ёмкости (\texttt{\_\_fuel\_tank} и \texttt{\_\_engine\_oil\_capacity}) и должны быть неотрицательными.
\end{itemize}
Реализовать метод вывода всех установленных через сеттеры значений закрытых полей экземпляра класса.
На основе этого класса реализовать три подхода к управлению доступом:
\begin{enumerate}
    \item \textbf{С использованием объекта \texttt{property}}:  
    Для каждого поля определить отдельные методы-геттеры и сеттеры (например, \texttt{get\_speed}, \texttt{set\_speed}), а затем создать свойство:  
    \begin{verbatim}
speed = property(get_speed, set_speed)
    \end{verbatim}  
    Этот код должен располагаться после определения соответствующих методов. Первый аргумент — геттер, второй — сеттер.  
    Продемонстрировать работу на трёх экземплярах класса: создать \texttt{mytractor1}, \texttt{mytractor2}, \texttt{mytractor3}, установить значения через свойства и вывести их.
    \item \textbf{С использованием декораторов \texttt{@property} и \texttt{@<имя>.setter}}:  
    Создать новую версию класса, в которой геттеры оформляются с декоратором \texttt{@property}, а сеттеры — с декоратором вида \texttt{@speed.setter}. Имена методов должны совпадать и не содержать префиксов \texttt{get\_}/\texttt{set\_}.  
    Пример:  
    \begin{verbatim}
@property
def speed(self):
    return self.__speed
@speed.setter
def speed(self, value):
    if 0 <= value <= self.__max_speed:
        self.__speed = value
    else:
        raise ValueError("Недопустимая скорость")
    \end{verbatim}  
    Продемонстрировать работу на трёх экземплярах и сделать выводы об оптимизации кода по сравнению с первым подходом.
    \item \textbf{С использованием модуля \texttt{accessify}}:  
    Установить модуль командой \texttt{pip install accessify} и импортировать:  
    \begin{verbatim}
from accessify import private, protected
    \end{verbatim}  
    Сделать поля \texttt{max\_speed}, \texttt{capacity}, \texttt{fuel\_tank}, \texttt{engine\_oil\_capacity}, \texttt{luggage\_spaces} по-настоящему приватными с помощью функции \texttt{private} (например, как атрибуты класса до \texttt{\_\_init\_\_}). Удалить их из инициализатора.  
    Проверки в сеттерах реализовать через вспомогательные методы, помеченные декоратором \texttt{@private}.  
    Учитывать, что методы с \texttt{@private} нельзя вызывать из методов, использующих \texttt{@property}, поэтому для этой версии использовать только классические геттеры и сеттеры (\texttt{get\_...}, \texttt{set\_...}).  
    Продемонстрировать, что попытка доступа извне (включая \texttt{mytractor3.\_Tractor\_\_max\_speed}) \textbf{не даёт результата}, а вызов приватного метода или чтение приватного поля вызывает ошибку доступа.
\end{enumerate}
Для всех трёх подходов создать по три экземпляра трактора, установить значения полей с учётом всех ограничений и вывести текущие значения всех полей каждого экземпляра.
\item[20] Разработать класс \texttt{Snowmobile}, который будет описывать модель снегохода. В классе должны быть следующие поля с доступом уровня \textbf{private} (только внутри класса):
\begin{itemize}
    \item \texttt{\_\_speed}: скорость движения снегохода  
    \item \texttt{\_\_distance}: расстояние, которое снегоход проехал  
    \item \texttt{\_\_max\_speed}: максимальная разрешённая скорость движения снегохода  
    \item \texttt{\_\_passengers}: список пассажиров  
    \item \texttt{\_\_capacity}: максимальная вместимость пассажиров на снегоходе  
    \item \texttt{\_\_empty\_seats}: число свободных мест  
    \item \texttt{\_\_seats\_occupied}: число занятых мест на снегоходе  
    \item \texttt{\_\_fuel\_tank}: объём топливного бака  
    \item \texttt{\_\_fuel}: количество топлива в литрах  
    \item \texttt{\_\_engine\_oil\_capacity}: объём картера масла двигателя (литры)  
    \item \texttt{\_\_engine\_oil}: количество моторного масла в литрах  
    \item \texttt{\_\_luggage\_spaces}: количество багажных мест  
    \item \texttt{\_\_luggage}: багаж снегохода  
\end{itemize}
Уровень доступа к полям должен быть следующим:
\begin{itemize}
    \item \texttt{\_\_max\_speed}, \texttt{\_\_capacity}, \texttt{\_\_fuel\_tank}, \texttt{\_\_engine\_oil\_capacity}, \texttt{\_\_luggage\_spaces}: \textbf{только чтение} (через геттеры)  
    \item \texttt{\_\_speed}, \texttt{\_\_distance}, \texttt{\_\_passengers}, \texttt{\_\_empty\_seats}, \texttt{\_\_seats\_occupied}, \texttt{\_\_fuel}, \texttt{\_\_engine\_oil}, \texttt{\_\_luggage}: \textbf{чтение и запись} (через геттеры и сеттеры)
\end{itemize}
Требования к сеттерам:
\begin{itemize}
    \item Для полей \texttt{\_\_empty\_seats} и \texttt{\_\_seats\_occupied} в сеттерах необходимо проверять, что передаваемое значение не превышает \texttt{\_\_capacity} и неотрицательно.  
    \item Для поля \texttt{\_\_passengers} в сеттере необходимо проверять, что количество пассажиров (длина списка) не превышает \texttt{\_\_capacity}.  
    \item Для поля \texttt{\_\_speed} в сеттере необходимо проверять, что заданная скорость не превышает \texttt{\_\_max\_speed} и неотрицательна.  
    \item Для поля \texttt{\_\_luggage} в сеттере необходимо проверять, что количество единиц багажа не превышает \texttt{\_\_luggage\_spaces}.
    \item Для полей \texttt{\_\_fuel} и \texttt{\_\_engine\_oil} значения не должны превышать соответствующие ёмкости (\texttt{\_\_fuel\_tank} и \texttt{\_\_engine\_oil\_capacity}) и должны быть неотрицательными.
\end{itemize}
Реализовать метод вывода всех установленных через сеттеры значений закрытых полей экземпляра класса.
На основе этого класса реализовать три подхода к управлению доступом:
\begin{enumerate}
    \item \textbf{С использованием объекта \texttt{property}}:  
    Для каждого поля определить отдельные методы-геттеры и сеттеры (например, \texttt{get\_speed}, \texttt{set\_speed}), а затем создать свойство:  
    \begin{verbatim}
speed = property(get_speed, set_speed)
    \end{verbatim}  
    Этот код должен располагаться после определения соответствующих методов. Первый аргумент — геттер, второй — сеттер.  
    Продемонстрировать работу на трёх экземплярах класса: создать \texttt{mysnow1}, \texttt{mysnow2}, \texttt{mysnow3}, установить значения через свойства и вывести их.
    \item \textbf{С использованием декораторов \texttt{@property} и \texttt{@<имя>.setter}}:  
    Создать новую версию класса, в которой геттеры оформляются с декоратором \texttt{@property}, а сеттеры — с декоратором вида \texttt{@speed.setter}. Имена методов должны совпадать и не содержать префиксов \texttt{get\_}/\texttt{set\_}.  
    Пример:  
    \begin{verbatim}
@property
def speed(self):
    return self.__speed
@speed.setter
def speed(self, value):
    if 0 <= value <= self.__max_speed:
        self.__speed = value
    else:
        raise ValueError("Недопустимая скорость")
    \end{verbatim}  
    Продемонстрировать работу на трёх экземплярах и сделать выводы об оптимизации кода по сравнению с первым подходом.
    \item \textbf{С использованием модуля \texttt{accessify}}:  
    Установить модуль командой \texttt{pip install accessify} и импортировать:  
    \begin{verbatim}
from accessify import private, protected
    \end{verbatim}  
    Сделать поля \texttt{max\_speed}, \texttt{capacity}, \texttt{fuel\_tank}, \texttt{engine\_oil\_capacity}, \texttt{luggage\_spaces} по-настоящему приватными с помощью функции \texttt{private} (например, как атрибуты класса до \texttt{\_\_init\_\_}). Удалить их из инициализатора.  
    Проверки в сеттерах реализовать через вспомогательные методы, помеченные декоратором \texttt{@private}.  
    Учитывать, что методы с \texttt{@private} нельзя вызывать из методов, использующих \texttt{@property}, поэтому для этой версии использовать только классические геттеры и сеттеры (\texttt{get\_...}, \texttt{set\_...}).  
    Продемонстрировать, что попытка доступа извне (включая \texttt{mysnow3.\_Snowmobile\_\_max\_speed}) \textbf{не даёт результата}, а вызов приватного метода или чтение приватного поля вызывает ошибку доступа.
\end{enumerate}
Для всех трёх подходов создать по три экземпляра снегохода, установить значения полей с учётом всех ограничений и вывести текущие значения всех полей каждого экземпляра.
\item[21] Разработать класс \texttt{ATV}, который будет описывать модель вездехода (quad bike). В классе должны быть следующие поля с доступом уровня \textbf{private} (только внутри класса):
\begin{itemize}
    \item \texttt{\_\_speed}: скорость движения вездехода  
    \item \texttt{\_\_distance}: расстояние, которое вездеход проехал  
    \item \texttt{\_\_max\_speed}: максимальная разрешённая скорость движения вездехода  
    \item \texttt{\_\_passengers}: список пассажиров  
    \item \texttt{\_\_capacity}: максимальная вместимость пассажиров на вездеходе  
    \item \texttt{\_\_empty\_seats}: число свободных мест  
    \item \texttt{\_\_seats\_occupied}: число занятых мест на вездеходе  
    \item \texttt{\_\_fuel\_tank}: объём топливного бака  
    \item \texttt{\_\_fuel}: количество топлива в литрах  
    \item \texttt{\_\_engine\_oil\_capacity}: объём картера масла двигателя (литры)  
    \item \texttt{\_\_engine\_oil}: количество моторного масла в литрах  
    \item \texttt{\_\_luggage\_spaces}: количество багажных мест  
    \item \texttt{\_\_luggage}: багаж вездехода  
\end{itemize}
Уровень доступа к полям должен быть следующим:
\begin{itemize}
    \item \texttt{\_\_max\_speed}, \texttt{\_\_capacity}, \texttt{\_\_fuel\_tank}, \texttt{\_\_engine\_oil\_capacity}, \texttt{\_\_luggage\_spaces}: \textbf{только чтение} (через геттеры)  
    \item \texttt{\_\_speed}, \texttt{\_\_distance}, \texttt{\_\_passengers}, \texttt{\_\_empty\_seats}, \texttt{\_\_seats\_occupied}, \texttt{\_\_fuel}, \texttt{\_\_engine\_oil}, \texttt{\_\_luggage}: \textbf{чтение и запись} (через геттеры и сеттеры)
\end{itemize}
Требования к сеттерам:
\begin{itemize}
    \item Для полей \texttt{\_\_empty\_seats} и \texttt{\_\_seats\_occupied} в сеттерах необходимо проверять, что передаваемое значение не превышает \texttt{\_\_capacity} и неотрицательно.  
    \item Для поля \texttt{\_\_passengers} в сеттере необходимо проверять, что количество пассажиров (длина списка) не превышает \texttt{\_\_capacity}.  
    \item Для поля \texttt{\_\_speed} в сеттере необходимо проверять, что заданная скорость не превышает \texttt{\_\_max\_speed} и неотрицательна.  
    \item Для поля \texttt{\_\_luggage} в сеттере необходимо проверять, что количество единиц багажа не превышает \texttt{\_\_luggage\_spaces}.
    \item Для полей \texttt{\_\_fuel} и \texttt{\_\_engine\_oil} значения не должны превышать соответствующие ёмкости (\texttt{\_\_fuel\_tank} и \texttt{\_\_engine\_oil\_capacity}) и должны быть неотрицательными.
\end{itemize}
Реализовать метод вывода всех установленных через сеттеры значений закрытых полей экземпляра класса.
На основе этого класса реализовать три подхода к управлению доступом:
\begin{enumerate}
    \item \textbf{С использованием объекта \texttt{property}}:  
    Для каждого поля определить отдельные методы-геттеры и сеттеры (например, \texttt{get\_speed}, \texttt{set\_speed}), а затем создать свойство:  
    \begin{verbatim}
speed = property(get_speed, set_speed)
    \end{verbatim}  
    Этот код должен располагаться после определения соответствующих методов. Первый аргумент — геттер, второй — сеттер.  
    Продемонстрировать работу на трёх экземплярах класса: создать \texttt{myatv1}, \texttt{myatv2}, \texttt{myatv3}, установить значения через свойства и вывести их.
    \item \textbf{С использованием декораторов \texttt{@property} и \texttt{@<имя>.setter}}:  
    Создать новую версию класса, в которой геттеры оформляются с декоратором \texttt{@property}, а сеттеры — с декоратором вида \texttt{@speed.setter}. Имена методов должны совпадать и не содержать префиксов \texttt{get\_}/\texttt{set\_}.  
    Пример:  
    \begin{verbatim}
@property
def speed(self):
    return self.__speed
@speed.setter
def speed(self, value):
    if 0 <= value <= self.__max_speed:
        self.__speed = value
    else:
        raise ValueError("Недопустимая скорость")
    \end{verbatim}  
    Продемонстрировать работу на трёх экземплярах и сделать выводы об оптимизации кода по сравнению с первым подходом.
    \item \textbf{С использованием модуля \texttt{accessify}}:  
    Установить модуль командой \texttt{pip install accessify} и импортировать:  
    \begin{verbatim}
from accessify import private, protected
    \end{verbatim}  
    Сделать поля \texttt{max\_speed}, \texttt{capacity}, \texttt{fuel\_tank}, \texttt{engine\_oil\_capacity}, \texttt{luggage\_spaces} по-настоящему приватными с помощью функции \texttt{private} (например, как атрибуты класса до \texttt{\_\_init\_\_}). Удалить их из инициализатора.  
    Проверки в сеттерах реализовать через вспомогательные методы, помеченные декоратором \texttt{@private}.  
    Учитывать, что методы с \texttt{@private} нельзя вызывать из методов, использующих \texttt{@property}, поэтому для этой версии использовать только классические геттеры и сеттеры (\texttt{get\_...}, \texttt{set\_...}).  
    Продемонстрировать, что попытка доступа извне (включая \texttt{myatv3.\_ATV\_\_max\_speed}) \textbf{не даёт результата}, а вызов приватного метода или чтение приватного поля вызывает ошибку доступа.
\end{enumerate}
Для всех трёх подходов создать по три экземпляра вездехода, установить значения полей с учётом всех ограничений и вывести текущие значения всех полей каждого экземпляра.
\item[22] Разработать класс \texttt{Hovercraft}, который будет описывать модель судна на воздушной подушке. В классе должны быть следующие поля с доступом уровня \textbf{private} (только внутри класса):
\begin{itemize}
    \item \texttt{\_\_speed}: скорость движения судна на воздушной подушке  
    \item \texttt{\_\_distance}: расстояние, которое судно на воздушной подушке прошло  
    \item \texttt{\_\_max\_speed}: максимальная разрешённая скорость движения судна на воздушной подушке  
    \item \texttt{\_\_passengers}: список пассажиров  
    \item \texttt{\_\_capacity}: максимальная вместимость пассажиров на судне на воздушной подушке  
    \item \texttt{\_\_empty\_seats}: число свободных мест  
    \item \texttt{\_\_seats\_occupied}: число занятых мест на судне на воздушной подушке  
    \item \texttt{\_\_fuel\_tank}: объём топливного бака  
    \item \texttt{\_\_fuel}: количество топлива в литрах  
    \item \texttt{\_\_engine\_oil\_capacity}: объём картера масла двигателя (литры)  
    \item \texttt{\_\_engine\_oil}: количество моторного масла в литрах  
    \item \texttt{\_\_luggage\_spaces}: количество багажных мест  
    \item \texttt{\_\_luggage}: багаж судна на воздушной подушке  
\end{itemize}
Уровень доступа к полям должен быть следующим:
\begin{itemize}
    \item \texttt{\_\_max\_speed}, \texttt{\_\_capacity}, \texttt{\_\_fuel\_tank}, \texttt{\_\_engine\_oil\_capacity}, \texttt{\_\_luggage\_spaces}: \textbf{только чтение} (через геттеры)  
    \item \texttt{\_\_speed}, \texttt{\_\_distance}, \texttt{\_\_passengers}, \texttt{\_\_empty\_seats}, \texttt{\_\_seats\_occupied}, \texttt{\_\_fuel}, \texttt{\_\_engine\_oil}, \texttt{\_\_luggage}: \textbf{чтение и запись} (через геттеры и сеттеры)
\end{itemize}
Требования к сеттерам:
\begin{itemize}
    \item Для полей \texttt{\_\_empty\_seats} и \texttt{\_\_seats\_occupied} в сеттерах необходимо проверять, что передаваемое значение не превышает \texttt{\_\_capacity} и неотрицательно.  
    \item Для поля \texttt{\_\_passengers} в сеттере необходимо проверять, что количество пассажиров (длина списка) не превышает \texttt{\_\_capacity}.  
    \item Для поля \texttt{\_\_speed} в сеттере необходимо проверять, что заданная скорость не превышает \texttt{\_\_max\_speed} и неотрицательна.  
    \item Для поля \texttt{\_\_luggage} в сеттере необходимо проверять, что количество единиц багажа не превышает \texttt{\_\_luggage\_spaces}.
    \item Для полей \texttt{\_\_fuel} и \texttt{\_\_engine\_oil} значения не должны превышать соответствующие ёмкости (\texttt{\_\_fuel\_tank} и \texttt{\_\_engine\_oil\_capacity}) и должны быть неотрицательными.
\end{itemize}
Реализовать метод вывода всех установленных через сеттеры значений закрытых полей экземпляра класса.
На основе этого класса реализовать три подхода к управлению доступом:
\begin{enumerate}
    \item \textbf{С использованием объекта \texttt{property}}:  
    Для каждого поля определить отдельные методы-геттеры и сеттеры (например, \texttt{get\_speed}, \texttt{set\_speed}), а затем создать свойство:  
    \begin{verbatim}
speed = property(get_speed, set_speed)
    \end{verbatim}  
    Этот код должен располагаться после определения соответствующих методов. Первый аргумент — геттер, второй — сеттер.  
    Продемонстрировать работу на трёх экземплярах класса: создать \texttt{myhover1}, \texttt{myhover2}, \texttt{myhover3}, установить значения через свойства и вывести их.
    \item \textbf{С использованием декораторов \texttt{@property} и \texttt{@<имя>.setter}}:  
    Создать новую версию класса, в которой геттеры оформляются с декоратором \texttt{@property}, а сеттеры — с декоратором вида \texttt{@speed.setter}. Имена методов должны совпадать и не содержать префиксов \texttt{get\_}/\texttt{set\_}.  
    Пример:  
    \begin{verbatim}
@property
def speed(self):
    return self.__speed
@speed.setter
def speed(self, value):
    if 0 <= value <= self.__max_speed:
        self.__speed = value
    else:
        raise ValueError("Недопустимая скорость")
    \end{verbatim}  
    Продемонстрировать работу на трёх экземплярах и сделать выводы об оптимизации кода по сравнению с первым подходом.
    \item \textbf{С использованием модуля \texttt{accessify}}:  
    Установить модуль командой \texttt{pip install accessify} и импортировать:  
    \begin{verbatim}
from accessify import private, protected
    \end{verbatim}  
    Сделать поля \texttt{max\_speed}, \texttt{capacity}, \texttt{fuel\_tank}, \texttt{engine\_oil\_capacity}, \texttt{luggage\_spaces} по-настоящему приватными с помощью функции \texttt{private} (например, как атрибуты класса до \texttt{\_\_init\_\_}). Удалить их из инициализатора.  
    Проверки в сеттерах реализовать через вспомогательные методы, помеченные декоратором \texttt{@private}.  
    Учитывать, что методы с \texttt{@private} нельзя вызывать из методов, использующих \texttt{@property}, поэтому для этой версии использовать только классические геттеры и сеттеры (\texttt{get\_...}, \texttt{set\_...}).  
    Продемонстрировать, что попытка доступа извне (включая \texttt{myhover3.\_Hovercraft\_\_max\_speed}) \textbf{не даёт результата}, а вызов приватного метода или чтение приватного поля вызывает ошибку доступа.
\end{enumerate}
Для всех трёх подходов создать по три экземпляра судна на воздушной подушке, установить значения полей с учётом всех ограничений и вывести текущие значения всех полей каждого экземпляра.
\item[23] Разработать класс \texttt{Rocket}, который будет описывать модель ракеты. В классе должны быть следующие поля с доступом уровня \textbf{private} (только внутри класса):
\begin{itemize}
    \item \texttt{\_\_speed}: скорость движения ракеты  
    \item \texttt{\_\_distance}: расстояние, которое ракета пролетела  
    \item \texttt{\_\_max\_speed}: максимальная разрешённая скорость движения ракеты  
    \item \texttt{\_\_passengers}: список пассажиров  
    \item \texttt{\_\_capacity}: максимальная вместимость пассажиров в ракете  
    \item \texttt{\_\_empty\_seats}: число свободных мест  
    \item \texttt{\_\_seats\_occupied}: число занятых мест в ракете  
    \item \texttt{\_\_fuel\_tank}: объём топливного бака  
    \item \texttt{\_\_fuel}: количество топлива в литрах  
    \item \texttt{\_\_engine\_oil\_capacity}: объём картера масла двигателя (литры)  
    \item \texttt{\_\_engine\_oil}: количество моторного масла в литрах  
    \item \texttt{\_\_luggage\_spaces}: количество багажных мест  
    \item \texttt{\_\_luggage}: багаж ракеты  
\end{itemize}
Уровень доступа к полям должен быть следующим:
\begin{itemize}
    \item \texttt{\_\_max\_speed}, \texttt{\_\_capacity}, \texttt{\_\_fuel\_tank}, \texttt{\_\_engine\_oil\_capacity}, \texttt{\_\_luggage\_spaces}: \textbf{только чтение} (через геттеры)  
    \item \texttt{\_\_speed}, \texttt{\_\_distance}, \texttt{\_\_passengers}, \texttt{\_\_empty\_seats}, \texttt{\_\_seats\_occupied}, \texttt{\_\_fuel}, \texttt{\_\_engine\_oil}, \texttt{\_\_luggage}: \textbf{чтение и запись} (через геттеры и сеттеры)
\end{itemize}
Требования к сеттерам:
\begin{itemize}
    \item Для полей \texttt{\_\_empty\_seats} и \texttt{\_\_seats\_occupied} в сеттерах необходимо проверять, что передаваемое значение не превышает \texttt{\_\_capacity} и неотрицательно.  
    \item Для поля \texttt{\_\_passengers} в сеттере необходимо проверять, что количество пассажиров (длина списка) не превышает \texttt{\_\_capacity}.  
    \item Для поля \texttt{\_\_speed} в сеттере необходимо проверять, что заданная скорость не превышает \texttt{\_\_max\_speed} и неотрицательна.  
    \item Для поля \texttt{\_\_luggage} в сеттере необходимо проверять, что количество единиц багажа не превышает \texttt{\_\_luggage\_spaces}.
    \item Для полей \texttt{\_\_fuel} и \texttt{\_\_engine\_oil} значения не должны превышать соответствующие ёмкости (\texttt{\_\_fuel\_tank} и \texttt{\_\_engine\_oil\_capacity}) и должны быть неотрицательными.
\end{itemize}
Реализовать метод вывода всех установленных через сеттеры значений закрытых полей экземпляра класса.
На основе этого класса реализовать три подхода к управлению доступом:
\begin{enumerate}
    \item \textbf{С использованием объекта \texttt{property}}:  
    Для каждого поля определить отдельные методы-геттеры и сеттеры (например, \texttt{get\_speed}, \texttt{set\_speed}), а затем создать свойство:  
    \begin{verbatim}
speed = property(get_speed, set_speed)
    \end{verbatim}  
    Этот код должен располагаться после определения соответствующих методов. Первый аргумент — геттер, второй — сеттер.  
    Продемонстрировать работу на трёх экземплярах класса: создать \texttt{myrocket1}, \texttt{myrocket2}, \texttt{myrocket3}, установить значения через свойства и вывести их.
    \item \textbf{С использованием декораторов \texttt{@property} и \texttt{@<имя>.setter}}:  
    Создать новую версию класса, в которой геттеры оформляются с декоратором \texttt{@property}, а сеттеры — с декоратором вида \texttt{@speed.setter}. Имена методов должны совпадать и не содержать префиксов \texttt{get\_}/\texttt{set\_}.  
    Пример:  
    \begin{verbatim}
@property
def speed(self):
    return self.__speed
@speed.setter
def speed(self, value):
    if 0 <= value <= self.__max_speed:
        self.__speed = value
    else:
        raise ValueError("Недопустимая скорость")
    \end{verbatim}  
    Продемонстрировать работу на трёх экземплярах и сделать выводы об оптимизации кода по сравнению с первым подходом.
    \item \textbf{С использованием модуля \texttt{accessify}}:  
    Установить модуль командой \texttt{pip install accessify} и импортировать:  
    \begin{verbatim}
from accessify import private, protected
    \end{verbatim}  
    Сделать поля \texttt{max\_speed}, \texttt{capacity}, \texttt{fuel\_tank}, \texttt{engine\_oil\_capacity}, \texttt{luggage\_spaces} по-настоящему приватными с помощью функции \texttt{private} (например, как атрибуты класса до \texttt{\_\_init\_\_}). Удалить их из инициализатора.  
    Проверки в сеттерах реализовать через вспомогательные методы, помеченные декоратором \texttt{@private}.  
    Учитывать, что методы с \texttt{@private} нельзя вызывать из методов, использующих \texttt{@property}, поэтому для этой версии использовать только классические геттеры и сеттеры (\texttt{get\_...}, \texttt{set\_...}).  
    Продемонстрировать, что попытка доступа извне (включая \texttt{myrocket3.\_Rocket\_\_max\_speed}) \textbf{не даёт результата}, а вызов приватного метода или чтение приватного поля вызывает ошибку доступа.
\end{enumerate}
Для всех трёх подходов создать по три экземпляра ракеты, установить значения полей с учётом всех ограничений и вывести текущие значения всех полей каждого экземпляра.
\item[24] Разработать класс \texttt{Glider}, который будет описывать модель планера. В классе должны быть следующие поля с доступом уровня \textbf{private} (только внутри класса):
\begin{itemize}
    \item \texttt{\_\_speed}: скорость движения планера  
    \item \texttt{\_\_distance}: расстояние, которое планер пролетел  
    \item \texttt{\_\_max\_speed}: максимальная разрешённая скорость движения планера  
    \item \texttt{\_\_passengers}: список пассажиров  
    \item \texttt{\_\_capacity}: максимальная вместимость пассажиров в планере  
    \item \texttt{\_\_empty\_seats}: число свободных мест  
    \item \texttt{\_\_seats\_occupied}: число занятых мест в планере  
    \item \texttt{\_\_fuel\_tank}: объём топливного бака  
    \item \texttt{\_\_fuel}: количество топлива в литрах  
    \item \texttt{\_\_engine\_oil\_capacity}: объём картера масла двигателя (литры)  
    \item \texttt{\_\_engine\_oil}: количество моторного масла в литрах  
    \item \texttt{\_\_luggage\_spaces}: количество багажных мест  
    \item \texttt{\_\_luggage}: багаж планера  
\end{itemize}
Уровень доступа к полям должен быть следующим:
\begin{itemize}
    \item \texttt{\_\_max\_speed}, \texttt{\_\_capacity}, \texttt{\_\_fuel\_tank}, \texttt{\_\_engine\_oil\_capacity}, \texttt{\_\_luggage\_spaces}: \textbf{только чтение} (через геттеры)  
    \item \texttt{\_\_speed}, \texttt{\_\_distance}, \texttt{\_\_passengers}, \texttt{\_\_empty\_seats}, \texttt{\_\_seats\_occupied}, \texttt{\_\_fuel}, \texttt{\_\_engine\_oil}, \texttt{\_\_luggage}: \textbf{чтение и запись} (через геттеры и сеттеры)
\end{itemize}
Требования к сеттерам:
\begin{itemize}
    \item Для полей \texttt{\_\_empty\_seats} и \texttt{\_\_seats\_occupied} в сеттерах необходимо проверять, что передаваемое значение не превышает \texttt{\_\_capacity} и неотрицательно.  
    \item Для поля \texttt{\_\_passengers} в сеттере необходимо проверять, что количество пассажиров (длина списка) не превышает \texttt{\_\_capacity}.  
    \item Для поля \texttt{\_\_speed} в сеттере необходимо проверять, что заданная скорость не превышает \texttt{\_\_max\_speed} и неотрицательна.  
    \item Для поля \texttt{\_\_luggage} в сеттере необходимо проверять, что количество единиц багажа не превышает \texttt{\_\_luggage\_spaces}.
    \item Для полей \texttt{\_\_fuel} и \texttt{\_\_engine\_oil} значения не должны превышать соответствующие ёмкости (\texttt{\_\_fuel\_tank} и \texttt{\_\_engine\_oil\_capacity}) и должны быть неотрицательными.
\end{itemize}
Реализовать метод вывода всех установленных через сеттеры значений закрытых полей экземпляра класса.
На основе этого класса реализовать три подхода к управлению доступом:
\begin{enumerate}
    \item \textbf{С использованием объекта \texttt{property}}:  
    Для каждого поля определить отдельные методы-геттеры и сеттеры (например, \texttt{get\_speed}, \texttt{set\_speed}), а затем создать свойство:  
    \begin{verbatim}
speed = property(get_speed, set_speed)
    \end{verbatim}  
    Этот код должен располагаться после определения соответствующих методов. Первый аргумент — геттер, второй — сеттер.  
    Продемонстрировать работу на трёх экземплярах класса: создать \texttt{myglider1}, \texttt{myglider2}, \texttt{myglider3}, установить значения через свойства и вывести их.
    \item \textbf{С использованием декораторов \texttt{@property} и \texttt{@<имя>.setter}}:  
    Создать новую версию класса, в которой геттеры оформляются с декоратором \texttt{@property}, а сеттеры — с декоратором вида \texttt{@speed.setter}. Имена методов должны совпадать и не содержать префиксов \texttt{get\_}/\texttt{set\_}.  
    Пример:  
    \begin{verbatim}
@property
def speed(self):
    return self.__speed
@speed.setter
def speed(self, value):
    if 0 <= value <= self.__max_speed:
        self.__speed = value
    else:
        raise ValueError("Недопустимая скорость")
    \end{verbatim}  
    Продемонстрировать работу на трёх экземплярах и сделать выводы об оптимизации кода по сравнению с первым подходом.
    \item \textbf{С использованием модуля \texttt{accessify}}:  
    Установить модуль командой \texttt{pip install accessify} и импортировать:  
    \begin{verbatim}
from accessify import private, protected
    \end{verbatim}  
    Сделать поля \texttt{max\_speed}, \texttt{capacity}, \texttt{fuel\_tank}, \texttt{engine\_oil\_capacity}, \texttt{luggage\_spaces} по-настоящему приватными с помощью функции \texttt{private} (например, как атрибуты класса до \texttt{\_\_init\_\_}). Удалить их из инициализатора.  
    Проверки в сеттерах реализовать через вспомогательные методы, помеченные декоратором \texttt{@private}.  
    Учитывать, что методы с \texttt{@private} нельзя вызывать из методов, использующих \texttt{@property}, поэтому для этой версии использовать только классические геттеры и сеттеры (\texttt{get\_...}, \texttt{set\_...}).  
    Продемонстрировать, что попытка доступа извне (включая \texttt{myglider3.\_Glider\_\_max\_speed}) \textbf{не даёт результата}, а вызов приватного метода или чтение приватного поля вызывает ошибку доступа.
\end{enumerate}
Для всех трёх подходов создать по три экземпляра планера, установить значения полей с учётом всех ограничений и вывести текущие значения всех полей каждого экземпляра.
\item[25] Разработать класс \texttt{Zeppelin}, который будет описывать модель дирижабля. В классе должны быть следующие поля с доступом уровня \textbf{private} (только внутри класса):
\begin{itemize}
    \item \texttt{\_\_speed}: скорость движения дирижабля  
    \item \texttt{\_\_distance}: расстояние, которое дирижабль пролетел  
    \item \texttt{\_\_max\_speed}: максимальная разрешённая скорость движения дирижабля  
    \item \texttt{\_\_passengers}: список пассажиров  
    \item \texttt{\_\_capacity}: максимальная вместимость пассажиров в дирижабле  
    \item \texttt{\_\_empty\_seats}: число свободных мест  
    \item \texttt{\_\_seats\_occupied}: число занятых мест в дирижабле  
    \item \texttt{\_\_fuel\_tank}: объём топливного бака  
    \item \texttt{\_\_fuel}: количество топлива в литрах  
    \item \texttt{\_\_engine\_oil\_capacity}: объём картера масла двигателя (литры)  
    \item \texttt{\_\_engine\_oil}: количество моторного масла в литрах  
    \item \texttt{\_\_luggage\_spaces}: количество багажных мест  
    \item \texttt{\_\_luggage}: багаж дирижабля  
\end{itemize}
Уровень доступа к полям должен быть следующим:
\begin{itemize}
    \item \texttt{\_\_max\_speed}, \texttt{\_\_capacity}, \texttt{\_\_fuel\_tank}, \texttt{\_\_engine\_oil\_capacity}, \texttt{\_\_luggage\_spaces}: \textbf{только чтение} (через геттеры)  
    \item \texttt{\_\_speed}, \texttt{\_\_distance}, \texttt{\_\_passengers}, \texttt{\_\_empty\_seats}, \texttt{\_\_seats\_occupied}, \texttt{\_\_fuel}, \texttt{\_\_engine\_oil}, \texttt{\_\_luggage}: \textbf{чтение и запись} (через геттеры и сеттеры)
\end{itemize}
Требования к сеттерам:
\begin{itemize}
    \item Для полей \texttt{\_\_empty\_seats} и \texttt{\_\_seats\_occupied} в сеттерах необходимо проверять, что передаваемое значение не превышает \texttt{\_\_capacity} и неотрицательно.  
    \item Для поля \texttt{\_\_passengers} в сеттере необходимо проверять, что количество пассажиров (длина списка) не превышает \texttt{\_\_capacity}.  
    \item Для поля \texttt{\_\_speed} в сеттере необходимо проверять, что заданная скорость не превышает \texttt{\_\_max\_speed} и неотрицательна.  
    \item Для поля \texttt{\_\_luggage} в сеттере необходимо проверять, что количество единиц багажа не превышает \texttt{\_\_luggage\_spaces}.
    \item Для полей \texttt{\_\_fuel} и \texttt{\_\_engine\_oil} значения не должны превышать соответствующие ёмкости (\texttt{\_\_fuel\_tank} и \texttt{\_\_engine\_oil\_capacity}) и должны быть неотрицательными.
\end{itemize}
Реализовать метод вывода всех установленных через сеттеры значений закрытых полей экземпляра класса.
На основе этого класса реализовать три подхода к управлению доступом:
\begin{enumerate}
    \item \textbf{С использованием объекта \texttt{property}}:  
    Для каждого поля определить отдельные методы-геттеры и сеттеры (например, \texttt{get\_speed}, \texttt{set\_speed}), а затем создать свойство:  
    \begin{verbatim}
speed = property(get_speed, set_speed)
    \end{verbatim}  
    Этот код должен располагаться после определения соответствующих методов. Первый аргумент — геттер, второй — сеттер.  
    Продемонстрировать работу на трёх экземплярах класса: создать \texttt{myzep1}, \texttt{myzep2}, \texttt{myzep3}, установить значения через свойства и вывести их.
    \item \textbf{С использованием декораторов \texttt{@property} и \texttt{@<имя>.setter}}:  
    Создать новую версию класса, в которой геттеры оформляются с декоратором \texttt{@property}, а сеттеры — с декоратором вида \texttt{@speed.setter}. Имена методов должны совпадать и не содержать префиксов \texttt{get\_}/\texttt{set\_}.  
    Пример:  
    \begin{verbatim}
@property
def speed(self):
    return self.__speed
@speed.setter
def speed(self, value):
    if 0 <= value <= self.__max_speed:
        self.__speed = value
    else:
        raise ValueError("Недопустимая скорость")
    \end{verbatim}  
    Продемонстрировать работу на трёх экземплярах и сделать выводы об оптимизации кода по сравнению с первым подходом.
    \item \textbf{С использованием модуля \texttt{accessify}}:  
    Установить модуль командой \texttt{pip install accessify} и импортировать:  
    \begin{verbatim}
from accessify import private, protected
    \end{verbatim}  
    Сделать поля \texttt{max\_speed}, \texttt{capacity}, \texttt{fuel\_tank}, \texttt{engine\_oil\_capacity}, \texttt{luggage\_spaces} по-настоящему приватными с помощью функции \texttt{private} (например, как атрибуты класса до \texttt{\_\_init\_\_}). Удалить их из инициализатора.  
    Проверки в сеттерах реализовать через вспомогательные методы, помеченные декоратором \texttt{@private}.  
    Учитывать, что методы с \texttt{@private} нельзя вызывать из методов, использующих \texttt{@property}, поэтому для этой версии использовать только классические геттеры и сеттеры (\texttt{get\_...}, \texttt{set\_...}).  
    Продемонстрировать, что попытка доступа извне (включая \texttt{myzep3.\_Zeppelin\_\_max\_speed}) \textbf{не даёт результата}, а вызов приватного метода или чтение приватного поля вызывает ошибку доступа.
\end{enumerate}
Для всех трёх подходов создать по три экземпляра дирижабля, установить значения полей с учётом всех ограничений и вывести текущие значения всех полей каждого экземпляра.
\item[26] Разработать класс \texttt{Ferry}, который будет описывать модель парома. В классе должны быть следующие поля с доступом уровня \textbf{private} (только внутри класса):
\begin{itemize}
    \item \texttt{\_\_speed}: скорость движения парома  
    \item \texttt{\_\_distance}: расстояние, которое паром прошёл  
    \item \texttt{\_\_max\_speed}: максимальная разрешённая скорость движения парома  
    \item \texttt{\_\_passengers}: список пассажиров  
    \item \texttt{\_\_capacity}: максимальная вместимость пассажиров на пароме  
    \item \texttt{\_\_empty\_seats}: число свободных мест  
    \item \texttt{\_\_seats\_occupied}: число занятых мест на пароме  
    \item \texttt{\_\_fuel\_tank}: объём топливного бака  
    \item \texttt{\_\_fuel}: количество топлива в литрах  
    \item \texttt{\_\_engine\_oil\_capacity}: объём картера масла двигателя (литры)  
    \item \texttt{\_\_engine\_oil}: количество моторного масла в литрах  
    \item \texttt{\_\_luggage\_spaces}: количество багажных мест  
    \item \texttt{\_\_luggage}: багаж парома  
\end{itemize}
Уровень доступа к полям должен быть следующим:
\begin{itemize}
    \item \texttt{\_\_max\_speed}, \texttt{\_\_capacity}, \texttt{\_\_fuel\_tank}, \texttt{\_\_engine\_oil\_capacity}, \texttt{\_\_luggage\_spaces}: \textbf{только чтение} (через геттеры)  
    \item \texttt{\_\_speed}, \texttt{\_\_distance}, \texttt{\_\_passengers}, \texttt{\_\_empty\_seats}, \texttt{\_\_seats\_occupied}, \texttt{\_\_fuel}, \texttt{\_\_engine\_oil}, \texttt{\_\_luggage}: \textbf{чтение и запись} (через геттеры и сеттеры)
\end{itemize}
Требования к сеттерам:
\begin{itemize}
    \item Для полей \texttt{\_\_empty\_seats} и \texttt{\_\_seats\_occupied} в сеттерах необходимо проверять, что передаваемое значение не превышает \texttt{\_\_capacity} и неотрицательно.  
    \item Для поля \texttt{\_\_passengers} в сеттере необходимо проверять, что количество пассажиров (длина списка) не превышает \texttt{\_\_capacity}.  
    \item Для поля \texttt{\_\_speed} в сеттере необходимо проверять, что заданная скорость не превышает \texttt{\_\_max\_speed} и неотрицательна.  
    \item Для поля \texttt{\_\_luggage} в сеттере необходимо проверять, что количество единиц багажа не превышает \texttt{\_\_luggage\_spaces}.
    \item Для полей \texttt{\_\_fuel} и \texttt{\_\_engine\_oil} значения не должны превышать соответствующие ёмкости (\texttt{\_\_fuel\_tank} и \texttt{\_\_engine\_oil\_capacity}) и должны быть неотрицательными.
\end{itemize}
Реализовать метод вывода всех установленных через сеттеры значений закрытых полей экземпляра класса.
На основе этого класса реализовать три подхода к управлению доступом:
\begin{enumerate}
    \item \textbf{С использованием объекта \texttt{property}}:  
    Для каждого поля определить отдельные методы-геттеры и сеттеры (например, \texttt{get\_speed}, \texttt{set\_speed}), а затем создать свойство:  
    \begin{verbatim}
speed = property(get_speed, set_speed)
    \end{verbatim}  
    Этот код должен располагаться после определения соответствующих методов. Первый аргумент — геттер, второй — сеттер.  
    Продемонстрировать работу на трёх экземплярах класса: создать \texttt{myferry1}, \texttt{myferry2}, \texttt{myferry3}, установить значения через свойства и вывести их.
    \item \textbf{С использованием декораторов \texttt{@property} и \texttt{@<имя>.setter}}:  
    Создать новую версию класса, в которой геттеры оформляются с декоратором \texttt{@property}, а сеттеры — с декоратором вида \texttt{@speed.setter}. Имена методов должны совпадать и не содержать префиксов \texttt{get\_}/\texttt{set\_}.  
    Пример:  
    \begin{verbatim}
@property
def speed(self):
    return self.__speed
@speed.setter
def speed(self, value):
    if 0 <= value <= self.__max_speed:
        self.__speed = value
    else:
        raise ValueError("Недопустимая скорость")
    \end{verbatim}  
    Продемонстрировать работу на трёх экземплярах и сделать выводы об оптимизации кода по сравнению с первым подходом.
    \item \textbf{С использованием модуля \texttt{accessify}}:  
    Установить модуль командой \texttt{pip install accessify} и импортировать:  
    \begin{verbatim}
from accessify import private, protected
    \end{verbatim}  
    Сделать поля \texttt{max\_speed}, \texttt{capacity}, \texttt{fuel\_tank}, \texttt{engine\_oil\_capacity}, \texttt{luggage\_spaces} по-настоящему приватными с помощью функции \texttt{private} (например, как атрибуты класса до \texttt{\_\_init\_\_}). Удалить их из инициализатора.  
    Проверки в сеттерах реализовать через вспомогательные методы, помеченные декоратором \texttt{@private}.  
    Учитывать, что методы с \texttt{@private} нельзя вызывать из методов, использующих \texttt{@property}, поэтому для этой версии использовать только классические геттеры и сеттеры (\texttt{get\_...}, \texttt{set\_...}).  
    Продемонстрировать, что попытка доступа извне (включая \texttt{myferry3.\_Ferry\_\_max\_speed}) \textbf{не даёт результата}, а вызов приватного метода или чтение приватного поля вызывает ошибку доступа.
\end{enumerate}
Для всех трёх подходов создать по три экземпляра парома, установить значения полей с учётом всех ограничений и вывести текущие значения всех полей каждого экземпляра.
\item[27] Разработать класс \texttt{Yacht}, который будет описывать модель яхты. В классе должны быть следующие поля с доступом уровня \textbf{private} (только внутри класса):
\begin{itemize}
    \item \texttt{\_\_speed}: скорость движения яхты  
    \item \texttt{\_\_distance}: расстояние, которое яхта прошла  
    \item \texttt{\_\_max\_speed}: максимальная разрешённая скорость движения яхты  
    \item \texttt{\_\_passengers}: список пассажиров  
    \item \texttt{\_\_capacity}: максимальная вместимость пассажиров на яхте  
    \item \texttt{\_\_empty\_seats}: число свободных мест  
    \item \texttt{\_\_seats\_occupied}: число занятых мест на яхте  
    \item \texttt{\_\_fuel\_tank}: объём топливного бака  
    \item \texttt{\_\_fuel}: количество топлива в литрах  
    \item \texttt{\_\_engine\_oil\_capacity}: объём картера масла двигателя (литры)  
    \item \texttt{\_\_engine\_oil}: количество моторного масла в литрах  
    \item \texttt{\_\_luggage\_spaces}: количество багажных мест  
    \item \texttt{\_\_luggage}: багаж яхты  
\end{itemize}
Уровень доступа к полям должен быть следующим:
\begin{itemize}
    \item \texttt{\_\_max\_speed}, \texttt{\_\_capacity}, \texttt{\_\_fuel\_tank}, \texttt{\_\_engine\_oil\_capacity}, \texttt{\_\_luggage\_spaces}: \textbf{только чтение} (через геттеры)  
    \item \texttt{\_\_speed}, \texttt{\_\_distance}, \texttt{\_\_passengers}, \texttt{\_\_empty\_seats}, \texttt{\_\_seats\_occupied}, \texttt{\_\_fuel}, \texttt{\_\_engine\_oil}, \texttt{\_\_luggage}: \textbf{чтение и запись} (через геттеры и сеттеры)
\end{itemize}
Требования к сеттерам:
\begin{itemize}
    \item Для полей \texttt{\_\_empty\_seats} и \texttt{\_\_seats\_occupied} в сеттерах необходимо проверять, что передаваемое значение не превышает \texttt{\_\_capacity} и неотрицательно.  
    \item Для поля \texttt{\_\_passengers} в сеттере необходимо проверять, что количество пассажиров (длина списка) не превышает \texttt{\_\_capacity}.  
    \item Для поля \texttt{\_\_speed} в сеттере необходимо проверять, что заданная скорость не превышает \texttt{\_\_max\_speed} и неотрицательна.  
    \item Для поля \texttt{\_\_luggage} в сеттере необходимо проверять, что количество единиц багажа не превышает \texttt{\_\_luggage\_spaces}.
    \item Для полей \texttt{\_\_fuel} и \texttt{\_\_engine\_oil} значения не должны превышать соответствующие ёмкости (\texttt{\_\_fuel\_tank} и \texttt{\_\_engine\_oil\_capacity}) и должны быть неотрицательными.
\end{itemize}
Реализовать метод вывода всех установленных через сеттеры значений закрытых полей экземпляра класса.
На основе этого класса реализовать три подхода к управлению доступом:
\begin{enumerate}
    \item \textbf{С использованием объекта \texttt{property}}:  
    Для каждого поля определить отдельные методы-геттеры и сеттеры (например, \texttt{get\_speed}, \texttt{set\_speed}), а затем создать свойство:  
    \begin{verbatim}
speed = property(get_speed, set_speed)
    \end{verbatim}  
    Этот код должен располагаться после определения соответствующих методов. Первый аргумент — геттер, второй — сеттер.  
    Продемонстрировать работу на трёх экземплярах класса: создать \texttt{myyacht1}, \texttt{myyacht2}, \texttt{myyacht3}, установить значения через свойства и вывести их.
    \item \textbf{С использованием декораторов \texttt{@property} и \texttt{@<имя>.setter}}:  
    Создать новую версию класса, в которой геттеры оформляются с декоратором \texttt{@property}, а сеттеры — с декоратором вида \texttt{@speed.setter}. Имена методов должны совпадать и не содержать префиксов \texttt{get\_}/\texttt{set\_}.  
    Пример:  
    \begin{verbatim}
@property
def speed(self):
    return self.__speed
@speed.setter
def speed(self, value):
    if 0 <= value <= self.__max_speed:
        self.__speed = value
    else:
        raise ValueError("Недопустимая скорость")
    \end{verbatim}  
    Продемонстрировать работу на трёх экземплярах и сделать выводы об оптимизации кода по сравнению с первым подходом.
    \item \textbf{С использованием модуля \texttt{accessify}}:  
    Установить модуль командой \texttt{pip install accessify} и импортировать:  
    \begin{verbatim}
from accessify import private, protected
    \end{verbatim}  
    Сделать поля \texttt{max\_speed}, \texttt{capacity}, \texttt{fuel\_tank}, \texttt{engine\_oil\_capacity}, \texttt{luggage\_spaces} по-настоящему приватными с помощью функции \texttt{private} (например, как атрибуты класса до \texttt{\_\_init\_\_}). Удалить их из инициализатора.  
    Проверки в сеттерах реализовать через вспомогательные методы, помеченные декоратором \texttt{@private}.  
    Учитывать, что методы с \texttt{@private} нельзя вызывать из методов, использующих \texttt{@property}, поэтому для этой версии использовать только классические геттеры и сеттеры (\texttt{get\_...}, \texttt{set\_...}).  
    Продемонстрировать, что попытка доступа извне (включая \texttt{myyacht3.\_Yacht\_\_max\_speed}) \textbf{не даёт результата}, а вызов приватного метода или чтение приватного поля вызывает ошибку доступа.
\end{enumerate}
Для всех трёх подходов создать по три экземпляра яхты, установить значения полей с учётом всех ограничений и вывести текущие значения всех полей каждого экземпляра.
\item[28] Разработать класс \texttt{Speedboat}, который будет описывать модель быстроходной лодки. В классе должны быть следующие поля с доступом уровня \textbf{private} (только внутри класса):
\begin{itemize}
    \item \texttt{\_\_speed}: скорость движения быстроходной лодки  
    \item \texttt{\_\_distance}: расстояние, которое быстроходная лодка прошла  
    \item \texttt{\_\_max\_speed}: максимальная разрешённая скорость движения быстроходной лодки  
    \item \texttt{\_\_passengers}: список пассажиров  
    \item \texttt{\_\_capacity}: максимальная вместимость пассажиров на быстроходной лодке  
    \item \texttt{\_\_empty\_seats}: число свободных мест  
    \item \texttt{\_\_seats\_occupied}: число занятых мест на быстроходной лодке  
    \item \texttt{\_\_fuel\_tank}: объём топливного бака  
    \item \texttt{\_\_fuel}: количество топлива в литрах  
    \item \texttt{\_\_engine\_oil\_capacity}: объём картера масла двигателя (литры)  
    \item \texttt{\_\_engine\_oil}: количество моторного масла в литрах  
    \item \texttt{\_\_luggage\_spaces}: количество багажных мест  
    \item \texttt{\_\_luggage}: багаж быстроходной лодки  
\end{itemize}
Уровень доступа к полям должен быть следующим:
\begin{itemize}
    \item \texttt{\_\_max\_speed}, \texttt{\_\_capacity}, \texttt{\_\_fuel\_tank}, \texttt{\_\_engine\_oil\_capacity}, \texttt{\_\_luggage\_spaces}: \textbf{только чтение} (через геттеры)  
    \item \texttt{\_\_speed}, \texttt{\_\_distance}, \texttt{\_\_passengers}, \texttt{\_\_empty\_seats}, \texttt{\_\_seats\_occupied}, \texttt{\_\_fuel}, \texttt{\_\_engine\_oil}, \texttt{\_\_luggage}: \textbf{чтение и запись} (через геттеры и сеттеры)
\end{itemize}
Требования к сеттерам:
\begin{itemize}
    \item Для полей \texttt{\_\_empty\_seats} и \texttt{\_\_seats\_occupied} в сеттерах необходимо проверять, что передаваемое значение не превышает \texttt{\_\_capacity} и неотрицательно.  
    \item Для поля \texttt{\_\_passengers} в сеттере необходимо проверять, что количество пассажиров (длина списка) не превышает \texttt{\_\_capacity}.  
    \item Для поля \texttt{\_\_speed} в сеттере необходимо проверять, что заданная скорость не превышает \texttt{\_\_max\_speed} и неотрицательна.  
    \item Для поля \texttt{\_\_luggage} в сеттере необходимо проверять, что количество единиц багажа не превышает \texttt{\_\_luggage\_spaces}.
    \item Для полей \texttt{\_\_fuel} и \texttt{\_\_engine\_oil} значения не должны превышать соответствующие ёмкости (\texttt{\_\_fuel\_tank} и \texttt{\_\_engine\_oil\_capacity}) и должны быть неотрицательными.
\end{itemize}
Реализовать метод вывода всех установленных через сеттеры значений закрытых полей экземпляра класса.
На основе этого класса реализовать три подхода к управлению доступом:
\begin{enumerate}
    \item \textbf{С использованием объекта \texttt{property}}:  
    Для каждого поля определить отдельные методы-геттеры и сеттеры (например, \texttt{get\_speed}, \texttt{set\_speed}), а затем создать свойство:  
    \begin{verbatim}
speed = property(get_speed, set_speed)
    \end{verbatim}  
    Этот код должен располагаться после определения соответствующих методов. Первый аргумент — геттер, второй — сеттер.  
    Продемонстрировать работу на трёх экземплярах класса: создать \texttt{myspeed1}, \texttt{myspeed2}, \texttt{myspeed3}, установить значения через свойства и вывести их.
    \item \textbf{С использованием декораторов \texttt{@property} и \texttt{@<имя>.setter}}:  
    Создать новую версию класса, в которой геттеры оформляются с декоратором \texttt{@property}, а сеттеры — с декоратором вида \texttt{@speed.setter}. Имена методов должны совпадать и не содержать префиксов \texttt{get\_}/\texttt{set\_}.  
    Пример:  
    \begin{verbatim}
@property
def speed(self):
    return self.__speed
@speed.setter
def speed(self, value):
    if 0 <= value <= self.__max_speed:
        self.__speed = value
    else:
        raise ValueError("Недопустимая скорость")
    \end{verbatim}  
    Продемонстрировать работу на трёх экземплярах и сделать выводы об оптимизации кода по сравнению с первым подходом.
    \item \textbf{С использованием модуля \texttt{accessify}}:  
    Установить модуль командой \texttt{pip install accessify} и импортировать:  
    \begin{verbatim}
from accessify import private, protected
    \end{verbatim}  
    Сделать поля \texttt{max\_speed}, \texttt{capacity}, \texttt{fuel\_tank}, \texttt{engine\_oil\_capacity}, \texttt{luggage\_spaces} по-настоящему приватными с помощью функции \texttt{private} (например, как атрибуты класса до \texttt{\_\_init\_\_}). Удалить их из инициализатора.  
    Проверки в сеттерах реализовать через вспомогательные методы, помеченные декоратором \texttt{@private}.  
    Учитывать, что методы с \texttt{@private} нельзя вызывать из методов, использующих \texttt{@property}, поэтому для этой версии использовать только классические геттеры и сеттеры (\texttt{get\_...}, \texttt{set\_...}).  
    Продемонстрировать, что попытка доступа извне (включая \texttt{myspeed3.\_Speedboat\_\_max\_speed}) \textbf{не даёт результата}, а вызов приватного метода или чтение приватного поля вызывает ошибку доступа.
\end{enumerate}
Для всех трёх подходов создать по три экземпляра быстроходной лодки, установить значения полей с учётом всех ограничений и вывести текущие значения всех полей каждого экземпляра.
\item[29] Разработать класс \texttt{CargoPlane}, который будет описывать модель грузового самолёта. В классе должны быть следующие поля с доступом уровня \textbf{private} (только внутри класса):
\begin{itemize}
    \item \texttt{\_\_speed}: скорость движения грузового самолёта  
    \item \texttt{\_\_distance}: расстояние, которое грузовой самолёт пролетел  
    \item \texttt{\_\_max\_speed}: максимальная разрешённая скорость движения грузового самолёта  
    \item \texttt{\_\_passengers}: список пассажиров  
    \item \texttt{\_\_capacity}: максимальная вместимость пассажиров в грузовом самолёте  
    \item \texttt{\_\_empty\_seats}: число свободных мест  
    \item \texttt{\_\_seats\_occupied}: число занятых мест в грузовом самолёте  
    \item \texttt{\_\_fuel\_tank}: объём топливного бака  
    \item \texttt{\_\_fuel}: количество топлива в литрах  
    \item \texttt{\_\_engine\_oil\_capacity}: объём картера масла двигателя (литры)  
    \item \texttt{\_\_engine\_oil}: количество моторного масла в литрах  
    \item \texttt{\_\_luggage\_spaces}: количество багажных мест  
    \item \texttt{\_\_luggage}: багаж грузового самолёта  
\end{itemize}
Уровень доступа к полям должен быть следующим:
\begin{itemize}
    \item \texttt{\_\_max\_speed}, \texttt{\_\_capacity}, \texttt{\_\_fuel\_tank}, \texttt{\_\_engine\_oil\_capacity}, \texttt{\_\_luggage\_spaces}: \textbf{только чтение} (через геттеры)  
    \item \texttt{\_\_speed}, \texttt{\_\_distance}, \texttt{\_\_passengers}, \texttt{\_\_empty\_seats}, \texttt{\_\_seats\_occupied}, \texttt{\_\_fuel}, \texttt{\_\_engine\_oil}, \texttt{\_\_luggage}: \textbf{чтение и запись} (через геттеры и сеттеры)
\end{itemize}
Требования к сеттерам:
\begin{itemize}
    \item Для полей \texttt{\_\_empty\_seats} и \texttt{\_\_seats\_occupied} в сеттерах необходимо проверять, что передаваемое значение не превышает \texttt{\_\_capacity} и неотрицательно.  
    \item Для поля \texttt{\_\_passengers} в сеттере необходимо проверять, что количество пассажиров (длина списка) не превышает \texttt{\_\_capacity}.  
    \item Для поля \texttt{\_\_speed} в сеттере необходимо проверять, что заданная скорость не превышает \texttt{\_\_max\_speed} и неотрицательна.  
    \item Для поля \texttt{\_\_luggage} в сеттере необходимо проверять, что количество единиц багажа не превышает \texttt{\_\_luggage\_spaces}.
    \item Для полей \texttt{\_\_fuel} и \texttt{\_\_engine\_oil} значения не должны превышать соответствующие ёмкости (\texttt{\_\_fuel\_tank} и \texttt{\_\_engine\_oil\_capacity}) и должны быть неотрицательными.
\end{itemize}
Реализовать метод вывода всех установленных через сеттеры значений закрытых полей экземпляра класса.
На основе этого класса реализовать три подхода к управлению доступом:
\begin{enumerate}
    \item \textbf{С использованием объекта \texttt{property}}:  
    Для каждого поля определить отдельные методы-геттеры и сеттеры (например, \texttt{get\_speed}, \texttt{set\_speed}), а затем создать свойство:  
    \begin{verbatim}
speed = property(get_speed, set_speed)
    \end{verbatim}  
    Этот код должен располагаться после определения соответствующих методов. Первый аргумент — геттер, второй — сеттер.  
    Продемонстрировать работу на трёх экземплярах класса: создать \texttt{mycargo1}, \texttt{mycargo2}, \texttt{mycargo3}, установить значения через свойства и вывести их.
    \item \textbf{С использованием декораторов \texttt{@property} и \texttt{@<имя>.setter}}:  
    Создать новую версию класса, в которой геттеры оформляются с декоратором \texttt{@property}, а сеттеры — с декоратором вида \texttt{@speed.setter}. Имена методов должны совпадать и не содержать префиксов \texttt{get\_}/\texttt{set\_}.  
    Пример:  
    \begin{verbatim}
@property
def speed(self):
    return self.__speed
@speed.setter
def speed(self, value):
    if 0 <= value <= self.__max_speed:
        self.__speed = value
    else:
        raise ValueError("Недопустимая скорость")
    \end{verbatim}  
    Продемонстрировать работу на трёх экземплярах и сделать выводы об оптимизации кода по сравнению с первым подходом.
    \item \textbf{С использованием модуля \texttt{accessify}}:  
    Установить модуль командой \texttt{pip install accessify} и импортировать:  
    \begin{verbatim}
from accessify import private, protected
    \end{verbatim}  
    Сделать поля \texttt{max\_speed}, \texttt{capacity}, \texttt{fuel\_tank}, \texttt{engine\_oil\_capacity}, \texttt{luggage\_spaces} по-настоящему приватными с помощью функции \texttt{private} (например, как атрибуты класса до \texttt{\_\_init\_\_}). Удалить их из инициализатора.  
    Проверки в сеттерах реализовать через вспомогательные методы, помеченные декоратором \texttt{@private}.  
    Учитывать, что методы с \texttt{@private} нельзя вызывать из методов, использующих \texttt{@property}, поэтому для этой версии использовать только классические геттеры и сеттеры (\texttt{get\_...}, \texttt{set\_...}).  
    Продемонстрировать, что попытка доступа извне (включая \texttt{mycargo3.\_CargoPlane\_\_max\_speed}) \textbf{не даёт результата}, а вызов приватного метода или чтение приватного поля вызывает ошибку доступа.
\end{enumerate}
Для всех трёх подходов создать по три экземпляра грузового самолёта, установить значения полей с учётом всех ограничений и вывести текущие значения всех полей каждого экземпляра.
\item[30] Разработать класс \texttt{PassengerPlane}, который будет описывать модель пассажирского самолёта. В классе должны быть следующие поля с доступом уровня \textbf{private} (только внутри класса):
\begin{itemize}
    \item \texttt{\_\_speed}: скорость движения пассажирского самолёта  
    \item \texttt{\_\_distance}: расстояние, которое пассажирский самолёт пролетел  
    \item \texttt{\_\_max\_speed}: максимальная разрешённая скорость движения пассажирского самолёта  
    \item \texttt{\_\_passengers}: список пассажиров  
    \item \texttt{\_\_capacity}: максимальная вместимость пассажиров в пассажирском самолёте  
    \item \texttt{\_\_empty\_seats}: число свободных мест  
    \item \texttt{\_\_seats\_occupied}: число занятых мест в пассажирском самолёте  
    \item \texttt{\_\_fuel\_tank}: объём топливного бака  
    \item \texttt{\_\_fuel}: количество топлива в литрах  
    \item \texttt{\_\_engine\_oil\_capacity}: объём картера масла двигателя (литры)  
    \item \texttt{\_\_engine\_oil}: количество моторного масла в литрах  
    \item \texttt{\_\_luggage\_spaces}: количество багажных мест  
    \item \texttt{\_\_luggage}: багаж пассажирского самолёта  
\end{itemize}
Уровень доступа к полям должен быть следующим:
\begin{itemize}
    \item \texttt{\_\_max\_speed}, \texttt{\_\_capacity}, \texttt{\_\_fuel\_tank}, \texttt{\_\_engine\_oil\_capacity}, \texttt{\_\_luggage\_spaces}: \textbf{только чтение} (через геттеры)  
    \item \texttt{\_\_speed}, \texttt{\_\_distance}, \texttt{\_\_passengers}, \texttt{\_\_empty\_seats}, \texttt{\_\_seats\_occupied}, \texttt{\_\_fuel}, \texttt{\_\_engine\_oil}, \texttt{\_\_luggage}: \textbf{чтение и запись} (через геттеры и сеттеры)
\end{itemize}
Требования к сеттерам:
\begin{itemize}
    \item Для полей \texttt{\_\_empty\_seats} и \texttt{\_\_seats\_occupied} в сеттерах необходимо проверять, что передаваемое значение не превышает \texttt{\_\_capacity} и неотрицательно.  
    \item Для поля \texttt{\_\_passengers} в сеттере необходимо проверять, что количество пассажиров (длина списка) не превышает \texttt{\_\_capacity}.  
    \item Для поля \texttt{\_\_speed} в сеттере необходимо проверять, что заданная скорость не превышает \texttt{\_\_max\_speed} и неотрицательна.  
    \item Для поля \texttt{\_\_luggage} в сеттере необходимо проверять, что количество единиц багажа не превышает \texttt{\_\_luggage\_spaces}.
    \item Для полей \texttt{\_\_fuel} и \texttt{\_\_engine\_oil} значения не должны превышать соответствующие ёмкости (\texttt{\_\_fuel\_tank} и \texttt{\_\_engine\_oil\_capacity}) и должны быть неотрицательными.
\end{itemize}
Реализовать метод вывода всех установленных через сеттеры значений закрытых полей экземпляра класса.
На основе этого класса реализовать три подхода к управлению доступом:
\begin{enumerate}
    \item \textbf{С использованием объекта \texttt{property}}:  
    Для каждого поля определить отдельные методы-геттеры и сеттеры (например, \texttt{get\_speed}, \texttt{set\_speed}), а затем создать свойство:  
    \begin{verbatim}
speed = property(get_speed, set_speed)
    \end{verbatim}  
    Этот код должен располагаться после определения соответствующих методов. Первый аргумент — геттер, второй — сеттер.  
    Продемонстрировать работу на трёх экземплярах класса: создать \texttt{mypass1}, \texttt{mypass2}, \texttt{mypass3}, установить значения через свойства и вывести их.
    \item \textbf{С использованием декораторов \texttt{@property} и \texttt{@<имя>.setter}}:  
    Создать новую версию класса, в которой геттеры оформляются с декоратором \texttt{@property}, а сеттеры — с декоратором вида \texttt{@speed.setter}. Имена методов должны совпадать и не содержать префиксов \texttt{get\_}/\texttt{set\_}.  
    Пример:  
    \begin{verbatim}
@property
def speed(self):
    return self.__speed
@speed.setter
def speed(self, value):
    if 0 <= value <= self.__max_speed:
        self.__speed = value
    else:
        raise ValueError("Недопустимая скорость")
    \end{verbatim}  
    Продемонстрировать работу на трёх экземплярах и сделать выводы об оптимизации кода по сравнению с первым подходом.
    \item \textbf{С использованием модуля \texttt{accessify}}:  
    Установить модуль командой \texttt{pip install accessify} и импортировать:  
    \begin{verbatim}
from accessify import private, protected
    \end{verbatim}  
    Сделать поля \texttt{max\_speed}, \texttt{capacity}, \texttt{fuel\_tank}, \texttt{engine\_oil\_capacity}, \texttt{luggage\_spaces} по-настоящему приватными с помощью функции \texttt{private} (например, как атрибуты класса до \texttt{\_\_init\_\_}). Удалить их из инициализатора.  
    Проверки в сеттерах реализовать через вспомогательные методы, помеченные декоратором \texttt{@private}.  
    Учитывать, что методы с \texttt{@private} нельзя вызывать из методов, использующих \texttt{@property}, поэтому для этой версии использовать только классические геттеры и сеттеры (\texttt{get\_...}, \texttt{set\_...}).  
    Продемонстрировать, что попытка доступа извне (включая \texttt{mypass3.\_PassengerPlane\_\_max\_speed}) \textbf{не даёт результата}, а вызов приватного метода или чтение приватного поля вызывает ошибку доступа.
\end{enumerate}
Для всех трёх подходов создать по три экземпляра пассажирского самолёта, установить значения полей с учётом всех ограничений и вывести текущие значения всех полей каждого экземпляра.
\item[31] Разработать класс \texttt{MetroCar}, который будет описывать модель вагона метро. В классе должны быть следующие поля с доступом уровня \textbf{private} (только внутри класса):
\begin{itemize}
    \item \texttt{\_\_speed}: скорость движения вагона метро  
    \item \texttt{\_\_distance}: расстояние, которое вагон метро проехал  
    \item \texttt{\_\_max\_speed}: максимальная разрешённая скорость движения вагона метро  
    \item \texttt{\_\_passengers}: список пассажиров  
    \item \texttt{\_\_capacity}: максимальная вместимость пассажиров в вагоне метро  
    \item \texttt{\_\_empty\_seats}: число свободных мест  
    \item \texttt{\_\_seats\_occupied}: число занятых мест в вагоне метро  
    \item \texttt{\_\_fuel\_tank}: объём топливного бака  
    \item \texttt{\_\_fuel}: количество топлива в литрах  
    \item \texttt{\_\_engine\_oil\_capacity}: объём картера масла двигателя (литры)  
    \item \texttt{\_\_engine\_oil}: количество моторного масла в литрах  
    \item \texttt{\_\_luggage\_spaces}: количество багажных мест  
    \item \texttt{\_\_luggage}: багаж вагона метро  
\end{itemize}
Уровень доступа к полям должен быть следующим:
\begin{itemize}
    \item \texttt{\_\_max\_speed}, \texttt{\_\_capacity}, \texttt{\_\_fuel\_tank}, \texttt{\_\_engine\_oil\_capacity}, \texttt{\_\_luggage\_spaces}: \textbf{только чтение} (через геттеры)  
    \item \texttt{\_\_speed}, \texttt{\_\_distance}, \texttt{\_\_passengers}, \texttt{\_\_empty\_seats}, \texttt{\_\_seats\_occupied}, \texttt{\_\_fuel}, \texttt{\_\_engine\_oil}, \texttt{\_\_luggage}: \textbf{чтение и запись} (через геттеры и сеттеры)
\end{itemize}
Требования к сеттерам:
\begin{itemize}
    \item Для полей \texttt{\_\_empty\_seats} и \texttt{\_\_seats\_occupied} в сеттерах необходимо проверять, что передаваемое значение не превышает \texttt{\_\_capacity} и неотрицательно.  
    \item Для поля \texttt{\_\_passengers} в сеттере необходимо проверять, что количество пассажиров (длина списка) не превышает \texttt{\_\_capacity}.  
    \item Для поля \texttt{\_\_speed} в сеттере необходимо проверять, что заданная скорость не превышает \texttt{\_\_max\_speed} и неотрицательна.  
    \item Для поля \texttt{\_\_luggage} в сеттере необходимо проверять, что количество единиц багажа не превышает \texttt{\_\_luggage\_spaces}.
    \item Для полей \texttt{\_\_fuel} и \texttt{\_\_engine\_oil} значения не должны превышать соответствующие ёмкости (\texttt{\_\_fuel\_tank} и \texttt{\_\_engine\_oil\_capacity}) и должны быть неотрицательными.
\end{itemize}
Реализовать метод вывода всех установленных через сеттеры значений закрытых полей экземпляра класса.
На основе этого класса реализовать три подхода к управлению доступом:
\begin{enumerate}
    \item \textbf{С использованием объекта \texttt{property}}:  
    Для каждого поля определить отдельные методы-геттеры и сеттеры (например, \texttt{get\_speed}, \texttt{set\_speed}), а затем создать свойство:  
    \begin{verbatim}
speed = property(get_speed, set_speed)
    \end{verbatim}  
    Этот код должен располагаться после определения соответствующих методов. Первый аргумент — геттер, второй — сеттер.  
    Продемонстрировать работу на трёх экземплярах класса: создать \texttt{mymetro1}, \texttt{mymetro2}, \texttt{mymetro3}, установить значения через свойства и вывести их.
    \item \textbf{С использованием декораторов \texttt{@property} и \texttt{@<имя>.setter}}:  
    Создать новую версию класса, в которой геттеры оформляются с декоратором \texttt{@property}, а сеттеры — с декоратором вида \texttt{@speed.setter}. Имена методов должны совпадать и не содержать префиксов \texttt{get\_}/\texttt{set\_}.  
    Пример:  
    \begin{verbatim}
@property
def speed(self):
    return self.__speed
@speed.setter
def speed(self, value):
    if 0 <= value <= self.__max_speed:
        self.__speed = value
    else:
        raise ValueError("Недопустимая скорость")
    \end{verbatim}  
    Продемонстрировать работу на трёх экземплярах и сделать выводы об оптимизации кода по сравнению с первым подходом.
    \item \textbf{С использованием модуля \texttt{accessify}}:  
    Установить модуль командой \texttt{pip install accessify} и импортировать:  
    \begin{verbatim}
from accessify import private, protected
    \end{verbatim}  
    Сделать поля \texttt{max\_speed}, \texttt{capacity}, \texttt{fuel\_tank}, \texttt{engine\_oil\_capacity}, \texttt{luggage\_spaces} по-настоящему приватными с помощью функции \texttt{private} (например, как атрибуты класса до \texttt{\_\_init\_\_}). Удалить их из инициализатора.  
    Проверки в сеттерах реализовать через вспомогательные методы, помеченные декоратором \texttt{@private}.  
    Учитывать, что методы с \texttt{@private} нельзя вызывать из методов, использующих \texttt{@property}, поэтому для этой версии использовать только классические геттеры и сеттеры (\texttt{get\_...}, \texttt{set\_...}).  
    Продемонстрировать, что попытка доступа извне (включая \texttt{mymetro3.\_MetroCar\_\_max\_speed}) \textbf{не даёт результата}, а вызов приватного метода или чтение приватного поля вызывает ошибку доступа.
\end{enumerate}
Для всех трёх подходов создать по три экземпляра вагона метро, установить значения полей с учётом всех ограничений и вывести текущие значения всех полей каждого экземпляра.
\item[32] Разработать класс \texttt{Trolleybus}, который будет описывать модель троллейбуса. В классе должны быть следующие поля с доступом уровня \textbf{private} (только внутри класса):
\begin{itemize}
    \item \texttt{\_\_speed}: скорость движения троллейбуса  
    \item \texttt{\_\_distance}: расстояние, которое троллейбус проехал  
    \item \texttt{\_\_max\_speed}: максимальная разрешённая скорость движения троллейбуса  
    \item \texttt{\_\_passengers}: список пассажиров  
    \item \texttt{\_\_capacity}: максимальная вместимость пассажиров в троллейбусе  
    \item \texttt{\_\_empty\_seats}: число свободных мест  
    \item \texttt{\_\_seats\_occupied}: число занятых мест в троллейбусе  
    \item \texttt{\_\_fuel\_tank}: объём топливного бака  
    \item \texttt{\_\_fuel}: количество топлива в литрах  
    \item \texttt{\_\_engine\_oil\_capacity}: объём картера масла двигателя (литры)  
    \item \texttt{\_\_engine\_oil}: количество моторного масла в литрах  
    \item \texttt{\_\_luggage\_spaces}: количество багажных мест  
    \item \texttt{\_\_luggage}: багаж троллейбуса  
\end{itemize}
Уровень доступа к полям должен быть следующим:
\begin{itemize}
    \item \texttt{\_\_max\_speed}, \texttt{\_\_capacity}, \texttt{\_\_fuel\_tank}, \texttt{\_\_engine\_oil\_capacity}, \texttt{\_\_luggage\_spaces}: \textbf{только чтение} (через геттеры)  
    \item \texttt{\_\_speed}, \texttt{\_\_distance}, \texttt{\_\_passengers}, \texttt{\_\_empty\_seats}, \texttt{\_\_seats\_occupied}, \texttt{\_\_fuel}, \texttt{\_\_engine\_oil}, \texttt{\_\_luggage}: \textbf{чтение и запись} (через геттеры и сеттеры)
\end{itemize}
Требования к сеттерам:
\begin{itemize}
    \item Для полей \texttt{\_\_empty\_seats} и \texttt{\_\_seats\_occupied} в сеттерах необходимо проверять, что передаваемое значение не превышает \texttt{\_\_capacity} и неотрицательно.  
    \item Для поля \texttt{\_\_passengers} в сеттере необходимо проверять, что количество пассажиров (длина списка) не превышает \texttt{\_\_capacity}.  
    \item Для поля \texttt{\_\_speed} в сеттере необходимо проверять, что заданная скорость не превышает \texttt{\_\_max\_speed} и неотрицательна.  
    \item Для поля \texttt{\_\_luggage} в сеттере необходимо проверять, что количество единиц багажа не превышает \texttt{\_\_luggage\_spaces}.
    \item Для полей \texttt{\_\_fuel} и \texttt{\_\_engine\_oil} значения не должны превышать соответствующие ёмкости (\texttt{\_\_fuel\_tank} и \texttt{\_\_engine\_oil\_capacity}) и должны быть неотрицательными.
\end{itemize}
Реализовать метод вывода всех установленных через сеттеры значений закрытых полей экземпляра класса.
На основе этого класса реализовать три подхода к управлению доступом:
\begin{enumerate}
    \item \textbf{С использованием объекта \texttt{property}}:  
    Для каждого поля определить отдельные методы-геттеры и сеттеры (например, \texttt{get\_speed}, \texttt{set\_speed}), а затем создать свойство:  
    \begin{verbatim}
speed = property(get_speed, set_speed)
    \end{verbatim}  
    Этот код должен располагаться после определения соответствующих методов. Первый аргумент — геттер, второй — сеттер.  
    Продемонстрировать работу на трёх экземплярах класса: создать \texttt{mytrol1}, \texttt{mytrol2}, \texttt{mytrol3}, установить значения через свойства и вывести их.
    \item \textbf{С использованием декораторов \texttt{@property} и \texttt{@<имя>.setter}}:  
    Создать новую версию класса, в которой геттеры оформляются с декоратором \texttt{@property}, а сеттеры — с декоратором вида \texttt{@speed.setter}. Имена методов должны совпадать и не содержать префиксов \texttt{get\_}/\texttt{set\_}.  
    Пример:  
    \begin{verbatim}
@property
def speed(self):
    return self.__speed
@speed.setter
def speed(self, value):
    if 0 <= value <= self.__max_speed:
        self.__speed = value
    else:
        raise ValueError("Недопустимая скорость")
    \end{verbatim}  
    Продемонстрировать работу на трёх экземплярах и сделать выводы об оптимизации кода по сравнению с первым подходом.
    \item \textbf{С использованием модуля \texttt{accessify}}:  
    Установить модуль командой \texttt{pip install accessify} и импортировать:  
    \begin{verbatim}
from accessify import private, protected
    \end{verbatim}  
    Сделать поля \texttt{max\_speed}, \texttt{capacity}, \texttt{fuel\_tank}, \texttt{engine\_oil\_capacity}, \texttt{luggage\_spaces} по-настоящему приватными с помощью функции \texttt{private} (например, как атрибуты класса до \texttt{\_\_init\_\_}). Удалить их из инициализатора.  
    Проверки в сеттерах реализовать через вспомогательные методы, помеченные декоратором \texttt{@private}.  
    Учитывать, что методы с \texttt{@private} нельзя вызывать из методов, использующих \texttt{@property}, поэтому для этой версии использовать только классические геттеры и сеттеры (\texttt{get\_...}, \texttt{set\_...}).  
    Продемонстрировать, что попытка доступа извне (включая \texttt{mytrol3.\_Trolleybus\_\_max\_speed}) \textbf{не даёт результата}, а вызов приватного метода или чтение приватного поля вызывает ошибку доступа.
\end{enumerate}
Для всех трёх подходов создать по три экземпляра троллейбуса, установить значения полей с учётом всех ограничений и вывести текущие значения всех полей каждого экземпляра.
\item[33] Разработать класс \texttt{ElectricCar}, который будет описывать модель электромобиля. В классе должны быть следующие поля с доступом уровня \textbf{private} (только внутри класса):
\begin{itemize}
    \item \texttt{\_\_speed}: скорость движения электромобиля  
    \item \texttt{\_\_distance}: расстояние, которое электромобиль проехал  
    \item \texttt{\_\_max\_speed}: максимальная разрешённая скорость движения электромобиля  
    \item \texttt{\_\_passengers}: список пассажиров  
    \item \texttt{\_\_capacity}: максимальная вместимость пассажиров в электромобиле  
    \item \texttt{\_\_empty\_seats}: число свободных мест  
    \item \texttt{\_\_seats\_occupied}: число занятых мест в электромобиле  
    \item \texttt{\_\_fuel\_tank}: объём топливного бака  
    \item \texttt{\_\_fuel}: количество топлива в литрах  
    \item \texttt{\_\_engine\_oil\_capacity}: объём картера масла двигателя (литры)  
    \item \texttt{\_\_engine\_oil}: количество моторного масла в литрах  
    \item \texttt{\_\_luggage\_spaces}: количество багажных мест  
    \item \texttt{\_\_luggage}: багаж электромобиля  
\end{itemize}
Уровень доступа к полям должен быть следующим:
\begin{itemize}
    \item \texttt{\_\_max\_speed}, \texttt{\_\_capacity}, \texttt{\_\_fuel\_tank}, \texttt{\_\_engine\_oil\_capacity}, \texttt{\_\_luggage\_spaces}: \textbf{только чтение} (через геттеры)  
    \item \texttt{\_\_speed}, \texttt{\_\_distance}, \texttt{\_\_passengers}, \texttt{\_\_empty\_seats}, \texttt{\_\_seats\_occupied}, \texttt{\_\_fuel}, \texttt{\_\_engine\_oil}, \texttt{\_\_luggage}: \textbf{чтение и запись} (через геттеры и сеттеры)
\end{itemize}
Требования к сеттерам:
\begin{itemize}
    \item Для полей \texttt{\_\_empty\_seats} и \texttt{\_\_seats\_occupied} в сеттерах необходимо проверять, что передаваемое значение не превышает \texttt{\_\_capacity} и неотрицательно.  
    \item Для поля \texttt{\_\_passengers} в сеттере необходимо проверять, что количество пассажиров (длина списка) не превышает \texttt{\_\_capacity}.  
    \item Для поля \texttt{\_\_speed} в сеттере необходимо проверять, что заданная скорость не превышает \texttt{\_\_max\_speed} и неотрицательна.  
    \item Для поля \texttt{\_\_luggage} в сеттере необходимо проверять, что количество единиц багажа не превышает \texttt{\_\_luggage\_spaces}.
    \item Для полей \texttt{\_\_fuel} и \texttt{\_\_engine\_oil} значения не должны превышать соответствующие ёмкости (\texttt{\_\_fuel\_tank} и \texttt{\_\_engine\_oil\_capacity}) и должны быть неотрицательными.
\end{itemize}
Реализовать метод вывода всех установленных через сеттеры значений закрытых полей экземпляра класса.
На основе этого класса реализовать три подхода к управлению доступом:
\begin{enumerate}
    \item \textbf{С использованием объекта \texttt{property}}:  
    Для каждого поля определить отдельные методы-геттеры и сеттеры (например, \texttt{get\_speed}, \texttt{set\_speed}), а затем создать свойство:  
    \begin{verbatim}
speed = property(get_speed, set_speed)
    \end{verbatim}  
    Этот код должен располагаться после определения соответствующих методов. Первый аргумент — геттер, второй — сеттер.  
    Продемонстрировать работу на трёх экземплярах класса: создать \texttt{myev1}, \texttt{myev2}, \texttt{myev3}, установить значения через свойства и вывести их.
    \item \textbf{С использованием декораторов \texttt{@property} и \texttt{@<имя>.setter}}:  
    Создать новую версию класса, в которой геттеры оформляются с декоратором \texttt{@property}, а сеттеры — с декоратором вида \texttt{@speed.setter}. Имена методов должны совпадать и не содержать префиксов \texttt{get\_}/\texttt{set\_}.  
    Пример:  
    \begin{verbatim}
@property
def speed(self):
    return self.__speed
@speed.setter
def speed(self, value):
    if 0 <= value <= self.__max_speed:
        self.__speed = value
    else:
        raise ValueError("Недопустимая скорость")
    \end{verbatim}  
    Продемонстрировать работу на трёх экземплярах и сделать выводы об оптимизации кода по сравнению с первым подходом.
    \item \textbf{С использованием модуля \texttt{accessify}}:  
    Установить модуль командой \texttt{pip install accessify} и импортировать:  
    \begin{verbatim}
from accessify import private, protected
    \end{verbatim}  
    Сделать поля \texttt{max\_speed}, \texttt{capacity}, \texttt{fuel\_tank}, \texttt{engine\_oil\_capacity}, \texttt{luggage\_spaces} по-настоящему приватными с помощью функции \texttt{private} (например, как атрибуты класса до \texttt{\_\_init\_\_}). Удалить их из инициализатора.  
    Проверки в сеттерах реализовать через вспомогательные методы, помеченные декоратором \texttt{@private}.  
    Учитывать, что методы с \texttt{@private} нельзя вызывать из методов, использующих \texttt{@property}, поэтому для этой версии использовать только классические геттеры и сеттеры (\texttt{get\_...}, \texttt{set\_...}).  
    Продемонстрировать, что попытка доступа извне (включая \texttt{myev3.\_ElectricCar\_\_max\_speed}) \textbf{не даёт результата}, а вызов приватного метода или чтение приватного поля вызывает ошибку доступа.
\end{enumerate}
Для всех трёх подходов создать по три экземпляра электромобиля, установить значения полей с учётом всех ограничений и вывести текущие значения всех полей каждого экземпляра.
\item[34] Разработать класс \texttt{Hydrofoil}, который будет описывать модель гидрофойла. В классе должны быть следующие поля с доступом уровня \textbf{private} (только внутри класса):
\begin{itemize}
    \item \texttt{\_\_speed}: скорость движения гидрофойла  
    \item \texttt{\_\_distance}: расстояние, которое гидрофойл прошёл  
    \item \texttt{\_\_max\_speed}: максимальная разрешённая скорость движения гидрофойла  
    \item \texttt{\_\_passengers}: список пассажиров  
    \item \texttt{\_\_capacity}: максимальная вместимость пассажиров на гидрофойле  
    \item \texttt{\_\_empty\_seats}: число свободных мест  
    \item \texttt{\_\_seats\_occupied}: число занятых мест на гидрофойле  
    \item \texttt{\_\_fuel\_tank}: объём топливного бака  
    \item \texttt{\_\_fuel}: количество топлива в литрах  
    \item \texttt{\_\_engine\_oil\_capacity}: объём картера масла двигателя (литры)  
    \item \texttt{\_\_engine\_oil}: количество моторного масла в литрах  
    \item \texttt{\_\_luggage\_spaces}: количество багажных мест  
    \item \texttt{\_\_luggage}: багаж гидрофойла  
\end{itemize}
Уровень доступа к полям должен быть следующим:
\begin{itemize}
    \item \texttt{\_\_max\_speed}, \texttt{\_\_capacity}, \texttt{\_\_fuel\_tank}, \texttt{\_\_engine\_oil\_capacity}, \texttt{\_\_luggage\_spaces}: \textbf{только чтение} (через геттеры)  
    \item \texttt{\_\_speed}, \texttt{\_\_distance}, \texttt{\_\_passengers}, \texttt{\_\_empty\_seats}, \texttt{\_\_seats\_occupied}, \texttt{\_\_fuel}, \texttt{\_\_engine\_oil}, \texttt{\_\_luggage}: \textbf{чтение и запись} (через геттеры и сеттеры)
\end{itemize}
Требования к сеттерам:
\begin{itemize}
    \item Для полей \texttt{\_\_empty\_seats} и \texttt{\_\_seats\_occupied} в сеттерах необходимо проверять, что передаваемое значение не превышает \texttt{\_\_capacity} и неотрицательно.  
    \item Для поля \texttt{\_\_passengers} в сеттере необходимо проверять, что количество пассажиров (длина списка) не превышает \texttt{\_\_capacity}.  
    \item Для поля \texttt{\_\_speed} в сеттере необходимо проверять, что заданная скорость не превышает \texttt{\_\_max\_speed} и неотрицательна.  
    \item Для поля \texttt{\_\_luggage} в сеттере необходимо проверять, что количество единиц багажа не превышает \texttt{\_\_luggage\_spaces}.
    \item Для полей \texttt{\_\_fuel} и \texttt{\_\_engine\_oil} значения не должны превышать соответствующие ёмкости (\texttt{\_\_fuel\_tank} и \texttt{\_\_engine\_oil\_capacity}) и должны быть неотрицательными.
\end{itemize}
Реализовать метод вывода всех установленных через сеттеры значений закрытых полей экземпляра класса.
На основе этого класса реализовать три подхода к управлению доступом:
\begin{enumerate}
    \item \textbf{С использованием объекта \texttt{property}}:  
    Для каждого поля определить отдельные методы-геттеры и сеттеры (например, \texttt{get\_speed}, \texttt{set\_speed}), а затем создать свойство:  
    \begin{verbatim}
speed = property(get_speed, set_speed)
    \end{verbatim}  
    Этот код должен располагаться после определения соответствующих методов. Первый аргумент — геттер, второй — сеттер.  
    Продемонстрировать работу на трёх экземплярах класса: создать \texttt{myhydro1}, \texttt{myhydro2}, \texttt{myhydro3}, установить значения через свойства и вывести их.
    \item \textbf{С использованием декораторов \texttt{@property} и \texttt{@<имя>.setter}}:  
    Создать новую версию класса, в которой геттеры оформляются с декоратором \texttt{@property}, а сеттеры — с декоратором вида \texttt{@speed.setter}. Имена методов должны совпадать и не содержать префиксов \texttt{get\_}/\texttt{set\_}.  
    Пример:  
    \begin{verbatim}
@property
def speed(self):
    return self.__speed
@speed.setter
def speed(self, value):
    if 0 <= value <= self.__max_speed:
        self.__speed = value
    else:
        raise ValueError("Недопустимая скорость")
    \end{verbatim}  
    Продемонстрировать работу на трёх экземплярах и сделать выводы об оптимизации кода по сравнению с первым подходом.
    \item \textbf{С использованием модуля \texttt{accessify}}:  
    Установить модуль командой \texttt{pip install accessify} и импортировать:  
    \begin{verbatim}
from accessify import private, protected
    \end{verbatim}  
    Сделать поля \texttt{max\_speed}, \texttt{capacity}, \texttt{fuel\_tank}, \texttt{engine\_oil\_capacity}, \texttt{luggage\_spaces} по-настоящему приватными с помощью функции \texttt{private} (например, как атрибуты класса до \texttt{\_\_init\_\_}). Удалить их из инициализатора.  
    Проверки в сеттерах реализовать через вспомогательные методы, помеченные декоратором \texttt{@private}.  
    Учитывать, что методы с \texttt{@private} нельзя вызывать из методов, использующих \texttt{@property}, поэтому для этой версии использовать только классические геттеры и сеттеры (\texttt{get\_...}, \texttt{set\_...}).  
    Продемонстрировать, что попытка доступа извне (включая \texttt{myhydro3.\_Hydrofoil\_\_max\_speed}) \textbf{не даёт результата}, а вызов приватного метода или чтение приватного поля вызывает ошибку доступа.
\end{enumerate}
Для всех трёх подходов создать по три экземпляра гидрофойла, установить значения полей с учётом всех ограничений и вывести текущие значения всех полей каждого экземпляра.
\item[35] Разработать класс \texttt{Segway}, который будет описывать модель сигвея. В классе должны быть следующие поля с доступом уровня \textbf{private} (только внутри класса):
\begin{itemize}
    \item \texttt{\_\_speed}: скорость движения сигвея  
    \item \texttt{\_\_distance}: расстояние, которое сигвей проехал  
    \item \texttt{\_\_max\_speed}: максимальная разрешённая скорость движения сигвея  
    \item \texttt{\_\_passengers}: список пассажиров  
    \item \texttt{\_\_capacity}: максимальная вместимость пассажиров на сигвее  
    \item \texttt{\_\_empty\_seats}: число свободных мест  
    \item \texttt{\_\_seats\_occupied}: число занятых мест на сигвее  
    \item \texttt{\_\_fuel\_tank}: объём топливного бака  
    \item \texttt{\_\_fuel}: количество топлива в литрах  
    \item \texttt{\_\_engine\_oil\_capacity}: объём картера масла двигателя (литры)  
    \item \texttt{\_\_engine\_oil}: количество моторного масла в литрах  
    \item \texttt{\_\_luggage\_spaces}: количество багажных мест  
    \item \texttt{\_\_luggage}: багаж сигвея  
\end{itemize}
Уровень доступа к полям должен быть следующим:
\begin{itemize}
    \item \texttt{\_\_max\_speed}, \texttt{\_\_capacity}, \texttt{\_\_fuel\_tank}, \texttt{\_\_engine\_oil\_capacity}, \texttt{\_\_luggage\_spaces}: \textbf{только чтение} (через геттеры)  
    \item \texttt{\_\_speed}, \texttt{\_\_distance}, \texttt{\_\_passengers}, \texttt{\_\_empty\_seats}, \texttt{\_\_seats\_occupied}, \texttt{\_\_fuel}, \texttt{\_\_engine\_oil}, \texttt{\_\_luggage}: \textbf{чтение и запись} (через геттеры и сеттеры)
\end{itemize}
Требования к сеттерам:
\begin{itemize}
    \item Для полей \texttt{\_\_empty\_seats} и \texttt{\_\_seats\_occupied} в сеттерах необходимо проверять, что передаваемое значение не превышает \texttt{\_\_capacity} и неотрицательно.  
    \item Для поля \texttt{\_\_passengers} в сеттере необходимо проверять, что количество пассажиров (длина списка) не превышает \texttt{\_\_capacity}.  
    \item Для поля \texttt{\_\_speed} в сеттере необходимо проверять, что заданная скорость не превышает \texttt{\_\_max\_speed} и неотрицательна.  
    \item Для поля \texttt{\_\_luggage} в сеттере необходимо проверять, что количество единиц багажа не превышает \texttt{\_\_luggage\_spaces}.
    \item Для полей \texttt{\_\_fuel} и \texttt{\_\_engine\_oil} значения не должны превышать соответствующие ёмкости (\texttt{\_\_fuel\_tank} и \texttt{\_\_engine\_oil\_capacity}) и должны быть неотрицательными.
\end{itemize}
Реализовать метод вывода всех установленных через сеттеры значений закрытых полей экземпляра класса.
На основе этого класса реализовать три подхода к управлению доступом:
\begin{enumerate}
    \item \textbf{С использованием объекта \texttt{property}}:  
    Для каждого поля определить отдельные методы-геттеры и сеттеры (например, \texttt{get\_speed}, \texttt{set\_speed}), а затем создать свойство:  
    \begin{verbatim}
speed = property(get_speed, set_speed)
    \end{verbatim}  
    Этот код должен располагаться после определения соответствующих методов. Первый аргумент — геттер, второй — сеттер.  
    Продемонстрировать работу на трёх экземплярах класса: создать \texttt{myseg1}, \texttt{myseg2}, \texttt{myseg3}, установить значения через свойства и вывести их.
    \item \textbf{С использованием декораторов \texttt{@property} и \texttt{@<имя>.setter}}:  
    Создать новую версию класса, в которой геттеры оформляются с декоратором \texttt{@property}, а сеттеры — с декоратором вида \texttt{@speed.setter}. Имена методов должны совпадать и не содержать префиксов \texttt{get\_}/\texttt{set\_}.  
    Пример:  
    \begin{verbatim}
@property
def speed(self):
    return self.__speed
@speed.setter
def speed(self, value):
    if 0 <= value <= self.__max_speed:
        self.__speed = value
    else:
        raise ValueError("Недопустимая скорость")
    \end{verbatim}  
    Продемонстрировать работу на трёх экземплярах и сделать выводы об оптимизации кода по сравнению с первым подходом.
    \item \textbf{С использованием модуля \texttt{accessify}}:  
    Установить модуль командой \texttt{pip install accessify} и импортировать:  
    \begin{verbatim}
from accessify import private, protected
    \end{verbatim}  
    Сделать поля \texttt{max\_speed}, \texttt{capacity}, \texttt{fuel\_tank}, \texttt{engine\_oil\_capacity}, \texttt{luggage\_spaces} по-настоящему приватными с помощью функции \texttt{private} (например, как атрибуты класса до \texttt{\_\_init\_\_}). Удалить их из инициализатора.  
    Проверки в сеттерах реализовать через вспомогательные методы, помеченные декоратором \texttt{@private}.  
    Учитывать, что методы с \texttt{@private} нельзя вызывать из методов, использующих \texttt{@property}, поэтому для этой версии использовать только классические геттеры и сеттеры (\texttt{get\_...}, \texttt{set\_...}).  
    Продемонстрировать, что попытка доступа извне (включая \texttt{myseg3.\_Segway\_\_max\_speed}) \textbf{не даёт результата}, а вызов приватного метода или чтение приватного поля вызывает ошибку доступа.
\end{enumerate}
Для всех трёх подходов создать по три экземпляра сигвея, установить значения полей с учётом всех ограничений и вывести текущие значения всех полей каждого экземпляра.
\end{enumerate}

\subsubsection{Задача 2}



\textbf{Инструкция:} Напишите функцию и соответствующие unit-тесты, покрывающие все важные случаи. Для заданий с ветвлениями (\texttt{if}/\texttt{elif}/\texttt{else}) обязательно проверяйте все ветви.  

\textbf{Пояснения:}  
\begin{itemize}
    \item \textbf{Triangle types:} 
        \begin{itemize}
            \item \texttt{equilateral} — все стороны равны  
            \item \texttt{isosceles} — две стороны равны  
            \item \texttt{scalene} — все стороны разные  
            \item \texttt{invalid} — невозможно построить треугольник  
        \end{itemize}
    \item \textbf{BMI (Body Mass Index):} индекс массы тела. Категории: \texttt{Underweight}, \texttt{Normal}, \texttt{Overweight}, \texttt{Obese}  
    \item \textbf{Palindrome:} строка или число, читающееся одинаково слева направо и справа налево  
    \item \textbf{Perfect number:} число, равное сумме своих делителей, исключая само число  
    \item \textbf{Triangle angles:} 
        \begin{itemize}
            \item \texttt{acute} — все углы < 90°  
            \item \texttt{right} — один угол = 90°  
            \item \texttt{obtuse} — один угол > 90°  
            \item \texttt{invalid} — треугольник не существует  
        \end{itemize}
    \item \textbf{Traffic fine:} штраф за превышение скорости. Функция должна учитывать разные зоны (\texttt{residential}, \texttt{city}, \texttt{highway}) и уровни превышения скорости.
\end{itemize}

\begin{enumerate}
    \item \texttt{classify\_triangle(a, b, c)} — возвращает тип треугольника.  
    \item \texttt{classify\_number(n)} — возвращает `"positive even"`, `"positive odd"`, `"negative even"`, `"negative odd"`, `"zero"`.  
    \item \texttt{middle\_value(a, b, c)} — возвращает среднее (не арифметическое) число среди трёх, через сравнения.  
    \item \texttt{median\_of\_three(a, b, c)} — медиана трёх чисел через if/elif/else.  
    \item \texttt{is\_leap\_year(year)} — проверяет високосный год (делится на 4, но не на 100, или на 400).  
    \item \texttt{bmi\_category(weight, height)} — возвращает категорию BMI.  
    \item \texttt{categorize\_temperature(temp)} — диапазоны: `"freezing"` $\leqslant$ 0°C, `"cold"` 1–10°C, `"cool"` 11–20°C, `"warm"` 21–30°C, `"hot"` >30°C.  
    \item \texttt{triangle\_area\_type(a, b, c)} — возвращает `"acute"`, `"right"`, `"obtuse"` или `"invalid"`.  
    \item \texttt{quadrant(x, y)} — возвращает номер четверти (1–4) или `"origin"`/`"axis"`.  
    \item \texttt{days\_in\_month(month, leap)} — возвращает число дней в месяце; \texttt{leap} = True для високосного года.  
    \item \texttt{traffic\_fine(speed, zone)} — вычисляет штраф за превышение скорости.  
    \begin{itemize}
        \item \textbf{speed} — скорость автомобиля (км/ч)  
        \item \textbf{zone} — тип зоны: `"residential"`, `"city"`, `"highway"`  
        \item \textbf{правила:}  
        \begin{itemize}
            \item `"residential"`: превышение >20 км/ч → 200, >10 км/ч → 100, иначе 0  
            \item `"city"`: превышение >30 → 150, >15 → 75, иначе 0  
            \item `"highway"`: превышение >40 → 100, >20 → 50, иначе 0  
            \item некорректная зона → `"invalid zone"`  
        \end{itemize}
        \item \textbf{требования:} использовать ветвления if/elif/else, проверить все сценарии превышения и отсутствие превышения.  
    \end{itemize}
    \item \texttt{compare\_three\_numbers(a, b, c)} — возвращает `"all equal"`, `"all different"`, или `"two equal"`.  
    \item \texttt{max\_digit(n)} — наибольшая цифра числа.  
    \item \texttt{triangle\_angle\_category(a, b, c)} — вычисляет углы через теорему косинусов и возвращает `"acute"`, `"right"`, `"obtuse"`, `"invalid"`.  
    \item \texttt{next\_day(day, month, leap)} — возвращает следующий день месяца, учитывая количество дней.  
    \item \texttt{is\_inside\_rectangle(x, y, x1, y1, x2, y2)} — проверка попадания точки в прямоугольник.  
    \item \texttt{discount(price)} — возврат цены со скидкой по условию (>1000 → 10%, >500 → 5%, иначе 0%).  
    \item \texttt{closest\_to\_zero(lst)} — элемент списка, ближайший к нулю.  
    \item \texttt{season(month)} — `"Winter"`, `"Spring"`, `"Summer"`, `"Autumn"`.  
    \item \texttt{simple\_calculator(a, b, op)} — +, -, *, /; деление на 0 → `"error"`.  
    \item \texttt{sum\_positive(lst)} — сумма положительных чисел.  
    \item \texttt{sum\_even(lst)} — сумма чётных чисел.  
    \item \texttt{sum\_odd(lst)} — сумма нечётных чисел.  
    \item \texttt{reverse\_signs(lst)} — меняет знак всех элементов.  
    \item \texttt{nearest\_multiple(n, m)} — ближайшее к n кратное m.  
    \item \texttt{sort\_three(a, b, c)} — тройка чисел в порядке возрастания через if/elif/else.  
    \item \texttt{validate\_password(password)} — True, если длина $\geqslant$8, есть цифра и заглавная буква.  
    \item \texttt{is\_perfect(n)} — True, если число совершенное.  
    \item \texttt{sum\_digits(n)} — сумма цифр числа.  
    \item \texttt{count\_vowels(s)} — количество гласных.  
    \item \texttt{is\_palindrome(s)} — True, если строка читается одинаково слева направо и справа налево.  
    \item \texttt{remove\_duplicates(lst)} — возвращает список без повторов.  
    \item \texttt{factorial\_iterative(n)} — факториал через цикл, без рекурсии.  
    \item \texttt{fizz\_buzz(n)} — числа от 1 до n с заменой кратных 3 → `"Fizz"`, кратных 5 → `"Buzz"`, кратных 15 → `"FizzBuzz"`.  
\end{enumerate}

