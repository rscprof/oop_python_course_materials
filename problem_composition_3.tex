\item \textbf{Моделирование поединка между двумя дуэлянтами}

\begin{enumerate}
    \item Импортируйте функцию \texttt{randint} из модуля \texttt{random}.

    \item Создайте класс \texttt{Duelist} («Дуэлянт»).  
    В конструкторе класса должны задаваться имя дуэлянта и его начальное здоровье (по умолчанию — 100 единиц).  
    Также реализуйте следующие методы:
    \begin{itemize}
        \item \texttt{set\_name} — позволяет изменить имя дуэлянта;
        \item \texttt{attack} — моделирует атаку на другого дуэлянта: генерирует случайный урон в диапазоне от 10 до 30 и уменьшает здоровье противника на эту величину.
    \end{itemize}

    \item Создайте класс \texttt{Duel} («Дуэль»).  
    Его конструктор принимает двух дуэлянтов и сохраняет их как внутренние атрибуты. Также в конструкторе инициализируется пустая строка для хранения результата поединка.

    \item Реализуйте метод \texttt{fight}, который моделирует сам поединок:
    \begin{itemize}
        \item Поединок продолжается, пока у обоих дуэлянтов здоровье больше нуля;
        \item На каждом шаге случайным образом (с равной вероятностью) выбирается, кто из дуэлянтов наносит удар;
        \item После каждой атаки, если здоровье любого из участников стало меньше или равно нулю, оно устанавливается в ноль.
    \end{itemize}

    \item После завершения поединка определите его исход:
    \begin{itemize}
        \item Если у первого дуэлянта осталось здоровье, а у второго — нет, побеждает первый;
        \item Если у второго осталось здоровье, а у первого — нет, побеждает второй;
        \item Если здоровье обоих участников равно нулю, объявляется ничья.
    \end{itemize}
    Результат сохраняется в виде понятной строки (например, «Алексей побеждает!» или «Ничья!»).

    \item Добавьте метод \texttt{who\_wins}, который выводит на экран сохранённый результат поединка.
\end{enumerate}

\item \textbf{Моделирование боя между двумя боксёрами}

\begin{enumerate}
    \item Импортируйте функцию \texttt{randint} из модуля \texttt{random}.

    \item Создайте класс \texttt{Boxer} («Боксёр»).  
    В конструкторе задаются имя и начальное здоровье (по умолчанию — 100).  
    Реализуйте методы:
    \begin{itemize}
        \item \texttt{set\_name} — изменение имени боксёра;
        \item \texttt{punch} — нанесение удара противнику с уроном от 10 до 30.
    \end{itemize}

    \item Создайте класс \texttt{BoxingMatch} («Боксёрский поединок»).  
    Его конструктор принимает двух боксёров и инициализирует атрибут для хранения результата.

    \item Реализуйте метод \texttt{match}, моделирующий бой по тем же правилам, что и в первом задании: случайный выбор атакующего, цикл до тех пор, пока у одного из участников не закончится здоровье, коррекция здоровья до нуля при необходимости.

    \item После завершения боя определите победителя или ничью и сохраните результат в виде строки.

    \item Добавьте метод \texttt{announce\_winner}, выводящий результат на экран.
\end{enumerate}

\item \textbf{Моделирование поединка между двумя шахматистами}

\begin{enumerate}
    \item Импортируйте функцию \texttt{randint}.

    \item Создайте класс \texttt{ChessPlayer} («Шахматист»).  
    В конструкторе задаются имя и начальное здоровье (по умолчанию — 100).  
    Реализуйте методы:
    \begin{itemize}
        \item \texttt{set\_name} — изменение имени;
        \item \texttt{play\_move} — «ход» в рамках метафорического интеллектуального поединка, наносящий урон от 10 до 30.
    \end{itemize}

    \item Создайте класс \texttt{ChessGame} («Шахматная партия»), принимающий двух шахматистов и хранящий результат.

    \item Реализуйте метод \texttt{simulate}, моделирующий поединок: случайный выбор ходящего, цикл до обнуления здоровья одного или обоих участников.

    \item Определите победителя или ничью по оставшемуся здоровью.

    \item Добавьте метод \texttt{show\_result}, выводящий итог партии.
\end{enumerate}

\item \textbf{Моделирование схватки между двумя борцами}

\begin{enumerate}
    \item Импортируйте функцию \texttt{randint}.

    \item Создайте класс \texttt{Wrestler} («Борец») с именем и здоровьем (по умолчанию — 100).  
    Методы:
    \begin{itemize}
        \item \texttt{set\_name} — изменение имени;
        \item \texttt{grapple} — захват, наносящий урон от 10 до 30.
    \end{itemize}

    \item Создайте класс \texttt{WrestlingMatch} («Борцовский поединок»), принимающий двух борцов.

    \item Реализуйте метод \texttt{compete}, моделирующий схватку по стандартной схеме: случайный выбор атакующего, цикл до поражения одного или обоих.

    \item Определите исход схватки и сохраните его в виде строки.

    \item Добавьте метод \texttt{declare\_champion}, выводящий победителя.
\end{enumerate}

\item \textbf{Моделирование битвы между двумя магами}

\begin{enumerate}
    \item Импортируйте функцию \texttt{randint}.

    \item Создайте класс \texttt{Mage} («Маг») с именем и здоровьем (по умолчанию — 100).  
    Методы:
    \begin{itemize}
        \item \texttt{set\_name} — изменение имени;
        \item \texttt{cast\_spell} — заклинание, наносящее урон от 10 до 30.
    \end{itemize}

    \item Создайте класс \texttt{MagicDuel} («Магическая дуэль»), принимающий двух магов.

    \item Реализуйте метод \texttt{duel}, моделирующий битву по стандартной логике.

    \item Определите победителя или ничью.

    \item Добавьте метод \texttt{reveal\_winner}, выводящий результат.
\end{enumerate}

\item \textbf{Моделирование поединка между двумя киберспортсменами}

\begin{enumerate}
    \item Импортируйте функцию \texttt{randint}.

    \item Создайте класс \texttt{Gamer} («Киберспортсмен») с именем и здоровьем (по умолчанию — 100).  
    Методы:
    \begin{itemize}
        \item \texttt{set\_name} — изменение имени;
        \item \texttt{make\_move} — игровой ход в метафоре «кибер-боя», наносящий урон от 10 до 30.
    \end{itemize}

    \item Создайте класс \texttt{EsportsMatch} («Кибертурнир»), принимающий двух игроков.

    \item Реализуйте метод \texttt{play}, моделирующий поединок.

    \item Определите победителя или ничью.

    \item Добавьте метод \texttt{show\_champion}, выводящий чемпиона.
\end{enumerate}

\item \textbf{Моделирование соревнования между двумя пловцами}

\begin{enumerate}
    \item Импортируйте функцию \texttt{randint}.

    \item Создайте класс \texttt{Swimmer} («Пловец») с именем и здоровьем (по умолчанию — 100).  
    Методы:
    \begin{itemize}
        \item \texttt{set\_name} — изменение имени;
        \item \texttt{swim\_lap} — заплыв в игровой интерпретации как «атака», наносящая урон от 10 до 30.
    \end{itemize}

    \item Создайте класс \texttt{SwimRace} («Заплыв»), принимающий двух пловцов.

    \item Реализуйте метод \texttt{race}, моделирующий соревнование.

    \item Определите победителя или ничью.

    \item Добавьте метод \texttt{announce\_medal}, выводящий результат.
\end{enumerate}

\item \textbf{Моделирование боя между двумя роботами}

\begin{enumerate}
    \item Импортируйте функцию \texttt{randint}.

    \item Создайте класс \texttt{RobotFighter} («Боевой робот») с именем и здоровьем (по умолчанию — 100).  
    Методы:
    \begin{itemize}
        \item \texttt{set\_name} — изменение имени;
        \item \texttt{strike} — удар, наносящий урон от 10 до 30.
    \end{itemize}

    \item Создайте класс \texttt{RobotBattle} («Робобой»), принимающий двух роботов.

    \item Реализуйте метод \texttt{fight}, моделирующий сражение.

    \item Определите победителя или ничью.

    \item Добавьте метод \texttt{display\_result}, выводящий результат.
\end{enumerate}

\item \textbf{Моделирование дуэли между двумя ковбоями}

\begin{enumerate}
    \item Импортируйте функцию \texttt{randint}.

    \item Создайте класс \texttt{Cowboy} («Ковбой») с именем и здоровьем (по умолчанию — 100).  
    Методы:
    \begin{itemize}
        \item \texttt{set\_name} — изменение имени;
        \item \texttt{draw} — выстрел, наносящий урон от 10 до 30.
    \end{itemize}

    \item Создайте класс \texttt{Showdown} («Разборка»), принимающий двух ковбоев.

    \item Реализуйте метод \texttt{shootout}, моделирующий перестрелку.

    \item Определите победителя или ничью.

    \item Добавьте метод \texttt{proclaim\_winner}, выводящий результат.
\end{enumerate}

\item \textbf{Моделирование битвы между двумя ниндзя}

\begin{enumerate}
    \item Импортируйте функцию \texttt{randint}.

    \item Создайте класс \texttt{Ninja} («Ниндзя») с именем и здоровьем (по умолчанию — 100).  
    Методы:
    \begin{itemize}
        \item \texttt{set\_name} — изменение имени;
        \item \texttt{throw\_shuriken} — бросок сюрикена, наносящий урон от 10 до 30.
    \end{itemize}

    \item Создайте класс \texttt{NinjaClash} («Столкновение ниндзя»), принимающий двух бойцов.

    \item Реализуйте метод \texttt{clash}, моделирующий битву.

    \item Определите победителя или ничью.

    \item Добавьте метод \texttt{declare\_victor}, выводящий результат.
\end{enumerate}

\item \textbf{Моделирование поединка между двумя пиратами}

\begin{enumerate}
    \item Импортируйте функцию \texttt{randint}.

    \item Создайте класс \texttt{Pirate} («Пират») с именем и здоровьем (по умолчанию — 100).  
    Методы:
    \begin{itemize}
        \item \texttt{set\_name} — изменение имени;
        \item \texttt{sword\_fight} — удар мечом, наносящий урон от 10 до 30.
    \end{itemize}

    \item Создайте класс \texttt{PirateDuel} («Пиратская дуэль»), принимающий двух пиратов.

    \item Реализуйте метод \texttt{battle}, моделирующий схватку.

    \item Определите победителя или ничью.

    \item Добавьте метод \texttt{shout\_winner}, выводящий результат.
\end{enumerate}

\item \textbf{Моделирование схватки между двумя гладиаторами}

\begin{enumerate}
    \item Импортируйте функцию \texttt{randint}.

    \item Создайте класс \texttt{Gladiator} («Гладиатор») с именем и здоровьем (по умолчанию — 100).  
    Методы:
    \begin{itemize}
        \item \texttt{set\_name} — изменение имени;
        \item \texttt{attack\_with\_sword} — удар мечом, наносящий урон от 10 до 30.
    \end{itemize}

    \item Создайте класс \texttt{ArenaFight} («Арена»), принимающий двух гладиаторов.

    \item Реализуйте метод \texttt{fight\_to\_death}, моделирующий бой до конца.

    \item Определите победителя или ничью.

    \item Добавьте метод \texttt{crowd\_cheers}, выводящий результат.
\end{enumerate}

\item \textbf{Моделирование поединка между двумя самураями}

\begin{enumerate}
    \item Импортируйте функцию \texttt{randint}.

    \item Создайте класс \texttt{Samurai} («Самурай») с именем и здоровьем (по умолчанию — 100).  
    Методы:
    \begin{itemize}
        \item \texttt{set\_name} — изменение имени;
        \item \texttt{katana\_strike} — удар катаной, наносящий урон от 10 до 30.
    \end{itemize}

    \item Создайте класс \texttt{SamuraiDuel} («Самурайская дуэль»), принимающий двух самураев.

    \item Реализуйте метод \texttt{duel}, моделирующий поединок.

    \item Определите победителя или ничью.

    \item Добавьте метод \texttt{bow\_to\_winner}, выводящий результат.
\end{enumerate}

\item \textbf{Моделирование поединка между двумя драконами}

\begin{enumerate}
    \item Импортируйте функцию \texttt{randint}.

    \item Создайте класс \texttt{Dragon} («Дракон») с именем и здоровьем (по умолчанию — 100).  
    Методы:
    \begin{itemize}
        \item \texttt{set\_name} — изменение имени;
        \item \texttt{breathe\_fire} — огненное дыхание, наносящее урон от 10 до 30.
    \end{itemize}

    \item Создайте класс \texttt{DragonBattle} («Битва драконов»), принимающий двух драконов.

    \item Реализуйте метод \texttt{clash}, моделирующий сражение.

    \item Определите победителя или ничью.

    \item Добавьте метод \texttt{roar\_victory}, выводящий результат.
\end{enumerate}

\item \textbf{Моделирование битвы между двумя титанами}

\begin{enumerate}
    \item Импортируйте функцию \texttt{randint}.

    \item Создайте класс \texttt{Titan} («Титан») с именем и здоровьем (по умолчанию — 100).  
    Методы:
    \begin{itemize}
        \item \texttt{set\_name} — изменение имени;
        \item \texttt{stomp} — топот, наносящий урон от 10 до 30.
    \end{itemize}

    \item Создайте класс \texttt{TitanClash} («Столкновение титанов»), принимающий двух титанов.

    \item Реализуйте метод \texttt{battle}, моделирующий битву.

    \item Определите победителя или ничью.

    \item Добавьте метод \texttt{earth\_shakes}, выводящий результат.
\end{enumerate}

\item \textbf{Моделирование поединка между двумя рыцарями}

\begin{enumerate}
    \item Импортируйте функцию \texttt{randint}.

    \item Создайте класс \texttt{Knight} («Рыцарь») с именем и здоровьем (по умолчанию — 100).  
    Методы:
    \begin{itemize}
        \item \texttt{set\_name} — изменение имени;
        \item \texttt{lance\_charge} — рывок с копьём, наносящий урон от 10 до 30.
    \end{itemize}

    \item Создайте класс \texttt{Joust} («Турнир»), принимающий двух рыцарей.

    \item Реализуйте метод \texttt{tournament}, моделирующий поединок.

    \item Определите победителя или ничью.

    \item Добавьте метод \texttt{king\_declares}, выводящий результат.
\end{enumerate}

\item \textbf{Моделирование поединка между двумя ведьмами}

\begin{enumerate}
    \item Импортируйте функцию \texttt{randint}.

    \item Создайте класс \texttt{Witch} («Ведьма») с именем и здоровьем (по умолчанию — 100).  
    Методы:
    \begin{itemize}
        \item \texttt{set\_name} — изменение имени;
        \item \texttt{brew\_curse} — наложение проклятия, наносящего урон от 10 до 30.
    \end{itemize}

    \item Создайте класс \texttt{WitchDuel} («Ведьмин поединок»), принимающий двух ведьм.

    \item Реализуйте метод \texttt{hex\_battle}, моделирующий битву.

    \item Определите победителя или ничью.

    \item Добавьте метод \texttt{cackle\_in\_triumph}, выводящий результат.
\end{enumerate}

\item \textbf{Моделирование боя между двумя зомби}

\begin{enumerate}
    \item Импортируйте функцию \texttt{randint}.

    \item Создайте класс \texttt{Zombie} («Зомби») с именем и здоровьем (по умолчанию — 100).  
    Методы:
    \begin{itemize}
        \item \texttt{set\_name} — изменение имени;
        \item \texttt{bite} — укус, наносящий урон от 10 до 30.
    \end{itemize}

    \item Создайте класс \texttt{ZombieFight} («Зомби-битва»), принимающий двух зомби.

    \item Реализуйте метод \texttt{apocalypse}, моделирующий сражение.

    \item Определите победителя или ничью.

    \item Добавьте метод \texttt{groan\_winner}, выводящий результат.
\end{enumerate}

\item \textbf{Моделирование схватки между двумя вампирами}

\begin{enumerate}
    \item Импортируйте функцию \texttt{randint}.

    \item Создайте класс \texttt{Vampire} («Вампир») с именем и здоровьем (по умолчанию — 100).  
    Методы:
    \begin{itemize}
        \item \texttt{set\_name} — изменение имени;
        \item \texttt{drain} — высасывание жизненных сил, наносящее урон от 10 до 30.
    \end{itemize}

    \item Создайте класс \texttt{VampireDuel} («Вампирская дуэль»), принимающий двух вампиров.

    \item Реализуйте метод \texttt{night\_fight}, моделирующий ночную битву.

    \item Определите победителя или ничью.

    \item Добавьте метод \texttt{howl\_at\_moon}, выводящий результат.
\end{enumerate}

\item \textbf{Моделирование битвы между двумя оборотнями}

\begin{enumerate}
    \item Импортируйте функцию \texttt{randint}.

    \item Создайте класс \texttt{Werewolf} («Оборотень») с именем и здоровьем (по умолчанию — 100).  
    Методы:
    \begin{itemize}
        \item \texttt{set\_name} — изменение имени;
        \item \texttt{claw} — удар когтями, наносящий урон от 10 до 30.
    \end{itemize}

    \item Создайте класс \texttt{MoonBattle} («Лунная битва»), принимающий двух оборотней.

    \item Реализуйте метод \texttt{howl\_and\_fight}, моделирующий сражение.

    \item Определите победителя или ничью.

    \item Добавьте метод \texttt{moon\_witnesses}, выводящий результат.
\end{enumerate}

\item \textbf{Моделирование поединка между двумя призраками}

\begin{enumerate}
    \item Импортируйте функцию \texttt{randint}.

    \item Создайте класс \texttt{Ghost} («Призрак») с именем и здоровьем (по умолчанию — 100).  
    Методы:
    \begin{itemize}
        \item \texttt{set\_name} — изменение имени;
        \item \texttt{terrify} — устрашение, наносящее урон от 10 до 30.
    \end{itemize}

    \item Создайте класс \texttt{HauntedDuel} («Призрачная дуэль»), принимающий двух призраков.

    \item Реализуйте метод \texttt{scare\_off}, моделирующий поединок.

    \item Определите победителя или ничью.

    \item Добавьте метод \texttt{echo\_victory}, выводящий результат.
\end{enumerate}

\item \textbf{Моделирование боя между двумя гоблинами}

\begin{enumerate}
    \item Импортируйте функцию \texttt{randint}.

    \item Создайте класс \texttt{Goblin} («Гоблин») с именем и здоровьем (по умолчанию — 100).  
    Методы:
    \begin{itemize}
        \item \texttt{set\_name} — изменение имени;
        \item \texttt{stab} — удар кинжалом, наносящий урон от 10 до 30.
    \end{itemize}

    \item Создайте класс \texttt{GoblinSkirmish} («Гоблинская стычка»), принимающий двух гоблинов.

    \item Реализуйте метод \texttt{loot\_fight}, моделирующий драку.

    \item Определите победителя или ничью.

    \item Добавьте метод \texttt{squeal\_winner}, выводящий результат.
\end{enumerate}

\item \textbf{Моделирование схватки между двумя орками}

\begin{enumerate}
    \item Импортируйте функцию \texttt{randint}.

    \item Создайте класс \texttt{Orc} («Орк») с именем и здоровьем (по умолчанию — 100).  
    Методы:
    \begin{itemize}
        \item \texttt{set\_name} — изменение имени;
        \item \texttt{bash} — мощный удар, наносящий урон от 10 до 30.
    \end{itemize}

    \item Создайте класс \texttt{OrcBattle} («Орковская битва»), принимающий двух орков.

    \item Реализуйте метод \texttt{war\_cry}, моделирующий сражение.

    \item Определите победителя или ничью.

    \item Добавьте метод \texttt{grunt\_victory}, выводящий результат.
\end{enumerate}

\item \textbf{Моделирование битвы между двумя эльфами}

\begin{enumerate}
    \item Импортируйте функцию \texttt{randint}.

    \item Создайте класс \texttt{Elf} («Эльф») с именем и здоровьем (по умолчанию — 100).  
    Методы:
    \begin{itemize}
        \item \texttt{set\_name} — изменение имени;
        \item \texttt{arrow\_shot} — выстрел из лука, наносящий урон от 10 до 30.
    \end{itemize}

    \item Создайте класс \texttt{ElvenDuel} («Эльфийская дуэль»), принимающий двух эльфов.

    \item Реализуйте метод \texttt{forest\_clash}, моделирующий поединок.

    \item Определите победителя или ничью.

    \item Добавьте метод \texttt{whisper\_winner}, выводящий результат.
\end{enumerate}

\item \textbf{Моделирование поединка между двумя гномами}

\begin{enumerate}
    \item Импортируйте функцию \texttt{randint}.

    \item Создайте класс \texttt{Dwarf} («Гном») с именем и здоровьем (по умолчанию — 100).  
    Методы:
    \begin{itemize}
        \item \texttt{set\_name} — изменение имени;
        \item \texttt{swing\_axe} — удар топором, наносящий урон от 10 до 30.
    \end{itemize}

    \item Создайте класс \texttt{DwarfFight} («Гномья драка»), принимающий двух гномов.

    \item Реализуйте метод \texttt{mine\_battle}, моделирующий сражение.

    \item Определите победителя или ничью.

    \item Добавьте метод \texttt{roar\_ale}, выводящий результат.
\end{enumerate}

\item \textbf{Моделирование боя между двумя кентаврами}

\begin{enumerate}
    \item Импортируйте функцию \texttt{randint}.

    \item Создайте класс \texttt{Centaur} («Кентавр») с именем и здоровьем (по умолчанию — 100).  
    Методы:
    \begin{itemize}
        \item \texttt{set\_name} — изменение имени;
        \item \texttt{gallop\_attack} — атака в галопе, наносящая урон от 10 до 30.
    \end{itemize}

    \item Создайте класс \texttt{CentaurClash} («Столкновение кентавров»), принимающий двух кентавров.

    \item Реализуйте метод \texttt{plain\_duel}, моделирующий поединок.

    \item Определите победителя или ничью.

    \item Добавьте метод \texttt{neigh\_victory}, выводящий результат.
\end{enumerate}

\item \textbf{Моделирование схватки между двумя минотаврами}

\begin{enumerate}
    \item Импортируйте функцию \texttt{randint}.

    \item Создайте класс \texttt{Minotaur} («Минотавр») с именем и здоровьем (по умолчанию — 100).  
    Методы:
    \begin{itemize}
        \item \texttt{set\_name} — изменение имени;
        \item \texttt{gore} — удар рогами, наносящий урон от 10 до 30.
    \end{itemize}

    \item Создайте класс \texttt{LabyrinthFight} («Лабиринтная битва»), принимающий двух минотавров.

    \item Реализуйте метод \texttt{maze\_battle}, моделирующий сражение.

    \item Определите победителя или ничью.

    \item Добавьте метод \texttt{bellow\_winner}, выводящий результат.
\end{enumerate}

\item \textbf{Моделирование битвы между двумя фениксами}

\begin{enumerate}
    \item Импортируйте функцию \texttt{randint}.

    \item Создайте класс \texttt{Phoenix} («Феникс») с именем и здоровьем (по умолчанию — 100).  
    Методы:
    \begin{itemize}
        \item \texttt{set\_name} — изменение имени;
        \item \texttt{rebirth\_strike} — удар, связанный с возрождением, наносящий урон от 10 до 30.
    \end{itemize}

    \item Создайте класс \texttt{PhoenixClash} («Столкновение фениксов»), принимающий двух фениксов.

    \item Реализуйте метод \texttt{ash\_duel}, моделирующий поединок.

    \item Определите победителя или ничью.

    \item Добавьте метод \texttt{soar\_victorious}, выводящий результат.
\end{enumerate}

\item \textbf{Моделирование поединка между двумя единорогами}

\begin{enumerate}
    \item Импортируйте функцию \texttt{randint}.

    \item Создайте класс \texttt{Unicorn} («Единорог») с именем и здоровьем (по умолчанию — 100).  
    Методы:
    \begin{itemize}
        \item \texttt{set\_name} — изменение имени;
        \item \texttt{horn\_charge} — удар рогом, наносящий урон от 10 до 30.
    \end{itemize}

    \item Создайте класс \texttt{UnicornDuel} («Дуэль единорогов»), принимающий двух единорогов.

    \item Реализуйте метод \texttt{meadow\_clash}, моделирующий поединок.

    \item Определите победителя или ничью.

    \item Добавьте метод \texttt{gallop\_in\_glory}, выводящий результат.
\end{enumerate}

\item \textbf{Моделирование боя между двумя троллями}

\begin{enumerate}
    \item Импортируйте функцию \texttt{randint}.

    \item Создайте класс \texttt{Troll} («Тролль») с именем и здоровьем (по умолчанию — 100).  
    Методы:
    \begin{itemize}
        \item \texttt{set\_name} — изменение имени;
        \item \texttt{club\_smash} — удар дубиной, наносящий урон от 10 до 30.
    \end{itemize}

    \item Создайте класс \texttt{TrollFight} («Троллья драка»), принимающий двух троллей.

    \item Реализуйте метод \texttt{bridge\_battle}, моделирующий сражение.

    \item Определите победителя или ничью.

    \item Добавьте метод \texttt{grunt\_and\_laugh}, выводящий результат.
\end{enumerate}

\item \textbf{Моделирование схватки между двумя грифонами}

\begin{enumerate}
    \item Импортируйте функцию \texttt{randint}.

    \item Создайте класс \texttt{Griffin} («Грифон») с именем и здоровьем (по умолчанию — 100).  
    Методы:
    \begin{itemize}
        \item \texttt{set\_name} — изменение имени;
        \item \texttt{dive\_attack} — пикирующая атака, наносящая урон от 10 до 30.
    \end{itemize}

    \item Создайте класс \texttt{GriffinClash} («Столкновение грифонов»), принимающий двух грифонов.

    \item Реализуйте метод \texttt{aerial\_duel}, моделирующий воздушный поединок.

    \item Определите победителя или ничью.

    \item Добавьте метод \texttt{screech\_victory}, выводящий результат.
\end{enumerate}

\item \textbf{Моделирование битвы между двумя драконоборцами}

\begin{enumerate}
    \item Импортируйте функцию \texttt{randint}.

    \item Создайте класс \texttt{Dragonslayer} («Драконоборец») с именем и здоровьем (по умолчанию — 100).  
    Методы:
    \begin{itemize}
        \item \texttt{set\_name} — изменение имени;
        \item \texttt{slay} — удар, направленный на убийство, наносящий урон от 10 до 30.
    \end{itemize}

    \item Создайте класс \texttt{SlayerDuel} («Дуэль драконоборцев»), принимающий двух героев.

    \item Реализуйте метод \texttt{heroic\_fight}, моделирующий битву.

    \item Определите победителя или ничью.

    \item Добавьте метод \texttt{bard\_sings}, выводящий результат.
\end{enumerate}

\item \textbf{Моделирование поединка между двумя наёмниками}

\begin{enumerate}
    \item Импортируйте функцию \texttt{randint}.

    \item Создайте класс \texttt{Mercenary} («Наёмник») с именем и здоровьем (по умолчанию — 100).  
    Методы:
    \begin{itemize}
        \item \texttt{set\_name} — изменение имени;
        \item \texttt{strike\_for\_hire} — удар за плату, наносящий урон от 10 до 30.
    \end{itemize}

    \item Создайте класс \texttt{MercenaryClash} («Стычка наёмников»), принимающий двух бойцов.

    \item Реализуйте метод \texttt{contract\_battle}, моделирующий сражение.

    \item Определите победителя или ничью.

    \item Добавьте метод \texttt{count\_coins}, выводящий результат.
\end{enumerate}

\item \textbf{Моделирование боя между двумя ассасинами}

\begin{enumerate}
    \item Импортируйте функцию \texttt{randint}.

    \item Создайте класс \texttt{Assassin} («Ассасин») с именем и здоровьем (по умолчанию — 100).  
    Методы:
    \begin{itemize}
        \item \texttt{set\_name} — изменение имени;
        \item \texttt{backstab} — удар в спину, наносящий урон от 10 до 30.
    \end{itemize}

    \item Создайте класс \texttt{ShadowDuel} («Теневая дуэль»), принимающий двух ассасинов.

    \item Реализуйте метод \texttt{night\_kill}, моделирующий ночной бой.

    \item Определите победителя или ничью.

    \item Добавьте метод \texttt{vanish\_in\_dark}, выводящий результат.
\end{enumerate}

\item \textbf{Моделирование сражения между двумя солдатами} (оригинальный вариант)

\begin{enumerate}
    \item Импортируйте функцию \texttt{randint} из модуля \texttt{random}.

    \item Создайте класс \texttt{Soldier} («Солдат»).  
    В конструкторе задаются имя и начальное здоровье (по умолчанию — 100).  
    Реализуйте методы:
    \begin{itemize}
        \item \texttt{set\_name} — изменение имени;
        \item \texttt{attack} — атака противника с уроном от 10 до 30.
    \end{itemize}

    \item Создайте класс \texttt{Battle} («Сражение»), принимающий двух солдат и хранящий результат.

    \item Реализуйте метод \texttt{battle}, моделирующий бой:
    \begin{itemize}
        \item Поединок продолжается, пока у обоих солдат здоровье больше нуля;
        \item Атакующий выбирается случайно;
        \item После каждой атаки здоровье, упавшее до нуля или ниже, устанавливается в ноль.
    \end{itemize}

    \item После завершения боя определите исход: победа одного из солдат или ничья — и сохраните результат в виде строки.

    \item Добавьте метод \texttt{who\_win}, выводящий результат на экран.
\end{enumerate}
