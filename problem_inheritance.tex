\subsection{Семинар <<Конструкторы, наследование и полиморфизм. 1 часть>>  
(2 часа)}


В ходе работы решите 4 задачи. 
Предполагается, что пользователь класса не имеет права обращаться к свойствам напрямую 
(соблюдая принцип инкапсуляции), а должен использовать методы. 

Продемонстрируйте работоспособность всех методов (из задания) 
посредством создания запускаемых файлов, где осуществляется 
вызов методов для разных ситуаций 
(без ручного ввода, но с выводом результатов в консоль). 

Каждый класс должен сохраняться в отдельном исходном файле. 
Необходимо соблюдать все стандартные требования к качеству кода 
(отступы, именования переменных, классов, методов, 
проверка корректности входных данных).
Для каждого класса создайте отдельный запускаемый файл для проверки всех его методов 
(допускается использование других классов в этих тестах).

Все предлагаемые классы в заданиях упрощенные; для использования в production-окружении они требуют серьезной доработки. Суть задания — в отработке базовых навыков, а не в идеальном моделировании предложенных ситуаций.

Для сдачи работы будьте готовы пояснить или аналогично заданию модифицировать любую часть кода, а также ответить на вопросы:
\begin{enumerate}
    \item Что обозначает свойство наследования в парадигме ООП?
    \item Что обозначает свойство полиморфизма в парадигме ООП?
    \item Опишите реализацию наследования в Python
    \item Как создать конструктор в Python
    \item Как реализовать абстрактный класс в Python (и что это значит)
    \item Как реализовать абстрактные методы в Python (и что это значит)
\end{enumerate}

Если вы нашли в задачнике ошибки, опечатки и другие недостатки, то вы можете сделать pull-request. 

\textbf{Срок сдачи работы (начала сдачи):} через одно занятие после его выдачи. В последующие сроки оценка будет снижаться (при отсутствии оправдывающих документов).

\subsubsection{Задача 1}

\begin{enumerate}
    \item 

Напишите программу, которая создаёт класс
\texttt{Circle} с методами для вычисления площади
и длины окружности (периметра). Программа должна запрашивать у пользователя радиус
и выводить вычисленные площадь и длину окружности.

\subsection*{Инструкции:}
\begin{enumerate}
\item Создайте класс \texttt{Circle} с методом
\texttt{\_\_init\_\_}, который принимает радиус окружности в
качестве аргумента и сохраняет его в атрибуте \texttt{self.\_\_radius}.

\item Создайте метод \texttt{calculate\_circle\_area},
без аргументов, который вычисляет площадь круга по формуле:
\[
\pi \cdot \texttt{\_\_radius}^2
\]

\item Создайте метод \texttt{calculate\_circle\_perimeter} без аргументов,
который вычисляет длину окружности по формуле:
\[
2 \cdot \pi \cdot \texttt{\_\_radius}
\]

\item Напишите цикл, который повторяется 10 раз. В каждой итерации программа должна:
\begin{enumerate}
\item запрашивать у пользователя радиус окружности,
\item создавать экземпляр класса \texttt{Circle} с этим радиусом,
\item вычислять площадь и длину окружности с помощью соответствующих методов,
\item выводить результаты на экран.
\end{enumerate}
\end{enumerate}

\subsection*{Пример использования:}
\begin{verbatim}
radius = 3
circle = Circle(radius)
area = circle.calculate_circle_area()
perimeter = circle.calculate_circle_perimeter()
print(f"Площадь окружности равна: {area}")
print(f"Периметр окружности равен: {perimeter}")
\end{verbatim}

\textbf{Вывод:}
\begin{verbatim}
Площадь окружности равна: 28.274333882308138
Периметр окружности равен: 18.84955592153876
\end{verbatim}

\item 
Напишите программу, которая создаёт класс \texttt{Square} с методами для вычисления площади
и периметра. Программа должна запрашивать у пользователя длину стороны
и выводить вычисленные площадь и периметр.

\subsection*{Инструкции:}
\begin{enumerate}
\item Создайте класс \texttt{Square} с методом
\texttt{\_\_init\_\_}, который принимает длину стороны квадрата в
качестве аргумента и сохраняет её в атрибуте \texttt{self.\_\_side}.

\item Создайте метод \texttt{calculate\_area},
без аргументов, который вычисляет площадь квадрата по формуле:
\[
\texttt{\_\_side}^2
\]

\item Создайте метод \texttt{calculate\_perimeter} без аргументов,
который вычисляет периметр квадрата по формуле:
\[
4 \cdot \texttt{\_\_side}
\]

\item Напишите цикл, который повторяется 10 раз. В каждой итерации программа должна:
\begin{enumerate}
\item запрашивать у пользователя длину стороны квадрата,
\item создавать экземпляр класса \texttt{Square} с этой длиной,
\item вычислять площадь и периметр с помощью соответствующих методов,
\item выводить результаты на экран.
\end{enumerate}
\end{enumerate}

\subsection*{Пример использования:}
\begin{verbatim}
side = 5
square = Square(side)
area = square.calculate_area()
perimeter = square.calculate_perimeter()
print(f"Площадь квадрата равна: {area}")
print(f"Периметр квадрата равен: {perimeter}")
\end{verbatim}

\textbf{Вывод:}
\begin{verbatim}
Площадь квадрата равна: 25
Периметр квадрата равен: 20
\end{verbatim}

\item
Напишите программу, которая создаёт класс \texttt{Rectangle} с методами для вычисления площади
и периметра. Программа должна запрашивать у пользователя ширину прямоугольника
(при соотношении сторон 1:2) и выводить вычисленные площадь и периметр.

\subsection*{Инструкции:}
\begin{enumerate}
\item Создайте класс \texttt{Rectangle} с методом
\texttt{\_\_init\_\_}, который принимает ширину прямоугольника в
качестве аргумента и сохраняет её в атрибуте \texttt{self.\_\_width}.
Высота прямоугольника должна быть в два раза больше ширины.

\item Создайте метод \texttt{calculate\_area},
без аргументов, который вычисляет площадь прямоугольника по формуле:
\[
\texttt{\_\_width} \cdot (2 \cdot \texttt{\_\_width})
\]

\item Создайте метод \texttt{calculate\_perimeter} без аргументов,
который вычисляет периметр прямоугольника по формуле:
\[
2 \cdot (\texttt{\_\_width} + 2 \cdot \texttt{\_\_width})
\]

\item Напишите цикл, который повторяется 10 раз. В каждой итерации программа должна:
\begin{enumerate}
\item запрашивать у пользователя ширину прямоугольника,
\item создавать экземпляр класса \texttt{Rectangle} с этой шириной,
\item вычислять площадь и периметр с помощью соответствующих методов,
\item выводить результаты на экран.
\end{enumerate}
\end{enumerate}

\subsection*{Пример использования:}
\begin{verbatim}
width = 3
rectangle = Rectangle(width)
area = rectangle.calculate_area()
perimeter = rectangle.calculate_perimeter()
print(f"Площадь прямоугольника равна: {area}")
print(f"Периметр прямоугольника равен: {perimeter}")
\end{verbatim}

\textbf{Вывод:}
\begin{verbatim}
Площадь прямоугольника равна: 18
Периметр прямоугольника равен: 18
\end{verbatim}

\item
Напишите программу, которая создаёт класс \texttt{Triangle} с методами для вычисления площади
и периметра. Программа должна запрашивать у пользователя длину стороны
равностороннего треугольника и выводить вычисленные площадь и периметр.

\subsection*{Инструкции:}
\begin{enumerate}
\item Создайте класс \texttt{Triangle} с методом
\texttt{\_\_init\_\_}, который принимает длину стороны треугольника в
качестве аргумента и сохраняет её в атрибуте \texttt{self.\_\_side}.

\item Создайте метод \texttt{calculate\_area},
без аргументов, который вычисляет площадь равностороннего треугольника по формуле:
\[
\frac{\sqrt{3}}{4} \cdot \texttt{\_\_side}^2
\]

\item Создайте метод \texttt{calculate\_perimeter} без аргументов,
который вычисляет периметр треугольника по формуле:
\[
3 \cdot \texttt{\_\_side}
\]

\item Напишите цикл, который повторяется 10 раз. В каждой итерации программа должна:
\begin{enumerate}
\item запрашивать у пользователя длину стороны треугольника,
\item создавать экземпляр класса \texttt{Triangle} с этой длиной,
\item вычислять площадь и периметр с помощью соответствующих методов,
\item выводить результаты на экран.
\end{enumerate}
\end{enumerate}

\subsection*{Пример использования:}
\begin{verbatim}
side = 4
triangle = Triangle(side)
area = triangle.calculate_area()
perimeter = triangle.calculate_perimeter()
print(f"Площадь треугольника равна: {area}")
print(f"Периметр треугольника равен: {perimeter}")
\end{verbatim}

\textbf{Вывод:}
\begin{verbatim}
Площадь треугольника равна: 6.928203230275509
Периметр треугольника равен: 12
\end{verbatim}

\item
Напишите программу, которая создаёт класс \texttt{Sphere} с методами для вычисления площади поверхности
и объёма. Программа должна запрашивать у пользователя радиус сферы
и выводить вычисленные площадь поверхности и объём.

\subsection*{Инструкции:}
\begin{enumerate}
\item Создайте класс \texttt{Sphere} с методом
\texttt{\_\_init\_\_}, который принимает радиус сферы в
качестве аргумента и сохраняет его в атрибуте \texttt{self.\_\_radius}.

\item Создайте метод \texttt{calculate\_surface\_area},
без аргументов, который вычисляет площадь поверхности сферы по формуле:
\[
4 \cdot \pi \cdot \texttt{\_\_radius}^2
\]

\item Создайте метод \texttt{calculate\_volume} без аргументов,
который вычисляет объём сферы по формуле:
\[
\frac{4}{3} \cdot \pi \cdot \texttt{\_\_radius}^3
\]

\item Напишите цикл, который повторяется 10 раз. В каждой итерации программа должна:
\begin{enumerate}
\item запрашивать у пользователя радиус сферы,
\item создавать экземпляр класса \texttt{Sphere} с этим радиусом,
\item вычислять площадь поверхности и объём с помощью соответствующих методов,
\item выводить результаты на экран.
\end{enumerate}
\end{enumerate}

\subsection*{Пример использования:}
\begin{verbatim}
radius = 2
sphere = Sphere(radius)
surface_area = sphere.calculate_surface_area()
volume = sphere.calculate_volume()
print(f"Площадь поверхности сферы равна: {surface_area}")
print(f"Объём сферы равен: {volume}")
\end{verbatim}

\textbf{Вывод:}
\begin{verbatim}
Площадь поверхности сферы равна: 50.26548245743669
Объём сферы равен: 33.510321638291124
\end{verbatim}

\item
Напишите программу, которая создаёт класс \texttt{Cylinder} с методами для вычисления объёма
и площади боковой поверхности. Программа должна запрашивать у пользователя радиус основания
и выводить вычисленные объём и площадь боковой поверхности (высота цилиндра фиксирована и равна 5).

\subsection*{Инструкции:}
\begin{enumerate}
\item Создайте класс \texttt{Cylinder} с методом
\texttt{\_\_init\_\_}, который принимает радиус основания цилиндра в
качестве аргумента и сохраняет его в атрибуте \texttt{self.\_\_radius}.
Высота цилиндра фиксирована и равна 5.

\item Создайте метод \texttt{calculate\_volume},
без аргументов, который вычисляет объём цилиндра по формуле:
\[
\pi \cdot \texttt{\_\_radius}^2 \cdot 5
\]

\item Создайте метод \texttt{calculate\_lateral\_area} без аргументов,
который вычисляет площадь боковой поверхности цилиндра по формуле:
\[
2 \cdot \pi \cdot \texttt{\_\_radius} \cdot 5
\]

\item Напишите цикл, который повторяется 10 раз. В каждой итерации программа должна:
\begin{enumerate}
\item запрашивать у пользователя радиус основания цилиндра,
\item создавать экземпляр класса \texttt{Cylinder} с этим радиусом,
\item вычислять объём и площадь боковой поверхности с помощью соответствующих методов,
\item выводить результаты на экран.
\end{enumerate}
\end{enumerate}

\subsection*{Пример использования:}
\begin{verbatim}
radius = 3
cylinder = Cylinder(radius)
volume = cylinder.calculate_volume()
lateral_area = cylinder.calculate_lateral_area()
print(f"Объём цилиндра равен: {volume}")
print(f"Площадь боковой поверхности равна: {lateral_area}")
\end{verbatim}

\textbf{Вывод:}
\begin{verbatim}
Объём цилиндра равен: 141.3716694115407
Площадь боковой поверхности равна: 94.24777960769379
\end{verbatim}

\item
Напишите программу, которая создаёт класс \texttt{Cone} с методами для вычисления объёма
и площади боковой поверхности. Программа должна запрашивать у пользователя радиус основания
и выводить вычисленные объём и площадь боковой поверхности (высота конуса фиксирована и равна 10).

\subsection*{Инструкции:}
\begin{enumerate}
\item Создайте класс \texttt{Cone} с методом
\texttt{\_\_init\_\_}, который принимает радиус основания конуса в
качестве аргумента и сохраняет его в атрибуте \texttt{self.\_\_radius}.
Высота конуса фиксирована и равна 10.

\item Создайте метод \texttt{calculate\_volume},
без аргументов, который вычисляет объём конуса по формуле:
\[
\frac{1}{3} \cdot \pi \cdot \texttt{\_\_radius}^2 \cdot 10
\]

\item Создайте метод \texttt{calculate\_lateral\_area} без аргументов,
который вычисляет площадь боковой поверхности конуса по формуле:
\[
\pi \cdot \texttt{\_\_radius} \cdot \sqrt{\texttt{\_\_radius}^2 + 10^2}
\]

\item Напишите цикл, который повторяется 10 раз. В каждой итерации программа должна:
\begin{enumerate}
\item запрашивать у пользователя радиус основания конуса,
\item создавать экземпляр класса \texttt{Cone} с этим радиусом,
\item вычислять объём и площадь боковой поверхности с помощью соответствующих методов,
\item выводить результаты на экран.
\end{enumerate}
\end{enumerate}

\subsection*{Пример использования:}
\begin{verbatim}
radius = 3
cone = Cone(radius)
volume = cone.calculate_volume()
lateral_area = cone.calculate_lateral_area()
print(f"Объём конуса равен: {volume}")
print(f"Площадь боковой поверхности равна: {lateral_area}")
\end{verbatim}

\textbf{Вывод:}
\begin{verbatim}
Объём конуса равен: 94.24777960769379
Площадь боковой поверхности равна: 94.86832980505137
\end{verbatim}

\item
Напишите программу, которая создаёт класс \texttt{Cube} с методами для вычисления объёма
и площади полной поверхности. Программа должна запрашивать у пользователя длину ребра куба
и выводить вычисленные объём и площадь.

\subsection*{Инструкции:}
\begin{enumerate}
\item Создайте класс \texttt{Cube} с методом
\texttt{\_\_init\_\_}, который принимает длину ребра куба в
качестве аргумента и сохраняет её в атрибуте \texttt{self.\_\_side}.

\item Создайте метод \texttt{calculate\_volume},
без аргументов, который вычисляет объём куба по формуле:
\[
\texttt{\_\_side}^3
\]

\item Создайте метод \texttt{calculate\_surface\_area} без аргументов,
который вычисляет площадь полной поверхности куба по формуле:
\[
6 \cdot \texttt{\_\_side}^2
\]

\item Напишите цикл, который повторяется 10 раз. В каждой итерации программа должна:
\begin{enumerate}
\item запрашивать у пользователя длину ребра куба,
\item создавать экземпляр класса \texttt{Cube} с этой длиной,
\item вычислять объём и площадь полной поверхности с помощью соответствующих методов,
\item выводить результаты на экран.
\end{enumerate}
\end{enumerate}

\subsection*{Пример использования:}
\begin{verbatim}
side = 4
cube = Cube(side)
volume = cube.calculate_volume()
surface_area = cube.calculate_surface_area()
print(f"Объём куба равен: {volume}")
print(f"Площадь полной поверхности равна: {surface_area}")
\end{verbatim}

\textbf{Вывод:}
\begin{verbatim}
Объём куба равен: 64
Площадь полной поверхности равна: 96
\end{verbatim}

\item
Напишите программу, которая создаёт класс \texttt{Parallelogram} с методами для вычисления площади
и периметра. Программа должна запрашивать у пользователя длину основания параллелограмма
и выводить вычисленные площадь и периметр (высота параллелограмма фиксирована и равна 8, 
а боковая сторона равна 6).

\subsection*{Инструкции:}
\begin{enumerate}
\item Создайте класс \texttt{Parallelogram} с методом
\texttt{\_\_init\_\_}, который принимает длину основания параллелограмма в
качестве аргумента и сохраняет её в атрибуте \texttt{self.\_\_base}.
Высота параллелограмма фиксирована и равна 8, а боковая сторона равна 6.

\item Создайте метод \texttt{calculate\_area},
без аргументов, который вычисляет площадь параллелограмма по формуле:
\[
\texttt{\_\_base} \cdot 8
\]

\item Создайте метод \texttt{calculate\_perimeter} без аргументов,
который вычисляет периметр параллелограмма по формуле:
\[
2 \cdot (\texttt{\_\_base} + 6)
\]

\item Напишите цикл, который повторяется 10 раз. В каждой итерации программа должна:
\begin{enumerate}
\item запрашивать у пользователя длину основания параллелограмма,
\item создавать экземпляр класса \texttt{Parallelogram} с этой длиной,
\item вычислять площадь и периметр с помощью соответствующих методов,
\item выводить результаты на экран.
\end{enumerate}
\end{enumerate}

\subsection*{Пример использования:}
\begin{verbatim}
base = 5
parallelogram = Parallelogram(base)
area = parallelogram.calculate_area()
perimeter = parallelogram.calculate_perimeter()
print(f"Площадь параллелограмма равна: {area}")
print(f"Периметр параллелограмма равен: {perimeter}")
\end{verbatim}

\textbf{Вывод:}
\begin{verbatim}
Площадь параллелограмма равна: 40
Периметр параллелограмма равен: 22
\end{verbatim}


\item
Напишите программу, которая создаёт класс \texttt{Ellipse} с методами для вычисления площади
и приближённого значения периметра. Программа должна запрашивать у пользователя длину большой полуоси
и выводить вычисленные площадь и периметр (длина малой полуоси фиксирована и равна 3).

\subsection*{Инструкции:}
\begin{enumerate}
\item Создайте класс \texttt{Ellipse} с методом
\texttt{\_\_init\_\_}, который принимает длину большой полуоси эллипса в
качестве аргумента и сохраняет её в атрибуте \texttt{self.\_\_major\_axis}.
Длина малой полуоси фиксирована и равна 3.

\item Создайте метод \texttt{calculate\_area},
без аргументов, который вычисляет площадь эллипса по формуле:
\[
\pi \cdot \texttt{\_\_major\_axis} \cdot 3
\]

\item Создайте метод \texttt{calculate\_perimeter} без аргументов,
который вычисляет приближённое значение периметра эллипса по формуле Рамануджана:
\[
\pi \cdot \left(3(\texttt{\_\_major\_axis} + 3) - \sqrt{(3\texttt{\_\_major\_axis} + 3)(\texttt{\_\_major\_axis} + 9)}\right)
\]

\item Напишите цикл, который повторяется 10 раз. В каждой итерации программа должна:
\begin{enumerate}
\item запрашивать у пользователя длину большой полуоси эллипса,
\item создавать экземпляр класса \texttt{Ellipse} с этой длиной,
\item вычислять площадь и периметр с помощью соответствующих методов,
\item выводить результаты на экран.
\end{enumerate}
\end{enumerate}

\subsection*{Пример использования:}
\begin{verbatim}
major_axis = 5
ellipse = Ellipse(major_axis)
area = ellipse.calculate_area()
perimeter = ellipse.calculate_perimeter()
print(f"Площадь эллипса равна: {area}")
print(f"Периметр эллипса равен: {perimeter}")
\end{verbatim}

\textbf{Вывод:}
\begin{verbatim}
Площадь эллипса равна: 47.12388980384689
Периметр эллипса равен: 25.74488980384689
\end{verbatim}

\item
Напишите программу, которая создаёт класс \texttt{BankAccount} с методами для вычисления начисленных процентов
и суммы налога на доход. Программа должна запрашивать у пользователя начальный баланс счёта
и выводить вычисленные проценты и налог (процентная ставка фиксирована и равна 5\%, 
налоговая ставка на доход фиксирована и равна 13\%).

\subsection*{Инструкции:}
\begin{enumerate}
\item Создайте класс \texttt{BankAccount} с методом
\texttt{\_\_init\_\_}, который принимает начальный баланс счёта в
качестве аргумента и сохраняет его в атрибуте \texttt{self.\_\_balance}.

\item Создайте метод \texttt{calculate\_interest},
без аргументов, который вычисляет начисленные проценты по формуле:
\[
\texttt{\_\_balance} \cdot 0.05
\]

\item Создайте метод \texttt{calculate\_tax} без аргументов,
который вычисляет сумму налога на полученный доход (проценты) по формуле:
\[
(\texttt{\_\_balance} \cdot 0.05) \cdot 0.13
\]

\item Напишите цикл, который повторяется 10 раз. В каждой итерации программа должна:
\begin{enumerate}
\item запрашивать у пользователя начальный баланс счёта,
\item создавать экземпляр класса \texttt{BankAccount} с этим балансом,
\item вычислять начисленные проценты и сумму налога с помощью соответствующих методов,
\item выводить результаты на экран.
\end{enumerate}
\end{enumerate}

\subsection*{Пример использования:}
\begin{verbatim}
balance = 1000
account = BankAccount(balance)
interest = account.calculate_interest()
tax = account.calculate_tax()
print(f"Начисленные проценты: {interest}")
print(f"Сумма налога на доход: {tax}")
\end{verbatim}

\textbf{Вывод:}
\begin{verbatim}
Начисленные проценты: 50.0
Сумма налога на доход: 6.5
\end{verbatim}

\item
Напишите программу, которая создаёт класс \texttt{TemperatureConverter} с методами для преобразования температуры
из градусов Цельсия в Фаренгейты и Кельвины. Программа должна запрашивать у пользователя температуру в Цельсиях
и выводить преобразованные значения.

\subsection*{Инструкции:}
\begin{enumerate}
\item Создайте класс \texttt{TemperatureConverter} с методом
\texttt{\_\_init\_\_}, который принимает температуру в градусах Цельсия в
качестве аргумента и сохраняет её в атрибуте \texttt{self.\_\_celsius}.

\item Создайте метод \texttt{to\_fahrenheit},
без аргументов, который преобразует температуру в Фаренгейты по формуле:
\[
(\texttt{\_\_celsius} \times \frac{9}{5}) + 32
\]

\item Создайте метод \texttt{to\_kelvin} без аргументов,
который преобразует температуру в Кельвины по формуле:
\[
\texttt{\_\_celsius} + 273.15
\]

\item Напишите цикл, который повторяется 10 раз. В каждой итерации программа должна:
\begin{enumerate}
\item запрашивать у пользователя температуру в градусах Цельсия,
\item создавать экземпляр класса \texttt{TemperatureConverter} с этим значением,
\item вычислять температуру в Фаренгейтах и Кельвинах с помощью соответствующих методов,
\item выводить результаты на экран.
\end{enumerate}
\end{enumerate}

\subsection*{Пример использования:}
\begin{verbatim}
celsius = 25
converter = TemperatureConverter(celsius)
fahrenheit = converter.to_fahrenheit()
kelvin = converter.to_kelvin()
print(f"Температура в Фаренгейтах: {fahrenheit}")
print(f"Температура в Кельвинах: {kelvin}")
\end{verbatim}

\textbf{Вывод:}
\begin{verbatim}
Температура в Фаренгейтах: 77.0
Температура в Кельвинах: 298.15
\end{verbatim}

\item
Напишите программу, которая создаёт класс \texttt{DistanceConverter} с методами для преобразования расстояния
из метров в километры и мили. Программа должна запрашивать у пользователя расстояние в метрах
и выводить преобразованные значения.

\subsection*{Инструкции:}
\begin{enumerate}
\item Создайте класс \texttt{DistanceConverter} с методом
\texttt{\_\_init\_\_}, который принимает расстояние в метрах в
качестве аргумента и сохраняет его в атрибуте \texttt{self.\_\_meters}.

\item Создайте метод \texttt{to\_kilometers},
без аргументов, который преобразует расстояние в километры по формуле:
\[
\texttt{\_\_meters} \div 1000
\]

\item Создайте метод \texttt{to\_miles} без аргументов,
который преобразует расстояние в мили по формуле:
\[
\texttt{\_\_meters} \div 1609.344
\]

\item Напишите цикл, который повторяется 10 раз. В каждой итерации программа должна:
\begin{enumerate}
\item запрашивать у пользователя расстояние в метрах,
\item создавать экземпляр класса \texttt{DistanceConverter} с этим значением,
\item вычислять расстояние в километрах и милях с помощью соответствующих методов,
\item выводить результаты на экран.
\end{enumerate}
\end{enumerate}

\subsection*{Пример использования:}
\begin{verbatim}
meters = 1609.344
converter = DistanceConverter(meters)
kilometers = converter.to_kilometers()
miles = converter.to_miles()
print(f"Расстояние в километрах: {kilometers}")
print(f"Расстояние в милях: {miles}")
\end{verbatim}

\textbf{Вывод:}
\begin{verbatim}
Расстояние в километрах: 1.609344
Расстояние в милях: 1.0
\end{verbatim}

\item
Напишите программу, которая создаёт класс \texttt{WeightConverter} с методами для преобразования массы
из килограммов в граммы и фунты. Программа должна запрашивать у пользователя массу в килограммах
и выводить преобразованные значения.

\subsection*{Инструкции:}
\begin{enumerate}
\item Создайте класс \texttt{WeightConverter} с методом
\texttt{\_\_init\_\_}, который принимает массу в килограммах в
качестве аргумента и сохраняет её в атрибуте \texttt{self.\_\_kg}.

\item Создайте метод \texttt{to\_grams},
без аргументов, который преобразует массу в граммы по формуле:
\[
\texttt{\_\_kg} \times 1000
\]

\item Создайте метод \texttt{to\_pounds} без аргументов,
который преобразует массу в фунты по формуле:
\[
\texttt{\_\_kg} \times 2.20462
\]

\item Напишите цикл, который повторяется 10 раз. В каждой итерации программа должна:
\begin{enumerate}
\item запрашивать у пользователя массу в килограммах,
\item создавать экземпляр класса \texttt{WeightConverter} с этим значением,
\item вычислять массу в граммах и фунтах с помощью соответствующих методов,
\item выводить результаты на экран.
\end{enumerate}
\end{enumerate}

\subsection*{Пример использования:}
\begin{verbatim}
kg = 2.5
converter = WeightConverter(kg)
grams = converter.to_grams()
pounds = converter.to_pounds()
print(f"Масса в граммах: {grams}")
print(f"Масса в фунтах: {pounds}")
\end{verbatim}

\textbf{Вывод:}
\begin{verbatim}
Масса в граммах: 2500.0
Масса в фунтах: 5.51155
\end{verbatim}

\item 


Напишите программу, которая создаёт класс \texttt{TimeConverter} с методами для преобразования времени
из секунд в минуты и часы. Программа должна запрашивать у пользователя время в секундах
и выводить преобразованные значения.

\subsection*{Инструкции:}
\begin{enumerate}
\item Создайте класс \texttt{TimeConverter} с методом
\texttt{\_\_init\_\_}, который принимает время в секундах в
качестве аргумента и сохраняет его в атрибуте \texttt{self.\_\_seconds}.

\item Создайте метод \texttt{to\_minutes},
без аргументов, который преобразует время в минуты по формуле:
\[
\texttt{\_\_seconds} \div 60
\]

\item Создайте метод \texttt{to\_hours} без аргументов,
который преобразует время в часы по формуле:
\[
\texttt{\_\_seconds} \div 3600
\]

\item Напишите цикл, который повторяется 10 раз. В каждой итерации программа должна:
\begin{enumerate}
\item запрашивать у пользователя время в секундах,
\item создавать экземпляр класса \texttt{TimeConverter} с этим значением,
\item вычислять время в минутах и часах с помощью соответствующих методов,
\item выводить результаты на экран.
\end{enumerate}
\end{enumerate}

\subsection*{Пример использования:}
\begin{verbatim}
seconds = 7200
converter = TimeConverter(seconds)
minutes = converter.to_minutes()
hours = converter.to_hours()
print(f"Время в минутах: {minutes}")
print(f"Время в часах: {hours}")
\end{verbatim}

\textbf{Вывод:}
\begin{verbatim}
Время в минутах: 120.0
Время в часах: 2.0
\end{verbatim}

\item


Напишите программу, которая создаёт класс \texttt{SpeedConverter} с методами для преобразования скорости
из километров в час в метры в секунду и мили в час. Программа должна запрашивать у пользователя скорость в км/ч
и выводить преобразованные значения.

\subsection*{Инструкции:}
\begin{enumerate}
\item Создайте класс \texttt{SpeedConverter} с методом
\texttt{\_\_init\_\_}, который принимает скорость в км/ч в
качестве аргумента и сохраняет её в атрибуте \texttt{self.\_\_kmh}.

\item Создайте метод \texttt{to\_ms},
без аргументов, который преобразует скорость в м/с по формуле:
\[
\texttt{\_\_kmh} \times \frac{1000}{3600}
\]

\item Создайте метод \texttt{to\_mph} без аргументов,
который преобразует скорость в мили/ч по формуле:
\[
\texttt{\_\_kmh} \div 1.60934
\]

\item Напишите цикл, который повторяется 10 раз. В каждой итерации программа должна:
\begin{enumerate}
\item запрашивать у пользователя скорость в км/ч,
\item создавать экземпляр класса \texttt{SpeedConverter} с этим значением,
\item вычислять скорость в м/с и милях/ч с помощью соответствующих методов,
\item выводить результаты на экран.
\end{enumerate}
\end{enumerate}

\subsection*{Пример использования:}
\begin{verbatim}
kmh = 100
converter = SpeedConverter(kmh)
ms = converter.to_ms()
mph = converter.to_mph()
print(f"Скорость в м/с: {ms}")
print(f"Скорость в милях/ч: {mph}")
\end{verbatim}

\textbf{Вывод:}
\begin{verbatim}
Скорость в м/с: 27.77777777777778
Скорость в милях/ч: 62.13727366498068
\end{verbatim}

\item 

Напишите программу, которая создаёт класс \texttt{AreaConverter} с методами для преобразования площади
из квадратных метров в гектары и акры. Программа должна запрашивать у пользователя площадь в м²
и выводить преобразованные значения.

\subsection*{Инструкции:}
\begin{enumerate}
\item Создайте класс \texttt{AreaConverter} с методом
\texttt{\_\_init\_\_}, который принимает площадь в м² в
качестве аргумента и сохраняет её в атрибуте \texttt{self.\_\_sq\_meters}.

\item Создайте метод \texttt{to\_hectares},
без аргументов, который преобразует площадь в гектары по формуле:
\[
\texttt{\_\_sq\_meters} \div 10000
\]

\item Создайте метод \texttt{to\_acres} без аргументов,
который преобразует площадь в акры по формуле:
\[
\texttt{\_\_sq\_meters} \div 4046.86
\]

\item Напишите цикл, который повторяется 10 раз. В каждой итерации программа должна:
\begin{enumerate}
\item запрашивать у пользователя площадь в м²,
\item создавать экземпляр класса \texttt{AreaConverter} с этим значением,
\item вычислять площадь в гектарах и акрах с помощью соответствующих методов,
\item выводить результаты на экран.
\end{enumerate}
\end{enumerate}

\subsection*{Пример использования:}
\begin{verbatim}
sq_meters = 10000
converter = AreaConverter(sq_meters)
hectares = converter.to_hectares()
acres = converter.to_acres()
print(f"Площадь в гектарах: {hectares}")
print(f"Площадь в акрах: {acres}")
\end{verbatim}

\textbf{Вывод:}
\begin{verbatim}
Площадь в гектары: 1.0
Площадь в акрах: 2.4710514233241505
\end{verbatim}

\item 

Напишите программу, которая создаёт класс \texttt{VolumeConverter} с методами для преобразования объёма
из литров в галлоны и кубические метры. Программа должна запрашивать у пользователя объём в литрах
и выводить преобразованные значения.

\subsection*{Инструкции:}
\begin{enumerate}
\item Создайте класс \texttt{VolumeConverter} с методом
\texttt{\_\_init\_\_}, который принимает объём в литрах в
качестве аргумента и сохраняет его в атрибуте \texttt{self.\_\_liters}.

\item Создайте метод \texttt{to\_gallons},
без аргументов, который преобразует объём в галлоны по формуле:
\[
\texttt{\_\_liters} \div 3.78541
\]

\item Создайте метод \texttt{to\_cubic\_meters} без аргументов,
который преобразует объём в кубические метры по формуле:
\[
\texttt{\_\_liters} \div 1000
\]

\item Напишите цикл, который повторяется 10 раз. В каждой итерации программа должна:
\begin{enumerate}
\item запрашивать у пользователя объём в литрах,
\item создавать экземпляр класса \texttt{VolumeConverter} с этим значением,
\item вычислять объём в галлонах и кубических метрах с помощью соответствующих методов,
\item выводить результаты на экран.
\end{enumerate}
\end{enumerate}

\subsection*{Пример использования:}
\begin{verbatim}
liters = 10
converter = VolumeConverter(liters)
gallons = converter.to_gallons()
cubic_meters = converter.to_cubic_meters()
print(f"Объём в галлонах: {gallons}")
print(f"Объём в кубических метрах: {cubic_meters}")
\end{verbatim}

\textbf{Вывод:}
\begin{verbatim}
Объём в галлонах: 2.641720523581484
Объём в кубических метрах: 0.01
\end{verbatim}

\item

Напишите программу, которая создаёт класс \texttt{EnergyConverter} с методами для преобразования энергии
из джоулей в калории и киловатт-часы. Программа должна запрашивать у пользователя энергию в джоулях
и выводить преобразованные значения.

\subsection*{Инструкции:}
\begin{enumerate}
\item Создайте класс \texttt{EnergyConverter} с методом
\texttt{\_\_init\_\_}, который принимает энергию в джоулях в
качестве аргумента и сохраняет её в атрибуте \texttt{self.\_\_joules}.

\item Создайте метод \texttt{to\_calories},
без аргументов, который преобразует энергию в калории по формуле:
\[
\texttt{\_\_joules} \div 4.184
\]

\item Создайте метод \texttt{to\_kwh} без аргументов,
который преобразует энергию в киловатт-часы по формуле:
\[
\texttt{\_\_joules} \div 3.6 \times 10^6
\]

\item Напишите цикл, который повторяется 10 раз. В каждой итерации программа должна:
\begin{enumerate}
\item запрашивать у пользователя энергию в джоулях,
\item создавать экземпляр класса \texttt{EnergyConverter} с этим значением,
\item вычислять энергию в калориях и киловатт-часах с помощью соответствующих методов,
\item выводить результаты на экран.
\end{enumerate}
\end{enumerate}

\subsection*{Пример использования:}
\begin{verbatim}
joules = 10000
converter = EnergyConverter(joules)
calories = converter.to_calories()
kwh = converter.to_kwh()
print(f"Энергия в калориях: {calories}")
print(f"Энергия в киловатт-часах: {kwh}")
\end{verbatim}

\textbf{Вывод:}
\begin{verbatim}
Энергия в калориях: 2390.057361376673
Энергия в киловатт-часах: 0.002777777777777778
\end{verbatim}

\item 

Напишите программу, которая создаёт класс \texttt{PowerConverter} с методами для преобразования мощности
из ватт в лошадиные силы и киловатты. Программа должна запрашивать у пользователя мощность в ваттах
и выводить преобразованные значения.

\subsection*{Инструкции:}
\begin{enumerate}
\item Создайте класс \texttt{PowerConverter} с методом
\texttt{\_\_init\_\_}, который принимает мощность в ваттах в
качестве аргумента и сохраняет её в атрибуте \texttt{self.\_\_watts}.

\item Создайте метод \texttt{to\_horsepower},
без аргументов, который преобразует мощность в лошадиные силы по формуле:
\[
\texttt{\_\_watts} \div 745.7
\]

\item Создайте метод \texttt{to\_kilowatts} без аргументов,
который преобразует мощность в киловатты по формуле:
\[
\texttt{\_\_watts} \div 1000
\]

\item Напишите цикл, который повторяется 10 раз. В каждой итерации программа должна:
\begin{enumerate}
\item запрашивать у пользователя мощность в ваттах,
\item создавать экземпляр класса \texttt{PowerConverter} с этим значением,
\item вычислять мощность в л.с. и киловаттах с помощью соответствующих методов,
\item выводить результаты на экран.
\end{enumerate}
\end{enumerate}

\subsection*{Пример использования:}
\begin{verbatim}
watts = 1000
converter = PowerConverter(watts)
horsepower = converter.to_horsepower()
kilowatts = converter.to_kilowatts()
print(f"Мощность в л.с.: {horsepower}")
print(f"Мощность в киловаттах: {kilowatts}")
\end{verbatim}

\textbf{Вывод:}
\begin{verbatim}
Мощность в л.с.: 1.3410220903956017
Мощность в киловаттах: 1.0
\end{verbatim}


\item

Напишите программу, которая создаёт класс \texttt{PressureConverter} с методами для преобразования давления
из паскалей в атмосферы и бары. Программа должна запрашивать у пользователя давление в паскалях
и выводить преобразованные значения.

\subsection*{Инструкции:}
\begin{enumerate}
\item Создайте класс \texttt{PressureConverter} с методом
\texttt{\_\_init\_\_}, который принимает давление в паскалях в
качестве аргумента и сохраняет его в атрибуте \texttt{self.\_\_pascals}.

\item Создайте метод \texttt{to\_atm},
без аргументов, который преобразует давление в атмосферы по формуле:
\[
\texttt{\_\_pascals} \div 101325
\]

\item Создайте метод \texttt{to\_bar} без аргументов,
который преобразует давление в бары по формуле:
\[
\texttt{\_\_pascals} \div 100000
\]

\item Напишите цикл, который повторяется 10 раз. В каждой итерации программа должна:
\begin{enumerate}
\item запрашивать у пользователя давление в паскалях,
\item создавать экземпляр класса \texttt{PressureConverter} с этим значением,
\item вычислять давление в атмосферах и барах с помощью соответствующих методов,
\item выводить результаты на экран.
\end{enumerate}
\end{enumerate}

\subsection*{Пример использования:}
\begin{verbatim}
pascals = 101325
converter = PressureConverter(pascals)
atm = converter.to_atm()
bar = converter.to_bar()
print(f"Давление в атмосферах: {atm}")
print(f"Давление в барах: {bar}")
\end{verbatim}

\textbf{Вывод:}
\begin{verbatim}
Давление в атмосферах: 1.0
Давление в барах: 1.01325
\end{verbatim}

\item 

Напишите программу, которая создаёт класс \texttt{ForceConverter} с методами для преобразования силы
из ньютонов в дины и фунты-силы. Программа должна запрашивать у пользователя силу в ньютонах
и выводить преобразованные значения.

\subsection*{Инструкции:}
\begin{enumerate}
\item Создайте класс \texttt{ForceConverter} с методом
\texttt{\_\_init\_\_}, который принимает силу в ньютонах в
качестве аргумента и сохраняет её в атрибуте \texttt{self.\_\_newtons}.

\item Создайте метод \texttt{to\_dyne},
без аргументов, который преобразует силу в дины по формуле:
\[
\texttt{\_\_newtons} \times 100000
\]

\item Создайте метод \texttt{to\_pound\_force} без аргументов,
который преобразует силу в фунты-силы по формуле:
\[
\texttt{\_\_newtons} \div 4.44822
\]

\item Напишите цикл, который повторяется 10 раз. В каждой итерации программа должна:
\begin{enumerate}
\item запрашивать у пользователя силу в ньютонах,
\item создавать экземпляр класса \texttt{ForceConverter} с этим значением,
\item вычислять силу в динах и фунтах-силы с помощью соответствующих методов,
\item выводить результаты на экран.
\end{enumerate}
\end{enumerate}

\subsection*{Пример использования:}
\begin{verbatim}
newtons = 10
converter = ForceConverter(newtons)
dyne = converter.to_dyne()
pound_force = converter.to_pound_force()
print(f"Сила в динах: {dyne}")
print(f"Сила в фунтах-силы: {pound_force}")
\end{verbatim}

\textbf{Вывод:}
\begin{verbatim}
Сила в динах: 1000000.0
Сила в фунтах-силы: 2.248089430997145
\end{verbatim}

\item 

\subsection*{Задание: Конвертер силы}
Напишите программу, которая создаёт класс \texttt{ForceConverter} с методами для преобразования силы
из ньютонов в дины и фунты-силы. Программа должна запрашивать у пользователя силу в ньютонах
и выводить преобразованные значения.

\subsection*{Инструкции:}
\begin{enumerate}
\item Создайте класс \texttt{ForceConverter} с методом
\texttt{\_\_init\_\_}, который принимает силу в ньютонах в
качестве аргумента и сохраняет её в атрибуте \texttt{self.\_\_newtons}.

\item Создайте метод \texttt{to\_dyne},
без аргументов, который преобразует силу в дины по формуле:
\[
\texttt{\_\_newtons} \times 100000
\]

\item Создайте метод \texttt{to\_pound\_force} без аргументов,
который преобразует силу в фунты-силы по формуле:
\[
\texttt{\_\_newtons} \div 4.44822
\]

\item Напишите цикл, который повторяется 10 раз. В каждой итерации программа должна:
\begin{enumerate}
\item запрашивать у пользователя силу в ньютонах,
\item создавать экземпляр класса \texttt{ForceConverter} с этим значением,
\item вычислять силу в динах и фунтах-силы с помощью соответствующих методов,
\item выводить результаты на экран.
\end{enumerate}
\end{enumerate}

\subsection*{Пример использования:}
\begin{verbatim}
newtons = 10
converter = ForceConverter(newtons)
dyne = converter.to_dyne()
pound_force = converter.to_pound_force()
print(f"Сила в динах: {dyne}")
print(f"Сила в фунтах-силы: {pound_force}")
\end{verbatim}

\textbf{Вывод:}
\begin{verbatim}
Сила в динах: 1000000.0
Сила в фунтах-силы: 2.248089430997145
\end{verbatim}

\item

Напишите программу, которая создаёт класс \texttt{ResistanceConverter} с методами для преобразования электрического сопротивления
из омов в килоомы и мегаомы. Программа должна запрашивать у пользователя сопротивление в омах
и выводить преобразованные значения.

\subsection*{Инструкции:}
\begin{enumerate}
\item Создайте класс \texttt{ResistanceConverter} с методом
\texttt{\_\_init\_\_}, который принимает сопротивление в омах в
качестве аргумента и сохраняет его в атрибуте \texttt{self.\_\_ohms}.

\item Создайте метод \texttt{to\_kiloohms},
без аргументов, который преобразует сопротивление в килоомы по формуле:
\[
\texttt{\_\_ohms} \div 1000
\]

\item Создайте метод \texttt{to\_megaohms} без аргументов,
который преобразует сопротивление в мегаомы по формуле:
\[
\texttt{\_\_ohms} \div 1000000
\]

\item Напишите цикл, который повторяется 10 раз. В каждой итерации программа должна:
\begin{enumerate}
\item запрашивать у пользователя сопротивление в омах,
\item создавать экземпляр класса \texttt{ResistanceConverter} с этим значением,
\item вычислять сопротивление в килоомах и мегаомах с помощью соответствующих методов,
\item выводить результаты на экран.
\end{enumerate}
\end{enumerate}

\subsection*{Пример использования:}
\begin{verbatim}
ohms = 10000
converter = ResistanceConverter(ohms)
kiloohms = converter.to_kiloohms()
megaohms = converter.to_megaohms()
print(f"Сопротивление в килоомах: {kiloohms}")
print(f"Сопротивление в мегаомах: {megaohms}")
\end{verbatim}

\textbf{Вывод:}
\begin{verbatim}
Сопротивление в килоомах: 10.0
Сопротивление в мегаомах: 0.01
\end{verbatim}

\item 


\section*{Дополнительные задания}

\item
Напишите программу, которая создаёт класс \texttt{Pentagon} с методами для вычисления площади
и периметра правильного пятиугольника. Программа должна запрашивать у пользователя длину сторону
и выводить вычисленные площадь и периметр.

\subsection*{Инструкции:}
\begin{enumerate}
\item Создайте класс \texttt{Pentagon} с методом
\texttt{\_\_init\_\_}, который принимает длину стороны пятиугольника в
качестве аргумента и сохраняет её в атрибуте \texttt{self.\_\_side}.

\item Создайте метод \texttt{calculate\_area},
без аргументов, который вычисляет площадь правильного пятиугольника по формуле:
\[
\frac{1}{4} \sqrt{5(5 + 2\sqrt{5})} \cdot \texttt{\_\_side}^2
\]

\item Создайте метод \texttt{calculate\_perimeter} без аргументов,
который вычисляет периметр пятиугольника по формуле:
\[
5 \cdot \texttt{\_\_side}
\]

\item Напишите цикл, который повторяется 10 раз. В каждой итерации программа должна:
\begin{enumerate}
\item запрашивать у пользователя длину стороны пятиугольника,
\item создавать экземпляр класса \texttt{Pentagon} с этой длиной,
\item вычислять площадь и периметр с помощью соответствующих методов,
\item выводить результаты на экран.
\end{enumerate}
\end{enumerate}

\subsection*{Пример использования:}
\begin{verbatim}
side = 5
pentagon = Pentagon(side)
area = pentagon.calculate_area()
perimeter = pentagon.calculate_perimeter()
print(f"Площадь пятиугольника: {area}")
print(f"Периметр пятиугольника: {perimeter}")
\end{verbatim}

\textbf{Вывод:}
\begin{verbatim}
Площадь пятиугольника: 43.01193501472417
Периметр пятиугольника: 25
\end{verbatim}

\item
Напишите программу, которая создаёт класс \texttt{Hexagon} с методами для вычисления площади
и периметра правильного шестиугольника. Программа должна запрашивать у пользователя длину стороны
и выводить вычисленные площадь и периметр.

\subsection*{Инструкции:}
\begin{enumerate}
\item Создайте класс \texttt{Hexagon} с методом
\texttt{\_\_init\_\_}, который принимает длину стороны шестиугольника в
качестве аргумента и сохраняет её в атрибуте \texttt{self.\_\_side}.

\item Создайте метод \texttt{calculate\_area},
без аргументов, который вычисляет площадь правильного шестиугольника по формуле:
\[
\frac{3\sqrt{3}}{2} \cdot \texttt{\_\_side}^2
\]

\item Создайте метод \texttt{calculate\_perimeter} без аргументов,
который вычисляет периметр шестиугольника по формуле:
\[
6 \cdot \texttt{\_\_side}
\]

\item Напишите цикл, который повторяется 10 раз. В каждой итерации программа должна:
\begin{enumerate}
\item запрашивать у пользователя длину стороны шестиугольника,
\item создавать экземпляр класса \texttt{Hexagon} с этой длиной,
\item вычислять площадь и периметр с помощью соответствующих методов,
\item выводить результаты на экран.
\end{enumerate}
\end{enumerate}

\subsection*{Пример использования:}
\begin{verbatim}
side = 4
hexagon = Hexagon(side)
area = hexagon.calculate_area()
perimeter = hexagon.calculate_perimeter()
print(f"Площадь шестиугольника: {area}")
print(f"Периметр шестиугольника: {perimeter}")
\end{verbatim}

\textbf{Вывод:}
\begin{verbatim}
Площадь шестиугольника: 41.569219381653056
Периметр шестиугольника: 24
\end{verbatim}

\item
Напишите программу, которая создаёт класс \texttt{AngleConverter} с методами для преобразования углов
из градусов в радианы и грады. Программа должна запрашивать у пользователя угол в градусах
и выводить преобразованные значения.

\subsection*{Инструкции:}
\begin{enumerate}
\item Создайте класс \texttt{AngleConverter} с методом
\texttt{\_\_init\_\_}, который принимает угол в градусах в
качестве аргумента и сохраняет его в атрибуте \texttt{self.\_\_degrees}.

\item Создайте метод \texttt{to\_radians},
без аргументов, который преобразует угол в радианы по формуле:
\[
\texttt{\_\_degrees} \times \frac{\pi}{180}
\]

\item Создайте метод \texttt{to\_gradians} без аргументов,
который преобразует угол в грады по формуле:
\[
\texttt{\_\_degrees} \times \frac{10}{9}
\]

\item Напишите цикл, который повторяется 10 раз. В каждой итерации программа должна:
\begin{enumerate}
\item запрашивать у пользователя угол в градусах,
\item создавать экземпляр класса \texttt{AngleConverter} с этим значением,
\item вычислять угол в радианах и градах с помощью соответствующих методов,
\item выводить результаты на экран.
\end{enumerate}
\end{enumerate}

\subsection*{Пример использования:}
\begin{verbatim}
degrees = 90
converter = AngleConverter(degrees)
radians = converter.to_radians()
gradians = converter.to_gradians()
print(f"Угол в радианах: {radians}")
print(f"Угол в градах: {gradians}")
\end{verbatim}

\textbf{Вывод:}
\begin{verbatim}
Угол в радианах: 1.5707963267948966
Угол в градах: 100.0
\end{verbatim}

\item
Напишите программу, которая создаёт класс \texttt{Tetrahedron} с методами для вычисления объёма
и площади поверхности правильного тетраэдра. Программа должна запрашивать у пользователя длину ребра
и выводить вычисленные объём и площадь поверхности.

\subsection*{Инструкции:}
\begin{enumerate}
\item Создайте класс \texttt{Tetrahedron} с методом
\texttt{\_\_init\_\_}, который принимает длину ребра тетраэдра в
качестве аргумента и сохраняет её в атрибуте \texttt{self.\_\_edge}.

\item Создайте метод \texttt{calculate\_volume},
без аргументов, который вычисляет объём тетраэдра по формуле:
\[
\frac{\texttt{\_\_edge}^3}{6\sqrt{2}}
\]

\item Создайте метод \texttt{calculate\_surface\_area} без аргументов,
который вычисляет площадь поверхности тетраэдра по формуле:
\[
\sqrt{3} \cdot \texttt{\_\_edge}^2
\]

\item Напишите цикл, который повторяется 10 раз. В каждой итерации программа должна:
\begin{enumerate}
\item запрашивать у пользователя длину ребра тетраэдра,
\item создавать экземпляр класса \texttt{Tetrahedron} с этой длиной,
\item вычислять объём и площадь поверхности с помощью соответствующих методов,
\item выводить результаты на экран.
\end{enumerate}
\end{enumerate}

\subsection*{Пример использования:}
\begin{verbatim}
edge = 3
tetrahedron = Tetrahedron(edge)
volume = tetrahedron.calculate_volume()
surface_area = tetrahedron.calculate_surface_area()
print(f"Объём тетраэдра: {volume}")
print(f"Площадь поверхности: {surface_area}")
\end{verbatim}

\textbf{Вывод:}
\begin{verbatim}
Объём тетраэдра: 3.181980515339464
Площадь поверхности: 15.588457268119896
\end{verbatim}

\item
Напишите программу, которая создаёт класс \texttt{CubicMeterConverter} с методами для преобразования объёма
из кубических метров в литры и кубические футы. Программа должна запрашивать у пользователя объём в кубометрах
и выводить преобразованные значения.

\subsection*{Инструкции:}
\begin{enumerate}
\item Создайте класс \texttt{CubicMeterConverter} с методом
\texttt{\_\_init\_\_}, который принимает объём в кубических метрах в
качестве аргумента и сохраняет его в атрибуте \texttt{self.\_\_cubic\_meters}.

\item Создайте метод \texttt{to\_liters},
без аргументов, который преобразует объём в литры по формуле:
\[
\texttt{\_\_cubic\_meters} \times 1000
\]

\item Создайте метод \texttt{\_\_cubic\_feet} без аргументов,
который преобразует объём в кубические футы по формуле:
\[
\texttt{\_\_cubic\_meters} \times 35.3147
\]

\item Напишите цикл, который повторяется 10 раз. В каждой итерации программа должна:
\begin{enumerate}
\item запрашивать у пользователя объём в кубических метрах,
\item создавать экземпляр класса \texttt{CubicMeterConverter} с этим значением,
\item вычислять объём в литрах и кубических футах с помощью соответствующих методов,
\item выводить результаты на экран.
\end{enumerate}
\end{enumerate}

\subsection*{Пример использования:}
\begin{verbatim}
cubic_meters = 2
converter = CubicMeterConverter(cubic_meters)
liters = converter.to_liters()
cubic_feet = converter.to_cubic_feet()
print(f"Объём в литрах: {liters}")
print(f"Объём в кубических футах: {cubic_feet}")
\end{verbatim}

\textbf{Вывод:}
\begin{verbatim}
Объём в литрах: 2000.0
Объём в кубических футах: 70.6294
\end{verbatim}

\item
Напишите программу, которая создаёт класс \texttt{RightTriangle} с методами для вычисления гипотенузы
и площади прямоугольного треугольника. Программа должна запрашивать у пользователя длину одного катета
(второй катет фиксирован и равен 4) и выводить вычисленные гипотенузу и площадь.

\subsection*{Инструкции:}
\begin{enumerate}
\item Создайте класс \texttt{RightTriangle} с методом
\texttt{\_\_init\_\_}, который принимает длину первого катета в
качестве аргумента и сохраняет его в атрибуте \texttt{self.\_\_cathetus}.
Второй катет фиксирован и равен 4.

\item Создайте метод \texttt{calculate\_hypotenuse},
без аргументов, который вычисляет гипотенузу по формуле:
\[
\sqrt{\texttt{\_\_cathetus}^2 + 4^2}
\]

\item Создайте метод \texttt{calculate\_area} без аргументов,
который вычисляет площадь треугольника по формуле:
\[
\frac{\texttt{\_\_cathetus} \times 4}{2}
\]

\item Напишите цикл, который повторяется 10 раз. В каждой итерации программа должна:
\begin{enumerate}
\item запрашивать у пользователя длину катета,
\item создавать экземпляр класса \texttt{RightTriangle} с этой длиной,
\item вычислять гипотенузу и площадь с помощью соответствующих методов,
\item выводить результаты на экран.
\end{enumerate}
\end{enumerate}

\subsection*{Пример использования:}
\begin{verbatim}
cathetus = 3
triangle = RightTriangle(cathetus)
hypotenuse = triangle.calculate_hypotenuse()
area = triangle.calculate_area()
print(f"Гипотенуза: {hypotenuse}")
print(f"Площадь: {area}")
\end{verbatim}

\textbf{Вывод:}
\begin{verbatim}
Гипотенуза: 5.0
Площадь: 6.0
\end{verbatim}

\end{enumerate}

\subsubsection{Задача 2}

\begin{enumerate}
    \item

Написать программу, которая создаёт класс \texttt{LeapYearChecker} 
для определения високосного года. В классе должен быть статический метод
\texttt{is\_leap\_year} и возвращать \texttt{True}, если год високосный, 
и \texttt{False} в противном случае. 
Программа также должна использовать цикл для проверки каждого года от 
2000 до 2099 и вывода результата на экран.

\subsection*{Инструкции:}
\begin{enumerate}
    \item Создайте класс \texttt{LeapYearChecker}.
    \item Создайте \textbf{статический} метод \texttt{is\_leap\_year}, который принимает год в качестве аргумента и проверяет, является ли год високосным. Если год делится на 4 без остатка и не делится на 100 без остатка, или делится на 400 без остатка, то возвращает \texttt{True}. В противном случае возвращает \texttt{False}.
    \item Используйте цикл для проверки каждого года от 2000 до 2099 (включительно), вызывая статический метод \texttt{is\_leap\_year} и выводя результат на экран.
\end{enumerate}

\subsection*{Пример использования:}
\begin{lstlisting}[language=Python]
    v = LeapYearChecker.is_leap_year(1999)
\end{lstlisting}
Вывод (первые и последние строки):
\begin{verbatim}
2000 True
2001 False
...
2098 False
2099 False
\end{verbatim}

\item
Написать программу, которая создаёт класс \texttt{PrimeChecker} 
для определения простого числа. В классе должен быть статический метод
\texttt{is\_prime} и возвращать \texttt{True}, если число простое, 
и \texttt{False} в противном случае. 
Программа также должна использовать цикл для проверки каждого числа от 
1 до 100 и вывода результата на экран.

\subsection*{Инструкции:}
\begin{enumerate}
    \item Создайте класс \texttt{PrimeChecker}.
    \item Создайте \textbf{статический} метод \texttt{is\_prime}, который принимает число в качестве аргумента и проверяет, является ли число простым. Простое число делится только на 1 и на само себя.
    \item Используйте цикл для проверки каждого числа от 1 до 100 (включительно), вызывая статический метод \texttt{is\_prime} и выводя результат на экран.
\end{enumerate}

\subsection*{Пример использования:}
\begin{lstlisting}[language=Python]
    v = PrimeChecker.is_prime(17)
\end{lstlisting}
Вывод (первые и последние строки):
\begin{verbatim}
1 False
2 True
3 True
...
98 False
99 False
100 False
\end{verbatim}

\item
Написать программу, которая создаёт класс \texttt{EvenChecker} 
для определения чётности числа. В классе должен быть статический метод
\texttt{is\_even} и возвращать \texttt{True}, если число чётное, 
и \texttt{False} в противном случае. 
Программа также должна использовать цикл для проверки каждого числа от 
1 до 50 и вывода результата на экран.

\subsection*{Инструкции:}
\begin{enumerate}
    \item Создайте класс \texttt{EvenChecker}.
    \item Создайте \textbf{статический} метод \texttt{is\_even}, который принимает число в качестве аргумента и проверяет, является ли число чётным.
    \item Используйте цикл для проверки каждого числа от 1 до 50 (включительно), вызывая статический метод \texttt{is\_even} и выводя результат на экран.
\end{enumerate}

\subsection*{Пример использования:}
\begin{lstlisting}[language=Python]
    v = EvenChecker.is_even(25)
\end{lstlisting}
Вывод (первые и последние строки):
\begin{verbatim}
1 False
2 True
3 False
...
48 True
49 False
50 True
\end{verbatim}

\item
Написать программу, которая создаёт класс \texttt{SquareChecker} 
для определения квадратного числа. В классе должен быть статический метод
\texttt{is\_square} и возвращать \texttt{True}, если число является квадратом целого числа, 
и \texttt{False} в противном случае. 
Программа также должна использовать цикл для проверки каждого числа от 
1 до 100 и вывода результата на экран.

\subsection*{Инструкции:}
\begin{enumerate}
    \item Создайте класс \texttt{SquareChecker}.
    \item Создайте \textbf{статический} метод \texttt{is\_square}, который принимает число в качестве аргумента и проверяет, является ли число квадратом целого числа.
    \item Используйте цикл для проверки каждого числа от 1 до 100 (включительно), вызывая статический метод \texttt{is\_square} и выводя результат на экран.
\end{enumerate}

\subsection*{Пример использования:}
\begin{lstlisting}[language=Python]
    v = SquareChecker.is_square(36)
\end{lstlisting}
Вывод (первые и последние строки):
\begin{verbatim}
1 True
2 False
3 False
...
99 False
100 True
\end{verbatim}

\item
Написать программу, которая создаёт класс \texttt{FactorialCalculator} 
для вычисления факториала числа. В классе должен быть статический метод
\texttt{factorial} и возвращать факториал числа. 
Программа также должна использовать цикл для вычисления факториала каждого числа от 
1 до 10 и вывода результата на экран.

\subsection*{Инструкции:}
\begin{enumerate}
    \item Создайте класс \texttt{FactorialCalculator}.
    \item Создайте \textbf{статический} метод \texttt{factorial}, который принимает число в качестве аргумента и возвращает его факториал.
    \item Используйте цикл для вычисления факториала каждого числа от 1 до 10 (включительно), вызывая статический метод \texttt{factorial} и выводя результат на экран.
\end{enumerate}

\subsection*{Пример использования:}
\begin{lstlisting}[language=Python]
    v = FactorialCalculator.factorial(5)
\end{lstlisting}
Вывод (первые и последние строки):
\begin{verbatim}
1 1
2 2
3 6
...
9 362880
10 3628800
\end{verbatim}

\item
Написать программу, которая создаёт класс \texttt{PalindromeChecker} 
для определения палиндрома числа. В классе должен быть статический метод
\texttt{is\_palindrome} и возвращать \texttt{True}, если число является палиндромом, 
и \texttt{False} в противном случае. 
Программа также должна использовать цикл для проверки каждого числа от 
100 до 200 и вывода результата на экран.

\subsection*{Инструкции:}
\begin{enumerate}
    \item Создайте класс \texttt{PalindromeChecker}.
    \item Создайте \textbf{статический} метод \texttt{is\_palindrome}, который принимает число в качестве аргумента и проверяет, является ли число палиндромом (читается одинаково слева направо и справа налево).
    \item Используйте цикл для проверки каждого числа от 100 до 200 (включительно), вызывая статический метод \texttt{is\_palindrome} и выводя результат на экран.
\end{enumerate}

\subsection*{Пример использования:}
\begin{lstlisting}[language=Python]
    v = PalindromeChecker.is_palindrome(121)
\end{lstlisting}
Вывод (первые и последние строки):
\begin{verbatim}
100 False
101 True
102 False
...
199 False
200 False
\end{verbatim}

\item
Написать программу, которая создаёт класс \texttt{ArmstrongChecker} 
для определения числа Армстронга. В классе должен быть статический метод
\texttt{is\_armstrong} и возвращать \texttt{True}, если число является числом Армстронга, 
и \texttt{False} в противном случае. 
Программа также должна использовать цикл для проверки каждого числа от 
100 до 500 и вывода результата на экран.

\subsection*{Инструкции:}
\begin{enumerate}
    \item Создайте класс \texttt{ArmstrongChecker}.
    \item Создайте \textbf{статический} метод \texttt{is\_armstrong}, который принимает число в качестве аргумента и проверяет, является ли число числом Армстронга (сумма цифр в степени, равной количеству цифр, равна самому числу).
    \item Используйте цикл для проверки каждого числа от 100 до 500 (включительно), вызывая статический метод \texttt{is\_armstrong} и выводя результат на экран.
\end{enumerate}

\subsection*{Пример использования:}
\begin{lstlisting}[language=Python]
    v = ArmstrongChecker.is_armstrong(153)
\end{lstlisting}
Вывод (первые и последние строки):
\begin{verbatim}
100 False
101 False
102 False
...
499 False
500 False
\end{verbatim}

\item
Написать программу, которая создаёт класс \texttt{PerfectNumberChecker} 
для определения совершенного числа. В классе должен быть статический метод
\texttt{is\_perfect} и возвращать \texttt{True}, если число является совершенным, 
и \texttt{False} в противном случае. 
Программа также должна использовать цикл для проверки каждого числа от 
1 до 1000 и вывода результата на экран.

\subsection*{Инструкции:}
\begin{enumerate}
    \item Создайте класс \texttt{PerfectNumberChecker}.
    \item Создайте \textbf{статический} метод \texttt{is\_perfect}, который принимает число в качестве аргумента и проверяет, является ли число совершенным (сумма делителей равна числу).
    \item Используйте цикл для проверки каждого числа от 1 до 1000 (включительно), вызывая статический метод \texttt{is\_perfect} и выводя результат на экран.
\end{enumerate}

\subsection*{Пример использования:}
\begin{lstlisting}[language=Python]
    v = PerfectNumberChecker.is_perfect(28)
\end{lstlisting}
Вывод (первые и последние строки):
\begin{verbatim}
1 False
2 False
3 False
...
998 False
999 False
1000 False
\end{verbatim}

\item
Написать программу, которая создаёт класс \texttt{FibonacciChecker} 
для проверки числа Фибоначчи. В классе должен быть статический метод
\texttt{is\_fibonacci} и возвращать \texttt{True}, если число является числом Фибоначчи, 
и \texttt{False} в противном случае. 
Программа также должна использовать цикл для проверки каждого числа от 
1 до 100 и вывода результата на экран.

\subsection*{Инструкции:}
\begin{enumerate}
    \item Создайте класс \texttt{FibonacciChecker}.
    \item Создайте \textbf{статический} метод \texttt{is\_fibonacci}, который принимает число в качестве аргумента и проверяет, является ли число числом Фибоначчи.
    \item Используйте цикл для проверки каждого числа от 1 до 100 (включительно), вызывая статический метод \texttt{is\_fibonacci} и выводя результат на экран.
\end{enumerate}

\subsection*{Пример использования:}
\begin{lstlisting}[language=Python]
    v = FibonacciChecker.is_fibonacci(21)
\end{lstlisting}
Вывод (первые и последние строки):
\begin{verbatim}
1 True
2 True
3 True
...
98 False
99 False
100 False
\end{verbatim}

\item
Написать программу, которая создаёт класс \texttt{PowerOfTwoChecker} 
для проверки степени двойки. В классе должен быть статический метод
\texttt{is\_power\_of\_two} и возвращать \texttt{True}, если число является степенью двойки, 
и \texttt{False} в противном случае. 
Программа также должна использовать цикл для проверки каждого числа от 
1 до 128 и вывода результата на экран.

\subsection*{Инструкции:}
\begin{enumerate}
    \item Создайте класс \texttt{PowerOfTwoChecker}.
    \item Создайте \textbf{статический} метод \texttt{is\_power\_of\_two}, который принимает число в качестве аргумента и проверяет, является ли число степенью двойки.
    \item Используйте цикл для проверки каждого числа от 1 до 128 (включительно), вызывая статический метод \texttt{is\_power\_of\_two} и выводя результат на экран.
\end{enumerate}

\subsection*{Пример использования:}
\begin{lstlisting}[language=Python]
    v = PowerOfTwoChecker.is_power_of_two(64)
\end{lstlisting}
Вывод (первые и последние строки):
\begin{verbatim}
1 True
2 True
3 False
...
127 False
128 True
\end{verbatim}

\item
Написать программу, которая создаёт класс \texttt{SumOfDigitsCalculator} 
для вычисления суммы цифр числа. В классе должен быть статический метод
\texttt{sum\_of\_digits} и возвращать сумму цифр. 
Программа также должна использовать цикл для вычисления суммы цифр каждого числа от 
1 до 50 и вывода результата на экран.

\subsection*{Инструкции:}
\begin{enumerate}
    \item Создайте класс \texttt{SumOfDigitsCalculator}.
    \item Создайте \textbf{статический} метод \texttt{sum\_of\_digits}, который принимает число в качестве аргумента и возвращает сумму его цифр.
    \item Используйте цикл для вычисления суммы цифр каждого числа от 1 до 50 (включительно), вызывая статический метод \texttt{sum\_of\_digits} и выводя результат на экран.
\end{enumerate}

\subsection*{Пример использования:}
\begin{lstlisting}[language=Python]
    v = SumOfDigitsCalculator.sum_of_digits(123)
\end{lstlisting}
Вывод (первые и последние строки):
\begin{verbatim}
1 1
2 2
3 3
...
49 13
50 5
\end{verbatim}

\item
Написать программу, которая создаёт класс \texttt{PrimeSumCalculator} 
для вычисления суммы простых чисел в диапазоне. В классе должен быть статический метод
\texttt{sum\_of\_primes} и возвращать сумму простых чисел в заданном диапазоне. 
Программа также должна использовать цикл для вычисления суммы простых чисел от 
1 до 100 и вывода результата на экран.

\subsection*{Инструкции:}
\begin{enumerate}
    \item Создайте класс \texttt{PrimeSumCalculator}.
    \item Создайте \textbf{статический} метод \texttt{sum\_of\_primes}, который принимает два аргумента (начало и конец диапазона) и возвращает сумму простых чисел в этом диапазоне.
    \item Используйте метод для вычисления суммы простых чисел от 1 до 100 и выведите результат.
\end{enumerate}

\subsection*{Пример использования:}
\begin{lstlisting}[language=Python]
    v = PrimeSumCalculator.sum_of_primes(1, 10)
\end{lstlisting}
Вывод:
\begin{verbatim}
Сумма простых чисел от 1 до 100: 1060
\end{verbatim}

\item
Написать программу, которая создаёт класс \texttt{DigitCountCalculator} 
для подсчёта количества цифр в числе. В классе должен быть статический метод
\texttt{digit\_count} и возвращать количество цифр. 
Программа также должна использовать цикл для подсчёта цифр каждого числа от 
1 до 100 и вывода результата на экран.

\subsection*{Инструкции:}
\begin{enumerate}
    \item Создайте класс \texttt{DigitCountCalculator}.
    \item Создайте \textbf{статический} метод \texttt{digit\_count}, который принимает число в качестве аргумента и возвращает количество его цифр.
    \item Используйте цикл для подсчёта цифр каждого числа от 1 до 100 (включительно), вызывая статический метод \texttt{digit\_count} и выводя результат на экран.
\end{enumerate}

\subsection*{Пример использования:}
\begin{lstlisting}[language=Python]
    v = DigitCountCalculator.digit_count(12345)
\end{lstlisting}
Вывод (первые и последние строки):
\begin{verbatim}
1 1
2 1
3 1
...
99 2
100 3
\end{verbatim}

\item
Написать программу, которая создаёт класс \texttt{BinaryConverter} 
для преобразования числа в двоичное представление. В классе должен быть статический метод
\texttt{to\_binary} и возвращать строку с двоичным представлением числа. 
Программа также должна использовать цикл для преобразования каждого числа от 
1 до 16 и вывода результата на экран.

\subsection*{Инструкции:}
\begin{enumerate}
    \item Создайте класс \texttt{BinaryConverter}.
    \item Создайте \textbf{статический} метод \texttt{to\_binary}, который принимает число в качестве аргумента и возвращает его двоичное представление в виде строки.
    \item Используйте цикл для преобразования каждого числа от 1 до 16 (включительно), вызывая статический метод \texttt{to\_binary} и выводя результат на экран.
\end{enumerate}

\subsection*{Пример использования:}
\begin{lstlisting}[language=Python]
    v = BinaryConverter.to_binary(10)
\end{lstlisting}
Вывод (первые и последние строки):
\begin{verbatim}
1 1
2 10
3 11
...
15 1111
16 10000
\end{verbatim}

\item
Написать программу, которая создаёт класс \texttt{HexConverter} 
для преобразования числа в шестнадцатеричное представление. В классе должен быть статический метод
\texttt{to\_hex} и возвращать строку с шестнадцатеричным представлением числа. 
Программа также должна использовать цикл для преобразования каждого числа от 
1 до 20 и вывода результата на экран.

\subsection*{Инструкции:}
\begin{enumerate}
    \item Создайте класс \texttt{HexConverter}.
    \item Создайте \textbf{статический} метод \texttt{to\_hex}, который принимает число в качестве аргумента и возвращает его шестнадцатеричное представление в виде строки.
    \item Используйте цикл для преобразования каждого числа от 1 до 20 (включительно), вызывая статический метод \texttt{to\_hex} и выводя результат на экран.
\end{enumerate}

\subsection*{Пример использования:}
\begin{lstlisting}[language=Python]
    v = HexConverter.to_hex(255)
\end{lstlisting}
Вывод (первые и последние строки):
\begin{verbatim}
1 1
2 2
3 3
...
19 13
20 14
\end{verbatim}

\item
Написать программу, которая создаёт класс \texttt{DivisorChecker} 
для проверки делителей числа. В классе должен быть статический метод
\texttt{get\_divisors} и возвращать список делителей числа. 
Программа также должна использовать цикл для вывода делителей каждого числа от 
1 до 20 и вывода результата на экран.

\subsection*{Инструкции:}
\begin{enumerate}
    \item Создайте класс \texttt{DivisorChecker}.
    \item Создайте \textbf{статический} метод \texttt{get\_divisors}, который принимает число в качестве аргумента и возвращает список его делителей.
    \item Используйте цикл для вывода делителей каждого числа от 1 до 20 (включительно), вызывая статический метод \texttt{get\_divisors} и выводя результат на экран.
\end{enumerate}

\subsection*{Пример использования:}
\begin{lstlisting}[language=Python]
    v = DivisorChecker.get_divisors(12)
\end{lstlisting}
Вывод (первые и последние строки):
\begin{verbatim}
1 [1]
2 [1, 2]
3 [1, 3]
...
19 [1, 19]
20 [1, 2, 4, 5, 10, 20]
\end{verbatim}

\item
Написать программу, которая создаёт класс \texttt{Multiplier} 
для создания таблицы умножения. В классе должен быть статический метод
\texttt{multiply\_table} и выводить таблицу умножения для заданного числа. 
Программа также должна использовать цикл для вывода таблицы умножения для чисел от 
1 до 10 и вывода результата на экран.

\subsection*{Инструкции:}
\begin{enumerate}
    \item Создайте класс \texttt{Multiplier}.
    \item Создайте \textbf{статический} метод \texttt{multiply\_table}, который принимает число в качестве аргумента и выводит таблицу умножения для этого числа от 1 до 10.
    \item Используйте цикл для вывода таблицы умножения для чисел от 1 до 10 (включительно), вызывая статический метод \texttt{multiply\_table} и выводя результат на экран.
\end{enumerate}

\subsection*{Пример использования:}
\begin{lstlisting}[language=Python]
    Multiplier.multiply_table(5)
\end{lstlisting}
Вывод (для числа 5):
\begin{verbatim}
5 * 1 = 5
5 * 2 = 10
...
5 * 10 = 50
\end{verbatim}

\item
Написать программу, которая создаёт класс \texttt{GCDCalculator} 
для вычисления НОД двух чисел. В классе должен быть статический метод
\texttt{gcd} и возвращать наибольший общий делитель. 
Программа также должна использовать цикл для вычисления НОД чисел 
(1,1), (2,4), (3,9), ..., (10,100) и вывода результата на экран.

\subsection*{Инструкции:}
\begin{enumerate}
    \item Создайте класс \texttt{GCDCalculator}.
    \item Создайте \textbf{статический} метод \texttt{gcd}, который принимает два числа в качестве аргументов и возвращает их наибольший общий делитель.
    \item Используйте цикл для вычисления НОД пар чисел (1,1), (2,4), (3,9), ..., (10,100), вызывая статический метод \texttt{gcd} и выводя результат на экран.
\end{enumerate}

\subsection*{Пример использования:}
\begin{lstlisting}[language=Python]
    v = GCDCalculator.gcd(48, 18)
\end{lstlisting}
Вывод:
\begin{verbatim}
НОД(1, 1) = 1
НОД(2, 4) = 2
НОД(3, 9) = 3
...
НОД(10, 100) = 10
\end{verbatim}

\item
Написать программу, которая создаёт класс \texttt{LCMCalculator} 
для вычисления НОК двух чисел. В классе должен быть статический метод
\texttt{lcm} и возвращать наименьшее общее кратное. 
Программа также должна использовать цикл для вычисления НОК чисел 
(1,1), (2,3), (3,5), ..., (10,11) и вывода результата на экран.

\subsection*{Инструкции:}
\begin{enumerate}
    \item Создайте класс \texttt{LCMCalculator}.
    \item Создайте \textbf{статический} метод \texttt{lcm}, который принимает два числа в качестве аргументов и возвращает их наименьшее общее кратное.
    \item Используйте цикл для вычисления НОК пар чисел (1,1), (2,3), (3,5), ..., (10,11), вызывая статический метод \texttt{lcm} и выводя результат на экран.
\end{enumerate}

\subsection*{Пример использования:}
\begin{lstlisting}[language=Python]
    v = LCMCalculator.lcm(4, 6)
\end{lstlisting}
Вывод:
\begin{verbatim}
НОК(1, 1) = 1
НОК(2, 3) = 6
НОК(3, 5) = 15
...
НОК(10, 11) = 110
\end{verbatim}

\item
Написать программу, которая создаёт класс \texttt{DigitReverse} 
для разворота цифр числа. В классе должен быть статический метод
\texttt{reverse\_digits} и возвращать число с обратным порядком цифр. 
Программа также должна использовать цикл для разворота каждого числа от 
10 до 20 и вывода результата на экран.

\subsection*{Инструкции:}
\begin{enumerate}
    \item Создайте класс \texttt{DigitReverse}.
    \item Создайте \textbf{статический} метод \texttt{reverse\_digits}, который принимает число в качестве аргумента и возвращает число с обратным порядком цифр.
    \item Используйте цикл для разворота каждого числа от 10 до 20 (включительно), вызывая статический метод \texttt{reverse\_digits} и выводя результат на экран.
\end{enumerate}

\subsection*{Пример использования:}
\begin{lstlisting}[language=Python]
    v = DigitReverse.reverse_digits(123)
\end{lstlisting}
Вывод:
\begin{verbatim}
10 1
11 11
12 21
13 31
...
19 91
20 2
\end{verbatim}

\item
Написать программу, которая создаёт класс \texttt{NumberTypeChecker} 
для определения типа числа (положительное/отрицательное/ноль). В классе должен быть статический метод
\texttt{check\_number\_type} и возвращать строку с типом числа. 
Программа также должна использовать цикл для проверки чисел 
[-5, -4, -3, -2, -1, 0, 1, 2, 3, 4, 5] и вывода результата на экран.

\subsection*{Инструкции:}
\begin{enumerate}
    \item Создайте класс \texttt{NumberTypeChecker}.
    \item Создайте \textbf{статический} метод \texttt{check\_number\_type}, который принимает число в качестве аргумента и возвращает строку "positive", "negative" или "zero".
    \item Используйте цикл для проверки чисел [-5, -4, -3, -2, -1, 0, 1, 2, 3, 4, 5], вызывая статический метод \texttt{check\_number\_type} и выводя результат на экран.
\end{enumerate}

\subsection*{Пример использования:}
\begin{lstlisting}[language=Python]
    v = NumberTypeChecker.check_number_type(-7)
\end{lstlisting}
Вывод:
\begin{verbatim}
-5 negative
-4 negative
-3 negative
-2 negative
-1 negative
0 zero
1 positive
2 positive
3 positive
4 positive
5 positive
\end{verbatim}

\item
Написать программу, которая создаёт класс \texttt{FactorialChecker} 
для проверки факториала числа. В классе должен быть статический метод
\texttt{is\_factorial} и возвращать \texttt{True}, если число является факториалом какого-либо числа, 
и \texttt{False} в противном случае. 
Программа также должна использовать цикл для проверки каждого числа от 
1 до 120 и вывода результата на экран.

\subsection*{Инструкции:}
\begin{enumerate}
    \item Создайте класс \texttt{FactorialChecker}.
    \item Создайте \textbf{статический} метод \texttt{is\_factorial}, который принимает число в качестве аргумента и проверяет, является ли число факториалом какого-либо числа.
    \item Используйте цикл для проверки каждого числа от 1 до 120 (включительно), вызывая статический метод \texttt{is\_factorial} и выводя результат на экран.
\end{enumerate}

\subsection*{Пример использования:}
\begin{lstlisting}[language=Python]
    v = FactorialChecker.is_factorial(24)
\end{lstlisting}
Вывод (первые и последние строки):
\begin{verbatim}
1 True
2 True
3 False
...
119 False
120 True
\end{verbatim}

\item
Написать программу, которая создаёт класс \texttt{PowerChecker} 
для проверки степени числа. В классе должен быть статический метод
\texttt{is\_power} и возвращать \texttt{True}, если число является степенью заданного основания, 
и \texttt{False} в противном случае. 
Программа также должна использовать цикл для проверки каждого числа от 
1 до 100 относительно основания 3 и вывода результата на экран.

\subsection*{Инструкции:}
\begin{enumerate}
    \item Создайте класс \texttt{PowerChecker}.
    \item Создайте \textbf{статический} метод \texttt{is\_power}, который принимает число и основание в качестве аргументов и проверяет, является ли число степенью основания.
    \item Используйте цикл для проверки каждого числа от 1 до 100 (включительно) относительно основания 3, вызывая статический метод \texttt{is\_power} и выводя результат на экран.
\end{enumerate}

\subsection*{Пример использования:}
\begin{lstlisting}[language=Python]
    v = PowerChecker.is_power(81, 3)
\end{lstlisting}
Вывод (первые и последние строки):
\begin{verbatim}
1 True
2 False
3 True
...
99 False
100 False
\end{verbatim}

\item
Написать программу, которая создаёт класс \texttt{DigitProductCalculator} 
для вычисления произведения цифр числа. В классе должен быть статический метод
\texttt{digit\_product} и возвращать произведение цифр. 
Программа также должна использовать цикл для вычисления произведения цифр каждого числа от 
1 до 50 и вывода результата на экран.

\subsection*{Инструкции:}
\begin{enumerate}
    \item Создайте класс \texttt{DigitProductCalculator}.
    \item Создайте \textbf{статический} метод \texttt{digit\_product}, который принимает число в качестве аргумента и возвращает произведение его цифр.
    \item Используйте цикл для вычисления произведения цифр каждого числа от 1 до 50 (включительно), вызывая статический метод \texttt{digit\_product} и выводя результат на экран.
\end{enumerate}

\subsection*{Пример использования:}
\begin{lstlisting}[language=Python]
    v = DigitProductCalculator.digit_product(123)
\end{lstlisting}
Вывод (первые и последние строки):
\begin{verbatim}
1 1
2 2
3 3
...
49 36
50 0
\end{verbatim}

\item
Написать программу, которая создаёт класс \texttt{NumberLengthChecker} 
для проверки длины числа. В классе должен быть статический метод
\texttt{get\_length} и возвращать количество цифр в числе. 
Программа также должна использовать цикл для проверки длины каждого числа от 
1 до 1000 с шагом 100 и вывода результата на экран.

\subsection*{Инструкции:}
\begin{enumerate}
    \item Создайте класс \texttt{NumberLengthChecker}.
    \item Создайте \textbf{статический} метод \texttt{get\_length}, который принимает число в качестве аргумента и возвращает количество его цифр.
    \item Используйте цикл для проверки длины чисел 1, 100, 200, 300, 400, 500, 600, 700, 800, 900, 1000, вызывая статический метод \texttt{get\_length} и выводя результат на экран.
\end{enumerate}

\subsection*{Пример использования:}
\begin{lstlisting}[language=Python]
    v = NumberLengthChecker.get_length(12345)
\end{lstlisting}
Вывод:
\begin{verbatim}
1 1
100 3
200 3
300 3
400 3
500 3
600 3
700 3
800 3
900 3
1000 4
\end{verbatim}

\item
Написать программу, которая создаёт класс \texttt{NumberSquareSumCalculator} 
для вычисления суммы квадратов чисел. В классе должен быть статический метод
\texttt{square\_sum} и возвращать сумму квадратов чисел в диапазоне. 
Программа также должна использовать метод для вычисления суммы квадратов чисел от 
1 до 10 и вывода результата на экран.

\subsection*{Инструкции:}
\begin{enumerate}
    \item Создайте класс \texttt{NumberSquareSumCalculator}.
    \item Создайте \textbf{статический} метод \texttt{square\_sum}, который принимает два аргумента (начало и конец диапазона) и возвращает сумму квадратов чисел в этом диапазоне.
    \item Используйте метод для вычисления суммы квадратов чисел от 1 до 10 и выведите результат.
\end{enumerate}

\subsection*{Пример использования:}
\begin{lstlisting}[language=Python]
    v = NumberSquareSumCalculator.square_sum(1, 3)
\end{lstlisting}
Вывод:
\begin{verbatim}
Сумма квадратов чисел от 1 до 10: 385
\end{verbatim}

\item
Написать программу, которая создаёт класс \texttt{NumberCubeSumCalculator} 
для вычисления суммы кубов чисел. В классе должен быть статический метод
\texttt{cube\_sum} и возвращать сумму кубов чисел в диапазоне. 
Программа также должна использовать метод для вычисления суммы кубов чисел от 
1 до 10 и вывода результата на экран.

\subsection*{Инструкции:}
\begin{enumerate}
    \item Создайте класс \texttt{NumberCubeSumCalculator}.
    \item Создайте \textbf{статический} метод \texttt{cube\_sum}, который принимает два аргумента (начало и конец диапазона) и возвращает сумму кубов чисел в этом диапазоне.
    \item Используйте метод для вычисления суммы кубов чисел от 1 до 10 и выведите результат.
\end{enumerate}

\subsection*{Пример использования:}
\begin{lstlisting}[language=Python]
    v = NumberCubeSumCalculator.cube_sum(1, 3)
\end{lstlisting}
Вывод:
\begin{verbatim}
Сумма кубов чисел от 1 до 10: 3025
\end{verbatim}

\item
Написать программу, которая создаёт класс \texttt{NumberRangeChecker} 
для проверки числа на принадлежность диапазону. В классе должен быть статический метод
\texttt{in\_range} и возвращать \texttt{True}, если число находится в заданном диапазоне, 
и \texttt{False} в противном случае. 
Программа также должна использовать цикл для проверки чисел от 
-5 до 5 на принадлежность диапазону [0, 10] и вывода результата на экран.

\subsection*{Инструкции:}
\begin{enumerate}
    \item Создайте класс \texttt{NumberRangeChecker}.
    \item Создайте \textbf{статический} метод \texttt{in\_range}, который принимает число, начало и конец диапазона и проверяет, находится ли число в этом диапазоне.
    \item Используйте цикл для проверки чисел от -5 до 5 (включительно) на принадлежность диапазону [0, 10], вызывая статический метод \texttt{in\_range} и выводя результат на экран.
\end{enumerate}

\subsection*{Пример использования:}
\begin{lstlisting}[language=Python]
    v = NumberRangeChecker.in_range(5, 0, 10)
\end{lstlisting}
Вывод:
\begin{verbatim}
-5 False
-4 False
-3 False
-2 False
-1 False
0 True
1 True
2 True
3 True
4 True
5 True
\end{verbatim}

\item
Написать программу, которая создаёт класс \texttt{NumberSignChecker} 
для проверки знака числа. В классе должен быть статический метод
\texttt{get\_sign} и возвращать строку с знаком числа (+, - или 0). 
Программа также должна использовать цикл для проверки чисел 
[-5, -4, -3, -2, -1, 0, 1, 2, 3, 4, 5] и вывода результата на экран.

\subsection*{Инструкции:}
\begin{enumerate}
    \item Создайте класс \texttt{NumberSignChecker}.
    \item Создайте \textbf{статический} метод \texttt{get\_sign}, который принимает число в качестве аргумента и возвращает строку с его знаком (+, - или 0).
    \item Используйте цикл для проверки чисел [-5, -4, -3, -2, -1, 0, 1, 2, 3, 4, 5], вызывая статический метод \texttt{get\_sign} и выводя результат на экран.
\end{enumerate}

\subsection*{Пример использования:}
\begin{lstlisting}[language=Python]
    v = NumberSignChecker.get_sign(-7)
\end{lstlisting}
Вывод:
\begin{verbatim}
-5 -
-4 -
-3 -
-2 -
-1 -
0 0
1 +
2 +
3 +
4 +
5 +
\end{verbatim}

\item
Написать программу, которая создаёт класс \texttt{NumberPalindromeChecker} 
для проверки палиндрома числа. В классе должен быть статический метод
\texttt{is\_palindrome} и возвращать \texttt{True}, если число является палиндромом, 
и \texttt{False} в противном случае. 
Программа также должна использовать цикл для проверки каждого числа от 
100 до 150 и вывода результата на экран.

\subsection*{Инструкции:}
\begin{enumerate}
    \item Создайте класс \texttt{NumberPalindromeChecker}.
    \item Создайте \textbf{статический} метод \texttt{is\_palindrome}, который принимает число в качестве аргумента и проверяет, является ли число палиндромом.
    \item Используйте цикл для проверки каждого числа от 100 до 150 (включительно), вызывая статический метод \texttt{is\_palindrome} и выводя результат на экран.
\end{enumerate}

\subsection*{Пример использования:}
\begin{lstlisting}[language=Python]
    v = NumberPalindromeChecker.is_palindrome(121)
\end{lstlisting}
Вывод (первые и последние строки):
\begin{verbatim}
100 False
101 True
102 False
...
149 False
150 False
\end{verbatim}

\item
Написать программу, которая создаёт класс \texttt{NumberAscendingChecker} 
для проверки, что цифры числа идут в порядке возрастания. В классе должен быть статический метод
\texttt{is\_ascending} и возвращать \texttt{True}, если цифры числа идут в порядке возрастания, 
и \texttt{False} в противном случае. 
Программа также должна использовать цикл для проверки каждого числа от 
10 до 100 и вывода результата на экран.

\subsection*{Инструкции:}
\begin{enumerate}
    \item Создайте класс \texttt{NumberAscendingChecker}.
    \item Создайте \textbf{статический} метод \texttt{is\_ascending}, который принимает число в качестве аргумента и проверяет, идут ли его цифры в порядке возрастания.
    \item Используйте цикл для проверки каждого числа от 10 до 100 (включительно), вызывая статический метод \texttt{is\_ascending} и выводя результат на экран.
\end{enumerate}

\subsection*{Пример использования:}
\begin{lstlisting}[language=Python]
    v = NumberAscendingChecker.is_ascending(123)
\end{lstlisting}
Вывод (первые и последние строки):
\begin{verbatim}
10 False
11 False
12 True
13 True
...
98 False
99 False
100 False
\end{verbatim}

\item
Написать программу, которая создаёт класс \texttt{NumberDescendingChecker} 
для проверки, что цифры числа идут в порядке убывания. В классе должен быть статический метод
\texttt{is\_descending} и возвращать \texttt{True}, если цифры числа идут в порядке убывания, 
и \texttt{False} в противном случае. 
Программа также должна использовать цикл для проверки каждого числа от 
10 до 100 и вывода результата на экран.

\subsection*{Инструкции:}
\begin{enumerate}
    \item Создайте класс \texttt{NumberDescendingChecker}.
    \item Создайте \textbf{статический} метод \texttt{is\_descending}, который принимает число в качестве аргумента и проверяет, идут ли его цифры в порядке убывания.
    \item Используйте цикл для проверки каждого числа от 10 до 100 (включительно), вызывая статический метод \texttt{is\_descending} и выводя результат на экран.
\end{enumerate}

\subsection*{Пример использования:}
\begin{lstlisting}[language=Python]
    v = NumberDescendingChecker.is_descending(321)
\end{lstlisting}
Вывод (первые и последние строки):
\begin{verbatim}
10 False
11 False
12 False
13 False
...
98 True
99 True
100 False
\end{verbatim}

\item
Написать программу, которая создаёт класс \texttt{NumberPrimeDigitChecker} 
для проверки, что все цифры числа простые. В классе должен быть статический метод
\texttt{all\_digits\_prime} и возвращать \texttt{True}, если все цифры числа простые, 
и \texttt{False} в противном случае. 
Программа также должна использовать цикл для проверки каждого числа от 
10 до 100 и вывода результата на экран.

\subsection*{Инструкции:}
\begin{enumerate}
    \item Создайте класс \texttt{NumberPrimeDigitChecker}.
    \item Создайте \textbf{статический} метод \texttt{all\_digits\_prime}, который принимает число в качестве аргумента и проверяет, являются ли все его цифры простыми числами.
    \item Используйте цикл для проверки каждого числа от 10 до 100 (включительно), вызывая статический метод \texttt{all\_digits\_prime} и выводя результат на экран.
\end{enumerate}

\subsection*{Пример использования:}
\begin{lstlisting}[language=Python]
    v = NumberPrimeDigitChecker.all_digits_prime(23)
\end{lstlisting}
Вывод (первые и последние строки):
\begin{verbatim}
10 False
11 False
12 False
13 False
...
98 False
99 False
100 False
\end{verbatim}

\item
Написать программу, которая создаёт класс \texttt{NumberEvenDigitChecker} 
для проверки, что все цифры числа чётные. В классе должен быть статический метод
\texttt{all\_digits\_even} и возвравать \texttt{True}, если все цифры числа чётные, 
и \texttt{False} в противном случае. 
Программа также должна использовать цикл для проверки каждого числа от 
10 до 100 и вывода результата на экран.

\subsection*{Инструкции:}
\begin{enumerate}
    \item Создайте класс \texttt{NumberEvenDigitChecker}.
    \item Создайте \textbf{статический} метод \texttt{all\_digits\_even}, который принимает число в качестве аргумента и проверяет, являются ли все его цифры чётными.
    \item Используйте цикл для проверки каждого числа от 10 до 100 (включительно), вызывая статический метод \texttt{all\_digits\_even} и выводя результат на экран.
\end{enumerate}

\subsection*{Пример использования:}
\begin{lstlisting}[language=Python]
    v = NumberEvenDigitChecker.all_digits_even(24)
\end{lstlisting}
Вывод (первые и последние строки):
\begin{verbatim}
10 False
11 False
12 False
13 False
...
98 False
99 False
100 False
\end{verbatim}

\item
Написать программу, которая создаёт класс \texttt{NumberOddDigitChecker} 
для проверки, что все цифры числа нечётные. В классе должен быть статический метод
\texttt{all\_digits\_odd} и возвращать \texttt{True}, если все цифры числа нечётные, 
и \texttt{False} в противном случае. 
Программа также должна использовать цикл для проверки каждого числа от 
10 до 100 и вывода результата на экран.

\subsection*{Инструкции:}
\begin{enumerate}
    \item Создайте класс \texttt{NumberOddDigitChecker}.
    \item Создайте \textbf{статический} метод \texttt{all\_digits\_odd}, который принимает число в качестве аргумента и проверяет, являются ли все его цифры нечётными.
    \item Используйте цикл для проверки каждого числа от 10 до 100 (включительно), вызывая статический метод \texttt{all\_digits\_odd} и выводя результат на экран.
\end{enumerate}

\subsection*{Пример использования:}
\begin{lstlisting}[language=Python]
    v = NumberOddDigitChecker.all_digits_odd(135)
\end{lstlisting}
Вывод (первые и последние строки):
\begin{verbatim}
10 False
11 True
12 False
13 True
...
98 False
99 True
100 False
\end{verbatim}


\end{enumerate}

\subsubsection{Задача 3}

\begin{enumerate}

\item
Написать программу на Python, которая создает класс \texttt{Person} для представления сотрудника персонала. Класс должен содержать закрытые атрибуты \texttt{\_\_name}, \texttt{\_\_country}, \texttt{\_\_date\_of\_birth} и метод \texttt{calculate\_age}. Доступ к атрибутам только через методы-геттеры. Создать экземпляры и вывести информацию о каждом человеке.

\subsection*{Инструкции:}
\begin{enumerate}
    \item Создайте класс \texttt{Person} с методом \texttt{\_\_init\_\_}, который принимает имя, страну и дату рождения.
    \item Создайте методы-геттеры: \texttt{get\_name()}, \texttt{get\_country()}, \texttt{get\_date\_of\_birth()}.
    \item Создайте метод \texttt{calculate\_age()} для вычисления возраста.
    \item Создайте несколько экземпляров класса \texttt{Person}.
    \item Выведите данные каждого человека через методы класса.
\end{enumerate}

\subsection*{Пример использования:}
\begin{lstlisting}[caption=Пример кода]
from datetime import date

person1 = Person("Иванов Иван Иванович", "Россия", date(1946, 8, 15))
person2 = Person("Петров Сергей Александрович", "Белоруссия", date(1982, 10, 22))

print("Персона 1:")
print("Имя: ", person1.get_name())
print("Страна: ", person1.get_country())
print("Дата рождения: ", person1.get_date_of_birth())
print("Возраст: ", person1.calculate_age())

print("Персона 2:")
print("Имя: ", person2.get_name())
print("Страна: ", person2.get_country())
print("Дата рождения: ", person2.get_date_of_birth())
print("Возраст: ", person2.calculate_age())
\end{lstlisting}

\subsection*{Вывод:}
\begin{lstlisting}[caption=Ожидаемый вывод]
Персона 1:
Имя:  Иванов Иван Иванович
Страна:  Россия
Дата рождения:  1946-08-15
Возраст:  77
Персона 2:
Имя:  Петров Сергей Александрович
Страна:  Белоруссия
Дата рождения:  1982-10-22
Возраст:  41
\end{lstlisting}

% ================= Вариант 2 =================
\item
Создайте класс \texttt{Student} с закрытыми атрибутами \texttt{\_\_full\_name}, \texttt{\_\_enrollment\_date}, \texttt{\_\_major}. Реализуйте методы-геттеры и метод \texttt{study\_duration()} для вычисления количества лет с момента зачисления.

\subsection*{Инструкции:}
\begin{enumerate}
    \item Создайте класс \texttt{Student} с методом \texttt{\_\_init\_\_}.
    \item Методы-геттеры: \texttt{get\_full\_name()}, \texttt{get\_enrollment\_date()}, \texttt{get\_major()}.
    \item Метод \texttt{study\_duration()} вычисляет количество лет с зачисления.
    \item Создайте несколько экземпляров класса.
    \item Выведите данные каждого студента.
\end{enumerate}

\subsection*{Пример использования:}
\begin{lstlisting}[caption=Пример кода]
from datetime import date

student1 = Student("Сидоров Алексей", date(2018, 9, 1), "Математика")
student2 = Student("Иванова Мария", date(2021, 9, 1), "Физика")

print("Студент 1:")
print("Имя: ", student1.get_full_name())
print("Направление: ", student1.get_major())
print("Дата зачисления: ", student1.get_enrollment_date())
print("Стаж учёбы: ", student1.study_duration())

print("Студент 2:")
print("Имя: ", student2.get_full_name())
print("Направление: ", student2.get_major())
print("Дата зачисления: ", student2.get_enrollment_date())
print("Стаж учёбы: ", student2.study_duration())
\end{lstlisting}

\subsection*{Вывод:}
\begin{lstlisting}[caption=Ожидаемый вывод]
Студент 1:
Имя:  Сидоров Алексей
Направление:  Математика
Дата зачисления:  2018-09-01
Стаж учёбы:  5
Студент 2:
Имя:  Иванова Мария
Направление:  Физика
Дата зачисления:  2021-09-01
Стаж учёбы:  2
\end{lstlisting}

\item
Создайте класс \texttt{Employee} с закрытыми атрибутами \texttt{\_\_name}, \texttt{\_\_position}, \texttt{\_\_hire\_date}. Реализуйте методы-геттеры и метод \texttt{work\_experience()} для вычисления количества лет работы.

\subsection*{Инструкции:}
\begin{enumerate}
    \item Создайте класс \texttt{Employee} с методом \texttt{\_\_init\_\_}.
    \item Методы-геттеры: \texttt{get\_name()}, \texttt{get\_position()}, \texttt{get\_hire\_date()}.
    \item Метод \texttt{work\_experience()} вычисляет стаж в годах.
    \item Создайте несколько экземпляров класса.
    \item Выведите данные каждого сотрудника.
\end{enumerate}

\subsection*{Пример использования:}
\begin{lstlisting}[caption=Пример кода]
from datetime import date

emp1 = Employee("Кузнецов Дмитрий", "Инженер", date(2010, 5, 10))
emp2 = Employee("Смирнова Ольга", "Менеджер", date(2015, 8, 1))

print("Сотрудник 1:")
print("Имя: ", emp1.get_name())
print("Должность: ", emp1.get_position())
print("Дата приёма: ", emp1.get_hire_date())
print("Стаж: ", emp1.work_experience())

print("Сотрудник 2:")
print("Имя: ", emp2.get_name())
print("Должность: ", emp2.get_position())
print("Дата приёма: ", emp2.get_hire_date())
print("Стаж: ", emp2.work_experience())
\end{lstlisting}

\subsection*{Вывод:}
\begin{lstlisting}[caption=Ожидаемый вывод]
Сотрудник 1:
Имя:  Кузнецов Дмитрий
Должность:  Инженер
Дата приёма:  2010-05-10
Стаж:  17
Сотрудник 2:
Имя:  Смирнова Ольга
Должность:  Менеджер
Дата приёма:  2015-08-01
Стаж:  8
\end{lstlisting}

% ================= Вариант 4 =================
\item
Создайте класс \texttt{Book} с закрытыми атрибутами \texttt{\_\_title}, \texttt{\_\_author}, \texttt{\_\_publish\_date}. Реализуйте геттеры и метод \texttt{book\_age()} для вычисления возраста книги.

\subsection*{Инструкции:}
\begin{enumerate}
    \item Создайте класс \texttt{Book}.
    \item Методы-геттеры: \texttt{get\_title()}, \texttt{get\_author()}, \texttt{get\_publish\_date()}.
    \item Метод \texttt{book\_age()} вычисляет возраст книги.
    \item Создайте экземпляры класса.
    \item Выведите данные каждой книги.
\end{enumerate}

\subsection*{Пример использования:}
\begin{lstlisting}[caption=Пример кода]
from datetime import date

book1 = Book("Программирование на Python", "Иванов И.И.", date(2015, 3, 10))
book2 = Book("Алгебра", "Петров П.П.", date(2000, 9, 1))

print("Книга 1:")
print("Название: ", book1.get_title())
print("Автор: ", book1.get_author())
print("Дата публикации: ", book1.get_publish_date())
print("Возраст книги: ", book1.book_age())

print("Книга 2:")
print("Название: ", book2.get_title())
print("Автор: ", book2.get_author())
print("Дата публикации: ", book2.get_publish_date())
print("Возраст книги: ", book2.book_age())
\end{lstlisting}

\subsection*{Вывод:}
\begin{lstlisting}[caption=Ожидаемый вывод]
Книга 1:
Название:  Программирование на Python
Автор:  Иванов И.И.
Дата публикации:  2015-03-10
Возраст книги:  8
Книга 2:
Название:  Алгебра
Автор:  Петров П.П.
Дата публикации:  2000-09-01
Возраст книги:  23
\end{lstlisting}

% ================= Вариант 5 =================
\item
Создайте класс \texttt{Car} с закрытыми атрибутами \texttt{\_\_model}, \texttt{\_\_manufacturer}, \texttt{\_\_production\_date}. Геттеры и метод \texttt{car\_age()} для вычисления возраста автомобиля.

\subsection*{Инструкции:}
\begin{enumerate}
    \item Создайте класс \texttt{Car}.
    \item Методы-геттеры: \texttt{get\_model()}, \texttt{get\_manufacturer()}, \texttt{get\_production\_date()}.
    \item Метод \texttt{car\_age()} вычисляет возраст автомобиля.
    \item Создайте экземпляры класса.
    \item Выведите данные каждого автомобиля.
\end{enumerate}

\subsection*{Пример использования:}
\begin{lstlisting}[caption=Пример кода]
from datetime import date

car1 = Car("Camry", "Toyota", date(2012, 6, 15))
car2 = Car("Focus", "Ford", date(2018, 4, 20))

print("Автомобиль 1:")
print("Модель: ", car1.get_model())
print("Производитель: ", car1.get_manufacturer())
print("Дата выпуска: ", car1.get_production_date())
print("Возраст авто: ", car1.car_age())

print("Автомобиль 2:")
print("Модель: ", car2.get_model())
print("Производитель: ", car2.get_manufacturer())
print("Дата выпуска: ", car2.get_production_date())
print("Возраст авто: ", car2.car_age())
\end{lstlisting}

\subsection*{Вывод:}
\begin{lstlisting}[caption=Ожидаемый вывод]
Автомобиль 1:
Модель:  Camry
Производитель:  Toyota
Дата выпуска:  2012-06-15
Возраст авто:  11
Автомобиль 2:
Модель:  Focus
Производитель:  Ford
Дата выпуска:  2018-04-20
Возраст авто:  5
\end{lstlisting}

% ================= Вариант 6 =================
\item
Создайте класс \texttt{Pet} с закрытыми атрибутами \texttt{\_\_name}, \texttt{\_\_species}, \texttt{\_\_birth\_date}. Реализуйте методы-геттеры и метод \texttt{pet\_age()} для вычисления возраста питомца. Создайте несколько экземпляров и выведите их данные.

\subsection*{Инструкции:}
\begin{enumerate}
    \item Создайте класс \texttt{Pet} с методом \texttt{\_\_init\_\_}.
    \item Методы-геттеры: \texttt{get\_name()}, \texttt{get\_species()}, \texttt{get\_birth\_date()}.
    \item Метод \texttt{pet\_age()} вычисляет возраст питомца в годах.
    \item Создайте несколько экземпляров класса.
    \item Выведите данные каждого питомца через методы класса.
\end{enumerate}

\subsection*{Пример использования:}
\begin{lstlisting}[caption=Пример кода]
from datetime import date

pet1 = Pet("Барсик", "Кошка", date(2018, 5, 12))
pet2 = Pet("Рекс", "Собака", date(2015, 8, 1))

print("Питомец 1:")
print("Имя: ", pet1.get_name())
print("Вид: ", pet1.get_species())
print("Дата рождения: ", pet1.get_birth_date())
print("Возраст: ", pet1.pet_age())

print("Питомец 2:")
print("Имя: ", pet2.get_name())
print("Вид: ", pet2.get_species())
print("Дата рождения: ", pet2.get_birth_date())
print("Возраст: ", pet2.pet_age())
\end{lstlisting}

\subsection*{Вывод:}
\begin{lstlisting}[caption=Ожидаемый вывод]
Питомец 1:
Имя:  Барсик
Вид:  Кошка
Дата рождения:  2018-05-12
Возраст:  7
Питомец 2:
Имя:  Рекс
Вид:  Собака
Дата рождения:  2015-08-01
Возраст:  10
\end{lstlisting}

% ================= Вариант 7 =================
\item
Создайте класс \texttt{Membership} с закрытыми атрибутами \texttt{\_\_member\_name}, \texttt{\_\_membership\_type}, \texttt{\_\_join\_date}. Реализуйте методы-геттеры и метод \texttt{membership\_duration()} для вычисления длительности членства в годах.

\subsection*{Инструкции:}
\begin{enumerate}
    \item Создайте класс \texttt{Membership}.
    \item Методы-геттеры: \texttt{get\_member\_name()}, \texttt{get\_membership\_type()}, \texttt{get\_join\_date()}.
    \item Метод \texttt{membership\_duration()} вычисляет длительность членства в годах.
    \item Создайте несколько экземпляров.
    \item Выведите данные каждого участника.
\end{enumerate}

\subsection*{Пример использования:}
\begin{lstlisting}[caption=Пример кода]
from datetime import date

member1 = Membership("Иванов Иван", "Золотой", date(2018, 3, 15))
member2 = Membership("Петров Петр", "Серебряный", date(2020, 6, 1))

print("Член 1:")
print("Имя: ", member1.get_member_name())
print("Тип членства: ", member1.get_membership_type())
print("Дата вступления: ", member1.get_join_date())
print("Длительность членства: ", member1.membership_duration())

print("Член 2:")
print("Имя: ", member2.get_member_name())
print("Тип членства: ", member2.get_membership_type())
print("Дата вступления: ", member2.get_join_date())
print("Длительность членства: ", member2.membership_duration())
\end{lstlisting}

\subsection*{Вывод:}
\begin{lstlisting}[caption=Ожидаемый вывод]
Член 1:
Имя:  Иванов Иван
Тип членства:  Золотой
Дата вступления:  2018-03-15
Длительность членства:  5
Член 2:
Имя:  Петров Петр
Тип членства:  Серебряный
Дата вступления:  2020-06-01
Длительность членства:  3
\end{lstlisting}

% ================= Вариант 8 =================
\item
Создайте класс \texttt{Event} с закрытыми атрибутами \texttt{\_\_event\_name}, \texttt{\_\_location}, \texttt{\_\_event\_date}. Реализуйте методы-геттеры и метод \texttt{days\_until\_event()} для вычисления количества дней до события.

\subsection*{Инструкции:}
\begin{enumerate}
    \item Создайте класс \texttt{Event}.
    \item Методы-геттеры: \texttt{get\_event\_name()}, \texttt{get\_location()}, \texttt{get\_event\_date()}.
    \item Метод \texttt{days\_until\_event()} вычисляет дни до события.
    \item Создайте несколько экземпляров.
    \item Выведите данные каждого события.
\end{enumerate}

\subsection*{Пример использования:}
\begin{lstlisting}[caption=Пример кода]
from datetime import date

event1 = Event("Концерт", "Стадион", date(2025, 12, 1))
event2 = Event("Выставка", "Музей", date(2025, 11, 20))

print("Событие 1:")
print("Название: ", event1.get_event_name())
print("Место: ", event1.get_location())
print("Дата: ", event1.get_event_date())
print("Дней до события: ", event1.days_until_event())

print("Событие 2:")
print("Название: ", event2.get_event_name())
print("Место: ", event2.get_location())
print("Дата: ", event2.get_event_date())
print("Дней до события: ", event2.days_until_event())
\end{lstlisting}

\subsection*{Вывод:}
\begin{lstlisting}[caption=Ожидаемый вывод]
Событие 1:
Название:  Концерт
Место:  Стадион
Дата:  2025-12-01
Дней до события:  112
Событие 2:
Название:  Выставка
Место:  Музей
Дата:  2025-11-20
Дней до события:  101
\end{lstlisting}

% ================= Вариант 9 =================
\item
Создайте класс \texttt{Course} с закрытыми атрибутами \texttt{\_\_course\_name}, \texttt{\_\_start\_date}, \texttt{\_\_duration\_weeks}. Реализуйте методы-геттеры и метод \texttt{weeks\_elapsed()} для вычисления прошедших недель с начала курса.

\subsection*{Инструкции:}
\begin{enumerate}
    \item Создайте класс \texttt{Course}.
    \item Методы-геттеры: \texttt{get\_course\_name()}, \texttt{get\_start\_date()}, \texttt{get\_duration\_weeks()}.
    \item Метод \texttt{weeks\_elapsed()} вычисляет количество прошедших недель.
    \item Создайте несколько экземпляров.
    \item Выведите данные каждого курса.
\end{enumerate}

\subsection*{Пример использования:}
\begin{lstlisting}[caption=Пример кода]
from datetime import date

course1 = Course("Python", date(2025, 1, 1), 12)
course2 = Course("Алгебра", date(2025, 2, 1), 10)

print("Курс 1:")
print("Название: ", course1.get_course_name())
print("Дата начала: ", course1.get_start_date())
print("Продолжительность (недель): ", course1.get_duration_weeks())
print("Прошло недель: ", course1.weeks_elapsed())

print("Курс 2:")
print("Название: ", course2.get_course_name())
print("Дата начала: ", course2.get_start_date())
print("Продолжительность (недель): ", course2.get_duration_weeks())
print("Прошло недель: ", course2.weeks_elapsed())
\end{lstlisting}

\subsection*{Вывод:}
\begin{lstlisting}[caption=Ожидаемый вывод]
Курс 1:
Название:  Python
Дата начала:  2025-01-01
Продолжительность (недель):  12
Прошло недель:  36
Курс 2:
Название:  Алгебра
Дата начала:  2025-02-01
Продолжительность (недель):  10
Прошло недель:  31
\end{lstlisting}

% ================= Вариант 10 =================
\item
Создайте класс \texttt{Subscription} с закрытыми атрибутами \texttt{\_\_user}, \texttt{\_\_plan}, \texttt{\_\_start\_date}. Реализуйте методы-геттеры и метод \texttt{subscription\_age()} для вычисления возраста подписки в годах.

\subsection*{Инструкции:}
\begin{enumerate}
    \item Создайте класс \texttt{Subscription}.
    \item Методы-геттеры: \texttt{get\_user()}, \texttt{get\_plan()}, \texttt{get\_start\_date()}.
    \item Метод \texttt{subscription\_age()} вычисляет возраст подписки.
    \item Создайте несколько экземпляров.
    \item Выведите данные каждой подписки.
\end{enumerate}

\subsection*{Пример использования:}
\begin{lstlisting}[caption=Пример кода]
from datetime import date

sub1 = Subscription("Иванов И.", "Premium", date(2021, 3, 1))
sub2 = Subscription("Петров П.", "Basic", date(2022, 7, 15))

print("Подписка 1:")
print("Пользователь: ", sub1.get_user())
print("План: ", sub1.get_plan())
print("Дата начала: ", sub1.get_start_date())
print("Возраст подписки: ", sub1.subscription_age())

print("Подписка 2:")
print("Пользователь: ", sub2.get_user())
print("План: ", sub2.get_plan())
print("Дата начала: ", sub2.get_start_date())
print("Возраст подписки: ", sub2.subscription_age())
\end{lstlisting}

\subsection*{Вывод:}
\begin{lstlisting}[caption=Ожидаемый вывод]
Подписка 1:
Пользователь:  Иванов И.
План:  Premium
Дата начала:  2021-03-01
Возраст подписки:  4
Подписка 2:
Пользователь:  Петров П.
План:  Basic
Дата начала:  2022-07-15
Возраст подписки:  3
\end{lstlisting}

% ================= Вариант 11 =================
\item
Создайте класс \texttt{Flight} с закрытыми атрибутами \texttt{\_\_flight\_number}, \texttt{\_\_departure\_date}, \texttt{\_\_destination}. Реализуйте методы-геттеры и метод \texttt{days\_until\_departure()} для вычисления количества дней до вылета.

\subsection*{Инструкции:}
\begin{enumerate}
    \item Создайте класс \texttt{Flight}.
    \item Методы-геттеры: \texttt{get\_flight\_number()}, \texttt{get\_departure\_date()}, \texttt{get\_destination()}.
    \item Метод \texttt{days\_until\_departure()} вычисляет количество дней до вылета.
    \item Создайте несколько экземпляров.
    \item Выведите данные каждого рейса.
\end{enumerate}

\subsection*{Пример использования:}
\begin{lstlisting}[caption=Пример кода]
from datetime import date

flight1 = Flight("SU123", date(2025, 10, 15), "Москва")
flight2 = Flight("AF456", date(2025, 11, 1), "Париж")

print("Рейс 1:")
print("Номер: ", flight1.get_flight_number())
print("Дата вылета: ", flight1.get_departure_date())
print("Пункт назначения: ", flight1.get_destination())
print("Дней до вылета: ", flight1.days_until_departure())

print("Рейс 2:")
print("Номер: ", flight2.get_flight_number())
print("Дата вылета: ", flight2.get_departure_date())
print("Пункт назначения: ", flight2.get_destination())
print("Дней до вылета: ", flight2.days_until_departure())
\end{lstlisting}

\subsection*{Вывод:}
\begin{lstlisting}[caption=Ожидаемый вывод]
Рейс 1:
Номер:  SU123
Дата вылета:  2025-10-15
Пункт назначения:  Москва
Дней до вылета:  54
Рейс 2:
Номер:  AF456
Дата вылета:  2025-11-01
Пункт назначения:  Париж
Дней до вылета:  71
\end{lstlisting}

% ================= Вариант 12 =================
\item
Создайте класс \texttt{Project} с закрытыми атрибутами \texttt{\_\_project\_name}, \texttt{\_\_start\_date}, \texttt{\_\_deadline}. Реализуйте методы-геттеры и метод \texttt{days\_remaining()} для вычисления количества дней до завершения проекта.

\subsection*{Инструкции:}
\begin{enumerate}
    \item Создайте класс \texttt{Project}.
    \item Методы-геттеры: \texttt{get\_project\_name()}, \texttt{get\_start\_date()}, \texttt{get\_deadline()}.
    \item Метод \texttt{days\_remaining()} вычисляет дни до дедлайна.
    \item Создайте несколько экземпляров.
    \item Выведите данные каждого проекта.
\end{enumerate}

\subsection*{Пример использования:}
\begin{lstlisting}[caption=Пример кода]
from datetime import date

project1 = Project("Разработка сайта", date(2025, 9, 1), date(2025, 12, 1))
project2 = Project("Анализ данных", date(2025, 10, 1), date(2025, 11, 30))

print("Проект 1:")
print("Название: ", project1.get_project_name())
print("Дата начала: ", project1.get_start_date())
print("Дедлайн: ", project1.get_deadline())
print("Дней до завершения: ", project1.days_remaining())

print("Проект 2:")
print("Название: ", project2.get_project_name())
print("Дата начала: ", project2.get_start_date())
print("Дедлайн: ", project2.get_deadline())
print("Дней до завершения: ", project2.days_remaining())
\end{lstlisting}

\subsection*{Вывод:}
\begin{lstlisting}[caption=Ожидаемый вывод]
Проект 1:
Название:  Разработка сайта
Дата начала:  2025-09-01
Дедлайн:  2025-12-01
Дней до завершения:  101
Проект 2:
Название:  Анализ данных
Дата начала:  2025-10-01
Дедлайн:  2025-11-30
Дней до завершения:  91
\end{lstlisting}

% ================= Вариант 13 =================
\item
Создайте класс \texttt{Doctor} с закрытыми атрибутами \texttt{\_\_full\_name}, \texttt{\_\_specialty}, \texttt{\_\_birth\_date}. Реализуйте методы-геттеры и метод \texttt{calculate\_age()} для вычисления возраста врача.

\subsection*{Инструкции:}
\begin{enumerate}
    \item Создайте класс \texttt{Doctor}.
    \item Методы-геттеры: \texttt{get\_full\_name()}, \texttt{get\_specialty()}, \texttt{get\_birth\_date()}.
    \item Метод \texttt{calculate\_age()} вычисляет возраст.
    \item Создайте несколько экземпляров.
    \item Выведите данные каждого врача.
\end{enumerate}

\subsection*{Пример использования:}
\begin{lstlisting}[caption=Пример кода]
from datetime import date

doc1 = Doctor("Иванов И.И.", "Терапевт", date(1980, 5, 12))
doc2 = Doctor("Петров П.П.", "Хирург", date(1975, 8, 1))

print("Врач 1:")
print("Имя: ", doc1.get_full_name())
print("Специальность: ", doc1.get_specialty())
print("Дата рождения: ", doc1.get_birth_date())
print("Возраст: ", doc1.calculate_age())

print("Врач 2:")
print("Имя: ", doc2.get_full_name())
print("Специальность: ", doc2.get_specialty())
print("Дата рождения: ", doc2.get_birth_date())
print("Возраст: ", doc2.calculate_age())
\end{lstlisting}

\subsection*{Вывод:}
\begin{lstlisting}[caption=Ожидаемый вывод]
Врач 1:
Имя:  Иванов И.И.
Специальность:  Терапевт
Дата рождения:  1980-05-12
Возраст:  45
Врач 2:
Имя:  Петров П.П.
Специальность:  Хирург
Дата рождения:  1975-08-01
Возраст:  50
\end{lstlisting}

% ================= Вариант 14 =================
\item
Создайте класс \texttt{Patient} с закрытыми атрибутами \texttt{\_\_full\_name}, \texttt{\_\_admission\_date}, \texttt{\_\_diagnosis}. Реализуйте методы-геттеры и метод \texttt{hospital\_stay()} для вычисления количества дней пребывания в больнице.

\subsection*{Инструкции:}
\begin{enumerate}
    \item Создайте класс \texttt{Patient}.
    \item Методы-геттеры: \texttt{get\_full\_name()}, \texttt{get\_admission\_date()}, \texttt{get\_diagnosis()}.
    \item Метод \texttt{hospital\_stay()} вычисляет дни пребывания.
    \item Создайте несколько экземпляров.
    \item Выведите данные каждого пациента.
\end{enumerate}

\subsection*{Пример использования:}
\begin{lstlisting}[caption=Пример кода]
from datetime import date

patient1 = Patient("Сидоров С.С.", date(2025, 9, 1), "ОРВИ")
patient2 = Patient("Кузнецов К.К.", date(2025, 8, 28), "Грипп")

print("Пациент 1:")
print("Имя: ", patient1.get_full_name())
print("Дата госпитализации: ", patient1.get_admission_date())
print("Диагноз: ", patient1.get_diagnosis())
print("Дней в больнице: ", patient1.hospital_stay())

print("Пациент 2:")
print("Имя: ", patient2.get_full_name())
print("Дата госпитализации: ", patient2.get_admission_date())
print("Диагноз: ", patient2.get_diagnosis())
print("Дней в больнице: ", patient2.hospital_stay())
\end{lstlisting}

\subsection*{Вывод:}
\begin{lstlisting}[caption=Ожидаемый вывод]
Пациент 1:
Имя:  Сидоров С.С.
Дата госпитализации:  2025-09-01
Диагноз:  ОРВИ
Дней в больнице:  15
Пациент 2:
Имя:  Кузнецов К.К.
Дата госпитализации:  2025-08-28
Диагноз:  Грипп
Дней в больнице:  19
\end{lstlisting}

% ================= Вариант 15 =================
\item
Создайте класс \texttt{Concert} с закрытыми атрибутами \texttt{\_\_artist}, \texttt{\_\_venue}, \texttt{\_\_concert\_date}. Реализуйте методы-геттеры и метод \texttt{days\_until\_concert()}.

\subsection*{Инструкции:}
\begin{enumerate}
    \item Создайте класс \texttt{Concert}.
    \item Методы-геттеры: \texttt{get\_artist()}, \texttt{get\_venue()}, \texttt{get\_concert\_date()}.
    \item Метод \texttt{days\_until\_concert()} вычисляет дни до концерта.
    \item Создайте несколько экземпляров.
    \item Выведите данные каждого концерта.
\end{enumerate}

\subsection*{Пример использования:}
\begin{lstlisting}[caption=Пример кода]
from datetime import date

concert1 = Concert("Imagine Dragons", "Лужники", date(2025, 10, 10))
concert2 = Concert("Coldplay", "O2 Arena", date(2025, 11, 5))

print("Концерт 1:")
print("Исполнитель: ", concert1.get_artist())
print("Место: ", concert1.get_venue())
print("Дата: ", concert1.get_concert_date())
print("Дней до концерта: ", concert1.days_until_concert())

print("Концерт 2:")
print("Исполнитель: ", concert2.get_artist())
print("Место: ", concert2.get_venue())
print("Дата: ", concert2.get_concert_date())
print("Дней до концерта: ", concert2.days_until_concert())
\end{lstlisting}

\subsection*{Вывод:}
\begin{lstlisting}[caption=Ожидаемый вывод]
Концерт 1:
Исполнитель:  Imagine Dragons
Место:  Лужники
Дата:  2025-10-10
Дней до концерта:  49
Концерт 2:
Исполнитель:  Coldplay
Место:  O2 Arena
Дата:  2025-11-05
Дней до концерта:  75
\end{lstlisting}

% ================= Вариант 16 =================
\item
Создайте класс \texttt{Holiday} с закрытыми атрибутами \texttt{\_\_name}, \texttt{\_\_country}, \texttt{\_\_holiday\_date}. Реализуйте методы-геттеры и метод \texttt{days\_until\_holiday()}.

\subsection*{Инструкции:}
\begin{enumerate}
    \item Создайте класс \texttt{Holiday}.
    \item Методы-геттеры: \texttt{get\_name()}, \texttt{get\_country()}, \texttt{get\_holiday\_date()}.
    \item Метод \texttt{days\_until\_holiday()} вычисляет дни до праздника.
    \item Создайте несколько экземпляров.
    \item Выведите данные каждого праздника.
\end{enumerate}

\subsection*{Пример использования:}
\begin{lstlisting}[caption=Пример кода]
from datetime import date

holiday1 = Holiday("Новый Год", "Россия", date(2026, 1, 1))
holiday2 = Holiday("Рождество", "Германия", date(2025, 12, 25))

print("Праздник 1:")
print("Название: ", holiday1.get_name())
print("Страна: ", holiday1.get_country())
print("Дата: ", holiday1.get_holiday_date())
print("Дней до праздника: ", holiday1.days_until_holiday())

print("Праздник 2:")
print("Название: ", holiday2.get_name())
print("Страна: ", holiday2.get_country())
print("Дата: ", holiday2.get_holiday_date())
print("Дней до праздника: ", holiday2.days_until_holiday())
\end{lstlisting}

\subsection*{Вывод:}
\begin{lstlisting}[caption=Ожидаемый вывод]
Праздник 1:
Название:  Новый Год
Страна:  Россия
Дата:  2026-01-01
Дней до праздника:  83
Праздник 2:
Название:  Рождество
Страна:  Германия
Дата:  2025-12-25
Дней до праздника:  67
\end{lstlisting}
% ================= Вариант 17 =================
\item
Создайте класс \texttt{Employee} с закрытыми атрибутами \texttt{\_\_full\_name}, \texttt{\_\_position}, \texttt{\_\_hire\_date}. Реализуйте методы-геттеры и метод \texttt{years\_worked()} для вычисления стажа работы в годах.

\subsection*{Инструкции:}
\begin{enumerate}
    \item Создайте класс \texttt{Employee}.
    \item Методы-геттеры: \texttt{get\_full\_name()}, \texttt{get\_position()}, \texttt{get\_hire\_date()}.
    \item Метод \texttt{years\_worked()} вычисляет стаж в годах.
    \item Создайте несколько экземпляров.
    \item Выведите данные каждого сотрудника.
\end{enumerate}

\subsection*{Пример использования:}
\begin{lstlisting}[caption=Пример кода]
from datetime import date

emp1 = Employee("Иванов И.И.", "Менеджер", date(2015, 4, 1))
emp2 = Employee("Петров П.П.", "Разработчик", date(2018, 7, 15))

print("Сотрудник 1:")
print("Имя: ", emp1.get_full_name())
print("Должность: ", emp1.get_position())
print("Дата приема: ", emp1.get_hire_date())
print("Стаж: ", emp1.years_worked())

print("Сотрудник 2:")
print("Имя: ", emp2.get_full_name())
print("Должность: ", emp2.get_position())
print("Дата приема: ", emp2.get_hire_date())
print("Стаж: ", emp2.years_worked())
\end{lstlisting}

\subsection*{Вывод:}
\begin{lstlisting}[caption=Ожидаемый вывод]
Сотрудник 1:
Имя:  Иванов И.И.
Должность:  Менеджер
Дата приема:  2015-04-01
Стаж:  10
Сотрудник 2:
Имя:  Петров П.П.
Должность:  Разработчик
Дата приема:  2018-07-15
Стаж:  7
\end{lstlisting}

% ================= Вариант 18 =================
\item
Создайте класс \texttt{LibraryBook} с закрытыми атрибутами \texttt{\_\_title}, \texttt{\_\_author}, \texttt{\_\_publication\_date}. Реализуйте методы-геттеры и метод \texttt{book\_age()} для вычисления возраста книги.

\subsection*{Инструкции:}
\begin{enumerate}
    \item Создайте класс \texttt{LibraryBook}.
    \item Методы-геттеры: \texttt{get\_title()}, \texttt{get\_author()}, \texttt{get\_publication\_date()}.
    \item Метод \texttt{book\_age()} вычисляет возраст книги в годах.
    \item Создайте несколько экземпляров.
    \item Выведите данные каждой книги.
\end{enumerate}

\subsection*{Пример использования:}
\begin{lstlisting}[caption=Пример кода]
from datetime import date

book1 = LibraryBook("Война и мир", "Толстой", date(1869, 1, 1))
book2 = LibraryBook("Мастер и Маргарита", "Булгаков", date(1967, 5, 1))

print("Книга 1:")
print("Название: ", book1.get_title())
print("Автор: ", book1.get_author())
print("Дата публикации: ", book1.get_publication_date())
print("Возраст книги: ", book1.book_age())

print("Книга 2:")
print("Название: ", book2.get_title())
print("Автор: ", book2.get_author())
print("Дата публикации: ", book2.get_publication_date())
print("Возраст книги: ", book2.book_age())
\end{lstlisting}

\subsection*{Вывод:}
\begin{lstlisting}[caption=Ожидаемый вывод]
Книга 1:
Название:  Война и мир
Автор:  Толстой
Дата публикации:  1869-01-01
Возраст книги:  156
Книга 2:
Название:  Мастер и Маргарита
Автор:  Булгаков
Дата публикации:  1967-05-01
Возраст книги:  59
\end{lstlisting}

% ================= Вариант 19 =================
\item
Создайте класс \texttt{Vehicle} с закрытыми атрибутами \texttt{\_\_brand}, \texttt{\_\_model}, \texttt{\_\_manufacture\_date}. Реализуйте методы-геттеры и метод \texttt{vehicle\_age()}.

\subsection*{Инструкции:}
\begin{enumerate}
    \item Создайте класс \texttt{Vehicle}.
    \item Методы-геттеры: \texttt{get\_brand()}, \texttt{get\_model()}, \texttt{get\_manufacture\_date()}.
    \item Метод \texttt{vehicle\_age()} вычисляет возраст транспортного средства.
    \item Создайте несколько экземпляров.
    \item Выведите данные каждого транспортного средства.
\end{enumerate}

\subsection*{Пример использования:}
\begin{lstlisting}[caption=Пример кода]
from datetime import date

vehicle1 = Vehicle("Toyota", "Camry", date(2015, 5, 1))
vehicle2 = Vehicle("BMW", "X5", date(2018, 3, 10))

print("Транспорт 1:")
print("Марка: ", vehicle1.get_brand())
print("Модель: ", vehicle1.get_model())
print("Дата производства: ", vehicle1.get_manufacture_date())
print("Возраст: ", vehicle1.vehicle_age())

print("Транспорт 2:")
print("Марка: ", vehicle2.get_brand())
print("Модель: ", vehicle2.get_model())
print("Дата производства: ", vehicle2.get_manufacture_date())
print("Возраст: ", vehicle2.vehicle_age())
\end{lstlisting}

\subsection*{Вывод:}
\begin{lstlisting}[caption=Ожидаемый вывод]
Транспорт 1:
Марка:  Toyota
Модель:  Camry
Дата производства:  2015-05-01
Возраст:  10
Транспорт 2:
Марка:  BMW
Модель:  X5
Дата производства:  2018-03-10
Возраст:  7
\end{lstlisting}

% ================= Вариант 20 =================
\item
Создайте класс \texttt{Student} с закрытыми атрибутами \texttt{\_\_full\_name}, \texttt{\_\_enrollment\_date}, \texttt{\_\_major}. Реализуйте методы-геттеры и метод \texttt{study\_years()}.

\subsection*{Инструкции:}
\begin{enumerate}
    \item Создайте класс \texttt{Student}.
    \item Методы-геттеры: \texttt{get\_full\_name()}, \texttt{get\_enrollment\_date()}, \texttt{get\_major()}.
    \item Метод \texttt{study\_years()} вычисляет количество лет учебы.
    \item Создайте несколько экземпляров.
    \item Выведите данные каждого студента.
\end{enumerate}

\subsection*{Пример использования:}
\begin{lstlisting}[caption=Пример кода]
from datetime import date

student1 = Student("Иванов И.И.", date(2020, 9, 1), "Математика")
student2 = Student("Петров П.П.", date(2021, 9, 1), "Физика")

print("Студент 1:")
print("Имя: ", student1.get_full_name())
print("Дата зачисления: ", student1.get_enrollment_date())
print("Специальность: ", student1.get_major())
print("Лет учебы: ", student1.study_years())

print("Студент 2:")
print("Имя: ", student2.get_full_name())
print("Дата зачисления: ", student2.get_enrollment_date())
print("Специальность: ", student2.get_major())
print("Лет учебы: ", student2.study_years())
\end{lstlisting}

\subsection*{Вывод:}
\begin{lstlisting}[caption=Ожидаемый вывод]
Студент 1:
Имя:  Иванов И.И.
Дата зачисления:  2020-09-01
Специальность:  Математика
Лет учебы:  5
Студент 2:
Имя:  Петров П.П.
Дата зачисления:  2021-09-01
Специальность:  Физика
Лет учебы:  4
\end{lstlisting}

% ================= Вариант 21 =================
\item
Создайте класс \texttt{Ticket} с закрытыми атрибутами \texttt{\_\_ticket\_number}, \texttt{\_\_issue\_date}, \texttt{\_\_valid\_until}. Реализуйте методы-геттеры и метод \texttt{days\_valid()}.

\subsection*{Инструкции:}
\begin{enumerate}
    \item Создайте класс \texttt{Ticket}.
    \item Методы-геттеры: \texttt{get\_ticket\_number()}, \texttt{get\_issue\_date()}, \texttt{get\_valid\_until()}.
    \item Метод \texttt{days\_valid()} вычисляет дни до окончания действия билета.
    \item Создайте несколько экземпляров.
    \item Выведите данные каждого билета.
\end{enumerate}

\subsection*{Пример использования:}
\begin{lstlisting}[caption=Пример кода]
from datetime import date

ticket1 = Ticket("A123", date(2025, 9, 1), date(2025, 12, 1))
ticket2 = Ticket("B456", date(2025, 10, 1), date(2026, 1, 1))

print("Билет 1:")
print("Номер: ", ticket1.get_ticket_number())
print("Дата выдачи: ", ticket1.get_issue_date())
print("Действителен до: ", ticket1.get_valid_until())
print("Дней до окончания: ", ticket1.days_valid())

print("Билет 2:")
print("Номер: ", ticket2.get_ticket_number())
print("Дата выдачи: ", ticket2.get_issue_date())
print("Действителен до: ", ticket2.get_valid_until())
print("Дней до окончания: ", ticket2.days_valid())
\end{lstlisting}

\subsection*{Вывод:}
\begin{lstlisting}[caption=Ожидаемый вывод]
Билет 1:
Номер:  A123
Дата выдачи:  2025-09-01
Действителен до:  2025-12-01
Дней до окончания:  91
Билет 2:
Номер:  B456
Дата выдачи:  2025-10-01
Действителен до:  2026-01-01
Дней до окончания:  92
\end{lstlisting}

% ================= Вариант 22 =================
\item
Создайте класс \texttt{Appointment} с закрытыми атрибутами \texttt{\_\_client}, \texttt{\_\_service}, \texttt{\_\_appointment\_date}. Реализуйте методы-геттеры и метод \texttt{days\_until\_appointment()}.
\subsection*{Инструкции:}
\begin{enumerate}
    \item Создайте класс \texttt{Appointment}.
    \item Методы-геттеры: \texttt{get\_client()}, \texttt{get\_service()}, \texttt{get\_appointment\_date()}.
    \item Метод \texttt{days\_until\_appointment()} вычисляет дни до приёма.
    \item Создайте несколько экземпляров.
    \item Выведите данные каждого приёма.
\end{enumerate}

\subsection*{Пример использования:}
\begin{lstlisting}[caption=Пример кода]
from datetime import date

app1 = Appointment("Иванов И.", "Массаж", date(2025, 10, 5))
app2 = Appointment("Петров П.", "Стрижка", date(2025, 10, 15))

print("Приём 1:")
print("Клиент: ", app1.get_client())
print("Услуга: ", app1.get_service())
print("Дата: ", app1.get_appointment_date())
print("Дней до приёма: ", app1.days_until_appointment())

print("Приём 2:")
print("Клиент: ", app2.get_client())
print("Услуга: ", app2.get_service())
print("Дата: ", app2.get_appointment_date())
print("Дней до приёма: ", app2.days_until_appointment())
\end{lstlisting}

\subsection*{Вывод:}
\begin{lstlisting}[caption=Ожидаемый вывод]
Приём 1:
Клиент:  Иванов И.
Услуга:  Массаж
Дата:  2025-10-05
Дней до приёма:  44
Приём 2:
Клиент:  Петров П.
Услуга:  Стрижка
Дата:  2025-10-15
Дней до приёма:  54
\end{lstlisting}
% ================= Вариант 23 =================
\item
Создайте класс \texttt{Subscription} с закрытыми атрибутами \texttt{\_\_subscriber}, \texttt{\_\_start\_date}, \texttt{\_\_end\_date}. Реализуйте методы-геттеры и метод \texttt{days\_remaining()}.

\subsection*{Инструкции:}
\begin{enumerate}
    \item Создайте класс \texttt{Subscription}.
    \item Методы-геттеры: \texttt{get\_subscriber()}, \texttt{get\_start\_date()}, \texttt{get\_end\_date()}.
    \item Метод \texttt{days\_remaining()} вычисляет дни до окончания подписки.
    \item Создайте несколько экземпляров.
    \item Выведите данные каждой подписки.
\end{enumerate}

\subsection*{Пример использования:}
\begin{lstlisting}[caption=Пример кода]
from datetime import date

sub1 = Subscription("Иванов И.", date(2025, 1, 1), date(2025, 12, 31))
sub2 = Subscription("Петров П.", date(2025, 6, 1), date(2026, 5, 31))

print("Подписка 1:")
print("Абонент: ", sub1.get_subscriber())
print("Дата начала: ", sub1.get_start_date())
print("Дата окончания: ", sub1.get_end_date())
print("Дней до окончания: ", sub1.days_remaining())

print("Подписка 2:")
print("Абонент: ", sub2.get_subscriber())
print("Дата начала: ", sub2.get_start_date())
print("Дата окончания: ", sub2.get_end_date())
print("Дней до окончания: ", sub2.days_remaining())
\end{lstlisting}

\subsection*{Вывод:}
\begin{lstlisting}[caption=Ожидаемый вывод]
Подписка 1:
Абонент:  Иванов И.
Дата начала:  2025-01-01
Дата окончания:  2025-12-31
Дней до окончания:  113
Подписка 2:
Абонент:  Петров П.
Дата начала:  2025-06-01
Дата окончания:  2026-05-31
Дней до окончания:  245
\end{lstlisting}

% ================= Вариант 24 =================
\item
Создайте класс \texttt{MembershipCard} с закрытыми атрибутами \texttt{\_\_owner}, \texttt{\_\_issue\_date}, \texttt{\_\_expiry\_date}. Реализуйте методы-геттеры и метод \texttt{days\_until\_expiry()}.

\subsection*{Инструкции:}
\begin{enumerate}
    \item Создайте класс \texttt{MembershipCard}.
    \item Методы-геттеры: \texttt{get\_owner()}, \texttt{get\_issue\_date()}, \texttt{get\_expiry\_date()}.
    \item Метод \texttt{days\_until\_expiry()} вычисляет дни до истечения действия карты.
    \item Создайте несколько экземпляров.
    \item Выведите данные каждой карты.
\end{enumerate}

\subsection*{Пример использования:}
\begin{lstlisting}[caption=Пример кода]
from datetime import date

card1 = MembershipCard("Иванов И.", date(2025, 1, 1), date(2026, 1, 1))
card2 = MembershipCard("Петров П.", date(2025, 5, 1), date(2026, 5, 1))

print("Карта 1:")
print("Владелец: ", card1.get_owner())
print("Дата выдачи: ", card1.get_issue_date())
print("Срок действия: ", card1.get_expiry_date())
print("Дней до окончания: ", card1.days_until_expiry())

print("Карта 2:")
print("Владелец: ", card2.get_owner())
print("Дата выдачи: ", card2.get_issue_date())
print("Срок действия: ", card2.get_expiry_date())
print("Дней до окончания: ", card2.days_until_expiry())
\end{lstlisting}

\subsection*{Вывод:}
\begin{lstlisting}[caption=Ожидаемый вывод]
Карта 1:
Владелец:  Иванов И.
Дата выдачи:  2025-01-01
Срок действия:  2026-01-01
Дней до окончания:  113
Карта 2:
Владелец:  Петров П.
Дата выдачи:  2025-05-01
Срок действия:  2026-05-01
Дней до окончания:  204
\end{lstlisting}

% ================= Вариант 25 =================
\item
Создайте класс \texttt{Event} с закрытыми атрибутами \texttt{\_\_title}, \texttt{\_\_location}, \texttt{\_\_event\_date}. Реализуйте методы-геттеры и метод \texttt{days\_until\_event()}.

\subsection*{Инструкции:}
\begin{enumerate}
    \item Создайте класс \texttt{Event}.
    \item Методы-геттеры: \texttt{get\_title()}, \texttt{get\_location()}, \texttt{get\_event\_date()}.
    \item Метод \texttt{days\_until\_event()} вычисляет дни до события.
    \item Создайте несколько экземпляров.
    \item Выведите данные каждого события.
\end{enumerate}

\subsection*{Пример использования:}
\begin{lstlisting}[caption=Пример кода]
from datetime import date

event1 = Event("Фестиваль науки", "Москва", date(2025, 10, 20))
event2 = Event("Конференция IT", "Санкт-Петербург", date(2025, 11, 10))

print("Событие 1:")
print("Название: ", event1.get_title())
print("Место: ", event1.get_location())
print("Дата: ", event1.get_event_date())
print("Дней до события: ", event1.days_until_event())

print("Событие 2:")
print("Название: ", event2.get_title())
print("Место: ", event2.get_location())
print("Дата: ", event2.get_event_date())
print("Дней до события: ", event2.days_until_event())
\end{lstlisting}

\subsection*{Вывод:}
\begin{lstlisting}[caption=Ожидаемый вывод]
Событие 1:
Название:  Фестиваль науки
Место:  Москва
Дата:  2025-10-20
Дней до события:  59
Событие 2:
Название:  Конференция IT
Место:  Санкт-Петербург
Дата:  2025-11-10
Дней до события:  80
\end{lstlisting}

% ================= Вариант 26 =================
\item
Создайте класс \texttt{CarRental} с закрытыми атрибутами \texttt{\_\_client}, \texttt{\_\_rental\_date}, \texttt{\_\_return\_date}. Реализуйте методы-геттеры и метод \texttt{rental\_duration()}.

\subsection*{Инструкции:}
\begin{enumerate}
    \item Создайте класс \texttt{CarRental}.
    \item Методы-геттеры: \texttt{get\_client()}, \texttt{get\_rental\_date()}, \texttt{get\_return\_date()}.
    \item Метод \texttt{rental\_duration()} вычисляет длительность аренды в днях.
    \item Создайте несколько экземпляров.
    \item Выведите данные каждой аренды.
\end{enumerate}

\subsection*{Пример использования:}
\begin{lstlisting}[caption=Пример кода]
from datetime import date

rental1 = CarRental("Иванов И.", date(2025, 10, 1), date(2025, 10, 10))
rental2 = CarRental("Петров П.", date(2025, 11, 1), date(2025, 11, 5))

print("Аренда 1:")
print("Клиент: ", rental1.get_client())
print("Дата аренды: ", rental1.get_rental_date())
print("Дата возврата: ", rental1.get_return_date())
print("Длительность аренды: ", rental1.rental_duration())

print("Аренда 2:")
print("Клиент: ", rental2.get_client())
print("Дата аренды: ", rental2.get_rental_date())
print("Дата возврата: ", rental2.get_return_date())
print("Длительность аренды: ", rental2.rental_duration())
\end{lstlisting}

\subsection*{Вывод:}
\begin{lstlisting}[caption=Ожидаемый вывод]
Аренда 1:
Клиент:  Иванов И.
Дата аренды:  2025-10-01
Дата возврата:  2025-10-10
Длительность аренды:  9
Аренда 2:
Клиент:  Петров П.
Дата аренды:  2025-11-01
Дата возврата:  2025-11-05
Длительность аренды:  4
\end{lstlisting}

% ================= Вариант 27 =================
\item
Создайте класс \texttt{Visa} с закрытыми атрибутами \texttt{\_\_holder}, \texttt{\_\_issue\_date}, \texttt{\_\_expiry\_date}. Реализуйте методы-геттеры и метод \texttt{days\_until\_expiry()}.

\subsection*{Инструкции:}
\begin{enumerate}
    \item Создайте класс \texttt{Visa}.
    \item Методы-геттеры: \texttt{get\_holder()}, \texttt{get\_issue\_date()}, \texttt{get\_expiry\_date()}.
    \item Метод \texttt{days\_until\_expiry()} вычисляет дни до окончания визы.
    \item Создайте несколько экземпляров.
    \item Выведите данные каждой визы.
\end{enumerate}

\subsection*{Пример использования:}
\begin{lstlisting}[caption=Пример кода]
from datetime import date

visa1 = Visa("Иванов И.", date(2025, 1, 1), date(2026, 1, 1))
visa2 = Visa("Петров П.", date(2025, 6, 1), date(2026, 6, 1))

print("Виза 1:")
print("Держатель: ", visa1.get_holder())
print("Дата выдачи: ", visa1.get_issue_date())
print("Дата окончания: ", visa1.get_expiry_date())
print("Дней до окончания: ", visa1.days_until_expiry())

print("Виза 2:")
print("Держатель: ", visa2.get_holder())
print("Дата выдачи: ", visa2.get_issue_date())
print("Дата окончания: ", visa2.get_expiry_date())
print("Дней до окончания: ", visa2.days_until_expiry())
\end{lstlisting}

\subsection*{Вывод:}
\begin{lstlisting}[caption=Ожидаемый вывод]
Виза 1:
Держатель:  Иванов И.
Дата выдачи:  2025-01-01
Дата окончания:  2026-01-01
Дней до окончания:  113
Виза 2:
Держатель:  Петров П.
Дата выдачи:  2025-06-01
Дата окончания:  2026-06-01
Дней до окончания:  204
\end{lstlisting}

% ================= Вариант 28 =================
\item
Создайте класс \texttt{Reservation} с закрытыми атрибутами \texttt{\_\_guest}, \texttt{\_\_checkin\_date}, \texttt{\_\_checkout\_date}. Реализуйте методы-геттеры и метод \texttt{stay\_duration()}.

\subsection*{Инструкции:}
\begin{enumerate}
    \item Создайте класс \texttt{Reservation}.
    \item Методы-геттеры: \texttt{get\_guest()}, \texttt{get\_checkin\_date()}, \texttt{get\_checkout\_date()}.
    \item Метод \texttt{stay\_duration()} вычисляет продолжительность пребывания в днях.
    \item Создайте несколько экземпляров.
    \item Выведите данные каждой брони.
\end{enumerate}

\subsection*{Пример использования:}
\begin{lstlisting}[caption=Пример кода]
from datetime import date

res1 = Reservation("Иванов И.", date(2025, 10, 1), date(2025, 10, 7))
res2 = Reservation("Петров П.", date(2025, 11, 5), date(2025, 11, 12))

print("Бронь 1:")
print("Гость: ", res1.get_guest())
print("Дата заезда: ", res1.get_checkin_date())
print("Дата выезда: ", res1.get_checkout_date())
print("Продолжительность пребывания: ", res1.stay_duration())

print("Бронь 2:")
print("Гость: ", res2.get_guest())
print("Дата заезда: ", res2.get_checkin_date())
print("Дата выезда: ", res2.get_checkout_date())
print("Продолжительность пребывания: ", res2.stay_duration())
\end{lstlisting}

\subsection*{Вывод:}
\begin{lstlisting}[caption=Ожидаемый вывод]
Бронь 1:
Гость:  Иванов И.
Дата заезда:  2025-10-01
Дата выезда:  2025-10-07
Продолжительность пребывания:  6
Бронь 2:
Гость:  Петров П.
Дата заезда:  2025-11-05
Дата выезда:  2025-11-12
Продолжительность пребывания:  7
\end{lstlisting}

% ================= Вариант 29 =================
\item
Создайте класс \texttt{Conference} с закрытыми атрибутами \texttt{\_\_name}, \texttt{\_\_city}, \texttt{\_\_start\_date}. Реализуйте методы-геттеры и метод \texttt{days\_until\_start()}.

\subsection*{Инструкции:}
\begin{enumerate}
    \item Создайте класс \texttt{Conference}.
    \item Методы-геттеры: \texttt{get\_name()}, \texttt{get\_city()}, \texttt{get\_start\_date()}.
    \item Метод \texttt{days\_until\_start()} вычисляет дни до начала конференции.
    \item Создайте несколько экземпляров.
    \item Выведите данные каждой конференции.
\end{enumerate}

\subsection*{Пример использования:}
\begin{lstlisting}[caption=Пример кода]
from datetime import date

conf1 = Conference("PythonConf", "Москва", date(2025, 10, 20))
conf2 = Conference("DataScience Summit", "Санкт-Петербург", date(2025, 11, 15))

print("Конференция 1:")
print("Название: ", conf1.get_name())
print("Город: ", conf1.get_city())
print("Дата начала: ", conf1.get_start_date())
print("Дней до начала: ", conf1.days_until_start())

print("Конференция 2:")
print("Название: ", conf2.get_name())
print("Город: ", conf2.get_city())
print("Дата начала: ", conf2.get_start_date())
print("Дней до начала: ", conf2.days_until_start())
\end{lstlisting}

\subsection*{Вывод:}
\begin{lstlisting}[caption=Ожидаемый вывод]
Конференция 1:
Название:  PythonConf
Город:  Москва
Дата начала:  2025-10-20
Дней до начала:  59
Конференция 2:
Название:  DataScience Summit
Город:  Санкт-Петербург
Дата начала:  2025-11-15
Дней до начала:  85
\end{lstlisting}

% ================= Вариант 30 =================
\item
Создайте класс \texttt{Medication} с закрытыми атрибутами \texttt{\_\_name}, \texttt{\_\_manufacturer}, \texttt{\_\_expiry\_date}. Реализуйте методы-геттеры и метод \texttt{days\_until\_expiry()}.

\subsection*{Инструкции:}
\begin{enumerate}
    \item Создайте класс \texttt{Medication}.
    \item Методы-геттеры: \texttt{get\_name()}, \texttt{get\_manufacturer()}, \texttt{get\_expiry\_date()}.
    \item Метод \texttt{days\_until\_expiry()} вычисляет дни до окончания срока годности.
    \item Создайте несколько экземпляров.
    \item Выведите данные каждого лекарства.
\end{enumerate}

\subsection*{Пример использования:}
\begin{lstlisting}[caption=Пример кода]
from datetime import date

med1 = Medication("Парацетамол", "Фармком", date(2026, 1, 1))
med2 = Medication("Ибупрофен", "БиоФарм", date(2025, 12, 1))

print("Лекарство 1:")
print("Название: ", med1.get_name())
print("Производитель: ", med1.get_manufacturer())
print("Срок годности: ", med1.get_expiry_date())
print("Дней до окончания: ", med1.days_until_expiry())

print("Лекарство 2:")
print("Название: ", med2.get_name())
print("Производитель: ", med2.get_manufacturer())
print("Срок годности: ", med2.get_expiry_date())
print("Дней до окончания: ", med2.days_until_expiry())
\end{lstlisting}

\subsection*{Вывод:}
\begin{lstlisting}[caption=Ожидаемый вывод]
Лекарство 1:
Название:  Парацетамол
Производитель:  Фармком
Срок годности:  2026-01-01
Дней до окончания:  113
Лекарство 2:
Название:  Ибупрофен
Производитель:  БиоФарм
Срок годности:  2025-12-01
Дней до окончания:  92
\end{lstlisting}

% ================= Вариант 31 =================
\item
Создайте класс \texttt{Project} с закрытыми атрибутами \texttt{\_\_title}, \texttt{\_\_start\_date}, \texttt{\_\_deadline}. Реализуйте методы-геттеры и метод \texttt{days\_until\_deadline()}.

\subsection*{Инструкции:}
\begin{enumerate}
    \item Создайте класс \texttt{Project}.
    \item Методы-геттеры: \texttt{get\_title()}, \texttt{get\_start\_date()}, \texttt{get\_deadline()}.
    \item Метод \texttt{days\_until\_deadline()} вычисляет дни до дедлайна.
    \item Создайте несколько экземпляров.
    \item Выведите данные каждого проекта.
\end{enumerate}

\subsection*{Пример использования:}
\begin{lstlisting}[caption=Пример кода]
from datetime import date

proj1 = Project("Разработка сайта", date(2025, 9, 1), date(2025, 12, 1))
proj2 = Project("Мобильное приложение", date(2025, 10, 1), date(2026, 1, 15))

print("Проект 1:")
print("Название: ", proj1.get_title())
print("Дата начала: ", proj1.get_start_date())
print("Дедлайн: ", proj1.get_deadline())
print("Дней до дедлайна: ", proj1.days_until_deadline())

print("Проект 2:")
print("Название: ", proj2.get_title())
print("Дата начала: ", proj2.get_start_date())
print("Дедлайн: ", proj2.get_deadline())
print("Дней до дедлайна: ", proj2.days_until_deadline())
\end{lstlisting}

\subsection*{Вывод:}
\begin{lstlisting}[caption=Ожидаемый вывод]
Проект 1:
Название:  Разработка сайта
Дата начала:  2025-09-01
Дедлайн:  2025-12-01
Дней до дедлайна:  91
Проект 2:
Название:  Мобильное приложение
Дата начала:  2025-10-01
Дедлайн:  2026-01-15
Дней до дедлайна:  106
\end{lstlisting}


\end{enumerate}


\subsubsection{Задача 4}


\textbf{Задача 4}

\begin{enumerate}
    \item 

    Написать программу на Python, которая создает абстрактный класс \texttt{Shape} для представления геометрической фигуры. 
Класс должен содержать абстрактные методы \texttt{calculate\_area} и \texttt{calculate\_perimeter}, 
которые вычисляют площадь и периметр фигуры соответственно. 
Программа также должна создавать дочерние классы \texttt{Circle}, \texttt{Rectangle} и \texttt{Triangle}, 
которые наследуют от класса \texttt{Shape} и реализуют специфические для каждого класса методы вычисления площади и периметра.

\textbf{Инструкции:}
\begin{enumerate}
    \item Создайте абстрактный класс \texttt{Shape} (с использованием модуля \texttt{abc}) с абстрактными методами 
    \texttt{calculate\_area()} и \texttt{calculate\_perimeter()}.
    \item Создайте класс \texttt{Circle} с конструктором \texttt{\_\_init\_\_(self, radius)}, 
    который принимает радиус окружности в качестве аргумента и сохраняет его в приватном атрибуте \texttt{\_\_radius}.  
    Добавьте \texttt{@property}-геттер \texttt{radius} для получения значения радиуса.  
    Реализуйте методы \texttt{calculate\_area()} и \texttt{calculate\_perimeter()} для вычисления площади и периметра окружности.
    \item Создайте класс \texttt{Rectangle} с конструктором \texttt{\_\_init\_\_(self, length, width)}, 
    который принимает длину и ширину прямоугольника в качестве аргументов и сохраняет их в приватных атрибутах 
    \texttt{\_length} и \texttt{\_width}.  
    Добавьте \texttt{@property}-геттеры \texttt{length} и \texttt{width} для получения значений атрибутов.  
    Реализуйте методы \texttt{calculate\_area()} и \texttt{calculate\_perimeter()} для вычисления площади и периметра прямоугольника.
    \item Создайте класс \texttt{Triangle} с конструктором \texttt{\_\_init\_\_(self, base, height, side1, side2, side3)}, 
    который принимает основание, высоту и три стороны треугольника в качестве аргументов и сохраняет их в приватных атрибутах 
    \texttt{\_base}, \texttt{\_height}, \texttt{\_side1}, \texttt{\_side2} и \texttt{\_side3}.  
    Добавьте \texttt{@property}-геттеры \texttt{base}, \texttt{height}, \texttt{side1}, \texttt{side2}, \texttt{side3}.  
    Реализуйте методы \texttt{calculate\_area()} и \texttt{calculate\_perimeter()} для вычисления площади и периметра треугольника.
    \item Создайте экземпляр каждого класса и вызовите методы \texttt{calculate\_area()} и \texttt{calculate\_perimeter()} 
    для вычисления площади и периметра фигуры. Выведите результаты на экран, используя геттеры для доступа к атрибутам.
\end{enumerate}

\textbf{Пример использования:}
\begin{verbatim}
# Вычисление параметров окружности.
r = 7
circle = Circle(r)
print("Радиус окружности:", circle.radius)
print("Площадь окружности:", circle.calculate_area())
print("Периметр окружности:", circle.calculate_perimeter())
\end{verbatim}

\textbf{Примечание:} В этом примере используется библиотека \texttt{math} для вычисления числа $\pi$ и квадратного корня.

\textbf{Вывод:}
\begin{verbatim}
Радиус окружности: 7
Площадь окружности: 153.93804002589985
Периметр окружности: 43.982297150257104
\end{verbatim}

Далее вывод для прямоугольника и треугольника.

\item
Написать программу на Python, которая создает абстрактный класс \texttt{ElectricalComponent} (с использованием модуля \texttt{abc}) для представления электрических элементов. 
Класс должен содержать абстрактные методы \texttt{calculate\_power()} и \texttt{calculate\_energy()}. 
Программа также должна создавать дочерние классы \texttt{Resistor}, \texttt{Capacitor} и \texttt{Inductor}, 
которые наследуют от класса \texttt{ElectricalComponent} и реализуют специфические для каждого класса методы вычисления мощности и энергии.

\textbf{Подсказка по формулам:}
\begin{itemize}
    \item \texttt{Resistor}: $P = U^2 / R$, $E = P \cdot t$
    \item \texttt{Capacitor}: $P = V \cdot I$, $E = 0.5 \cdot C \cdot V^2$
    \item \texttt{Inductor}: $P = L \cdot I^2$, $E = 0.5 \cdot L \cdot I^2$
\end{itemize}

\textbf{Инструкции:}
\begin{enumerate}
    \item Создайте абстрактный класс \texttt{ElectricalComponent} с методами \texttt{calculate\_power()} и \texttt{calculate\_energy()}, используя модуль \texttt{abc}.
    \item Создайте класс \texttt{Resistor} с конструктором \texttt{\_\_init\_\_(self, voltage, resistance, time)}, который сохраняет приватные атрибуты \texttt{\_\_voltage}, \texttt{\_\_resistance}, \texttt{\_\_time}. Добавьте \texttt{@property}-геттеры для всех атрибутов. Реализуйте методы вычисления мощности и энергии.
    \item Создайте класс \texttt{Capacitor} с конструктором \texttt{\_\_init\_\_(self, voltage, current, capacitance)}, приватными атрибутами \texttt{\_\_voltage}, \texttt{\_\_current}, \texttt{\_\_capacitance} и геттерами. Реализуйте методы.
    \item Создайте класс \texttt{Inductor} с конструктором \texttt{\_\_init\_\_(self, inductance, current)}, приватными атрибутами \texttt{\_\_inductance}, \texttt{\_\_current} и геттерами. Реализуйте методы.
    \item Создайте экземпляр каждого класса и вызовите методы \texttt{calculate\_power()} и \texttt{calculate\_energy()}, используя геттеры для доступа к атрибутам. Выведите результаты на экран.
\end{enumerate}

\textbf{Пример использования:}
\begin{verbatim}
r = Resistor(10, 5, 10)
print("Сопротивление резистора:", r.resistance)
print("Мощность резистора:", r.calculate_power())
print("Энергия резистора:", r.calculate_energy())
\end{verbatim}

\textbf{Вывод:}
\begin{verbatim}
Сопротивление резистора: 5
Мощность резистора: 20
Энергия резистора: 200
\end{verbatim}

Далее вывод для конденсатора и катушки индуктивности.

\item

Написать программу на Python, которая создает абстрактный класс \texttt{MotionObject} (с использованием модуля \texttt{abc}) для представления движущихся тел. 
Класс должен содержать абстрактные методы \texttt{calculate\_kinetic\_energy()} и \texttt{calculate\_momentum()}. 
Программа также должна создавать дочерние классы \texttt{LinearBody}, \texttt{RotatingBody} и \texttt{FallingBody}, 
которые наследуют от класса \texttt{MotionObject} и реализуют специфические для каждого класса методы вычисления кинетической энергии и импульса.

\textbf{Подсказка по формулам:}
\begin{itemize}
    \item \texttt{LinearBody}: $KE = 0.5 \cdot m \cdot v^2$, $p = m \cdot v$
    \item \texttt{RotatingBody}: $KE = 0.5 \cdot I \cdot \omega^2$, $p = I \cdot \omega$
    \item \texttt{FallingBody}: $KE = m \cdot g \cdot h$, $p = m \cdot v$
\end{itemize}

\textbf{Инструкции:}
\begin{enumerate}
    \item Создайте абстрактный класс \texttt{MotionObject} с методами \texttt{calculate\_kinetic\_energy()} и \texttt{calculate\_momentum()}, используя модуль \texttt{abc}.
    \item Создайте класс \texttt{LinearBody} с конструктором \texttt{\_\_init\_\_(self, mass, velocity)}, приватными атрибутами \texttt{\_\_mass}, \texttt{\_\_velocity} и геттерами. Реализуйте методы.
    \item Создайте класс \texttt{RotatingBody} с конструктором \texttt{\_\_init\_\_(self, moment\_of\_inertia, angular\_velocity)}, приватными атрибутами \texttt{\_\_moment\_of\_inertia}, \texttt{\_\_angular\_velocity} и геттерами. Реализуйте методы.
    \item Создайте класс \texttt{FallingBody} с конструктором \texttt{\_\_init\_\_(self, mass, height, velocity)}, приватными атрибутами \texttt{\_\_mass}, \texttt{\_\_height}, \texttt{\_\_velocity} и геттерами. Реализуйте методы.
    \item Создайте экземпляр каждого класса и вызовите методы \texttt{calculate\_kinetic\_energy()} и \texttt{calculate\_momentum()}, используя геттеры для доступа к атрибутам. Выведите результаты на экран.
\end{enumerate}

\textbf{Пример использования:}
\begin{verbatim}
body = LinearBody(2, 3)
print("Масса тела:", body.mass)
print("Кинетическая энергия:", body.calculate_kinetic_energy())
print("Импульс:", body.calculate_momentum())
\end{verbatim}

\textbf{Вывод:}
\begin{verbatim}
Масса тела: 2
Кинетическая энергия: 6
Импульс: 6
\end{verbatim}

Далее вывод для вращающегося тела и падающего тела.

\item
Написать программу на Python, которая создает абстрактный класс \texttt{Investment} (с использованием модуля \texttt{abc}) для финансовых вложений. 
Класс должен содержать абстрактные методы \texttt{calculate\_simple\_interest()} и \texttt{calculate\_total\_value()}. 
Программа также должна создавать дочерние классы \texttt{ShortTerm}, \texttt{LongTerm} и \texttt{CompoundInvestment}, 
которые наследуют от класса \texttt{Investment} и реализуют специфические методы вычисления процентов и итоговой суммы.

\textbf{Подсказка по формулам:}
\begin{itemize}
    \item \texttt{ShortTerm}: $SI = P \cdot R \cdot T / 100$, $Total = P + SI$
    \item \texttt{LongTerm}: $SI = P \cdot R \cdot T / 100 + 50$, $Total = P + SI$
    \item \texttt{CompoundInvestment}: $Total = P \cdot (1 + R/100)^T$, $SI = Total - P$
\end{itemize}

\textbf{Инструкции:}
\begin{enumerate}
    \item Создайте абстрактный класс \texttt{Investment} с методами \texttt{calculate\_simple\_interest()} и \texttt{calculate\_total\_value()}, используя модуль \texttt{abc}.
    \item Создайте класс \texttt{ShortTerm} с конструктором \texttt{\_\_init\_\_(self, principal, rate, time)}, приватными атрибутами \texttt{\_\_principal}, \texttt{\_\_rate}, \texttt{\_\_time} и геттерами. Реализуйте методы.
    \item Создайте класс \texttt{LongTerm} с конструктором \texttt{\_\_init\_\_(self, principal, rate, time)}, приватными атрибутами \texttt{\_\_principal}, \texttt{\_\_rate}, \texttt{\_\_time} и геттерами. Реализуйте методы.
    \item Создайте класс \texttt{CompoundInvestment} с конструктором \texttt{\_\_init\_\_(self, principal, rate, time)}, приватными атрибутами \texttt{\_\_principal}, \texttt{\_\_rate}, \texttt{\_\_time} и геттерами. Реализуйте методы.
    \item Создайте экземпляр каждого класса и вызовите методы \texttt{calculate\_simple\_interest()} и \texttt{calculate\_total\_value()}, используя геттеры для доступа к атрибутам. Выведите результаты на экран.
\end{enumerate}

\textbf{Пример использования:}
\begin{verbatim}
inv = ShortTerm(1000, 5, 2)
print("Начальная сумма:", inv.principal)
print("Простой процент:", inv.calculate_simple_interest())
print("Итоговая сумма:", inv.calculate_total_value())
\end{verbatim}

\textbf{Вывод:}
\begin{verbatim}
Начальная сумма: 1000
Простой процент: 100
Итоговая сумма: 1100
\end{verbatim}

Далее вывод для долгосрочного и сложного вложения.

\item
Написать программу на Python, которая создает абстрактный класс \texttt{Solid} (с использованием модуля \texttt{abc}) для твердого тела. 
Класс должен содержать абстрактные методы \texttt{calculate\_volume()} и \texttt{calculate\_surface\_area()}. 
Программа также должна создавать дочерние классы \texttt{Cube}, \texttt{RectangularPrism} и \texttt{Cylinder}, 
которые наследуют от класса \texttt{Solid} и реализуют специфические методы вычисления объема и площади поверхности.

\textbf{Подсказка по формулам:}
\begin{itemize}
    \item \texttt{Cube}: $V = a^3$, $S = 6 \cdot a^2$
    \item \texttt{RectangularPrism}: $V = l \cdot w \cdot h$, $S = 2(lw + lh + wh)$
    \item \texttt{Cylinder}: $V = \pi r^2 h$, $S = 2 \pi r (r + h)$
\end{itemize}

\textbf{Инструкции:}
\begin{enumerate}
    \item Создайте абстрактный класс \texttt{Solid} с методами \texttt{calculate\_volume()} и \texttt{calculate\_surface\_area()}, используя модуль \texttt{abc}.
    \item Создайте класс \texttt{Cube} с конструктором \texttt{\_\_init\_\_(self, side)}, приватным атрибутом \texttt{\_\_side} и геттером. Реализуйте методы.
    \item Создайте класс \texttt{RectangularPrism} с конструктором \texttt{\_\_init\_\_(self, length, width, height)}, приватными атрибутами \texttt{\_\_length}, \texttt{\_\_width}, \texttt{\_\_height} и геттерами. Реализуйте методы.
    \item Создайте класс \texttt{Cylinder} с конструктором \texttt{\_\_init\_\_(self, radius, height)}, приватными атрибутами \texttt{\_\_radius}, \texttt{\_\_height} и геттерами. Реализуйте методы.
    \item Создайте экземпляр каждого класса и вызовите методы \texttt{calculate\_volume()} и \texttt{calculate\_surface\_area()}, используя геттеры. Выведите результаты на экран.
\end{enumerate}

\textbf{Пример использования:}
\begin{verbatim}
cube = Cube(3)
print("Сторона куба:", cube.side)
print("Объем куба:", cube.calculate_volume())
print("Площадь поверхности куба:", cube.calculate_surface_area())
\end{verbatim}

\textbf{Вывод:}
\begin{verbatim}
Сторона куба: 3
Объем куба: 27
Площадь поверхности куба: 54
\end{verbatim}

Далее вывод для прямоугольного параллелепипеда и цилиндра.
\item
Написать программу на Python, которая создает абстрактный класс \texttt{ChemicalSubstance} (с использованием модуля \texttt{abc}) для химических веществ. 
Класс должен содержать абстрактные методы \texttt{calculate\_molar\_mass()} и \texttt{calculate\_density()}. 
Программа также должна создавать дочерние классы \texttt{Element}, \texttt{Compound} и \texttt{Mixture}, 
которые наследуют от класса \texttt{ChemicalSubstance} и реализуют специфические методы вычисления молярной массы и плотности.

\textbf{Подсказка по формулам:}
\begin{itemize}
    \item \texttt{Element}: $M = atomic\_mass$, $\rho = mass / volume$
    \item \texttt{Compound}: $M = \sum (fraction \cdot atomic\_mass)$, $\rho = mass / volume$
    \item \texttt{Mixture}: $M = \sum (fraction \cdot molar\_mass)$, $\rho = \sum (fraction \cdot density)$
\end{itemize}

\textbf{Инструкции:}
\begin{enumerate}
    \item Создайте абстрактный класс \texttt{ChemicalSubstance} с методами \texttt{calculate\_molar\_mass()} и \texttt{calculate\_density()}, используя модуль \texttt{abc}.
    \item Создайте класс \texttt{Element} с конструктором \texttt{\_\_init\_\_(self, atomic\_mass, mass, volume)}, приватными атрибутами и геттерами. Реализуйте методы.
    \item Создайте класс \texttt{Compound} с конструктором \texttt{\_\_init\_\_(self, fractions, atomic\_masses, mass, volume)}, приватными атрибутами и геттерами. Реализуйте методы.
    \item Создайте класс \texttt{Mixture} с конструктором \texttt{\_\_init\_\_(self, fractions, molar\_masses, densities)}, приватными атрибутами и геттерами. Реализуйте методы.
    \item Создайте экземпляр каждого класса и вызовите методы, используя геттеры, и выведите результаты.
\end{enumerate}

\textbf{Пример использования:}
\begin{verbatim}
el = Element(12, 24, 2)
print("Атомная масса элемента:", el.atomic_mass)
print("Молярная масса:", el.calculate_molar_mass())
print("Плотность:", el.calculate_density())
\end{verbatim}

\textbf{Вывод:}
\begin{verbatim}
Атомная масса элемента: 12
Молярная масса: 12
Плотность: 12
\end{verbatim}

Далее вывод для соединения и смеси.

\item
Написать программу на Python, которая создает абстрактный класс \texttt{BankAccount} (с использованием модуля \texttt{abc}) для банковских счетов. 
Класс должен содержать абстрактные методы \texttt{calculate\_interest()} и \texttt{calculate\_balance()}. 
Программа также должна создавать дочерние классы \texttt{Savings}, \texttt{Checking} и \texttt{FixedDeposit}, 
которые наследуют от класса \texttt{BankAccount} и реализуют специфические методы вычисления процентов и баланса.

\textbf{Подсказка по формулам:}
\begin{itemize}
    \item \texttt{Savings}: $Interest = balance \cdot rate \cdot time / 100$, $Balance = balance + Interest$
    \item \texttt{Checking}: $Interest = balance \cdot rate \cdot time / 100 - fee$, $Balance = balance + Interest$
    \item \texttt{FixedDeposit}: $Balance = principal \cdot (1 + rate/100)^time$, $Interest = Balance - principal$
\end{itemize}

\textbf{Инструкции:}
\begin{enumerate}
    \item Создайте абстрактный класс \texttt{BankAccount} с методами \texttt{calculate\_interest()} и \texttt{calculate\_balance()}, используя модуль \texttt{abc}.
    \item Создайте класс \texttt{Savings} с конструктором \texttt{\_\_init\_\_(self, balance, rate, time)}, приватными атрибутами и геттерами. Реализуйте методы.
    \item Создайте класс \texttt{Checking} с конструктором \texttt{\_\_init\_\_(self, balance, rate, time, fee)}, приватными атрибутами и геттерами. Реализуйте методы.
    \item Создайте класс \texttt{FixedDeposit} с конструктором \texttt{\_\_init\_\_(self, principal, rate, time)}, приватными атрибутами и геттерами. Реализуйте методы.
    \item Создайте экземпляр каждого класса и вызовите методы, используя геттеры, и выведите результаты.
\end{enumerate}

\textbf{Пример использования:}
\begin{verbatim}
s = Savings(1000, 5, 2)
print("Баланс на сберегательном счете:", s.balance)
print("Проценты:", s.calculate_interest())
print("Итоговый баланс:", s.calculate_balance())
\end{verbatim}

\textbf{Вывод:}
\begin{verbatim}
Баланс на сберегательном счете: 1000
Проценты: 100
Итоговый баланс: 1100
\end{verbatim}

Далее вывод для расчетного счета и срочного депозита.

\item
Написать программу на Python, которая создает абстрактный класс \texttt{Shape3D} (с использованием модуля \texttt{abc}) для трехмерных фигур. 
Класс должен содержать абстрактные методы \texttt{calculate\_volume()} и \texttt{calculate\_surface\_area()}. 
Программа также должна создавать дочерние классы \texttt{Sphere}, \texttt{Cone} и \texttt{Pyramid}, 
которые наследуют от класса \texttt{Shape3D} и реализуют специфические методы вычисления объема и площади поверхности.

\textbf{Подсказка по формулам:}
\begin{itemize}
    \item \texttt{Sphere}: $V = 4/3 \cdot \pi r^3$, $S = 4 \cdot \pi r^2$
    \item \texttt{Cone}: $V = 1/3 \cdot \pi r^2 h$, $S = \pi r (r + \sqrt{r^2 + h^2})$
    \item \texttt{Pyramid}: $V = 1/3 \cdot base\_area \cdot height$, $S = base\_area + lateral\_area$
\end{itemize}

\textbf{Инструкции:}
\begin{enumerate}
    \item Создайте абстрактный класс \texttt{Shape3D} с методами \texttt{calculate\_volume()} и \texttt{calculate\_surface\_area()}, используя модуль \texttt{abc}.
    \item Создайте класс \texttt{Sphere} с конструктором \texttt{\_\_init\_\_(self, radius)}, приватным атрибутом и геттером. Реализуйте методы.
    \item Создайте класс \texttt{Cone} с конструктором \texttt{\_\_init\_\_(self, radius, height)}, приватными атрибутами и геттерами. Реализуйте методы.
    \item Создайте класс \texttt{Pyramid} с конструктором \texttt{\_\_init\_\_(self, base\_area, lateral\_area, height)}, приватными атрибутами и геттерами. Реализуйте методы.
    \item Создайте экземпляр каждого класса и вызовите методы, используя геттеры, и выведите результаты.
\end{enumerate}

\textbf{Пример использования:}
\begin{verbatim}
s = Sphere(3)
print("Радиус сферы:", s.radius)
print("Объем сферы:", s.calculate_volume())
print("Площадь поверхности сферы:", s.calculate_surface_area())
\end{verbatim}

\textbf{Вывод:}
\begin{verbatim}
Радиус сферы: 3
Объем сферы: 113.097
Площадь поверхности сферы: 113.097
\end{verbatim}

Далее вывод для конуса и пирамиды.

\item
Написать программу на Python, которая создает абстрактный класс \texttt{Vehicle} (с использованием модуля \texttt{abc}) для транспортных средств. 
Класс должен содержать абстрактные методы \texttt{calculate\_fuel\_consumption()} и \texttt{calculate\_range()}. 
Программа также должна создавать дочерние классы \texttt{Car}, \texttt{Truck} и \texttt{Motorcycle}, 
которые наследуют от класса \texttt{Vehicle} и реализуют специфические методы вычисления расхода топлива и запаса хода.

\textbf{Подсказка по формулам:}
\begin{itemize}
    \item \texttt{Car}: $fuel = distance / efficiency$, $range = tank\_capacity \cdot efficiency$
    \item \texttt{Truck}: $fuel = (distance / efficiency) \cdot load\_factor$, $range = tank\_capacity \cdot efficiency / load\_factor$
    \item \texttt{Motorcycle}: $fuel = distance / efficiency \cdot 0.8$, $range = tank\_capacity \cdot efficiency \cdot 1.2$
\end{itemize}

\textbf{Инструкции:}
\begin{enumerate}
    \item Создайте абстрактный класс \texttt{Vehicle} с методами \texttt{calculate\_fuel\_consumption()} и \texttt{calculate\_range()}, используя модуль \texttt{abc}.
    \item Создайте класс \texttt{Car} с конструктором \texttt{\_\_init\_\_(self, efficiency, distance, tank\_capacity)}, приватными атрибутами и геттерами. Реализуйте методы.
    \item Создайте класс \texttt{Truck} с конструктором \texttt{\_\_init\_\_(self, efficiency, distance, tank\_capacity, load\_factor)}, приватными атрибутами и геттерами. Реализуйте методы.
    \item Создайте класс \texttt{Motorcycle} с конструктором \texttt{\_\_init\_\_(self, efficiency, distance, tank\_capacity)}, приватными атрибутами и геттерами. Реализуйте методы.
    \item Создайте экземпляр каждого класса и вызовите методы, используя геттеры, и выведите результаты.
\end{enumerate}

\textbf{Пример использования:}
\begin{verbatim}
car = Car(15, 150, 50)
print("Эффективность автомобиля:", car.efficiency)
print("Расход топлива:", car.calculate_fuel_consumption())
print("Запас хода:", car.calculate_range())
\end{verbatim}

\textbf{Вывод:}
\begin{verbatim}
Эффективность автомобиля: 15
Расход топлива: 10
Запас хода: 750
\end{verbatim}

Далее вывод для грузовика и мотоцикла.

\item
Написать программу на Python, которая создает абстрактный класс \texttt{Plant} (с использованием модуля \texttt{abc}) для растений. 
Класс должен содержать абстрактные методы \texttt{calculate\_growth()} и \texttt{calculate\_water\_needs()}. 
Программа также должна создавать дочерние классы \texttt{Tree}, \texttt{Flower} и \texttt{Shrub}, 
которые наследуют от класса \texttt{Plant} и реализуют специфические методы вычисления роста и потребности в воде.

\textbf{Подсказка по формулам:}
\begin{itemize}
    \item \texttt{Tree}: $growth = height\_rate \cdot time$, $water = area \cdot water\_rate$
    \item \texttt{Flower}: $growth = height\_rate \cdot time \cdot 0.5$, $water = area \cdot water\_rate \cdot 0.3$
    \item \texttt{Shrub}: $growth = height\_rate \cdot time \cdot 0.8$, $water = area \cdot water\_rate \cdot 0.6$
\end{itemize}

\textbf{Инструкции:}
\begin{enumerate}
    \item Создайте абстрактный класс \texttt{Plant} с методами \texttt{calculate\_growth()} и \texttt{calculate\_water\_needs()}, используя модуль \texttt{abc}.
    \item Создайте класс \texttt{Tree} с конструктором \texttt{\_\_init\_\_(self, height\_rate, time, area, water\_rate)}, приватными атрибутами и геттерами. Реализуйте методы.
    \item Создайте класс \texttt{Flower} с конструктором \texttt{\_\_init\_\_(self, height\_rate, time, area, water\_rate)}, приватными атрибутами и геттерами. Реализуйте методы.
    \item Создайте класс \texttt{Shrub} с конструктором \texttt{\_\_init\_\_(self, height\_rate, time, area, water\_rate)}, приватными атрибутами и геттерами. Реализуйте методы.
    \item Создайте экземпляр каждого класса и вызовите методы, используя геттеры, и выведите результаты.
\end{enumerate}

\textbf{Пример использования:}
\begin{verbatim}
tree = Tree(2, 5, 10, 3)
print("Скорость роста дерева:", tree.height_rate)
print("Рост:", tree.calculate_growth())
print("Потребность в воде:", tree.calculate_water_needs())
\end{verbatim}

\textbf{Вывод:}
\begin{verbatim}
Скорость роста дерева: 2
Рост: 10
Потребность в воде: 30
\end{verbatim}

Далее вывод для цветка и кустарника.

\item
Написать программу на Python, которая создает абстрактный класс \texttt{Sensor} (с использованием модуля \texttt{abc}) для измерительных датчиков. 
Класс должен содержать абстрактные методы \texttt{calculate\_signal()} и \texttt{calculate\_accuracy()}. 
Программа также должна создавать дочерние классы \texttt{TemperatureSensor}, \texttt{PressureSensor} и \texttt{LightSensor}, 
которые наследуют от класса \texttt{Sensor} и реализуют специфические методы вычисления сигнала и точности.

\textbf{Подсказка по формулам:}
\begin{itemize}
    \item \texttt{TemperatureSensor}: $signal = voltage \cdot sensitivity$, $accuracy = tolerance$
    \item \texttt{PressureSensor}: $signal = pressure \cdot sensitivity$, $accuracy = tolerance \cdot 1.1$
    \item \texttt{LightSensor}: $signal = intensity \cdot sensitivity$, $accuracy = tolerance \cdot 0.9$
\end{itemize}

\textbf{Инструкции:}
\begin{enumerate}
    \item Создайте абстрактный класс \texttt{Sensor} с методами \texttt{calculate\_signal()} и \texttt{calculate\_accuracy()}, используя модуль \texttt{abc}.
    \item Создайте класс \texttt{TemperatureSensor} с конструктором \texttt{\_\_init\_\_(self, voltage, sensitivity, tolerance)}, приватными атрибутами и геттерами. Реализуйте методы.
    \item Создайте класс \texttt{PressureSensor} с конструктором \texttt{\_\_init\_\_(self, pressure, sensitivity, tolerance)}, приватными атрибутами и геттерами. Реализуйте методы.
    \item Создайте класс \texttt{LightSensor} с конструктором \texttt{\_\_init\_\_(self, intensity, sensitivity, tolerance)}, приватными атрибутами и геттерами. Реализуйте методы.
    \item Создайте экземпляр каждого класса и вызовите методы, используя геттеры, и выведите результаты.
\end{enumerate}

\textbf{Пример использования:}
\begin{verbatim}
temp_sensor = TemperatureSensor(5, 2, 0.1)
print("Напряжение:", temp_sensor.voltage)
print("Сигнал:", temp_sensor.calculate_signal())
print("Точность:", temp_sensor.calculate_accuracy())
\end{verbatim}

\textbf{Вывод:}
\begin{verbatim}
Напряжение: 5
Сигнал: 10
Точность: 0.1
\end{verbatim}

Далее вывод для датчиков давления и света.

\item
Написать программу на Python, которая создает абстрактный класс \texttt{CookingIngredient} (с использованием модуля \texttt{abc}) для ингредиентов. 
Класс должен содержать абстрактные методы \texttt{calculate\_calories()} и \texttt{calculate\_mass()}. 
Программа также должна создавать дочерние классы \texttt{Vegetable}, \texttt{Meat} и \texttt{Grain}, 
которые наследуют от класса \texttt{CookingIngredient} и реализуют специфические методы вычисления калорий и массы.

\textbf{Подсказка по формулам:}
\begin{itemize}
    \item \texttt{Vegetable}: $calories = weight \cdot cal\_per\_100g / 100$, $mass = weight$
    \item \texttt{Meat}: $calories = weight \cdot cal\_per\_100g / 100 \cdot 1.2$, $mass = weight$
    \item \texttt{Grain}: $calories = weight \cdot cal\_per\_100g / 100 \cdot 1.1$, $mass = weight$
\end{itemize}

\textbf{Инструкции:}
\begin{enumerate}
    \item Создайте абстрактный класс \texttt{CookingIngredient} с методами \texttt{calculate\_calories()} и \texttt{calculate\_mass()}, используя модуль \texttt{abc}.
    \item Создайте класс \texttt{Vegetable} с конструктором \texttt{\_\_init\_\_(self, weight, cal\_per\_100g)}, приватными атрибутами и геттерами. Реализуйте методы.
    \item Создайте класс \texttt{Meat} с конструктором \texttt{\_\_init\_\_(self, weight, cal\_per\_100g)}, приватными атрибутами и геттерами. Реализуйте методы.
    \item Создайте класс \texttt{Grain} с конструктором \texttt{\_\_init\_\_(self, weight, cal\_per\_100g)}, приватными атрибутами и геттерами. Реализуйте методы.
    \item Создайте экземпляр каждого класса и вызовите методы, используя геттеры, и выведите результаты.
\end{enumerate}

\textbf{Пример использования:}
\begin{verbatim}
veg = Vegetable(200, 30)
print("Вес овоща:", veg.weight)
print("Калории:", veg.calculate_calories())
print("Масса:", veg.calculate_mass())
\end{verbatim}

\textbf{Вывод:}
\begin{verbatim}
Вес овоща: 200
Калории: 60
Масса: 200
\end{verbatim}

Далее вывод для мяса и зерна.

\item
Написать программу на Python, которая создает абстрактный класс \texttt{ElectronicDevice} (с использованием модуля \texttt{abc}) для электронных устройств. 
Класс должен содержать абстрактные методы \texttt{calculate\_power()} и \texttt{calculate\_efficiency()}. 
Программа также должна создавать дочерние классы \texttt{Laptop}, \texttt{Smartphone} и \texttt{Tablet}, 
которые наследуют от класса \texttt{ElectronicDevice} и реализуют специфические методы вычисления мощности и эффективности.

\textbf{Подсказка по формулам:}
\begin{itemize}
    \item \texttt{Laptop}: $power = voltage \cdot current$, $efficiency = useful\_power / power$
    \item \texttt{Smartphone}: $power = voltage \cdot current \cdot 0.8$, $efficiency = useful\_power / power$
    \item \texttt{Tablet}: $power = voltage \cdot current \cdot 0.9$, $efficiency = useful\_power / power$
\end{itemize}

\textbf{Инструкции:}
\begin{enumerate}
    \item Создайте абстрактный класс \texttt{ElectronicDevice} с методами \texttt{calculate\_power()} и \texttt{calculate\_efficiency()}, используя модуль \texttt{abc}.
    \item Создайте класс \texttt{Laptop} с конструктором \texttt{\_\_init\_\_(self, voltage, current, useful\_power)}, приватными атрибутами и геттерами. Реализуйте методы.
    \item Создайте класс \texttt{Smartphone} с конструктором \texttt{\_\_init\_\_(self, voltage, current, useful\_power)}, приватными атрибутами и геттерами. Реализуйте методы.
    \item Создайте класс \texttt{Tablet} с конструктором \texttt{\_\_init\_\_(self, voltage, current, useful\_power)}, приватными атрибутами и геттерами. Реализуйте методы.
    \item Создайте экземпляр каждого класса и вызовите методы, используя геттеры, и выведите результаты.
\end{enumerate}

\textbf{Пример использования:}
\begin{verbatim}
laptop = Laptop(19, 3, 50)
print("Напряжение ноутбука:", laptop.voltage)
print("Мощность:", laptop.calculate_power())
print("Эффективность:", laptop.calculate_efficiency())
\end{verbatim}

\textbf{Вывод:}
\begin{verbatim}
Напряжение ноутбука: 19
Мощность: 57
Эффективность: 0.877
\end{verbatim}

Далее вывод для смартфона и планшета.

\item
Написать программу на Python, которая создает абстрактный класс \texttt{MusicalInstrument} (с использованием модуля \texttt{abc}) для музыкальных инструментов. 
Класс должен содержать абстрактные методы \texttt{calculate\_sound\_level()} и \texttt{calculate\_frequency()}. 
Программа также должна создавать дочерние классы \texttt{Piano}, \texttt{Guitar} и \texttt{Flute}, 
которые наследуют от класса \texttt{MusicalInstrument} и реализуют специфические методы вычисления уровня звука и частоты.

\textbf{Подсказка по формулам:}
\begin{itemize}
    \item \texttt{Piano}: $sound\_level = keys \cdot intensity$, $frequency = 440 \cdot 2^{(note-49)/12}$
    \item \texttt{Guitar}: $sound\_level = strings \cdot intensity \cdot 0.8$, $frequency = 440 \cdot 2^{(note-49)/12}$
    \item \texttt{Flute}: $sound\_level = holes \cdot intensity \cdot 0.9$, $frequency = 440 \cdot 2^{(note-49)/12}$
\end{itemize}

\textbf{Инструкции:}
\begin{enumerate}
    \item Создайте абстрактный класс \texttt{MusicalInstrument} с методами \texttt{calculate\_sound\_level()} и \texttt{calculate\_frequency()}, используя модуль \texttt{abc}.
    \item Создайте класс \texttt{Piano} с конструктором \texttt{\_\_init\_\_(self, keys, intensity, note)}, приватными атрибутами и геттерами. Реализуйте методы.
    \item Создайте класс \texttt{Guitar} с конструктором \texttt{\_\_init\_\_(self, strings, intensity, note)}, приватными атрибутами и геттерами. Реализуйте методы.
    \item Создайте класс \texttt{Flute} с конструктором \texttt{\_\_init\_\_(self, holes, intensity, note)}, приватными атрибутами и геттерами. Реализуйте методы.
    \item Создайте экземпляр каждого класса и вызовите методы, используя геттеры, и выведите результаты.
\end{enumerate}

\textbf{Пример использования:}
\begin{verbatim}
piano = Piano(88, 5, 49)
print("Клавиши:", piano.keys)
print("Уровень звука:", piano.calculate_sound_level())
print("Частота:", piano.calculate_frequency())
\end{verbatim}

\textbf{Вывод:}
\begin{verbatim}
Клавиши: 88
Уровень звука: 440
Частота: 440
\end{verbatim}

Далее вывод для гитары и флейты.

\item
Написать программу на Python, которая создает абстрактный класс \texttt{Workout} (с использованием модуля \texttt{abc}) для физических упражнений. 
Класс должен содержать абстрактные методы \texttt{calculate\_calories\_burned()} и \texttt{calculate\_duration()}. 
Программа также должна создавать дочерние классы \texttt{Cardio}, \texttt{Strength} и \texttt{Flexibility}, 
которые наследуют от класса \texttt{Workout} и реализуют специфические методы вычисления сожженных калорий и длительности тренировки.

\textbf{Подсказка по формулам:}
\begin{itemize}
    \item \texttt{Cardio}: $calories = weight \cdot time \cdot 0.1$, $duration = time$
    \item \texttt{Strength}: $calories = weight \cdot time \cdot 0.08$, $duration = time$
    \item \texttt{Flexibility}: $calories = weight \cdot time \cdot 0.05$, $duration = time$
\end{itemize}

\textbf{Инструкции:}
\begin{enumerate}
    \item Создайте абстрактный класс \texttt{Workout} с методами \texttt{calculate\_calories\_burned()} и \texttt{calculate\_duration()}, используя модуль \texttt{abc}.
    \item Создайте класс \texttt{Cardio} с конструктором \texttt{\_\_init\_\_(self, weight, time)}, приватными атрибутами и геттерами. Реализуйте методы.
    \item Создайте класс \texttt{Strength} с конструктором \texttt{\_\_init\_\_(self, weight, time)}, приватными атрибутами и геттерами. Реализуйте методы.
    \item Создайте класс \texttt{Flexibility} с конструктором \texttt{\_\_init\_\_(self, weight, time)}, приватными атрибутами и геттерами. Реализуйте методы.
    \item Создайте экземпляр каждого класса и вызовите методы, используя геттеры, и выведите результаты.
\end{enumerate}

\textbf{Пример использования:}
\begin{verbatim}
cardio = Cardio(70, 30)
print("Вес:", cardio.weight)
print("Сожженные калории:", cardio.calculate_calories_burned())
print("Длительность:", cardio.calculate_duration())
\end{verbatim}

\textbf{Вывод:}
\begin{verbatim}
Вес: 70
Сожженные калории: 210
Длительность: 30
\end{verbatim}

Далее вывод для силовой и растяжки.

\item
Написать программу на Python, которая создает абстрактный класс \texttt{ComputerComponent} (с использованием модуля \texttt{abc}) для компонентов компьютера. 
Класс должен содержать абстрактные методы \texttt{calculate\_power\_consumption()} и \texttt{calculate\_cost()}. 
Программа также должна создавать дочерние классы \texttt{CPU}, \texttt{GPU} и \texttt{RAM}, 
которые наследуют от класса \texttt{ComputerComponent} и реализуют специфические методы вычисления энергопотребления и стоимости.

\textbf{Подсказка по формулам:}
\begin{itemize}
    \item \texttt{CPU}: $power = cores \cdot frequency \cdot 10$, $cost = cores \cdot 50$
    \item \texttt{GPU}: $power = cores \cdot frequency \cdot 12$, $cost = cores \cdot 80$
    \item \texttt{RAM}: $power = size \cdot 3$, $cost = size \cdot 20$
\end{itemize}

\textbf{Инструкции:}
\begin{enumerate}
    \item Создайте абстрактный класс \texttt{ComputerComponent} с методами \texttt{calculate\_power\_consumption()} и \texttt{calculate\_cost()}, используя модуль \texttt{abc}.
    \item Создайте класс \texttt{CPU} с конструктором \texttt{\_\_init\_\_(self, cores, frequency)}, приватными атрибутами и геттерами. Реализуйте методы.
    \item Создайте класс \texttt{GPU} с конструктором \texttt{\_\_init\_\_(self, cores, frequency)}, приватными атрибутами и геттерами. Реализуйте методы.
    \item Создайте класс \texttt{RAM} с конструктором \texttt{\_\_init\_\_(self, size)}, приватным атрибутом и геттером. Реализуйте методы.
    \item Создайте экземпляр каждого класса и вызовите методы, используя геттеры, и выведите результаты.
\end{enumerate}

\textbf{Пример использования:}
\begin{verbatim}
cpu = CPU(4, 3.5)
print("Ядра CPU:", cpu.cores)
print("Энергопотребление:", cpu.calculate_power_consumption())
print("Стоимость:", cpu.calculate_cost())
\end{verbatim}

\textbf{Вывод:}
\begin{verbatim}
Ядра CPU: 4
Энергопотребление: 140
Стоимость: 200
\end{verbatim}

Далее вывод для GPU и RAM.

\item
Написать программу на Python, которая создает абстрактный класс \texttt{Building} (с использованием модуля \texttt{abc}) для зданий. 
Класс должен содержать абстрактные методы \texttt{calculate\_volume()} и \texttt{calculate\_floor\_area()}. 
Программа также должна создавать дочерние классы \texttt{House}, \texttt{Office} и \texttt{Warehouse}, 
которые наследуют от класса \texttt{Building} и реализуют специфические методы вычисления объема и площади.

\textbf{Подсказка по формулам:}
\begin{itemize}
    \item \texttt{House}: $volume = length \cdot width \cdot height$, $floor\_area = length \cdot width$
    \item \texttt{Office}: $volume = length \cdot width \cdot height \cdot 1.2$, $floor\_area = length \cdot width \cdot 1.1$
    \item \texttt{Warehouse}: $volume = length \cdot width \cdot height \cdot 1.5$, $floor\_area = length \cdot width \cdot 1.3$
\end{itemize}

\textbf{Инструкции:}
\begin{enumerate}
    \item Создайте абстрактный класс \texttt{Building} с методами \texttt{calculate\_volume()} и \texttt{calculate\_floor\_area()}, используя модуль \texttt{abc}.
    \item Создайте класс \texttt{House} с конструктором \texttt{\_\_init\_\_(self, length, width, height)}, приватными атрибутами и геттерами. Реализуйте методы.
    \item Создайте класс \texttt{Office} с конструктором \texttt{\_\_init\_\_(self, length, width, height)}, приватными атрибутами и геттерами. Реализуйте методы.
    \item Создайте класс \texttt{Warehouse} с конструктором \texttt{\_\_init\_\_(self, length, width, height)}, приватными атрибутами и геттерами. Реализуйте методы.
    \item Создайте экземпляр каждого класса и вызовите методы, используя геттеры, и выведите результаты.
\end{enumerate}

\textbf{Пример использования:}
\begin{verbatim}
house = House(10, 8, 3)
print("Длина дома:", house.length)
print("Объем:", house.calculate_volume())
print("Площадь пола:", house.calculate_floor_area())
\end{verbatim}

\textbf{Вывод:}
\begin{verbatim}
Длина дома: 10
Объем: 240
Площадь пола: 80
\end{verbatim}

Далее вывод для офиса и склада.

\item
Написать программу на Python, которая создает абстрактный класс \texttt{Vehicle} (с использованием модуля \texttt{abc}) для транспортных средств. 
Класс должен содержать абстрактные методы \texttt{calculate\_max\_speed()} и \texttt{calculate\_range()}. 
Программа также должна создавать дочерние классы \texttt{Car}, \texttt{Motorcycle} и \texttt{Bicycle}, 
которые наследуют от класса \texttt{Vehicle} и реализуют специфические методы вычисления максимальной скорости и дальности.

\textbf{Подсказка по формулам:}
\begin{itemize}
    \item \texttt{Car}: $max\_speed = engine\_power \cdot 2$, $range = fuel\_capacity \cdot 10$
    \item \texttt{Motorcycle}: $max\_speed = engine\_power \cdot 2.5$, $range = fuel\_capacity \cdot 8$
    \item \texttt{Bicycle}: $max\_speed = pedaling\_power \cdot 3$, $range = stamina \cdot 5$
\end{itemize}

\textbf{Инструкции:}
\begin{enumerate}
    \item Создайте абстрактный класс \texttt{Vehicle} с методами \texttt{calculate\_max\_speed()} и \texttt{calculate\_range()}, используя модуль \texttt{abc}.
    \item Создайте класс \texttt{Car} с конструктором \texttt{\_\_init\_\_(self, engine\_power, fuel\_capacity)}, приватными атрибутами и геттерами. Реализуйте методы.
    \item Создайте класс \texttt{Motorcycle} с конструктором \texttt{\_\_init\_\_(self, engine\_power, fuel\_capacity)}, приватными атрибутами и геттерами. Реализуйте методы.
    \item Создайте класс \texttt{Bicycle} с конструктором \texttt{\_\_init\_\_(self, pedaling\_power, stamina)}, приватными атрибутами и геттерами. Реализуйте методы.
    \item Создайте экземпляр каждого класса и вызовите методы, используя геттеры, и выведите результаты.
\end{enumerate}

\textbf{Пример использования:}
\begin{verbatim}
car = Car(150, 50)
print("Мощность двигателя автомобиля:", car.engine_power)
print("Максимальная скорость:", car.calculate_max_speed())
print("Дальность:", car.calculate_range())
\end{verbatim}

\textbf{Вывод:}
\begin{verbatim}
Мощность двигателя автомобиля: 150
Максимальная скорость: 300
Дальность: 500
\end{verbatim}

Далее вывод для мотоцикла и велосипеда.

\item
Написать программу на Python, которая создает абстрактный класс \texttt{BankAccount} (с использованием модуля \texttt{abc}) для банковских счетов. 
Класс должен содержать абстрактные методы \texttt{calculate\_interest()} и \texttt{calculate\_fees()}. 
Программа также должна создавать дочерние классы \texttt{SavingsAccount}, \texttt{CheckingAccount} и \texttt{InvestmentAccount}, 
которые наследуют от класса \texttt{BankAccount} и реализуют специфические методы вычисления процентов и комиссий.

\textbf{Подсказка по формулам:}
\begin{itemize}
    \item \texttt{SavingsAccount}: $interest = balance \cdot 0.03$, $fees = 5$
    \item \texttt{CheckingAccount}: $interest = balance \cdot 0.01$, $fees = 2$
    \item \texttt{InvestmentAccount}: $interest = balance \cdot 0.05$, $fees = 10$
\end{itemize}

\textbf{Инструкции:}
\begin{enumerate}
    \item Создайте абстрактный класс \texttt{BankAccount} с методами \texttt{calculate\_interest()} и \texttt{calculate\_fees()}, используя модуль \texttt{abc}.
    \item Создайте класс \texttt{SavingsAccount} с конструктором \texttt{\_\_init\_\_(self, balance)}, приватными атрибутами и геттерами. Реализуйте методы.
    \item Создайте класс \texttt{CheckingAccount} с конструктором \texttt{\_\_init\_\_(self, balance)}, приватными атрибутами и геттерами. Реализуйте методы.
    \item Создайте класс \texttt{InvestmentAccount} с конструктором \texttt{\_\_init\_\_(self, balance)}, приватными атрибутами и геттерами. Реализуйте методы.
    \item Создайте экземпляр каждого класса и вызовите методы, используя геттеры, и выведите результаты.
\end{enumerate}

\textbf{Пример использования:}
\begin{verbatim}
savings = SavingsAccount(1000)
print("Баланс сберегательного счета:", savings.balance)
print("Проценты:", savings.calculate_interest())
print("Комиссии:", savings.calculate_fees())
\end{verbatim}

\textbf{Вывод:}
\begin{verbatim}
Баланс сберегательного счета: 1000
Проценты: 30.0
Комиссии: 5
\end{verbatim}

Далее вывод для расчетного и инвестиционного счета.

\item
Написать программу на Python, которая создает абстрактный класс \texttt{Appliance} (с использованием модуля \texttt{abc}) для бытовой техники. 
Класс должен содержать абстрактные методы \texttt{calculate\_energy\_usage()} и \texttt{calculate\_operating\_cost()}. 
Программа также должна создавать дочерние классы \texttt{Refrigerator}, \texttt{WashingMachine} и \texttt{Microwave}, 
которые наследуют от класса \texttt{Appliance} и реализуют специфические методы вычисления энергопотребления и стоимости эксплуатации.

\textbf{Подсказка по формулам:}
\begin{itemize}
    \item \texttt{Refrigerator}: $energy = power \cdot hours$, $cost = energy \cdot 0.12$
    \item \texttt{WashingMachine}: $energy = power \cdot hours \cdot 1.1$, $cost = energy \cdot 0.12$
    \item \texttt{Microwave}: $energy = power \cdot hours \cdot 0.8$, $cost = energy \cdot 0.12$
\end{itemize}

\textbf{Инструкции:}
\begin{enumerate}
    \item Создайте абстрактный класс \texttt{Appliance} с методами \texttt{calculate\_energy\_usage()} и \texttt{calculate\_operating\_cost()}, используя модуль \texttt{abc}.
    \item Создайте класс \texttt{Refrigerator} с конструктором \texttt{\_\_init\_\_(self, power, hours)}, приватными атрибутами и геттерами. Реализуйте методы.
    \item Создайте класс \texttt{WashingMachine} с конструктором \texttt{\_\_init\_\_(self, power, hours)}, приватными атрибутами и геттерами. Реализуйте методы.
    \item Создайте класс \texttt{Microwave} с конструктором \texttt{\_\_init\_\_(self, power, hours)}, приватными атрибутами и геттерами. Реализуйте методы.
    \item Создайте экземпляр каждого класса и вызовите методы, используя геттеры, и выведите результаты.
\end{enumerate}

\textbf{Пример использования:}
\begin{verbatim}
fridge = Refrigerator(150, 24)
print("Мощность холодильника:", fridge.power)
print("Энергопотребление:", fridge.calculate_energy_usage())
print("Стоимость эксплуатации:", fridge.calculate_operating_cost())
\end{verbatim}

\textbf{Вывод:}
\begin{verbatim}
Мощность холодильника: 150
Энергопотребление: 3600
Стоимость эксплуатации: 432.0
\end{verbatim}

Далее вывод для стиральной машины и микроволновки.

\item
Написать программу на Python, которая создает абстрактный класс \texttt{Planet} (с использованием модуля \texttt{abc}) для планет. 
Класс должен содержать абстрактные методы \texttt{calculate\_surface\_area()} и \texttt{calculate\_gravity()}. 
Программа также должна создавать дочерние классы \texttt{Earth}, \texttt{Mars} и \texttt{Jupiter}, 
которые наследуют от класса \texttt{Planet} и реализуют специфические методы вычисления площади поверхности и силы гравитации.

\textbf{Подсказка по формулам:}
\begin{itemize}
    \item \texttt{Earth}: $surface\_area = 4 \cdot \pi \cdot radius^2$, $gravity = G \cdot mass / radius^2$
    \item \texttt{Mars}: $surface\_area = 4 \cdot \pi \cdot radius^2 \cdot 0.95$, $gravity = G \cdot mass / radius^2 \cdot 0.38$
    \item \texttt{Jupiter}: $surface\_area = 4 \cdot \pi \cdot radius^2 \cdot 11.2$, $gravity = G \cdot mass / radius^2 \cdot 2.5$
\end{itemize}

\textbf{Инструкции:}
\begin{enumerate}
    \item Создайте абстрактный класс \texttt{Planet} с методами \texttt{calculate\_surface\_area()} и \texttt{calculate\_gravity()}, используя модуль \texttt{abc}.
    \item Создайте класс \texttt{Earth} с конструктором \texttt{\_\_init\_\_(self, radius, mass)}, приватными атрибутами и геттерами. Реализуйте методы.
    \item Создайте класс \texttt{Mars} с конструктором \texttt{\_\_init\_\_(self, radius, mass)}, приватными атрибутами и геттерами. Реализуйте методы.
    \item Создайте класс \texttt{Jupiter} с конструктором \texttt{\_\_init\_\_(self, radius, mass)}, приватными атрибутами и геттерами. Реализуйте методы.
    \item Создайте экземпляр каждого класса и вызовите методы, используя геттеры, и выведите результаты.
\end{enumerate}

\textbf{Пример использования:}
\begin{verbatim}
earth = Earth(6371, 5.97e24)
print("Радиус Земли:", earth.radius)
print("Площадь поверхности:", earth.calculate_surface_area())
print("Сила гравитации:", earth.calculate_gravity())
\end{verbatim}

\textbf{Вывод:}
\begin{verbatim}
Радиус Земли: 6371
Площадь поверхности: 510064471
Сила гравитации: 9.8
\end{verbatim}

Далее вывод для Марса и Юпитера.

\item
Написать программу на Python, которая создает абстрактный класс \texttt{FoodItem} (с использованием модуля \texttt{abc}) для пищевых продуктов. 
Класс должен содержать абстрактные методы \texttt{calculate\_calories()} и \texttt{calculate\_price()}. 
Программа также должна создавать дочерние классы \texttt{Fruit}, \texttt{Vegetable} и \texttt{Meat}, 
которые наследуют от класса \texttt{FoodItem} и реализуют специфические методы вычисления калорийности и стоимости.

\textbf{Подсказка по формулам:}
\begin{itemize}
    \item \texttt{Fruit}: $calories = weight \cdot 0.52$, $price = weight \cdot 3$
    \item \texttt{Vegetable}: $calories = weight \cdot 0.3$, $price = weight \cdot 2$
    \item \texttt{Meat}: $calories = weight \cdot 2.5$, $price = weight \cdot 10$
\end{itemize}

\textbf{Инструкции:}
\begin{enumerate}
    \item Создайте абстрактный класс \texttt{FoodItem} с методами \texttt{calculate\_calories()} и \texttt{calculate\_price()}, используя модуль \texttt{abc}.
    \item Создайте класс \texttt{Fruit} с конструктором \texttt{\_\_init\_\_(self, weight)}, приватным атрибутом и геттером. Реализуйте методы.
    \item Создайте класс \texttt{Vegetable} с конструктором \texttt{\_\_init\_\_(self, weight)}, приватным атрибутом и геттером. Реализуйте методы.
    \item Создайте класс \texttt{Meat} с конструктором \texttt{\_\_init\_\_(self, weight)}, приватным атрибутом и геттером. Реализуйте методы.
    \item Создайте экземпляр каждого класса и вызовите методы, используя геттеры, и выведите результаты.
\end{enumerate}

\textbf{Пример использования:}
\begin{verbatim}
apple = Fruit(150)
print("Вес фрукта:", apple.weight)
print("Калории:", apple.calculate_calories())
print("Стоимость:", apple.calculate_price())
\end{verbatim}

\textbf{Вывод:}
\begin{verbatim}
Вес фрукта: 150
Калории: 78.0
Стоимость: 450
\end{verbatim}

Далее вывод для овощей и мяса.

\item
Написать программу на Python, которая создает абстрактный класс \texttt{Tool} (с использованием модуля \texttt{abc}) для инструментов. 
Класс должен содержать абстрактные методы \texttt{calculate\_efficiency()} и \texttt{calculate\_durability()}. 
Программа также должна создавать дочерние классы \texttt{Hammer}, \texttt{Screwdriver} и \texttt{Wrench}, 
которые наследуют от класса \texttt{Tool} и реализуют специфические методы вычисления эффективности и прочности.

\textbf{Подсказка по формулам:}
\begin{itemize}
    \item \texttt{Hammer}: $efficiency = weight \cdot swing\_speed$, $durability = material\_hardness \cdot 10$
    \item \texttt{Screwdriver}: $efficiency = length \cdot torque$, $durability = material\_hardness \cdot 8$
    \item \texttt{Wrench}: $efficiency = size \cdot torque$, $durability = material\_hardness \cdot 12$
\end{itemize}

\textbf{Инструкции:}
\begin{enumerate}
    \item Создайте абстрактный класс \texttt{Tool} с методами \texttt{calculate\_efficiency()} и \texttt{calculate\_durability()}, используя модуль \texttt{abc}.
    \item Создайте класс \texttt{Hammer} с конструктором \texttt{\_\_init\_\_(self, weight, swing\_speed, material\_hardness)}, приватными атрибутами и геттерами. Реализуйте методы.
    \item Создайте класс \texttt{Screwdriver} с конструктором \texttt{\_\_init\_\_(self, length, torque, material\_hardness)}, приватными атрибутами и геттерами. Реализуйте методы.
    \item Создайте класс \texttt{Wrench} с конструктором \texttt{\_\_init\_\_(self, size, torque, material\_hardness)}, приватными атрибутами и геттерами. Реализуйте методы.
    \item Создайте экземпляр каждого класса и вызовите методы, используя геттеры, и выведите результаты.
\end{enumerate}

\textbf{Пример использования:}
\begin{verbatim}
hammer = Hammer(2, 5, 7)
print("Вес молотка:", hammer.weight)
print("Эффективность:", hammer.calculate_efficiency())
print("Прочность:", hammer.calculate_durability())
\end{verbatim}

\textbf{Вывод:}
\begin{verbatim}
Вес молотка: 2
Эффективность: 10
Прочность: 70
\end{verbatim}

Далее вывод для отвертки и ключа.

\item
Написать программу на Python, которая создает абстрактный класс \texttt{Book} (с использованием модуля \texttt{abc}) для книг. 
Класс должен содержать абстрактные методы \texttt{calculate\_reading\_time()} и \texttt{calculate\_cost()}. 
Программа также должна создавать дочерние классы \texttt{Fiction}, \texttt{NonFiction} и \texttt{Comics}, 
которые наследуют от класса \texttt{Book} и реализуют специфические методы вычисления времени чтения и стоимости.

\textbf{Подсказка по формулам:}
\begin{itemize}
    \item \texttt{Fiction}: $reading\_time = pages \cdot 2$, $cost = pages \cdot 1.5$
    \item \texttt{NonFiction}: $reading\_time = pages \cdot 2.5$, $cost = pages \cdot 2$
    \item \texttt{Comics}: $reading\_time = pages \cdot 1$, $cost = pages \cdot 1$
\end{itemize}

\textbf{Инструкции:}
\begin{enumerate}
    \item Создайте абстрактный класс \texttt{Book} с методами \texttt{calculate\_reading\_time()} и \texttt{calculate\_cost()}, используя модуль \texttt{abc}.
    \item Создайте класс \texttt{Fiction} с конструктором \texttt{\_\_init\_\_(self, pages)}, приватным атрибутом и геттером. Реализуйте методы.
    \item Создайте класс \texttt{NonFiction} с конструктором \texttt{\_\_init\_\_(self, pages)}, приватным атрибутом и геттером. Реализуйте методы.
    \item Создайте класс \texttt{Comics} с конструктором \texttt{\_\_init\_\_(self, pages)}, приватным атрибутом и геттером. Реализуйте методы.
    \item Создайте экземпляр каждого класса и вызовите методы, используя геттеры, и выведите результаты.
\end{enumerate}

\textbf{Пример использования:}
\begin{verbatim}
novel = Fiction(300)
print("Количество страниц:", novel.pages)
print("Время чтения:", novel.calculate_reading_time())
print("Стоимость:", novel.calculate_cost())
\end{verbatim}

\textbf{Вывод:}
\begin{verbatim}
Количество страниц: 300
Время чтения: 600
Стоимость: 450.0
\end{verbatim}

Далее вывод для научной литературы и комиксов.

\item
Написать программу на Python, которая создает абстрактный класс \texttt{ElectronicDevice} (с использованием модуля \texttt{abc}) для электронных устройств. 
Класс должен содержать абстрактные методы \texttt{calculate\_power\_consumption()} и \texttt{calculate\_battery\_life()}. 
Программа также должна создавать дочерние классы \texttt{Smartphone}, \texttt{Laptop} и \texttt{Tablet}, 
которые наследуют от класса \texttt{ElectronicDevice} и реализуют специфические методы вычисления потребляемой мощности и времени работы от батареи.

\textbf{Подсказка по формулам:}
\begin{itemize}
    \item \texttt{Smartphone}: $power = voltage \cdot current \cdot hours$, $battery\_life = battery\_capacity / current$
    \item \texttt{Laptop}: $power = voltage \cdot current \cdot hours \cdot 1.5$, $battery\_life = battery\_capacity / (current \cdot 1.5)$
    \item \texttt{Tablet}: $power = voltage \cdot current \cdot hours \cdot 1.2$, $battery\_life = battery\_capacity / (current \cdot 1.2)$
\end{itemize}

\textbf{Инструкции:}
\begin{enumerate}
    \item Создайте абстрактный класс \texttt{ElectronicDevice} с методами \texttt{calculate\_power\_consumption()} и \texttt{calculate\_battery\_life()}, используя модуль \texttt{abc}.
    \item Создайте класс \texttt{Smartphone} с конструктором \texttt{\_\_init\_\_(self, voltage, current, hours, battery\_capacity)}, приватными атрибутами и геттерами. Реализуйте методы.
    \item Создайте класс \texttt{Laptop} с конструктором \texttt{\_\_init\_\_(self, voltage, current, hours, battery\_capacity)}, приватными атрибутами и геттерами. Реализуйте методы.
    \item Создайте класс \texttt{Tablet} с конструктором \texttt{\_\_init\_\_(self, voltage, current, hours, battery\_capacity)}, приватными атрибутами и геттерами. Реализуйте методы.
    \item Создайте экземпляр каждого класса и вызовите методы, используя геттеры, и выведите результаты.
\end{enumerate}

\textbf{Пример использования:}
\begin{verbatim}
phone = Smartphone(5, 1, 10, 5000)
print("Напряжение смартфона:", phone.voltage)
print("Потребляемая мощность:", phone.calculate_power_consumption())
print("Время работы от батареи:", phone.calculate_battery_life())
\end{verbatim}

\textbf{Вывод:}
\begin{verbatim}
Напряжение смартфона: 5
Потребляемая мощность: 50
Время работы от батареи: 5000.0
\end{verbatim}

Далее вывод для ноутбука и планшета.

\item
Написать программу на Python, которая создает абстрактный класс \texttt{MusicalInstrument} (с использованием модуля \texttt{abc}) для музыкальных инструментов. 
Класс должен содержать абстрактные методы \texttt{calculate\_sound\_volume()} и \texttt{calculate\_weight()}. 
Программа также должна создавать дочерние классы \texttt{Piano}, \texttt{Guitar} и \texttt{Drum}, 
которые наследуют от класса \texttt{MusicalInstrument} и реализуют специфические методы вычисления громкости и веса.

\textbf{Подсказка по формулам:}
\begin{itemize}
    \item \texttt{Piano}: $volume = keys \cdot 2$, $weight = base\_weight \cdot 3$
    \item \texttt{Guitar}: $volume = strings \cdot 3$, $weight = base\_weight \cdot 1.5$
    \item \texttt{Drum}: $volume = diameter \cdot 4$, $weight = base\_weight \cdot 2$
\end{itemize}

\textbf{Инструкции:}
\begin{enumerate}
    \item Создайте абстрактный класс \texttt{MusicalInstrument} с методами \texttt{calculate\_sound\_volume()} и \texttt{calculate\_weight()}, используя модуль \texttt{abc}.
    \item Создайте класс \texttt{Piano} с конструктором \texttt{\_\_init\_\_(self, keys, base\_weight)}, приватными атрибутами и геттерами. Реализуйте методы.
    \item Создайте класс \texttt{Guitar} с конструктором \texttt{\_\_init\_\_(self, strings, base\_weight)}, приватными атрибутами и геттерами. Реализуйте методы.
    \item Создайте класс \texttt{Drum} с конструктором \texttt{\_\_init\_\_(self, diameter, base\_weight)}, приватными атрибутами и геттерами. Реализуйте методы.
    \item Создайте экземпляр каждого класса и вызовите методы, используя геттеры, и выведите результаты.
\end{enumerate}

\textbf{Пример использования:}
\begin{verbatim}
piano = Piano(88, 200)
print("Количество клавиш:", piano.keys)
print("Громкость:", piano.calculate_sound_volume())
print("Вес:", piano.calculate_weight())
\end{verbatim}

\textbf{Вывод:}
\begin{verbatim}
Количество клавиш: 88
Громкость: 176
Вес: 600
\end{verbatim}

Далее вывод для гитары и барабана.

\item
Написать программу на Python, которая создает абстрактный класс \texttt{VehiclePart} (с использованием модуля \texttt{abc}) для частей транспортного средства. 
Класс должен содержать абстрактные методы \texttt{calculate\_durability()} и \texttt{calculate\_maintenance\_cost()}. 
Программа также должна создавать дочерние классы \texttt{Engine}, \texttt{Wheel} и \texttt{Brake}, 
которые наследуют от класса \texttt{VehiclePart} и реализуют специфические методы вычисления долговечности и стоимости обслуживания.

\textbf{Подсказка по формулам:}
\begin{itemize}
    \item \texttt{Engine}: $durability = hours\_run \cdot 1.2$, $maintenance = base\_cost \cdot 5$
    \item \texttt{Wheel}: $durability = rotation\_count \cdot 0.8$, $maintenance = base\_cost \cdot 2$
    \item \texttt{Brake}: $durability = pressure\_applied \cdot 0.5$, $maintenance = base\_cost \cdot 3$
\end{itemize}

\textbf{Инструкции:}
\begin{enumerate}
    \item Создайте абстрактный класс \texttt{VehiclePart} с методами \texttt{calculate\_durability()} и \texttt{calculate\_maintenance\_cost()}, используя модуль \texttt{abc}.
    \item Создайте класс \texttt{Engine} с конструктором \texttt{\_\_init\_\_(self, hours\_run, base\_cost)}, приватными атрибутами и геттерами. Реализуйте методы.
    \item Создайте класс \texttt{Wheel} с конструктором \texttt{\_\_init\_\_(self, rotation\_count, base\_cost)}, приватными атрибутами и геттерами. Реализуйте методы.
    \item Создайте класс \texttt{Brake} с конструктором \texttt{\_\_init\_\_(self, pressure\_applied, base\_cost)}, приватными атрибутами и геттерами. Реализуйте методы.
    \item Создайте экземпляр каждого класса и вызовите методы, используя геттеры, и выведите результаты.
\end{enumerate}

\textbf{Пример использования:}
\begin{verbatim}
engine = Engine(1000, 200)
print("Наработка двигателя:", engine.hours_run)
print("Долговечность:", engine.calculate_durability())
print("Стоимость обслуживания:", engine.calculate_maintenance_cost())
\end{verbatim}

\textbf{Вывод:}
\begin{verbatim}
Наработка двигателя: 1000
Долговечность: 1200.0
Стоимость обслуживания: 1000
\end{verbatim}

Далее вывод для колес и тормозов.

\item
Написать программу на Python, которая создает абстрактный класс \texttt{Appliance} (с использованием модуля \texttt{abc}) для бытовых приборов. 
Класс должен содержать абстрактные методы \texttt{calculate\_energy\_consumption()} и \texttt{calculate\_cost()}. 
Программа также должна создавать дочерние классы \texttt{Refrigerator}, \texttt{WashingMachine} и \texttt{Microwave}, 
которые наследуют от класса \texttt{Appliance} и реализуют специфические методы вычисления потребляемой энергии и стоимости эксплуатации.

\textbf{Подсказка по формулам:}
\begin{itemize}
    \item \texttt{Refrigerator}: $energy = power \cdot hours \cdot 30$, $cost = energy \cdot rate$
    \item \texttt{WashingMachine}: $energy = power \cdot hours \cdot 1.5$, $cost = energy \cdot rate$
    \item \texttt{Microwave}: $energy = power \cdot hours \cdot 0.8$, $cost = energy \cdot rate$
\end{itemize}

\textbf{Инструкции:}
\begin{enumerate}
    \item Создайте абстрактный класс \texttt{Appliance} с методами \texttt{calculate\_energy\_consumption()} и \texttt{calculate\_cost()}, используя модуль \texttt{abc}.
    \item Создайте класс \texttt{Refrigerator} с конструктором \texttt{\_\_init\_\_(self, power, hours, rate)}, приватными атрибутами и геттерами. Реализуйте методы.
    \item Создайте класс \texttt{WashingMachine} с конструктором \texttt{\_\_init\_\_(self, power, hours, rate)}, приватными атрибутами и геттерами. Реализуйте методы.
    \item Создайте класс \texttt{Microwave} с конструктором \texttt{\_\_init\_\_(self, power, hours, rate)}, приватными атрибутами и геттерами. Реализуйте методы.
    \item Создайте экземпляр каждого класса и вызовите методы, используя геттеры, и выведите результаты.
\end{enumerate}

\textbf{Пример использования:}
\begin{verbatim}
fridge = Refrigerator(150, 24, 0.1)
print("Мощность холодильника:", fridge.power)
print("Энергопотребление:", fridge.calculate_energy_consumption())
print("Стоимость эксплуатации:", fridge.calculate_cost())
\end{verbatim}

\textbf{Вывод:}
\begin{verbatim}
Мощность холодильника: 150
Энергопотребление: 108000
Стоимость эксплуатации: 10800.0
\end{verbatim}

Далее вывод для стиральной машины и микроволновки.

\item
Написать программу на Python, которая создает абстрактный класс \texttt{SportActivity} (с использованием модуля \texttt{abc}) для спортивных занятий. 
Класс должен содержать абстрактные методы \texttt{calculate\_calories\_burned()} и \texttt{calculate\_duration()}. 
Программа также должна создавать дочерние классы \texttt{Running}, \texttt{Swimming} и \texttt{Cycling}, 
которые наследуют от класса \texttt{SportActivity} и реализуют специфические методы вычисления сожженных калорий и продолжительности.

\textbf{Подсказка по формулам:}
\begin{itemize}
    \item \texttt{Running}: $calories = weight \cdot distance \cdot 1.036$, $duration = distance / speed$
    \item \texttt{Swimming}: $calories = weight \cdot distance \cdot 1.5$, $duration = distance / speed$
    \item \texttt{Cycling}: $calories = weight \cdot distance \cdot 0.8$, $duration = distance / speed$
\end{itemize}

\textbf{Инструкции:}
\begin{enumerate}
    \item Создайте абстрактный класс \texttt{SportActivity} с методами \texttt{calculate\_calories\_burned()} и \texttt{calculate\_duration()}, используя модуль \texttt{abc}.
    \item Создайте класс \texttt{Running} с конструктором \texttt{\_\_init\_\_(self, weight, distance, speed)}, приватными атрибутами и геттерами. Реализуйте методы.
    \item Создайте класс \texttt{Swimming} с конструктором \texttt{\_\_init\_\_(self, weight, distance, speed)}, приватными атрибутами и геттерами. Реализуйте методы.
    \item Создайте класс \texttt{Cycling} с конструктором \texttt{\_\_init\_\_(self, weight, distance, speed)}, приватными атрибутами и геттерами. Реализуйте методы.
    \item Создайте экземпляр каждого класса и вызовите методы, используя геттеры, и выведите результаты.
\end{enumerate}

\textbf{Пример использования:}
\begin{verbatim}
run = Running(70, 5, 10)
print("Вес бегуна:", run.weight)
print("Сожженные калории:", run.calculate_calories_burned())
print("Продолжительность:", run.calculate_duration())
\end{verbatim}

\textbf{Вывод:}
\begin{verbatim}
Вес бегуна: 70
Сожженные калории: 362.6
Продолжительность: 0.5
\end{verbatim}

Далее вывод для плавания и езды на велосипеде.

\item
Написать программу на Python, которая создает абстрактный класс \texttt{BuildingMaterial} (с использованием модуля \texttt{abc}) для строительных материалов. 
Класс должен содержать абстрактные методы \texttt{calculate\_strength()} и \texttt{calculate\_cost()}. 
Программа также должна создавать дочерние классы \texttt{Concrete}, \texttt{Wood}, \texttt{Steel}, 
которые наследуют от класса \texttt{BuildingMaterial} и реализуют специфические методы вычисления прочности и стоимости.

\textbf{Подсказка по формулам:}
\begin{itemize}
    \item \texttt{Concrete}: $strength = density \cdot compressive\_factor$, $cost = volume \cdot price\_per\_m3$
    \item \texttt{Wood}: $strength = density \cdot elastic\_factor$, $cost = volume \cdot price\_per\_m3$
    \item \texttt{Steel}: $strength = density \cdot tensile\_factor$, $cost = volume \cdot price\_per\_m3$
\end{itemize}

\textbf{Инструкции:}
\begin{enumerate}
    \item Создайте абстрактный класс \texttt{BuildingMaterial} с методами \texttt{calculate\_strength()} и \texttt{calculate\_cost()}, используя модуль \texttt{abc}.
    \item Создайте класс \texttt{Concrete} с конструктором \texttt{\_\_init\_\_(self, density, compressive\_factor, volume, price\_per\_m3)}, приватными атрибутами и геттерами. Реализуйте методы.
    \item Создайте класс \texttt{Wood} с конструктором \texttt{\_\_init\_\_(self, density, elastic\_factor, volume, price\_per\_m3)}, приватными атрибутами и геттерами. Реализуйте методы.
    \item Создайте класс \texttt{Steel} с конструктором \texttt{\_\_init\_\_(self, density, tensile\_factor, volume, price\_per\_m3)}, приватными атрибутами и геттерами. Реализуйте методы.
    \item Создайте экземпляр каждого класса и вызовите методы, используя геттеры, и выведите результаты.
\end{enumerate}

\textbf{Пример использования:}
\begin{verbatim}
concrete = Concrete(2400, 30, 2, 100)
print("Плотность бетона:", concrete.density)
print("Прочность:", concrete.calculate_strength())
print("Стоимость:", concrete.calculate_cost())
\end{verbatim}

\textbf{Вывод:}
\begin{verbatim}
Плотность бетона: 2400
Прочность: 72000
Стоимость: 200
\end{verbatim}

Далее вывод для древесины и стали.

\item
Написать программу на Python, которая создает абстрактный класс \texttt{TransportVehicle} (с использованием модуля \texttt{abc}) для транспортных средств. 
Класс должен содержать абстрактные методы \texttt{calculate\_range()} и \texttt{calculate\_fuel\_cost()}. 
Программа также должна создавать дочерние классы \texttt{Car}, \texttt{Motorcycle} и \texttt{ElectricScooter}, 
которые наследуют от класса \texttt{TransportVehicle} и реализуют специфические методы вычисления дальности хода и стоимости топлива/энергии.

\textbf{Подсказка по формулам:}
\begin{itemize}
    \item \texttt{Car}: $range = tank\_capacity / consumption \cdot 100$, $fuel\_cost = tank\_capacity \cdot fuel\_price$
    \item \texttt{Motorcycle}: $range = tank\_capacity / consumption \cdot 120$, $fuel\_cost = tank\_capacity \cdot fuel\_price$
    \item \texttt{ElectricScooter}: $range = battery\_capacity / consumption \cdot 100$, $fuel\_cost = battery\_capacity \cdot electricity\_rate$
\end{itemize}

\textbf{Инструкции:}
\begin{enumerate}
    \item Создайте абстрактный класс \texttt{TransportVehicle} с методами \texttt{calculate\_range()} и \texttt{calculate\_fuel\_cost()}, используя модуль \texttt{abc}.
    \item Создайте класс \texttt{Car} с конструктором \texttt{\_\_init\_\_(self, tank\_capacity, consumption, fuel\_price)}, приватными атрибутами и геттерами. Реализуйте методы.
    \item Создайте класс \texttt{Motorcycle} с конструктором \texttt{\_\_init\_\_(self, tank\_capacity, consumption, fuel\_price)}, приватными атрибутами и геттерами. Реализуйте методы.
    \item Создайте класс \texttt{ElectricScooter} с конструктором \texttt{\_\_init\_\_(self, battery\_capacity, consumption, electricity\_rate)}, приватными атрибутами и геттерами. Реализуйте методы.
    \item Создайте экземпляр каждого класса и вызовите методы, используя геттеры, и выведите результаты.
\end{enumerate}

\textbf{Пример использования:}
\begin{verbatim}
car = Car(50, 8, 1.5)
print("Ёмкость бака автомобиля:", car.tank_capacity)
print("Дальность хода:", car.calculate_range())
print("Стоимость топлива:", car.calculate_fuel_cost())
\end{verbatim}

\textbf{Вывод:}
\begin{verbatim}
Ёмкость бака автомобиля: 50
Дальность хода: 625.0
Стоимость топлива: 75.0
\end{verbatim}

Далее вывод для Motorcycle и ElectricScooter.



\end{enumerate}